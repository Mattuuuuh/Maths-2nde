				% ENABLE or DISABLE font change
				% use XeLaTeX if true
\newif\ifdys
				\dystrue
				\dysfalse

\newif\ifsolutions
				\solutionstrue
				%\solutionsfalse

% DYSLEXIA SWITCH
\newif\ifdys
		
				% ENABLE or DISABLE font change
				% use XeLaTeX if true
				\dystrue
				\dysfalse


\ifdys

\documentclass[a4paper, 14pt]{extarticle}
\usepackage{amsmath,amsfonts,amsthm,amssymb,mathtools}

\tracinglostchars=3 % Report an error if a font does not have a symbol.
\usepackage{fontspec}
\usepackage{unicode-math}
\defaultfontfeatures{ Ligatures=TeX,
                      Scale=MatchUppercase }

\setmainfont{OpenDyslexic}[Scale=1.0]
\setmathfont{Fira Math} % Or maybe try KPMath-Sans?
\setmathfont{OpenDyslexic Italic}[range=it/{Latin,latin}]
\setmathfont{OpenDyslexic}[range=up/{Latin,latin,num}]

\else

\documentclass[a4paper, 12pt]{extarticle}

\usepackage[utf8x]{inputenc}
%fonts
\usepackage{amsmath,amsfonts,amsthm,amssymb,mathtools}
% comment below to default to computer modern
\usepackage{libertinus,libertinust1math}

\fi


\usepackage[french]{babel}
\usepackage[
a4paper,
margin=2cm,
nomarginpar,% We don't want any margin paragraphs
]{geometry}
\usepackage{icomma}

\usepackage{fancyhdr}
\usepackage{array}
\usepackage{hyperref}

\usepackage{multicol, enumerate}
\newcolumntype{P}[1]{>{\centering\arraybackslash}p{#1}}


\usepackage{stackengine}
\newcommand\xrowht[2][0]{\addstackgap[.5\dimexpr#2\relax]{\vphantom{#1}}}

% theorems

\theoremstyle{plain}
\newtheorem{theorem}{Th\'eor\`eme}
\newtheorem*{sol}{Solution}
\theoremstyle{definition}
\newtheorem{ex}{Exercice}
\newtheorem*{rpl}{Rappel}
\newtheorem{enigme}{Énigme}

% corps
\usepackage{calrsfs}
\newcommand{\C}{\mathcal{C}}
\newcommand{\R}{\mathbb{R}}
\newcommand{\Rnn}{\mathbb{R}^{2n}}
\newcommand{\Z}{\mathbb{Z}}
\newcommand{\N}{\mathbb{N}}
\newcommand{\Q}{\mathbb{Q}}

% variance
\newcommand{\Var}[1]{\text{Var}(#1)}

% domain
\newcommand{\D}{\mathcal{D}}


% date
\usepackage{advdate}
\AdvanceDate[0]


% plots
\usepackage{pgfplots}

% table line break
\usepackage{makecell}
%tablestuff
\def\arraystretch{2}
\setlength\tabcolsep{15pt}

%subfigures
\usepackage{subcaption}

\definecolor{myg}{RGB}{56, 140, 70}
\definecolor{myb}{RGB}{45, 111, 177}
\definecolor{myr}{RGB}{199, 68, 64}

% fake sections with no title to move around the merged pdf
\newcommand{\fakesection}[1]{%
  \par\refstepcounter{section}% Increase section counter
  \sectionmark{#1}% Add section mark (header)
  \addcontentsline{toc}{section}{\protect\numberline{\thesection}#1}% Add section to ToC
  % Add more content here, if needed.
}


% SOLUTION SWITCH
\newif\ifsolutions
				\solutionstrue
				%\solutionsfalse

\ifsolutions
	\newcommand{\exe}[2]{
		\begin{ex} #1  \end{ex}
		\begin{sol} #2 \end{sol}
	}
\else
	\newcommand{\exe}[2]{
		\begin{ex} #1  \end{ex}
	}
	
\fi


% tableaux var, signe
\usepackage{tkz-tab}


%pinfty minfty
\newcommand{\pinfty}{{+}\infty}
\newcommand{\minfty}{{-}\infty}

\begin{document}


\AdvanceDate[0]

\begin{document}
\pagestyle{fancy}
\fancyhead[L]{Seconde 13}
\fancyhead[C]{\textbf{Probabilités 1 \ifsolutions -- Solutions  \fi}}
\fancyhead[R]{\today}

\exe{
	Donner l'univers $\Omega$ de chacune des expériences aléatoires suivantes.
	\begin{multicols}{2}
	\begin{enumerate}[label=---]
		\item Un lancer de dé équilibré à six faces.
		\item Un lancer de dé pipé (truqué) à six faces.
		\item Un lancer de pièce de monnaie.
		\item Deux lancers de dés  à $6$ faces simultanés.
		\item Deux lancers de dés à $6$ faces, l'un après l'autre.
		\item Couleur d'une carte tirée au hasard parmis un jeu de $52$ cartes.
	\end{enumerate}
	\end{multicols}
}{

	\begin{multicols}{2}
	\begin{enumerate}
		\item $\Omega = \{ 1 ; 2 ; 3 ;4 ;5 ;6\}$.
		\item $\Omega = \{ 1 ; 2 ; 3 ;4 ;5 ;6\}$.
		\item $\Omega = \{ \text{Pile} ; \text{Face} \}$.
		\item $\Omega = \left\{ \{a ; b \} \text{ où } a, b \in \{ 1 ; 2 ; 3 ;4 ;5 ;6\} \right\}$.
		\item $\Omega = \left\{ ( a ;b ) \text{ où } a, b \in \{ 1 ; 2 ; 3 ;4 ;5 ;6\} \right\}$.
		\item $\Omega = \{ \text{Carreau} ; \text{Cœur} ; \text{Trèfle} ; \text{Pique} \}$.
	\end{enumerate}
	\end{multicols}


}

\exe{\label{ex:1}
	On considère le lancer d'un D20, dé à 20 faces numérotées de $1$ à $20$.
	On note le nombre de la face du dessus.
	\begin{enumerate}
		\item Donner l'univers des issues possibles $\Omega$ ainsi que son cardinal $|\Omega|$.
		\item Pour chacun des événements suivants, donner l'ensemble des issues associé ainsi que son cardinal.
		%\begin{multicols}{2}
		\begin{enumerate}[label=\roman*)]
		\item A : \og le nombre est pair \fg
		\item B : \og le nombre est multiple de $3$ \fg
		\item C : \og le nombre est multiple de $5$ \fg
		\item D : \og le nombre est pair ou multiple de $3$ \fg
		\item E : \og le nombre est multiple de $6$ \fg
		\item F : \og le nombre est multiple de 3 et pair \fg
		\end{enumerate}
		%\end{multicols}
	\end{enumerate}
}
{	
	\begin{enumerate}
		\item $\Omega = \{ 1 ; 2 ; \dots ; 19 ; 20 \} = \left\{ i \in \N \text{ tq. } 1 \leq i \leq 20 \right\}$ de cardinal $|\Omega| = 20$.
		\item 
		%\begin{multicols}{2}
		\begin{enumerate}[label=\roman*)]
		\item $A = \{ 2 ; 4 ; \dots ; 18 ; 20 \} = \left\{ i \in \N \text{ tq. } 1 \leq i \leq 20 \text{ et } 2|i \right\}$ de cardinal $|A| = 10$
		\item $B = \{ 3 ; 6 ; \dots ; 15 ; 18 \} = \left\{ i \in \N \text{ tq. } 1 \leq i \leq 20 \text{ et } 3|i \right\}$ de cardinal $|B| = 6$
		\item $C = \{ 5 ; 10 ; 15 ; 20 \} = \left\{ i \in \N \text{ tq. } 1 \leq i \leq 20 \text{ et } 5|i \right\}$ de cardinal $|C| = 4$
		\item $D = \{ 2 ; 3 ; 4 ; 6 ; 8 ; 9 ; 10 ; 12 ; 14 ; 15 ; 16 ; 18 ; 20 \}$ de cardinal $|D| = 13$
		\item $E = \{ 6 ; 12 ; 18 \}$ de cardinal $|E| = 3$.
		\item $F = \{ 6 ; 12 ; 18 \}$ de cardinal $|F| = 3$.
		\end{enumerate}
		%\end{multicols}
	\end{enumerate}
}

\exe{
	On suppose le D20 de l'exercice \ref{ex:1} bien équilibré.
	\begin{enumerate}
		\item Calculer la probabilité de chacun des événements de l'exercice \ref{ex:1}.
		\item Vérifier que $D = A \cup B$.
		\item Vérifier que $E = A \cap B$.
		\item Vérifier l'identité
			\[ P(A \cup B) = P(A) + P(B) - P(A\cap B). \]
	\end{enumerate}
}{
	\begin{enumerate}
		\item 
		Comme le dé est bien équilibré, chaque issue admet la même probabilité, soit $\frac1{\Omega} = \frac1{20}$.
		En outre, d'après le cours, pour un événement $V = \{ e_1 ; e_2 ; \dots \} \subset \Omega$, sa probabilité est la somme des probabilités des issues lui appartenant.
		C'est donc 
			\[ P(V) = \dfrac{|V|}{|\Omega|} = \dfrac{|V|}{20}, \]
		résultat qu'on applique aux $6$ événements de l'exercice \ref{ex:1} pour trouver
			\begin{align*}
				P(A) = \dfrac{10}{20} = \dfrac12 && P(B) = \dfrac{6}{20} = \dfrac3{10} && P(C) = \dfrac15 \\
				P(D) = \dfrac{13}{20} && P(E) = \dfrac{3}{20} && P(F) = \dfrac{3}{20}
			\end{align*}
		\item On vérifie que $D = A \cup B$ en considérant tous les éléments qui appartiennent à $A$ \textbf{ou} à $B$ (ou inclusif : un élément qui appartient aux deux ensembles appartient à l'union) et en vérifiant qu'on trouve bien $D$.
		\item On vérifie que $E = A \cap B$ en considérant tous les éléments qui appartiennent à $A$ \textbf{et} à $B$ et en vérifiant qu'on trouve bien $E$.
		\item 
		Le membre de gauche de l'identité donne
			\[ P (A \cup B) = P(D) = \dfrac{13}{20}, \]
		d'après les questions précédentes.
		Le membre de droite de l'identité est égal à 
			\[ P(A) + P(B) - P(A\cap B) = \dfrac{10}{20} + \dfrac{6}{20}  - \dfrac{3}{20} = \dfrac{13}{20}.\]
		L'identité est donc bien vérifée dans ce cas !
	\end{enumerate}
}

\exe{
	Compléter le tableau de probabilités suivant, concernant le numéro de la face du dessus obtenue après un lancer d'un D6 pipé.
	\begin{center}
	\begin{tabular}{|c|c|c|c|c|c|c|} \hline
		Résultat & 1 & 2 & 3 & 4 & 5 & 6 \\ \hline
		Probabilité & $0,1$ & $0,2$ & $0,1$ & $0,15$ & $0,25$ & \ifsolutions $\color{red} 0,2$ \fi \\ \hline
	\end{tabular}
	\end{center}
	
	Calculer les probabilités suivantes.
		\begin{enumerate}[label=\roman*)]
			\item P(\og obtenir un nombre pair \fg) 
			\item P(\og obtenir un nombre impair \fg) 
			\item P(\og obtenir un nombre pair \fg)   + P(\og obtenir un nombre impair \fg) 
			\item P(\og obtenir $2$ ou $5$ \fg) 
			\item P(\og obtenir ni $2$ ni $5$ \fg) 
			\item P(\og obtenir $2$ ou $5$ \fg)  + P(\og obtenir ni $2$ ni $5$ \fg) 
		\end{enumerate}
}{
	On complète d'abord le tableau en sachant que l'univers de l'expérience est $\Omega = \{ 1 ; 2 ;3 ; 4 ; 5 ; 6 \}$ et que la somme des probabilités du tableau est donc nécessairement $1$.
	En effet, $P(\Omega) = 1$, car $\Omega$ correspond à l'événement \og obtenir une des issues possibles \fg, qui est un événement sûr.

	\begin{enumerate}[label=\roman*)]
		\item P(\og obtenir un nombre pair \fg) $ = P(2) + P(4) + P(6) = 0,2 + 0,15 + 0,2 = 0,55$
		\item P(\og obtenir un nombre impair \fg) $ = P(1) + P(3) + P(5) = 0,1 + 0,1 + 0,25 = 0,45$.
		\item P(\og obtenir un nombre pair \fg)   + P(\og obtenir un nombre impair \fg) $ = 0,55 + 0,45 = 1$
		\item P(\og obtenir $2$ ou $5$ \fg) $ = P(2) + P(5) = 0,2 + 0,25 = 0,45$
		\item P(\og obtenir ni $2$ ni $5$ \fg) $ = P(1) + P(3) + P(4) + P(6) = 0,1 + 0,1 + 0,15 + 0,2 = 0,55$
		\item P(\og obtenir $2$ ou $5$ \fg)  + P(\og obtenir ni $2$ ni $5$ \fg) $ = 0,45 + 0,55 = 1$
	\end{enumerate}
	
	Le fait que la somme soit $1$ n'est pas suprenant : les événements comptent séparément toutes les issues possibles.
	On appelle ces événements \emph{complémentaires} : ils sont disjoints (les deux événements ne peuvent pas arriver en même temps), et ensembles ils forment l'univers tout entier (chaque issue appartient à un événement ou à l'autre).
	
	Les derniers événements motivent la relation 
		\[ \overline{A \cup B} = \overline{A} \cap \overline{B} \]
	qui, avec des mots, dit
		\begin{center}
			\og non (A ou B) \fg $=$ \og ni A, ni B \fg $=$ \og (non A) et (non B) \fg.
		\end{center}
}

\exe{
	On lance un D6 pipé avant de noter le numéro de la face du dessus.
	Les probabilités de chacunes des issues vérifient
		\[ P(1) = 2 P(2) = P(3) = 2 P(4) = P(5) = 2P(6). \]
	Calculer $P($\og le résultat est divisible par $3$ \fg$)$.
}{
	Posons $p=P(2) \in [0;1]$ et exprimons chaque probabilité en fonction de $p$.
		\begin{align*}
			P(1) = 2p && P(2) = p && P(3) = 2p && P(4) = p && P(5) = 2p && P(6) = p.
		\end{align*}
	La relation $P(\Omega) = 1$ nous donne l'équation suivante à résoudre pour $p$.
		\begin{align*}
			P(1) + P(2) + P(3) + P(4) + P(5) + P (6) &= 1 \\
			2p + p + 2p + p + 2p + p &= 1 \\
			9p &= 1 \\
			p &= \dfrac19
		\end{align*}
	On en déduit le tableau de probabilités suivant.
	\begin{center}
	\begin{tabular}{|c|c|c|c|c|c|c|} \hline
		Résultat & 1 & 2 & 3 & 4 & 5 & 6 \\ \hline
		Probabilité & $\dfrac29$ & $\dfrac19$ & $\dfrac29$ & $\dfrac19$ & $\dfrac29$ & $\dfrac19$ \\ \hline
	\end{tabular}
	\end{center}
	Et on conclut que $P($\og le résultat est divisible par $3$ \fg$) = P(3) + P(6) = \dfrac29 + \dfrac19 = \dfrac13.$
}

\exe{
	L'univers associé à une expérience aléatoire est $\{ a, b, c\}$.
	La loi de probabilité $P$ vérifie $P(a) = t^2$, $P(b) = -t$, et $P(c) = \frac14$, pour un réel $t \in \R$.
	
	Développer le carré $\left(t-\frac12\right)^2$ et déterminer $t$.
}{
	On développe le carré à l'aide de l'identité remarquable
		\[ (a-b)^2 = a^2 + b^2 - 2ab, \]
	où, ici, on a $a=t$ et $b=\frac12.$
		\begin{align*}
			\left(t-\dfrac12\right)^2 &= t^2 + \left(\dfrac12\right)^2 - 2 \cdot t \cdot \dfrac12 \\
									&= t^2 + \dfrac14 - t
		\end{align*}
	On cherche désormais le $t\in\R$ pour lequel $P$ est une loi de probabilité. 
	Un loi vérifie les deux propriétés suivantes :
		\begin{itemize}
			\item $P(\omega) \in [0;1]$ pour chaque issue $\omega \in \Omega$ ; et
			\item $P(\Omega) = 1$.
		\end{itemize}
	La deuxième identité donne donc
		\begin{align*}
			P(a) + P(b) + P(c) = 1 && \iff && t^2 - t + \dfrac14 = 1.
		\end{align*}
	Le carré développé nous permet d'écrire
		\[ \left(t-\dfrac12\right)^2 = 1, \]
	et donc
		\[ \left|t-\dfrac12\right| = \sqrt{1} = 1, \]
	en utilisant le fait que $\sqrt{x^2} = |x|.$
	L'expression à l'intérieur de la valeur absolue est donc soit $+1$, soit $-1$, et on a donc deux alternatives :
		\begin{align*}
			t-\dfrac12 = 1 && \text{ ou } && t - \dfrac12 = -1 \\
			t = \dfrac32 && \text{ ou } && t = -\dfrac12.
		\end{align*}
	Pour s'entraîner à ce genre de résolution, voir la feuille d'exercices Fonctions 3.
	
	Comme les probabilités sont des nombres entre $0$ et $1$, on peut écarter la première solution car $P(a) = t^2$ serait strictement supérieur à $1$, et $P(b) = -t$, serait strictement négatif.
	Il ne reste donc que $t = -\frac12$, qui donne le tableau de probabilités suivant.
	\begin{center}
	\begin{tabular}{|c|c|c|c|} \hline
		Issue & $a$ & $b$ & $c$ \\ \hline
		Probabilité & $\frac14$ & $\frac12$ & $\frac14$ \\ \hline
	\end{tabular}
	\end{center}
}

\exe{
	Une joueuse de tennis a une probabilité $0,58$ de réussir son premier service et une probabilité de $0,06$ de faire une double faute.
	Quelle est la probabilité qu'elle réussisse seulement son deuxième service ?
}{
 	Les issues du service sont $\Omega = \{ \text{Premier service réussi} ; \text{Premier service raté, deuxième réussi} ; \text{Premier et deuxième services ratés} \}$.
	 La somme des probabilité vaut $1$ ($P(\Omega) = 1$), et donc on a nécessairement
	 	\[ P(\text{Premier service raté, deuxième réussi}) = 1 - 0,58 - 0,06 = 0,36. \]
}

\exe{
	Dessiner des diagrammes de Venn pour motiver les relations suivantes.
		\begin{align*}
			\overline{A\cup B} = \overline{A} \cap \overline{B}, && \text{et} && \overline{A \cap B} = \overline{A} \cup \overline{B}.
		\end{align*}

}{
	En classe.
}

\exe{
	On considère un lancer d'un D$300$ bien équilibré, dé à $300$ faces numérotées de $1$ à $300$.
	Après le lancer, on note le numéro de la face supérieure.
	Posons $A$ : \og le résultat n'est pas multiple de $10$ \fg, et $B$ : \og le résultat n'est pas divisible par $7$ \fg.
	
	\begin{enumerate}
		\item Exprimer avec des mots les événements suivants.
			\begin{multicols}{2}
			\begin{enumerate}[label=\roman*)]
				\item $\overline{A}$
				\item $\overline{B}$
				\item $\overline{A} \cap \overline{B}$
				\item $\overline{A} \cup \overline{B}$
				%\item $\overline{A \cup B}$
				%\item $\overline{A \cap B}$
			\end{enumerate}
			\end{multicols}
		\item Calculer les probabilités $P\left( \overline{A} \right)$ et $P\left( \overline{B} \right)$.
		\item Calculer les probabilités $P\left(\overline{A} \cap \overline{B}\right)$ puis $P\left(\overline{A} \cup \overline{B}\right)$.
		\item Calculer la probabilité que le résultat obtenu soit supérieur ou égal à $10$.
	\end{enumerate}
}{
	\begin{enumerate}
		\item 
			\begin{enumerate}[label=\roman*)]
				\item $\overline{A}$ : \og le résultat est multiple de $10$ \fg
				\item $\overline{B}$ : \og le résultat est divisible par $7$ \fg
				\item $\overline{A} \cap \overline{B}$  : \og le résultat est multiple de $10$ \textbf{et} divisible par $7$ \fg
				\item $\overline{A} \cup \overline{B}$ : \og le résultat est multiple de $10$ \textbf{ou} divisible par $7$ \fg
			\end{enumerate}
		\item 
		Le dé étant bien équilibré, nous sommes en situation d'équiprobabilité et 
			\[ P(E) = \dfrac{|E|}{|\Omega|} = \dfrac{|E|}{300} \]
		pour n'importe quel événement $E\subseteq\Omega$.
		Il s'agit donc de calculer le cardinal de $ \overline{A}$ et de $ \overline{B}$, c'est-à-dire le nombre d'entiers entre $1$ et $300$ qui sont multiples de $10$ puis qui sont multiples de $7$.
		
		D'une part, les multiples de $10$ s'écrivent
			\begin{align*}
				10 \times 1 && 10 \times 2 && 10 \times 3 && \dots && 10 \times 30 = 300,
			\end{align*}
		et donc $|\overline{A}| = 30$, et $P\left( \overline{A} \right) = \dfrac{30}{300} = \dfrac1{10}$.
		
		D'autre part, les multiples de $7$ s'écrivent
			\begin{align*}
				7 \times 1 && 7\times 2 && 7 \times 3 && \dots
			\end{align*}
		Pour savoir jusqu'où aller, on calcule $\dfrac{300}{7} \approx 42,86$, donc la liste s'arrête à $7 \times 42 = 294$.
		Par conséquent, $|\overline{B}| = 42$, et $P\left( \overline{B} \right) = \dfrac{42}{300} = \dfrac{7}{50}$.
		
		\item 
		On calcule d'abord $P\left(\overline{A} \cap \overline{B}\right)$ en comptant les multiples de $10$ et de $7$ inférieurs à $300$.
		Ceux-ci sont $\{ 70 ; 140 ; 210 ; 280 \}$ par énumération (et en fait car $7$ et $10$ sont premiers entre eux).
		D'où $P\left(\overline{A} \cap \overline{B}\right) = \dfrac{4}{300} = \dfrac{1}{75}$.
		
		On utilise l'inclusion-exclusion pour conclure :
			\begin{align*}
				P\left(\overline{A} \cup \overline{B}\right) &= P\left( \overline{A} \right) + P\left( \overline{B} \right) - P\left(\overline{A} \cap \overline{B}\right) \\
																&= \dfrac{30}{300} + \dfrac{42}{300} -  \dfrac{4}{300} \\
																&= \dfrac{68}{300} = \dfrac{17}{75}.
			\end{align*}
	
		Remarquons qu'on a ainsi compté le nombre d'entier entre $1$ et $300$ qui sont multiples de $7$ ou de $10$ : il y en a $68$.

		\item 
		L'événement correspond à l'ensemble 
			\[ \{10 ; 11 ; 12 ; \dots ; 299 ; 300 \}, \]
		de cardinal $300-10+1 = 291$ (pour se convaincre de la nécessité du $+1$, essayer de calculer le cardinal de $\{10 ; 11\}$, puis $\{10 ; 11 ; 12 \}$, etc...).
		En conclusion,
			\[ P(\text{\og résultat supérieur ou égal à $10$ \fg}) = \dfrac{291}{300} = \dfrac{97}{100}. \]
	\end{enumerate}

}

\exe{
	On lance deux D$6$ équilibrés, dés à $6$ faces, au même moment. Les deux dés sont absolument identiques.
	\begin{enumerate}
		\item Donner l'univers $\Omega$ et son cardinal $|\Omega|$. Est-ce une situation d'équiprobabilité ?
		\item Quelle est la probabilité d'obtenir un double $6$ ?
		\item Quelle est la probabilité d'obtenir deux résultats différents ?
	\end{enumerate}
}{	
	\begin{enumerate}
		\item
		L'ordre des résultats n'importe pas car les dés sont identiques et les jets effectués au même moment.		
		On a donc 
		\begin{align*}
			\Omega &= \left\{ \{a ; b \} \text{ où } a, b \in \{ 1 ; 2 ; 3 ;4 ;5 ;6\} \right\} \\
					&= \left\{
						\begin{aligned}
						&\{1 ; 1\}, \{1 ; 2\}, \{1 ; 3\}, \{1 ; 4\}, \{1 ; 5\}, \{1 ; 6\}, \\
						&\{2 ; 2\}, \{2 ; 3\},\{2 ; 4\}, \{2 ; 5\}, \{2 ; 6\}, \\
						& \{3 ; 3\}, \{3 ; 4\}, \{3 ; 5\}, \{3 ; 6\}, \\
						& \{4 ; 4\}, \{4 ; 5\}, \{4 ; 6\}, \\
						& \{5 ; 5\}, \{5 ; 6\}, \\
						& \{6 ; 6\}
						\end{aligned}
						\right\}
		\end{align*}
		On compte $|\Omega| = 6 + 5 + 4 + 3 + 2 + 1 = 21$.
		
		Un autre façon de compter sans énumérer est la suivante : chaque paire de résultats distincts est comptée deux fois au lieu d'une. 
		Il y a $6$ paires de résultats identiques, donc $36-6 = 30$ paires de résultats distincts.
		Par conséquent, $|\Omega| = \dfrac{30}2 + 6 = 21$.
		
		Pour comprendre la probabilité de chaque événement, il est possible de différencier les deux dés (par exemple en leur assignant une couleur), et de regrouper les issues qui donnent le même résultat lorsque les dés sont identiques.
		Par exemple, \og obtenir $5$ puis obtenir $6$\fg et \og obtenir $6$ puis obtenir $5$ \fg sont deux issues différentes équiprobables si les dés sont différentiables, mais pas si les dés sont identiques et les lancers simultanés.
		La situation n'est donc pas une situation d'équiprobabilité car toutes les issues n'ont pas la même probabilité d'arriver.
		Par exemple,
			\[ P( \{2 ; 2\} ) = \dfrac1{36} \qquad \text{ et } \qquad P(\{5 ; 6\}) = \dfrac2{36} = \dfrac1{18}, \]
		car il y a deux façons d'obtenir l'issue $\{5 ; 6\}$.
		\item Il n'y a qu'une seule façon d'obtenir un double $6$ : que les deux dés résultent en un $6$.
		Ainsi $P( \{6 ; 6\} ) = \dfrac16 \times \dfrac16 = \dfrac1{36}$.
		\item 
		On additionne 
			\[ P( \{1 ; 1\} ) + P( \{2 ; 2\} )+ P( \{3 ; 3\} )+ P( \{4 ; 4\} )+ P( \{5 ; 5\} )+ P( \{6 ; 6\} ) = \dfrac{6}{36} = \dfrac16. \]
	\end{enumerate}
}

\exe{
	On lance $3$ fois de suite une pièce de monnaie bien équilibrée.
	On note par $P$ (pile) ou $F$ (face) le résultat de chaque lancer.
	Donner $\Omega$, l'univers de l'expérience, et $|\Omega|$ son cardinal.
	
	Calculer la probabilité des événements suivants.
		\begin{enumerate}
			\item Obtenir $3$ fois face.
			\item Le deuxième lancer donne pile.
			\item Le troisième lancer est différent du premier.
			\item On obtient au moins une fois pile.
		\end{enumerate}
}{
	Comme les lancers sont distingués, il y a $8$ issues possibles.
		\[ \Omega = \{ FFF ; FFP ; FPF ; FPP ; PFF ; PFP ; PPF ; PPP \} \]
	Le cardinal de l'univers est $|\Omega| = 8$.
	On aurait pû aussi noter les issues avec des parenthèses, p.ex. $(F;P;F)$, mais pas avec des accolades $\{ \cdot \}$.
	\begin{enumerate}
		\item
		Les probabilités se multiplient, on a donc $P(FFF) = \dfrac12 \times \dfrac12 \times \dfrac12 = \left( \dfrac12 \right)^3 = \dfrac18$.
		En fait, nous sommes en situation d'équiprobabilité, et $|\Omega| = 8$.
		\item 
		Les lancers sont indépendants (le résultat des précédents n'influe en rien celui des prochains), donc la probabilité que le deuxième donne pile est $\frac12$.
		On aurait également pu sommer la probabilité des événements concernés :
			\[ P(FPF) + P(PPF) +  P(FPP) + P(PPP) = \dfrac48 = \dfrac12. \]
		\item
		Il y a quatre issues qui correspondent à cet événement. 
			\[ P(PFF) + P(PPF) + P(FFP) + P(FPP) = \dfrac48 = \dfrac12. \]
		On aurait pû tout aussi bien supprimer le deuxième lancer, car il n'a aucune influence sur les autres --- cela donne le même résultat.
		\item 
		Lorsqu'on étudie un événement de la forme \og au moins [\dots] \fg, il est toujours utile de passer par le complémentaire.
		L'événement complémentaire est \og on obtient trois fois face \fg, dont la probabilité est $P(FFF) = \dfrac18.$
		La probabilité recherché est donc $1-\dfrac18 = \dfrac78$.
		
		On aurait également pû énumérer les issues de l'événement et sommer leur probabilité. 
		Seul l'événement $FFF$ n'apparaît alors pas dans cette somme qui vaut $\dfrac78$.
	\end{enumerate}
}

\exe{
	On lance deux D$6$ équilibrés, dés à $6$ faces l'un après l'autre. Les deux dés sont distinguables car de couleurs différentes.
	\begin{enumerate}
		\item Donner l'univers $\Omega$ et son cardinal $|\Omega|$. Est-ce une situation d'équiprobabilité ?
		\item Quelle est la probabilité d'obtenir un double $6$ ?
		\item Quelle est la probabilité qu'après $10$ tels lancers, on obtienne au moins une fois un double $6$ ?
	\end{enumerate}
}{
	\begin{enumerate}
		\item Donner l'univers $\Omega$ et son cardinal $|\Omega|$. Est-ce une situation d'équiprobabilité ?
		L'univers est formé par tous les couples $(a ;b)$ de résultats.
		On utilise des parenthèses ici car on distingue le premier du deuxième lancer.
			\[ \Omega = \left\{ (a ; b) \text{ où } a, b \in \{ 1 ; 2 ; 3 ; 4 ;5 ; 6 \} \right\}, \]
		de cardinal $|\Omega| = 6 \times 6 = 36$.
		
		La situation est bien d'équiprobabilité car il y a $36$ issues et chacune admet comme probabilité $\dfrac16 \times \dfrac16 = \dfrac1{36}$, car les dés sont bien équilibrés.
		
		\item 
		La probabilité de l'issue $(6;6)$ est $\dfrac1{36}$ par équiprobabilité.
		
		\item Quelle est la probabilité qu'après $10$ tels lancers, on obtienne au moins une fois un double $6$ ?
		Lorsqu'on étudie un événement de la forme \og au moins [\dots] \fg, il est toujours utile de passer par le complémentaire.
		L'événement complémentaire est \og obtenir aucun double $6$ \fg, dont la probabilité est
			\[ \left( \dfrac{35}{36} \right)^{10} \approx 0,75. \]
		En effet, on peut construire un arbre réduit à deux événements  : \og double 6 \fg (probabilité $\frac1{36}$) et \og pas double 6 \fg (probabilité $\frac{35}{36}$), de profondeur $10$.
		La feuille qui correspond à \og obtenir aucun double $6$ \fg est obtenue en obtenant \og pas double 6 \fg dix fois dans l'arbre.
		
		La probabilité de l'événement \og on obtient aucun double $6$ \fg est donc
			\[ \left( \dfrac{35}{36} \right)^{10} \approx 0,75. \]
		On conclut en faisant $1-0,75 = 0,25 = \frac14$, probabilité approximative qu'au moins un des $10$ lancers donne un double $6$.
	\end{enumerate}


}

\end{document}

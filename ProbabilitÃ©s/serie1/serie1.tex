				% ENABLE or DISABLE font change
				% use XeLaTeX if true
\newif\ifdys
				\dystrue
				\dysfalse

\newif\ifsolutions
				\solutionstrue
				\solutionsfalse

% DYSLEXIA SWITCH
\newif\ifdys
		
				% ENABLE or DISABLE font change
				% use XeLaTeX if true
				\dystrue
				\dysfalse


\ifdys

\documentclass[a4paper, 14pt]{extarticle}
\usepackage{amsmath,amsfonts,amsthm,amssymb,mathtools}

\tracinglostchars=3 % Report an error if a font does not have a symbol.
\usepackage{fontspec}
\usepackage{unicode-math}
\defaultfontfeatures{ Ligatures=TeX,
                      Scale=MatchUppercase }

\setmainfont{OpenDyslexic}[Scale=1.0]
\setmathfont{Fira Math} % Or maybe try KPMath-Sans?
\setmathfont{OpenDyslexic Italic}[range=it/{Latin,latin}]
\setmathfont{OpenDyslexic}[range=up/{Latin,latin,num}]

\else

\documentclass[a4paper, 12pt]{extarticle}

\usepackage[utf8x]{inputenc}
%fonts
\usepackage{amsmath,amsfonts,amsthm,amssymb,mathtools}
% comment below to default to computer modern
\usepackage{libertinus,libertinust1math}

\fi


\usepackage[french]{babel}
\usepackage[
a4paper,
margin=2cm,
nomarginpar,% We don't want any margin paragraphs
]{geometry}
\usepackage{icomma}

\usepackage{fancyhdr}
\usepackage{array}
\usepackage{hyperref}

\usepackage{multicol, enumerate}
\newcolumntype{P}[1]{>{\centering\arraybackslash}p{#1}}


\usepackage{stackengine}
\newcommand\xrowht[2][0]{\addstackgap[.5\dimexpr#2\relax]{\vphantom{#1}}}

% theorems

\theoremstyle{plain}
\newtheorem{theorem}{Th\'eor\`eme}
\newtheorem*{sol}{Solution}
\theoremstyle{definition}
\newtheorem{ex}{Exercice}
\newtheorem*{rpl}{Rappel}
\newtheorem{enigme}{Énigme}

% corps
\usepackage{calrsfs}
\newcommand{\C}{\mathcal{C}}
\newcommand{\R}{\mathbb{R}}
\newcommand{\Rnn}{\mathbb{R}^{2n}}
\newcommand{\Z}{\mathbb{Z}}
\newcommand{\N}{\mathbb{N}}
\newcommand{\Q}{\mathbb{Q}}

% variance
\newcommand{\Var}[1]{\text{Var}(#1)}

% domain
\newcommand{\D}{\mathcal{D}}


% date
\usepackage{advdate}
\AdvanceDate[0]


% plots
\usepackage{pgfplots}

% table line break
\usepackage{makecell}
%tablestuff
\def\arraystretch{2}
\setlength\tabcolsep{15pt}

%subfigures
\usepackage{subcaption}

\definecolor{myg}{RGB}{56, 140, 70}
\definecolor{myb}{RGB}{45, 111, 177}
\definecolor{myr}{RGB}{199, 68, 64}

% fake sections with no title to move around the merged pdf
\newcommand{\fakesection}[1]{%
  \par\refstepcounter{section}% Increase section counter
  \sectionmark{#1}% Add section mark (header)
  \addcontentsline{toc}{section}{\protect\numberline{\thesection}#1}% Add section to ToC
  % Add more content here, if needed.
}


% SOLUTION SWITCH
\newif\ifsolutions
				\solutionstrue
				%\solutionsfalse

\ifsolutions
	\newcommand{\exe}[2]{
		\begin{ex} #1  \end{ex}
		\begin{sol} #2 \end{sol}
	}
\else
	\newcommand{\exe}[2]{
		\begin{ex} #1  \end{ex}
	}
	
\fi


% tableaux var, signe
\usepackage{tkz-tab}


%pinfty minfty
\newcommand{\pinfty}{{+}\infty}
\newcommand{\minfty}{{-}\infty}

\begin{document}


\AdvanceDate[0]

\begin{document}
\pagestyle{fancy}
\fancyhead[L]{Seconde 13}
\fancyhead[C]{\textbf{Probabilités 1 \ifsolutions -- Solutions  \fi}}
\fancyhead[R]{\today}

\exe{
	Donner l'univers $\Omega$ de chacune des expériences aléatoires suivantes.
	\begin{multicols}{2}
	\begin{enumerate}[label=---]
		\item Un lancer de dé équilibré à six faces.
		\item Un lancer de dé pipé (truqué) à six faces.
		\item Un lancer de pièce de monnaie.
		\item Deux lancers de dés  à $6$ faces simultanés.
		\item Deux lancers de dés à $6$ faces, l'un après l'autre.
		\item Couleur d'une carte tirée au hasard parmis un jeu de $52$ cartes.
	\end{enumerate}
	\end{multicols}
}{}

\exe{\label{ex:1}
	On considère le lancer d'un D20, dé à 20 faces numérotées de $1$ à $20$.
	On note le nombre de la face du dessus.
	\begin{enumerate}
		\item Donner l'univers des issues possibles $\Omega$ ainsi que son cardinal $|\Omega|$.
		\item Pour chacun des événements suivants, donner l'ensemble des issues associé ainsi que son cardinal.
		%\begin{multicols}{2}
		\begin{enumerate}[label=\roman*)]
		\item A : \og le nombre est pair \fg
		\item B : \og le nombre est multiple de $3$ \fg
		\item C : \og le nombre est multiple de $5$ \fg
		\item D : \og le nombre est pair ou multiple de $3$ \fg
		\item E : \og le nombre est multiple de $6$ \fg
		\item F : \og le nombre est multiple de 3 et pair \fg
		\end{enumerate}
		%\end{multicols}
	\end{enumerate}
}
{}

\exe{
	On suppose le D20 de l'exercice \ref{ex:1} bien équilibré.
	\begin{enumerate}
		\item Calculer la probabilité de chacun des événements de l'exercice \ref{ex:1}.
		\item Vérifier que $D = A \cup B$.
		\item Vérifier que $E = A \cap B$.
		\item Vérifier l'identité
			\[ P(A \cup B) = P(A) + P(B) - P(A\cap B). \]
	\end{enumerate}
}{}

\exe{
	On lance un D6 pipé avant de noter le numéro de la face du dessus.
	Les probabilités de chacunes des issues vérifient
		\[ P(1) = 2 P(2) = P(3) = 2 P(4) = P(5) = 2P(6). \]
	Calculer $P($\og le résultat est divisible par $3$ \fg$)$.
}{}

\exe{
	Compléter le tableau de probabilités suivant, concernant le numéro de la face du dessus obtenue après un lancer d'un D6 pipé.
	\begin{center}
	\begin{tabular}{|c|c|c|c|c|c|c|} \hline
		Résultat & 1 & 2 & 3 & 4 & 5 & 6 \\ \hline
		Probabilité & $0,1$ & $0,2$ & $0,1$ & $0,15$ & $0,25$ & \\ \hline
	\end{tabular}
	\end{center}
	
	Calculer les probabilités suivantes.
		\begin{enumerate}[label=\roman*)]
			\item P(\og obtenir un nombre pair \fg) 
			\item P(\og obtenir un nombre impair \fg) 
			\item P(\og obtenir un nombre pair \fg)   + P(\og obtenir un nombre impair \fg) 
			\item P(\og obtenir $2$ ou $5$ \fg) 
			\item P(\og obtenir ni $2$ ni $5$ \fg) 
			\item P(\og obtenir $2$ ou $5$ \fg)  + P(\og obtenir ni $2$ ni $5$ \fg) 
		\end{enumerate}
}{}

\exe{
	L'univers associé à une expérience aléatoire est $\{ a, b, c\}$.
	La loi de probabilité $P$ vérifie $P(a) = t^2$, $P(b) = -t$, et $P(c) = \frac14$, pour un réel $t \in \R$.
	
	Développer le carré $\left(t-\frac12\right)^2$ et déterminer $t$.
}{}

\exe{
	Une joueuse de tennis a une probabilité $0,58$ de réussir son premier service et une probabilité de $0,06$ de faire une double faute.
	Quelle est la probabilité qu'elle réussisse seulement son deuxième service ?
}{}

\exe{
	Dessiner des diagrammes de Venn pour motiver les relations suivantes.
		\begin{align*}
			\overline{A\cup B} = \overline{A} \cap \overline{B}, && \text{et} && \overline{A \cap B} = \overline{A} \cup \overline{B}.
		\end{align*}

}{}

\exe{
	On considère un lancer d'un D$300$ bien équilibré, dé à $300$ faces numérotées de $1$ à $300$.
	Après le lancer, on note le numéro de la face supérieure.
	Posons $A$ : \og le résultat est n'est pas multiple de $10$ \fg, et $B$ : \og le résultat n'est pas divisible par $7$ \fg.
	
	\begin{enumerate}
		\item Exprimer avec des mots les événements suivants.
			\begin{multicols}{2}
			\begin{enumerate}[label=\roman*)]
				\item $\overline{A}$
				\item $\overline{B}$
				\item $\overline{A} \cap \overline{B}$
				\item $\overline{A} \cup \overline{B}$
				%\item $\overline{A \cup B}$
				%\item $\overline{A \cap B}$
			\end{enumerate}
			\end{multicols}
		\item Calculer les probabilités $P\left( \overline{A} \right)$ et $P\left( \overline{B} \right)$.
		\item Calculer les probabilités $P\left(\overline{A} \cap \overline{B}\right)$ puis $P\left(\overline{A} \cup \overline{B}\right)$.
		\item Calculer la probabilité que le résultat obtenu est supérieur ou égal à $10$.
	\end{enumerate}
}{}


\exe{
	On lance deux D$6$ équilibrés, dés à $6$ faces l'un après l'autre. Les deux dés sont distinguables car de couleurs différentes.
	\begin{enumerate}
		\item Donner l'univers $\Omega$ et son cardinal $|\Omega|$. Est-ce une situation d'équiprobabilité ?
		\item Quelle est la probabilité d'obtenir un double $6$ ?
		\item Quelle est la probabilité qu'après $10$ tels lancers, on obtienne au moins une fois un double $6$ ?
	\end{enumerate}
}{}

\exe{
	On lance deux D$6$ équilibrés, dés à $6$ faces, au même moment. Les deux dés sont absolument identiques.
	\begin{enumerate}
		\item Donner l'univers $\Omega$ et son cardinal $|\Omega|$. Est-ce une situation d'équiprobabilité ?
		\item Quelle est la probabilité d'obtenir un double $6$ ?
		\item Quelle est la probabilité d'obtenir deux résultats différents ?
	\end{enumerate}
}{}

\exe{
	On lance $3$ fois de suite une pièce de monnaie bien équilibrée.
	On note par $P$ (pile) ou $F$ (face) le résultat de chaque lancer.
	Donner $\Omega$, l'univers de l'expérience, et $|\Omega|$ son cardinal.
	
	Calculer la probabilité des événements suivants.
		\begin{enumerate}
			\item Obtenir $3$ fois face.
			\item Le deuxième lancer donne pile.
			\item Le troisième lancer est différent du premier.
			\item On obtient au moins une fois pile.
		\end{enumerate}
}{}

\end{document}

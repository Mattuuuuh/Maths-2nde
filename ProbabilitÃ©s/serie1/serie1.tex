				% ENABLE or DISABLE font change
				% use XeLaTeX if true
\newif\ifdys
				\dystrue
				\dysfalse

\newif\ifsolutions
				\solutionstrue
				\solutionsfalse

% DYSLEXIA SWITCH
\newif\ifdys
		
				% ENABLE or DISABLE font change
				% use XeLaTeX if true
				\dystrue
				\dysfalse


\ifdys

\documentclass[a4paper, 14pt]{extarticle}
\usepackage{amsmath,amsfonts,amsthm,amssymb,mathtools}

\tracinglostchars=3 % Report an error if a font does not have a symbol.
\usepackage{fontspec}
\usepackage{unicode-math}
\defaultfontfeatures{ Ligatures=TeX,
                      Scale=MatchUppercase }

\setmainfont{OpenDyslexic}[Scale=1.0]
\setmathfont{Fira Math} % Or maybe try KPMath-Sans?
\setmathfont{OpenDyslexic Italic}[range=it/{Latin,latin}]
\setmathfont{OpenDyslexic}[range=up/{Latin,latin,num}]

\else

\documentclass[a4paper, 12pt]{extarticle}

\usepackage[utf8x]{inputenc}
%fonts
\usepackage{amsmath,amsfonts,amsthm,amssymb,mathtools}
% comment below to default to computer modern
\usepackage{libertinus,libertinust1math}

\fi


\usepackage[french]{babel}
\usepackage[
a4paper,
margin=2cm,
nomarginpar,% We don't want any margin paragraphs
]{geometry}
\usepackage{icomma}

\usepackage{fancyhdr}
\usepackage{array}
\usepackage{hyperref}

\usepackage{multicol, enumerate}
\newcolumntype{P}[1]{>{\centering\arraybackslash}p{#1}}


\usepackage{stackengine}
\newcommand\xrowht[2][0]{\addstackgap[.5\dimexpr#2\relax]{\vphantom{#1}}}

% theorems

\theoremstyle{plain}
\newtheorem{theorem}{Th\'eor\`eme}
\newtheorem*{sol}{Solution}
\theoremstyle{definition}
\newtheorem{ex}{Exercice}
\newtheorem*{rpl}{Rappel}
\newtheorem{enigme}{Énigme}

% corps
\usepackage{calrsfs}
\newcommand{\C}{\mathcal{C}}
\newcommand{\R}{\mathbb{R}}
\newcommand{\Rnn}{\mathbb{R}^{2n}}
\newcommand{\Z}{\mathbb{Z}}
\newcommand{\N}{\mathbb{N}}
\newcommand{\Q}{\mathbb{Q}}

% variance
\newcommand{\Var}[1]{\text{Var}(#1)}

% domain
\newcommand{\D}{\mathcal{D}}


% date
\usepackage{advdate}
\AdvanceDate[0]


% plots
\usepackage{pgfplots}

% table line break
\usepackage{makecell}
%tablestuff
\def\arraystretch{2}
\setlength\tabcolsep{15pt}

%subfigures
\usepackage{subcaption}

\definecolor{myg}{RGB}{56, 140, 70}
\definecolor{myb}{RGB}{45, 111, 177}
\definecolor{myr}{RGB}{199, 68, 64}

% fake sections with no title to move around the merged pdf
\newcommand{\fakesection}[1]{%
  \par\refstepcounter{section}% Increase section counter
  \sectionmark{#1}% Add section mark (header)
  \addcontentsline{toc}{section}{\protect\numberline{\thesection}#1}% Add section to ToC
  % Add more content here, if needed.
}


% SOLUTION SWITCH
\newif\ifsolutions
				\solutionstrue
				%\solutionsfalse

\ifsolutions
	\newcommand{\exe}[2]{
		\begin{ex} #1  \end{ex}
		\begin{sol} #2 \end{sol}
	}
\else
	\newcommand{\exe}[2]{
		\begin{ex} #1  \end{ex}
	}
	
\fi


% tableaux var, signe
\usepackage{tkz-tab}


%pinfty minfty
\newcommand{\pinfty}{{+}\infty}
\newcommand{\minfty}{{-}\infty}

\begin{document}


\AdvanceDate[0]

\begin{document}
\pagestyle{fancy}
\fancyhead[L]{Seconde 13}
\fancyhead[C]{\textbf{Probabilités 1 \ifsolutions -- Solutions  \fi}}
\fancyhead[R]{\today}

\exe{
	Donner l'univers $\Omega$ de chacune des expériences aléatoires suivantes.
	\begin{multicols}{2}
	\begin{enumerate}[label=---]
		\item Un lancer de dé équilibré à six faces.
		\item Un lancer de dé pipé (truqué) à six faces.
		\item Un lancer de pièce de monnaie.
		\item Deux lancers de dés  à $6$ faces simultanés.
		\item Deux lancers de dés à $6$ faces, l'un après l'autre.
		\item Couleur d'une carte tirée au hasard parmis un jeu de $52$ cartes.
	\end{enumerate}
	\end{multicols}
}{}

\exe{\label{ex:1}
	On considère le lancer d'un D20, dé à 20 faces numérotées de $1$ à $20$.
	On note le nombre de la face du dessus.
	\begin{enumerate}
		\item Donner l'univers des issues possibles $\Omega$ ainsi que son cardinal $|\Omega|$.
		\item Pour chacun des événements suivants, donner l'ensemble des issues associé ainsi que son cardinal.
		%\begin{multicols}{2}
		\begin{enumerate}[label=\roman*)]
		\item A : \og le nombre est pair \fg
		\item B : \og le nombre est multiple de $3$ \fg
		\item C : \og le nombre est multiple de $5$ \fg
		\item D : \og le nombre est pair ou multiple de $3$ \fg
		\item E : \og le nombre est multiple de $6$ \fg
		\item F : \og le nombre est multiple de 3 et pair \fg
		\end{enumerate}
		%\end{multicols}
	\end{enumerate}
}
{}

\exe{
	On suppose le D20 de l'exercice \ref{ex:1} bien équilibré.
	\begin{enumerate}
		\item Calculer la probabilité de chacun des événements de l'exercice \ref{ex:1}.
		\item Vérifier que $D = A \cup B$.
		\item Vérifier que $E = A \cap B$.
		\item Vérifier l'identité
			\[ P(A \cup B) = P(A) + P(B) - P(A\cap B). \]
	\end{enumerate}
}{}

\exe{
	On lance un D6 pipé avant de noter le numéro de la face du dessus.
	Les probabilités de chacunes des issues vérifient
		\[ P(1) = 2 P(2) = P(3) = 2 P(4) = P(5) = 2P(6). \]
	Calculer $P($\og le résultat est divisible par $3$ \fg$)$.
}{}

\exe{
	Compléter le tableau de probabilités suivant, concernant le numéro de la face du dessus obtenue après un lancer d'un D6 pipé.
	\begin{center}
	\begin{tabular}{|c|c|c|c|c|c|c|} \hline
		Résultat & 1 & 2 & 3 & 4 & 5 & 6 \\ \hline
		Probabilité & $0,1$ & $0,2$ & $0,1$ & $0,15$ & $0,25$ & \\ \hline
	\end{tabular}
	\end{center}
	
	Calculer les probabilités suivantes.
		\begin{enumerate}[label=\roman*)]
			\item P(\og obtenir un nombre pair \fg) 
			\item P(\og obtenir un nombre impair \fg) 
			\item P(\og obtenir un nombre pair \fg)   + P(\og obtenir un nombre impair \fg) 
			\item P(\og obtenir $2$ ou $5$ \fg) 
			\item P(\og obtenir ni $2$ ni $5$ \fg) 
			\item P(\og obtenir $2$ ou $5$ \fg)  + P(\og obtenir ni $2$ ni $5$ \fg) 
		\end{enumerate}
}{}

\exe{
	L'univers associé à une expérience aléatoire est $\{ a, b, c\}$.
	La loi de probabilité $P$ vérifie $P(a) = t^2$, $P(b) = -t$, et $P(c) = \frac14$, pour un réel $t \in \R$.
	
	Développer le carré $\left(t-\frac12\right)^2$ et déterminer $t$.
}{}

\exe{
	Une joueuse de tennis a une probabilité $0,58$ de réussir son premier service et une probabilité de $0,06$ de faire une double faute.
	Quelle est la probabilité qu'elle réussisse seulement son deuxième service ?
}{}

\exe{
	Dessiner des diagrammes de Venn pour motiver les relations suivantes.
		\begin{align*}
			\overline{A\cup B} = \overline{A} \cap \overline{B}, && \text{et} && \overline{A \cap B} = \overline{A} \cup \overline{B}.
		\end{align*}

}{}

\exe{
	On considère un lancer d'un D$300$ bien équilibré, dé à $300$ faces numérotées de $1$ à $300$.
	Après le lancer, on note le numéro de la face supérieure.
	Posons $A$ : \og le résultat est n'est pas multiple de $10$ \fg, et $B$ : \og le résultat n'est pas divisible par $7$ \fg.
	
	\begin{enumerate}
		\item Exprimer avec des mots les événements suivants.
			\begin{multicols}{2}
			\begin{enumerate}[label=\roman*)]
				\item $\overline{A}$
				\item $\overline{B}$
				\item $\overline{A} \cap \overline{B}$
				\item $\overline{A} \cup \overline{B}$
				%\item $\overline{A \cup B}$
				%\item $\overline{A \cap B}$
			\end{enumerate}
			\end{multicols}
		\item Calculer les probabilités $P\left( \overline{A} \right)$ et $P\left( \overline{B} \right)$.
		\item Calculer les probabilités $P\left(\overline{A} \cap \overline{B}\right)$ puis $P\left(\overline{A} \cup \overline{B}\right)$.
		\item Calculer la probabilité que le résultat obtenu est supérieur ou égal à $10$.
	\end{enumerate}
}{}


\exe{
	On lance deux D$6$ équilibrés, dés à $6$ faces l'un après l'autre. Les deux dés sont distinguables car de couleurs différentes.
	\begin{enumerate}
		\item Donner l'univers $\Omega$ et son cardinal $|\Omega|$. Est-ce une situation d'équiprobabilité ?
		\item Quelle est la probabilité d'obtenir un double $6$ ?
		\item Quelle est la probabilité qu'après $10$ tels lancers, on obtienne au moins une fois un double $6$ ?
	\end{enumerate}
}{}

\exe{
	On lance deux D$6$ équilibrés, dés à $6$ faces, au même moment. Les deux dés sont absolument identiques.
	\begin{enumerate}
		\item Donner l'univers $\Omega$ et son cardinal $|\Omega|$. Est-ce une situation d'équiprobabilité ?
		\item Quelle est la probabilité d'obtenir un double $6$ ?
		\item Quelle est la probabilité d'obtenir deux résultats différents ?
	\end{enumerate}
}{}

\exe{
	On lance $3$ fois de suite une pièce de monnaie bien équilibrée.
	On note par $P$ (pile) ou $F$ (face) le résultat de chaque lancer.
	Donner $\Omega$, l'univers de l'expérience, et $|\Omega|$ son cardinal.
	
	Calculer la probabilité des événements suivants.
		\begin{enumerate}
			\item Obtenir $3$ fois face.
			\item Le deuxième lancer donne pile.
			\item Le troisième lancer est différent du premier.
			\item On obtient au moins une fois pile.
		\end{enumerate}
}{}

Introduction aux arbres avec l'exo ci-dessus.

\exe{	
	On tire un boule dans une urne contenant $2$ boules rouges et $4$ boules vertes.
	\begin{enumerate}[label=$\bullet$]
		\item Si la boule tirée est verte, on la met de côté et on retire une nouvelle boule
		\item Si la boule tirée est rouge, on la remet dans l'urne et on retire une nouvelle boule
	\end{enumerate}
	On distingue les quatre événements suivants :
		\begin{multicols}{2}
		\begin{enumerate}[label=]
			\item v : \og la première boule tirée est verte \fg
			\item r : \og la première boule tirée est rouge \fg
			\item V : \og la deuxième boule tirée est verte \fg
			\item R : \og la deuxième boule tirée est rouge \fg
		\end{enumerate}
		\end{multicols}
	\begin{center}
	\begin{tikzpicture}
		% depth 1
		\foreach \i in {-3, 3}
		\draw[-, thick, black] (0,0) node {$\bullet$} -- (\i,-2);
		% depth 2
		\foreach \i in {-3, 3} \foreach \j in {-1, 1}
			\draw[-, thick, black] (\i,-2) node {$\bullet$} -- (\i+\j,-4) node {$\bullet$};
			
		\draw (-3,-2) node[above left] {$v$};
		\draw (3,-2) node[above right] {$r$};
			
		\draw (-4,-4) node[below] {$V$};
		\draw (2,-4) node[below] {$V$};
		\draw (-2,-4) node[below] {$R$};
		\draw (4,-4) node[below] {$R$};
	\end{tikzpicture}
	\end{center}
	Compléter l'arbre, calculer $P(V)$ et $P(R)$.
}{}

\exe{
	Compléter l'arbre correspondant à une expérience aléatoire à deux épreuves d'univers $\{A ; B ; C ; D\}$ et répondre aux questions suivantes.
	\begin{center}
	\begin{tikzpicture}
		% depth 1
		\draw[-, thick, black] (0,0) node {$\bullet$} -- (3,-2) node[midway, above right] {};
		\draw[-, thick, black] (0,0) node {$\bullet$} -- (-3,-2) node[midway, above left] {$0,7$};
		% depth 2
		\draw[-, thick, black] (-3,-2) node {$\bullet$} -- (-1,-4) node[midway, above right] {$\frac49$};
		\draw[-, thick, black] (-3,-2) node {$\bullet$} -- (-3,-4);
		\draw[-, thick, black] (-3,-2) node {$\bullet$} -- (-5,-4) node[midway, above left] {$\frac13$};
		
		\draw[-, thick, black] (3,-2) node {$\bullet$} -- (1,-4) node[midway, above left] {$\frac16$};
		\draw[-, thick, black] (3,-2) node {$\bullet$} -- (3,-4) node[pos=.6, left] {$\frac12$};
		\draw[-, thick, black] (3,-2) node {$\bullet$} -- (5,-4);
		
		\draw (1,-4) node {$\bullet$};
		\draw (3,-4) node {$\bullet$};
		\draw (5,-4) node {$\bullet$};
		\draw (1,-4) node[below] {$D$};
		\draw (3,-4) node[below] {$B$};
		\draw (5,-4) node[below] {$C$};
		
		\draw (-1,-4) node {$\bullet$};
		\draw (-3,-4) node {$\bullet$};
		\draw (-5,-4) node {$\bullet$};
		\draw (-1,-4) node[below] {$C$};
		\draw (-3,-4) node[below] {$B$};
		\draw (-5,-4) node[below] {$A$};
	\end{tikzpicture}
	\end{center}
	
	\begin{multicols}{2}
	\begin{enumerate}
		\item Calculer $P(D)$.
		\item Calculer $P(B)$.
		\item Calculer $P(D \cup B)$.
		\item Calculer $P(A\cup C)$.
	\end{enumerate}
	\end{multicols}

}{}

\exe{
	On tire un boule dans une urne contenant $3$ boules bleues et $4$ boules vertes.
	\begin{enumerate}[label=$\bullet$]
		\item Si la boule tirée est verte, on jette un dé équilibré à $3$ faces
		\item Si la boule tirée est rouge, on jette un dé équilibré à $6$ faces
	\end{enumerate}

	Créer un arbre de probabilité pour cette situation et répondre aux questions suivantes.
	\begin{enumerate}
		\item Quelle est la probabilité d'obtenir un nombre pair ?
		\item Quelle est la probabilité d'obtenir un multiple de $3$ ?
		\item Quelle est la probabilité d'obtenir un nombre pair \textbf{ou} un multiple de $3$ ?
		\item Calculer la probabilité d'obtenir $6$ et vérifier la formule d'inclusion-exclusion.
	\end{enumerate}

}{}

%% TODO : implémenter en python et générer les valeurs
%% N = 10 000 c'est un peu facile en fait
\exe{	
	Dans une urne opaque se trouvent un nombre inconnu de boules rouges, bleues, et vertes.
	On tire aléatoirement une boule de l'urne, on note sa couleur, et on la remet dans l'urne.
	Les résultats sont décrits dans le tableau suivant.
	\begin{center}
	\begin{tabular}{|c|c|c|c|} \hline
		Couleur & Rouge & Bleu & Vert \\ \hline
		Nombre de tirages & 3332 & 5005 & 1663 \\ \hline
		Fréquence & & & \\ \hline
	\end{tabular}
	\end{center}
	
	Compléter le tableau, modéliser la réalité en définissant un univers et une loi de probabilité, et l'utiliser pour répondre aux questions suivantes.
	\begin{enumerate}
		\item Quelle est la probabilité d'obtenir deux boules rouges en deux tirages ?
		\item Quelle est la probabilité d'obtenir deux boules vertes en trois tirages ?
		\item Quelle est la probabilité d'obtenir au moins une boule bleue en quatre tirages ?
	\end{enumerate}
}{}


\newpage

%\subsection*{Exercices supplémentaires}

\exe{[$\star$]
	Soit $\Omega = \{ 0 ; 1; \dots ; 499; 500\} \subseteq \N$ et $A, B \subseteq \Omega$ définis par
		\begin{align*}
			A = \{ n \in \Omega \text{ tq. } 2|n \}, && B = \{ n \in \Omega \text{ tq. } 5|n\}.
		\end{align*}
	\begin{enumerate}
		\item
		Montrer qu'un nombre est multiple de $2$ \textbf{et} de $5$ si et seulement s'il est multiple de $10$.
		\item
		Donner $|A|$, le nombre d'entiers naturels inférieurs ou égaux à $500$ qui sont multiples de $2$.
		\item
		Donner $|B|$, le nombre d'entiers naturels inférieurs ou égaux à $500$ qui sont multiples de $5$.
		\item
		Donner $|A \cap B|$, le nombre d'entiers naturels inférieurs ou égaux à $500$ qui sont multiples de $10$.
		\item
		En déduire $|A \cup B|$, le nombre d'entiers naturels inférieurs ou égaux à $500$ qui sont multiples de $2$ \textbf{ou} de $5$.
	\end{enumerate}
}{}

\exe{[$\star$]
	Soient $A, B$ deux événements.
	Montrer que les événements $ \overline{A} \cap B$ et $A \cap B$ sont disjoints.
	En déduire la relation suivante.
		\[ P\left( \overline{A} \cap B \right) = P(B) - P(A \cap B) \]
}{}

\exe{[$\star$]
	Soit $\Omega = \{ 1 ; 2 ; 3 ; \dots ; n \}$ l'ensemble des $n$ premiers entiers naturels non nuls.
	On construit un sous-ensemble $E \subseteq \Omega$ de la façon suivante :
	pour chaque entier de $\Omega$, on jette une pièce équilibrée pour savoir si on l'inclut ou non dans $E$.
	
	\begin{enumerate}
		\item
		Combien de sous-ensembles $E$ est-il possible d'obtenir ?
		\item
		Quelle est la probabilité que l'ensemble $E$ contienne $4$ (si $n\geq4$) ?
		\item
		Quelle est la probabilité d'obtenir $E = \{ 1; 3 ; 5\}$ (si $n\geq 5$) ?
		\item
		Quelle est la probabilité que l'ensemble contienne au moins $2$ éléments (si $n\geq2$) ?
	\end{enumerate}


}{}

\exe{[$\star$]
	On mélange un jeu classique de $52$ cartes puis on retourne les cartes une à une en les prenant successivement du haut du paquet.
	Donner $|\Omega|$, le nombre d'ordres différents des $52$ cartes.
}{}

\exe{[$\star$]
	On mélange un jeu classique de $52$ cartes puis on retourne les $5$ premières cartes en gardant leur ordre en mémoire.
	Donner $|\Omega|$, le nombre quintuples ordonnés de cartes. 
	On utilisera la notation $(1 ; 2; 3; 4; 5)$ pour une issue possible car les tuples sont ordonnés.
}{}

\exe{[$\star$]
	Deux personnes veulent faire un pari équilibré mais aucune n'a confiance en l'autre : elles ne peuvent donc pas utiliser un lancer simple de pièce de monnaie, de peur que son propriétaire l'ait truqué.
	Posons $p$ et $1-p$ la probabilité d'obtenir pile et face respectivement après un lancer d'une pièce de monnaie.
	\begin{enumerate}
		\item Montrer que la pièce est équilibrée si et seulement si $p=\frac12$. On ne supposera \emph{pas} que la pièce est équilibrée dans la suite.
		\item Dessiner un arbre de probabilité correspondant à deux lancers successifs.
		\item Déduire que les probabilités d'obtenir pile puis face et d'obtenir face puis pile sont égales.
		\item Donner un protocole garantissant un pari équiprobable en utilisant une pièce de monnaie possiblement truquée.
	\end{enumerate}
}{}

\exe{[$\star\star$]
	On mélange un jeu classique de $52$ cartes puis on retourne les $5$ premières cartes sans prendre en compte leur ordre.
	Donner le nombre de quintuples ordonnés donnant lieu à la même issue de l'expérience. Par exemple, échanger l'ordre des deux premières cartes ne change pas l'ensemble qu'ils forment.
	On utilisera la notation $\{1 ; 2; 3; 4; 5\}$ pour une issue possible car les ensembles ne sont pas ordonnés.
	
	En déduire $|\Omega|$, le nombre d'ensembles de $5$ cartes.
}{}

\exe{[$\star\star$]
	On mélange un jeu classique de $52$ cartes puis on retourne les cartes une à une en les prenant successivement du haut du paquet.
	Quelle est la probabilité que l'as de trèfle soit le premier as retourné ?

	On ajoute une carte joker au paquet avant de le remélanger et de retourner encore une fois les cartes une à une.
	Quelle est la probabilité que le joker apparaisse après exactement un as, et avant les 4 autres ?
}{}

\exe{[$\star$]
	On choisit deux nombres $x$ et $y$ aléatoirement et uniformément entre $-1$ et $1$, et on considère la distance $d$ du point $(x;y)$ à l'origine.
	On souhaite calculer la probabilité $p$ de l'événement \og la distance $d$ est inférieure à $1$ \fg.
	
	Faire un dessin de la situation et montrer que $p = \dfrac{\pi}4$.
	
	Créer un protocole pour approximer la valeur de $\pi$ en admettant qu'on puisse choisir un nombre uniformément entre $0$ et $1$.
}{}

\exe{[$\star\star$ Problème de Monty Hall]
	 Supposez que vous êtes sur le plateau d'un jeu télévisé, face à trois portes et que vous devez choisir d'en ouvrir une seule, en sachant que derrière l'une d'elles se trouve une voiture et derrière les deux autres des chèvres. Vous choisissez une porte, disons la numéro 1, et le présentateur, qui sait, lui, ce qu'il y a derrière chaque porte, ouvre une autre porte, disons la numéro 3, qui découvre une chèvre. Il vous demande alors : \og désirez-vous ouvrir la porte numéro 2 ? \fg. Avez-vous intérêt à changer votre choix ?\footnote{Traduction de $\href{https://en.wikipedia.org/wiki/Monty\_Hall\_problem}{https://en.wikipedia.org/wiki/Monty\_Hall\_problem}$.}
}{}

\exe{[$\star\star$ Paradoxe des anniversaires]
	On choisit $5$ personnes au hasard dans la population en supposant que chaque date d'anniversaire est équiprobable de probabilité $\frac1{365}$.
	
	Montrer, en passant par l'événement complémentaire, que la probabilité que deux personnes partagent la même date d'anniversaire est d'environ $2,71\%$.
	
	Montrer qu'en choisissant $20$ personnes, cette probabilité atteint environ $41,14\%$.
}{}


\end{document}

				% ENABLE or DISABLE font change
				% use XeLaTeX if true
\newif\ifdys
				\dystrue
				\dysfalse

\newif\ifsolutions
				\solutionstrue
				\solutionsfalse

% DYSLEXIA SWITCH
\newif\ifdys
		
				% ENABLE or DISABLE font change
				% use XeLaTeX if true
				\dystrue
				\dysfalse


\ifdys

\documentclass[a4paper, 14pt]{extarticle}
\usepackage{amsmath,amsfonts,amsthm,amssymb,mathtools}

\tracinglostchars=3 % Report an error if a font does not have a symbol.
\usepackage{fontspec}
\usepackage{unicode-math}
\defaultfontfeatures{ Ligatures=TeX,
                      Scale=MatchUppercase }

\setmainfont{OpenDyslexic}[Scale=1.0]
\setmathfont{Fira Math} % Or maybe try KPMath-Sans?
\setmathfont{OpenDyslexic Italic}[range=it/{Latin,latin}]
\setmathfont{OpenDyslexic}[range=up/{Latin,latin,num}]

\else

\documentclass[a4paper, 12pt]{extarticle}

\usepackage[utf8x]{inputenc}
%fonts
\usepackage{amsmath,amsfonts,amsthm,amssymb,mathtools}
% comment below to default to computer modern
\usepackage{libertinus,libertinust1math}

\fi


\usepackage[french]{babel}
\usepackage[
a4paper,
margin=2cm,
nomarginpar,% We don't want any margin paragraphs
]{geometry}
\usepackage{icomma}

\usepackage{fancyhdr}
\usepackage{array}
\usepackage{hyperref}

\usepackage{multicol, enumerate}
\newcolumntype{P}[1]{>{\centering\arraybackslash}p{#1}}


\usepackage{stackengine}
\newcommand\xrowht[2][0]{\addstackgap[.5\dimexpr#2\relax]{\vphantom{#1}}}

% theorems

\theoremstyle{plain}
\newtheorem{theorem}{Th\'eor\`eme}
\newtheorem*{sol}{Solution}
\theoremstyle{definition}
\newtheorem{ex}{Exercice}
\newtheorem*{rpl}{Rappel}
\newtheorem{enigme}{Énigme}

% corps
\usepackage{calrsfs}
\newcommand{\C}{\mathcal{C}}
\newcommand{\R}{\mathbb{R}}
\newcommand{\Rnn}{\mathbb{R}^{2n}}
\newcommand{\Z}{\mathbb{Z}}
\newcommand{\N}{\mathbb{N}}
\newcommand{\Q}{\mathbb{Q}}

% variance
\newcommand{\Var}[1]{\text{Var}(#1)}

% domain
\newcommand{\D}{\mathcal{D}}


% date
\usepackage{advdate}
\AdvanceDate[0]


% plots
\usepackage{pgfplots}

% table line break
\usepackage{makecell}
%tablestuff
\def\arraystretch{2}
\setlength\tabcolsep{15pt}

%subfigures
\usepackage{subcaption}

\definecolor{myg}{RGB}{56, 140, 70}
\definecolor{myb}{RGB}{45, 111, 177}
\definecolor{myr}{RGB}{199, 68, 64}

% fake sections with no title to move around the merged pdf
\newcommand{\fakesection}[1]{%
  \par\refstepcounter{section}% Increase section counter
  \sectionmark{#1}% Add section mark (header)
  \addcontentsline{toc}{section}{\protect\numberline{\thesection}#1}% Add section to ToC
  % Add more content here, if needed.
}


% SOLUTION SWITCH
\newif\ifsolutions
				\solutionstrue
				%\solutionsfalse

\ifsolutions
	\newcommand{\exe}[2]{
		\begin{ex} #1  \end{ex}
		\begin{sol} #2 \end{sol}
	}
\else
	\newcommand{\exe}[2]{
		\begin{ex} #1  \end{ex}
	}
	
\fi


% tableaux var, signe
\usepackage{tkz-tab}


%pinfty minfty
\newcommand{\pinfty}{{+}\infty}
\newcommand{\minfty}{{-}\infty}

\begin{document}


\AdvanceDate[0]

\begin{document}
\pagestyle{fancy}
\fancyhead[L]{Seconde 13}
\fancyhead[C]{\textbf{Probabilités $\star$ \ifsolutions -- Solutions  \fi}}
\fancyhead[R]{\today}

\exe{[$\star$]
	Soit $\Omega = \{ 0 ; 1; \dots ; 499; 500\} \subseteq \N$ et $A, B \subseteq \Omega$ définis par
		\begin{align*}
			A = \{ n \in \Omega \text{ tq. } 2|n \}, && B = \{ n \in \Omega \text{ tq. } 5|n\}.
		\end{align*}
	\begin{enumerate}
		\item
		Montrer qu'un nombre est multiple de $2$ \textbf{et} de $5$ si et seulement s'il est multiple de $10$.
		\item
		Donner $|A|$, le nombre d'entiers naturels inférieurs ou égaux à $500$ qui sont multiples de $2$.
		\item
		Donner $|B|$, le nombre d'entiers naturels inférieurs ou égaux à $500$ qui sont multiples de $5$.
		\item
		Donner $|A \cap B|$, le nombre d'entiers naturels inférieurs ou égaux à $500$ qui sont multiples de $10$.
		\item
		En déduire $|A \cup B|$, le nombre d'entiers naturels inférieurs ou égaux à $500$ qui sont multiples de $2$ \textbf{ou} de $5$.
	\end{enumerate}
}{}

\exe{[$\star$]
	Soient $A, B$ deux événements.
	Montrer que les événements $ \overline{A} \cap B$ et $A \cap B$ sont disjoints.
	En déduire la relation suivante.
		\[ P\left( \overline{A} \cap B \right) = P(B) - P(A \cap B) \]
}{}

\exe{[$\star$]
	Soit $\Omega = \{ 1 ; 2 ; 3 ; \dots ; n \}$ l'ensemble des $n$ premiers entiers naturels non nuls.
	On construit un sous-ensemble $E \subseteq \Omega$ de la façon suivante :
	pour chaque entier de $\Omega$, on jette une pièce équilibrée pour savoir si on l'inclut ou non dans $E$.
	
	\begin{enumerate}
		\item
		Combien de sous-ensembles $E$ est-il possible d'obtenir ?
		\item
		Quelle est la probabilité que l'ensemble $E$ contienne $4$ (si $n\geq4$) ?
		\item
		Quelle est la probabilité d'obtenir $E = \{ 1; 3 ; 5\}$ (si $n\geq 5$) ?
		\item
		Quelle est la probabilité que l'ensemble contienne au moins $2$ éléments (si $n\geq2$) ?
	\end{enumerate}


}{}

\exe{[$\star$]
	On mélange un jeu classique de $52$ cartes puis on retourne les $5$ premières cartes en gardant leur ordre en mémoire.
	Donner $|\Omega|$, le nombre quintuples ordonnés de cartes. 
	On utilisera la notation $(1 ; 2; 3; 4; 5)$ pour une issue possible car les tuples sont ordonnés.
}{}

\exe{[$\star$]
	On mélange un jeu classique de $52$ cartes puis on retourne les cartes une à une en les prenant successivement du haut du paquet.
	Donner $|\Omega|$, le nombre d'ordres différents des $52$ cartes.
}{}

\exe{[$\star$]
	On mélange un jeu classique de $52$ cartes puis on retourne les $5$ premières cartes sans prendre en compte leur ordre.
	
	Donner le nombre de quintuples ordonnés donnant lieu à la même issue de l'expérience. Par exemple, échanger l'ordre des deux premières cartes ne change pas l'ensemble qu'ils forment.
	On utilisera la notation $\{1 ; 2; 3; 4; 5\}$ pour une issue possible car l'ordre n'importe pas.
	
	En déduire $|\Omega|$, le nombre d'ensembles de $5$ cartes.
}{}

\exe{[$\star$]
	Deux personnes veulent faire un pile ou face mais aucune n'a confiance en l'autre : elles ne peuvent donc pas utiliser un lancer simple de pièce de monnaie, de peur que son propriétaire l'ait truquée.
	Posons $p$ et $1-p$ la probabilité d'obtenir pile et face respectivement après un lancer d'une pièce de monnaie possiblement truquée.
	\begin{enumerate}
		\item Montrer que la pièce est équilibrée si et seulement si $p=\frac12$. On ne suppose pas que la pièce est équilibrée dans la suite.
		\item Dessiner un arbre de probabilité correspondant à deux lancers successifs.
		\item Déduire que les probabilités d'obtenir pile puis face et d'obtenir face puis pile sont égales.
		\item Donner un protocole garantissant un pari équitable en utilisant une pièce de monnaie possiblement truquée.
	\end{enumerate}
}{}

\exe{[$\star$]
	On mélange un jeu classique de $52$ cartes puis on retourne les cartes une à une en les prenant successivement du haut du paquet.
	Quelle est la probabilité que l'as de trèfle soit le premier as retourné ?

	On ajoute une carte joker au paquet avant de le remélanger et de retourner encore une fois les cartes une à une.
	Quelle est la probabilité que le joker apparaisse après exactement un as, et avant les 4 autres ?
}{}

\exe{[$\star$]
	On choisit deux nombres $x$ et $y$ aléatoirement et uniformément entre $-1$ et $1$, et on considère la distance $d$ du point $(x;y)$ à l'origine.
	On souhaite calculer la probabilité $p$ de l'événement \og la distance $d$ est inférieure à $1$ \fg.
	
	Faire un dessin de la situation et montrer que $p = \dfrac{\pi}4$.
	
	Créer un protocole pour approximer la valeur de $\pi$ en admettant qu'on puisse choisir un nombre uniformément entre $0$ et $1$.
}{}

\exe{[$\star\star$ Problème de Monty Hall]
	 Supposez que vous êtes sur le plateau d'un jeu télévisé, face à trois portes et que vous devez choisir d'en ouvrir une seule, en sachant que derrière l'une d'elles se trouve une voiture et derrière les deux autres des chèvres. Vous choisissez une porte, disons la numéro 1, et le présentateur, qui sait, lui, ce qu'il y a derrière chaque porte, ouvre une autre porte, disons la numéro 3, qui découvre une chèvre. Il vous demande alors : \og désirez-vous ouvrir la porte numéro 2 ? \fg. Avez-vous intérêt à changer votre choix ?\footnote{Traduction de $\href{https://en.wikipedia.org/wiki/Monty\_Hall\_problem}{https://en.wikipedia.org/wiki/Monty\_Hall\_problem}$.}
}{}

\exe{[$\star\star$ Paradoxe des anniversaires]
	On choisit $5$ personnes au hasard dans la population en supposant que chaque date d'anniversaire est équiprobable de probabilité $\frac1{365}$.
	
	Montrer, en passant par l'événement complémentaire, que la probabilité que deux personnes partagent la même date d'anniversaire est d'environ $2,71\%$.
	
	Montrer qu'en choisissant $20$ personnes, cette probabilité atteint environ $41,14\%$.
}{}


\end{document}

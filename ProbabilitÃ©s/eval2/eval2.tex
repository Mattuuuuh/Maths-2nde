				% ENABLE or DISABLE font change
				% use XeLaTeX if true
\newif\ifdys
				\dystrue
				\dysfalse

\newif\ifsolutions
				\solutionstrue
				\solutionsfalse

% DYSLEXIA SWITCH
\newif\ifdys
		
				% ENABLE or DISABLE font change
				% use XeLaTeX if true
				\dystrue
				\dysfalse


\ifdys

\documentclass[a4paper, 14pt]{extarticle}
\usepackage{amsmath,amsfonts,amsthm,amssymb,mathtools}

\tracinglostchars=3 % Report an error if a font does not have a symbol.
\usepackage{fontspec}
\usepackage{unicode-math}
\defaultfontfeatures{ Ligatures=TeX,
                      Scale=MatchUppercase }

\setmainfont{OpenDyslexic}[Scale=1.0]
\setmathfont{Fira Math} % Or maybe try KPMath-Sans?
\setmathfont{OpenDyslexic Italic}[range=it/{Latin,latin}]
\setmathfont{OpenDyslexic}[range=up/{Latin,latin,num}]

\else

\documentclass[a4paper, 12pt]{extarticle}

\usepackage[utf8x]{inputenc}
%fonts
\usepackage{amsmath,amsfonts,amsthm,amssymb,mathtools}
% comment below to default to computer modern
\usepackage{libertinus,libertinust1math}

\fi


\usepackage[french]{babel}
\usepackage[
a4paper,
margin=2cm,
nomarginpar,% We don't want any margin paragraphs
]{geometry}
\usepackage{icomma}

\usepackage{fancyhdr}
\usepackage{array}
\usepackage{hyperref}

\usepackage{multicol, enumerate}
\newcolumntype{P}[1]{>{\centering\arraybackslash}p{#1}}


\usepackage{stackengine}
\newcommand\xrowht[2][0]{\addstackgap[.5\dimexpr#2\relax]{\vphantom{#1}}}

% theorems

\theoremstyle{plain}
\newtheorem{theorem}{Th\'eor\`eme}
\newtheorem*{sol}{Solution}
\theoremstyle{definition}
\newtheorem{ex}{Exercice}
\newtheorem*{rpl}{Rappel}
\newtheorem{enigme}{Énigme}

% corps
\usepackage{calrsfs}
\newcommand{\C}{\mathcal{C}}
\newcommand{\R}{\mathbb{R}}
\newcommand{\Rnn}{\mathbb{R}^{2n}}
\newcommand{\Z}{\mathbb{Z}}
\newcommand{\N}{\mathbb{N}}
\newcommand{\Q}{\mathbb{Q}}

% variance
\newcommand{\Var}[1]{\text{Var}(#1)}

% domain
\newcommand{\D}{\mathcal{D}}


% date
\usepackage{advdate}
\AdvanceDate[0]


% plots
\usepackage{pgfplots}

% table line break
\usepackage{makecell}
%tablestuff
\def\arraystretch{2}
\setlength\tabcolsep{15pt}

%subfigures
\usepackage{subcaption}

\definecolor{myg}{RGB}{56, 140, 70}
\definecolor{myb}{RGB}{45, 111, 177}
\definecolor{myr}{RGB}{199, 68, 64}

% fake sections with no title to move around the merged pdf
\newcommand{\fakesection}[1]{%
  \par\refstepcounter{section}% Increase section counter
  \sectionmark{#1}% Add section mark (header)
  \addcontentsline{toc}{section}{\protect\numberline{\thesection}#1}% Add section to ToC
  % Add more content here, if needed.
}


% SOLUTION SWITCH
\newif\ifsolutions
				\solutionstrue
				%\solutionsfalse

\ifsolutions
	\newcommand{\exe}[2]{
		\begin{ex} #1  \end{ex}
		\begin{sol} #2 \end{sol}
	}
\else
	\newcommand{\exe}[2]{
		\begin{ex} #1  \end{ex}
	}
	
\fi


% tableaux var, signe
\usepackage{tkz-tab}


%pinfty minfty
\newcommand{\pinfty}{{+}\infty}
\newcommand{\minfty}{{-}\infty}

\begin{document}


\AdvanceDate[2]

\begin{document}
\pagestyle{fancy}
\fancyhead[L]{Seconde 13}
\fancyhead[C]{\textbf{Évaluation blanche : Probabilités \ifsolutions -- Solutions  \fi}}
\fancyhead[R]{\today}


%\exe{
%	Démontrer, à l'aide d'un ou plusieurs diagrammes de Venn, la formule d'inclusion-exclusion
%		\[ P(A\cup B) = P(A) + P(B) - P(A\cap B). \]
%}{}

\exe{
	On considère un univers
		\[ \Omega = \{ -3 ; -2 ; 1 ; 4; 5; 7 ; 13 \}, \]
	ainsi que deux sous-ensembles 
		\begin{align*}
			A = \{ -3 ; 1 ; 5 \} && \text{ et } && B = \{ 1 ; 4 ; 5 ; 7 \}.
		\end{align*}
	\begin{enumerate}
		\item Donner $\overline{A}$, le complémentaire de $A$ dans l'univers $\Omega$.
		\item Vérifier l'égalité
			\[ |A\cup B| = |A| + |B| - |A\cap B| \]
		en calculant le membre de gauche puis le membre de droite et en comparant les valeurs obtenues.
	\end{enumerate}

}{}


\exe{
	Démontrer, à l'aide d'un ou plusieurs diagrammes de Venn, la relation d'ensembles
		\[ A = \left(A \cap B \right) \cup \left(A \cap \overline{B} \right). \]
}{}


\exe{
	On jette simultanément deux D$6$, dés à $6$ faces numérotées de $1$ à $6$, et on calcule la somme des faces du dessus.
	Les dés sont supposés bien équilibrés.
	
	\begin{enumerate}
		\item Donner l'univers $\Omega$ de l'expérience.
		\item Calculer la probabilité d'obtenir exactement $4$.
		\item Est-ce une situation d'équiprobabilité ?
		\item[4($\star$).] Montrer que $P(k) = P(14-k)$ pour tout $k \in \Omega$.
	\end{enumerate}
}{}

\exe{
	On lance une pièce de monnaie bien équilibrée.
	\begin{enumerate}[label=$\bullet$]
		\item Si elle tombe sur pile, on extrait au hasard une boule dans l'urne 1 qui contient 1 boule noire, 3 boules blanches et 4 boules rouges
		\item Sinon, on extrait une boule dans l'urne 2 qui contient 4 boules noires et 3 boules blanches.
	\end{enumerate}

	Créer un arbre de probabilité pour cette situation et répondre aux questions suivantes.
	\begin{enumerate}
		\item Quelle est la probabilité d'obtenir une boule rouge ?
		\item Quelle est la probabilité d'obtenir une boule noire ou une boule blanche ?
	\end{enumerate}

}
{}


\exe{	
	Dans une urne opaque se trouvent un nombre inconnu de boules rouges, bleues, et vertes.
	On tire aléatoirement une boule de l'urne, on note sa couleur, et on la remet dans l'urne.
	Les résultats sont décrits dans le tableau suivant.
	\begin{center}
	\begin{tabular}{|c|c|c|c|} \hline
		Couleur & Rouge & Bleu & Vert \\ \hline
		Nombre de tirages & 64~444 & 42~962 & 35~802 \\ \hline
		Fréquence & & & \\ \hline
	\end{tabular}
	\end{center}
	
	Compléter le tableau en arrondissant à $10^{-2}$ près.
	Modéliser la réalité en définissant un univers et une loi de probabilité, et l'utiliser pour répondre aux questions suivantes.
	\begin{enumerate}
		\item Quelle est la probabilité d'obtenir exactement deux boules bleues en deux tirages ?
		\item Quelle est la probabilité d'obtenir exactement deux boules rouges en trois tirages ?
		\item Quelle est la probabilité d'obtenir au moins une boule bleue en $10$ tirages ?
	\end{enumerate}
}{}


\newpage

\subsection*{Bonus}


\exe{
	Une expérience aléatoire à deux épreuves d'univers $\{A ; B ; C ; D\}$ admet un arbre de probabilité binaire comme ci-dessous, où $t\in\R$ est un paramètre réel encore inconnu.

	\begin{center}
	\begin{tikzpicture}
		% depth 1
		\foreach \i in {-3, 3}
		\draw[-, thick, black] (0,0) node {$\bullet$} -- (\i,-2);
		% depth 2
		\foreach \i in {-3, 3} \foreach \j in {-1, 1}
			\draw[-, thick, black] (\i,-2) node {$\bullet$} -- (\i+\j,-4) node {$\bullet$};
		
		% probas
		\draw (-1.5,-1) node[above left] {$-t$};
		\draw (1.5,-1) node[above right] {$1+t$};
		
		\draw (-3.5,-3) node[above left] {$-t$};
		\draw (-2.5,-3) node[above right] {$1+t$};
		
		\draw (2.5,-3) node[above left] {$-t$};
		\draw (3.5,-3) node[above right] {$1+t$};
		
		% issues
		\draw (-4,-4) node[below] {A};
		\draw (2,-4) node[below] {B};
		\draw (-2,-4) node[below] {C};
		\draw (4,-4) node[below] {D};
	\end{tikzpicture}
	\end{center}
	
	\begin{enumerate}
		\item Montrer qu'on a forcément 
			\[ -1 \leq t \leq 0. \]
		\item Déterminer le paramètre $t$ tel que
			\[ P(D) = \frac19. \]
	\end{enumerate}
}{}

\end{document}

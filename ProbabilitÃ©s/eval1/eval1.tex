				% ENABLE or DISABLE font change
				% use XeLaTeX if true
\newif\ifdys
				\dystrue
				\dysfalse

\newif\ifsolutions
				\solutionstrue
				\solutionsfalse

% DYSLEXIA SWITCH
\newif\ifdys
		
				% ENABLE or DISABLE font change
				% use XeLaTeX if true
				\dystrue
				\dysfalse


\ifdys

\documentclass[a4paper, 14pt]{extarticle}
\usepackage{amsmath,amsfonts,amsthm,amssymb,mathtools}

\tracinglostchars=3 % Report an error if a font does not have a symbol.
\usepackage{fontspec}
\usepackage{unicode-math}
\defaultfontfeatures{ Ligatures=TeX,
                      Scale=MatchUppercase }

\setmainfont{OpenDyslexic}[Scale=1.0]
\setmathfont{Fira Math} % Or maybe try KPMath-Sans?
\setmathfont{OpenDyslexic Italic}[range=it/{Latin,latin}]
\setmathfont{OpenDyslexic}[range=up/{Latin,latin,num}]

\else

\documentclass[a4paper, 12pt]{extarticle}

\usepackage[utf8x]{inputenc}
%fonts
\usepackage{amsmath,amsfonts,amsthm,amssymb,mathtools}
% comment below to default to computer modern
\usepackage{libertinus,libertinust1math}

\fi


\usepackage[french]{babel}
\usepackage[
a4paper,
margin=2cm,
nomarginpar,% We don't want any margin paragraphs
]{geometry}
\usepackage{icomma}

\usepackage{fancyhdr}
\usepackage{array}
\usepackage{hyperref}

\usepackage{multicol, enumerate}
\newcolumntype{P}[1]{>{\centering\arraybackslash}p{#1}}


\usepackage{stackengine}
\newcommand\xrowht[2][0]{\addstackgap[.5\dimexpr#2\relax]{\vphantom{#1}}}

% theorems

\theoremstyle{plain}
\newtheorem{theorem}{Th\'eor\`eme}
\newtheorem*{sol}{Solution}
\theoremstyle{definition}
\newtheorem{ex}{Exercice}
\newtheorem*{rpl}{Rappel}
\newtheorem{enigme}{Énigme}

% corps
\usepackage{calrsfs}
\newcommand{\C}{\mathcal{C}}
\newcommand{\R}{\mathbb{R}}
\newcommand{\Rnn}{\mathbb{R}^{2n}}
\newcommand{\Z}{\mathbb{Z}}
\newcommand{\N}{\mathbb{N}}
\newcommand{\Q}{\mathbb{Q}}

% variance
\newcommand{\Var}[1]{\text{Var}(#1)}

% domain
\newcommand{\D}{\mathcal{D}}


% date
\usepackage{advdate}
\AdvanceDate[0]


% plots
\usepackage{pgfplots}

% table line break
\usepackage{makecell}
%tablestuff
\def\arraystretch{2}
\setlength\tabcolsep{15pt}

%subfigures
\usepackage{subcaption}

\definecolor{myg}{RGB}{56, 140, 70}
\definecolor{myb}{RGB}{45, 111, 177}
\definecolor{myr}{RGB}{199, 68, 64}

% fake sections with no title to move around the merged pdf
\newcommand{\fakesection}[1]{%
  \par\refstepcounter{section}% Increase section counter
  \sectionmark{#1}% Add section mark (header)
  \addcontentsline{toc}{section}{\protect\numberline{\thesection}#1}% Add section to ToC
  % Add more content here, if needed.
}


% SOLUTION SWITCH
\newif\ifsolutions
				\solutionstrue
				%\solutionsfalse

\ifsolutions
	\newcommand{\exe}[2]{
		\begin{ex} #1  \end{ex}
		\begin{sol} #2 \end{sol}
	}
\else
	\newcommand{\exe}[2]{
		\begin{ex} #1  \end{ex}
	}
	
\fi


% tableaux var, signe
\usepackage{tkz-tab}


%pinfty minfty
\newcommand{\pinfty}{{+}\infty}
\newcommand{\minfty}{{-}\infty}

\begin{document}


\AdvanceDate[2]

\begin{document}
\pagestyle{fancy}
\fancyhead[L]{Seconde 13}
\fancyhead[C]{\textbf{Probabilités 3 -- Exercices à rendre \ifsolutions -- Solutions  \fi}}
\fancyhead[R]{\today}

\exe{
	Compléter le tableau de probabilités suivant, concernant le numéro de la face du dessus obtenue après un lancer d'un D6 pipé.
	\begin{center}
	\begin{tabular}{|c|c|c|c|c|c|c|} \hline
		Résultat & 1 & 2 & 3 & 4 & 5 & 6 \\ \hline
		Probabilité & $0,1$ & $0,2$ & $0,1$ & $0,15$ & $0,25$ & \ifsolutions $\color{red} 0,2$ \fi \\ \hline
	\end{tabular}
	\end{center}
	
	Calculer les probabilités suivantes.
		\begin{multicols}{2}
		\begin{enumerate}[label=\roman*)]
			\item P(\og obtenir un nombre pair \fg) 
			\item P(\og obtenir un nombre impair \fg) 
			\item P(\og ne pas obtenir $5$ \fg)
		\end{enumerate}
		\end{multicols}
}{
	On complète d'abord le tableau en sachant que l'univers de l'expérience est $\Omega = \{ 1 ; 2 ;3 ; 4 ; 5 ; 6 \}$ et que la somme des probabilités du tableau est donc nécessairement $1$.
	En effet, $P(\Omega) = 1$, car $\Omega$ correspond à l'événement \og obtenir une des issues possibles \fg, qui est un événement sûr.

	\begin{enumerate}[label=\roman*)]
		\item P(\og obtenir un nombre pair \fg) $ = P(2) + P(4) + P(6) = 0,2 + 0,15 + 0,2 = 0,55$
		\item P(\og obtenir un nombre impair \fg) $ = P(1) + P(3) + P(5) = 0,1 + 0,1 + 0,25 = 0,45$.
		\item P(\og obtenir un nombre pair \fg)   + P(\og obtenir un nombre impair \fg) $ = 0,55 + 0,45 = 1$
		\item P(\og obtenir $2$ ou $5$ \fg) $ = P(2) + P(5) = 0,2 + 0,25 = 0,45$
		\item P(\og obtenir ni $2$ ni $5$ \fg) $ = P(1) + P(3) + P(4) + P(6) = 0,1 + 0,1 + 0,15 + 0,2 = 0,55$
		\item P(\og obtenir $2$ ou $5$ \fg)  + P(\og obtenir ni $2$ ni $5$ \fg) $ = 0,45 + 0,55 = 1$
	\end{enumerate}
	
	Le fait que la somme soit $1$ n'est pas suprenant : les événements comptent séparément toutes les issues possibles.
	On appelle ces événements \emph{complémentaires} : ils sont disjoints (les deux événements ne peuvent pas arriver en même temps), et ensembles ils forment l'univers tout entier (chaque issue appartient à un événement ou à l'autre).
	
	Les derniers événements motivent la relation 
		\[ \overline{A \cup B} = \overline{A} \cap \overline{B} \]
	qui, avec des mots, dit
		\begin{center}
			\og non (A ou B) \fg $=$ \og ni A, ni B \fg $=$ \og (non A) et (non B) \fg.
		\end{center}
}

\exe{
	Dans une entreprise du bâtiment de $360$ employés, on a observé que $35\%$ des employés sont des cadres, et que $45\%$ des employés sont des femmes.
	Parmis les femmes, seules $48$ sont des cadres.
	
	On choisit un employé de l'entreprise uniformément au hasard : chaque personne a la même probabilité d'être choisie.
	On dénote les événements
		\begin{center}
			F : \og la personne est une femme \fg \qquad et \qquad C : \og la personne est un cadre \fg.
		\end{center}
	\begin{enumerate}
		%\item Réaliser un diagramme de Venn permettant de décrire les données de l'énoncé.
		\item Déterminer $P\left(\overline{F}\right)$ et $P(C)$.
		\item Calculer la probabilité que la personne choisie soit une femme cadre.
		\item Calculer $P(F \cup C)$ à l'aide de la formule d'inclusion-exclusion.
		\item 
		Justifier, en traduisant les événements $\overline{F} \cap \overline{C}$ et $\overline{F \cup C}$ par des phrases, l'égalité
			\[ \overline{F} \cap \overline{C} = \overline{F \cup C}, \]
		puis calculer $P(\overline{F} \cap \overline{C})$.
		\item À l'aide des événements $F$ et $C$, traduire l'événement \og la personne est un homme cadre \fg ~puis calculer sa probabilité.
	\end{enumerate}
}{}


\exe{
	On lance un D6 pipé avant de noter le numéro de la face du dessus.
	Les probabilités de chacunes des issues vérifient
		\[ P(1) = 2 P(2) = 4P(3) = 4 P(4) = 2P(5) = P(6). \]
		
	\begin{center}
	\begin{tabular}{|c|c|c|c|c|c|c|} \hline
		Résultat & 1 & 2 & 3 & 4 & 5 & 6 \\ \hline
		Probabilité & & & & & & \ifsolutions $\color{red} 0,2$ \fi \\ \hline
	\end{tabular}
	\end{center}
	\vspace{5pt}
	\begin{multicols}{2}
	\begin{enumerate}
		\item Compléter le tableau de probabilités.
		\item Calculer $P($\og le résultat est pair \fg$)$.
	\end{enumerate}
	\end{multicols}
}{
	Posons $p=P(2) \in [0;1]$ et exprimons chaque probabilité en fonction de $p$.
		\begin{align*}
			P(1) = 2p && P(2) = p && P(3) = 2p && P(4) = p && P(5) = 2p && P(6) = p.
		\end{align*}
	La relation $P(\Omega) = 1$ nous donne l'équation suivante à résoudre pour $p$.
		\begin{align*}
			P(1) + P(2) + P(3) + P(4) + P(5) + P (6) &= 1 \\
			2p + p + 2p + p + 2p + p &= 1 \\
			9p &= 1 \\
			p &= \dfrac19
		\end{align*}
	On en déduit le tableau de probabilités suivant.
	\begin{center}
	\begin{tabular}{|c|c|c|c|c|c|c|} \hline
		Résultat & 1 & 2 & 3 & 4 & 5 & 6 \\ \hline
		Probabilité & $\dfrac29$ & $\dfrac19$ & $\dfrac29$ & $\dfrac19$ & $\dfrac29$ & $\dfrac19$ \\ \hline
	\end{tabular}
	\end{center}
	Et on conclut que $P($\og le résultat est divisible par $3$ \fg$) = P(3) + P(6) = \dfrac29 + \dfrac19 = \dfrac13.$
}

\exe{
	L'univers associé à une expérience aléatoire est $\{ A, B, C\}$.
	La loi de probabilité $P$ est donnée par le tableau ci-dessous et dépend d'un réel $t\in\R$ encore inconnu.
	\begin{multicols}{2}
	\begin{enumerate}
		\item Développer le carré $\left(t-\dfrac35\right)^2$.
		\item Déterminer $t$.
		\item Compléter la dernière ligne du tableau.
	\end{enumerate}
	\end{multicols}
	\begin{center}
	\begin{tabular}{|c|c|c|c|} \hline
		Résultat & $A$ & $B$ & $C$ \\ \hline
		Probabilité & $\dfrac9{25}$ & $t^2$ & $-\dfrac65 t$ \\ \hline
		Probabilité ($t = \dots\dots$) & $\dfrac9{25}$ & & \\ \hline
	\end{tabular}
	\end{center}
}{
	On développe le carré à l'aide de l'identité remarquable
		\[ (a-b)^2 = a^2 + b^2 - 2ab, \]
	où, ici, on a $a=t$ et $b=\frac12.$
		\begin{align*}
			\left(t-\dfrac12\right)^2 &= t^2 + \left(\dfrac12\right)^2 - 2 \cdot t \cdot \dfrac12 \\
									&= t^2 + \dfrac14 - t
		\end{align*}
	On cherche désormais le $t\in\R$ pour lequel $P$ est une loi de probabilité. 
	Un loi vérifie les deux propriétés suivantes :
		\begin{itemize}
			\item $P(\omega) \in [0;1]$ pour chaque issue $\omega \in \Omega$ ; et
			\item $P(\Omega) = 1$.
		\end{itemize}
	La deuxième identité donne donc
		\begin{align*}
			P(a) + P(b) + P(c) = 1 && \iff && t^2 - t + \dfrac14 = 1.
		\end{align*}
	Le carré développé nous permet d'écrire
		\[ \left(t-\dfrac12\right)^2 = 1, \]
	et donc
		\[ \left|t-\dfrac12\right| = \sqrt{1} = 1, \]
	en utilisant le fait que $\sqrt{x^2} = |x|.$
	L'expression à l'intérieur de la valeur absolue est donc soit $+1$, soit $-1$, et on a donc deux alternatives :
		\begin{align*}
			t-\dfrac12 = 1 && \text{ ou } && t - \dfrac12 = -1 \\
			t = \dfrac32 && \text{ ou } && t = -\dfrac12.
		\end{align*}
	Pour s'entraîner à ce genre de résolution, voir la feuille d'exercices Fonctions 3.
	
	Comme les probabilités sont des nombres entre $0$ et $1$, on peut écarter la première solution car $P(a) = t^2$ serait strictement supérieur à $1$, et $P(b) = -t$, serait strictement négatif.
	Il ne reste donc que $t = -\frac12$, qui donne le tableau de probabilités suivant.
	\begin{center}
	\begin{tabular}{|c|c|c|c|} \hline
		Issue & $a$ & $b$ & $c$ \\ \hline
		Probabilité & $\dfrac14$ & $\dfrac12$ & $\dfrac14$ \\ \hline
	\end{tabular}
	\end{center}
}

\end{document}

				% ENABLE or DISABLE font change
				% use XeLaTeX if true
\newif\ifdys
				\dystrue
				\dysfalse

\newif\ifsolutions
				\solutionstrue
				\solutionsfalse

% DYSLEXIA SWITCH
\newif\ifdys
		
				% ENABLE or DISABLE font change
				% use XeLaTeX if true
				\dystrue
				\dysfalse


\ifdys

\documentclass[a4paper, 14pt]{extarticle}
\usepackage{amsmath,amsfonts,amsthm,amssymb,mathtools}

\tracinglostchars=3 % Report an error if a font does not have a symbol.
\usepackage{fontspec}
\usepackage{unicode-math}
\defaultfontfeatures{ Ligatures=TeX,
                      Scale=MatchUppercase }

\setmainfont{OpenDyslexic}[Scale=1.0]
\setmathfont{Fira Math} % Or maybe try KPMath-Sans?
\setmathfont{OpenDyslexic Italic}[range=it/{Latin,latin}]
\setmathfont{OpenDyslexic}[range=up/{Latin,latin,num}]

\else

\documentclass[a4paper, 12pt]{extarticle}

\usepackage[utf8x]{inputenc}
%fonts
\usepackage{amsmath,amsfonts,amsthm,amssymb,mathtools}
% comment below to default to computer modern
\usepackage{libertinus,libertinust1math}

\fi


\usepackage[french]{babel}
\usepackage[
a4paper,
margin=2cm,
nomarginpar,% We don't want any margin paragraphs
]{geometry}
\usepackage{icomma}

\usepackage{fancyhdr}
\usepackage{array}
\usepackage{hyperref}

\usepackage{multicol, enumerate}
\newcolumntype{P}[1]{>{\centering\arraybackslash}p{#1}}


\usepackage{stackengine}
\newcommand\xrowht[2][0]{\addstackgap[.5\dimexpr#2\relax]{\vphantom{#1}}}

% theorems

\theoremstyle{plain}
\newtheorem{theorem}{Th\'eor\`eme}
\newtheorem*{sol}{Solution}
\theoremstyle{definition}
\newtheorem{ex}{Exercice}
\newtheorem*{rpl}{Rappel}
\newtheorem{enigme}{Énigme}

% corps
\usepackage{calrsfs}
\newcommand{\C}{\mathcal{C}}
\newcommand{\R}{\mathbb{R}}
\newcommand{\Rnn}{\mathbb{R}^{2n}}
\newcommand{\Z}{\mathbb{Z}}
\newcommand{\N}{\mathbb{N}}
\newcommand{\Q}{\mathbb{Q}}

% variance
\newcommand{\Var}[1]{\text{Var}(#1)}

% domain
\newcommand{\D}{\mathcal{D}}


% date
\usepackage{advdate}
\AdvanceDate[0]


% plots
\usepackage{pgfplots}

% table line break
\usepackage{makecell}
%tablestuff
\def\arraystretch{2}
\setlength\tabcolsep{15pt}

%subfigures
\usepackage{subcaption}

\definecolor{myg}{RGB}{56, 140, 70}
\definecolor{myb}{RGB}{45, 111, 177}
\definecolor{myr}{RGB}{199, 68, 64}

% fake sections with no title to move around the merged pdf
\newcommand{\fakesection}[1]{%
  \par\refstepcounter{section}% Increase section counter
  \sectionmark{#1}% Add section mark (header)
  \addcontentsline{toc}{section}{\protect\numberline{\thesection}#1}% Add section to ToC
  % Add more content here, if needed.
}


% SOLUTION SWITCH
\newif\ifsolutions
				\solutionstrue
				%\solutionsfalse

\ifsolutions
	\newcommand{\exe}[2]{
		\begin{ex} #1  \end{ex}
		\begin{sol} #2 \end{sol}
	}
\else
	\newcommand{\exe}[2]{
		\begin{ex} #1  \end{ex}
	}
	
\fi


% tableaux var, signe
\usepackage{tkz-tab}


%pinfty minfty
\newcommand{\pinfty}{{+}\infty}
\newcommand{\minfty}{{-}\infty}

\begin{document}


\AdvanceDate[1]

\begin{document}
\pagestyle{fancy}
\fancyhead[L]{Seconde 13}
\fancyhead[C]{\textbf{Fonctions de référence \ifsolutions \, -- Solutions  \fi}}
\fancyhead[R]{\today}

\begin{figure}

\begin{subfigure}{.5\textwidth}
\begin{mintedbox}{python}
# exemple de fonction affine
def aff(x):
	return -2*x + 1

# la fonction carré
def carre(x):
	return # à compléter

# la fonction inverse
def inv(x):
	return # à compléter
	
# la fonction cube
def cube(x):
	return # à compléter
	
# la fonction racine carrée
import numpy as np
def racine(x):
	return np.sqrt(x)
\end{mintedbox}
\end{subfigure}
\begin{subfigure}{.5\textwidth}
\begin{mintedbox}{python}
import numpy as np
import matplotlib.pyplot as plt

# N, nombre d'antécédents du domaine
N = 2
# N antécédents entre -3 et 3
x = np.linspace(-3,3,N)
# images par f
y = f(x)

# graphe les couples (x ; y)
plt.plot(x,y)
# affiche un quadrillage
plt.grid()
# affiche le graphe
plt.show()
\end{mintedbox}
\end{subfigure}

\caption{rr.}
\label{fig:1}
\end{figure}

\exe{
	Implémenter la fonction affine $f(x) = -2x+1$ et grapher sa courbe représentative sur le domaine $[-3;3]$ en choisissant $2$ points.
}{}


	\begin{multicols}{2}
	\begin{tikzpicture}[scale=1.1]
		\begin{axis}[xmin = -3, xmax=3, ymin=-10, ymax=10, axis x line=middle, axis y line=middle, axis line style=->, grid=both,
		ytick={-10,-8,...,10},
		%ytick={-4,-3,...,2,3}, xtick={-11, -10,...,-4,-3},
	    	%every y tick label/.style={
	        %anchor=near yticklabel opposite,
	       % xshift=0.2em,
	    	%}
	    	]
		\end{axis}
	\end{tikzpicture}
	
	\hfill
	\begin{tikzpicture}
		\tkzTabInit
		 [espcl=6]
	       		{$x$ / 1 , Variation de $f(x)$ / 2}
	       		{$-3$,$3$}
	\end{tikzpicture}
	
	\begin{tikzpicture}
		\tkzTabInit
		 [espcl=6]
	       		{$x$ / 1 , Signe de $f(x)$ / 1}
	       		{$-3$,$3$}
	\end{tikzpicture}
	\end{multicols}


\end{document}

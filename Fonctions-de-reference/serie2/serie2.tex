				% ENABLE or DISABLE font change
				% use XeLaTeX if true
\newif\ifdys
				\dystrue
				\dysfalse

\newif\ifsolutions
				\solutionstrue
				\solutionsfalse

% DYSLEXIA SWITCH
\newif\ifdys
		
				% ENABLE or DISABLE font change
				% use XeLaTeX if true
				\dystrue
				\dysfalse


\ifdys

\documentclass[a4paper, 14pt]{extarticle}
\usepackage{amsmath,amsfonts,amsthm,amssymb,mathtools}

\tracinglostchars=3 % Report an error if a font does not have a symbol.
\usepackage{fontspec}
\usepackage{unicode-math}
\defaultfontfeatures{ Ligatures=TeX,
                      Scale=MatchUppercase }

\setmainfont{OpenDyslexic}[Scale=1.0]
\setmathfont{Fira Math} % Or maybe try KPMath-Sans?
\setmathfont{OpenDyslexic Italic}[range=it/{Latin,latin}]
\setmathfont{OpenDyslexic}[range=up/{Latin,latin,num}]

\else

\documentclass[a4paper, 12pt]{extarticle}

\usepackage[utf8x]{inputenc}
%fonts
\usepackage{amsmath,amsfonts,amsthm,amssymb,mathtools}
% comment below to default to computer modern
\usepackage{libertinus,libertinust1math}

\fi


\usepackage[french]{babel}
\usepackage[
a4paper,
margin=2cm,
nomarginpar,% We don't want any margin paragraphs
]{geometry}
\usepackage{icomma}

\usepackage{fancyhdr}
\usepackage{array}
\usepackage{hyperref}

\usepackage{multicol, enumerate}
\newcolumntype{P}[1]{>{\centering\arraybackslash}p{#1}}


\usepackage{stackengine}
\newcommand\xrowht[2][0]{\addstackgap[.5\dimexpr#2\relax]{\vphantom{#1}}}

% theorems

\theoremstyle{plain}
\newtheorem{theorem}{Th\'eor\`eme}
\newtheorem*{sol}{Solution}
\theoremstyle{definition}
\newtheorem{ex}{Exercice}
\newtheorem*{rpl}{Rappel}
\newtheorem{enigme}{Énigme}

% corps
\usepackage{calrsfs}
\newcommand{\C}{\mathcal{C}}
\newcommand{\R}{\mathbb{R}}
\newcommand{\Rnn}{\mathbb{R}^{2n}}
\newcommand{\Z}{\mathbb{Z}}
\newcommand{\N}{\mathbb{N}}
\newcommand{\Q}{\mathbb{Q}}

% variance
\newcommand{\Var}[1]{\text{Var}(#1)}

% domain
\newcommand{\D}{\mathcal{D}}


% date
\usepackage{advdate}
\AdvanceDate[0]


% plots
\usepackage{pgfplots}

% table line break
\usepackage{makecell}
%tablestuff
\def\arraystretch{2}
\setlength\tabcolsep{15pt}

%subfigures
\usepackage{subcaption}

\definecolor{myg}{RGB}{56, 140, 70}
\definecolor{myb}{RGB}{45, 111, 177}
\definecolor{myr}{RGB}{199, 68, 64}

% fake sections with no title to move around the merged pdf
\newcommand{\fakesection}[1]{%
  \par\refstepcounter{section}% Increase section counter
  \sectionmark{#1}% Add section mark (header)
  \addcontentsline{toc}{section}{\protect\numberline{\thesection}#1}% Add section to ToC
  % Add more content here, if needed.
}


% SOLUTION SWITCH
\newif\ifsolutions
				\solutionstrue
				%\solutionsfalse

\ifsolutions
	\newcommand{\exe}[2]{
		\begin{ex} #1  \end{ex}
		\begin{sol} #2 \end{sol}
	}
\else
	\newcommand{\exe}[2]{
		\begin{ex} #1  \end{ex}
	}
	
\fi


% tableaux var, signe
\usepackage{tkz-tab}


%pinfty minfty
\newcommand{\pinfty}{{+}\infty}
\newcommand{\minfty}{{-}\infty}

\begin{document}


\AdvanceDate[1]

\begin{document}
\pagestyle{fancy}
\fancyhead[L]{Seconde 13}
\fancyhead[C]{\textbf{Fonctions parentes : transformations des abscisses \ifsolutions \, -- Solutions  \fi}}
\fancyhead[R]{\today}

\ex{
	On donne la courbe $\C_f$ graphiquement ci-dessous sur le domaine $\D = [-10 ; 7]$.
	
	\begin{enumerate}
		\item Esquisser, dans le repère de gauche, la courbe de $g(x) = f(x+1)$.
		Quel est le domaine de $g$ ?
		\item Esquisser, dans le repère de droite, la courbe de $h(x) = f(x-2)$.
		Quel est le domaine de $h$ ?
	\end{enumerate}
	
	\begin{multicols}{2}
	\begin{tikzpicture}[scale=1.1]
		\begin{axis}[xmin = -10, xmax=7, ymin=-3.25, ymax=3.25, axis x line=middle, axis y line=middle, axis line style=->, grid=both,
		ytick={-4,-3,...,4},
	    	]
		% g cos
		\addplot[no marks, myb, -, very thick] expression[domain=-10:7, samples=100]{2*sin(x*30)}
		node[pos=.1, above]{$\mathcal{C}_f$};
		\ifsolutions
		\addplot[no marks, myg, -, very thick] expression[domain=-10:6, samples=100]{2*sin((x+1)*30)}
		node[pos=.7, above]{$\mathcal{C}_g$};
		\fi
		\end{axis}
	\end{tikzpicture}
	
	
	\begin{tikzpicture}[scale=1.1]
		\begin{axis}[xmin = -10, xmax=7, ymin=-3.25, ymax=3.25, axis x line=middle, axis y line=middle, axis line style=->, grid=both,
		ytick={-4,-3,...,4},
	    	]
		% g cos
		\addplot[no marks, myb, -, very thick] expression[domain=-10:7, samples=100]{2*sin(x*30)}
		node[pos=.1, above]{$\mathcal{C}_f$};
		\ifsolutions
		\addplot[no marks, myr, -, very thick] expression[domain=-8:7, samples=100]{2*sin((x-2)*30)}
		node[pos=.25, above]{$\mathcal{C}_h$};
		\fi
		\end{axis}
	\end{tikzpicture}
	\end{multicols}
}

\ex{
	On donne la courbe $\C_f$ graphiquement ci-dessous.
	
	\begin{enumerate}
		\item Esquisser la courbe de $g(x) = f(-x)$. Que dire de $\C_g$ par rapport à $\C_f$ ?
		\item Esquisser la courbe de $h(x) = -f(-x)$.Que dire de $\C_h$ par rapport à $\C_f$ ?
	\end{enumerate}
	
	\begin{multicols}{2}
	\begin{tikzpicture}[scale=1.1]
		\begin{axis}[xmin = -9, xmax=9, ymin=-4.25, ymax=3.25, axis x line=middle, axis y line=middle, axis line style=->, grid=both,
		ytick={-4,-3,...,4},
		xtick={-8, -6, ..., 8},
	    	]
		\addplot[myb, -, very thick] expression[domain=-9:9, samples=100]{(x+6)*(x-2)*(x+12)*(x-9)/1000}
		node[pos=.35, above]{$\mathcal{C}_f$};
		\ifsolutions
		\addplot[no marks, myg, -, very thick] expression[domain=-9:9, samples=100]{(-x+6)*(-x-2)*(-x+12)*(-x-9)/1000}
		node[pos=.7, above]{$\mathcal{C}_g$};
		 \fi
		\end{axis}
	\end{tikzpicture}
	
	\begin{tikzpicture}[scale=1.1]
		\begin{axis}[xmin = -9, xmax=9, ymin=-4.25, ymax=3.25, axis x line=middle, axis y line=middle, axis line style=->, grid=both,
		ytick={-4,-3,...,4},
		xtick={-8, -6, ..., 8},
	    	]
		\addplot[myb, -, very thick] expression[domain=-9:9, samples=100]{(x+6)*(x-2)*(x+12)*(x-9)/1000}
		node[pos=.35, above]{$\mathcal{C}_f$};
		\ifsolutions
		\addplot[no marks, myr, -, very thick] expression[domain=-9:9, samples=100]{-(-x+6)*(-x-2)*(-x+12)*(-x-9)/1000}
		node[pos=.15, above]{$\mathcal{C}_g$};
		 \fi
		\end{axis}
	\end{tikzpicture}
	\end{multicols}
}

\begin{dfn*}
	On dit d'une fonction réelle $f$ définie sur $\R$ qu'elle est \emph{paire} si, pour tous les $x\in\R$,
		\[ f(-x) = \ifsolutions f(x) \else \qquad\qquad \fi. \]
	On dit qu'elle est \emph{impaire} si, pour tous les $x\in\R$,
		\[ f(-x) = \ifsolutions -f(x)\else \qquad\qquad \fi. \]
\end{dfn*}

%\newpage

\exe{
	Donner le domaine de définition de chaque fonction suivante, puis montrer qu'elles sont paires.
		
		\begin{multicols}{2}
		\begin{enumerate}
			\item $f(x) = x^2$
			\item $g(x) = 4x^2 + 7$
			\item $h(x) = \dfrac{1}{4x^2 + 7}$
			\item $k(x) = x^4$
		\end{enumerate}
		\end{multicols}
}

\exe{
	Donner le domaine de définition de chaque fonction suivante, puis montrer qu'elles sont impaires.
		
		\begin{multicols}{2}
		\begin{enumerate}
			\item $f(x) = 3x$
			\item $g(x) = x^3$
			\item $h(x) = x^3 - 2x$
			\item $k(x) = (x^3 - 2x)^3$
		\end{enumerate}
		\end{multicols}
}

\exe{
	Soit $f$ une fonction réelle définie sur $\R$ tout entier.
	Montrer que, pour tout $x\in\R$,
		\[ f(x) = \dfrac{f(x) + f(-x)}2 + \dfrac{f(x) - f(-x)}2. \]
	En déduire que $f$ est la somme d'une fonction paire et d'une fonction impaire.
}{}

\hrule 

\subsection*{Interpolation de Lagrange}

\begin{enigme}\label{enigme:1}
	Compléter la suite logique suivante.
		\begin{center}
			$3$ --- $4$ --- ?
		\end{center}
\end{enigme}

\exe{
	Le but de l'exercice est de répondre à l'énigme \ref{enigme:1} de deux façons différentes.
	On cherche deux fonctions polynomiales $f$ et $g$ telles que 
		\begin{enumerate}[label=(\roman*)]
			\item $ f(1) = g(1) = 3$ ;
			\item $f(2) = g(2) = 4$ ; et
			\item $f(3) \neq g(3)$.
		\end{enumerate}
	On aura ainsi deux façons cohérentes de compléter la suite de nombres :
		\begin{center}
			$3$ --- $4$ --- $f(3)$, \hspace{3cm} et \hspace{3cm} $3$ --- $4$ --- $g(3)$.
		\end{center}
	
	\begin{enumerate}
		\item Montrer que $a(x) = (x-2)(x-3)$ s'annule en $2$ et en $3$, mais pas en $1$.
		\item Montrer que $b(x) = (x-1)(x-3)$ s'annule en $1$ et en $3$, mais pas en $2$.
		\item Montrer que $c(x) = (x-1)(x-2)$ s'annule en $1$ et $2$, mais pas en $3$.
		\item \underline{\smash{Sans développer l'expression}}, montrer que la fonction
			\[ f(x) = \dfrac{3}{a(1)} a(x) + \dfrac{4}{b(2)} b(x)  + \dfrac{5}{c(3)} c(x) \]
		vérifie que
			\begin{enumerate}[label=(\roman*)]
				\item $f(1) = 3$ ;
				\item $f(2) = 4$ ; et
				\item $f(3) = 5$.
			\end{enumerate}
		\item Développer et réduire l'expression de $f$ pour trouver $f(x) = x+2$.
		\item Construire une fonction $g$ telle que
			\begin{enumerate}[label=(\roman*)]
				\item $g(1) = 3$ ;
				\item $g(2) = 4$ ; et
				\item $g(3) = 7$.
			\end{enumerate}
		\item Développer et réduire l'expression de $g$ pour trouver $g(x) = x^2 - 2x + 4$.
	\end{enumerate}
}{}

\begin{enigme}[Bonus]\label{enigme:2}
	Compléter la suite logique suivante de la façon souhaitée en trouvant un polynôme $f$ de degré 3\footnote{C'est-à-dire $f(x) = ax^3 + bx^2 + cx + d$ pour certains coefficients $a, b, c, d \in \R$. Le degré est la plus haute puissance de $x$.} tel que la suite soit $f(1)$ — $f(2)$ — $f(3)$ — $f(4)$.
		\begin{center}
			$3$ --- $4$ --- $5$ --- ?
		\end{center}
\end{enigme}

\end{document}

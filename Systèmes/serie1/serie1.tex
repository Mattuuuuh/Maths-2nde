				% ENABLE or DISABLE font change
				% use XeLaTeX if true
\newif\ifdys
				\dystrue
				\dysfalse

\newif\ifsolutions
				\solutionstrue
				\solutionsfalse

%!TEX encoding = UTF8
%!TEX root =notes.tex


%%%%%%%%%%%%%%%%%%%%%%%%%%%%%%%%%
% PACKAGE IMPORTS
%%%%%%%%%%%%%%%%%%%%%%%%%%%%%%%%%


\usepackage[french]{babel}

\usepackage[tmargin=2cm,rmargin=1in,lmargin=1in,margin=0.85in,bmargin=2cm,footskip=.2in]{geometry}
\usepackage{amsmath,amsfonts,amsthm,amssymb,mathtools}
\usepackage[varbb]{newpxmath}
\usepackage{xfrac}
\usepackage[makeroom]{cancel}
\usepackage{mathtools}
\usepackage{bookmark}
\usepackage{enumitem}
\usepackage{hyperref,theoremref}
\hypersetup{
	pdftitle={Assignment},
	colorlinks=true, linkcolor=doc!90,
	bookmarksnumbered=true,
	bookmarksopen=true
}
\usepackage[most,many,breakable]{tcolorbox}
\usepackage{xcolor}
\usepackage{varwidth}
\usepackage{varwidth}
\usepackage{etoolbox}
%\usepackage{authblk}
\usepackage{nameref}
\usepackage{multicol,array}
\usepackage{tikz-cd}
\usepackage[ruled,vlined,linesnumbered]{algorithm2e}
\usepackage{comment} % enables the use of multi-line comments (\ifx \fi) 
\usepackage{import}
\usepackage{xifthen}
\usepackage{pdfpages}
\usepackage{transparent}


\newcommand\mycommfont[1]{\footnotesize\ttfamily\textcolor{blue}{#1}}
\SetCommentSty{mycommfont}
\newcommand{\incfig}[1]{%
    \def\svgwidth{\columnwidth}
    \import{./figures/}{#1.pdf_tex}
}

\usepackage{tikzsymbols}
%\renewcommand\qedsymbol{$\Laughey$}


%\usepackage{import}
%\usepackage{xifthen}
%\usepackage{pdfpages}
%\usepackage{transparent}


%%%%%%%%%%%%%%%%%%%%%%%%%%%%%%
% SELF MADE COLORS
%%%%%%%%%%%%%%%%%%%%%%%%%%%%%%



\definecolor{myg}{RGB}{56, 140, 70}
\definecolor{myb}{RGB}{45, 111, 177}
\definecolor{myr}{RGB}{199, 68, 64}
\definecolor{mytheorembg}{HTML}{F2F2F9}
\definecolor{mytheoremfr}{HTML}{00007B}
\definecolor{mylenmabg}{HTML}{FFFAF8}
\definecolor{mylenmafr}{HTML}{983b0f}
\definecolor{mypropbg}{HTML}{f2fbfc}
\definecolor{mypropfr}{HTML}{191971}
\definecolor{myexamplebg}{HTML}{F2FBF8}
\definecolor{myexamplefr}{HTML}{88D6D1}
\definecolor{myexampleti}{HTML}{2A7F7F}
\definecolor{mydefinitbg}{HTML}{E5E5FF}
\definecolor{mydefinitfr}{HTML}{3F3FA3}
\definecolor{notesgreen}{RGB}{0,162,0}
\definecolor{myp}{RGB}{197, 92, 212}
\definecolor{mygr}{HTML}{2C3338}
\definecolor{myred}{RGB}{127,0,0}
\definecolor{myyellow}{RGB}{169,121,69}
\definecolor{myexercisebg}{HTML}{F2FBF8}
\definecolor{myexercisefg}{HTML}{88D6D1}


%%%%%%%%%%%%%%%%%%%%%%%%%%%%
% TCOLORBOX SETUPS
%%%%%%%%%%%%%%%%%%%%%%%%%%%%

\setlength{\parindent}{1cm}
%================================
% THEOREM BOX
%================================

\tcbuselibrary{theorems,skins,hooks}
\newtcbtheorem[number within=chapter]{Theorem}{Théorème}
{%
	enhanced,
	breakable,
	colback = mytheorembg,
	frame hidden,
	boxrule = 0sp,
	borderline west = {2pt}{0pt}{mytheoremfr},
	sharp corners,
	detach title,
	before upper = \tcbtitle\par\smallskip,
	coltitle = mytheoremfr,
	fonttitle = \bfseries\sffamily,
	description font = \mdseries,
	separator sign none,
	segmentation style={solid, mytheoremfr},
}
{th}


\tcbuselibrary{theorems,skins,hooks}
\newtcolorbox{Theoremcon}
{%
	enhanced
	,breakable
	,colback = mytheorembg
	,frame hidden
	,boxrule = 0sp
	,borderline west = {2pt}{0pt}{mytheoremfr}
	,sharp corners
	,description font = \mdseries
	,separator sign none
}

%================================
% Corollery
%================================
\tcbuselibrary{theorems,skins,hooks}
\newtcbtheorem[use counter=tcb@cnt@Theorem]{Corollary}{Corollaire}
{%
	enhanced
	,breakable
	,colback = myp!10
	,frame hidden
	,boxrule = 0sp
	,borderline west = {2pt}{0pt}{myp!85!black}
	,sharp corners
	,detach title
	,before upper = \tcbtitle\par\smallskip
	,coltitle = myp!85!black
	,fonttitle = \bfseries\sffamily
	,description font = \mdseries
	,separator sign none
	,segmentation style={solid, myp!85!black}
}
{th}

%================================
% LENMA
%================================

\tcbuselibrary{theorems,skins,hooks}
\newtcbtheorem[use counter=tcb@cnt@Theorem]{Lemma}{Lemme}
{%
	enhanced,
	breakable,
	colback = mylenmabg,
	frame hidden,
	boxrule = 0sp,
	borderline west = {2pt}{0pt}{mylenmafr},
	sharp corners,
	detach title,
	before upper = \tcbtitle\par\smallskip,
	coltitle = mylenmafr,
	fonttitle = \bfseries\sffamily,
	description font = \mdseries,
	separator sign none,
	segmentation style={solid, mylenmafr},
}
{th}


%================================
% PROPOSITION
%================================

\tcbuselibrary{theorems,skins,hooks}
\newtcbtheorem[use counter=tcb@cnt@Theorem]{Prop}{Proposition}
{%
	enhanced,
	breakable,
	colback = mypropbg,
	frame hidden,
	boxrule = 0sp,
	borderline west = {2pt}{0pt}{mypropfr},
	sharp corners,
	detach title,
	before upper = \tcbtitle\par\smallskip,
	coltitle = mypropfr,
	fonttitle = \bfseries\sffamily,
	description font = \mdseries,
	separator sign none,
	segmentation style={solid, mypropfr},
}
{th}


%================================
% CLAIM
%================================

\tcbuselibrary{theorems,skins,hooks}
\newtcbtheorem[use counter=tcb@cnt@Theorem]{claim}{Claim}
{%
	enhanced
	,breakable
	,colback = myg!10
	,frame hidden
	,boxrule = 0sp
	,borderline west = {2pt}{0pt}{myg}
	,sharp corners
	,detach title
	,before upper = \tcbtitle\par\smallskip
	,coltitle = myg!85!black
	,fonttitle = \bfseries\sffamily
	,description font = \mdseries
	,separator sign none
	,segmentation style={solid, myg!85!black}
}
{th}



%================================
% Exercise
%================================

\tcbuselibrary{theorems,skins,hooks}
\newtcbtheorem[use counter=tcb@cnt@Theorem]{Exercise}{Exercice}
{%
	enhanced,
	breakable,
	colback = myexercisebg,
	frame hidden,
	boxrule = 0sp,
	borderline west = {2pt}{0pt}{myexercisefg},
	sharp corners,
	detach title,
	before upper = \tcbtitle\par\smallskip,
	coltitle = myexercisefg,
	fonttitle = \bfseries\sffamily,
	description font = \mdseries,
	separator sign none,
	segmentation style={solid, myexercisefg},
}
{th}

%================================
% EXAMPLE BOX
%================================

\newtcbtheorem[use counter=tcb@cnt@Theorem]{Example}{Exemple}
{%
	colback = myexamplebg
	,breakable
	,colframe = myexamplefr
	,coltitle = myexampleti
	,boxrule = 1pt
	,sharp corners
	,detach title
	,before upper=\tcbtitle\par\smallskip
	,fonttitle = \bfseries
	,description font = \mdseries
	,separator sign none
	,description delimiters parenthesis
}
{ex}

%================================
% DEFINITION BOX
%================================

\newtcbtheorem[use counter=tcb@cnt@Theorem]{Definition}{Définition}{enhanced,
	before skip=2mm,after skip=2mm, colback=red!5,colframe=red!80!black,boxrule=0.5mm,
	attach boxed title to top left={xshift=1cm,yshift*=1mm-\tcboxedtitleheight}, varwidth boxed title*=-3cm,
	boxed title style={frame code={
					\path[fill=tcbcolback]
					([yshift=-1mm,xshift=-1mm]frame.north west)
					arc[start angle=0,end angle=180,radius=1mm]
					([yshift=-1mm,xshift=1mm]frame.north east)
					arc[start angle=180,end angle=0,radius=1mm];
					\path[left color=tcbcolback!60!black,right color=tcbcolback!60!black,
						middle color=tcbcolback!80!black]
					([xshift=-2mm]frame.north west) -- ([xshift=2mm]frame.north east)
					[rounded corners=1mm]-- ([xshift=1mm,yshift=-1mm]frame.north east)
					-- (frame.south east) -- (frame.south west)
					-- ([xshift=-1mm,yshift=-1mm]frame.north west)
					[sharp corners]-- cycle;
				},interior engine=empty,
		},
	fonttitle=\bfseries,
	title={#2},#1}{def}

%================================
% Solution BOX
%================================

\makeatletter
\newtcbtheorem[use counter=tcb@cnt@Theorem]{question}{Question}{enhanced,
	breakable,
	colback=white,
	colframe=myb!80!black,
	attach boxed title to top left={yshift*=-\tcboxedtitleheight},
	fonttitle=\bfseries,
	title={#2},
	boxed title size=title,
	boxed title style={%
			sharp corners,
			rounded corners=northwest,
			colback=tcbcolframe,
			boxrule=0pt,
		},
	underlay boxed title={%
			\path[fill=tcbcolframe] (title.south west)--(title.south east)
			to[out=0, in=180] ([xshift=5mm]title.east)--
			(title.center-|frame.east)
			[rounded corners=\kvtcb@arc] |-
			(frame.north) -| cycle;
		},
	#1
}{def}
\makeatother

%================================
% SOLUTION BOX
%================================

\makeatletter
\newtcolorbox{solution}{enhanced,
	breakable,
	colback=white,
	colframe=myg!80!black,
	attach boxed title to top left={yshift*=-\tcboxedtitleheight},
	title=Solution,
	boxed title size=title,
	boxed title style={%
			sharp corners,
			rounded corners=northwest,
			colback=tcbcolframe,
			boxrule=0pt,
		},
	underlay boxed title={%
			\path[fill=tcbcolframe] (title.south west)--(title.south east)
			to[out=0, in=180] ([xshift=5mm]title.east)--
			(title.center-|frame.east)
			[rounded corners=\kvtcb@arc] |-
			(frame.north) -| cycle;
		},
}
\makeatother

%================================
% Question BOX
%================================

\makeatletter
\newtcbtheorem[use counter=tcb@cnt@Theorem]{qstion}{Question}{enhanced,
	breakable,
	colback=white,
	colframe=mygr,
	attach boxed title to top left={yshift*=-\tcboxedtitleheight},
	fonttitle=\bfseries,
	title={#2},
	boxed title size=title,
	boxed title style={%
			sharp corners,
			rounded corners=northwest,
			colback=tcbcolframe,
			boxrule=0pt,
		},
	underlay boxed title={%
			\path[fill=tcbcolframe] (title.south west)--(title.south east)
			to[out=0, in=180] ([xshift=5mm]title.east)--
			(title.center-|frame.east)
			[rounded corners=\kvtcb@arc] |-
			(frame.north) -| cycle;
		},
	#1
}{def}
\makeatother

\newtcbtheorem[number within=chapter]{wconc}{Wrong Concept}{
	breakable,
	enhanced,
	colback=white,
	colframe=myr,
	arc=0pt,
	outer arc=0pt,
	fonttitle=\bfseries\sffamily\large,
	colbacktitle=myr,
	attach boxed title to top left={},
	boxed title style={
			enhanced,
			skin=enhancedfirst jigsaw,
			arc=3pt,
			bottom=0pt,
			interior style={fill=myr}
		},
	#1
}{def}



%================================
% NOTE BOX
%================================

\usetikzlibrary{arrows,calc,shadows.blur}
\tcbuselibrary{skins}
\newtcolorbox{note}[1][]{%
	enhanced jigsaw,
	colback=gray!20!white,%
	colframe=gray!80!black,
	size=small,
	boxrule=1pt,
	title=\colorbox{white!100}{\textbf{ Remarque }},
	halign title=flush center,
	coltitle=black,
	breakable,
	drop shadow=black!50!white,
	attach boxed title to top left={xshift=1cm,yshift=-\tcboxedtitleheight/2,yshifttext=-\tcboxedtitleheight/2},
	minipage boxed title=2.6cm,
	boxed title style={%
			colback=white,
			size=fbox,
			boxrule=1pt,
			boxsep=2pt,
			underlay={%
					\coordinate (dotA) at ($(interior.west) + (-0.5pt,0)$);
					\coordinate (dotB) at ($(interior.east) + (0.5pt,0)$);
					\begin{scope}
						\clip (interior.north west) rectangle ([xshift=3ex]interior.east);
						\filldraw [white, blur shadow={shadow opacity=60, shadow yshift=-.75ex}, rounded corners=2pt] (interior.north west) rectangle (interior.south east);
					\end{scope}
					\begin{scope}[gray!80!black]
						\fill (dotA) circle (2pt);
						\fill (dotB) circle (2pt);
					\end{scope}
				},
		},
	#1,
}

%================================
% STRATÉGIE BOX
%================================

\usetikzlibrary{arrows,calc,shadows.blur}
\tcbuselibrary{skins}
\newtcolorbox{strategy}[1][]{%
	enhanced jigsaw,
	colback=myb!20!white,%
	colframe=gray!80!black,
	size=small,
	boxrule=1pt,
	title=\colorbox{white!100}{\textbf{ Stratégie }},
	halign title=flush center,
	coltitle=black,
	breakable,
	drop shadow=black!50!white,
	attach boxed title to top left={xshift=1cm,yshift=-\tcboxedtitleheight/2,yshifttext=-\tcboxedtitleheight/2},
	minipage boxed title=2.5cm,
	boxed title style={%
			colback=white,
			size=fbox,
			boxrule=1pt,
			boxsep=2pt,
			underlay={%
					\coordinate (dotA) at ($(interior.west) + (-0.5pt,0)$);
					\coordinate (dotB) at ($(interior.east) + (0.5pt,0)$);
					\begin{scope}
						\clip (interior.north west) rectangle ([xshift=3ex]interior.east);
						\filldraw [white, blur shadow={shadow opacity=60, shadow yshift=-.75ex}, rounded corners=2pt] (interior.north west) rectangle (interior.south east);
					\end{scope}
					\begin{scope}[gray!80!black]
						\fill (dotA) circle (2pt);
						\fill (dotB) circle (2pt);
					\end{scope}
				},
		},
	#1,
}

%================================
% MÉTHODE BOX
%================================

\usetikzlibrary{arrows,calc,shadows.blur}
\tcbuselibrary{skins}
\newtcolorbox{methode}[1][]{%
	enhanced jigsaw,
	colback=white,%
	colframe=gray!80!black,
	size=small,
	boxrule=1pt,
	title=\textbf{Méthode},
	halign title=flush center,
	coltitle=black,
	breakable,
	drop shadow=black!50!white,
	attach boxed title to top left={xshift=1cm,yshift=-\tcboxedtitleheight/2,yshifttext=-\tcboxedtitleheight/2},
	minipage boxed title=2.5cm,
	boxed title style={%
			colback=white,
			size=fbox,
			boxrule=1pt,
			boxsep=2pt,
			underlay={%
					\coordinate (dotA) at ($(interior.west) + (-0.5pt,0)$);
					\coordinate (dotB) at ($(interior.east) + (0.5pt,0)$);
					\begin{scope}
						\clip (interior.north west) rectangle ([xshift=3ex]interior.east);
						\filldraw [white, blur shadow={shadow opacity=60, shadow yshift=-.75ex}, rounded corners=2pt] (interior.north west) rectangle (interior.south east);
					\end{scope}
					\begin{scope}[gray!80!black]
						\fill (dotA) circle (2pt);
						\fill (dotB) circle (2pt);
					\end{scope}
				},
		},
	#1,
}

%%%%%%%%%%%%%%%%%%%%%%%%%%%%%%%%%%%%%%%%%%%
% TABLE OF CONTENTS
%%%%%%%%%%%%%%%%%%%%%%%%%%%%%%%%%%%%%%%%%%%

\usepackage{tikz}

\definecolor{doc}{RGB}{0,60,110}
\usepackage{titletoc}
\contentsmargin{0cm}
\titlecontents{chapter}[3.7pc]
{\addvspace{30pt}%
	\begin{tikzpicture}[remember picture, overlay]%
		\draw[fill=doc!60,draw=doc!60] (-7,-.1) rectangle (-0.2,.6);%
		\pgftext[left,x=-3.5cm,y=0.2cm]{\color{white}\Large\sc\bfseries Chapitre\ \thecontentslabel};%
	\end{tikzpicture}\color{doc!60}\large\sc\bfseries}%
{}
{}
{\;\titlerule\;\large\sc\bfseries Page \thecontentspage
	\begin{tikzpicture}[remember picture, overlay]
		\draw[fill=doc!60,draw=doc!60] (2pt,0) rectangle (4,0.1pt);
	\end{tikzpicture}}%
\titlecontents{section}[3.7pc]
{\addvspace{2pt}}
{\contentslabel[\thecontentslabel]{2pc}}
{}
{\hfill\small \thecontentspage}
[]
\titlecontents*{subsection}[3.7pc]
{\addvspace{-1pt}\small}
{}
{}
{\ --- \small\thecontentspage}
[ \textbullet\ ][]

\makeatletter
\renewcommand{\tableofcontents}{%
	\chapter*{%
	  \vspace*{-20\p@}%
	  \begin{tikzpicture}[remember picture, overlay]%
		  \pgftext[right,x=15cm,y=0.2cm]{\color{doc!60}\Huge\sc\bfseries \contentsname};%
		  \draw[fill=doc!60,draw=doc!60] (13,-.75) rectangle (20,1);%
		  \clip (13,-.75) rectangle (20,1);
		  \pgftext[right,x=15cm,y=0.2cm]{\color{white}\Huge\sc\bfseries \contentsname};%
	  \end{tikzpicture}}%
	\@starttoc{toc}}
\makeatother


%%%%%%%%%%%%%%%%%%%%%%%%%%%%%%%%%%%%%%%%%%%
% MINTED FOR PYTHON ALGORITHMS
%%%%%%%%%%%%%%%%%%%%%%%%%%%%%%%%%%%%%%%%%%%

\usepackage{tcolorbox}
\tcbuselibrary{minted,breakable,xparse,skins}
\definecolor{bg}{gray}{0.95}
\DeclareTCBListing{mintedbox}{O{}m!O{}}{%
  breakable=true,
  listing engine=minted,
  listing only,
  minted language=#2,
  minted style=default,
  minted options={%
    linenos,
    gobble=0,
    breaklines=true,
    breakafter=,,
    fontsize=\small,
    numbersep=8pt,
    #1},
  boxsep=0pt,
  left skip=0pt,
  right skip=0pt,
  left=25pt,
  right=0pt,
  top=3pt,
  bottom=3pt,
  arc=5pt,
  leftrule=0pt,
  rightrule=0pt,
  bottomrule=2pt,
  toprule=2pt,
  colback=bg,
  colframe=orange!70,
  enhanced,
  overlay={%
    \begin{tcbclipinterior}
    \fill[orange!20!white] (frame.south west) rectangle ([xshift=20pt]frame.north west);
    \end{tcbclipinterior}},
  #3}
  
  
 % for braces
\usetikzlibrary{decorations.pathreplacing}


\AdvanceDate[0]

\begin{document}
\pagestyle{fancy}
\fancyhead[L]{Seconde 13}
\fancyhead[C]{\textbf{Systèmes d'équations linéaires \ifsolutions \\ Solutions  \fi}}
\fancyhead[R]{\today}

\ex{
	Pour chaque paire d'équations, ajouter entre elles le symbole
		\begin{enumerate}[itemindent=1.5cm]
			\item[$\implies$] si la première équation implique la seconde ;
			\item[$\impliedby$] si la deuxième équation implique la première ;
			\item[$\iff$] si les deux équations sont équivalentes ;
			\item[$\centernot\iff$] si les deux équations sont indépendantes.
		\end{enumerate}
	Spécifier l'opération appliquée pour obtenir la deuxième équation à partir de la première le cas échéant.
	
	Par abus de notation et pour gagner du temps, l'absence de symbole signifie en général que les équations sont équivalentes, mais il faut toujours faire attention à que ça soit bien le cas !
	
	\begin{center}
	\begin{tabular}{ccc|c|c}
		Équation 1 & Symbole & Équation 2 & Opération & \thead{Les équations sont-\\ elles équivalentes ?} \\ \hline
		$3x + 2y = 1$ & \ifsol{$\iff$} & $6x  + 4y = 2$ & \ifsol{$\times2$} & \ifsol{Oui} \\ \hline
		$-x + 2y = 1$ & \ifsol{$\iff$} & $-x + 2y - 1 = 0$ & \ifsol{$-1$} & \ifsol{Oui} \\ \hline
		$3x + 2y = 1$ & \ifsol{$\centernot\iff$} & $-3x - 5y = -3$ & & \ifsol{Non} \\ \hline
		$x^2 = -3x$ & \ifsol{$\impliedby$} & $x = -3$ & & \ifsol{Non} \\ \hline
		$-10x - y = 3$ & \ifsol{$\implies$} & $0=0$ & \ifsol{$\times0$} & \ifsol{Non} \\ \hline
		$x-y= 3$ & \ifsol{$\iff$} & $-x + y = -3$ & \ifsol{$\times(-1)$} & \ifsol{Oui} \\ \hline
		$x -y = 0$ &\ifsol{$\iff$} & $x=y$ & \ifsol{$+y$} & \ifsol{Oui} \\ \hline
		$-x + 2y = 3$ & \ifsol{$\implies$} & $0=0$ & \ifsol{$\times0$} & \ifsol{Non} \\ \hline
		$2x + 3y + 4 = 2x + 3y + 5$ & \ifsol{$\iff$} & $1=0$ & \ifsol{$+(-2x-3y)$} & \ifsol{Oui} \\ \hline
		$-x - y = 1$ & \ifsol{$\centernot\iff$} & $-x - y = 1$ & & \ifsol{Non} \\ \hline
		$0=0$ & \ifsol{$\impliedby$} & $23x - 10y = 6$ & & \ifsol{Non} \\ \hline
		$x + 2y = 3$ & \ifsol{$\implies$} & $x^2 + 2yx = 3x$ & \ifsol{$\times x$} & \ifsol{Non} \\ \hline
		$x-3y  =0$ & \ifsol{$\iff$} & $-2x+6y = 0$ & \ifsol{$\times(-2)$} & \ifsol{Oui} \\ \hline
		$-2x + 3y = -3$ & \ifsol{$\centernot\iff$} & $4x - 6y = -6$ & & \ifsol{Non} \\ \hline
		$x^2 = x$ & \ifsol{$\impliedby$} & $x = 1$ & & \ifsol{Non} \\ \hline
		$x = 3$ & \ifsol{$\implies$} & $x^2 = 9$ & \ifsol{Mise au carré} & \ifsol{Non} \\ \hline
	\end{tabular}
	\end{center}
}{}

% inutile en fait, à part peut-être l'identification du coeff devant x.
%\ex{
%	Pour chaque équation d'inconnue $x\in\R$, 
%		\begin{enumerate}
%			\item donner le coefficient qui multiplie $x$ ; puis
%			\item créer une équation équivalente telle que le coefficient multipliant $x$ soit $1$.
%		\end{enumerate}
%	
%	\begin{multicols}{2}
%	\begin{enumerate}[label=\roman*), itemsep=10pt]
%		\item $2x + 4y = 6 \iff \ifsol{x + 2y = 3}$
%		\item $-x + 2y = 10 \iff \ifsol{x - 2y = -10}$
%		\item $-2x + 4y = -12 \iff \ifsol{x - 2x = 6}$
%		\item $x - y = 3 \iff \ifsol{x -y = 3}$
%		\item $3x - 7y = 0 \iff \ifsol{x - \frac73 y = 0}$
%		\item $2x - y = -3 \iff \ifsol{x - \frac12 y = -\frac32}$
%	\end{enumerate}
%	\end{multicols}
%}{}

\newpage

\ex{\label{ex:3}
	Pour chaque système d'équations d'inconnues $x, y \in \R$, donner un système équivalent tel que les coefficients multipliant $x$ soient opposés (c'est-à-dire leur somme soit nulle).
	\begin{multicols}{2}
	\begin{enumerate}[label=\roman*), itemsep=20pt]
		\item $\systeme{x + 3y = 3{,}, -2x + 2y = 2.} \iff \ifsolutions\systeme{2x+6y=6{,},-2x+2y=2.}\else\systeme{,}\fi$
		\item $\systeme{x - 7y = 1{,}, -3x + 5y = 13.} \iff \ifsolutions\systeme{3x - 21y = 3{,}, -3x + 5y = 13.}\else\systeme{,}\fi$
		\item $\systeme{2x - y = 3{,}, -x + 2y = -5.} \iff \ifsolutions\systeme{2x - y = 3{,}, -2x + 4y = -10.}\else\systeme{,}\fi$
		\item $\systeme{4x  - y = -4{,},  x + 5y = 2.} \iff  \ifsolutions\systeme{4x  - y = -4{,},  -4x - 20y = -8.}\else\systeme{,}\fi$
		\item $\systeme{-2x - 2y = -5{,}, 3x + 7y = 10.} \iff \ifsolutions\systeme{-6x - 6y = -15{,}, 6x + 14y = 20.}\else\systeme{,}\fi$
		\item $\systeme{7x - y = 10{,}, 5x + 2y = 1.} \iff \ifsolutions\systeme{35x - 5y = 50{,}, -35x - 14y = -7.}\else\systeme{,}\fi$
	\end{enumerate}
	\end{multicols}
}{}

\exe{
	Pour chaque système de l'exercice \ref{ex:3}, 
		\begin{enumerate}
			\item combiner les équations pour trouver $y$ ; 
			\item trouver $x$ ; 
			\item vérifier que le couple $(x ;y)$ obtenu soit bien solution du système initial.
		\end{enumerate}
}{ \, \\
	\begin{enumerate}[label=Système \roman*) :, itemsep=20pt]
		\item La somme des deux équations donne
			\[ 8y = 8 \iff y = 1. \]
		En remplaçant $y$ par $1$ dans la première équation, on trouve $x + 3 = 3$, et donc $x =0$.
			\[ (x ; y ) = (0; 1). \]
		
		\item La somme des deux équations donne
			\[ -16y = 16 \iff y = -1. \]
		En remplaçant $y$ par $-1$ dans la première équation, on trouve $x + 7 = 1$, et donc $x =-6$.
			\[ (x ; y ) = (-6; -1). \]
		
		\item La somme des deux équations donne
			\[ 3y = -7 \iff y = -\dfrac73. \]
		En remplaçant $y$ par $-\dfrac73$ dans la première équation, on trouve $2x +\dfrac73 = 3$, et donc $x =\dfrac13$.
			\[ (x ; y ) = \left(\dfrac13; -\dfrac73 \right). \]
		
		\item La somme des deux équations donne
			\[ -21y = -12 \iff y = \dfrac{12}{21} = \dfrac47. \]
		En remplaçant $y$ par $ \dfrac47$ dans la deuxième équation, on trouve $x +\dfrac{20}7 = 2$, et donc $x =-\dfrac67$.
			\[ (x ; y ) = \left(-\dfrac67; \dfrac47 \right). \]
		
		\item La somme des deux équations donne
			\[ 8y = 5 \iff y = \dfrac58. \]
		En remplaçant $y$ par $\dfrac58$ dans la deuxième équation, on trouve $3x +\dfrac{35}8 = 10 \iff x =\dfrac{45}{24} = \dfrac{15}8$.
			\[ (x ; y ) = \left(\dfrac{15}8; \dfrac58 \right). \]
		
		\item La somme des deux équations donne
			\[ -19y = 43 \iff y = -\dfrac{43}{19}. \]
		En remplaçant $y$ par $-\dfrac{43}{19}$ dans la première équation, on trouve $7x +\dfrac{43}{19}= 10 \iff x =\dfrac{147}{133} = \dfrac{21}{19}$.
			\[ (x ; y ) = \left( \dfrac{21}{19};  -\dfrac{43}{19} \right). \]
	\end{enumerate}
}

\exe{
	Donner les solutions réelles $(x;y)$ des systèmes suivantes.
	
	\begin{multicols}{2}
	\begin{enumerate}[label=\roman*), itemsep=20pt]
		\item $\sys{2x+4y=0}{-2x -2y = 2}$
		\item $\sys{-6x+3y=-3}{x -3y = -2}$
		\item $\sys{x+y = 2}{x+y=2}$
		\item $\sys{8x-4y = 6+2x-y}{7x+12y=-5+y}$
		\item $\sys{x-y = 1}{-x+y=10}$
		\item $\sys{\dfrac12 x - \dfrac23 y= -1}{ ,\dfrac15 x + \dfrac72 y= 5}$
		\item $\sys{2x - 8y = 2}{-4x+16y=-1}$
		\item $\sys{5x + y = -2x-1-y}{8x = -1+2y}$
		\item $\systeme[yx]{2y + 12x = -3{,}, , {-6}x=y + \dfrac32}$
	\end{enumerate}
	\end{multicols}
}{\, \\
	
	\begin{enumerate}[label=\roman*), itemsep=20pt]
		\item $(x ; y) = (-2 ; 1)$
		\item $(x ; y) = (1;1)$
		\item Les deux équations sont redondantes : on a en fait qu'une seule contrainte pour deux variables et donc un degré liberté. Pour chaque choix de $x$, on peut trouver un $y$ tel que $(x;y)$ soit solution.
		L'ensemble des solutions est
			\[ \{ (x ; y) \text{ tq. } x+y=2, x, y \in \R \} = \{ (x ; y) \text{ tq. } y = -x + 2, \text{ où $x$ parcourt $\R$} \} = \C_f, \]
		où $f(x) = -x+2$.
		\item $(x ; y) = \left(\dfrac{17}{29} ; -\dfrac{24}{29}\right)$
		\item Les deux équations sont contradictoires et aucune solution n'existe.
		\item $(x ; y) = \left(-\dfrac{10}{113} ; \dfrac{162}{113}\right)$
		\item Les deux équations sont contradictoires et aucune solution n'existe.
		\item $(x ; y) = \left(-\dfrac{2}{15} ; -\dfrac{1}{30}\right)$
		\item Les deux équations sont redondantes : on a en fait qu'une seule contrainte pour deux variables et donc un degré liberté. Pour chaque choix de $x$, on peut trouver un $y$ tel que $(x;y)$ soit solution.
		L'ensemble des solutions est
			\[ \left\{ (x ; y) \text{ tq. } -6x -y = \dfrac32, x, y \in \R \right\} = \left\{ (x ; y) \text{ tq. } y= -6x - \dfrac32 \text{ où $x$ parcourt $\R$} \right\} = \C_f, \]
		où $f(x) =-6x - \dfrac32$.
	\end{enumerate}
}

\end{document}

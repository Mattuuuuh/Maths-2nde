				% ENABLE or DISABLE font change
				% use XeLaTeX if true
\newif\ifdys
				\dystrue
				\dysfalse

\newif\ifsolutions
				\solutionstrue
				\solutionsfalse

% DYSLEXIA SWITCH
\newif\ifdys
		
				% ENABLE or DISABLE font change
				% use XeLaTeX if true
				\dystrue
				\dysfalse


\ifdys

\documentclass[a4paper, 14pt]{extarticle}
\usepackage{amsmath,amsfonts,amsthm,amssymb,mathtools}

\tracinglostchars=3 % Report an error if a font does not have a symbol.
\usepackage{fontspec}
\usepackage{unicode-math}
\defaultfontfeatures{ Ligatures=TeX,
                      Scale=MatchUppercase }

\setmainfont{OpenDyslexic}[Scale=1.0]
\setmathfont{Fira Math} % Or maybe try KPMath-Sans?
\setmathfont{OpenDyslexic Italic}[range=it/{Latin,latin}]
\setmathfont{OpenDyslexic}[range=up/{Latin,latin,num}]

\else

\documentclass[a4paper, 12pt]{extarticle}

\usepackage[utf8x]{inputenc}
%fonts
\usepackage{amsmath,amsfonts,amsthm,amssymb,mathtools}
% comment below to default to computer modern
\usepackage{libertinus,libertinust1math}

\fi


\usepackage[french]{babel}
\usepackage[
a4paper,
margin=2cm,
nomarginpar,% We don't want any margin paragraphs
]{geometry}
\usepackage{icomma}

\usepackage{fancyhdr}
\usepackage{array}
\usepackage{hyperref}

\usepackage{multicol, enumerate}
\newcolumntype{P}[1]{>{\centering\arraybackslash}p{#1}}


\usepackage{stackengine}
\newcommand\xrowht[2][0]{\addstackgap[.5\dimexpr#2\relax]{\vphantom{#1}}}

% theorems

\theoremstyle{plain}
\newtheorem{theorem}{Th\'eor\`eme}
\newtheorem*{sol}{Solution}
\theoremstyle{definition}
\newtheorem{ex}{Exercice}
\newtheorem*{rpl}{Rappel}
\newtheorem{enigme}{Énigme}

% corps
\usepackage{calrsfs}
\newcommand{\C}{\mathcal{C}}
\newcommand{\R}{\mathbb{R}}
\newcommand{\Rnn}{\mathbb{R}^{2n}}
\newcommand{\Z}{\mathbb{Z}}
\newcommand{\N}{\mathbb{N}}
\newcommand{\Q}{\mathbb{Q}}

% variance
\newcommand{\Var}[1]{\text{Var}(#1)}

% domain
\newcommand{\D}{\mathcal{D}}


% date
\usepackage{advdate}
\AdvanceDate[0]


% plots
\usepackage{pgfplots}

% table line break
\usepackage{makecell}
%tablestuff
\def\arraystretch{2}
\setlength\tabcolsep{15pt}

%subfigures
\usepackage{subcaption}

\definecolor{myg}{RGB}{56, 140, 70}
\definecolor{myb}{RGB}{45, 111, 177}
\definecolor{myr}{RGB}{199, 68, 64}

% fake sections with no title to move around the merged pdf
\newcommand{\fakesection}[1]{%
  \par\refstepcounter{section}% Increase section counter
  \sectionmark{#1}% Add section mark (header)
  \addcontentsline{toc}{section}{\protect\numberline{\thesection}#1}% Add section to ToC
  % Add more content here, if needed.
}


% SOLUTION SWITCH
\newif\ifsolutions
				\solutionstrue
				%\solutionsfalse

\ifsolutions
	\newcommand{\exe}[2]{
		\begin{ex} #1  \end{ex}
		\begin{sol} #2 \end{sol}
	}
\else
	\newcommand{\exe}[2]{
		\begin{ex} #1  \end{ex}
	}
	
\fi


% tableaux var, signe
\usepackage{tkz-tab}


%pinfty minfty
\newcommand{\pinfty}{{+}\infty}
\newcommand{\minfty}{{-}\infty}

\begin{document}


\AdvanceDate[0]

\begin{document}
\pagestyle{fancy}
\fancyhead[L]{Seconde 13}
\fancyhead[C]{\textbf{Échantillonnage : loi de grands nombres \ifsolutions \, -- Solutions  \fi}}
\fancyhead[R]{\today}

% Si $p$ est la probabilité d'une issue et $f$ sa fréquence observée dans un échantillon, calculer la proportion des cas où l'écart entre $p$ et $f$ est inférieur ou égal à $\frac1{\sqrt{n}}$.

\exe{
	On lance un grand nombre de fois un D6 équilibré, dé à 6 faces, et on note la face du dessus.
	Tous les lancers sont \underline{\smash{identiques}} et \underline{\smash{indépendants}} les uns des autres.
	Au bout de 50 lancers, on écrit le nombre de fois que la face 6 a été obtenue depuis le début en séparant, dans le tableau ci-dessous, les résultat en groupes de 10 lancers.
	
	\begin{tabular}{|c|c|c|c|}\hline
		$N$ & 10 & 10 & 10  \\ \hline
		Nombre de 6 obtenus &&& \\ \hline
		Fréquence $f$ de 6 &&& \\ \hline
		Distance entre $f$ et $p$ &&& \\ \hline
	\end{tabular}
	
	\begin{enumerate}
		\item Donner $p$, la probabilité d'obtenir 6 après un seul lancer.
		\item Pour chaque groupe de 10 lancers, calculer la fréquence $f$ du nombre de 6 obtenus.
		\item Pour chaque groupe de 10 lancers, calculer la distance entre $p$ et $f$ à $10^{-3}$ près.
		\item Quelle est la proportion de groupes de 10 lancers pour lesquels la distance est inférieure à $0,1$ ? Et $0,05$ ? 
	\end{enumerate}
}{}

\exe{
	À quel intervalle correspondent les ensembles suivants ?
		\begin{align*}
			E_1 &= \bigl\{ x \in \R \text{ tq. } \bigl|x\bigr| \leq 20 \bigr\} \\
			E_2 &= \bigl\{ x \in \R \text{ tq. } \bigl|x - 4\bigr| \leq 10 \bigr\} \\
			E_3 &= \left\{ x \in \R \text{ tq. } \left|x - \dfrac16\right| \leq 2\sqrt{3} \right\}
		\end{align*}
}{}

\exe{

}{}

\exe{	
	On lance un grand nombre de fois un D$6$, dé à $6$ faces.
	On suppose que tous les lancers sont \underline{\smash{identiques}} et \underline{\smash{indépendants}} les uns des autres.
	Les résultats sont décrits dans le tableau suivant.
	\begin{center}
	\begin{tabular}{|c|c|c|c|c|c|c|} \hline
		Face & 1 & 2 & 3 & 4 & 5 & 6 \\ \hline
		Nombre de tirages & 19780 & 29557 & 9952 & 14769 & 19903 & 4955 \\ \hline
		Fréquence & \ifsolutions $\color{red} 0,2$ \fi & \ifsolutions $\color{red} 0,3$ \fi & \ifsolutions $\color{red} 0,1$ \fi & \ifsolutions $\color{red} 0,15$ \fi & \ifsolutions $\color{red} 0,2$ \fi & \ifsolutions $\color{red} 0,05$ \fi \\ \hline
	\end{tabular}
	\end{center}
	
	\begin{enumerate}
		\item Calculer $N$, le nombre de lancers.
		\item Donner un intervalle de confiance à $95\%$ 
	\end{enumerate}
}{}


\end{document}

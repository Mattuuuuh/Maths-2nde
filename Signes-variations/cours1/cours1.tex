				% ENABLE or DISABLE font change
				% use XeLaTeX if true
\newif\ifdys
				\dystrue
				\dysfalse

\newif\ifsolutions
				\solutionstrue
				\solutionsfalse

% DYSLEXIA SWITCH
\newif\ifdys
		
				% ENABLE or DISABLE font change
				% use XeLaTeX if true
				\dystrue
				\dysfalse


\ifdys

\documentclass[a4paper, 14pt]{extarticle}
\usepackage{amsmath,amsfonts,amsthm,amssymb,mathtools}

\tracinglostchars=3 % Report an error if a font does not have a symbol.
\usepackage{fontspec}
\usepackage{unicode-math}
\defaultfontfeatures{ Ligatures=TeX,
                      Scale=MatchUppercase }

\setmainfont{OpenDyslexic}[Scale=1.0]
\setmathfont{Fira Math} % Or maybe try KPMath-Sans?
\setmathfont{OpenDyslexic Italic}[range=it/{Latin,latin}]
\setmathfont{OpenDyslexic}[range=up/{Latin,latin,num}]

\else

\documentclass[a4paper, 12pt]{extarticle}

\usepackage[utf8x]{inputenc}
%fonts
\usepackage{amsmath,amsfonts,amsthm,amssymb,mathtools}
% comment below to default to computer modern
\usepackage{libertinus,libertinust1math}

\fi


\usepackage[french]{babel}
\usepackage[
a4paper,
margin=2cm,
nomarginpar,% We don't want any margin paragraphs
]{geometry}
\usepackage{icomma}

\usepackage{fancyhdr}
\usepackage{array}
\usepackage{hyperref}

\usepackage{multicol, enumerate}
\newcolumntype{P}[1]{>{\centering\arraybackslash}p{#1}}


\usepackage{stackengine}
\newcommand\xrowht[2][0]{\addstackgap[.5\dimexpr#2\relax]{\vphantom{#1}}}

% theorems

\theoremstyle{plain}
\newtheorem{theorem}{Th\'eor\`eme}
\newtheorem*{sol}{Solution}
\theoremstyle{definition}
\newtheorem{ex}{Exercice}
\newtheorem*{rpl}{Rappel}
\newtheorem{enigme}{Énigme}

% corps
\usepackage{calrsfs}
\newcommand{\C}{\mathcal{C}}
\newcommand{\R}{\mathbb{R}}
\newcommand{\Rnn}{\mathbb{R}^{2n}}
\newcommand{\Z}{\mathbb{Z}}
\newcommand{\N}{\mathbb{N}}
\newcommand{\Q}{\mathbb{Q}}

% variance
\newcommand{\Var}[1]{\text{Var}(#1)}

% domain
\newcommand{\D}{\mathcal{D}}


% date
\usepackage{advdate}
\AdvanceDate[0]


% plots
\usepackage{pgfplots}

% table line break
\usepackage{makecell}
%tablestuff
\def\arraystretch{2}
\setlength\tabcolsep{15pt}

%subfigures
\usepackage{subcaption}

\definecolor{myg}{RGB}{56, 140, 70}
\definecolor{myb}{RGB}{45, 111, 177}
\definecolor{myr}{RGB}{199, 68, 64}

% fake sections with no title to move around the merged pdf
\newcommand{\fakesection}[1]{%
  \par\refstepcounter{section}% Increase section counter
  \sectionmark{#1}% Add section mark (header)
  \addcontentsline{toc}{section}{\protect\numberline{\thesection}#1}% Add section to ToC
  % Add more content here, if needed.
}


% SOLUTION SWITCH
\newif\ifsolutions
				\solutionstrue
				%\solutionsfalse

\ifsolutions
	\newcommand{\exe}[2]{
		\begin{ex} #1  \end{ex}
		\begin{sol} #2 \end{sol}
	}
\else
	\newcommand{\exe}[2]{
		\begin{ex} #1  \end{ex}
	}
	
\fi


% tableaux var, signe
\usepackage{tkz-tab}


%pinfty minfty
\newcommand{\pinfty}{{+}\infty}
\newcommand{\minfty}{{-}\infty}

\begin{document}


\AdvanceDate[0]

\begin{document}
\pagestyle{fancy}
\fancyhead[L]{Seconde 13}
\fancyhead[C]{\textbf{Cours --- Variations de fonctions parentes \ifsolutions -- Solutions  \fi}}
\fancyhead[R]{\today}

\ex{
	On donne la courbe $\C_f$ graphiquement ci-dessous.
	
	\begin{enumerate}
		\item Dessiner, dans le même repère, la fonction $g(x) = f(x) + 1$.
		\item Dessiner, dans le même repère, la fonction $h(x) = f(x) -2$.
	\end{enumerate}
	
	\begin{center}
	\begin{tikzpicture}[scale=1]
		\begin{axis}[xmin = -10, xmax=10, ymin=-4.25, ymax=4.25, axis x line=middle, axis y line=middle, axis line style=->, grid=both,
		%ytick={-4,-3,...,2,3}, xtick={-11, -10,...,-4,-3},
	    	%every y tick label/.style={
	        %anchor=near yticklabel opposite,
	        %xshift=0.2em,
	    	%}
	    	]
		% g cos
		\addplot[no marks, myb, -, very thick] expression[domain=-10:10, samples=50]{.01*(x+6)*(x+3)*(x-5)}
		node[pos=.5, below]{$\mathcal{C}_f$};
		\ifsolutions
		\addplot[no marks, myr, -, very thick] expression[domain=-10:10, samples=50]{.01*(x+6)*(x+3)*(x-5)+1}
		node[pos=.2, above]{$\mathcal{C}_g$};
		\addplot[no marks, myg, -, very thick] expression[domain=-10:10, samples=50]{.01*(x+6)*(x+3)*(x-5)-2}
		node[pos=.5,below]{$\mathcal{C}_h$};
		\fi
		\end{axis}
	\end{tikzpicture}
	\end{center}
}

\begin{propriete}{ajout d'une constante}{}
	Soit $c\in\R$ un nombre réel et $g(x) = f(x)+c$.
	\begin{itemize}
		\item Pour obtenir $\C_g$, on translate $\C_f$ \ifsolutions verticalement de $c$ unités. \fi
		\item Les variations de $g$ et de $f$ sont \ifsolutions identiques. \fi
	\end{itemize}
\end{propriete}


\ex{
	On donne la courbe $\C_f$ graphiquement ci-dessous.
	
	\begin{enumerate}
		\item Dessiner, dans le même repère, la fonction $g(x) = f(x+2)$.
		\item Dessiner, dans le même repère, la fonction $h(x) = f(x-5)$.
	\end{enumerate}
	
	\begin{center}
	\begin{tikzpicture}[scale=1]
		\begin{axis}[xmin = -10, xmax=10, ymin=-4.25, ymax=4.25, axis x line=middle, axis y line=middle, axis line style=->, grid=both,
		%ytick={-4,-3,...,2,3}, xtick={-11, -10,...,-4,-3},
	    	%every y tick label/.style={
	        %anchor=near yticklabel opposite,
	        %xshift=0.2em,
	    	%}
	    	]
		\addplot[no marks, myb, -, very thick] expression[domain=-10:10, samples=50]{.01*(x+6)*(x+3)*(x-5)}
		node[pos=.25, above]{$\mathcal{C}_f$};
		\ifsolutions
		\addplot[no marks, myr, -, very thick] expression[domain=-10:10, samples=50]{.01*(x+8)*(x+5)*(x-3)}
		node[pos=.1, above]{$\mathcal{C}_g$};
		\addplot[no marks, myg, -, very thick] expression[domain=-10:10, samples=50]{.01*(x+1)*(x-2)*(x-10)}
		node[pos=.95, below]{$\mathcal{C}_h$};
		\fi
		\end{axis}
	\end{tikzpicture}
	\end{center}
	
}{}

\begin{propriete}{translation d'une constante}{}
	Soit $c\in\R$ un nombre réel et $g(x) = f(x+c)$.
	\begin{itemize}
		\item Pour obtenir $\C_g$, on translate $\C_f$ \ifsolutions horizontalement de $-c$ unités. \fi
		\item Les variations de $g$ et de $f$ sont \ifsolutions identiques mais les intervalles décalés de $-c$ unités. \fi
	\end{itemize}
\end{propriete}

\newpage

\ex{
	On donne la courbe $\C_f$ graphiquement ci-dessous.
	
	\begin{enumerate}
		\item Dessiner, dans le même repère, la fonction $g(x) = 2f(x)$.
		\item Dessiner, dans le même repère, la fonction $h(x) = -f(x)$.
	\end{enumerate}
	
	\begin{center}
	\begin{tikzpicture}[scale=1]
		\begin{axis}[xmin = -10, xmax=10, ymin=-4.25, ymax=4.25, axis x line=middle, axis y line=middle, axis line style=->, grid=both,
		%ytick={-4,-3,...,2,3}, xtick={-11, -10,...,-4,-3},
	    	%every y tick label/.style={
	        %anchor=near yticklabel opposite,
	        %xshift=0.2em,
	    	%}
	    	]
		\addplot[no marks, myb, -, very thick] expression[domain=-10:10, samples=50]{.01*(x+6)*(x+3)*(x-5)}
		node[pos=.45, above]{$\mathcal{C}_f$};
		\ifsolutions
		\addplot[no marks, myr, -, very thick] expression[domain=-10:10, samples=50]{.02*(x+6)*(x+3)*(x-5)}
		node[pos=.4, below]{$\mathcal{C}_g$};
		\addplot[no marks, myg, -, very thick] expression[domain=-10:10, samples=50]{-.01*(x+6)*(x+3)*(x-5)}
		node[pos=.45, above]{$\mathcal{C}_h$};
		\fi
		\end{axis}
	\end{tikzpicture}
	\end{center}
}{}

\begin{propriete}{multiplication par une constante}{}
	Soit $c\in\R$ un nombre réel et $g(x) = c\cdot f(x)$.
	\begin{itemize}
		\item Si $c>0$, alors les variations de $g$ et de $f$ sont \ifsolutions identiques. \fi
		\item Si $c<0$, alors les variations de $g$ et de $f$ sont \ifsolutions opposées (croissante devient décroissante ; décroissante devient croissante ; et constante reste idem). \fi
	\end{itemize}
\end{propriete}

\end{document}

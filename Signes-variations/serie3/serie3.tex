				% ENABLE or DISABLE font change
				% use XeLaTeX if true
\newif\ifdys
				\dystrue
				\dysfalse

\newif\ifsolutions
				\solutionstrue
				\solutionsfalse

% DYSLEXIA SWITCH
\newif\ifdys
		
				% ENABLE or DISABLE font change
				% use XeLaTeX if true
				\dystrue
				\dysfalse


\ifdys

\documentclass[a4paper, 14pt]{extarticle}
\usepackage{amsmath,amsfonts,amsthm,amssymb,mathtools}

\tracinglostchars=3 % Report an error if a font does not have a symbol.
\usepackage{fontspec}
\usepackage{unicode-math}
\defaultfontfeatures{ Ligatures=TeX,
                      Scale=MatchUppercase }

\setmainfont{OpenDyslexic}[Scale=1.0]
\setmathfont{Fira Math} % Or maybe try KPMath-Sans?
\setmathfont{OpenDyslexic Italic}[range=it/{Latin,latin}]
\setmathfont{OpenDyslexic}[range=up/{Latin,latin,num}]

\else

\documentclass[a4paper, 12pt]{extarticle}

\usepackage[utf8x]{inputenc}
%fonts
\usepackage{amsmath,amsfonts,amsthm,amssymb,mathtools}
% comment below to default to computer modern
\usepackage{libertinus,libertinust1math}

\fi


\usepackage[french]{babel}
\usepackage[
a4paper,
margin=2cm,
nomarginpar,% We don't want any margin paragraphs
]{geometry}
\usepackage{icomma}

\usepackage{fancyhdr}
\usepackage{array}
\usepackage{hyperref}

\usepackage{multicol, enumerate}
\newcolumntype{P}[1]{>{\centering\arraybackslash}p{#1}}


\usepackage{stackengine}
\newcommand\xrowht[2][0]{\addstackgap[.5\dimexpr#2\relax]{\vphantom{#1}}}

% theorems

\theoremstyle{plain}
\newtheorem{theorem}{Th\'eor\`eme}
\newtheorem*{sol}{Solution}
\theoremstyle{definition}
\newtheorem{ex}{Exercice}
\newtheorem*{rpl}{Rappel}
\newtheorem{enigme}{Énigme}

% corps
\usepackage{calrsfs}
\newcommand{\C}{\mathcal{C}}
\newcommand{\R}{\mathbb{R}}
\newcommand{\Rnn}{\mathbb{R}^{2n}}
\newcommand{\Z}{\mathbb{Z}}
\newcommand{\N}{\mathbb{N}}
\newcommand{\Q}{\mathbb{Q}}

% variance
\newcommand{\Var}[1]{\text{Var}(#1)}

% domain
\newcommand{\D}{\mathcal{D}}


% date
\usepackage{advdate}
\AdvanceDate[0]


% plots
\usepackage{pgfplots}

% table line break
\usepackage{makecell}
%tablestuff
\def\arraystretch{2}
\setlength\tabcolsep{15pt}

%subfigures
\usepackage{subcaption}

\definecolor{myg}{RGB}{56, 140, 70}
\definecolor{myb}{RGB}{45, 111, 177}
\definecolor{myr}{RGB}{199, 68, 64}

% fake sections with no title to move around the merged pdf
\newcommand{\fakesection}[1]{%
  \par\refstepcounter{section}% Increase section counter
  \sectionmark{#1}% Add section mark (header)
  \addcontentsline{toc}{section}{\protect\numberline{\thesection}#1}% Add section to ToC
  % Add more content here, if needed.
}


% SOLUTION SWITCH
\newif\ifsolutions
				\solutionstrue
				%\solutionsfalse

\ifsolutions
	\newcommand{\exe}[2]{
		\begin{ex} #1  \end{ex}
		\begin{sol} #2 \end{sol}
	}
\else
	\newcommand{\exe}[2]{
		\begin{ex} #1  \end{ex}
	}
	
\fi


% tableaux var, signe
\usepackage{tkz-tab}


%pinfty minfty
\newcommand{\pinfty}{{+}\infty}
\newcommand{\minfty}{{-}\infty}

\begin{document}


\AdvanceDate[0]

\begin{document}
\pagestyle{fancy}
\fancyhead[L]{Seconde 13}
\fancyhead[C]{\textbf{Signes et variations 3 \ifsolutions --- Solutions  \fi}}
\fancyhead[R]{\today}

\exe{\label{ex:1}
	On considère la fonction carré $f(x) = x^2$.
	\begin{enumerate}
		\item Donner $\D_f$, le domaine de définition de $f$.
		\item Esquisser $\C_f$ sur $\D = [-5 ; 5]$.
		\item Compléter le tableau de variations ci-dessous.
	\end{enumerate}
	
	\begin{center}
	\begin{tikzpicture}
		\tkzTabInit
		 [lgt=3]
	       		{$x$ / 1 , Variations de $x^2$ / 2}
	       		{$\minfty$,,\ifsolutions 0 \fi,,$\pinfty$}
	       		
		\ifsolutions
		\tkzTabVar
			{+/25, R/, -/0, R/, +/25}
		\fi
	\end{tikzpicture}
	\end{center}
}{

	 Aucune opération n'est illégale lorsqu'on calcule l'image d'un $x\in\R$ réel : ni division par zéro, ni racine carrée de nombre éventuellement négatif.
	Aucun $x$ n'est donc interdit, et $\D_f = \R$.
		
	\begin{center}
	\begin{tikzpicture}
	\begin{axis}[ymin=0, xmin=-5, xmax=5]
		\addplot[very thick, myb, domain=-5:5, samples=50] {x^2} node[pos=.5, above] {$\C_f$};
	\end{axis}
	\end{tikzpicture}
	\end{center}
}

\ex{
	À l'aide de l'exercice \ref{ex:1} et du cours, compléter les tableaux de variations des fonctions parentes à $x^2$.
	\begin{center}
	\begin{tikzpicture}
		\tkzTabInit
		 [lgt=3]
	       		{$x$ / 1 , Variation de $-2x^2$/ 2, Variation de $3x^2$ / 2, Variation de $1+x^2$ / 2, Variation de $2x^2-7$ / 2, Variation de $-3 - 5x^2$ / 2, Variation de $-x^2+17$ / 2}
	       		{\ifsolutions $\minfty$ \fi,,\ifsolutions 0 \fi,, \ifsolutions $\pinfty$ \fi}
	      
		\ifsolutions
		\tkzTabVar
			{-/, R/, +/0, R/, -/}
		\tkzTabVar
			{+/, R/, -/0, R/, +/}
		\tkzTabVar
			{+/, R/, -/1, R/, +/}
		\tkzTabVar
			{+/, R/, -/{-7}, R/, +/}
		\tkzTabVar
			{-/, R/, +/{-3}, R/, -/}
		\tkzTabVar
			{-/, R/, +/{17}, R/, -/}
		\fi
	\end{tikzpicture}
	\end{center}
}{}

\exe{
	Soit $f$ la fonction définie par
		\[ f(x) = -3 - (x+1)^2. \]
	\begin{enumerate}
		\item Donner $\D_f$.
		\item Montrer que $f$ atteint son maximum en $x^\star=-1$ et donner sa valeur.
	\end{enumerate}
	
}{
	\begin{enumerate}
		\item 
		Aucune opération n'est illégale lorsqu'on calcule l'image d'un $x\in\R$ réel : ni division par zéro, ni racine carrée de nombre éventuellement négatif.
		Aucun $x$ n'est donc interdit, et $\D_f = \R$.
		\item 
		Lorsqu'on a affaire à une forme canonique, on part systématiquement du fait qu'un carré est toujours positif pour enfin construire $f(x)$.
		Pour tout $x\in\R$ réel, on a donc
			\begin{align*}
				(x+1)^2 &\geq 0 \\
				-(x+1)^2 &\leq 0 \\
				-3 - (x+1)^2 &\leq -3 \\
				f(x) &\leq -3
			\end{align*}
		On en déduit que $f(x)$ est borné supérieurement par $-3$ pour tous les $x\in\R$ réels.
		
		Pour montrer que c'est un maximum atteint en $-1$, on calcule $f(-1) = -3 - (-1+1)^2 = -3 + 0^2 = -3$.
		En conclusion,
			\[ f(x) \leq f(-1)=-3, \]
		et ce pour tous les $x\in\R$. Par définition, $-3$ est le maximum de $f$, atteint en $-1$.
	\end{enumerate}
}

\exe{
	Soit $f$ la fonction définie sur $\R$ par
		\[ f(x) = -10 + 3(3x-1)^2. \]
	\begin{enumerate}
		\item Donner $\D_f$.
		\item Montrer que $f$ atteint son minimum en $x^\star=\dfrac13$ et donner sa valeur.
	\end{enumerate}
}{



		Lorsqu'on a affaire à une forme canonique, on part systématiquement du fait qu'un carré est toujours positif pour construire $f(x)$.
		Pour tout $x\in\R$ réel, on a donc
			\begin{align*}
				(3x-1)^2 &\geq 0 \\
				3(3x-1)^2 &\geq 0 \\
				-10 + 3(3x-1)^2 &\geq -10 \\
				f(x) &\geq -10
			\end{align*}
		On en déduit que $f(x)$ est borné inférieurement par $-10$ pour tous les $x\in\R$ réels.
		
		Pour montrer que c'est un minimum atteint en $\frac13$, on calcule $f(\frac13) = -10$.
		En conclusion,
			\[ f(x) \geq f\left(\dfrac13\right)=-10, \]
		et ce pour tous les $x\in\R$. Par définition, $-10$ est le minimum de $f$, atteint en $\frac13$.
}

\ifsolutions
\else
\newpage
\fi

\exe{[Vrai ou faux]
	Pour chaque proposition suivante, démontrer qu'elle est \underline{toujours vraie} ou montrer qu'elle \underline{peut être fausse} avec un contre-exemple.
	$f, g, h, F$ et $G$ sont des fonctions définies sur $\R$.
	\begin{enumerate}
		\item Si $f$ est affine et admet deux racines distinctes, alors $f(x) = 0$ pour tout $x\in\R$.
		\item Si $g(x) \geq 0$ pour tout $x\in\R$, alors $g$ n'admet aucune racine.
		\item Si $h(x) > 0$ pour tout $x\in\R$, alors $h$ n'admet aucune racine.
		\item Si $F$ n'admet aucune racine, alors $F(x) > 0$ pour tout $x\in\R$.
		\item $G(x) = (x-3)^2 (x-5)^2$ est toujours positive est admet exactement deux racines distinctes.
	\end{enumerate}
}{

	\begin{enumerate}
		\item 
		C'est vrai. 
		Si $f$ est affine et admet deux racines distinctes, alors sa courbe représentative est une droite qui intersecte deux fois l'axe des abscisses en deux endroits différents.
		$\C_f$ et l'axe des abscisses sont donc confondues et on a bien $f(x) = 0$ pour tout $x\in\R$.
		
		Algébriquement, on a $(x_A ; 0), (x_B ; 0) \in \C_f$ pour $x_A \neq x_B$.
		La formule du coefficient directeur donne $a = 0$, et une appartenance donne $b=0$.
		
		\item 
		C'est faux : être positif ou nul n'empêche pas d'être nul.
		On pourra prend la fonction carré $f(x) = x^2$ comme contre-exemple : $f(x) \geq 0$ pour tout $x\in\R$ et pourtant $f$ admet $0$ pour racine car $f(0) = 0$.
		
		\item 
		C'est vrai : être strictement positif empêche d'être nul. 
		C'est bien la différence entre la positivité stricte ($>$) et large ($\geq$).
		
		\item 
		C'est faux : si $F$ ne s'annule jamais, elle n'est pas nécessairement toujours strictement positive car elle pourrait aussi être toujours strictement négative !
		Prenons par exemple $f(x) = -1 -x^2$ qui vérifie $f(x) \leq -1 < 0$ pour tout $x\in\R$ réel.
		
		\item 
		C'est vrai.
		Un carré est toujours positif, donc $G(x)$ est le produit de deux positifs et est positif.
		En outre, si $G(x) = 0$, alors $(x-3)^2 = 0$ ou $(x-5)^2 = 0$.
		Il suit que $x=3$ ou $x=5$, et ce sont les deux seules racines de $G$.
		
	\end{enumerate}

}

\ex{
	On considère le tableau des variations des fonctions $f, g,$ et $h$ sur le domaine $\D = [7 ; 31]$.
	Pour chacune d'entre elles,
		\begin{enumerate}[label=\roman*)]
			\item donner son maximum sur $\D$ s'il est possible de le connaître.
			Sinon, tracer deux courbes fidèles au tableau et de maxima différents.
			\item donner son minimum sur $\D$ s'il est possible de le connaître.
			Sinon, tracer deux courbes fidèles au tableau et de minima différents.
		\end{enumerate}
	%Peut-on déduire le maximum et le minimum de chaque fonction sur $\D$ ?
	
	\begin{center}
	\begin{tikzpicture}
		\tkzTabInit
		 %[lgt=3,espcl=1.5]
	       		{$x$ / 1 , Variation de $f(x)$ / 2, Variation de $g(x)$ / 2, Variation de $h(x)$ / 2}
	       		{7,11,17,28,31}

		\tkzTabVar
			{+/,-/,+/,-/,+/}
		\tkzTabIma
			{1}{2}{1}{45}
		\tkzTabIma
			{1}{2}{2}{0}
		\tkzTabIma
			{3}{4}{3}{60}
		\tkzTabIma
			{3}{4}{4}{-10}
		\tkzTabIma
			{4}{5}{5}{30}
			
		\tkzTabVar
			{+/, +/,R/,-/,+/}
		\tkzTabIma
			{1}{2}{1}{60}
		\tkzTabIma
			{2}{4}{4}{25}
			
		\tkzTabVar
			{-/,+/,+/,R/,-/}
		\tkzTabIma
			{1}{2}{1}{10}
		\tkzTabIma
			{1}{2}{2}{40}
	\end{tikzpicture}
	\end{center}
}{}


\ex{[Vrai ou faux]
	Pour chaque proposition suivante, dire si elle est \underline{toujours vraie} ou si elle \underline{peut être fausse} à l'aide du tableau de variations suivant.
	
\def\arraystretch{1.5}
\setlength\tabcolsep{6pt}
	\begin{center}
	\begin{tikzpicture}
		\tkzTabInit
		 [lgt=3,espcl=1.5]
	       		{$x$ / 1 , Variation de $f(x)$ / 2}
	       		{7,$a$,11,$b$,$c$,28,$d$,31}
			
		\tkzTabVar
			{-/, +/,-/$-45$,R/,R/,+/$20$, R/, -/$-30$}
		\tkzTabIma
			{1}{2}{1}{$30$}
%		\tkzTabIma
%			{3}{6}{3}{25}
%		\tkzTabIma
%			{3}{6}{6}{70}
			
	\end{tikzpicture}
	\vfill
	\begin{tabular}{|c|c|c|} \hline
		Proposition & Vrai & Faux \\ \hline
		$f(a) = 35$ & & \tick \\ \hline
		$f(a) > 30$ & \tick & \\ \hline
		$f(b) \leq 20$ & \tick & \\ \hline
		$f(a) > f(b)$ & \tick & \\ \hline
	\end{tabular}	
	\hfill
	\begin{tabular}{|c|c|c|} \hline
		Proposition & Vrai & Faux \\ \hline
		$f(b) < f(c)$ & \tick & \\ \hline
		$f(c) < f(a)$ & \tick & \\ \hline
		$f(c) = 19$ & & \tick \\ \hline
		$f(c) < 30$ & \tick & \\ \hline
	\end{tabular}
	\hfill
	\begin{tabular}{|c|c|c|} \hline
		Proposition & Vrai & Faux \\ \hline
		$f(d) = -5$ & & \tick \\ \hline
		$f(d) < 0$ & & \tick \\ \hline
		$f(c) = f(d)$ & & \tick \\ \hline
		$f(d) \geq -30$ & \tick & \\ \hline
	\end{tabular}	
	
	\end{center}




}{}


\end{document}

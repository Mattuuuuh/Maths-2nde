				% ENABLE or DISABLE font change
				% use XeLaTeX if true
\newif\ifdys
				\dystrue
				\dysfalse

\newif\ifsolutions
				\solutionstrue
				\solutionsfalse

% DYSLEXIA SWITCH
\newif\ifdys
		
				% ENABLE or DISABLE font change
				% use XeLaTeX if true
				\dystrue
				\dysfalse


\ifdys

\documentclass[a4paper, 14pt]{extarticle}
\usepackage{amsmath,amsfonts,amsthm,amssymb,mathtools}

\tracinglostchars=3 % Report an error if a font does not have a symbol.
\usepackage{fontspec}
\usepackage{unicode-math}
\defaultfontfeatures{ Ligatures=TeX,
                      Scale=MatchUppercase }

\setmainfont{OpenDyslexic}[Scale=1.0]
\setmathfont{Fira Math} % Or maybe try KPMath-Sans?
\setmathfont{OpenDyslexic Italic}[range=it/{Latin,latin}]
\setmathfont{OpenDyslexic}[range=up/{Latin,latin,num}]

\else

\documentclass[a4paper, 12pt]{extarticle}

\usepackage[utf8x]{inputenc}
%fonts
\usepackage{amsmath,amsfonts,amsthm,amssymb,mathtools}
% comment below to default to computer modern
\usepackage{libertinus,libertinust1math}

\fi


\usepackage[french]{babel}
\usepackage[
a4paper,
margin=2cm,
nomarginpar,% We don't want any margin paragraphs
]{geometry}
\usepackage{icomma}

\usepackage{fancyhdr}
\usepackage{array}
\usepackage{hyperref}

\usepackage{multicol, enumerate}
\newcolumntype{P}[1]{>{\centering\arraybackslash}p{#1}}


\usepackage{stackengine}
\newcommand\xrowht[2][0]{\addstackgap[.5\dimexpr#2\relax]{\vphantom{#1}}}

% theorems

\theoremstyle{plain}
\newtheorem{theorem}{Th\'eor\`eme}
\newtheorem*{sol}{Solution}
\theoremstyle{definition}
\newtheorem{ex}{Exercice}
\newtheorem*{rpl}{Rappel}
\newtheorem{enigme}{Énigme}

% corps
\usepackage{calrsfs}
\newcommand{\C}{\mathcal{C}}
\newcommand{\R}{\mathbb{R}}
\newcommand{\Rnn}{\mathbb{R}^{2n}}
\newcommand{\Z}{\mathbb{Z}}
\newcommand{\N}{\mathbb{N}}
\newcommand{\Q}{\mathbb{Q}}

% variance
\newcommand{\Var}[1]{\text{Var}(#1)}

% domain
\newcommand{\D}{\mathcal{D}}


% date
\usepackage{advdate}
\AdvanceDate[0]


% plots
\usepackage{pgfplots}

% table line break
\usepackage{makecell}
%tablestuff
\def\arraystretch{2}
\setlength\tabcolsep{15pt}

%subfigures
\usepackage{subcaption}

\definecolor{myg}{RGB}{56, 140, 70}
\definecolor{myb}{RGB}{45, 111, 177}
\definecolor{myr}{RGB}{199, 68, 64}

% fake sections with no title to move around the merged pdf
\newcommand{\fakesection}[1]{%
  \par\refstepcounter{section}% Increase section counter
  \sectionmark{#1}% Add section mark (header)
  \addcontentsline{toc}{section}{\protect\numberline{\thesection}#1}% Add section to ToC
  % Add more content here, if needed.
}


% SOLUTION SWITCH
\newif\ifsolutions
				\solutionstrue
				%\solutionsfalse

\ifsolutions
	\newcommand{\exe}[2]{
		\begin{ex} #1  \end{ex}
		\begin{sol} #2 \end{sol}
	}
\else
	\newcommand{\exe}[2]{
		\begin{ex} #1  \end{ex}
	}
	
\fi


% tableaux var, signe
\usepackage{tkz-tab}


%pinfty minfty
\newcommand{\pinfty}{{+}\infty}
\newcommand{\minfty}{{-}\infty}

\begin{document}


\AdvanceDate[0]

\begin{document}
\pagestyle{fancy}
\fancyhead[L]{Seconde 13}
\fancyhead[C]{\textbf{Signes et variations 1 \ifsolutions -- Solutions  \fi}}
\fancyhead[R]{\today}


\exe{\label{ex:1}
	Pour chacune des fonctions affines $f$ suivantes, donner l'intervalle 
		\[ \{ x \in \R \text{ tq. } f(x) \geq 0 \}. \]
	
	\begin{multicols}{2}
	\begin{enumerate}
		\item $f(x) = x+3$
		\item $f(x) = 3x-10$
		\item $f(x) = x$
		\item $f(x) = -4x-12$
		\item $f(x) = -10x$
		\item $f(x) = 1$
	\end{enumerate}
	\end{multicols}
}{}

\exe{
	À l'aide de l'exercice \ref{ex:1}, donner le domaine de définition de chaque fonction suivante.
	
	\begin{enumerate}
		\item $f(x) = \sqrt{x+3}$
		\item $g(x) = \sqrt{3x-10}$
		\item $h(x) = \sqrt{-4x-12}$
	\end{enumerate}

}{}


\exe{
	\begin{multicols}{2}
	Exprimer par lecture graphique les ensembles suivants sous forme d'union d'intervalles.
		\begin{enumerate}
			\item $\{ x \in [-3 ; 3] \text{ tq. } f(x) \geq 0\}$
			\item $\{ x \in [-3 ; 3] \text{ tq. } g(x) \leq 0\}$
			\item $\{ x \in [-3 ; 3] \text{ tq. } f(x) \cdot g(x) \geq 0\}$
		\end{enumerate}
	\vfill
	
	\begin{center}
	\begin{tikzpicture}[>=stealth, scale=1]
	\begin{axis}[xmin = -3, xmax=3, ymin=-4, ymax=4, axis x line=middle, axis y line=middle, axis line style=<->, xlabel={}, ylabel={}, xtick = {-4, -3, ..., 4}, ytick = {-4, -3, ..., 4}, grid=both]
		
		% (g)
		\addplot[myr, very thick, domain =-3:3, samples=50] {(x+1)*x*(x-2)+.5}  node[pos = .77, right=2pt] {$\C_g$};
		
		% (f)
		\addplot[myb, very thick, domain =-3:3, samples=50] {2.9-1.2*x^2}  node[pos = .41, above=4pt] {$\C_f$};
	\end{axis}
	\end{tikzpicture}
	\end{center}
	\end{multicols}
}{}

\exe{\label{ex:4}
	À l'aide de l'exercice \ref{ex:1} remplir les tableaux de signes ci-dessous et donner
		\[ \{ x \in \R \text{ tq. } f(x) \geq 0 \} \]
	sous forme d'union d'intervalles.
	
	\begin{center}
	\begin{tikzpicture}
		\tkzTabInit
		 %[lgt=3,espcl=1.5]
		 [lgt=5]
	       		{$x$ / 1 , $x+3$ / 1, $3x-10$ / 1 , $f(x) = (x+3)(3x-10)$ / 1}
	       		{$\minfty$,,,$\pinfty$}
	\end{tikzpicture}
	
	\begin{tikzpicture}
		\tkzTabInit
		 %[lgt=3,espcl=1.5]
		 [lgt=5]
	       		{$x$ / 1 , $x$ / 1, $-4x-12$ / 1 , $f(x) = x(-4x-12)$ / 1}
	       		{,,,}
	\end{tikzpicture}
	
	
	\begin{tikzpicture}
		\tkzTabInit
		 %[lgt=3,espcl=1.5]
		 [lgt=5]
	       		{$x$ / 1 , $x+3$ / 1, $-10x$ / 1 , $3x-10$ / 1 , $f(x) = -10x(x+3)(3x-10)$ / 1}
	       		{,,,,}
	\end{tikzpicture}
	\end{center}
}{}


\exe{
	À l'aide de l'exercice \ref{ex:4}, donner le domaine de définition de la fonction
		\[ f(x) = \sqrt{-10x(x+3)(3x-10)}. \]
}{}

\exe{
	On souhaite connaître pour quels $x\in\D_f$ l'inégalité suivante est vérifée.
		\[ f(x) = \dfrac{3-x}{6x^2 - 5x - 4} \geq 0 \]
	
	\begin{enumerate}
		\item Montrer que $6x^2 - 5x - 4 = (3x-4)(2x+1)$.
		\item En déduire les valeurs interdites et donc $\D_f$.
		\item Remplir le tableau de signes ci-dessous.
		\item Exprimer $\{ x \in \D_f \text{ tq. } f(x) \geq 0 \}$ sous forme d'union d'intervalles.
	\end{enumerate}
	
	
	\begin{center}
	\begin{tikzpicture}
		\tkzTabInit
		 %[lgt=3,espcl=1.5]
	       		{$x$ / 1, $3-x$ / 1 , $3x-4$ / 1, $2x+1$ / 1 , $f(x)$ / 1}
	       		{,,,,,}
	\end{tikzpicture}
	\end{center}
}{}




\end{document}

				% ENABLE or DISABLE font change
				% use XeLaTeX if true
\newif\ifdys
				\dystrue
				\dysfalse

\newif\ifsolutions
				\solutionstrue
				\solutionsfalse

%!TEX encoding = UTF8
%!TEX root =notes.tex


%%%%%%%%%%%%%%%%%%%%%%%%%%%%%%%%%
% PACKAGE IMPORTS
%%%%%%%%%%%%%%%%%%%%%%%%%%%%%%%%%


\usepackage[french]{babel}

\usepackage[tmargin=2cm,rmargin=1in,lmargin=1in,margin=0.85in,bmargin=2cm,footskip=.2in]{geometry}
\usepackage{amsmath,amsfonts,amsthm,amssymb,mathtools}
\usepackage[varbb]{newpxmath}
\usepackage{xfrac}
\usepackage[makeroom]{cancel}
\usepackage{mathtools}
\usepackage{bookmark}
\usepackage{enumitem}
\usepackage{hyperref,theoremref}
\hypersetup{
	pdftitle={Assignment},
	colorlinks=true, linkcolor=doc!90,
	bookmarksnumbered=true,
	bookmarksopen=true
}
\usepackage[most,many,breakable]{tcolorbox}
\usepackage{xcolor}
\usepackage{varwidth}
\usepackage{varwidth}
\usepackage{etoolbox}
%\usepackage{authblk}
\usepackage{nameref}
\usepackage{multicol,array}
\usepackage{tikz-cd}
\usepackage[ruled,vlined,linesnumbered]{algorithm2e}
\usepackage{comment} % enables the use of multi-line comments (\ifx \fi) 
\usepackage{import}
\usepackage{xifthen}
\usepackage{pdfpages}
\usepackage{transparent}


\newcommand\mycommfont[1]{\footnotesize\ttfamily\textcolor{blue}{#1}}
\SetCommentSty{mycommfont}
\newcommand{\incfig}[1]{%
    \def\svgwidth{\columnwidth}
    \import{./figures/}{#1.pdf_tex}
}

\usepackage{tikzsymbols}
%\renewcommand\qedsymbol{$\Laughey$}


%\usepackage{import}
%\usepackage{xifthen}
%\usepackage{pdfpages}
%\usepackage{transparent}


%%%%%%%%%%%%%%%%%%%%%%%%%%%%%%
% SELF MADE COLORS
%%%%%%%%%%%%%%%%%%%%%%%%%%%%%%



\definecolor{myg}{RGB}{56, 140, 70}
\definecolor{myb}{RGB}{45, 111, 177}
\definecolor{myr}{RGB}{199, 68, 64}
\definecolor{mytheorembg}{HTML}{F2F2F9}
\definecolor{mytheoremfr}{HTML}{00007B}
\definecolor{mylenmabg}{HTML}{FFFAF8}
\definecolor{mylenmafr}{HTML}{983b0f}
\definecolor{mypropbg}{HTML}{f2fbfc}
\definecolor{mypropfr}{HTML}{191971}
\definecolor{myexamplebg}{HTML}{F2FBF8}
\definecolor{myexamplefr}{HTML}{88D6D1}
\definecolor{myexampleti}{HTML}{2A7F7F}
\definecolor{mydefinitbg}{HTML}{E5E5FF}
\definecolor{mydefinitfr}{HTML}{3F3FA3}
\definecolor{notesgreen}{RGB}{0,162,0}
\definecolor{myp}{RGB}{197, 92, 212}
\definecolor{mygr}{HTML}{2C3338}
\definecolor{myred}{RGB}{127,0,0}
\definecolor{myyellow}{RGB}{169,121,69}
\definecolor{myexercisebg}{HTML}{F2FBF8}
\definecolor{myexercisefg}{HTML}{88D6D1}


%%%%%%%%%%%%%%%%%%%%%%%%%%%%
% TCOLORBOX SETUPS
%%%%%%%%%%%%%%%%%%%%%%%%%%%%

\setlength{\parindent}{1cm}
%================================
% THEOREM BOX
%================================

\tcbuselibrary{theorems,skins,hooks}
\newtcbtheorem[number within=chapter]{Theorem}{Théorème}
{%
	enhanced,
	breakable,
	colback = mytheorembg,
	frame hidden,
	boxrule = 0sp,
	borderline west = {2pt}{0pt}{mytheoremfr},
	sharp corners,
	detach title,
	before upper = \tcbtitle\par\smallskip,
	coltitle = mytheoremfr,
	fonttitle = \bfseries\sffamily,
	description font = \mdseries,
	separator sign none,
	segmentation style={solid, mytheoremfr},
}
{th}


\tcbuselibrary{theorems,skins,hooks}
\newtcolorbox{Theoremcon}
{%
	enhanced
	,breakable
	,colback = mytheorembg
	,frame hidden
	,boxrule = 0sp
	,borderline west = {2pt}{0pt}{mytheoremfr}
	,sharp corners
	,description font = \mdseries
	,separator sign none
}

%================================
% Corollery
%================================
\tcbuselibrary{theorems,skins,hooks}
\newtcbtheorem[use counter=tcb@cnt@Theorem]{Corollary}{Corollaire}
{%
	enhanced
	,breakable
	,colback = myp!10
	,frame hidden
	,boxrule = 0sp
	,borderline west = {2pt}{0pt}{myp!85!black}
	,sharp corners
	,detach title
	,before upper = \tcbtitle\par\smallskip
	,coltitle = myp!85!black
	,fonttitle = \bfseries\sffamily
	,description font = \mdseries
	,separator sign none
	,segmentation style={solid, myp!85!black}
}
{th}

%================================
% LENMA
%================================

\tcbuselibrary{theorems,skins,hooks}
\newtcbtheorem[use counter=tcb@cnt@Theorem]{Lemma}{Lemme}
{%
	enhanced,
	breakable,
	colback = mylenmabg,
	frame hidden,
	boxrule = 0sp,
	borderline west = {2pt}{0pt}{mylenmafr},
	sharp corners,
	detach title,
	before upper = \tcbtitle\par\smallskip,
	coltitle = mylenmafr,
	fonttitle = \bfseries\sffamily,
	description font = \mdseries,
	separator sign none,
	segmentation style={solid, mylenmafr},
}
{th}


%================================
% PROPOSITION
%================================

\tcbuselibrary{theorems,skins,hooks}
\newtcbtheorem[use counter=tcb@cnt@Theorem]{Prop}{Proposition}
{%
	enhanced,
	breakable,
	colback = mypropbg,
	frame hidden,
	boxrule = 0sp,
	borderline west = {2pt}{0pt}{mypropfr},
	sharp corners,
	detach title,
	before upper = \tcbtitle\par\smallskip,
	coltitle = mypropfr,
	fonttitle = \bfseries\sffamily,
	description font = \mdseries,
	separator sign none,
	segmentation style={solid, mypropfr},
}
{th}


%================================
% CLAIM
%================================

\tcbuselibrary{theorems,skins,hooks}
\newtcbtheorem[use counter=tcb@cnt@Theorem]{claim}{Claim}
{%
	enhanced
	,breakable
	,colback = myg!10
	,frame hidden
	,boxrule = 0sp
	,borderline west = {2pt}{0pt}{myg}
	,sharp corners
	,detach title
	,before upper = \tcbtitle\par\smallskip
	,coltitle = myg!85!black
	,fonttitle = \bfseries\sffamily
	,description font = \mdseries
	,separator sign none
	,segmentation style={solid, myg!85!black}
}
{th}



%================================
% Exercise
%================================

\tcbuselibrary{theorems,skins,hooks}
\newtcbtheorem[use counter=tcb@cnt@Theorem]{Exercise}{Exercice}
{%
	enhanced,
	breakable,
	colback = myexercisebg,
	frame hidden,
	boxrule = 0sp,
	borderline west = {2pt}{0pt}{myexercisefg},
	sharp corners,
	detach title,
	before upper = \tcbtitle\par\smallskip,
	coltitle = myexercisefg,
	fonttitle = \bfseries\sffamily,
	description font = \mdseries,
	separator sign none,
	segmentation style={solid, myexercisefg},
}
{th}

%================================
% EXAMPLE BOX
%================================

\newtcbtheorem[use counter=tcb@cnt@Theorem]{Example}{Exemple}
{%
	colback = myexamplebg
	,breakable
	,colframe = myexamplefr
	,coltitle = myexampleti
	,boxrule = 1pt
	,sharp corners
	,detach title
	,before upper=\tcbtitle\par\smallskip
	,fonttitle = \bfseries
	,description font = \mdseries
	,separator sign none
	,description delimiters parenthesis
}
{ex}

%================================
% DEFINITION BOX
%================================

\newtcbtheorem[use counter=tcb@cnt@Theorem]{Definition}{Définition}{enhanced,
	before skip=2mm,after skip=2mm, colback=red!5,colframe=red!80!black,boxrule=0.5mm,
	attach boxed title to top left={xshift=1cm,yshift*=1mm-\tcboxedtitleheight}, varwidth boxed title*=-3cm,
	boxed title style={frame code={
					\path[fill=tcbcolback]
					([yshift=-1mm,xshift=-1mm]frame.north west)
					arc[start angle=0,end angle=180,radius=1mm]
					([yshift=-1mm,xshift=1mm]frame.north east)
					arc[start angle=180,end angle=0,radius=1mm];
					\path[left color=tcbcolback!60!black,right color=tcbcolback!60!black,
						middle color=tcbcolback!80!black]
					([xshift=-2mm]frame.north west) -- ([xshift=2mm]frame.north east)
					[rounded corners=1mm]-- ([xshift=1mm,yshift=-1mm]frame.north east)
					-- (frame.south east) -- (frame.south west)
					-- ([xshift=-1mm,yshift=-1mm]frame.north west)
					[sharp corners]-- cycle;
				},interior engine=empty,
		},
	fonttitle=\bfseries,
	title={#2},#1}{def}

%================================
% Solution BOX
%================================

\makeatletter
\newtcbtheorem[use counter=tcb@cnt@Theorem]{question}{Question}{enhanced,
	breakable,
	colback=white,
	colframe=myb!80!black,
	attach boxed title to top left={yshift*=-\tcboxedtitleheight},
	fonttitle=\bfseries,
	title={#2},
	boxed title size=title,
	boxed title style={%
			sharp corners,
			rounded corners=northwest,
			colback=tcbcolframe,
			boxrule=0pt,
		},
	underlay boxed title={%
			\path[fill=tcbcolframe] (title.south west)--(title.south east)
			to[out=0, in=180] ([xshift=5mm]title.east)--
			(title.center-|frame.east)
			[rounded corners=\kvtcb@arc] |-
			(frame.north) -| cycle;
		},
	#1
}{def}
\makeatother

%================================
% SOLUTION BOX
%================================

\makeatletter
\newtcolorbox{solution}{enhanced,
	breakable,
	colback=white,
	colframe=myg!80!black,
	attach boxed title to top left={yshift*=-\tcboxedtitleheight},
	title=Solution,
	boxed title size=title,
	boxed title style={%
			sharp corners,
			rounded corners=northwest,
			colback=tcbcolframe,
			boxrule=0pt,
		},
	underlay boxed title={%
			\path[fill=tcbcolframe] (title.south west)--(title.south east)
			to[out=0, in=180] ([xshift=5mm]title.east)--
			(title.center-|frame.east)
			[rounded corners=\kvtcb@arc] |-
			(frame.north) -| cycle;
		},
}
\makeatother

%================================
% Question BOX
%================================

\makeatletter
\newtcbtheorem[use counter=tcb@cnt@Theorem]{qstion}{Question}{enhanced,
	breakable,
	colback=white,
	colframe=mygr,
	attach boxed title to top left={yshift*=-\tcboxedtitleheight},
	fonttitle=\bfseries,
	title={#2},
	boxed title size=title,
	boxed title style={%
			sharp corners,
			rounded corners=northwest,
			colback=tcbcolframe,
			boxrule=0pt,
		},
	underlay boxed title={%
			\path[fill=tcbcolframe] (title.south west)--(title.south east)
			to[out=0, in=180] ([xshift=5mm]title.east)--
			(title.center-|frame.east)
			[rounded corners=\kvtcb@arc] |-
			(frame.north) -| cycle;
		},
	#1
}{def}
\makeatother

\newtcbtheorem[number within=chapter]{wconc}{Wrong Concept}{
	breakable,
	enhanced,
	colback=white,
	colframe=myr,
	arc=0pt,
	outer arc=0pt,
	fonttitle=\bfseries\sffamily\large,
	colbacktitle=myr,
	attach boxed title to top left={},
	boxed title style={
			enhanced,
			skin=enhancedfirst jigsaw,
			arc=3pt,
			bottom=0pt,
			interior style={fill=myr}
		},
	#1
}{def}



%================================
% NOTE BOX
%================================

\usetikzlibrary{arrows,calc,shadows.blur}
\tcbuselibrary{skins}
\newtcolorbox{note}[1][]{%
	enhanced jigsaw,
	colback=gray!20!white,%
	colframe=gray!80!black,
	size=small,
	boxrule=1pt,
	title=\colorbox{white!100}{\textbf{ Remarque }},
	halign title=flush center,
	coltitle=black,
	breakable,
	drop shadow=black!50!white,
	attach boxed title to top left={xshift=1cm,yshift=-\tcboxedtitleheight/2,yshifttext=-\tcboxedtitleheight/2},
	minipage boxed title=2.6cm,
	boxed title style={%
			colback=white,
			size=fbox,
			boxrule=1pt,
			boxsep=2pt,
			underlay={%
					\coordinate (dotA) at ($(interior.west) + (-0.5pt,0)$);
					\coordinate (dotB) at ($(interior.east) + (0.5pt,0)$);
					\begin{scope}
						\clip (interior.north west) rectangle ([xshift=3ex]interior.east);
						\filldraw [white, blur shadow={shadow opacity=60, shadow yshift=-.75ex}, rounded corners=2pt] (interior.north west) rectangle (interior.south east);
					\end{scope}
					\begin{scope}[gray!80!black]
						\fill (dotA) circle (2pt);
						\fill (dotB) circle (2pt);
					\end{scope}
				},
		},
	#1,
}

%================================
% STRATÉGIE BOX
%================================

\usetikzlibrary{arrows,calc,shadows.blur}
\tcbuselibrary{skins}
\newtcolorbox{strategy}[1][]{%
	enhanced jigsaw,
	colback=myb!20!white,%
	colframe=gray!80!black,
	size=small,
	boxrule=1pt,
	title=\colorbox{white!100}{\textbf{ Stratégie }},
	halign title=flush center,
	coltitle=black,
	breakable,
	drop shadow=black!50!white,
	attach boxed title to top left={xshift=1cm,yshift=-\tcboxedtitleheight/2,yshifttext=-\tcboxedtitleheight/2},
	minipage boxed title=2.5cm,
	boxed title style={%
			colback=white,
			size=fbox,
			boxrule=1pt,
			boxsep=2pt,
			underlay={%
					\coordinate (dotA) at ($(interior.west) + (-0.5pt,0)$);
					\coordinate (dotB) at ($(interior.east) + (0.5pt,0)$);
					\begin{scope}
						\clip (interior.north west) rectangle ([xshift=3ex]interior.east);
						\filldraw [white, blur shadow={shadow opacity=60, shadow yshift=-.75ex}, rounded corners=2pt] (interior.north west) rectangle (interior.south east);
					\end{scope}
					\begin{scope}[gray!80!black]
						\fill (dotA) circle (2pt);
						\fill (dotB) circle (2pt);
					\end{scope}
				},
		},
	#1,
}

%================================
% MÉTHODE BOX
%================================

\usetikzlibrary{arrows,calc,shadows.blur}
\tcbuselibrary{skins}
\newtcolorbox{methode}[1][]{%
	enhanced jigsaw,
	colback=white,%
	colframe=gray!80!black,
	size=small,
	boxrule=1pt,
	title=\textbf{Méthode},
	halign title=flush center,
	coltitle=black,
	breakable,
	drop shadow=black!50!white,
	attach boxed title to top left={xshift=1cm,yshift=-\tcboxedtitleheight/2,yshifttext=-\tcboxedtitleheight/2},
	minipage boxed title=2.5cm,
	boxed title style={%
			colback=white,
			size=fbox,
			boxrule=1pt,
			boxsep=2pt,
			underlay={%
					\coordinate (dotA) at ($(interior.west) + (-0.5pt,0)$);
					\coordinate (dotB) at ($(interior.east) + (0.5pt,0)$);
					\begin{scope}
						\clip (interior.north west) rectangle ([xshift=3ex]interior.east);
						\filldraw [white, blur shadow={shadow opacity=60, shadow yshift=-.75ex}, rounded corners=2pt] (interior.north west) rectangle (interior.south east);
					\end{scope}
					\begin{scope}[gray!80!black]
						\fill (dotA) circle (2pt);
						\fill (dotB) circle (2pt);
					\end{scope}
				},
		},
	#1,
}

%%%%%%%%%%%%%%%%%%%%%%%%%%%%%%%%%%%%%%%%%%%
% TABLE OF CONTENTS
%%%%%%%%%%%%%%%%%%%%%%%%%%%%%%%%%%%%%%%%%%%

\usepackage{tikz}

\definecolor{doc}{RGB}{0,60,110}
\usepackage{titletoc}
\contentsmargin{0cm}
\titlecontents{chapter}[3.7pc]
{\addvspace{30pt}%
	\begin{tikzpicture}[remember picture, overlay]%
		\draw[fill=doc!60,draw=doc!60] (-7,-.1) rectangle (-0.2,.6);%
		\pgftext[left,x=-3.5cm,y=0.2cm]{\color{white}\Large\sc\bfseries Chapitre\ \thecontentslabel};%
	\end{tikzpicture}\color{doc!60}\large\sc\bfseries}%
{}
{}
{\;\titlerule\;\large\sc\bfseries Page \thecontentspage
	\begin{tikzpicture}[remember picture, overlay]
		\draw[fill=doc!60,draw=doc!60] (2pt,0) rectangle (4,0.1pt);
	\end{tikzpicture}}%
\titlecontents{section}[3.7pc]
{\addvspace{2pt}}
{\contentslabel[\thecontentslabel]{2pc}}
{}
{\hfill\small \thecontentspage}
[]
\titlecontents*{subsection}[3.7pc]
{\addvspace{-1pt}\small}
{}
{}
{\ --- \small\thecontentspage}
[ \textbullet\ ][]

\makeatletter
\renewcommand{\tableofcontents}{%
	\chapter*{%
	  \vspace*{-20\p@}%
	  \begin{tikzpicture}[remember picture, overlay]%
		  \pgftext[right,x=15cm,y=0.2cm]{\color{doc!60}\Huge\sc\bfseries \contentsname};%
		  \draw[fill=doc!60,draw=doc!60] (13,-.75) rectangle (20,1);%
		  \clip (13,-.75) rectangle (20,1);
		  \pgftext[right,x=15cm,y=0.2cm]{\color{white}\Huge\sc\bfseries \contentsname};%
	  \end{tikzpicture}}%
	\@starttoc{toc}}
\makeatother


%%%%%%%%%%%%%%%%%%%%%%%%%%%%%%%%%%%%%%%%%%%
% MINTED FOR PYTHON ALGORITHMS
%%%%%%%%%%%%%%%%%%%%%%%%%%%%%%%%%%%%%%%%%%%

\usepackage{tcolorbox}
\tcbuselibrary{minted,breakable,xparse,skins}
\definecolor{bg}{gray}{0.95}
\DeclareTCBListing{mintedbox}{O{}m!O{}}{%
  breakable=true,
  listing engine=minted,
  listing only,
  minted language=#2,
  minted style=default,
  minted options={%
    linenos,
    gobble=0,
    breaklines=true,
    breakafter=,,
    fontsize=\small,
    numbersep=8pt,
    #1},
  boxsep=0pt,
  left skip=0pt,
  right skip=0pt,
  left=25pt,
  right=0pt,
  top=3pt,
  bottom=3pt,
  arc=5pt,
  leftrule=0pt,
  rightrule=0pt,
  bottomrule=2pt,
  toprule=2pt,
  colback=bg,
  colframe=orange!70,
  enhanced,
  overlay={%
    \begin{tcbclipinterior}
    \fill[orange!20!white] (frame.south west) rectangle ([xshift=20pt]frame.north west);
    \end{tcbclipinterior}},
  #3}
  
  
 % for braces
\usetikzlibrary{decorations.pathreplacing}


\AdvanceDate[0]

\begin{document}
\pagestyle{fancy}
\fancyhead[L]{Seconde 13}
\fancyhead[C]{\textbf{Signes et variations 1 \ifsolutions -- Solutions  \fi}}
\fancyhead[R]{\today}


\exe{\label{ex:1}
	Pour chacune des fonctions affines $f$ suivantes, donner l'intervalle 
		\[ \{ x \in \R \text{ tq. } f(x) \geq 0 \}. \]
	
	\begin{multicols}{2}
	\begin{enumerate}
		\item $f(x) = x+3$
		\item $f(x) = 3x-10$
		\item $f(x) = x$
		\item $f(x) = -4x-12$
		\item $f(x) = -10x$
		\item $f(x) = 1$
	\end{enumerate}
	\end{multicols}
}{
	\begin{enumerate}
		\item 
		L'inégalité $x+3 \geq 0$ est équivalente à $x \geq -3$.
		On obtient donc 
			\[ \{ x \in \R \text{ tq. } x+ 3 \geq 0 \} = \{ x \in \R \text{ tq. }x \geq -3 \} = [-3 ; \pinfty[. \]
		\item 
			\[ \left[ \dfrac{10}3 ; \pinfty \right[ \]
		\item
			\[ [0 ; \pinfty[ \]
		\item
		L'inégalité $-4x - 12 \geq 0$ est équivalente à $-4x \geq 12$.
		Pour continuer, on multiplie par $-\dfrac14$ qui est négatif, ce qui change donc le sens de l'inégalité : $x \leq -3$.
		On obtient donc
			\[ ]\minfty ; -3]. \]
		\item
			\[ ]\minfty ; 0] \]
		\item 
			Tous les $x\in\R$ vérifie que $f(x) \geq 0$ car $1 \geq 0$ est toujours vrai.
			Donc 
			\[ \{ x \in \R \text{ tq. } 1 \geq 0 \} = \R. \]
	\end{enumerate}

}

\ifdys
\exe{
	Exprimer par lecture graphique les ensembles suivants sous forme d'union d'intervalles.
		\begin{enumerate}
			\item $\{ x \in [-3 ; 3] \text{ tq. } f(x) \geq 0\}$
			\item $\{ x \in [-3 ; 3] \text{ tq. } g(x) \leq 0\}$
			\item $\{ x \in [-3 ; 3] \text{ tq. } f(x) \cdot g(x) \geq 0\}$
		\end{enumerate}
	
	\begin{center}
	\begin{tikzpicture}[>=stealth, scale=1]
	\begin{axis}[xmin = -3, xmax=3, ymin=-4, ymax=4, axis x line=middle, axis y line=middle, axis line style=<->, xlabel={}, ylabel={}, xtick = {-4, -3, ..., 4}, ytick = {-4, -3, ..., 4}, grid=both]
		
		% (g)
		\addplot[myr, very thick, domain =-3:3, samples=50] {(x+1)*x*(x-2)+.5}  node[pos = .77, right=2pt] {$\C_g$};
		
		% (f)
		\addplot[myb, very thick, domain =-3:3, samples=50] {2.9-1.2*x^2}  node[pos = .41, above=4pt] {$\C_f$};
	\end{axis}
	\end{tikzpicture}
	\end{center}
}{}
\else
\exe{
	\begin{multicols}{2}
	Exprimer par lecture graphique les ensembles suivants sous forme d'union d'intervalles.
		\begin{enumerate}
			\item $\{ x \in [-3 ; 3] \text{ tq. } f(x) \geq 0\}$
			\item $\{ x \in [-3 ; 3] \text{ tq. } g(x) \leq 0\}$
			\item $\{ x \in [-3 ; 3] \text{ tq. } f(x) \cdot g(x) \geq 0\}$
		\end{enumerate}
	\vfill
	
	\begin{center}
	\begin{tikzpicture}[>=stealth, scale=1]
	\begin{axis}[xmin = -3, xmax=3, ymin=-4, ymax=4, axis x line=middle, axis y line=middle, axis line style=<->, xlabel={}, ylabel={}, xtick = {-4, -3, ..., 4}, ytick = {-4, -3, ..., 4}, grid=both]
		
		% (g)
		\addplot[myr, very thick, domain =-3:3, samples=50] {(x+1)*x*(x-2)+.5}  node[pos = .77, right=2pt] {$\C_g$};
		
		% (f)
		\addplot[myb, very thick, domain =-3:3, samples=50] {2.9-1.2*x^2}  node[pos = .41, above=4pt] {$\C_f$};
	\end{axis}
	\end{tikzpicture}
	\end{center}
	\end{multicols}
}{
	L'ordonnée du point d'abscisse $x$ appartenant à $\C_f$ est $f(x)$.
	On lit donc son signe en regardant s'il est au-dessus ou en-dessous de l'axe des abscisses.
	
	\begin{enumerate}
		\item On a environ 
			\[ \{ x \in [-3 ; 3] \text{ tq. } f(x) \geq 0\} \approx [-1,5 ; 1,5]. \]
		\item On a environ 
			\[ \{ x \in [-3 ; 3] \text{ tq. } g(x) \leq 0\} \approx [-3 ; -1,2] \cup [1,9 ; 3]. \]
		On est en droit de se demander : comment sait-on que $g(-2) \leq 0$ si on ne peut pas lire sa valeur graphiquement ?
		On suppose en fait ici que $\C_g$ est une courbe continue (tracée sans lever le crayon). 
		Ainsi, si $g(-2)$ était positif, la courbe devrait nécessairement couper l'axe des abscisses, ce qui n'est visiblement pas le cas.
		\item Il y a deux situations dans lesquelles le produit $f(x) \cdot g(x)$ peut être positif : soit $f(x)$ et $g(x)$ sont positifs, soit ils sont tous deux négatifs.
		Graphiquement, on lit que
			\[ \{ x \in [-3 ; 3] \text{ tq. } f(x) \geq 0 \text{ et } g(x) \geq 0 \} \approx [-1,2 ; 0,2], \]
		et que (en faisant à nouveau une hypothèse de continuité)
			\[ \{ x \in [-3 ; 3] \text{ tq. } f(x) \leq 0 \text{ et } g(x) \leq 0 \} \approx [-3 ; -1,6] \cup [1,6 ; 1,9]. \]
		On prend l'union des deux pour obtenir
			\[ \{ x \in [-3 ; 3] \text{ tq. } f(x) \cdot g(x)  \geq 0 \} \approx [-1,2 ; 0,2] \cup [-3 ; -1,6] \cup [1,6 ; 1,9]. \]
	\end{enumerate}
}
\fi

\exe{
	À l'aide de l'exercice \ref{ex:1}, donner le domaine de définition de chaque fonction suivante.
	
	\begin{enumerate}
		\item $f(x) = \sqrt{x+3}$
		\item $g(x) = \sqrt{3x-10}$
		\item $h(x) = \sqrt{-4x-12}$
	\end{enumerate}

}{
	Le domaine de définition est le plus grand domaine sur lequel une fonction est bien définie.
	Ici, la seule chose qui peut se mal passer est qu'on demande à $f$ de calculer une racine carrée d'un nombre négatif.
	Ceci n'est pas possible car $x^2 \geq 0$ pour tout $x\in\R$, donc si on souhaite calculer $\sqrt{a}$ vérifiant $\sqrt{a}^2 = a$ par définition, on a nécessairement $a\geq0$.
	Par exemple, $\sqrt{-1}$ ne peut pas être un nombre réel, sinon $\sqrt{-1}^2 = -1$, et un carré serait négatif !
	
	On impose donc que l'expression sous une racine carrée soit positive ou nulle, et on prend tous les réels vérifiant cette propriété.
	\begin{enumerate}
		\item
			\[ \D_f = \{ x \in \R \text{ tq. } x + 3 \geq 0 \} = [-3; \pinfty[, \]
		d'après l'exercice \ref{ex:1}.
		\item
			\[ \D_f = \{ x \in \R \text{ tq. } 3x- 10 \geq 0 \} = \left[ \dfrac{10}3 ; \pinfty \right[, \]
		d'après l'exercice \ref{ex:1}.
		\item
			\[ \D_f = \{ x \in \R \text{ tq. } -4x-12 \geq 0 \} = ]\minfty; -3], \]
		d'après l'exercice \ref{ex:1}.
	\end{enumerate}
}

\exe{\label{ex:4}
	À l'aide de l'exercice \ref{ex:1} remplir les tableaux de signes ci-dessous et donner
		\begin{align*}
			\{ x \in \R \text{ tq. } f(x) \geq 0 \}, && \{ x \in \R \text{ tq. } g(x) \leq 0 \}, && \{ x \in \R \text{ tq. } h(x) \geq 0 \}
		\end{align*}
	sous forme d'union d'intervalles.
	
	\begin{center}
	\begin{tikzpicture}
		\tkzTabInit
		 %[lgt=3,espcl=1.5]
		 [lgt=5]
	       		{$x$ / 1 , $x+3$ / 1, $3x-10$ / 1 , $f(x) = (x+3)(3x-10)$ / \ifdys 2 \else1\fi}
	       		{$\minfty$,\ifsolutions $-3$ \fi,\ifsolutions $\dfrac{10}3$ \fi,$\pinfty$}
		
		\ifsolutions
		\tkzTabLine
			{,-,z,+,,+}
		\tkzTabLine
			{,-,,-,z,+}
		\tkzTabLine
			{,+, z, -, z, +}
		\fi
	\end{tikzpicture}
	
	\begin{tikzpicture}
		\tkzTabInit
		 %[lgt=3,espcl=1.5]
		 [lgt=5]
	       		{$x$ / 1 , $x$ / 1, $-4x-12$ / 1 , $g(x) = x(-4x-12)$ / \ifdys 2 \else1\fi}
	       		{\ifsolutions $\minfty$ \fi,\ifsolutions $-3$ \fi,\ifsolutions $0$ \fi,\ifsolutions $\pinfty$ \fi}
	       		
		\ifsolutions
		\tkzTabLine
			{,-,z,+,,+}
		\tkzTabLine
			{,+,,+,z,-}
		\tkzTabLine
			{,-, z, +, z, -}
		\fi
	\end{tikzpicture}
	
	
	\begin{tikzpicture}
		\tkzTabInit
		 %[lgt=3,espcl=1.5]
		 [lgt=5]
	       		{$x$ / 1 , $x+3$ / 1, $-10x$ / 1 , $3x-10$ / 1 , $h(x) = -10x(x+3)(3x-10)$ / \ifdys 2 \else1\fi}
	       		{\ifsolutions $\minfty$ \fi,\ifsolutions $-3$ \fi, \ifsolutions $0$ \fi, \ifsolutions $\dfrac{10}3$ \fi,\ifsolutions $\pinfty$ \fi}
	       		
		\ifsolutions
		\tkzTabLine
			{,-,z,+,,+,,+}
		\tkzTabLine
			{,+,,+,z,-,,-}
		\tkzTabLine
			{,-,,-,, -,z,+}
		\tkzTabLine
			{,+,z,-,z, +,z,-}
		\fi
	\end{tikzpicture}
	\end{center}
}{
	On remplit le tableau de signe en calculant d'abord quand chaque fonction affine s'annule : c'est exactement quand elle change de signe (du positif vers le négatif, ou l'inverse).
	
	Par exemple, $x+3$ s'annule en $x=-3$, et l'exercice \ref{ex:1} donne que $x+3$ est positif après $-3$, est donc nécessairement négatif avant.
	On fait idem pour $3x-10$, qui s'annule en $\dfrac{10}3$, et qui est positif après, et négatif avant.
	
	Le signe du produit $f(x) = (x+3)(3x-10)$ est déduit du signe des facteurs, à l'aide des règles ``positif $\times$ positif = positif ; négatif $\times$ négatif = positif ; positif $\times$ négatif = négatif''.
	De puis, lorsqu'un des deux facteurs des nul, le produit est forcément nul, car zéro multiplié par n'importe quoi est nul.
	
	Le deuxième tableau est similaire, en faisant attention au fait que le signe de $-4x-12$ est positif puis négatif (cf. exercice \ref{ex:1}).
	Ceci est en fait dû au fait qu'on ait multiplié par $-\dfrac14$ lors de la résolution de $-4x-12 \geq 0$.
	Le signe du coefficient directeur $a$ de la fonction affine donne donc l'allure de la droite associée : si $a>0$, la droite monte du négatif vers le positif en passant par $0$, et si $a <0$, la droite descend du positif vers le négatif en passant par $0$.
	
	Le cas $a=0$ est le cas de la fonction constante, dont l'étude de signe n'est pas très intéressante.
}


\exe{
	À l'aide de l'exercice \ref{ex:4}, donner le domaine de définition de la fonction
		\[ f(x) = \sqrt{-10x(x+3)(3x-10)}. \]
}{
	La seul contrainte à poser sur $x$ est que l'expression sous la racine soit positive ou nulle.
		\[ \D_f = \{ x \in \R \text{ tq. } -10x(x+3)(3x-10) \geq 0 \}. \]
	Or la dernière ligne du dernier tableau de signes de l'exercice \ref{ex:4} nous donne exactement les intervalles où $h(x) = -10x(x+3)(3x-10)$ est positive ou nulle :
		\[ \D_f = ]\minfty ; -3] \cup \left[0 ; \dfrac{10}3 \right]. \]
}

\exe{
	On souhaite connaître pour quels $x\in\D_f$ l'inégalité suivante est vérifée.
		\[ f(x) = \dfrac{2x+1}{7x^2 - 20x - 3} \geq 0 \]
	
	\begin{enumerate}
		\item Montrer que $7x^2 - 20x - 3 = (x-3)(7x+1)$.
		\item En déduire les valeurs interdites à $f$ et donc le domaine de définition $\D_f$.
		\item Remplir le tableau de signes ci-dessous.
		\item Exprimer $\{ x \in \D_f \text{ tq. } f(x) \geq 0 \}$ sous forme d'union d'intervalles.
	\end{enumerate}
	
	
	\begin{center}
	\begin{tikzpicture}
		\tkzTabInit
		 %[lgt=3,espcl=1.5]
	       		{$x$ / 1, $x-3$ / 1 , $7x+1$ / 1, $2x+1$ / 1 , $f(x)$ / 1}
	       		{\ifsolutions $\minfty$ \fi,\ifsolutions $-\dfrac12$ \fi, \ifsolutions $-\dfrac17$ \fi, \ifsolutions $3$ \fi,\ifsolutions $\pinfty$ \fi}
	       		
		\ifsolutions
		\tkzTabLine
			{,-,,-,,-,z,+}
		\tkzTabLine
			{,-,,-,z,+,,+}
		\tkzTabLine
			{,-,z,+,,+,,+}
		\tkzTabLine
			{,-,z,+,d, -,d,+}
		\fi
	\end{tikzpicture}
	\end{center}
}{
	\begin{enumerate}
		\item 
		Par double distributivité,
			\[ (x-3)(7x+1) = x^2 \cdot (7) + x \cdot (1 - 21) + (-3) = 7x^2 - 20x - 3. \]
		\item
		Il n'y a pas de racine carrée dans l'expression de $f$ : la seule opération illégale est la division par zéro.
		Le dénominateur $(x-3)(7x+1)$ est nul lorsqu'un des deux facteur est nul, c'est-à-dire lorsque $x=3$ ou $x= -\dfrac17$.
		Par conséquent, $\D_f = \R - \left\{ 3 ; -\dfrac17 \right\}$.
		
		\item 
		Dans la dernière ligne, on note par les doubles barres les valeurs interdites à $f$.
		Lorsque le numérateur $2x+1$ est nul, $f$ est bien nulle, mais $f$ n'est pas définie aux $x$ pour lesquels le dénominateur s'annule.
		\item 
			\[ \{ x \in \D_f \text{ tq. } f(x) \geq 0 \} = \left[ -\dfrac12 ; -\dfrac17 \right[ \cup ]3 ; \pinfty[. \]
		Les valeurs interdites $-\frac17$ et $3$ ne sont pas incluses car elles n'appartiennent pas à $\D_f$, ensemble dans lequel on pioche nos $x$.
	\end{enumerate}
}

\exe{
	Donner le domaine de définition $\D_f$ de la fonction
		\[ f(x) = \dfrac{\sqrt{7+2x}}{(3x+1) \sqrt{2-x}}, \]
	et trouver l'ensemble des $x\in\D_f$ vérifiant $f(x) \leq 0$.
}{
	On identifie quatre contraintes sur $x$ :
		\begin{enumerate}
			\item $7+2x \geq 0$ ;
			\item $3x+1 \neq 0$ ;
			\item $2-x \geq 0$ ; et
			\item $2-x \neq 0$.
		\end{enumerate}
	On traite d'abord les inégalités qui imposent $x \geq -\dfrac72$, et $x\leq 2$.
	Ainsi, $x$ appartient nécessairement à l'intervalle $\left[ -\dfrac72 ; 2 \right]$.
	
	Les autres contraintes imposent que $x \neq -\dfrac13$ et $x\neq 2$, ce qui perfore l'intervalle obtenu :
		\[ \D_f = \left[ -\dfrac72 ; 2 \right] - \left\{ -\dfrac13 ; 2 \right\} = \left[ -\dfrac72 ; -\dfrac13 \right[ \cup \left] -\dfrac13; 2 \right[. \]
	
	Pour obtenir le signe, remarquons qu'il n'est que nécessaire d'étudier le signe de $3x+1$, car les racines carrées sont toujours positives.
	Or $3x+1$ est négatif avant sa racine et positif après, ce qui implique que
		\[ \{ x \in \D_f \text{ tq. } f(x) \leq 0 \} = \D_f \cap \left]\minfty ; -\dfrac13 \right] = \left[ -\dfrac72 ; -\dfrac13 \right[. \]
}

\exe{
	Soit une fonction $f$ donnée par
		\[ f(x) = \dfrac{(4x-8)(x-3)}{-2x^2 + 5x + 12}. \]
	\begin{enumerate}
		\item Montrer que $-2x^2 + 5x + 12 = -2\left(x+\dfrac32\right)(x-4)$.
		\item En déduire $\D_f$.
		\item Donner $\{ x \in \D_f \text{ tq. } f(x) \leq 0 \}$ sous forme d'union d'intervalles.
	\end{enumerate}
}{
	\begin{enumerate}
		\item 
		Par double distributivité, on obtient
			\[ -2\left(x+\dfrac32\right)(x-4) = -2 \left( x^2 + \dfrac32 x - 4x - \dfrac32 \cdot 4 \right) = -2x^2 + 5x + 12. \]
		\item 
		Les seules contraintes sont que le dénominateur ne s'annule pas :
			\[ \D_f = \R - \left\{ - \dfrac32 ; 4 \right \}. \]
		\item
		On crée un tableau de signes utilisant que
			\[ f(x) = \dfrac{(4x-8)(x-3)}{-2\left(x+\dfrac32\right)(x-4)} = -\dfrac12 \cdot \dfrac{(4x-8)(x-3)}{\left(x+\dfrac32\right)(x-4)}. \]
		Remarquons, de manière importante, que le signe de $f$ est donc l'opposé du signe de $\dfrac{(4x-8)(x-3)}{\left(x+\dfrac32\right)(x-4)}$ à cause du $-2$ !
		On l'ajoute donc au tableau pour ne pas l'oublier...
	\end{enumerate}


	\begin{center}
	\begin{tikzpicture}
		\tkzTabInit
		 %[lgt=3,espcl=1.5]
	       		{$x$ / 1, $4x-8$ / 1 , $x-3$ / 1, $x+\dfrac32$ / 1 , $x-4$ / 1, $-2$ / 1, $f(x)$ / 1}
	       		{$\minfty$, $-\dfrac32$, $2$, $3$, $4$, $\pinfty$}
	       		
		\tkzTabLine
			{,-,,-,z,+,,+,,+,}
		\tkzTabLine
			{,-,,-,,-,z,+,,+,}
		\tkzTabLine
			{,-,z,+,,+,,+,,+,}
		\tkzTabLine
			{,-,,-,,-,,-,z,+,}
		\tkzTabLine
			{,-,,-,,-,,-,,-,}
		\tkzTabLine
			{,-,d,+,z,-,z,+,d,-,}
	\end{tikzpicture}
	\end{center}
	
	D'où 
		\[  \{ x \in \D_f \text{ tq. } f(x) \leq 0 \} = \left] \minfty ; -\dfrac32 \right[ \cup  \left[ 2 ; 3 \right]  \cup \left]4;\pinfty \right[. \]
}


\end{document}

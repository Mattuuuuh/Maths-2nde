				% ENABLE or DISABLE font change
				% use XeLaTeX if true
\newif\ifdys
				\dystrue
				\dysfalse

\newif\ifsolutions
				\solutionstrue
				\solutionsfalse

% DYSLEXIA SWITCH
\newif\ifdys
		
				% ENABLE or DISABLE font change
				% use XeLaTeX if true
				\dystrue
				\dysfalse


\ifdys

\documentclass[a4paper, 14pt]{extarticle}
\usepackage{amsmath,amsfonts,amsthm,amssymb,mathtools}

\tracinglostchars=3 % Report an error if a font does not have a symbol.
\usepackage{fontspec}
\usepackage{unicode-math}
\defaultfontfeatures{ Ligatures=TeX,
                      Scale=MatchUppercase }

\setmainfont{OpenDyslexic}[Scale=1.0]
\setmathfont{Fira Math} % Or maybe try KPMath-Sans?
\setmathfont{OpenDyslexic Italic}[range=it/{Latin,latin}]
\setmathfont{OpenDyslexic}[range=up/{Latin,latin,num}]

\else

\documentclass[a4paper, 12pt]{extarticle}

\usepackage[utf8x]{inputenc}
%fonts
\usepackage{amsmath,amsfonts,amsthm,amssymb,mathtools}
% comment below to default to computer modern
\usepackage{libertinus,libertinust1math}

\fi


\usepackage[french]{babel}
\usepackage[
a4paper,
margin=2cm,
nomarginpar,% We don't want any margin paragraphs
]{geometry}
\usepackage{icomma}

\usepackage{fancyhdr}
\usepackage{array}
\usepackage{hyperref}

\usepackage{multicol, enumerate}
\newcolumntype{P}[1]{>{\centering\arraybackslash}p{#1}}


\usepackage{stackengine}
\newcommand\xrowht[2][0]{\addstackgap[.5\dimexpr#2\relax]{\vphantom{#1}}}

% theorems

\theoremstyle{plain}
\newtheorem{theorem}{Th\'eor\`eme}
\newtheorem*{sol}{Solution}
\theoremstyle{definition}
\newtheorem{ex}{Exercice}
\newtheorem*{rpl}{Rappel}
\newtheorem{enigme}{Énigme}

% corps
\usepackage{calrsfs}
\newcommand{\C}{\mathcal{C}}
\newcommand{\R}{\mathbb{R}}
\newcommand{\Rnn}{\mathbb{R}^{2n}}
\newcommand{\Z}{\mathbb{Z}}
\newcommand{\N}{\mathbb{N}}
\newcommand{\Q}{\mathbb{Q}}

% variance
\newcommand{\Var}[1]{\text{Var}(#1)}

% domain
\newcommand{\D}{\mathcal{D}}


% date
\usepackage{advdate}
\AdvanceDate[0]


% plots
\usepackage{pgfplots}

% table line break
\usepackage{makecell}
%tablestuff
\def\arraystretch{2}
\setlength\tabcolsep{15pt}

%subfigures
\usepackage{subcaption}

\definecolor{myg}{RGB}{56, 140, 70}
\definecolor{myb}{RGB}{45, 111, 177}
\definecolor{myr}{RGB}{199, 68, 64}

% fake sections with no title to move around the merged pdf
\newcommand{\fakesection}[1]{%
  \par\refstepcounter{section}% Increase section counter
  \sectionmark{#1}% Add section mark (header)
  \addcontentsline{toc}{section}{\protect\numberline{\thesection}#1}% Add section to ToC
  % Add more content here, if needed.
}


% SOLUTION SWITCH
\newif\ifsolutions
				\solutionstrue
				%\solutionsfalse

\ifsolutions
	\newcommand{\exe}[2]{
		\begin{ex} #1  \end{ex}
		\begin{sol} #2 \end{sol}
	}
\else
	\newcommand{\exe}[2]{
		\begin{ex} #1  \end{ex}
	}
	
\fi


% tableaux var, signe
\usepackage{tkz-tab}


%pinfty minfty
\newcommand{\pinfty}{{+}\infty}
\newcommand{\minfty}{{-}\infty}

\begin{document}


\AdvanceDate[0]

\begin{document}
\pagestyle{fancy}
\fancyhead[L]{Seconde 13}
\fancyhead[C]{\textbf{Signes et variations 1 \ifsolutions -- Solutions  \fi}}
\fancyhead[R]{\today}


\exe{\label{ex:1}
	Pour chacune des fonctions affines $f$ suivantes, donner l'intervalle 
		\[ \{ x \in \R \text{ tq. } f(x) \geq 0 \}. \]
	
	\begin{multicols}{2}
	\begin{enumerate}
		\item $f(x) = x+3$
		\item $f(x) = 3x-10$
		\item $f(x) = x$
		\item $f(x) = -4x-12$
		\item $f(x) = -10x$
		\item $f(x) = 1$
	\end{enumerate}
	\end{multicols}
}{
	\begin{enumerate}
		\item 
		L'inégalité $x+3 \geq 0$ est équivalente à $x \geq -3$.
		On obtient donc 
			\[ \{ x \in \R \text{ tq. } x+ 3 \geq 0 \} = \{ x \in \R \text{ tq. }x \geq -3 \} = [-3 ; \pinfty[. \]
		\item 
			\[ \left[ \dfrac{10}3 ; \pinfty \right[ \]
		\item
			\[ [0 ; \pinfty[ \]
		\item
		L'inégalité $-4x - 12 \geq 0$ est équivalente à $-4x \geq 12$.
		Pour continuer, on multiplie par $-\dfrac14$ qui est négatif, ce qui change donc le sens de l'inégalité : $x \leq -3$.
		On obtient donc
			\[ ]\minfty ; -3]. \]
		\item
			\[ ]\minfty ; 0] \]
		\item 
			Tous les $x\in\R$ vérifie que $f(x) \geq 0$ car $1 \geq 0$ est toujours vrai.
			Donc 
			\[ \{ x \in \R \text{ tq. } 1 \geq 0 \} = \R. \]
	\end{enumerate}

}

\ifdys
\exe{
	Exprimer par lecture graphique les ensembles suivants sous forme d'union d'intervalles.
		\begin{enumerate}
			\item $\{ x \in [-3 ; 3] \text{ tq. } f(x) \geq 0\}$
			\item $\{ x \in [-3 ; 3] \text{ tq. } g(x) \leq 0\}$
			\item $\{ x \in [-3 ; 3] \text{ tq. } f(x) \cdot g(x) \geq 0\}$
		\end{enumerate}
	
	\begin{center}
	\begin{tikzpicture}[>=stealth, scale=1]
	\begin{axis}[xmin = -3, xmax=3, ymin=-4, ymax=4, axis x line=middle, axis y line=middle, axis line style=<->, xlabel={}, ylabel={}, xtick = {-4, -3, ..., 4}, ytick = {-4, -3, ..., 4}, grid=both]
		
		% (g)
		\addplot[myr, very thick, domain =-3:3, samples=50] {(x+1)*x*(x-2)+.5}  node[pos = .77, right=2pt] {$\C_g$};
		
		% (f)
		\addplot[myb, very thick, domain =-3:3, samples=50] {2.9-1.2*x^2}  node[pos = .41, above=4pt] {$\C_f$};
	\end{axis}
	\end{tikzpicture}
	\end{center}
}{}
\else
\exe{
	\begin{multicols}{2}
	Exprimer par lecture graphique les ensembles suivants sous forme d'union d'intervalles.
		\begin{enumerate}
			\item $\{ x \in [-3 ; 3] \text{ tq. } f(x) \geq 0\}$
			\item $\{ x \in [-3 ; 3] \text{ tq. } g(x) \leq 0\}$
			\item $\{ x \in [-3 ; 3] \text{ tq. } f(x) \cdot g(x) \geq 0\}$
		\end{enumerate}
	\vfill
	
	\begin{center}
	\begin{tikzpicture}[>=stealth, scale=1]
	\begin{axis}[xmin = -3, xmax=3, ymin=-4, ymax=4, axis x line=middle, axis y line=middle, axis line style=<->, xlabel={}, ylabel={}, xtick = {-4, -3, ..., 4}, ytick = {-4, -3, ..., 4}, grid=both]
		
		% (g)
		\addplot[myr, very thick, domain =-3:3, samples=50] {(x+1)*x*(x-2)+.5}  node[pos = .77, right=2pt] {$\C_g$};
		
		% (f)
		\addplot[myb, very thick, domain =-3:3, samples=50] {2.9-1.2*x^2}  node[pos = .41, above=4pt] {$\C_f$};
	\end{axis}
	\end{tikzpicture}
	\end{center}
	\end{multicols}
}{
	L'ordonnée du point d'abscisse $x$ appartenant à $\C_f$ est $f(x)$.
	On lit donc son signe en regardant s'il est au-dessus ou en-dessous de l'axe des abscisses.
	
	\begin{enumerate}
		\item On a environ 
			\[ \{ x \in [-3 ; 3] \text{ tq. } f(x) \geq 0\} \approx [-1,5 ; 1,5]. \]
		\item On a environ 
			\[ \{ x \in [-3 ; 3] \text{ tq. } g(x) \leq 0\} \approx [-3 ; -1,2] \cup [1,9 ; 3]. \]
		On est en droit de se demander : comment sait-on que $g(-2) \leq 0$ si on ne peut pas lire sa valeur graphiquement ?
		On suppose en fait ici que $\C_g$ est une courbe continue (tracée sans lever le crayon). 
		Ainsi, si $g(-2)$ était positif, la courbe devrait nécessairement couper l'axe des abscisses, ce qui n'est visiblement pas le cas.
		\item Il y a deux situations dans lesquelles le produit $f(x) \cdot g(x)$ peut être positif : soit $f(x)$ et $g(x)$ sont positifs, soit ils sont tous deux négatifs.
		Graphiquement, on lit que
			\[ \{ x \in [-3 ; 3] \text{ tq. } f(x) \geq 0 \text{ et } g(x) \geq 0 \} \approx [-1,2 ; 0,2], \]
		et que (en faisant à nouveau une hypothèse de continuité)
			\[ \{ x \in [-3 ; 3] \text{ tq. } f(x) \leq 0 \text{ et } g(x) \leq 0 \} \approx [-3 ; -1,6] \cup [1,6 ; 1,9]. \]
		On prend l'union des deux pour obtenir
			\[ \{ x \in [-3 ; 3] \text{ tq. } f(x) \cdot g(x)  \geq 0 \} \approx [-1,2 ; 0,2] \cup [-3 ; -1,6] \cup [1,6 ; 1,9]. \]
	\end{enumerate}
}
\fi

\exe{
	À l'aide de l'exercice \ref{ex:1}, donner le domaine de définition de chaque fonction suivante.
	
	\begin{enumerate}
		\item $f(x) = \sqrt{x+3}$
		\item $g(x) = \sqrt{3x-10}$
		\item $h(x) = \sqrt{-4x-12}$
	\end{enumerate}

}{
	Le domaine de définition est le plus grand domaine sur lequel une fonction est bien définie.
	Ici, la seule chose qui peut se mal passer est qu'on demande à $f$ de calculer une racine carrée d'un nombre négatif.
	Ceci n'est pas possible car $x^2 \geq 0$ pour tout $x\in\R$, donc si on souhaite calculer $\sqrt{a}$ vérifiant $\sqrt{a}^2 = a$ par définition, on a nécessairement $a\geq0$.
	Par exemple, $\sqrt{-1}$ ne peut pas être un nombre réel, sinon $\sqrt{-1}^2 = -1$, et un carré serait négatif !
	
	On impose donc que l'expression sous une racine carrée soit positive ou nulle, et on prend tous les réels vérifiant cette propriété.
	\begin{enumerate}
		\item
			\[ \D_f = \{ x \in \R \text{ tq. } x + 3 \geq 0 \} = [-3; \pinfty[, \]
		d'après l'exercice \ref{ex:1}.
		\item
			\[ \D_f = \{ x \in \R \text{ tq. } 3x- 10 \geq 0 \} = \left[ \dfrac{10}3 ; \pinfty \right[, \]
		d'après l'exercice \ref{ex:1}.
		\item
			\[ \D_f = \{ x \in \R \text{ tq. } -4x-12 \geq 0 \} = ]\minfty; -3], \]
		d'après l'exercice \ref{ex:1}.
	\end{enumerate}
}

\exe{\label{ex:4}
	À l'aide de l'exercice \ref{ex:1} remplir les tableaux de signes ci-dessous et donner
		\begin{align*}
			\{ x \in \R \text{ tq. } f(x) \geq 0 \}, && \{ x \in \R \text{ tq. } g(x) \leq 0 \}, && \{ x \in \R \text{ tq. } h(x) \geq 0 \}
		\end{align*}
	sous forme d'union d'intervalles.
	
	\begin{center}
	\begin{tikzpicture}
		\tkzTabInit
		 %[lgt=3,espcl=1.5]
		 [lgt=5]
	       		{$x$ / 1 , $x+3$ / 1, $3x-10$ / 1 , $f(x) = (x+3)(3x-10)$ / \ifdys 2 \else1\fi}
	       		{$\minfty$,\ifsolutions $-3$ \fi,\ifsolutions $\dfrac{10}3$ \fi,$\pinfty$}
		
		\ifsolutions
		\tkzTabLine
			{,-,z,+,,+}
		\tkzTabLine
			{,-,,-,z,+}
		\tkzTabLine
			{,+, z, -, z, +}
		\fi
	\end{tikzpicture}
	
	\begin{tikzpicture}
		\tkzTabInit
		 %[lgt=3,espcl=1.5]
		 [lgt=5]
	       		{$x$ / 1 , $x$ / 1, $-4x-12$ / 1 , $g(x) = x(-4x-12)$ / \ifdys 2 \else1\fi}
	       		{\ifsolutions $\minfty$ \fi,\ifsolutions $-3$ \fi,\ifsolutions $0$ \fi,\ifsolutions $\pinfty$ \fi}
	       		
		\ifsolutions
		\tkzTabLine
			{,-,z,+,,+}
		\tkzTabLine
			{,+,,+,z,-}
		\tkzTabLine
			{,-, z, +, z, -}
		\fi
	\end{tikzpicture}
	
	
	\begin{tikzpicture}
		\tkzTabInit
		 %[lgt=3,espcl=1.5]
		 [lgt=5]
	       		{$x$ / 1 , $x+3$ / 1, $-10x$ / 1 , $3x-10$ / 1 , $h(x) = -10x(x+3)(3x-10)$ / \ifdys 2 \else1\fi}
	       		{\ifsolutions $\minfty$ \fi,\ifsolutions $-3$ \fi, \ifsolutions $0$ \fi, \ifsolutions $\dfrac{10}3$ \fi,\ifsolutions $\pinfty$ \fi}
	       		
		\ifsolutions
		\tkzTabLine
			{,-,z,+,,+,,+}
		\tkzTabLine
			{,+,,+,z,-,,-}
		\tkzTabLine
			{,-,,-,, -,z,+}
		\tkzTabLine
			{,+,z,-,z, +,z,-}
		\fi
	\end{tikzpicture}
	\end{center}
}{
	On remplit le tableau de signe en calculant d'abord quand chaque fonction affine s'annule : c'est exactement quand elle change de signe (du positif vers le négatif, ou l'inverse).
	
	Par exemple, $x+3$ s'annule en $x=-3$, et l'exercice \ref{ex:1} donne que $x+3$ est positif après $-3$, est donc nécessairement négatif avant.
	On fait idem pour $3x-10$, qui s'annule en $\dfrac{10}3$, et qui est positif après, et négatif avant.
	
	Le signe du produit $f(x) = (x+3)(3x-10)$ est déduit du signe des facteurs, à l'aide des règles ``positif $\times$ positif = positif ; négatif $\times$ négatif = positif ; positif $\times$ négatif = négatif''.
	De plus, lorsqu'un des deux facteurs des nul, le produit est forcément nul, multiplier par zéro donne toujours zéro.
	
	Le deuxième tableau est similaire, en faisant attention au fait que le signe de $-4x-12$ est positif puis négatif (cf. exercice \ref{ex:1}).
	Ceci est en fait dû au fait qu'on ait multiplié par $-\dfrac14$ lors de la résolution de $-4x-12 \geq 0$.
	Le signe du coefficient directeur $a$ de la fonction affine donne donc l'allure de la droite associée : si $a>0$, la droite monte du négatif vers le positif en passant par $0$, et si $a <0$, la droite descend du positif vers le négatif en passant par $0$.
	
	Le cas $a=0$ est le cas de la fonction constante, dont l'étude de signe n'est pas très intéressante.
}


\exe{
	À l'aide de l'exercice \ref{ex:4}, donner le domaine de définition de la fonction
		\[ f(x) = \sqrt{-10x(x+3)(3x-10)}. \]
}{
	La seul contrainte à poser sur $x$ est que l'expression sous la racine soit positive ou nulle.
		\[ \D_f = \{ x \in \R \text{ tq. } -10x(x+3)(3x-10) \geq 0 \}. \]
	Or la dernière ligne du dernier tableau de signes de l'exercice \ref{ex:4} nous donne exactement les intervalles où $h(x) = -10x(x+3)(3x-10)$ est positive ou nulle :
		\[ \D_f = ]\minfty ; -3] \cup \left[0 ; \dfrac{10}3 \right]. \]
}

\exe{
	On souhaite connaître pour quels $x\in\D_f$ l'inégalité suivante est vérifée.
		\[ f(x) = \dfrac{2x+1}{7x^2 - 20x - 3} \geq 0 \]
	
	\begin{enumerate}
		\item Montrer que $7x^2 - 20x - 3 = (x-3)(7x+1)$.
		\item En déduire les valeurs interdites à $f$ et donc le domaine de définition $\D_f$.
		\item Remplir le tableau de signes ci-dessous.
		\item Exprimer $\{ x \in \D_f \text{ tq. } f(x) \geq 0 \}$ sous forme d'union d'intervalles.
	\end{enumerate}
	
	
	\begin{center}
	\begin{tikzpicture}
		\tkzTabInit
		 %[lgt=3,espcl=1.5]
	       		{$x$ / 1, $x-3$ / 1 , $7x+1$ / 1, $2x+1$ / 1 , $f(x)$ / 1}
	       		{\ifsolutions $\minfty$ \fi,\ifsolutions $-\dfrac12$ \fi, \ifsolutions $-\dfrac17$ \fi, \ifsolutions $3$ \fi,\ifsolutions $\pinfty$ \fi}
	       		
		\ifsolutions
		\tkzTabLine
			{,-,,-,,-,z,+}
		\tkzTabLine
			{,-,,-,z,+,,+}
		\tkzTabLine
			{,-,z,+,,+,,+}
		\tkzTabLine
			{,-,z,+,d, -,d,+}
		\fi
	\end{tikzpicture}
	\end{center}
}{
	\begin{enumerate}
		\item 
		Par double distributivité,
			\[ (x-3)(7x+1) = x^2 \cdot (7) + x \cdot (1 - 21) + (-3) = 7x^2 - 20x - 3. \]
		\item
		Il n'y a pas de racine carrée dans l'expression de $f$ : la seule opération illégale est la division par zéro.
		Le dénominateur $(x-3)(7x+1)$ est nul lorsqu'un des deux facteur est nul, c'est-à-dire lorsque $x=3$ ou $x= -\dfrac17$.
		Par conséquent, $\D_f = \R - \left\{ 3 ; -\dfrac17 \right\}$.
		
		\item 
		Dans la dernière ligne, on note par les doubles barres les valeurs interdites à $f$.
		Lorsque le numérateur $2x+1$ est nul, $f$ est bien nulle, mais $f$ n'est pas définie aux $x$ pour lesquels le dénominateur s'annule.
		\item 
			\[ \{ x \in \D_f \text{ tq. } f(x) \geq 0 \} = \left[ -\dfrac12 ; -\dfrac17 \right[ \cup ]3 ; \pinfty[. \]
		Les valeurs interdites $-\frac17$ et $3$ ne sont pas incluses car elles n'appartiennent pas à $\D_f$, ensemble dans lequel on pioche nos $x$.
	\end{enumerate}
}

\exe{
	Donner le domaine de définition $\D_f$ de la fonction
		\[ f(x) = \dfrac{\sqrt{7+2x}}{(3x+1) \sqrt{2-x}}, \]
	et trouver l'ensemble des $x\in\D_f$ vérifiant $f(x) \leq 0$.
}{
	On identifie quatre contraintes sur $x$ :
		\begin{enumerate}
			\item $7+2x \geq 0$ ;
			\item $3x+1 \neq 0$ ;
			\item $2-x \geq 0$ ; et
			\item $2-x \neq 0$.
		\end{enumerate}
	On traite d'abord les inégalités qui imposent $x \geq -\dfrac72$, et $x\leq 2$.
	Ainsi, $x$ appartient nécessairement à l'intervalle $\left[ -\dfrac72 ; 2 \right]$.
	
	Les autres contraintes imposent que $x \neq -\dfrac13$ et $x\neq 2$, ce qui perfore l'intervalle obtenu :
		\[ \D_f = \left[ -\dfrac72 ; 2 \right] - \left\{ -\dfrac13 ; 2 \right\} = \left[ -\dfrac72 ; -\dfrac13 \right[ \cup \left] -\dfrac13; 2 \right[. \]
	
	Pour obtenir le signe, remarquons qu'il n'est que nécessaire d'étudier le signe de $3x+1$, car les racines carrées sont toujours positives.
	Or $3x+1$ est négatif avant sa racine et positif après, ce qui implique que
		\[ \{ x \in \D_f \text{ tq. } f(x) \leq 0 \} = \D_f \cap \left]\minfty ; -\dfrac13 \right] = \left[ -\dfrac72 ; -\dfrac13 \right[. \]
}

\exe{
	Soit une fonction $f$ donnée par
		\[ f(x) = \dfrac{(4x-8)(x-3)}{-2x^2 + 5x + 12}. \]
	\begin{enumerate}
		\item Montrer que $-2x^2 + 5x + 12 = -2\left(x+\dfrac32\right)(x-4)$.
		\item En déduire $\D_f$.
		\item Donner $\{ x \in \D_f \text{ tq. } f(x) \leq 0 \}$ sous forme d'union d'intervalles.
	\end{enumerate}
}{
	\begin{enumerate}
		\item 
		Par double distributivité, on obtient
			\[ -2\left(x+\dfrac32\right)(x-4) = -2 \left( x^2 + \dfrac32 x - 4x - \dfrac32 \cdot 4 \right) = -2x^2 + 5x + 12. \]
		\item 
		Les seules contraintes sont que le dénominateur ne s'annule pas :
			\[ \D_f = \R - \left\{ - \dfrac32 ; 4 \right \}. \]
		\item
		On crée un tableau de signes utilisant que
			\[ f(x) = \dfrac{(4x-8)(x-3)}{-2\left(x+\dfrac32\right)(x-4)} = -\dfrac12 \cdot \dfrac{(4x-8)(x-3)}{\left(x+\dfrac32\right)(x-4)}. \]
		Remarquons, de manière importante, que le signe de $f$ est donc l'opposé du signe de $\dfrac{(4x-8)(x-3)}{\left(x+\dfrac32\right)(x-4)}$ à cause du $-2$ !
		On l'ajoute donc au tableau pour ne pas l'oublier...
	\end{enumerate}


	\begin{center}
	\begin{tikzpicture}
		\tkzTabInit
		 %[lgt=3,espcl=1.5]
	       		{$x$ / 1, $4x-8$ / 1 , $x-3$ / 1, $x+\dfrac32$ / 1 , $x-4$ / 1, $-2$ / 1, $f(x)$ / 1}
	       		{$\minfty$, $-\dfrac32$, $2$, $3$, $4$, $\pinfty$}
	       		
		\tkzTabLine
			{,-,,-,z,+,,+,,+,}
		\tkzTabLine
			{,-,,-,,-,z,+,,+,}
		\tkzTabLine
			{,-,z,+,,+,,+,,+,}
		\tkzTabLine
			{,-,,-,,-,,-,z,+,}
		\tkzTabLine
			{,-,,-,,-,,-,,-,}
		\tkzTabLine
			{,-,d,+,z,-,z,+,d,-,}
	\end{tikzpicture}
	\end{center}
	
	D'où 
		\[  \{ x \in \D_f \text{ tq. } f(x) \leq 0 \} = \left] \minfty ; -\dfrac32 \right[ \cup  \left[ 2 ; 3 \right]  \cup \left]4;\pinfty \right[. \]
}


\end{document}

%% INPUT PREAMBLE.TEX
%% THEN INPUT VARS_{i}.ADR
%% THEN RUN THIS
%% DYSLEXIA SWITCH
\newif\ifdys
		
				% ENABLE or DISABLE font change
				% use XeLaTeX if true
				\dystrue
				\dysfalse


\ifdys

\documentclass[a4paper, 14pt]{extarticle}
\usepackage{amsmath,amsfonts,amsthm,amssymb,mathtools}

\tracinglostchars=3 % Report an error if a font does not have a symbol.
\usepackage{fontspec}
\usepackage{unicode-math}
\defaultfontfeatures{ Ligatures=TeX,
                      Scale=MatchUppercase }

\setmainfont{OpenDyslexic}[Scale=1.0]
\setmathfont{Fira Math} % Or maybe try KPMath-Sans?
\setmathfont{OpenDyslexic Italic}[range=it/{Latin,latin}]
\setmathfont{OpenDyslexic}[range=up/{Latin,latin,num}]

\else

\documentclass[a4paper, 12pt]{extarticle}

\usepackage[utf8x]{inputenc}
%fonts
\usepackage{amsmath,amsfonts,amsthm,amssymb,mathtools}
% comment below to default to computer modern
\usepackage{libertinus,libertinust1math}

\fi


\usepackage[french]{babel}
\usepackage[
a4paper,
margin=2cm,
nomarginpar,% We don't want any margin paragraphs
]{geometry}
\usepackage{icomma}

\usepackage{fancyhdr}
\usepackage{array}
\usepackage{hyperref}

\usepackage{multicol, enumerate}
\newcolumntype{P}[1]{>{\centering\arraybackslash}p{#1}}


\usepackage{stackengine}
\newcommand\xrowht[2][0]{\addstackgap[.5\dimexpr#2\relax]{\vphantom{#1}}}

% theorems

\theoremstyle{plain}
\newtheorem{theorem}{Th\'eor\`eme}
\newtheorem*{sol}{Solution}
\theoremstyle{definition}
\newtheorem{ex}{Exercice}
\newtheorem*{rpl}{Rappel}
\newtheorem{enigme}{Énigme}

% corps
\usepackage{calrsfs}
\newcommand{\C}{\mathcal{C}}
\newcommand{\R}{\mathbb{R}}
\newcommand{\Rnn}{\mathbb{R}^{2n}}
\newcommand{\Z}{\mathbb{Z}}
\newcommand{\N}{\mathbb{N}}
\newcommand{\Q}{\mathbb{Q}}

% variance
\newcommand{\Var}[1]{\text{Var}(#1)}

% domain
\newcommand{\D}{\mathcal{D}}


% date
\usepackage{advdate}
\AdvanceDate[0]


% plots
\usepackage{pgfplots}

% table line break
\usepackage{makecell}
%tablestuff
\def\arraystretch{2}
\setlength\tabcolsep{15pt}

%subfigures
\usepackage{subcaption}

\definecolor{myg}{RGB}{56, 140, 70}
\definecolor{myb}{RGB}{45, 111, 177}
\definecolor{myr}{RGB}{199, 68, 64}

% fake sections with no title to move around the merged pdf
\newcommand{\fakesection}[1]{%
  \par\refstepcounter{section}% Increase section counter
  \sectionmark{#1}% Add section mark (header)
  \addcontentsline{toc}{section}{\protect\numberline{\thesection}#1}% Add section to ToC
  % Add more content here, if needed.
}


% SOLUTION SWITCH
\newif\ifsolutions
				\solutionstrue
				%\solutionsfalse

\ifsolutions
	\newcommand{\exe}[2]{
		\begin{ex} #1  \end{ex}
		\begin{sol} #2 \end{sol}
	}
\else
	\newcommand{\exe}[2]{
		\begin{ex} #1  \end{ex}
	}
	
\fi


% tableaux var, signe
\usepackage{tkz-tab}


%pinfty minfty
\newcommand{\pinfty}{{+}\infty}
\newcommand{\minfty}{{-}\infty}

\begin{document}

%\input{adr/vars_44284.adr}
%\newcommand{\seed}{TEST}

\pagestyle{fancy}
\fancyhead[L]{Seconde 13}
\fancyhead[C]{\textbf{Devoir Maison 4 --- \seed \ifsolutions \, --- Solutions  \fi}}
\fancyhead[R]{\today}

\exe{
	Considérons trois fonctions du deuxième degré définies sur $\R$.
	\begin{align*}
		f(x) = \fa x^2  \fb x \fc, && g(x) = \left(\gaI x \gbI\right)\left(\gaII x \gbII\right), && h(x) = -\hbeta + (\ha x \hb)^2.
	\end{align*}
	
	\begin{enumerate}
		\item Montrer que $g = f$ en développant $g(x)$.
		\item Remplir le tableau de signe de $f$ à l'aide de l'expression de $g$.
		\item Montrer que $h = f$ en développant $h(x)$.
		%\item Remplir le tableau de variations de $f$ à l'aide de l'expression de $h$.
		\item Donner le minimum de $f$ sur $\R$ et l'antécédent $x^\star \in \R$ qui le réalise.
	\end{enumerate}

	\begin{center}
	\begin{tikzpicture}
		\tkzTabInit
		 %[lgt=3,espcl=1.5]
	       		{$x$ / 1, /1, /1,  Signe de $f(x)$ / 1}
	       		{$\minfty$,,,,$\pinfty$}
	\end{tikzpicture}
	
%	\vfill
%	\begin{tikzpicture}
%		\tkzTabInit
%		 %[lgt=3,espcl=1.5]
%	       		{$x$ / 1,  Variation de $f(x)$ / 2}
%	       		{$\minfty$,,,,$\pinfty$}
%	\end{tikzpicture}
	\end{center}

}{}

\exe{
	Construire, en posant des nombres réels $a, b, c\in\R$, une fonction de la forme
			\[ f(x) = ax^2 + bx + c \]
%	telle que
%			\[ \max\{ f(x) \text{ où $x$ parcourt $\R$} \} = f(-1) = -3. \]
	telle que $-3$ soit le maximum de $f$ sur $\R$, atteint en $x^\star = -1$.
%	qui admet le tableau de variations suivant.
%	\begin{center}
%	\begin{tikzpicture}
%		\tkzTabInit
%		 %[lgt=3,espcl=1.5]
%	       		{$x$ / 1,  Variation de $f(x)$ / 2}
%	       		{$\minfty$,,$-1$,,$\pinfty$}
%	       		
%		\tkzTabVar
%			{-/, R/, +/$-3$, R/, -/}
%		\tkzTabIma
%			{1}{2}{1}{60}
%		\tkzTabIma
%			{2}{4}{4}{25}
%	\end{tikzpicture}
%	\end{center}

}{}

\exe{
	On souhaite connaître les solutions de l'équation quartique d'inconnue $x\in\R$ :
		\begin{align}
			\eqIa x^4 + \eqIb x^2 \eqIc = 0. \label{eq:1}
		\end{align}
	\begin{enumerate}
		\item Montrer que $\eqIa x^4 + \eqIb x^2 \eqIc = (\aV x^2 - \bV)(\cV - \dV x^2)$.
		\item En déduire l'ensemble des solutions de l'équation \eqref{eq:1}.
		Exprimer ces solutions sous la forme $q \sqrt{n}$ où $q\in\Q$ est rationnel et $n\in\N$ est un entier naturel.
		
		\item Substituer les nombres réels obtenus dans l'équation \eqref{eq:1} pour vérifier qu'ils sont bien solution.
	\end{enumerate}
}{}

\exe{
	Construire, en posant des nombres réels $a, b, c\in\R$, une équation de la forme
		\begin{align}
			ax^2 + bx + c = 0 \label{eq:2}
		\end{align}
	telle que l'ensemble des $x\in\R$ solutions de l'équation \eqref{eq:2} soit $\left\{ \qI ; \qII \right\}$.
}{}

\end{document}

				% ENABLE or DISABLE font change
				% use XeLaTeX if true
\newif\ifdys
				\dystrue
				\dysfalse

\newif\ifsolutions
				\solutionstrue
				\solutionsfalse

% DYSLEXIA SWITCH
\newif\ifdys
		
				% ENABLE or DISABLE font change
				% use XeLaTeX if true
				\dystrue
				\dysfalse


\ifdys

\documentclass[a4paper, 14pt]{extarticle}
\usepackage{amsmath,amsfonts,amsthm,amssymb,mathtools}

\tracinglostchars=3 % Report an error if a font does not have a symbol.
\usepackage{fontspec}
\usepackage{unicode-math}
\defaultfontfeatures{ Ligatures=TeX,
                      Scale=MatchUppercase }

\setmainfont{OpenDyslexic}[Scale=1.0]
\setmathfont{Fira Math} % Or maybe try KPMath-Sans?
\setmathfont{OpenDyslexic Italic}[range=it/{Latin,latin}]
\setmathfont{OpenDyslexic}[range=up/{Latin,latin,num}]

\else

\documentclass[a4paper, 12pt]{extarticle}

\usepackage[utf8x]{inputenc}
%fonts
\usepackage{amsmath,amsfonts,amsthm,amssymb,mathtools}
% comment below to default to computer modern
\usepackage{libertinus,libertinust1math}

\fi


\usepackage[french]{babel}
\usepackage[
a4paper,
margin=2cm,
nomarginpar,% We don't want any margin paragraphs
]{geometry}
\usepackage{icomma}

\usepackage{fancyhdr}
\usepackage{array}
\usepackage{hyperref}

\usepackage{multicol, enumerate}
\newcolumntype{P}[1]{>{\centering\arraybackslash}p{#1}}


\usepackage{stackengine}
\newcommand\xrowht[2][0]{\addstackgap[.5\dimexpr#2\relax]{\vphantom{#1}}}

% theorems

\theoremstyle{plain}
\newtheorem{theorem}{Th\'eor\`eme}
\newtheorem*{sol}{Solution}
\theoremstyle{definition}
\newtheorem{ex}{Exercice}
\newtheorem*{rpl}{Rappel}
\newtheorem{enigme}{Énigme}

% corps
\usepackage{calrsfs}
\newcommand{\C}{\mathcal{C}}
\newcommand{\R}{\mathbb{R}}
\newcommand{\Rnn}{\mathbb{R}^{2n}}
\newcommand{\Z}{\mathbb{Z}}
\newcommand{\N}{\mathbb{N}}
\newcommand{\Q}{\mathbb{Q}}

% variance
\newcommand{\Var}[1]{\text{Var}(#1)}

% domain
\newcommand{\D}{\mathcal{D}}


% date
\usepackage{advdate}
\AdvanceDate[0]


% plots
\usepackage{pgfplots}

% table line break
\usepackage{makecell}
%tablestuff
\def\arraystretch{2}
\setlength\tabcolsep{15pt}

%subfigures
\usepackage{subcaption}

\definecolor{myg}{RGB}{56, 140, 70}
\definecolor{myb}{RGB}{45, 111, 177}
\definecolor{myr}{RGB}{199, 68, 64}

% fake sections with no title to move around the merged pdf
\newcommand{\fakesection}[1]{%
  \par\refstepcounter{section}% Increase section counter
  \sectionmark{#1}% Add section mark (header)
  \addcontentsline{toc}{section}{\protect\numberline{\thesection}#1}% Add section to ToC
  % Add more content here, if needed.
}


% SOLUTION SWITCH
\newif\ifsolutions
				\solutionstrue
				%\solutionsfalse

\ifsolutions
	\newcommand{\exe}[2]{
		\begin{ex} #1  \end{ex}
		\begin{sol} #2 \end{sol}
	}
\else
	\newcommand{\exe}[2]{
		\begin{ex} #1  \end{ex}
	}
	
\fi


% tableaux var, signe
\usepackage{tkz-tab}


%pinfty minfty
\newcommand{\pinfty}{{+}\infty}
\newcommand{\minfty}{{-}\infty}

\begin{document}


\newdate{date}{09}{05}{2025}

%%%%%%%%%%%%%%%%%
%% ESTIMÉ 1H15 %%
%%%%%%%%%%%%%%%%%

\begin{document}
\pagestyle{fancy}
\fancyhead[L]{Seconde 13}
\fancyhead[C]{\textbf{Évaluation --- Signes et variations \ifsolutions \\ Solutions  \fi}}
\fancyhead[R]{\date}


\exe{[3pts]
	Donner le domaine de définition de chaque fonction suivante.
	\begin{multicols}{3}
	\begin{enumerate}
		\item $f(x) = \dfrac1{-2\left(x+1\right)}$
		\item $g(x) = \dfrac{\sqrt{2-x}}{10}$
		\item $h(x) = \dfrac{\sqrt{3x+6}}{x}$
		%\item $F(x) = \dfrac{1}{\sqrt{x+1}}$
	\end{enumerate}
	\end{multicols}

}{
}


\exe{[3pts]
	On considère la fonction $f$ donnée graphiquement ci-dessous sur le domaine $\D = [-3; 2,5]$.
	
	\begin{enumerate}
		\item Remplir approximativement les tableaux de signes et de variations de $f$ sur $\D$ ci-dessous.
		\item Tracer la courbe de $g(x) = -f(x)$ dans le même repère.
	\end{enumerate}
	\begin{center}
	\begin{tikzpicture}[>=stealth, scale=1]
	\begin{axis}[xmin = -3, xmax=2.5, ymin=-4, ymax=4, axis x line=middle, axis y line=middle, axis line style=<->, xlabel={}, ylabel={}, xtick = {-4, -3, ..., 4}, ytick = {-4, -3, ..., 4}, grid=both]
		% (f)
		\addplot[myb, very thick, domain =-3:2.5, samples=100] {-.2*(x+2.6)*(x+.5)*(x-.8)*(x-2.1)}  node[pos = .45, above] {$\C_f$};
	\end{axis}
	\end{tikzpicture}
	
	\begin{tikzpicture}
		\tkzTabInit
		 [lgt=3,espcl=3]
	       		{$x$ / 1,  Signe de $f(x)$ / 1 }
	       		{,,,,}
	\end{tikzpicture}
	\vspace{10pt}
	
	\begin{tikzpicture}
		\tkzTabInit
		 [lgt=3,espcl=3]
	       		{$x$ / 1,  Variation de $f(x)$ / 2}
	       		{,,,,}
	\end{tikzpicture}
	\end{center}
}{}

\exe{[4pts]
	On souhaite connaître pour quels $x\in\D_f$ l'inégalité suivante est vérifée.
		\[ f(x) = \dfrac{6 - x - x^2}{3(x-4)} \geq 0 \]
		%\[ f(x) = \dfrac{x+2}{(x+3)(2-x)} \geq 0 \]
	
	\begin{enumerate}
		\item Donner $\D_f$.
		\item Montrer que $6 - x - x^2 = (x+3)(2-x)$.
		%\item Donner les valeurs interdites à $f$ et son domaine de définition $\D_f$.
		\item Remplir le tableau de signes ci-dessous.
		\item Exprimer $\{ x \in \D_f \text{ tq. } f(x) \geq 0 \}$ sous forme d'union d'intervalles.
	\end{enumerate}
	
	
	\begin{center}
	\begin{tikzpicture}
		\tkzTabInit
		 [lgt=3,espcl=3]
	       		{ / 1,  / 1 ,  / 1,  / 1 ,  / 1}
	       		{,,,,}
	\end{tikzpicture}
	\end{center}
}{
}



\exe{[4pts]
	Pour chaque propriété, donner l'expression algébrique d'une fonction $f$ non identiquement nulle la vérifiant. Il s'agit ici de donner une expression de la forme $f(x) = \dots$, et non pas une courbe.
	\begin{multicols}{2}
	\begin{enumerate}
		\item $f$ s'annule en $10$.
		\item $f$ s'annule en $0$, en $1$, et en $-\dfrac45$.
		\item $f(x) \geq 0$ pour tout $x\in\R$.
		\item $\D_f = [-1 ; \pinfty[$.
	\end{enumerate}
	\end{multicols}
}{}


\exe{[2pts]
	Soit $f$ une fonction croissante sur $\R$.
	
	\begin{enumerate}
		\item Donner la définition d'une fonction croissante sur $\R$.
		\item  Démontrer, avec ou sans le cours, que la fonction $g$ donnée par 
		\[ g(x) = 10-2f(x) \]
		est décroissante sur $\R$.
	\end{enumerate}
}{}

\exe{[2pts]
	Soit $f$ la fonction définie par
		\[ f(x) = 17 - 4(2x-3)^2. \]
	\begin{enumerate}
		\item Donner $\D_f$.
		\item Démontrer soigneusement que $f$ atteint son maximum en $x^\star=\dfrac32$ et donner sa valeur.
	\end{enumerate}
}{}

\exe{[2pts]
	Pour chaque proposition du tableau ci-dessous, cocher \og Vrai \fg~si elle est \underline{toujours vraie} et \og Faux \fg~si elle \underline{peut être fausse} à l'aide du tableau de variations suivant.
	
\def\arraystretch{1.5}
\setlength\tabcolsep{6pt}
	\begin{center}
	\begin{tikzpicture}
		\tkzTabInit
		 [lgt=3,espcl=1.5]
	       		{$x$ / 1 , Variation de $f(x)$ / 2}
	       		{-5,$a$,$b$,3,$c$,5,8,$d$}
			
		\tkzTabVar
			{+/, -/,+/$45$,R/,R/,-/$20$, R/, +/}
%		\tkzTabIma
%			{1}{2}{1}{$30$}
%		\tkzTabIma
%			{3}{6}{3}{25}
%		\tkzTabIma
%			{3}{6}{6}{70}
			
	\end{tikzpicture}
	
	\vspace{1cm}
	
	\begin{tabular}{|c|c|c|} \hline
		Proposition & Vrai & Faux \\ \hline
		$a \geq -5$ & & \\ \hline
		$f(a) = 10$ & & \\ \hline
		$8 \leq d$ & & \\ \hline
		$f(-5) = 45$ & & \\ \hline
	\end{tabular}	
	\hfill
	\begin{tabular}{|c|c|c|} \hline
		Proposition & Vrai & Faux \\ \hline
		$f(3) > f(c)$ & & \\ \hline
		$f(c) = f(8)$ & & \\ \hline
		$f(8) \geq 0$ & & \\ \hline
		$f(-5) \geq 45$ & & \\ \hline
	\end{tabular}
	\hfill
	\begin{tabular}{|c|c|c|} \hline
		Proposition & Vrai & Faux \\ \hline
		$f(8) > f(d)$ & & \\ \hline
		$f(b) \leq 45$ & & \\ \hline
		$f(-5) \leq 45$ & & \\ \hline
		$f(a) \leq 100$ & & \\ \hline
	\end{tabular}	
	
	\end{center}

}{}

\subsection*{Bonus}

\exe{[2pts]
	Y a-t-il type d'exercice pour lequel vous avez révisé mais qui n'apparaît pas dans l'évaluation ?
	Si oui, donner un énoncé non trivial et résoudre l'exercice posé.
}{}


\end{document}

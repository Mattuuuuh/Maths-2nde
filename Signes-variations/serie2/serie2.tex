				% ENABLE or DISABLE font change
				% use XeLaTeX if true
\newif\ifdys
				\dystrue
				\dysfalse

\newif\ifsolutions
				\solutionstrue
				\solutionsfalse

% DYSLEXIA SWITCH
\newif\ifdys
		
				% ENABLE or DISABLE font change
				% use XeLaTeX if true
				\dystrue
				\dysfalse


\ifdys

\documentclass[a4paper, 14pt]{extarticle}
\usepackage{amsmath,amsfonts,amsthm,amssymb,mathtools}

\tracinglostchars=3 % Report an error if a font does not have a symbol.
\usepackage{fontspec}
\usepackage{unicode-math}
\defaultfontfeatures{ Ligatures=TeX,
                      Scale=MatchUppercase }

\setmainfont{OpenDyslexic}[Scale=1.0]
\setmathfont{Fira Math} % Or maybe try KPMath-Sans?
\setmathfont{OpenDyslexic Italic}[range=it/{Latin,latin}]
\setmathfont{OpenDyslexic}[range=up/{Latin,latin,num}]

\else

\documentclass[a4paper, 12pt]{extarticle}

\usepackage[utf8x]{inputenc}
%fonts
\usepackage{amsmath,amsfonts,amsthm,amssymb,mathtools}
% comment below to default to computer modern
\usepackage{libertinus,libertinust1math}

\fi


\usepackage[french]{babel}
\usepackage[
a4paper,
margin=2cm,
nomarginpar,% We don't want any margin paragraphs
]{geometry}
\usepackage{icomma}

\usepackage{fancyhdr}
\usepackage{array}
\usepackage{hyperref}

\usepackage{multicol, enumerate}
\newcolumntype{P}[1]{>{\centering\arraybackslash}p{#1}}


\usepackage{stackengine}
\newcommand\xrowht[2][0]{\addstackgap[.5\dimexpr#2\relax]{\vphantom{#1}}}

% theorems

\theoremstyle{plain}
\newtheorem{theorem}{Th\'eor\`eme}
\newtheorem*{sol}{Solution}
\theoremstyle{definition}
\newtheorem{ex}{Exercice}
\newtheorem*{rpl}{Rappel}
\newtheorem{enigme}{Énigme}

% corps
\usepackage{calrsfs}
\newcommand{\C}{\mathcal{C}}
\newcommand{\R}{\mathbb{R}}
\newcommand{\Rnn}{\mathbb{R}^{2n}}
\newcommand{\Z}{\mathbb{Z}}
\newcommand{\N}{\mathbb{N}}
\newcommand{\Q}{\mathbb{Q}}

% variance
\newcommand{\Var}[1]{\text{Var}(#1)}

% domain
\newcommand{\D}{\mathcal{D}}


% date
\usepackage{advdate}
\AdvanceDate[0]


% plots
\usepackage{pgfplots}

% table line break
\usepackage{makecell}
%tablestuff
\def\arraystretch{2}
\setlength\tabcolsep{15pt}

%subfigures
\usepackage{subcaption}

\definecolor{myg}{RGB}{56, 140, 70}
\definecolor{myb}{RGB}{45, 111, 177}
\definecolor{myr}{RGB}{199, 68, 64}

% fake sections with no title to move around the merged pdf
\newcommand{\fakesection}[1]{%
  \par\refstepcounter{section}% Increase section counter
  \sectionmark{#1}% Add section mark (header)
  \addcontentsline{toc}{section}{\protect\numberline{\thesection}#1}% Add section to ToC
  % Add more content here, if needed.
}


% SOLUTION SWITCH
\newif\ifsolutions
				\solutionstrue
				%\solutionsfalse

\ifsolutions
	\newcommand{\exe}[2]{
		\begin{ex} #1  \end{ex}
		\begin{sol} #2 \end{sol}
	}
\else
	\newcommand{\exe}[2]{
		\begin{ex} #1  \end{ex}
	}
	
\fi


% tableaux var, signe
\usepackage{tkz-tab}


%pinfty minfty
\newcommand{\pinfty}{{+}\infty}
\newcommand{\minfty}{{-}\infty}

\begin{document}


\AdvanceDate[0]

\begin{document}
\pagestyle{fancy}
\fancyhead[L]{Seconde 13}
\fancyhead[C]{\textbf{Signes et variations 2 \ifsolutions -- Solutions  \fi}}
\fancyhead[R]{\today}


\exe{
	Pour chaque propriété, donner une fonction $f$ sur $\R$ non identiquement nulle la vérifiant.
	\begin{enumerate}
		\item $f$ s'annule en $1$.
		\item $f$ s'annule en $-10$.
		\item $f$ s'annule en $0$.
		\item $f$ s'annule en $-10$ et en $1$.
		\item Les racines de $f$ sont $2, -3, \dfrac27$, et $0$.
	\end{enumerate}
}{}

\exe{
	On souhaite connaître les solutions de l'équation du deuxième degré d'inconnue $x\in\R$ :
		\begin{align}
			22x^2 - 125x + 22  = 0. \label{eq:1}
		\end{align}
	\begin{enumerate}
		\item Montrer que $22x^2 - 125x + 22 = (2x-11)(11x-2)$.
		\item En déduire l'ensemble des solutions de l'équation \eqref{eq:1}.
	\end{enumerate}
}{}

\exe{
	On souhaite connaître les solutions de l'équation du troisième degré d'inconnue $x\in\R$ :
		\begin{align}
			4x^3 - 6x^2 - 2x + 3  = 0. \label{eq:2}
		\end{align}
	\begin{enumerate}
		\item Montrer que $4x^3 - 6x^2 - 2x + 3 = (2x-3) \left(2x^2-1\right)$.
		\item En déduire l'ensemble des solutions de l'équation \eqref{eq:2}.
	\end{enumerate}
}{}

\exe{
	On souhaite connaître les solutions de l'équation du quatrième degré d'inconnue $x\in\R$ :
		\begin{align}
			16x^4 - 24x^2 + 9  = 0. \label{eq:3}
		\end{align}
	\begin{enumerate}
		\item Montrer que $16x^4 - 24x^2 + 9 = \left(4x^2-3\right)^2$.
		\item En déduire l'ensemble des solutions de l'équation \eqref{eq:3}.
	\end{enumerate}
}{}

\hrule


\exe{
	\begin{enumerate}
		\item Esquisser la courbe représentative de la fonction affine $f(x) =3x - 2$ sur le domaine $\D = [-10 ; 12]$.
		\item Remplir le tableau de variations et de signes ci-dessous.
	\end{enumerate}
	
	\begin{center}
	\begin{tikzpicture}
		\tkzTabInit
		 %[lgt=3,espcl=1.5]
	       		{$x$ / 1 , Variation de $3x-2$ / 2, Signe de $3x-2$ / 1}
	       		{,,,}
	\end{tikzpicture}
	\end{center}
}{}


\exe{
	\begin{enumerate}
		\item Esquisser la courbe représentative de la fonction affine $f(x) = -6x - 2$ sur le domaine $\D = [-10 ; 8]$.
		\item Remplir le tableau de variations et de signes ci-dessous.
	\end{enumerate}
	
	\begin{center}
	\begin{tikzpicture}
		\tkzTabInit
		 %[lgt=3,espcl=1.5]
	       		{$x$ / 1 , Variation de $-6x-2$ / 2, Signe de $-6x-2$ / 1}
	       		{,,,}
	\end{tikzpicture}
	\end{center}
}{}


\exe{
	\begin{enumerate}
		\item Esquisser la courbe représentative de la fonction affine $f(x) = -3$ sur le domaine $\D = [-12 ; -5]$.
		\item Remplir le tableau de variations et de signes ci-dessous.
	\end{enumerate}
	
	\begin{center}
	\begin{tikzpicture}
		\tkzTabInit
		 %[lgt=3,espcl=1.5]
	       		{$x$ / 1 , Variation de $-3$ / 2, Signe de $-3$ / 1}
	       		{,,,}
	\end{tikzpicture}
	\end{center}
}{}

\exe{
	Remplir approximativement les tableaux ci-dessous à l'aide des graphes de $\C_f$ et $\C_g$ sur le domaine $\D = [-11; -2,5]$.
	
	\begin{center}
	\begin{tikzpicture}[scale=1]
		\begin{axis}[xmin = -11, xmax=-2.5, ymin=-4.3, ymax=3.5, axis x line=middle, axis y line=middle, axis line style=->, grid=both, ytick={-4,-3,...,2,3}, xtick={-11, -10,...,-4,-3},
    every y tick label/.style={
        anchor=near yticklabel opposite,
        xshift=0.2em,
    }]
    			%f de l'éval f° affines
			\addplot[no marks, myb, -, very thick] expression[domain=-11:-8, samples=50]{5*(x+10)*(x^3 - 300*x - 1920)/80} 
			node[pos=.6, right]{$\mathcal{C}_f$};
			\addplot[no marks, myb, -, very thick] expression[domain=-8:-6, samples=2]{5*(x/2 + 3.2)};
			\addplot[no marks, myb, -, very thick] expression[domain=-6:-5, samples=50]{5*((x+5.5)^2 - 0.5^2 + 0.2)};
			\addplot[no marks, myb, -, very thick] expression[domain=-5:-2.5, samples=2]{5*0.2};
			
			% g sinus
			\addplot[no marks, myr, -, very thick] expression[domain=-11:-2.5, samples=500]{2*sin(x*60)}
			node[pos=.75, above]{$\mathcal{C}_g$};
		\end{axis}
	\end{tikzpicture}
	
	\begin{tikzpicture}
		\tkzTabInit
		 %[lgt=3,espcl=1.5]
	       		{$x$ / 1 , Signe de $f(x)$ / 1}
	       		{,,,,,}
	\end{tikzpicture}
	\vfill
	\begin{tikzpicture}
		\tkzTabInit
		 %[lgt=3,espcl=1.5]
	       		{$x$ / 1 , Variation de $f(x)$ / 2}
	       		{,,,,,}
	\end{tikzpicture}
	\vfill
	\begin{tikzpicture}
		\tkzTabInit
		 %[lgt=3,espcl=1.5]
	       		{$x$ / 1 , Signe de $g(x)$ / 1}
	       		{,,,,,}
	\end{tikzpicture}
	\vfill
	\begin{tikzpicture}
		\tkzTabInit
		 %[lgt=3,espcl=1.5]
	       		{$x$ / 1 , Variation de $g(x)$ / 2}
	       		{,,,,,}
	\end{tikzpicture}
	\end{center}
	
}{}


\end{document}

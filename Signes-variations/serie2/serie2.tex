				% ENABLE or DISABLE font change
				% use XeLaTeX if true
\newif\ifdys
				\dystrue
				\dysfalse

\newif\ifsolutions
				\solutionstrue
				\solutionsfalse

%!TEX encoding = UTF8
%!TEX root =notes.tex


%%%%%%%%%%%%%%%%%%%%%%%%%%%%%%%%%
% PACKAGE IMPORTS
%%%%%%%%%%%%%%%%%%%%%%%%%%%%%%%%%


\usepackage[french]{babel}

\usepackage[tmargin=2cm,rmargin=1in,lmargin=1in,margin=0.85in,bmargin=2cm,footskip=.2in]{geometry}
\usepackage{amsmath,amsfonts,amsthm,amssymb,mathtools}
\usepackage[varbb]{newpxmath}
\usepackage{xfrac}
\usepackage[makeroom]{cancel}
\usepackage{mathtools}
\usepackage{bookmark}
\usepackage{enumitem}
\usepackage{hyperref,theoremref}
\hypersetup{
	pdftitle={Assignment},
	colorlinks=true, linkcolor=doc!90,
	bookmarksnumbered=true,
	bookmarksopen=true
}
\usepackage[most,many,breakable]{tcolorbox}
\usepackage{xcolor}
\usepackage{varwidth}
\usepackage{varwidth}
\usepackage{etoolbox}
%\usepackage{authblk}
\usepackage{nameref}
\usepackage{multicol,array}
\usepackage{tikz-cd}
\usepackage[ruled,vlined,linesnumbered]{algorithm2e}
\usepackage{comment} % enables the use of multi-line comments (\ifx \fi) 
\usepackage{import}
\usepackage{xifthen}
\usepackage{pdfpages}
\usepackage{transparent}


\newcommand\mycommfont[1]{\footnotesize\ttfamily\textcolor{blue}{#1}}
\SetCommentSty{mycommfont}
\newcommand{\incfig}[1]{%
    \def\svgwidth{\columnwidth}
    \import{./figures/}{#1.pdf_tex}
}

\usepackage{tikzsymbols}
%\renewcommand\qedsymbol{$\Laughey$}


%\usepackage{import}
%\usepackage{xifthen}
%\usepackage{pdfpages}
%\usepackage{transparent}


%%%%%%%%%%%%%%%%%%%%%%%%%%%%%%
% SELF MADE COLORS
%%%%%%%%%%%%%%%%%%%%%%%%%%%%%%



\definecolor{myg}{RGB}{56, 140, 70}
\definecolor{myb}{RGB}{45, 111, 177}
\definecolor{myr}{RGB}{199, 68, 64}
\definecolor{mytheorembg}{HTML}{F2F2F9}
\definecolor{mytheoremfr}{HTML}{00007B}
\definecolor{mylenmabg}{HTML}{FFFAF8}
\definecolor{mylenmafr}{HTML}{983b0f}
\definecolor{mypropbg}{HTML}{f2fbfc}
\definecolor{mypropfr}{HTML}{191971}
\definecolor{myexamplebg}{HTML}{F2FBF8}
\definecolor{myexamplefr}{HTML}{88D6D1}
\definecolor{myexampleti}{HTML}{2A7F7F}
\definecolor{mydefinitbg}{HTML}{E5E5FF}
\definecolor{mydefinitfr}{HTML}{3F3FA3}
\definecolor{notesgreen}{RGB}{0,162,0}
\definecolor{myp}{RGB}{197, 92, 212}
\definecolor{mygr}{HTML}{2C3338}
\definecolor{myred}{RGB}{127,0,0}
\definecolor{myyellow}{RGB}{169,121,69}
\definecolor{myexercisebg}{HTML}{F2FBF8}
\definecolor{myexercisefg}{HTML}{88D6D1}


%%%%%%%%%%%%%%%%%%%%%%%%%%%%
% TCOLORBOX SETUPS
%%%%%%%%%%%%%%%%%%%%%%%%%%%%

\setlength{\parindent}{1cm}
%================================
% THEOREM BOX
%================================

\tcbuselibrary{theorems,skins,hooks}
\newtcbtheorem[number within=chapter]{Theorem}{Théorème}
{%
	enhanced,
	breakable,
	colback = mytheorembg,
	frame hidden,
	boxrule = 0sp,
	borderline west = {2pt}{0pt}{mytheoremfr},
	sharp corners,
	detach title,
	before upper = \tcbtitle\par\smallskip,
	coltitle = mytheoremfr,
	fonttitle = \bfseries\sffamily,
	description font = \mdseries,
	separator sign none,
	segmentation style={solid, mytheoremfr},
}
{th}


\tcbuselibrary{theorems,skins,hooks}
\newtcolorbox{Theoremcon}
{%
	enhanced
	,breakable
	,colback = mytheorembg
	,frame hidden
	,boxrule = 0sp
	,borderline west = {2pt}{0pt}{mytheoremfr}
	,sharp corners
	,description font = \mdseries
	,separator sign none
}

%================================
% Corollery
%================================
\tcbuselibrary{theorems,skins,hooks}
\newtcbtheorem[use counter=tcb@cnt@Theorem]{Corollary}{Corollaire}
{%
	enhanced
	,breakable
	,colback = myp!10
	,frame hidden
	,boxrule = 0sp
	,borderline west = {2pt}{0pt}{myp!85!black}
	,sharp corners
	,detach title
	,before upper = \tcbtitle\par\smallskip
	,coltitle = myp!85!black
	,fonttitle = \bfseries\sffamily
	,description font = \mdseries
	,separator sign none
	,segmentation style={solid, myp!85!black}
}
{th}

%================================
% LENMA
%================================

\tcbuselibrary{theorems,skins,hooks}
\newtcbtheorem[use counter=tcb@cnt@Theorem]{Lemma}{Lemme}
{%
	enhanced,
	breakable,
	colback = mylenmabg,
	frame hidden,
	boxrule = 0sp,
	borderline west = {2pt}{0pt}{mylenmafr},
	sharp corners,
	detach title,
	before upper = \tcbtitle\par\smallskip,
	coltitle = mylenmafr,
	fonttitle = \bfseries\sffamily,
	description font = \mdseries,
	separator sign none,
	segmentation style={solid, mylenmafr},
}
{th}


%================================
% PROPOSITION
%================================

\tcbuselibrary{theorems,skins,hooks}
\newtcbtheorem[use counter=tcb@cnt@Theorem]{Prop}{Proposition}
{%
	enhanced,
	breakable,
	colback = mypropbg,
	frame hidden,
	boxrule = 0sp,
	borderline west = {2pt}{0pt}{mypropfr},
	sharp corners,
	detach title,
	before upper = \tcbtitle\par\smallskip,
	coltitle = mypropfr,
	fonttitle = \bfseries\sffamily,
	description font = \mdseries,
	separator sign none,
	segmentation style={solid, mypropfr},
}
{th}


%================================
% CLAIM
%================================

\tcbuselibrary{theorems,skins,hooks}
\newtcbtheorem[use counter=tcb@cnt@Theorem]{claim}{Claim}
{%
	enhanced
	,breakable
	,colback = myg!10
	,frame hidden
	,boxrule = 0sp
	,borderline west = {2pt}{0pt}{myg}
	,sharp corners
	,detach title
	,before upper = \tcbtitle\par\smallskip
	,coltitle = myg!85!black
	,fonttitle = \bfseries\sffamily
	,description font = \mdseries
	,separator sign none
	,segmentation style={solid, myg!85!black}
}
{th}



%================================
% Exercise
%================================

\tcbuselibrary{theorems,skins,hooks}
\newtcbtheorem[use counter=tcb@cnt@Theorem]{Exercise}{Exercice}
{%
	enhanced,
	breakable,
	colback = myexercisebg,
	frame hidden,
	boxrule = 0sp,
	borderline west = {2pt}{0pt}{myexercisefg},
	sharp corners,
	detach title,
	before upper = \tcbtitle\par\smallskip,
	coltitle = myexercisefg,
	fonttitle = \bfseries\sffamily,
	description font = \mdseries,
	separator sign none,
	segmentation style={solid, myexercisefg},
}
{th}

%================================
% EXAMPLE BOX
%================================

\newtcbtheorem[use counter=tcb@cnt@Theorem]{Example}{Exemple}
{%
	colback = myexamplebg
	,breakable
	,colframe = myexamplefr
	,coltitle = myexampleti
	,boxrule = 1pt
	,sharp corners
	,detach title
	,before upper=\tcbtitle\par\smallskip
	,fonttitle = \bfseries
	,description font = \mdseries
	,separator sign none
	,description delimiters parenthesis
}
{ex}

%================================
% DEFINITION BOX
%================================

\newtcbtheorem[use counter=tcb@cnt@Theorem]{Definition}{Définition}{enhanced,
	before skip=2mm,after skip=2mm, colback=red!5,colframe=red!80!black,boxrule=0.5mm,
	attach boxed title to top left={xshift=1cm,yshift*=1mm-\tcboxedtitleheight}, varwidth boxed title*=-3cm,
	boxed title style={frame code={
					\path[fill=tcbcolback]
					([yshift=-1mm,xshift=-1mm]frame.north west)
					arc[start angle=0,end angle=180,radius=1mm]
					([yshift=-1mm,xshift=1mm]frame.north east)
					arc[start angle=180,end angle=0,radius=1mm];
					\path[left color=tcbcolback!60!black,right color=tcbcolback!60!black,
						middle color=tcbcolback!80!black]
					([xshift=-2mm]frame.north west) -- ([xshift=2mm]frame.north east)
					[rounded corners=1mm]-- ([xshift=1mm,yshift=-1mm]frame.north east)
					-- (frame.south east) -- (frame.south west)
					-- ([xshift=-1mm,yshift=-1mm]frame.north west)
					[sharp corners]-- cycle;
				},interior engine=empty,
		},
	fonttitle=\bfseries,
	title={#2},#1}{def}

%================================
% Solution BOX
%================================

\makeatletter
\newtcbtheorem[use counter=tcb@cnt@Theorem]{question}{Question}{enhanced,
	breakable,
	colback=white,
	colframe=myb!80!black,
	attach boxed title to top left={yshift*=-\tcboxedtitleheight},
	fonttitle=\bfseries,
	title={#2},
	boxed title size=title,
	boxed title style={%
			sharp corners,
			rounded corners=northwest,
			colback=tcbcolframe,
			boxrule=0pt,
		},
	underlay boxed title={%
			\path[fill=tcbcolframe] (title.south west)--(title.south east)
			to[out=0, in=180] ([xshift=5mm]title.east)--
			(title.center-|frame.east)
			[rounded corners=\kvtcb@arc] |-
			(frame.north) -| cycle;
		},
	#1
}{def}
\makeatother

%================================
% SOLUTION BOX
%================================

\makeatletter
\newtcolorbox{solution}{enhanced,
	breakable,
	colback=white,
	colframe=myg!80!black,
	attach boxed title to top left={yshift*=-\tcboxedtitleheight},
	title=Solution,
	boxed title size=title,
	boxed title style={%
			sharp corners,
			rounded corners=northwest,
			colback=tcbcolframe,
			boxrule=0pt,
		},
	underlay boxed title={%
			\path[fill=tcbcolframe] (title.south west)--(title.south east)
			to[out=0, in=180] ([xshift=5mm]title.east)--
			(title.center-|frame.east)
			[rounded corners=\kvtcb@arc] |-
			(frame.north) -| cycle;
		},
}
\makeatother

%================================
% Question BOX
%================================

\makeatletter
\newtcbtheorem[use counter=tcb@cnt@Theorem]{qstion}{Question}{enhanced,
	breakable,
	colback=white,
	colframe=mygr,
	attach boxed title to top left={yshift*=-\tcboxedtitleheight},
	fonttitle=\bfseries,
	title={#2},
	boxed title size=title,
	boxed title style={%
			sharp corners,
			rounded corners=northwest,
			colback=tcbcolframe,
			boxrule=0pt,
		},
	underlay boxed title={%
			\path[fill=tcbcolframe] (title.south west)--(title.south east)
			to[out=0, in=180] ([xshift=5mm]title.east)--
			(title.center-|frame.east)
			[rounded corners=\kvtcb@arc] |-
			(frame.north) -| cycle;
		},
	#1
}{def}
\makeatother

\newtcbtheorem[number within=chapter]{wconc}{Wrong Concept}{
	breakable,
	enhanced,
	colback=white,
	colframe=myr,
	arc=0pt,
	outer arc=0pt,
	fonttitle=\bfseries\sffamily\large,
	colbacktitle=myr,
	attach boxed title to top left={},
	boxed title style={
			enhanced,
			skin=enhancedfirst jigsaw,
			arc=3pt,
			bottom=0pt,
			interior style={fill=myr}
		},
	#1
}{def}



%================================
% NOTE BOX
%================================

\usetikzlibrary{arrows,calc,shadows.blur}
\tcbuselibrary{skins}
\newtcolorbox{note}[1][]{%
	enhanced jigsaw,
	colback=gray!20!white,%
	colframe=gray!80!black,
	size=small,
	boxrule=1pt,
	title=\colorbox{white!100}{\textbf{ Remarque }},
	halign title=flush center,
	coltitle=black,
	breakable,
	drop shadow=black!50!white,
	attach boxed title to top left={xshift=1cm,yshift=-\tcboxedtitleheight/2,yshifttext=-\tcboxedtitleheight/2},
	minipage boxed title=2.6cm,
	boxed title style={%
			colback=white,
			size=fbox,
			boxrule=1pt,
			boxsep=2pt,
			underlay={%
					\coordinate (dotA) at ($(interior.west) + (-0.5pt,0)$);
					\coordinate (dotB) at ($(interior.east) + (0.5pt,0)$);
					\begin{scope}
						\clip (interior.north west) rectangle ([xshift=3ex]interior.east);
						\filldraw [white, blur shadow={shadow opacity=60, shadow yshift=-.75ex}, rounded corners=2pt] (interior.north west) rectangle (interior.south east);
					\end{scope}
					\begin{scope}[gray!80!black]
						\fill (dotA) circle (2pt);
						\fill (dotB) circle (2pt);
					\end{scope}
				},
		},
	#1,
}

%================================
% STRATÉGIE BOX
%================================

\usetikzlibrary{arrows,calc,shadows.blur}
\tcbuselibrary{skins}
\newtcolorbox{strategy}[1][]{%
	enhanced jigsaw,
	colback=myb!20!white,%
	colframe=gray!80!black,
	size=small,
	boxrule=1pt,
	title=\colorbox{white!100}{\textbf{ Stratégie }},
	halign title=flush center,
	coltitle=black,
	breakable,
	drop shadow=black!50!white,
	attach boxed title to top left={xshift=1cm,yshift=-\tcboxedtitleheight/2,yshifttext=-\tcboxedtitleheight/2},
	minipage boxed title=2.5cm,
	boxed title style={%
			colback=white,
			size=fbox,
			boxrule=1pt,
			boxsep=2pt,
			underlay={%
					\coordinate (dotA) at ($(interior.west) + (-0.5pt,0)$);
					\coordinate (dotB) at ($(interior.east) + (0.5pt,0)$);
					\begin{scope}
						\clip (interior.north west) rectangle ([xshift=3ex]interior.east);
						\filldraw [white, blur shadow={shadow opacity=60, shadow yshift=-.75ex}, rounded corners=2pt] (interior.north west) rectangle (interior.south east);
					\end{scope}
					\begin{scope}[gray!80!black]
						\fill (dotA) circle (2pt);
						\fill (dotB) circle (2pt);
					\end{scope}
				},
		},
	#1,
}

%================================
% MÉTHODE BOX
%================================

\usetikzlibrary{arrows,calc,shadows.blur}
\tcbuselibrary{skins}
\newtcolorbox{methode}[1][]{%
	enhanced jigsaw,
	colback=white,%
	colframe=gray!80!black,
	size=small,
	boxrule=1pt,
	title=\textbf{Méthode},
	halign title=flush center,
	coltitle=black,
	breakable,
	drop shadow=black!50!white,
	attach boxed title to top left={xshift=1cm,yshift=-\tcboxedtitleheight/2,yshifttext=-\tcboxedtitleheight/2},
	minipage boxed title=2.5cm,
	boxed title style={%
			colback=white,
			size=fbox,
			boxrule=1pt,
			boxsep=2pt,
			underlay={%
					\coordinate (dotA) at ($(interior.west) + (-0.5pt,0)$);
					\coordinate (dotB) at ($(interior.east) + (0.5pt,0)$);
					\begin{scope}
						\clip (interior.north west) rectangle ([xshift=3ex]interior.east);
						\filldraw [white, blur shadow={shadow opacity=60, shadow yshift=-.75ex}, rounded corners=2pt] (interior.north west) rectangle (interior.south east);
					\end{scope}
					\begin{scope}[gray!80!black]
						\fill (dotA) circle (2pt);
						\fill (dotB) circle (2pt);
					\end{scope}
				},
		},
	#1,
}

%%%%%%%%%%%%%%%%%%%%%%%%%%%%%%%%%%%%%%%%%%%
% TABLE OF CONTENTS
%%%%%%%%%%%%%%%%%%%%%%%%%%%%%%%%%%%%%%%%%%%

\usepackage{tikz}

\definecolor{doc}{RGB}{0,60,110}
\usepackage{titletoc}
\contentsmargin{0cm}
\titlecontents{chapter}[3.7pc]
{\addvspace{30pt}%
	\begin{tikzpicture}[remember picture, overlay]%
		\draw[fill=doc!60,draw=doc!60] (-7,-.1) rectangle (-0.2,.6);%
		\pgftext[left,x=-3.5cm,y=0.2cm]{\color{white}\Large\sc\bfseries Chapitre\ \thecontentslabel};%
	\end{tikzpicture}\color{doc!60}\large\sc\bfseries}%
{}
{}
{\;\titlerule\;\large\sc\bfseries Page \thecontentspage
	\begin{tikzpicture}[remember picture, overlay]
		\draw[fill=doc!60,draw=doc!60] (2pt,0) rectangle (4,0.1pt);
	\end{tikzpicture}}%
\titlecontents{section}[3.7pc]
{\addvspace{2pt}}
{\contentslabel[\thecontentslabel]{2pc}}
{}
{\hfill\small \thecontentspage}
[]
\titlecontents*{subsection}[3.7pc]
{\addvspace{-1pt}\small}
{}
{}
{\ --- \small\thecontentspage}
[ \textbullet\ ][]

\makeatletter
\renewcommand{\tableofcontents}{%
	\chapter*{%
	  \vspace*{-20\p@}%
	  \begin{tikzpicture}[remember picture, overlay]%
		  \pgftext[right,x=15cm,y=0.2cm]{\color{doc!60}\Huge\sc\bfseries \contentsname};%
		  \draw[fill=doc!60,draw=doc!60] (13,-.75) rectangle (20,1);%
		  \clip (13,-.75) rectangle (20,1);
		  \pgftext[right,x=15cm,y=0.2cm]{\color{white}\Huge\sc\bfseries \contentsname};%
	  \end{tikzpicture}}%
	\@starttoc{toc}}
\makeatother


%%%%%%%%%%%%%%%%%%%%%%%%%%%%%%%%%%%%%%%%%%%
% MINTED FOR PYTHON ALGORITHMS
%%%%%%%%%%%%%%%%%%%%%%%%%%%%%%%%%%%%%%%%%%%

\usepackage{tcolorbox}
\tcbuselibrary{minted,breakable,xparse,skins}
\definecolor{bg}{gray}{0.95}
\DeclareTCBListing{mintedbox}{O{}m!O{}}{%
  breakable=true,
  listing engine=minted,
  listing only,
  minted language=#2,
  minted style=default,
  minted options={%
    linenos,
    gobble=0,
    breaklines=true,
    breakafter=,,
    fontsize=\small,
    numbersep=8pt,
    #1},
  boxsep=0pt,
  left skip=0pt,
  right skip=0pt,
  left=25pt,
  right=0pt,
  top=3pt,
  bottom=3pt,
  arc=5pt,
  leftrule=0pt,
  rightrule=0pt,
  bottomrule=2pt,
  toprule=2pt,
  colback=bg,
  colframe=orange!70,
  enhanced,
  overlay={%
    \begin{tcbclipinterior}
    \fill[orange!20!white] (frame.south west) rectangle ([xshift=20pt]frame.north west);
    \end{tcbclipinterior}},
  #3}
  
  
 % for braces
\usetikzlibrary{decorations.pathreplacing}


\AdvanceDate[0]

\begin{document}
\pagestyle{fancy}
\fancyhead[L]{Seconde 13}
\fancyhead[C]{\textbf{Signes et variations 2 \ifsolutions \\ Solutions  \fi}}
\fancyhead[R]{\today}


\exe{
	Pour chaque propriété, donner une fonction $f$ sur $\R$ non identiquement nulle la vérifiant.
	\begin{enumerate}
		\item $f$ s'annule en $1$.
		\item $f$ s'annule en $-10$.
		\item $f$ s'annule en $0$.
		\item $f$ s'annule en $-10$ et en $1$.
		\item Les racines de $f$ sont $2, -3, \dfrac27$, et $0$.
	\end{enumerate}
}{
	La fonction identiquement nulle $f(x) = 0$ n'est bien sûr pas autorisée sinon l'exercice n'a pas d'intérêt !
	\begin{enumerate}
		\item $f(x) = x-1$ fonctionne ainsi que $f(x) = 3(x-1)$, ou $f(x) = -(x-1) = 1-x$.
		\item $f(x) = x+10$ fonctionne, ainsi que $f(x) = (x+10)(x-1)$, et tous ses multiples.
		\item $f(x) = x$ ou $f(x) = 500x$.
		\item $f(x) = (x+10)(x-1) = x^2 +9x -10$.
		\item $f(x) = (x-2)(x+3)(x-\frac27)x$.
	\end{enumerate}

}

\exe{
	On souhaite connaître les solutions de l'équation du deuxième degré d'inconnue $x\in\R$ :
		\begin{align}
			22x^2 - 125x + 22  = 0. \label{eq:1}
		\end{align}
	\begin{enumerate}
		\item Montrer que $22x^2 - 125x + 22 = (2x-11)(11x-2)$.
		\item En déduire l'ensemble des solutions de l'équation \eqref{eq:1}.
	\end{enumerate}
}{
	\begin{enumerate}
		\item On part toujours de la forme factorisée qu'on développe par double distributivité :
			\[ (2x - 11)(11x - 2) = 22x^2 - 4x - 121x + 22 = 22x^2 - 125x + 22. \]
		\item On utilise que le produit est nul si est seulement si un des deux facteurs est nul.
		L'ensemble des solutions de \eqref{eq:1} est donc $\left\{ \dfrac{11}2 ; \dfrac2{11} \right\}$.
	\end{enumerate}
}

\exe{
	On souhaite connaître les solutions de l'équation du troisième degré d'inconnue $x\in\R$ :
		\begin{align}
			4x^3 - 6x^2 - 2x + 3  = 0. \label{eq:2}
		\end{align}
	\begin{enumerate}
		\item Montrer que $4x^3 - 6x^2 - 2x + 3 = (2x-3) \left(2x^2-1\right)$.
		\item En déduire l'ensemble des solutions de l'équation \eqref{eq:2}.
	\end{enumerate}
}{

	\begin{enumerate}
		\item On part toujours de la forme factorisée qu'on développe par double distributivité.
		On utilise aussi que $x^2 \cdot x = x \cdot x \cdot x = x^3$.
			\[ (2x - 3)(2x^2 - 1) = 4x^3 - 2x -6x^2 + 3. \]
		\item On utilise que le produit est nul si est seulement si un des deux facteurs est nul.
		Or l'équation $2x^2 - 1 = 0$ est équivalente à $x^2 = \dfrac12$.
		En prenant la racine et en utilisant que $\sqrt{x^2} = |x|$, on obtient
			\[ |x| = \sqrt{\dfrac12} = \dfrac1{\sqrt2}. \]
		Les deux valeurs $\pm \dfrac1{\sqrt2}$ (notation $\pm$ pour \og plus ou moins \fg) sont donc solutions.
		On en déduit que l'ensemble des solutions de \eqref{eq:2} est $\left\{ \dfrac32 ; \pm \dfrac1{\sqrt2} \right\} = \left\{ \dfrac32 ; \dfrac1{\sqrt2} ; - \dfrac1{\sqrt2} \right\}$
	\end{enumerate}


}

\exe{
	On souhaite connaître les solutions de l'équation du quatrième degré d'inconnue $x\in\R$ :
		\begin{align}
			16x^4 - 24x^2 + 9  = 0. \label{eq:3}
		\end{align}
	\begin{enumerate}
		\item Montrer que $16x^4 - 24x^2 + 9 = \left(4x^2-3\right)^2$.
		\item En déduire l'ensemble des solutions de l'équation \eqref{eq:3}.
	\end{enumerate}
}{

	\begin{enumerate}
		\item On part toujours de la forme factorisée qu'on développe à l'aide de l'identité remarquable $(a-b)^2 = a^2 + b^2 -2ab$.
		On utilise aussi que $(x^2)^2 = x^2 \cdot x^2 = x\cdot x\cdot x\cdot x = x^4$.
			\[ (4x^2 - 3)^2 = (4x^2)^2 + 3^2 - 2\cdot(4x^2)\cdot3 = 16x^4 + 9 -24x^2. \]
		\item On utilise que le produit est nul si est seulement si un des deux facteurs est nul.
		Ici, les deux facteurs sont les mêmes, car $(4x^2 - 3)^2 = (4x^2 - 3)(4x^2 - 3)$, donc on se remet à résoudre $(4x^2 - 3) = 0$.
		Ceci est équivalent à $x^2 = \dfrac34$, et donc l'ensemble des $x$ vérifiant \eqref{eq:3} est donné par $\left\{ \pm \sqrt{\dfrac34} \right\} = \left\{ \pm \dfrac12\sqrt{3} \right\}= \left\{ \dfrac12\sqrt{3} ;  -\dfrac12\sqrt{3} \right\}$.
	\end{enumerate}

}

\hrule


\ex{
	\begin{enumerate}
		\item Esquisser la courbe représentative de la fonction affine $f(x) =3x - 2$ sur le domaine $\D = [-10 ; 12]$.
		\item Remplir le tableau de variations et de signes ci-dessous.
	\end{enumerate}
	
	\begin{center}
	\begin{tikzpicture}
		\tkzTabInit
		 %[lgt=3,espcl=1.5]
	       		{$x$ / 1 , Variation de $3x-2$ / 2, Signe de $3x-2$ / 1}
	       		{ \ifsolutions $-10$ \fi,, \ifsolutions $\dfrac23$ \fi,, \ifsolutions $12$ \fi}
	       	
	       	\ifsolutions
		\tkzTabVar
			{-/$-32$, R/, R/, R/, +/$34$}
		\tkzTabIma
			{1}{5}{3}{0}
		\tkzTabLine
			{,,-,,z,,+}
		\fi
	\end{tikzpicture}
	\end{center}
}{}


\ex{
	\begin{enumerate}
		\item Esquisser la courbe représentative de la fonction affine $f(x) = -6x - 2$ sur le domaine $\D = [-10 ; 8]$.
		\item Remplir le tableau de variations et de signes ci-dessous.
	\end{enumerate}
	
	\begin{center}
	\begin{tikzpicture}
		\tkzTabInit
		 %[lgt=3,espcl=1.5]
	       		{$x$ / 1 , Variation de $-6x-2$ / 2, Signe de $-6x-2$ / 1}
	       		{ \ifsolutions $-10$ \fi,, \ifsolutions $-\dfrac13$ \fi,, \ifsolutions $8$ \fi}
	       	
	       	\ifsolutions
		\tkzTabVar
			{+/$58$, R/, R/, R/, -/$-50$}
		\tkzTabIma
			{1}{5}{3}{0}
		\tkzTabLine
			{,,+,,z,,-}
		\fi
	\end{tikzpicture}
	\end{center}
}{}


\ex{
	\begin{enumerate}
		\item Esquisser la courbe représentative de la fonction affine $f(x) = -3$ sur le domaine $\D = [-12 ; -5]$.
		\item Remplir le tableau de variations et de signes ci-dessous.
	\end{enumerate}
	
	\begin{center}
	\begin{tikzpicture}
		\tkzTabInit
		 %[lgt=3,espcl=1.5]
	       		{$x$ / 1 , Variation de $-3$ / 2, Signe de $-3$ / 1}
	       		{ \ifsolutions $-12$ \fi,,,, \ifsolutions $-5$ \fi}
	       	
	       	\ifsolutions
		\tkzTabVar
			{+/$-3$, R/, R/, R/,+/$-3$}
		\tkzTabLine
			{,,,,-,,}
		\fi
	\end{tikzpicture}
	\end{center}
}{}

\ex{
	Remplir approximativement les tableaux ci-dessous à l'aide des graphes de $\C_f$ et $\C_g$ sur le domaine $\D = [-11; -2,5]$.
	
	\begin{center}
	\begin{tikzpicture}[scale=1]
		\begin{axis}[xmin = -11, xmax=-2.5, ymin=-4.3, ymax=3.5, axis x line=middle, axis y line=middle, axis line style=->, grid=both, ytick={-4,-3,...,2,3}, xtick={-11, -10,...,-4,-3},
    every y tick label/.style={
        anchor=near yticklabel opposite,
        xshift=0.2em,
    }]
    			%f de l'éval f° affines
			\addplot[no marks, myb, -, very thick] expression[domain=-11:-8, samples=50]{5*(x+10)*(x^3 - 300*x - 1920)/80} 
			node[pos=.6, right]{$\mathcal{C}_f$};
			\addplot[no marks, myb, -, very thick] expression[domain=-8:-6, samples=2]{5*(x/2 + 3.2)};
			\addplot[no marks, myb, -, very thick] expression[domain=-6:-5, samples=50]{5*((x+5.5)^2 - 0.5^2 + 0.2)};
			\addplot[no marks, myb, -, very thick] expression[domain=-5:-2.5, samples=2]{5*0.2};
			
			% g sinus
			\addplot[no marks, myr, -, very thick] expression[domain=-11:-2.5, samples=500]{2*sin(x*60)}
			node[pos=.75, above]{$\mathcal{C}_g$};
		\end{axis}
	\end{tikzpicture}
	
	\begin{tikzpicture}
		\tkzTabInit
		 %[lgt=3,espcl=1.5]
	       		{$x$ / 1 , Signe de $f(x)$ / 1}
	       		{\ifsolutions $-11$ \fi,\ifsolutions $-9$ \fi,\ifsolutions $-6$ \fi,\ifsolutions $-3$ \fi, \ifsolutions {$-2,5$} \fi}
	       	
	       	\ifsolutions
		\tkzTabLine
			{,+,z,-,z,+,z,-}
		\fi
	\end{tikzpicture}
	\vfill
	\begin{tikzpicture}
		\tkzTabInit
		[lgt=3,espcl=3]
	       		{$x$ / 1 , Variation de $f(x)$ / 2}
	       		{\ifsolutions $-11$ \fi,\ifsolutions ${-10,5}$ \fi,\ifsolutions {$-7,5$} \fi,\ifsolutions {$-4,5$} \fi, \ifsolutions {$-2,5$} \fi}
	       	
	       	\ifsolutions
		\tkzTabVar
			{-/{$1,8$},+/{$2$}, -/{$-2$}, +/{$2$}, -/{$-1$}}
		\fi
	\end{tikzpicture}
	\vfill
	\begin{tikzpicture}
		\tkzTabInit
		 [lgt=3,espcl=2]
	       		{$x$ / 1 , Signe de $g(x)$ / 1}
	       		{\ifsolutions $-11$ \fi,\ifsolutions $-10$ \fi,\ifsolutions {$-8,4$} \fi,\ifsolutions {$-6,5$} \fi, \ifsolutions {$-5,8$} \fi, \ifsolutions {$-5,3$} \fi, \ifsolutions {$-2,5$} \fi}
	       	
	       	\ifsolutions
		\tkzTabLine
			{,-,z,+,z,-,z,+,z,-,z,+}
		\fi
	\end{tikzpicture}
	\vfill
	\begin{tikzpicture}
		\tkzTabInit
		 [lgt=3,espcl=2]
	       		{$x$ / 1 , Variation de $g(x)$ / 2}
	       		{\ifsolutions $-11$ \fi,\ifsolutions ${-9,1}$ \fi,\ifsolutions {$-8$} \fi,\ifsolutions {$-6$} \fi, \ifsolutions {$-5,5$} \fi, \ifsolutions {$-5$} \fi, \ifsolutions {$-2,5$} \fi}
	       	
	       	\ifsolutions
		\tkzTabVar
			{-/$-3$, +/${3,1}$, -/$-4$, +/$1$, -/${-0,2}$, +/$1$, +/$1$}
		\fi
	\end{tikzpicture}
	\end{center}
	
}{}


\end{document}

				% ENABLE or DISABLE font change
				% use XeLaTeX if true
\newif\ifdys
				\dystrue
				\dysfalse

\newif\ifsolutions
				\solutionstrue
				%\solutionsfalse

%!TEX encoding = UTF8
%!TEX root =notes.tex


%%%%%%%%%%%%%%%%%%%%%%%%%%%%%%%%%
% PACKAGE IMPORTS
%%%%%%%%%%%%%%%%%%%%%%%%%%%%%%%%%


\usepackage[french]{babel}

\usepackage[tmargin=2cm,rmargin=1in,lmargin=1in,margin=0.85in,bmargin=2cm,footskip=.2in]{geometry}
\usepackage{amsmath,amsfonts,amsthm,amssymb,mathtools}
\usepackage[varbb]{newpxmath}
\usepackage{xfrac}
\usepackage[makeroom]{cancel}
\usepackage{mathtools}
\usepackage{bookmark}
\usepackage{enumitem}
\usepackage{hyperref,theoremref}
\hypersetup{
	pdftitle={Assignment},
	colorlinks=true, linkcolor=doc!90,
	bookmarksnumbered=true,
	bookmarksopen=true
}
\usepackage[most,many,breakable]{tcolorbox}
\usepackage{xcolor}
\usepackage{varwidth}
\usepackage{varwidth}
\usepackage{etoolbox}
%\usepackage{authblk}
\usepackage{nameref}
\usepackage{multicol,array}
\usepackage{tikz-cd}
\usepackage[ruled,vlined,linesnumbered]{algorithm2e}
\usepackage{comment} % enables the use of multi-line comments (\ifx \fi) 
\usepackage{import}
\usepackage{xifthen}
\usepackage{pdfpages}
\usepackage{transparent}


\newcommand\mycommfont[1]{\footnotesize\ttfamily\textcolor{blue}{#1}}
\SetCommentSty{mycommfont}
\newcommand{\incfig}[1]{%
    \def\svgwidth{\columnwidth}
    \import{./figures/}{#1.pdf_tex}
}

\usepackage{tikzsymbols}
%\renewcommand\qedsymbol{$\Laughey$}


%\usepackage{import}
%\usepackage{xifthen}
%\usepackage{pdfpages}
%\usepackage{transparent}


%%%%%%%%%%%%%%%%%%%%%%%%%%%%%%
% SELF MADE COLORS
%%%%%%%%%%%%%%%%%%%%%%%%%%%%%%



\definecolor{myg}{RGB}{56, 140, 70}
\definecolor{myb}{RGB}{45, 111, 177}
\definecolor{myr}{RGB}{199, 68, 64}
\definecolor{mytheorembg}{HTML}{F2F2F9}
\definecolor{mytheoremfr}{HTML}{00007B}
\definecolor{mylenmabg}{HTML}{FFFAF8}
\definecolor{mylenmafr}{HTML}{983b0f}
\definecolor{mypropbg}{HTML}{f2fbfc}
\definecolor{mypropfr}{HTML}{191971}
\definecolor{myexamplebg}{HTML}{F2FBF8}
\definecolor{myexamplefr}{HTML}{88D6D1}
\definecolor{myexampleti}{HTML}{2A7F7F}
\definecolor{mydefinitbg}{HTML}{E5E5FF}
\definecolor{mydefinitfr}{HTML}{3F3FA3}
\definecolor{notesgreen}{RGB}{0,162,0}
\definecolor{myp}{RGB}{197, 92, 212}
\definecolor{mygr}{HTML}{2C3338}
\definecolor{myred}{RGB}{127,0,0}
\definecolor{myyellow}{RGB}{169,121,69}
\definecolor{myexercisebg}{HTML}{F2FBF8}
\definecolor{myexercisefg}{HTML}{88D6D1}


%%%%%%%%%%%%%%%%%%%%%%%%%%%%
% TCOLORBOX SETUPS
%%%%%%%%%%%%%%%%%%%%%%%%%%%%

\setlength{\parindent}{1cm}
%================================
% THEOREM BOX
%================================

\tcbuselibrary{theorems,skins,hooks}
\newtcbtheorem[number within=chapter]{Theorem}{Théorème}
{%
	enhanced,
	breakable,
	colback = mytheorembg,
	frame hidden,
	boxrule = 0sp,
	borderline west = {2pt}{0pt}{mytheoremfr},
	sharp corners,
	detach title,
	before upper = \tcbtitle\par\smallskip,
	coltitle = mytheoremfr,
	fonttitle = \bfseries\sffamily,
	description font = \mdseries,
	separator sign none,
	segmentation style={solid, mytheoremfr},
}
{th}


\tcbuselibrary{theorems,skins,hooks}
\newtcolorbox{Theoremcon}
{%
	enhanced
	,breakable
	,colback = mytheorembg
	,frame hidden
	,boxrule = 0sp
	,borderline west = {2pt}{0pt}{mytheoremfr}
	,sharp corners
	,description font = \mdseries
	,separator sign none
}

%================================
% Corollery
%================================
\tcbuselibrary{theorems,skins,hooks}
\newtcbtheorem[use counter=tcb@cnt@Theorem]{Corollary}{Corollaire}
{%
	enhanced
	,breakable
	,colback = myp!10
	,frame hidden
	,boxrule = 0sp
	,borderline west = {2pt}{0pt}{myp!85!black}
	,sharp corners
	,detach title
	,before upper = \tcbtitle\par\smallskip
	,coltitle = myp!85!black
	,fonttitle = \bfseries\sffamily
	,description font = \mdseries
	,separator sign none
	,segmentation style={solid, myp!85!black}
}
{th}

%================================
% LENMA
%================================

\tcbuselibrary{theorems,skins,hooks}
\newtcbtheorem[use counter=tcb@cnt@Theorem]{Lemma}{Lemme}
{%
	enhanced,
	breakable,
	colback = mylenmabg,
	frame hidden,
	boxrule = 0sp,
	borderline west = {2pt}{0pt}{mylenmafr},
	sharp corners,
	detach title,
	before upper = \tcbtitle\par\smallskip,
	coltitle = mylenmafr,
	fonttitle = \bfseries\sffamily,
	description font = \mdseries,
	separator sign none,
	segmentation style={solid, mylenmafr},
}
{th}


%================================
% PROPOSITION
%================================

\tcbuselibrary{theorems,skins,hooks}
\newtcbtheorem[use counter=tcb@cnt@Theorem]{Prop}{Proposition}
{%
	enhanced,
	breakable,
	colback = mypropbg,
	frame hidden,
	boxrule = 0sp,
	borderline west = {2pt}{0pt}{mypropfr},
	sharp corners,
	detach title,
	before upper = \tcbtitle\par\smallskip,
	coltitle = mypropfr,
	fonttitle = \bfseries\sffamily,
	description font = \mdseries,
	separator sign none,
	segmentation style={solid, mypropfr},
}
{th}


%================================
% CLAIM
%================================

\tcbuselibrary{theorems,skins,hooks}
\newtcbtheorem[use counter=tcb@cnt@Theorem]{claim}{Claim}
{%
	enhanced
	,breakable
	,colback = myg!10
	,frame hidden
	,boxrule = 0sp
	,borderline west = {2pt}{0pt}{myg}
	,sharp corners
	,detach title
	,before upper = \tcbtitle\par\smallskip
	,coltitle = myg!85!black
	,fonttitle = \bfseries\sffamily
	,description font = \mdseries
	,separator sign none
	,segmentation style={solid, myg!85!black}
}
{th}



%================================
% Exercise
%================================

\tcbuselibrary{theorems,skins,hooks}
\newtcbtheorem[use counter=tcb@cnt@Theorem]{Exercise}{Exercice}
{%
	enhanced,
	breakable,
	colback = myexercisebg,
	frame hidden,
	boxrule = 0sp,
	borderline west = {2pt}{0pt}{myexercisefg},
	sharp corners,
	detach title,
	before upper = \tcbtitle\par\smallskip,
	coltitle = myexercisefg,
	fonttitle = \bfseries\sffamily,
	description font = \mdseries,
	separator sign none,
	segmentation style={solid, myexercisefg},
}
{th}

%================================
% EXAMPLE BOX
%================================

\newtcbtheorem[use counter=tcb@cnt@Theorem]{Example}{Exemple}
{%
	colback = myexamplebg
	,breakable
	,colframe = myexamplefr
	,coltitle = myexampleti
	,boxrule = 1pt
	,sharp corners
	,detach title
	,before upper=\tcbtitle\par\smallskip
	,fonttitle = \bfseries
	,description font = \mdseries
	,separator sign none
	,description delimiters parenthesis
}
{ex}

%================================
% DEFINITION BOX
%================================

\newtcbtheorem[use counter=tcb@cnt@Theorem]{Definition}{Définition}{enhanced,
	before skip=2mm,after skip=2mm, colback=red!5,colframe=red!80!black,boxrule=0.5mm,
	attach boxed title to top left={xshift=1cm,yshift*=1mm-\tcboxedtitleheight}, varwidth boxed title*=-3cm,
	boxed title style={frame code={
					\path[fill=tcbcolback]
					([yshift=-1mm,xshift=-1mm]frame.north west)
					arc[start angle=0,end angle=180,radius=1mm]
					([yshift=-1mm,xshift=1mm]frame.north east)
					arc[start angle=180,end angle=0,radius=1mm];
					\path[left color=tcbcolback!60!black,right color=tcbcolback!60!black,
						middle color=tcbcolback!80!black]
					([xshift=-2mm]frame.north west) -- ([xshift=2mm]frame.north east)
					[rounded corners=1mm]-- ([xshift=1mm,yshift=-1mm]frame.north east)
					-- (frame.south east) -- (frame.south west)
					-- ([xshift=-1mm,yshift=-1mm]frame.north west)
					[sharp corners]-- cycle;
				},interior engine=empty,
		},
	fonttitle=\bfseries,
	title={#2},#1}{def}

%================================
% Solution BOX
%================================

\makeatletter
\newtcbtheorem[use counter=tcb@cnt@Theorem]{question}{Question}{enhanced,
	breakable,
	colback=white,
	colframe=myb!80!black,
	attach boxed title to top left={yshift*=-\tcboxedtitleheight},
	fonttitle=\bfseries,
	title={#2},
	boxed title size=title,
	boxed title style={%
			sharp corners,
			rounded corners=northwest,
			colback=tcbcolframe,
			boxrule=0pt,
		},
	underlay boxed title={%
			\path[fill=tcbcolframe] (title.south west)--(title.south east)
			to[out=0, in=180] ([xshift=5mm]title.east)--
			(title.center-|frame.east)
			[rounded corners=\kvtcb@arc] |-
			(frame.north) -| cycle;
		},
	#1
}{def}
\makeatother

%================================
% SOLUTION BOX
%================================

\makeatletter
\newtcolorbox{solution}{enhanced,
	breakable,
	colback=white,
	colframe=myg!80!black,
	attach boxed title to top left={yshift*=-\tcboxedtitleheight},
	title=Solution,
	boxed title size=title,
	boxed title style={%
			sharp corners,
			rounded corners=northwest,
			colback=tcbcolframe,
			boxrule=0pt,
		},
	underlay boxed title={%
			\path[fill=tcbcolframe] (title.south west)--(title.south east)
			to[out=0, in=180] ([xshift=5mm]title.east)--
			(title.center-|frame.east)
			[rounded corners=\kvtcb@arc] |-
			(frame.north) -| cycle;
		},
}
\makeatother

%================================
% Question BOX
%================================

\makeatletter
\newtcbtheorem[use counter=tcb@cnt@Theorem]{qstion}{Question}{enhanced,
	breakable,
	colback=white,
	colframe=mygr,
	attach boxed title to top left={yshift*=-\tcboxedtitleheight},
	fonttitle=\bfseries,
	title={#2},
	boxed title size=title,
	boxed title style={%
			sharp corners,
			rounded corners=northwest,
			colback=tcbcolframe,
			boxrule=0pt,
		},
	underlay boxed title={%
			\path[fill=tcbcolframe] (title.south west)--(title.south east)
			to[out=0, in=180] ([xshift=5mm]title.east)--
			(title.center-|frame.east)
			[rounded corners=\kvtcb@arc] |-
			(frame.north) -| cycle;
		},
	#1
}{def}
\makeatother

\newtcbtheorem[number within=chapter]{wconc}{Wrong Concept}{
	breakable,
	enhanced,
	colback=white,
	colframe=myr,
	arc=0pt,
	outer arc=0pt,
	fonttitle=\bfseries\sffamily\large,
	colbacktitle=myr,
	attach boxed title to top left={},
	boxed title style={
			enhanced,
			skin=enhancedfirst jigsaw,
			arc=3pt,
			bottom=0pt,
			interior style={fill=myr}
		},
	#1
}{def}



%================================
% NOTE BOX
%================================

\usetikzlibrary{arrows,calc,shadows.blur}
\tcbuselibrary{skins}
\newtcolorbox{note}[1][]{%
	enhanced jigsaw,
	colback=gray!20!white,%
	colframe=gray!80!black,
	size=small,
	boxrule=1pt,
	title=\colorbox{white!100}{\textbf{ Remarque }},
	halign title=flush center,
	coltitle=black,
	breakable,
	drop shadow=black!50!white,
	attach boxed title to top left={xshift=1cm,yshift=-\tcboxedtitleheight/2,yshifttext=-\tcboxedtitleheight/2},
	minipage boxed title=2.6cm,
	boxed title style={%
			colback=white,
			size=fbox,
			boxrule=1pt,
			boxsep=2pt,
			underlay={%
					\coordinate (dotA) at ($(interior.west) + (-0.5pt,0)$);
					\coordinate (dotB) at ($(interior.east) + (0.5pt,0)$);
					\begin{scope}
						\clip (interior.north west) rectangle ([xshift=3ex]interior.east);
						\filldraw [white, blur shadow={shadow opacity=60, shadow yshift=-.75ex}, rounded corners=2pt] (interior.north west) rectangle (interior.south east);
					\end{scope}
					\begin{scope}[gray!80!black]
						\fill (dotA) circle (2pt);
						\fill (dotB) circle (2pt);
					\end{scope}
				},
		},
	#1,
}

%================================
% STRATÉGIE BOX
%================================

\usetikzlibrary{arrows,calc,shadows.blur}
\tcbuselibrary{skins}
\newtcolorbox{strategy}[1][]{%
	enhanced jigsaw,
	colback=myb!20!white,%
	colframe=gray!80!black,
	size=small,
	boxrule=1pt,
	title=\colorbox{white!100}{\textbf{ Stratégie }},
	halign title=flush center,
	coltitle=black,
	breakable,
	drop shadow=black!50!white,
	attach boxed title to top left={xshift=1cm,yshift=-\tcboxedtitleheight/2,yshifttext=-\tcboxedtitleheight/2},
	minipage boxed title=2.5cm,
	boxed title style={%
			colback=white,
			size=fbox,
			boxrule=1pt,
			boxsep=2pt,
			underlay={%
					\coordinate (dotA) at ($(interior.west) + (-0.5pt,0)$);
					\coordinate (dotB) at ($(interior.east) + (0.5pt,0)$);
					\begin{scope}
						\clip (interior.north west) rectangle ([xshift=3ex]interior.east);
						\filldraw [white, blur shadow={shadow opacity=60, shadow yshift=-.75ex}, rounded corners=2pt] (interior.north west) rectangle (interior.south east);
					\end{scope}
					\begin{scope}[gray!80!black]
						\fill (dotA) circle (2pt);
						\fill (dotB) circle (2pt);
					\end{scope}
				},
		},
	#1,
}

%================================
% MÉTHODE BOX
%================================

\usetikzlibrary{arrows,calc,shadows.blur}
\tcbuselibrary{skins}
\newtcolorbox{methode}[1][]{%
	enhanced jigsaw,
	colback=white,%
	colframe=gray!80!black,
	size=small,
	boxrule=1pt,
	title=\textbf{Méthode},
	halign title=flush center,
	coltitle=black,
	breakable,
	drop shadow=black!50!white,
	attach boxed title to top left={xshift=1cm,yshift=-\tcboxedtitleheight/2,yshifttext=-\tcboxedtitleheight/2},
	minipage boxed title=2.5cm,
	boxed title style={%
			colback=white,
			size=fbox,
			boxrule=1pt,
			boxsep=2pt,
			underlay={%
					\coordinate (dotA) at ($(interior.west) + (-0.5pt,0)$);
					\coordinate (dotB) at ($(interior.east) + (0.5pt,0)$);
					\begin{scope}
						\clip (interior.north west) rectangle ([xshift=3ex]interior.east);
						\filldraw [white, blur shadow={shadow opacity=60, shadow yshift=-.75ex}, rounded corners=2pt] (interior.north west) rectangle (interior.south east);
					\end{scope}
					\begin{scope}[gray!80!black]
						\fill (dotA) circle (2pt);
						\fill (dotB) circle (2pt);
					\end{scope}
				},
		},
	#1,
}

%%%%%%%%%%%%%%%%%%%%%%%%%%%%%%%%%%%%%%%%%%%
% TABLE OF CONTENTS
%%%%%%%%%%%%%%%%%%%%%%%%%%%%%%%%%%%%%%%%%%%

\usepackage{tikz}

\definecolor{doc}{RGB}{0,60,110}
\usepackage{titletoc}
\contentsmargin{0cm}
\titlecontents{chapter}[3.7pc]
{\addvspace{30pt}%
	\begin{tikzpicture}[remember picture, overlay]%
		\draw[fill=doc!60,draw=doc!60] (-7,-.1) rectangle (-0.2,.6);%
		\pgftext[left,x=-3.5cm,y=0.2cm]{\color{white}\Large\sc\bfseries Chapitre\ \thecontentslabel};%
	\end{tikzpicture}\color{doc!60}\large\sc\bfseries}%
{}
{}
{\;\titlerule\;\large\sc\bfseries Page \thecontentspage
	\begin{tikzpicture}[remember picture, overlay]
		\draw[fill=doc!60,draw=doc!60] (2pt,0) rectangle (4,0.1pt);
	\end{tikzpicture}}%
\titlecontents{section}[3.7pc]
{\addvspace{2pt}}
{\contentslabel[\thecontentslabel]{2pc}}
{}
{\hfill\small \thecontentspage}
[]
\titlecontents*{subsection}[3.7pc]
{\addvspace{-1pt}\small}
{}
{}
{\ --- \small\thecontentspage}
[ \textbullet\ ][]

\makeatletter
\renewcommand{\tableofcontents}{%
	\chapter*{%
	  \vspace*{-20\p@}%
	  \begin{tikzpicture}[remember picture, overlay]%
		  \pgftext[right,x=15cm,y=0.2cm]{\color{doc!60}\Huge\sc\bfseries \contentsname};%
		  \draw[fill=doc!60,draw=doc!60] (13,-.75) rectangle (20,1);%
		  \clip (13,-.75) rectangle (20,1);
		  \pgftext[right,x=15cm,y=0.2cm]{\color{white}\Huge\sc\bfseries \contentsname};%
	  \end{tikzpicture}}%
	\@starttoc{toc}}
\makeatother


%%%%%%%%%%%%%%%%%%%%%%%%%%%%%%%%%%%%%%%%%%%
% MINTED FOR PYTHON ALGORITHMS
%%%%%%%%%%%%%%%%%%%%%%%%%%%%%%%%%%%%%%%%%%%

\usepackage{tcolorbox}
\tcbuselibrary{minted,breakable,xparse,skins}
\definecolor{bg}{gray}{0.95}
\DeclareTCBListing{mintedbox}{O{}m!O{}}{%
  breakable=true,
  listing engine=minted,
  listing only,
  minted language=#2,
  minted style=default,
  minted options={%
    linenos,
    gobble=0,
    breaklines=true,
    breakafter=,,
    fontsize=\small,
    numbersep=8pt,
    #1},
  boxsep=0pt,
  left skip=0pt,
  right skip=0pt,
  left=25pt,
  right=0pt,
  top=3pt,
  bottom=3pt,
  arc=5pt,
  leftrule=0pt,
  rightrule=0pt,
  bottomrule=2pt,
  toprule=2pt,
  colback=bg,
  colframe=orange!70,
  enhanced,
  overlay={%
    \begin{tcbclipinterior}
    \fill[orange!20!white] (frame.south west) rectangle ([xshift=20pt]frame.north west);
    \end{tcbclipinterior}},
  #3}
  
  
 % for braces
\usetikzlibrary{decorations.pathreplacing}


\AdvanceDate[0]

\begin{document}
\pagestyle{fancy}
\fancyhead[L]{Seconde 13}
\fancyhead[C]{\textbf{Exercices d'entraînement --- Signes et variations \ifsolutions \\ Solutions  \fi}}
\fancyhead[R]{\today}


\exe{
	Donner le domaine de définition de chaque fonction suivante.
	\begin{multicols}{2}
	\begin{enumerate}
		%\item $f(x) = \sqrt{2x-1}$
		%\item $g(x) = \sqrt{-x-3}$
		%\item $F(x) = \dfrac1{4\left(x+\dfrac23\right)\left(x-\dfrac{23}7\right)}$
		\item $F(x) = \dfrac1{4\left(x+\dfrac23\right)}$
		%\item $G(x) = \dfrac{\sqrt{8-7x}}{(x+1)(x-124)}$
		\item $G(x) = \sqrt{8-7x}$
	\end{enumerate}
	\end{multicols}

}{
	\begin{enumerate}
		\item La seule opération éventuellement illégale est la division par zéro : il nous incombe donc d'empêcher que cela arrive.
		Or cela arrive lorsque
			\begin{align*}
				4\left(x+\dfrac23\right) = 0 && \iff && x = -\dfrac23
			\end{align*}
		On empèche donc la valeur $-\frac23$ en perforant $\R$ :
			\[ \D_f = \R - \left\{ -\dfrac23 \right\}. \]
			
		\item La seule opération éventuellement illégale est la prise de racine carrée d'un nombre négatif. 
		Il nous incombe donc d'empêcher que cela arrive.
		On impose donc que
			\begin{align*}
				8 - 7x &\geq 0 \\
				8 &\geq 7x \\
				x &\leq \dfrac87
			\end{align*}
		Il suit que $\D_f = \left]\minfty ; \dfrac87 \right]$.
		
		Remarques importantes : prendre la racine carrée de $0$ est légal, donc les inégalités sont larges, et la borne supérieure de $\D_f$ est incluse. 
		De plus, on aurait pû déterminer les valeurs interdites en résolvant $8-7x < 0$ et en écrivant $\D_f = \R - \left]\dfrac87 ; \pinfty \right[$, ce qui donne le même résultat !
	\end{enumerate}
}


\ex{
	Remplir approximativement les tableaux ci-dessous à l'aide des graphes de $\C_f$ et $\C_g$ sur le domaine $\D = [-3; -3]$.
	\begin{center}
	\begin{tikzpicture}[>=stealth, scale=1]
	\begin{axis}[xmin = -3, xmax=3, ymin=-4, ymax=4, axis x line=middle, axis y line=middle, axis line style=<->, xlabel={}, ylabel={}, xtick = {-4, -3, ..., 4}, ytick = {-4, -3, ..., 4}, grid=both]
		
		% (g)
		\addplot[myr, very thick, domain =-3:3, samples=100] {2*(x+1)*x*(x-2)+.5}  node[pos = .72, right=2pt] {$\C_g$};
		
		% (f)
		\addplot[myb, very thick, domain =-3:3, samples=50] {-2.9+1.2*x^2}  node[pos = .3, left] {$\C_f$};
	\end{axis}
	\end{tikzpicture}
	\end{center}
%	\end{multicols}

	\begin{center}
	\begin{tikzpicture}
		\tkzTabInit
		 %[lgt=3,espcl=1.5]
		[lgt=3,espcl=4]
	       		{$x$ / 1 , Signe de $f(x)$ / 1}
	       		{\ifsolutions $-3$ \fi,\ifsolutions ${-1,5}$ \fi,\ifsolutions ${1,5}$ \fi,\ifsolutions $3$ \fi}
	       	
	       	\ifsolutions
		\tkzTabLine
			{,+,z,-,z,+,}
		\fi
	\end{tikzpicture}
	\vfill
	\begin{tikzpicture}
		\tkzTabInit
		[lgt=3,espcl=6]
	       		{$x$ / 1 , Variation de $f(x)$ / 2}
	       		{\ifsolutions $-3$ \fi,\ifsolutions ${0}$ \fi,\ifsolutions {$3$} \fi}
	       	
	       	\ifsolutions
		\tkzTabVar
			{+/,-/${-2,9}$, +/}
		\fi
	\end{tikzpicture}
	\vfill
	\begin{tikzpicture}
		\tkzTabInit
		 [lgt=3,espcl=3]
	       		{$x$ / 1 , Signe de $g(x)$ / 1}
	       		{\ifsolutions $-3$ \fi,\ifsolutions {$-1,1$} \fi,\ifsolutions {$0,1$} \fi,\ifsolutions {$1,9$} \fi, \ifsolutions {$3$} \fi}
	       	
	       	\ifsolutions
		\tkzTabLine
			{,-,z,+,z,-,z,+,}
		\fi
	\end{tikzpicture}
	\vfill
	\begin{tikzpicture}
		\tkzTabInit
		 [lgt=3,espcl=4]
	       		{$x$ / 1 , Variation de $g(x)$ / 2}
	       		{\ifsolutions $-3$ \fi,\ifsolutions ${-0,6}$ \fi,\ifsolutions {$1,2$} \fi,\ifsolutions {$3$} \fi}
	       	
	       	\ifsolutions
		\tkzTabVar
			{-/,+/{$1,8$}, -/{$-3,6$}, +/}
		\fi
	\end{tikzpicture}
	\end{center}


}{}

\exe{
	On souhaite connaître pour quels $x\in\D_f$ l'inégalité suivante est vérifée.
		\[ f(x) = \dfrac{x+2}{6 - x - x^2} \geq 0 \]
		%\[ f(x) = \dfrac{x+2}{(x+3)(2-x)} \geq 0 \]
	
	\begin{enumerate}
		\item Montrer que $6 - x - x^2 = (x+3)(2-x)$.
		\item En déduire les valeurs interdites à $f$ et donc le domaine de définition $\D_f$.
		%\item Donner les valeurs interdites à $f$ et son domaine de définition $\D_f$.
		\item Remplir le tableau de signes au dos.
		\item Exprimer $\{ x \in \D_f \text{ tq. } f(x) \geq 0 \}$ sous forme d'union d'intervalles.
	\end{enumerate}
	
	
	\begin{center}
	\begin{tikzpicture}
		\tkzTabInit
		 [lgt=3,espcl=3]
	       		{ {\ifsolutions $x$ \fi}/ 1,  {\ifsolutions $x+2$ \fi}/ 1 ,  {\ifsolutions $x+3$ \fi}/ 1,  {\ifsolutions $2-x$ \fi}/ 1 ,  {\ifsolutions $f(x)$ \fi}/ 1}
	       		{{\ifsolutions $\minfty$ \fi}, {\ifsolutions $-3$ \fi}, {\ifsolutions $-2$ \fi},{\ifsolutions $2$ \fi}, {\ifsolutions $\pinfty$ \fi}}
	       		
	       	
	       	\ifsolutions
		\tkzTabLine
			{,-,,-,z,+,,+}
		\tkzTabLine
			{,-,z,+,,+,,+}
		\tkzTabLine
			{,+,,+,,+,z,-}
		\tkzTabLine
			{,+,d,-,z,+,d,-}
		\fi
	\end{tikzpicture}
	\end{center}
}{

	\begin{enumerate}
		\item 
		Par distributivité, 
			\[ (x+3)(2-x) = x(2-x) + 3(2-x) = 2x - x^2 + 6 - 3x = 6 - x - x^2. \]
		
		\item
		La seule opération éventuellement illégale est la division par zéro : il nous incombe donc d'empêcher que cela arrive.
		Or cela arrive lorsque
			\begin{align*}
				6-x-x^2= 0 && \iff && (x+3)(2-x) = 0 && \iff && x=-3 \text{ ou } x=2.
			\end{align*}
		On empèche donc les valeurs $-3$ et $2$ en perforant $\R$ :
			\[ \D_f = \R - \left\{ -3 ; 2 \right\}. \]
		\item[4.] 
		On lit l'ensemble à l'aide du tableau de signes :
			\[ \{ x \in \D_f \text{ tq. } f(x) \geq 0 \} = ]\minfty ; -3[ \cup [-2 ; 2[. \]
		Remarques : l'inégalité $f(x) \geq 0$ est large donc $-2$ est inclu. Les valeurs interdites, elles, ne sont jamais incluses.
	\end{enumerate}
}

\exe{
	Pour chaque propriété, donner une fonction $f$ sur $\R$ non identiquement nulle la vérifiant.
	\begin{enumerate}
		\item $f$ s'annule en $3$.
		\item $f$ s'annule en $0$ et en $1$.
		\item $f$ s'annule en $0, -1$, et  $\dfrac12$.
	\end{enumerate}
}{

	\begin{enumerate}
		\item Par exemple $f(x) = x-3$.
		\item Par exemple $f(x) = 500x(x-1)$.
		\item Par exemple $f(x) = -10^{40} (x+1)x\left(x-\dfrac12\right)$.
	\end{enumerate}
}

\exe{[Vrai ou faux]
	Pour chaque proposition suivante, démontrer qu'elle est vraie ou donner un contre-exemple.
	$f, g, h,$ et $F$ sont des fonctions définies sur $\R$.
	\begin{enumerate}
		\item Si $f$ s'annule en $1$, alors $f$ s'annule en $-1$.
		\item La fonction $g(x)=x^2 (x-1)^2$ admet exactement deux racines et est toujours positive.
		\item Si $h$ s'annule en $3$ et en $4$, alors $h$ est de signe constant (toujours positif ou toujours négatif).
		\item Si $F$ est de signe constant (toujours positif ou toujours négatif), alors $F$ est constante.
		
	\end{enumerate}
}{

	\begin{enumerate}
		\item
		C'est faux. Par exemple, $f(x) = x-1$ s'annule en $1$ mais pas en $-1$ car $f(-1) = -2 \neq 0$.
		
		\item 
		C'est vrai.
		$g$ est produit de deux carrés qui sont toujours positifs et s'annules en $x=0$ et $x=1$.
		
		\item 
		C'est faux.
		On peut facilement dessiner la courbe d'une fonction qui s'annule en $3$ et $4$ et donc le signe varie.
		Par exemple une fonction sinusoïdale qui oscille.
		
		Algébriquement, $f(x) = (x-3)(x-4)$ fonctionne. Un tableau de signes ou les images $f(2,5) < 0$ et $f(5) > 0$ permettent de conclure.
		
		\item 
		C'est faux.
		On peut dessiner la courbe d'une fonction non constante (c'est-à-dire qui n'est pas une droite horizontale) et qui est toujours positive ou toujours négative.
		
		Algébriquement, la fonction carré est un contre-exemple, car $f(x) = x^2$ est toujours positive et n'est pas constante.
		
	\end{enumerate}
	
}


\exe{
	Soient $f$ et $g$ deux fonctions croissantes sur $\R$ tout entier.
	
	Montrer que la fonction $h(x) = f(x) + g(x)$ est également croissante sur $\R$.
}{
	On souhait démontrer que $h$ préserve l'ordre : si $x$ augmente, $h(x)$ augmente aussi.
	À cette fin, prenons $x < y$ deux nombres réels.
	Comme $f$ et $g$ sont croissantes, on a nécessairement
		\begin{align*}
			f(x) < f(y), && \text{ et } && g(x) < g(y).
		\end{align*}
	Il suit que
		\[ h(x) = f(x) + g(x) < f(y) + g(x) < f(y) + g(y) < h(y), \]
	ce qui conclut.
}

\exe{
	On pose l'intervalle $I=]3; 17[$.
	
	\begin{enumerate}
		\item Montrer que $3x+5 > 17 - x > 0$ sur $I$.
		\item Montrer que $7x-5 > 2(x+5) > 0$ sur $I$.
		\item En déduire que, pour tout $x\in I$,
			\[ \dfrac{(3x+5)(7x-5)}{(17-x)(x+5)} > 2. \]
		\item En développant $(3x+5)(7x-5)-2(17-x)(x+5)$, en déduire que $23x^2  > 4x + 195$ sur $I$.
	\end{enumerate}
}{
	
	\begin{enumerate}
		\item 
		La première inégalité à vérifier est $3x+5 > 17-x$.
		Celle-ci est équivalente à $4x > 12$, et aussi à $x > 3$, inégalité vraie sur $I$.
		
		La deuxième inégalité à vérifier est $17 - x > 0$, équivalente à $x < 17$, vraie sur $I$.
		
		\item
		La première inégalité à vérifier est $7x - 5 > 2(x+5)$, équivalente à $7x - 5 > 2x + 10$, et à $5x > 15$ puis $x> 3$.
		Cette inégalité est donc vraie sur $I$.
		
		La deuxième inégalité à vérifier est $2(x+5) > 0$, qui est équivalente à $x > -5$, ce qui est également vrai sur $I$.
		
		\item En déduire que, pour tout $x\in I$,
			\[ \dfrac{(3x+5)(7x-5)}{(17-x)(x+5)} > 2. \]
		
		La première inégalité donne, par division par $17-x$,
			\[ \dfrac{3x+5}{17-x} > \dfrac{17-x}{17-x} = 1, \]
		où on a utilisé que $17-x >0$ sur $I$ pour s'assurer que le sens de l'inégalité ne change pas.
		
		La deuxième inégalité donne, par division par $x+5$,
			\[ \dfrac{7x-5}{x+5} > \dfrac{2(x+5)}{x+5} = 2, \]
		où on a utilisé que $2(x+5) >0$ sur $I$ pour s'assurer que le sens de l'inégalité ne change pas.
		
		On multiplie les deux inégalités ensembles et, par positivité, on conclut.
			
		\item
		D'après la question précédente, on a
			 \[ \dfrac{(3x+5)(7x-5)}{(17-x)(x+5)}- 2 > 0 \]
		sur $I$.
		Le dénominateur étant positif, on multiplie par $(17-x)(x+5)$ sans changer le sens de l'inégalité pour trouver que
			\[ (3x+5)(7x-5)-2(17-x)(x+5) > 0 \]
		sur $I$.
		
		 On développe ensuite
		 	\begin{align*}
		 		(3x+5)(7x-5)-2(17-x)(x+5) &= 21x^2 + (-15+35)x - 25 - 2(-x^2 + x(17-5)+85)
		 		&= 21x^2 + 20x - 25 -2(-x^2 + 12x + 85) 
		 		&= 21x^2 + 20x - 25 + 2x^2 - 24x - 170
		 		&= 23x^2 - 4x - 195
			\end{align*}
		Il suit que, pour tout $x\in I$,
			\begin{align*}
				23x^2 - 4x - 195 > 0 && \iff && 23x^2 > 4x + 195.
			\end{align*}
		Un exercice très difficile, en somme !
	\end{enumerate}


}

%\subsection*{Exercices supplémentaires}

\exe{
	Donner une fonction $f$ telle que
		\begin{align*}
			f(2) = 0, && f(-1) = 0, && \text{ et } && f(3)=8.
		\end{align*}
}{
	On commence par prendre une fonction non identiquement nulle qui s'annule en $2$ et en $-1$.
		\[ g(x) = (x-2)(x+1). \]
	Or, $g(3) = (3-2)(3+1) = 4$, ce qui ne convient pas !
	Remarquons cependant qu'on peut toujours multiplier $g$ par une constante au choix sans changer ses racines.
	Par quoi multiplier $4$ pour obtenir $8$ ?
	
	On pose donc
		\[ f(x) = 2g(x) = 2(x-2)(x+1), \]
	qui vérifie alors $f(2) = f(-1) = 0$, et $f(3) = 8$, comme requis.
}

\exe{
	Donner une fonction $f$ de la forme $f(x) = ax^2 + bx + c$ où $a, b, c\in\Z$ sont des \underline{entiers relatifs} telle que
		\[ f\left(\dfrac13\right) = f\left(\dfrac{1}{2}\right) = 0. \]
}{
	On commence par prendre une fonction qui s'annule en $\frac13$ et en $\frac12$.
		\[ g(x) = \left(x - \dfrac13\right)\left( x - \dfrac12 \right) = x^2 - \dfrac56 x + \dfrac16. \]
	Petit problème, les coefficients de $g$ ne sont pas des entiers !
	Cependant, comme on peut multiplier $g$ par n'importe quel nombre sans changer ses racines, on se demande : par quoi multiplier $-\dfrac56$ et $\dfrac16$ pour obtenir un nombre entier relatif ?
	
	On pose donc
		\[ f(x) = 6g(x) = 6x^2 - 5x + 1, \]
	qui vérifie alors $f(\frac12) = f(\frac13) = 0$ (à vérifier !).
}

\exe{
	Soient $f$ et $g$ deux fonctions croissantes sur $\R$ tout entier.
	
	Montrer que la fonction composée $k(x) = f(g(x))$ est également croissante sur $\R$.
}{
	Soient $x < y$ deux réels.
	Par croissance de $g$, on a 
		\[ g(x) < g(y). \]
	On comprend désormais $g(x)$ et $g(y)$ comme deux antécédents réels à donner à $f$ : par croissance de $f$, on a alors
		\[ f(g(x)) < f(g(y)), \]
	ce qui conclut.


}

\end{document}

%!TEX root = ../eval1.tex

\exemulticols{2}{
	Compléter les pointillés en ajoutant un signe d'inclusion ou de non inclusion ($\subseteq$ ou $\not\subseteq$) entre les ensembles de nombres ci-contre.
}{
	\begin{center}
	\def\arraystretch{1.4}
	\setlength\tabcolsep{5pt}
	\begin{tabular}{|ccc|ccc|}\hline
		$\bigset{1 ; 3 ; 0}$ & \dots & $\Z$ & $\bigset{3 ; 4 ; -\frac12}$ & \dots & $\Z$ \\
		$\Z$ & \dots & $\R$ & $\bigset{4 ; -\frac13 ; \frac{141}{367}}$ & \dots & $\Q$ \\ \hline
	\end{tabular}
	\end{center}
}{exe:1}{
	\begin{center}
	\def\arraystretch{1.5}
	\setlength\tabcolsep{10pt}
	\begin{tabular}{|ccc|ccc|}\hline
		$\bigset{1 ; 3 ; 0}$ & $\subseteq$ & $\Z$ & $\bigset{3 ; 4 ; -\frac12}$ & $\not\subseteq$ & $\Z$ \\
		$\Z$ & $\subseteq$ & $\R$ & $\bigset{4 ; -\frac13 ; \frac{141}{367}}$ & $\subseteq$ & $\Q$ \\ \hline
	\end{tabular}
	\end{center}
}
% DYSLEXIA SWITCH
\newif\ifdys
		
				% ENABLE or DISABLE font change
				% use XeLaTeX if true
				\dystrue
				\dysfalse


\ifdys

\documentclass[a4paper, 14pt]{extarticle}
\usepackage{amsmath,amsfonts,amsthm,amssymb,mathtools}

\tracinglostchars=3 % Report an error if a font does not have a symbol.
\usepackage{fontspec}
\usepackage{unicode-math}
\defaultfontfeatures{ Ligatures=TeX,
                      Scale=MatchUppercase }

\setmainfont{OpenDyslexic}[Scale=1.0]
\setmathfont{Fira Math} % Or maybe try KPMath-Sans?
\setmathfont{OpenDyslexic Italic}[range=it/{Latin,latin}]
\setmathfont{OpenDyslexic}[range=up/{Latin,latin,num}]

\else

\documentclass[a4paper, 12pt]{extarticle}

\usepackage[utf8x]{inputenc}
%fonts
\usepackage{amsmath,amsfonts,amsthm,amssymb,mathtools}
% comment below to default to computer modern
\usepackage{libertinus,libertinust1math}

\fi


\usepackage[french]{babel}
\usepackage[
a4paper,
margin=2cm,
nomarginpar,% We don't want any margin paragraphs
]{geometry}
\usepackage{icomma}

\usepackage{fancyhdr}
\usepackage{array}
\usepackage{hyperref}

\usepackage{multicol, enumerate}
\newcolumntype{P}[1]{>{\centering\arraybackslash}p{#1}}


\usepackage{stackengine}
\newcommand\xrowht[2][0]{\addstackgap[.5\dimexpr#2\relax]{\vphantom{#1}}}

% theorems

\theoremstyle{plain}
\newtheorem{theorem}{Th\'eor\`eme}
\newtheorem*{sol}{Solution}
\theoremstyle{definition}
\newtheorem{ex}{Exercice}
\newtheorem*{rpl}{Rappel}
\newtheorem{enigme}{Énigme}

% corps
\usepackage{calrsfs}
\newcommand{\C}{\mathcal{C}}
\newcommand{\R}{\mathbb{R}}
\newcommand{\Rnn}{\mathbb{R}^{2n}}
\newcommand{\Z}{\mathbb{Z}}
\newcommand{\N}{\mathbb{N}}
\newcommand{\Q}{\mathbb{Q}}

% variance
\newcommand{\Var}[1]{\text{Var}(#1)}

% domain
\newcommand{\D}{\mathcal{D}}


% date
\usepackage{advdate}
\AdvanceDate[0]


% plots
\usepackage{pgfplots}

% table line break
\usepackage{makecell}
%tablestuff
\def\arraystretch{2}
\setlength\tabcolsep{15pt}

%subfigures
\usepackage{subcaption}

\definecolor{myg}{RGB}{56, 140, 70}
\definecolor{myb}{RGB}{45, 111, 177}
\definecolor{myr}{RGB}{199, 68, 64}

% fake sections with no title to move around the merged pdf
\newcommand{\fakesection}[1]{%
  \par\refstepcounter{section}% Increase section counter
  \sectionmark{#1}% Add section mark (header)
  \addcontentsline{toc}{section}{\protect\numberline{\thesection}#1}% Add section to ToC
  % Add more content here, if needed.
}


% SOLUTION SWITCH
\newif\ifsolutions
				\solutionstrue
				%\solutionsfalse

\ifsolutions
	\newcommand{\exe}[2]{
		\begin{ex} #1  \end{ex}
		\begin{sol} #2 \end{sol}
	}
\else
	\newcommand{\exe}[2]{
		\begin{ex} #1  \end{ex}
	}
	
\fi


% tableaux var, signe
\usepackage{tkz-tab}


%pinfty minfty
\newcommand{\pinfty}{{+}\infty}
\newcommand{\minfty}{{-}\infty}

\begin{document}


\SetDate[15/10/2025]
\reversemarginpar
\setlength{\marginparsep}{.5cm}

\newcommand{\exeI}{1}
\newcommand{\exeII}{2}
\newcommand{\exeIII}{3}
\newcommand{\exeIV}{4}
\newcommand{\exeV}{5}
\newcommand{\exeVI}{6}
\newcommand{\exeVII}{7}
\newcommand{\exeVIII}{8}
\newcommand{\exeIX}{9}
\newcommand{\exeX}{10}
\newcommand{\exeXI}{11}

\begin{document}
\pagestyle{fancy}
\fancyhead[L]{Seconde}
\fancyhead[C]{\textbf{Évaluation — Plan cartésien}}
\fancyhead[R]{\today}

\null\vspace{-30pt}
Consignes particulières : 
\begin{itemize}[label=$\bullet$]
	\item 
	La calculatrice est {interdite}. Une aide aux calculs est disponible à l'exercice \ref{exe:Trex}.
	\item
	Les sujets d'évaluation sont individuels. Écrire son nom avant de rendre son sujet.
	\item
	Les exercices \ref{exe:prénom} et \ref{exe:diagonale} peuvent être faits entièrement sur la feuille d'évaluation. 
	\item 
	On supposera l'existence d'un nombre positif noté $\sqrt{12}$ dont le carré vaut 12 pour résoudre l'exercice \ref{exe:equilateral}.
	En d'autres termes, $\sqrt{12} > 0$, et $\bigl(\sqrt{12}\bigr)^2 = 12$.
	\item
	L'évaluation fait 2 pages. La somme des points est \total{points}.
\end{itemize}

\marginpar{[pts]}
\hrule

%!TEX root = ../eval1.tex

\exe{2}{
	\begin{enumerate}
		\item Placer des points dans le repère ci-dessous et les relier afin qu'on puisse lire la première lettre de votre prénom.
		\item Donner les coordonnées de chaque point placé.
	\end{enumerate}
}{exe:prénom}{
	Le prénom du correcteur commençant par $M$, celui-ci propose les points $A(-4;0), B(-4;4), C(-2,5;2,5), D(-1;4), E(-1;0)$.
	\begin{center}
	\begin{tikzpicture}[>=stealth, scale=.8]
		\begin{axis}[xmin = -5.1, xmax=1.1, ymin=-1.1, ymax=5.1, axis x line=middle, axis y line=middle, axis line style=<->, xlabel={}, ylabel={}, grid=both, grid style = {opacity=.5}, clip=false, xtick distance = 2, ytick distance=1, x=2cm, y=1cm]
			\addplot[BLUE_E, mark=*, mark size = 1] (-4,0) node[above left] {$A(-4;0)$};
			\addplot[RED_E, mark=*, mark size = 1] (-4,4) node[above left] {$B(-4;4)$};
			\addplot[GREEN_E, mark=*, mark size = 1] (-2.5,2.5) node[below] {$C(-2,5;2,5)$};
			\addplot[BLUE_E, mark=*, mark size = 1] (-1,4) node[above right] {$D(-1;4)$};
			\addplot[RED_E, mark=*, mark size = 1] (-1,0) node[above right] {$E(-1;0)$};
			
			\draw[-, thick,dashed] (axis cs:-4,0) -- (axis cs:-4,4);
			\draw[-, thick, dashed] (axis cs:-4,4) -- (axis cs:-2.5,2.5);
			\draw[-, thick,dashed] (axis cs:-2.5,2.5) -- (axis cs:-1,4);
			\draw[-, thick,dashed] (axis cs:-1,4) -- (axis cs:-1,0);
		\end{axis}
	\end{tikzpicture}
	\end{center}
}


	\begin{center}
	\begin{tikzpicture}[>=stealth, scale=.8]
		\begin{axis}[xmin = -5.1, xmax=1.1, ymin=-1.1, ymax=5.1, axis x line=middle, axis y line=middle, axis line style=<->, xlabel={}, ylabel={}, grid=both, grid style = {opacity=.5}, clip=false, xtick distance = 2, ytick distance=1, x=2cm, y=1cm]
		\end{axis}
	\end{tikzpicture}
	\end{center}
	

%!TEX root = ../eval1.tex

\exemulticols{3}{
	Donner approximativement les coordonnées de chaque point du repère ci-contre.
	\begin{align*}
		&A\hspace{3cm} \\[10pt]
		&B \\[10pt]
		&C \\[10pt]
		&D \\[10pt]
		&E \\[10pt]
		&O
	\end{align*}
	%\vfill\null
}{
	\begin{center}
	\begin{tikzpicture}[>=stealth, scale=1.2]
		\begin{axis}[xmin = -4.9, xmax=4.9, ymin=-4.9, ymax=4.9, axis x line=middle, axis y line=middle, axis line style=<->, xlabel={}, ylabel={}, grid=both, grid style = {opacity=.5}, xtick distance=1, ytick distance=1]			
			\addplot[BLUE_E, mark=*, mark size = 1] (2,0) node[above] {$A$};
			\addplot[RED_E, mark=*, mark size = 1] (-3,3) node[above left] {$B$};
			\addplot[GREEN_E, mark=*, mark size = 1] (3,1.5) node[above right] {$C$};
			\addplot[PURPLE_E, mark=*, mark size = 1] (0,-1.5) node[right] {$D$};
			\addplot[GOLD_E, mark=*, mark size = 1] (-2.5,-3) node[left] {$E$};
			\addplot[BLACK, mark=*, mark size = 1] (0,0) node[above left] {$O$};
		\end{axis}
	\end{tikzpicture}
	\end{center}
}{exe:lecture-coord}{
	\begin{align*}
		A(3 ; 0) && B(-1 ; 3) && C(3 ; 3) &&
		D(0;-2) && E(-2,5 ; 0)
	\end{align*}
}


%!TEX root = ../eval1.tex

\exemulticols{5}{
	Considérons le triangle de sommets 
		\[ A(2;6), \quad B(8 ; -3), \quad\et\quad C(-10 ; -2).\]
	
	Les questions 1, 2, et 3 peuvent être faites séparément.
	\begin{enumerate}
		\item Tracer le triangle $ABC$ dans un repère.
		\item Montrer par le calcul que
		\begin{enumerate}[label=\roman*)]
			\item $AB^2 = 117$
			\item $AC^2 = 208$
			\item $BC^2 = 325$
		\end{enumerate}
		\item Que dire du triangle $ABC$ ?
	\end{enumerate}
}{
	%\hfill
	%Aide aux calculs
	
	\def\arraystretch{1.1}
	\setlength\tabcolsep{15pt}
	\hfill
	\begin{tabular}{|c|c|}\hline
		$n$ & $n^2$ \\ \hline
		11 & 121 \\ \hline
		12 & 144 \\ \hline
		13 & 169 \\ \hline
		14 & 196 \\ \hline
		15 & 225 \\ \hline
		16 & 256 \\ \hline
		17 & 289 \\ \hline
		18 & 324 \\ \hline
		19 & 361 \\ \hline
		20 & 400 \\ \hline
	\end{tabular}
}{exe:Trex}{
	\begin{center}
	\begin{tikzpicture}[>=stealth, scale=1]
		\begin{axis}[xmin = -9.1, xmax=9.1, ymin=-5.1, ymax=5.1,axis x line=middle, axis y line=middle, axis line style=<->, xlabel={}, ylabel={}, grid=both, grid style = {opacity=.5}, x=20pt, y=20pt]			
			\addplot[BLUE_E, mark=*, mark size = 1] (2,6) node[above] {$A$};
			\addplot[RED_E, mark=*, mark size = 1] (8,-3) node[right] {$B$};
			\addplot[GREEN_E, mark=*, mark size = 1] (-10,-2) node[left] {$C$};
			
			\draw[-, thick] (axis cs:2,6) -- (axis cs:8,-3) -- (axis cs:-10,-2) -- (axis cs:2,6);
		\end{axis}
	\end{tikzpicture}
	\end{center}
	
	\begin{enumerate}
		\item[2.] 
			\begin{align*}
				AB^2 &= \norm{A-B}^2 = \norm{(-6 ; 9)}^2 = 36 + 81 = 117, \\
				AC^2 &= \norm{A-C}^2 = \norm{(12 ; 8)}^2 = 144 + 64 = 208, \\
				BC^2 &= \norm{B-C}^2 = \norm{(18 ; -1)}^2 = 324 + 1 = 325.
			\end{align*}
		\item[3.] D'après la réciproque du théorème de Pythagore, le triangle est rectangle en $A$ car $BC^2 = AB^2 + AC^2$.
	\end{enumerate}
}

%!TEX root = ../eval1.tex

\exe{2}{
	Considérons les points
		\begin{align*}
			\point{A}{\dfrac{10}3}{-\dfrac76}, && \point{B}{\dfrac{13}3}{\dfrac{11}6}, && \point{C}{\dfrac{-5}3}{\dfrac{17}6}, && \point{D}{-\dfrac23}{\dfrac{35}6}.
		\end{align*}
	Montrer que le quadrilatère $CDBA$ est un parallélogramme en comparant le milieu de ses deux diagonales.
	
	\emph{Un parallélogramme est un quadrilatère dont les diagonales se coupent en leur milieu}.
}{exe:parallélogramme}{
	Le milieu du segment $[BC]$ est donné par
		\[ M = \dfrac12 (B+C) = (1;2,5),\]
	et le milieu du segment $[AD]$ est donné par
		\[ M' = \dfrac12 (A+D) = (1;2,5).\]
	Comme $M' = M$, le quadrilatère est bien un parallélogramme.

	\centering
	\begin{tikzpicture}[>=stealth, scale=1]
	\begin{axis}[xmin = -2.1, xmax=4.1, ymin=-1.1, ymax=6.1, axis x line=middle, axis y line=middle, axis line style=<->, xlabel={}, ylabel={}, grid=both, clip=false, ytick distance=1]
					
		\addplot[RED_E, mark=*, mark size = 1] (3,-1) node[right] {$A$};
		\addplot[RED_E, mark=*, mark size = 1] (4,2) node[right] {$B$};
		\addplot[RED_E, mark=*, mark size = 1] (-2,3) node[left] {$C$};
		\addplot[RED_E, mark=*, mark size = 1] (-1,6) node[left] {$D$};
		
		\draw[dashed, thick] (axis cs:3,-1) -- (axis cs:-1,6);
		\draw[dashed, thick] (axis cs:4,2) -- (axis cs:-2,3);
		
		\addplot[black, mark=*, mark size = 1] (1,2.5) node[above right] {$M$};
		
		\draw[thick] (axis cs:3,-1) -- (axis cs:4,2) -- (axis cs:-1,6) -- (axis cs:-2,3) -- (axis cs:3,-1);
	\end{axis}
	\end{tikzpicture}
}

%!TEX root = ../eval1.tex

\exe{3, difficulty=1}{
	Considérons les points $\point{A}{2}{1}, \point{B}{2}{5}, C\bigl(\sqrt{12}+2 ; 3\bigr)$.
	Montrer que le triangle $ABC$ est équilatéral en calculant le carré de la longueur de chaque côté.
	
	\emph{Un triangle équilatéral est un triangle dont les trois côtés ont la même longueur}.
}{exe:equilateral}{
	On calcule 
		\begin{align*}
			AB^2 = {0^2 + 4^2} = 16, && AC^2 = {\sqrt{12}^2 + 2^2} = 16, && BC^2 = 16.
		\end{align*}
}

\newpage

%!TEX root = ../eval1.tex

\exe{20, difficulty=3}{
	Considérons les points $O(0;0)$ et $P(3 ; 1)$ et $(d)$ la médiatrice du segment $[OP]$ : ce sont tous les points $M(x;y)$ à équidistance de $O$ et $P$.
	
	Montrer que $x$ et $y$ sont liés par la relation 
		\[ y = -3x + 5. \]
		
	\emph{Toute trace de recherche sera prise en compte.}
}{exe:médiatrice}{
	TODO
}

%!TEX root = ../eval1.tex

\exe{2, difficulty=1}{
	Exprimer les nombres suivants sous forme de fraction d'entiers.
	\begin{multicols}{2}
	\begin{enumerate}[label=]
		%\item $A = 0,666...$ ($6$ se répète à l'infini)
		\item $x = 0,555...$ ($5$ se répète à l'infini)
		%\item $C = 0,020202...$ ($02$ se répète à l'infini)
		\item $y = 1,545454...$ ($54$ se répète à l'infini)
	\end{enumerate} 
	\end{multicols}
}{exe:4}{
	\begin{enumerate}
		\item L'équation $10x = 5 + x$ implique que $x=\frac59$.
		\item L'équation 
			\[ 100y = 154,545454... = 153 + 1,545454... = 153 + y \]
		implique que $y = \frac{153}{99}$.
	\end{enumerate}
}

%!TEX root = ../eval1.tex

\exe{2, difficulty=1}{
	Soit $A(x;x)$ un point dépendant d'un paramètre réel $x\in\R$.
	
	Où se situe le point $A$ ? Placer ses positions possibles dans le repère ci-dessous.
	
	\begin{center}
	\begin{tikzpicture}[>=stealth, scale=.8]
		\begin{axis}[xmin = -5.1, xmax=5.1, ymin=-5.1, ymax=5.1, axis x line=middle, axis y line=middle, axis line style=<->, xlabel={}, ylabel={}, grid=both, grid style = {opacity=.5}, clip=false, xtick distance = 1, ytick distance=1, x=1cm, y=1cm]
		\end{axis}
	\end{tikzpicture}
	\end{center}
}{exe:diagonale}{

	On teste plusieurs valeurs de $x\in\R$, sans oublier des valeurs négatives ou non entières.
	
	En plaçant les points $(1;1), (-3; -3), (2,1 ; 2,1), (-0,5 ; -05), \dots$, on remarque que les positions possibles de $A$ se situent sur la diagonale ci-dessous.
	
	\begin{center}
	\begin{tikzpicture}[>=stealth, scale=.7]
		\begin{axis}[xmin = -5.1, xmax=5.1, ymin=-5.1, ymax=5.1, axis x line=middle, axis y line=middle, axis line style=<->, xlabel={}, ylabel={}, grid=both, grid style = {opacity=.5}, clip=false, xtick distance = 1, ytick distance=1, x=1cm, y=1cm]
		\draw[BLUE_E, -, very thick] (axis cs:-5,-5) -- (axis cs:5,5);
		\end{axis}
	\end{tikzpicture}
	\end{center}
}

%!TEX root = ../eval1.tex

\exe{1, difficulty=1}{
	Soit $A(x;x)$ un point dépendant d'un paramètre réel $x\in\R$.
	
	Où se situe le point $A$ ? Tracer ses positions possibles dans le repère ci-dessous.
	
	\begin{center}
	\begin{tikzpicture}[>=stealth, scale=.8]
		\begin{axis}[xmin = -5.1, xmax=5.1, ymin=-5.1, ymax=5.1, axis x line=middle, axis y line=middle, axis line style=<->, xlabel={}, ylabel={}, grid=both, grid style = {opacity=.5}, clip=false, xtick distance = 1, ytick distance=1, x=1cm, y=1cm]
		\end{axis}
	\end{tikzpicture}
	\end{center}
}{exe:diagonale}{
	TODO
}

%%%%%%%%%%%%

\newpage
\fancyhead[C]{\textbf{Solutions}}
\shipoutAnswer

\end{document}

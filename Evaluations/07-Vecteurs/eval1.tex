%!TEX encoding = UTF8
%!TEX root =notes.tex


%%%%%%%%%%%%%%%%%%%%%%%%%%%%%%%%%
% PACKAGE IMPORTS
%%%%%%%%%%%%%%%%%%%%%%%%%%%%%%%%%


\usepackage[french]{babel}

\usepackage[tmargin=2cm,rmargin=1in,lmargin=1in,margin=0.85in,bmargin=2cm,footskip=.2in]{geometry}
\usepackage{amsmath,amsfonts,amsthm,amssymb,mathtools}
\usepackage[varbb]{newpxmath}
\usepackage{xfrac}
\usepackage[makeroom]{cancel}
\usepackage{mathtools}
\usepackage{bookmark}
\usepackage{enumitem}
\usepackage{hyperref,theoremref}
\hypersetup{
	pdftitle={Assignment},
	colorlinks=true, linkcolor=doc!90,
	bookmarksnumbered=true,
	bookmarksopen=true
}
\usepackage[most,many,breakable]{tcolorbox}
\usepackage{xcolor}
\usepackage{varwidth}
\usepackage{varwidth}
\usepackage{etoolbox}
%\usepackage{authblk}
\usepackage{nameref}
\usepackage{multicol,array}
\usepackage{tikz-cd}
\usepackage[ruled,vlined,linesnumbered]{algorithm2e}
\usepackage{comment} % enables the use of multi-line comments (\ifx \fi) 
\usepackage{import}
\usepackage{xifthen}
\usepackage{pdfpages}
\usepackage{transparent}


\newcommand\mycommfont[1]{\footnotesize\ttfamily\textcolor{blue}{#1}}
\SetCommentSty{mycommfont}
\newcommand{\incfig}[1]{%
    \def\svgwidth{\columnwidth}
    \import{./figures/}{#1.pdf_tex}
}

\usepackage{tikzsymbols}
%\renewcommand\qedsymbol{$\Laughey$}


%\usepackage{import}
%\usepackage{xifthen}
%\usepackage{pdfpages}
%\usepackage{transparent}


%%%%%%%%%%%%%%%%%%%%%%%%%%%%%%
% SELF MADE COLORS
%%%%%%%%%%%%%%%%%%%%%%%%%%%%%%



\definecolor{myg}{RGB}{56, 140, 70}
\definecolor{myb}{RGB}{45, 111, 177}
\definecolor{myr}{RGB}{199, 68, 64}
\definecolor{mytheorembg}{HTML}{F2F2F9}
\definecolor{mytheoremfr}{HTML}{00007B}
\definecolor{mylenmabg}{HTML}{FFFAF8}
\definecolor{mylenmafr}{HTML}{983b0f}
\definecolor{mypropbg}{HTML}{f2fbfc}
\definecolor{mypropfr}{HTML}{191971}
\definecolor{myexamplebg}{HTML}{F2FBF8}
\definecolor{myexamplefr}{HTML}{88D6D1}
\definecolor{myexampleti}{HTML}{2A7F7F}
\definecolor{mydefinitbg}{HTML}{E5E5FF}
\definecolor{mydefinitfr}{HTML}{3F3FA3}
\definecolor{notesgreen}{RGB}{0,162,0}
\definecolor{myp}{RGB}{197, 92, 212}
\definecolor{mygr}{HTML}{2C3338}
\definecolor{myred}{RGB}{127,0,0}
\definecolor{myyellow}{RGB}{169,121,69}
\definecolor{myexercisebg}{HTML}{F2FBF8}
\definecolor{myexercisefg}{HTML}{88D6D1}


%%%%%%%%%%%%%%%%%%%%%%%%%%%%
% TCOLORBOX SETUPS
%%%%%%%%%%%%%%%%%%%%%%%%%%%%

\setlength{\parindent}{1cm}
%================================
% THEOREM BOX
%================================

\tcbuselibrary{theorems,skins,hooks}
\newtcbtheorem[number within=chapter]{Theorem}{Théorème}
{%
	enhanced,
	breakable,
	colback = mytheorembg,
	frame hidden,
	boxrule = 0sp,
	borderline west = {2pt}{0pt}{mytheoremfr},
	sharp corners,
	detach title,
	before upper = \tcbtitle\par\smallskip,
	coltitle = mytheoremfr,
	fonttitle = \bfseries\sffamily,
	description font = \mdseries,
	separator sign none,
	segmentation style={solid, mytheoremfr},
}
{th}


\tcbuselibrary{theorems,skins,hooks}
\newtcolorbox{Theoremcon}
{%
	enhanced
	,breakable
	,colback = mytheorembg
	,frame hidden
	,boxrule = 0sp
	,borderline west = {2pt}{0pt}{mytheoremfr}
	,sharp corners
	,description font = \mdseries
	,separator sign none
}

%================================
% Corollery
%================================
\tcbuselibrary{theorems,skins,hooks}
\newtcbtheorem[use counter=tcb@cnt@Theorem]{Corollary}{Corollaire}
{%
	enhanced
	,breakable
	,colback = myp!10
	,frame hidden
	,boxrule = 0sp
	,borderline west = {2pt}{0pt}{myp!85!black}
	,sharp corners
	,detach title
	,before upper = \tcbtitle\par\smallskip
	,coltitle = myp!85!black
	,fonttitle = \bfseries\sffamily
	,description font = \mdseries
	,separator sign none
	,segmentation style={solid, myp!85!black}
}
{th}

%================================
% LENMA
%================================

\tcbuselibrary{theorems,skins,hooks}
\newtcbtheorem[use counter=tcb@cnt@Theorem]{Lemma}{Lemme}
{%
	enhanced,
	breakable,
	colback = mylenmabg,
	frame hidden,
	boxrule = 0sp,
	borderline west = {2pt}{0pt}{mylenmafr},
	sharp corners,
	detach title,
	before upper = \tcbtitle\par\smallskip,
	coltitle = mylenmafr,
	fonttitle = \bfseries\sffamily,
	description font = \mdseries,
	separator sign none,
	segmentation style={solid, mylenmafr},
}
{th}


%================================
% PROPOSITION
%================================

\tcbuselibrary{theorems,skins,hooks}
\newtcbtheorem[use counter=tcb@cnt@Theorem]{Prop}{Proposition}
{%
	enhanced,
	breakable,
	colback = mypropbg,
	frame hidden,
	boxrule = 0sp,
	borderline west = {2pt}{0pt}{mypropfr},
	sharp corners,
	detach title,
	before upper = \tcbtitle\par\smallskip,
	coltitle = mypropfr,
	fonttitle = \bfseries\sffamily,
	description font = \mdseries,
	separator sign none,
	segmentation style={solid, mypropfr},
}
{th}


%================================
% CLAIM
%================================

\tcbuselibrary{theorems,skins,hooks}
\newtcbtheorem[use counter=tcb@cnt@Theorem]{claim}{Claim}
{%
	enhanced
	,breakable
	,colback = myg!10
	,frame hidden
	,boxrule = 0sp
	,borderline west = {2pt}{0pt}{myg}
	,sharp corners
	,detach title
	,before upper = \tcbtitle\par\smallskip
	,coltitle = myg!85!black
	,fonttitle = \bfseries\sffamily
	,description font = \mdseries
	,separator sign none
	,segmentation style={solid, myg!85!black}
}
{th}



%================================
% Exercise
%================================

\tcbuselibrary{theorems,skins,hooks}
\newtcbtheorem[use counter=tcb@cnt@Theorem]{Exercise}{Exercice}
{%
	enhanced,
	breakable,
	colback = myexercisebg,
	frame hidden,
	boxrule = 0sp,
	borderline west = {2pt}{0pt}{myexercisefg},
	sharp corners,
	detach title,
	before upper = \tcbtitle\par\smallskip,
	coltitle = myexercisefg,
	fonttitle = \bfseries\sffamily,
	description font = \mdseries,
	separator sign none,
	segmentation style={solid, myexercisefg},
}
{th}

%================================
% EXAMPLE BOX
%================================

\newtcbtheorem[use counter=tcb@cnt@Theorem]{Example}{Exemple}
{%
	colback = myexamplebg
	,breakable
	,colframe = myexamplefr
	,coltitle = myexampleti
	,boxrule = 1pt
	,sharp corners
	,detach title
	,before upper=\tcbtitle\par\smallskip
	,fonttitle = \bfseries
	,description font = \mdseries
	,separator sign none
	,description delimiters parenthesis
}
{ex}

%================================
% DEFINITION BOX
%================================

\newtcbtheorem[use counter=tcb@cnt@Theorem]{Definition}{Définition}{enhanced,
	before skip=2mm,after skip=2mm, colback=red!5,colframe=red!80!black,boxrule=0.5mm,
	attach boxed title to top left={xshift=1cm,yshift*=1mm-\tcboxedtitleheight}, varwidth boxed title*=-3cm,
	boxed title style={frame code={
					\path[fill=tcbcolback]
					([yshift=-1mm,xshift=-1mm]frame.north west)
					arc[start angle=0,end angle=180,radius=1mm]
					([yshift=-1mm,xshift=1mm]frame.north east)
					arc[start angle=180,end angle=0,radius=1mm];
					\path[left color=tcbcolback!60!black,right color=tcbcolback!60!black,
						middle color=tcbcolback!80!black]
					([xshift=-2mm]frame.north west) -- ([xshift=2mm]frame.north east)
					[rounded corners=1mm]-- ([xshift=1mm,yshift=-1mm]frame.north east)
					-- (frame.south east) -- (frame.south west)
					-- ([xshift=-1mm,yshift=-1mm]frame.north west)
					[sharp corners]-- cycle;
				},interior engine=empty,
		},
	fonttitle=\bfseries,
	title={#2},#1}{def}

%================================
% Solution BOX
%================================

\makeatletter
\newtcbtheorem[use counter=tcb@cnt@Theorem]{question}{Question}{enhanced,
	breakable,
	colback=white,
	colframe=myb!80!black,
	attach boxed title to top left={yshift*=-\tcboxedtitleheight},
	fonttitle=\bfseries,
	title={#2},
	boxed title size=title,
	boxed title style={%
			sharp corners,
			rounded corners=northwest,
			colback=tcbcolframe,
			boxrule=0pt,
		},
	underlay boxed title={%
			\path[fill=tcbcolframe] (title.south west)--(title.south east)
			to[out=0, in=180] ([xshift=5mm]title.east)--
			(title.center-|frame.east)
			[rounded corners=\kvtcb@arc] |-
			(frame.north) -| cycle;
		},
	#1
}{def}
\makeatother

%================================
% SOLUTION BOX
%================================

\makeatletter
\newtcolorbox{solution}{enhanced,
	breakable,
	colback=white,
	colframe=myg!80!black,
	attach boxed title to top left={yshift*=-\tcboxedtitleheight},
	title=Solution,
	boxed title size=title,
	boxed title style={%
			sharp corners,
			rounded corners=northwest,
			colback=tcbcolframe,
			boxrule=0pt,
		},
	underlay boxed title={%
			\path[fill=tcbcolframe] (title.south west)--(title.south east)
			to[out=0, in=180] ([xshift=5mm]title.east)--
			(title.center-|frame.east)
			[rounded corners=\kvtcb@arc] |-
			(frame.north) -| cycle;
		},
}
\makeatother

%================================
% Question BOX
%================================

\makeatletter
\newtcbtheorem[use counter=tcb@cnt@Theorem]{qstion}{Question}{enhanced,
	breakable,
	colback=white,
	colframe=mygr,
	attach boxed title to top left={yshift*=-\tcboxedtitleheight},
	fonttitle=\bfseries,
	title={#2},
	boxed title size=title,
	boxed title style={%
			sharp corners,
			rounded corners=northwest,
			colback=tcbcolframe,
			boxrule=0pt,
		},
	underlay boxed title={%
			\path[fill=tcbcolframe] (title.south west)--(title.south east)
			to[out=0, in=180] ([xshift=5mm]title.east)--
			(title.center-|frame.east)
			[rounded corners=\kvtcb@arc] |-
			(frame.north) -| cycle;
		},
	#1
}{def}
\makeatother

\newtcbtheorem[number within=chapter]{wconc}{Wrong Concept}{
	breakable,
	enhanced,
	colback=white,
	colframe=myr,
	arc=0pt,
	outer arc=0pt,
	fonttitle=\bfseries\sffamily\large,
	colbacktitle=myr,
	attach boxed title to top left={},
	boxed title style={
			enhanced,
			skin=enhancedfirst jigsaw,
			arc=3pt,
			bottom=0pt,
			interior style={fill=myr}
		},
	#1
}{def}



%================================
% NOTE BOX
%================================

\usetikzlibrary{arrows,calc,shadows.blur}
\tcbuselibrary{skins}
\newtcolorbox{note}[1][]{%
	enhanced jigsaw,
	colback=gray!20!white,%
	colframe=gray!80!black,
	size=small,
	boxrule=1pt,
	title=\colorbox{white!100}{\textbf{ Remarque }},
	halign title=flush center,
	coltitle=black,
	breakable,
	drop shadow=black!50!white,
	attach boxed title to top left={xshift=1cm,yshift=-\tcboxedtitleheight/2,yshifttext=-\tcboxedtitleheight/2},
	minipage boxed title=2.6cm,
	boxed title style={%
			colback=white,
			size=fbox,
			boxrule=1pt,
			boxsep=2pt,
			underlay={%
					\coordinate (dotA) at ($(interior.west) + (-0.5pt,0)$);
					\coordinate (dotB) at ($(interior.east) + (0.5pt,0)$);
					\begin{scope}
						\clip (interior.north west) rectangle ([xshift=3ex]interior.east);
						\filldraw [white, blur shadow={shadow opacity=60, shadow yshift=-.75ex}, rounded corners=2pt] (interior.north west) rectangle (interior.south east);
					\end{scope}
					\begin{scope}[gray!80!black]
						\fill (dotA) circle (2pt);
						\fill (dotB) circle (2pt);
					\end{scope}
				},
		},
	#1,
}

%================================
% STRATÉGIE BOX
%================================

\usetikzlibrary{arrows,calc,shadows.blur}
\tcbuselibrary{skins}
\newtcolorbox{strategy}[1][]{%
	enhanced jigsaw,
	colback=myb!20!white,%
	colframe=gray!80!black,
	size=small,
	boxrule=1pt,
	title=\colorbox{white!100}{\textbf{ Stratégie }},
	halign title=flush center,
	coltitle=black,
	breakable,
	drop shadow=black!50!white,
	attach boxed title to top left={xshift=1cm,yshift=-\tcboxedtitleheight/2,yshifttext=-\tcboxedtitleheight/2},
	minipage boxed title=2.5cm,
	boxed title style={%
			colback=white,
			size=fbox,
			boxrule=1pt,
			boxsep=2pt,
			underlay={%
					\coordinate (dotA) at ($(interior.west) + (-0.5pt,0)$);
					\coordinate (dotB) at ($(interior.east) + (0.5pt,0)$);
					\begin{scope}
						\clip (interior.north west) rectangle ([xshift=3ex]interior.east);
						\filldraw [white, blur shadow={shadow opacity=60, shadow yshift=-.75ex}, rounded corners=2pt] (interior.north west) rectangle (interior.south east);
					\end{scope}
					\begin{scope}[gray!80!black]
						\fill (dotA) circle (2pt);
						\fill (dotB) circle (2pt);
					\end{scope}
				},
		},
	#1,
}

%================================
% MÉTHODE BOX
%================================

\usetikzlibrary{arrows,calc,shadows.blur}
\tcbuselibrary{skins}
\newtcolorbox{methode}[1][]{%
	enhanced jigsaw,
	colback=white,%
	colframe=gray!80!black,
	size=small,
	boxrule=1pt,
	title=\textbf{Méthode},
	halign title=flush center,
	coltitle=black,
	breakable,
	drop shadow=black!50!white,
	attach boxed title to top left={xshift=1cm,yshift=-\tcboxedtitleheight/2,yshifttext=-\tcboxedtitleheight/2},
	minipage boxed title=2.5cm,
	boxed title style={%
			colback=white,
			size=fbox,
			boxrule=1pt,
			boxsep=2pt,
			underlay={%
					\coordinate (dotA) at ($(interior.west) + (-0.5pt,0)$);
					\coordinate (dotB) at ($(interior.east) + (0.5pt,0)$);
					\begin{scope}
						\clip (interior.north west) rectangle ([xshift=3ex]interior.east);
						\filldraw [white, blur shadow={shadow opacity=60, shadow yshift=-.75ex}, rounded corners=2pt] (interior.north west) rectangle (interior.south east);
					\end{scope}
					\begin{scope}[gray!80!black]
						\fill (dotA) circle (2pt);
						\fill (dotB) circle (2pt);
					\end{scope}
				},
		},
	#1,
}

%%%%%%%%%%%%%%%%%%%%%%%%%%%%%%%%%%%%%%%%%%%
% TABLE OF CONTENTS
%%%%%%%%%%%%%%%%%%%%%%%%%%%%%%%%%%%%%%%%%%%

\usepackage{tikz}

\definecolor{doc}{RGB}{0,60,110}
\usepackage{titletoc}
\contentsmargin{0cm}
\titlecontents{chapter}[3.7pc]
{\addvspace{30pt}%
	\begin{tikzpicture}[remember picture, overlay]%
		\draw[fill=doc!60,draw=doc!60] (-7,-.1) rectangle (-0.2,.6);%
		\pgftext[left,x=-3.5cm,y=0.2cm]{\color{white}\Large\sc\bfseries Chapitre\ \thecontentslabel};%
	\end{tikzpicture}\color{doc!60}\large\sc\bfseries}%
{}
{}
{\;\titlerule\;\large\sc\bfseries Page \thecontentspage
	\begin{tikzpicture}[remember picture, overlay]
		\draw[fill=doc!60,draw=doc!60] (2pt,0) rectangle (4,0.1pt);
	\end{tikzpicture}}%
\titlecontents{section}[3.7pc]
{\addvspace{2pt}}
{\contentslabel[\thecontentslabel]{2pc}}
{}
{\hfill\small \thecontentspage}
[]
\titlecontents*{subsection}[3.7pc]
{\addvspace{-1pt}\small}
{}
{}
{\ --- \small\thecontentspage}
[ \textbullet\ ][]

\makeatletter
\renewcommand{\tableofcontents}{%
	\chapter*{%
	  \vspace*{-20\p@}%
	  \begin{tikzpicture}[remember picture, overlay]%
		  \pgftext[right,x=15cm,y=0.2cm]{\color{doc!60}\Huge\sc\bfseries \contentsname};%
		  \draw[fill=doc!60,draw=doc!60] (13,-.75) rectangle (20,1);%
		  \clip (13,-.75) rectangle (20,1);
		  \pgftext[right,x=15cm,y=0.2cm]{\color{white}\Huge\sc\bfseries \contentsname};%
	  \end{tikzpicture}}%
	\@starttoc{toc}}
\makeatother


%%%%%%%%%%%%%%%%%%%%%%%%%%%%%%%%%%%%%%%%%%%
% MINTED FOR PYTHON ALGORITHMS
%%%%%%%%%%%%%%%%%%%%%%%%%%%%%%%%%%%%%%%%%%%

\usepackage{tcolorbox}
\tcbuselibrary{minted,breakable,xparse,skins}
\definecolor{bg}{gray}{0.95}
\DeclareTCBListing{mintedbox}{O{}m!O{}}{%
  breakable=true,
  listing engine=minted,
  listing only,
  minted language=#2,
  minted style=default,
  minted options={%
    linenos,
    gobble=0,
    breaklines=true,
    breakafter=,,
    fontsize=\small,
    numbersep=8pt,
    #1},
  boxsep=0pt,
  left skip=0pt,
  right skip=0pt,
  left=25pt,
  right=0pt,
  top=3pt,
  bottom=3pt,
  arc=5pt,
  leftrule=0pt,
  rightrule=0pt,
  bottomrule=2pt,
  toprule=2pt,
  colback=bg,
  colframe=orange!70,
  enhanced,
  overlay={%
    \begin{tcbclipinterior}
    \fill[orange!20!white] (frame.south west) rectangle ([xshift=20pt]frame.north west);
    \end{tcbclipinterior}},
  #3}
  
  
 % for braces
\usetikzlibrary{decorations.pathreplacing}

\def\arraystretch{1}
\setlength\tabcolsep{10pt}

\SetDate[22/02/2026]
\reversemarginpar
\setlength{\marginparsep}{.5cm}

\begin{document}
\pagestyle{fancy}
\fancyhead[L]{Seconde}
\fancyhead[C]{\textbf{Évaluation blanche — Vecteurs}}
\fancyhead[R]{\today}

\null\vspace{-20pt}
Consignes particulières : 
\begin{itemize}[label=$\bullet$]
	\item 
	La calculatrice est {interdite}.
	\item
	Sauf mention du contraire, toutes les réponses doivent être justifiées.
	%\item
	%L'exercice \ref{exe:prop-fond} peut être entièrement fait sur la feuille d'évaluation.
	\item
	L'évaluation fait \pageref{lastpage} page. La somme des points est \total{points}.
\end{itemize}

\marginpar{[pts]}
\hrule


\exe{6}{
	Considérons un point $A(0 ; -4)$, un vecteur $v = \pvec{1}{2}$, et une fonction $f(x) = 2x - 4$ définie sur $\R$.
	
	Posons $(d)$ la droite passant par $A$ et dirigée par $v$ et $\C_f$ la courbe représentative de $f$.
	
	\begin{enumerate}
		\item
		Donner $3$ points distincts appartenant à $(d)$.
		\item
		Donner un point $B$ appartenant à $\C_f$ distinct des points déjà cités
		\item
		Montrer que $B$ appartient à $(d)$.
		\item$(\star)$
		Montrer que $(d) = \C_f$.
	\end{enumerate}
}{exe:v-directeur}{
	
	\begin{enumerate}
		\item
		D'après le cours, $(d)$ est l'ensemble des translatés de $A$ par un multiple de $v$ :
			\[ (d) = \bigset{ A + k\cdot v \tq k \in \R}. \]
		En particulier, les points $A, A+v = (1 ; -2),$ et $A - v = (-1 ; -6)$ appartiennent à $(d)$.
		
		Notons que $A + \frac{\pi}{4}v = \left( \frac{\pi}4 ; \frac{-8 + \pi}2 \right)$ appartient aussi à $(d)$.
		
		\item
		Par définition, $\C_f$ est l'ensemble des points $\bigl(x ; f(x)\bigr)$ avec $x \in \R$.
		Prenons $x=7$ (notre nombre préféré) : le point $B(7 ; 10)$ appartient à $\C_f$.
		
		\item
		Pour montrer que $B$ appartient à $(d)$, il suffit de calculer $\vec{AB}$ et de montrer qu'il est colinéaire à $v$.
			\[ \vec{AB} = \pvec{7-0}{10-(-4)} = \pvec{7}{14} = 7u \]
		
		\item$(\star)$
		D'une part, la droite $(d)$ est l'ensemble des $A+k\cdot v$ avec $k\in\R$.
		C'est donc l'ensemble des points de la forme
			\[ A + k\cdot v = (0 ; -4) + k\pvec{1}{2} = (0 ; -4) + \pvec{k}{2k} = (k ; 2k-4). \]
		
		D'autre part, la courbe représentative $\C_f$ est l'ensemble des points de la forme 
			\[ \bigl(x ; f(x)\bigr) = (x ; 2x - 4) \]
		avec $x \in \R$.
		
		Ces ensembles sont donc égaux ; leur seul différence est la lettre utilisée pour les caractériser !
	\end{enumerate}

}

\exe{5}{
	Un élève de Seconde fournit la production écrite suivante.
	Celui présente quelques erreurs.
	
	\fbox{\parbox{\textwidth}{
		\textbf{\oldunderline{Énoncé :}}
		
		Considérons les points $A(-1; -6), B(-3; -12)$, et $C(5; 12)$.
		
		Le point $C$ appartient-il à la droite $(AB)$ ?
		
		
		\textbf{\underline{Réponse de l'élève :}}
		
		D'abord, je calcule le vecteur $\vec{AB}$ qui dirige la droite $(AB)$ :
			\begin{align}\label{eq:1}
				\vec{AB} = \pvec{x_A - x_B}{y_A - y_B} = \pvec{-1 - (-3)}{-6 - (-12)} = \pvec{4}{6}.
			\end{align}
		Ensuite, je calcule le vecteur $\vec{AC}$ :
			\begin{align}\label{eq:2}
				\vec{AC} = \pvec{x_C - x_A}{y_C-y_A} = \pvec{5-(-1)}{12-(-6)} = \pvec{4}{-6}.
			\end{align}
		Enfin, le déterminant $\det\bigl(\vec{AB}, \vec{AC}\bigr)$ est égal à
			\begin{align}\label{eq:3}
				\det\bigl(\vec{AB}, \vec{AC}\bigr) = (4)(-6) + (6)(4) = -24 + 24 = 0,
			\end{align}
		duquel je déduis que les vecteurs $\vec{AB}$ et $\vec{AC}$ sont colinéaires.
		
		En conclusion, les droites $(AB)$ et $(AC)$ sont parallèles et passent toutes les deux par $A$ : elles sont donc confondues et les points $A, B, C$ sont alignés.		
	
	}}
	
	\begin{enumerate}
		\item 
		Identifier les erreurs commises par l'élève. Une erreur par ligne de calcul (dénotées \eqref{eq:1}, \eqref{eq:2} et \eqref{eq:3}) est attendue.
		\item
		Répondre à l'énoncé.
	\end{enumerate}
}{exe:erreurs}{
	\begin{enumerate}
		\item
		La ligne \eqref{eq:1} contient deux erreurs : la formule du vecteur $\vec{AB}$ est fausse ($x_B - x_A$ et non $x_A - x_B$) ; et le calcul $-1-(-3)$ vaut $-1+3 = 2$ et non 4.
		
		La ligne \eqref{eq:2} contient également deux erreurs de signe : $5-(-1) = 5+1 = 6$ et $12-(-6) = 18$.
		
		La ligne \eqref{eq:3} utilise la formule $ad+bc$ qui n'est pas la bonne formule du déterminant $ad-bc$.
		
		\item
		En corrigeant les erreurs, on trouve $\vec{AB} = \pvec{-2}{-6}$ et $\vec{AC} = \pvec{6}{18}$.
		Comme $\vec{AC} = -3\vec{AC}$, les vecteurs sont colinéaires et la conclusion de l'élève reste correcte.
		Un calcul de déterminant permet également de conclure : 
			\[ \det\bigl(\vec{AB},\vec{AC}\bigr) = (-2)(18)-(-6)(6) = -36+36 = 0. \]	
	\end{enumerate}
}

\newpage

\exe{6}{
	Soit $A(2 ; -3)$ un point et $u = \pvec{-2}{4}, v = \frac17\pvec{6}{3}$ deux vecteurs.
	
	\begin{enumerate}
		\item
		Calculer les coordonnées \textbf{exactes} des points $B = A+u$ et $C = A+v$.
		\item
		Représenter les points $A, B, C$ dans un repère.
		Il est possible d'estimer les valeurs pour placer approximativement les points.
		\item
		Démontrer que le triangle $ABC$ est rectangle en $A$.
	\end{enumerate}
}{exe:Pyth}{
	\begin{enumerate}
		\item
		Les coordonnées de $B$ sont données par
			\[ B = A + u = (2 ; -3) + \pvec{-2}4 = (2-2 ; -3+4) = (0 ; 1). \]
		Celles de $C$ par
			\[ C = A + v = (2 ; -3) + \dfrac17\pvec{6}{3} = \left( 2 + \dfrac67 ; -3 + \dfrac37 \right) = \left( \dfrac{20}{7} ; \dfrac{-18}7 \right). \]
		
		\item
		On approxime $C \approx (3 ; -2,5)$ pour placer les points approximativement ci-dessous.

		\item
		Calculons le carré de la longueur de chaque côté :
			\begin{align*}
				AB^2 = \Vert \vec{AB}\Vert^2 = \Vert u \Vert^2 = (2)^2 + (4)^2 = 20, \\
				AC^2 = \Vert \vec{AC} \Vert^2 = \Vert v \Vert^2 = \left(\dfrac67\right)^2 + \left(\dfrac37\right)^2 = \dfrac{6^2 + 3^2}{7^2} = \dfrac{45}{49}, \\
				BC^2 = \Vert \vec{BC} \Vert^2 = \left\Vert \dfrac17\pvec{20}{-25} \right\Vert^2 = \dfrac{20^2 + 25^2}{7^2} = \dfrac{400 + 625}{49} = \dfrac{1025}{49},
			\end{align*}
		où on a utilisé que $25^2 = (20+5)^2 = 20^2 + 5^2 + 2(20)(5) = 400+25+200 = 625$ d'après notre super identité remarquable favorite.
		
		La réciproque du théorème de Pythagore demande de vérifier l'égalité $BC^2 \stackrel{?}{=} AC^2 + AB^2$, ce que l'on fait calmement car on aime les fractions :
			\[ AC^2 + AB^2 = 13 + \dfrac{45}{49} = \dfrac{20\times49 + 45}{49} = \dfrac{980+45}{49} = \dfrac{1025}{49} = BC^2, \]
		où on a utilisé que $20\times49 = 20\times50 - 20 = 980$.

	\end{enumerate}

	\begin{center}
	\begin{tikzpicture}[>=stealth, scale=.875]
		\begin{axis}[xmin = -0.5, xmax=3.5, ymin=-3.5, ymax=1.5, axis x line=middle, axis y line=middle, axis line style=<->, xlabel={}, ylabel={}, xtick distance=1, ytick distance=1, grid=both, x=2cm, y=2cm]
		\addplot[BLUE_E, mark=*, mark size = 1] (2,-3) node[right] {$A(2 ; -3)$};
		\addplot[GREEN_E, mark=*, mark size = 1] (0,1) node[above] {$B(0;1)$};
		\addplot[RED_E, mark=*, mark size = 1] (20/7,-18/7) node[right] {$C\approx(3 ; -2,5)$};
		\addplot[dashed, thick, GREY] (2,-3) -- (axis cs:0,1) -- (axis cs:20/7, -18/7) -- (axis cs:2,-3);
		\end{axis}
	\end{tikzpicture}
	\end{center}
}

\exe{4}{
	Construire en bas de page les vecteurs 
		\begin{align*}
			a = u+v+w, && b = -2u, && \text{ et } && c = 3u - \dfrac12v - \dfrac13w,
		\end{align*}
	où les vecteurs $u, v, w$ sont donnés dans le plan ci-dessous.
	Montrer les constructions.
	
	\begin{center}
		\begin{tikzpicture}[>=stealth, scale=1]
		\begin{axis}[xmin = -10, xmax=10, ymin=-10, ymax=10, axis x line=none, axis y line=none, axis line style=<->, xlabel={}, ylabel={}, ticks = none]
			\draw[very thick, ->, GREEN_E] (axis cs:-1,-1) -- (axis cs: 1,-4) node[right] {$u$};
			% u = (2, -3)
			\draw[very thick, ->, RED_E] (axis cs: -2,8) -- (axis cs: -5,-1) node[left] {$v$};
			% v = (-3, -9)
			\draw[very thick, ->, BLUE_E] (axis cs:0,1) -- (axis cs: 8,5) node[above] {$w$};
			% w = (8, 4)
		\end{axis}
		\end{tikzpicture}
	\end{center}
	\vspace{-1cm}
	\underline{\textbf{Constructions :}}
	%\vspace{5cm}
}{exe:Chasles}{
	On obtient quelque chose comme ça. 
	Multiplier par un scalaire $k$ multiplie la norme par $|k|$ et change le sens si $k<0$.
	Pour ajouter deux vecteurs, on prend un point au hasard, et on le translate par l'un puis par l'autre et on trace la translation qui envoie le point choisi au point d'arrivée (on met donc les vecteurs bout à bout).
	\begin{center}
		\begin{tikzpicture}[>=stealth, scale=1]
		\begin{axis}[xmin = -10, xmax=10, ymin=-10, ymax=10, axis x line=none, axis y line=none, axis line style=<->, xlabel={}, ylabel={}, ticks = none]
			\draw[very thick, ->, PINK] (axis cs:-5,0) -- (axis cs: 2,-8) node[midway, right] {$u+v+w$};
			\draw[very thick, ->, GOLD_E] (axis cs:0,0) -- (axis cs: -4,6) node[midway, right] {$-2u$};
			\draw[very thick, ->, TEAL_E] (axis cs:5,0) -- (axis cs: 9.83,-5.83) node[midway, right] {$3u - \frac12v - \frac13w$};
		\end{axis}
		\end{tikzpicture}
	\end{center}
}

%%%%%%%%%%%

\label{lastpage}
\newpage
\fancyhead[C]{\textbf{Solutions}}
\shipoutAnswer
	
\end{document}

%!TEX encoding = UTF8
%!TEX root =notes.tex


%%%%%%%%%%%%%%%%%%%%%%%%%%%%%%%%%
% PACKAGE IMPORTS
%%%%%%%%%%%%%%%%%%%%%%%%%%%%%%%%%


\usepackage[french]{babel}

\usepackage[tmargin=2cm,rmargin=1in,lmargin=1in,margin=0.85in,bmargin=2cm,footskip=.2in]{geometry}
\usepackage{amsmath,amsfonts,amsthm,amssymb,mathtools}
\usepackage[varbb]{newpxmath}
\usepackage{xfrac}
\usepackage[makeroom]{cancel}
\usepackage{mathtools}
\usepackage{bookmark}
\usepackage{enumitem}
\usepackage{hyperref,theoremref}
\hypersetup{
	pdftitle={Assignment},
	colorlinks=true, linkcolor=doc!90,
	bookmarksnumbered=true,
	bookmarksopen=true
}
\usepackage[most,many,breakable]{tcolorbox}
\usepackage{xcolor}
\usepackage{varwidth}
\usepackage{varwidth}
\usepackage{etoolbox}
%\usepackage{authblk}
\usepackage{nameref}
\usepackage{multicol,array}
\usepackage{tikz-cd}
\usepackage[ruled,vlined,linesnumbered]{algorithm2e}
\usepackage{comment} % enables the use of multi-line comments (\ifx \fi) 
\usepackage{import}
\usepackage{xifthen}
\usepackage{pdfpages}
\usepackage{transparent}


\newcommand\mycommfont[1]{\footnotesize\ttfamily\textcolor{blue}{#1}}
\SetCommentSty{mycommfont}
\newcommand{\incfig}[1]{%
    \def\svgwidth{\columnwidth}
    \import{./figures/}{#1.pdf_tex}
}

\usepackage{tikzsymbols}
%\renewcommand\qedsymbol{$\Laughey$}


%\usepackage{import}
%\usepackage{xifthen}
%\usepackage{pdfpages}
%\usepackage{transparent}


%%%%%%%%%%%%%%%%%%%%%%%%%%%%%%
% SELF MADE COLORS
%%%%%%%%%%%%%%%%%%%%%%%%%%%%%%



\definecolor{myg}{RGB}{56, 140, 70}
\definecolor{myb}{RGB}{45, 111, 177}
\definecolor{myr}{RGB}{199, 68, 64}
\definecolor{mytheorembg}{HTML}{F2F2F9}
\definecolor{mytheoremfr}{HTML}{00007B}
\definecolor{mylenmabg}{HTML}{FFFAF8}
\definecolor{mylenmafr}{HTML}{983b0f}
\definecolor{mypropbg}{HTML}{f2fbfc}
\definecolor{mypropfr}{HTML}{191971}
\definecolor{myexamplebg}{HTML}{F2FBF8}
\definecolor{myexamplefr}{HTML}{88D6D1}
\definecolor{myexampleti}{HTML}{2A7F7F}
\definecolor{mydefinitbg}{HTML}{E5E5FF}
\definecolor{mydefinitfr}{HTML}{3F3FA3}
\definecolor{notesgreen}{RGB}{0,162,0}
\definecolor{myp}{RGB}{197, 92, 212}
\definecolor{mygr}{HTML}{2C3338}
\definecolor{myred}{RGB}{127,0,0}
\definecolor{myyellow}{RGB}{169,121,69}
\definecolor{myexercisebg}{HTML}{F2FBF8}
\definecolor{myexercisefg}{HTML}{88D6D1}


%%%%%%%%%%%%%%%%%%%%%%%%%%%%
% TCOLORBOX SETUPS
%%%%%%%%%%%%%%%%%%%%%%%%%%%%

\setlength{\parindent}{1cm}
%================================
% THEOREM BOX
%================================

\tcbuselibrary{theorems,skins,hooks}
\newtcbtheorem[number within=chapter]{Theorem}{Théorème}
{%
	enhanced,
	breakable,
	colback = mytheorembg,
	frame hidden,
	boxrule = 0sp,
	borderline west = {2pt}{0pt}{mytheoremfr},
	sharp corners,
	detach title,
	before upper = \tcbtitle\par\smallskip,
	coltitle = mytheoremfr,
	fonttitle = \bfseries\sffamily,
	description font = \mdseries,
	separator sign none,
	segmentation style={solid, mytheoremfr},
}
{th}


\tcbuselibrary{theorems,skins,hooks}
\newtcolorbox{Theoremcon}
{%
	enhanced
	,breakable
	,colback = mytheorembg
	,frame hidden
	,boxrule = 0sp
	,borderline west = {2pt}{0pt}{mytheoremfr}
	,sharp corners
	,description font = \mdseries
	,separator sign none
}

%================================
% Corollery
%================================
\tcbuselibrary{theorems,skins,hooks}
\newtcbtheorem[use counter=tcb@cnt@Theorem]{Corollary}{Corollaire}
{%
	enhanced
	,breakable
	,colback = myp!10
	,frame hidden
	,boxrule = 0sp
	,borderline west = {2pt}{0pt}{myp!85!black}
	,sharp corners
	,detach title
	,before upper = \tcbtitle\par\smallskip
	,coltitle = myp!85!black
	,fonttitle = \bfseries\sffamily
	,description font = \mdseries
	,separator sign none
	,segmentation style={solid, myp!85!black}
}
{th}

%================================
% LENMA
%================================

\tcbuselibrary{theorems,skins,hooks}
\newtcbtheorem[use counter=tcb@cnt@Theorem]{Lemma}{Lemme}
{%
	enhanced,
	breakable,
	colback = mylenmabg,
	frame hidden,
	boxrule = 0sp,
	borderline west = {2pt}{0pt}{mylenmafr},
	sharp corners,
	detach title,
	before upper = \tcbtitle\par\smallskip,
	coltitle = mylenmafr,
	fonttitle = \bfseries\sffamily,
	description font = \mdseries,
	separator sign none,
	segmentation style={solid, mylenmafr},
}
{th}


%================================
% PROPOSITION
%================================

\tcbuselibrary{theorems,skins,hooks}
\newtcbtheorem[use counter=tcb@cnt@Theorem]{Prop}{Proposition}
{%
	enhanced,
	breakable,
	colback = mypropbg,
	frame hidden,
	boxrule = 0sp,
	borderline west = {2pt}{0pt}{mypropfr},
	sharp corners,
	detach title,
	before upper = \tcbtitle\par\smallskip,
	coltitle = mypropfr,
	fonttitle = \bfseries\sffamily,
	description font = \mdseries,
	separator sign none,
	segmentation style={solid, mypropfr},
}
{th}


%================================
% CLAIM
%================================

\tcbuselibrary{theorems,skins,hooks}
\newtcbtheorem[use counter=tcb@cnt@Theorem]{claim}{Claim}
{%
	enhanced
	,breakable
	,colback = myg!10
	,frame hidden
	,boxrule = 0sp
	,borderline west = {2pt}{0pt}{myg}
	,sharp corners
	,detach title
	,before upper = \tcbtitle\par\smallskip
	,coltitle = myg!85!black
	,fonttitle = \bfseries\sffamily
	,description font = \mdseries
	,separator sign none
	,segmentation style={solid, myg!85!black}
}
{th}



%================================
% Exercise
%================================

\tcbuselibrary{theorems,skins,hooks}
\newtcbtheorem[use counter=tcb@cnt@Theorem]{Exercise}{Exercice}
{%
	enhanced,
	breakable,
	colback = myexercisebg,
	frame hidden,
	boxrule = 0sp,
	borderline west = {2pt}{0pt}{myexercisefg},
	sharp corners,
	detach title,
	before upper = \tcbtitle\par\smallskip,
	coltitle = myexercisefg,
	fonttitle = \bfseries\sffamily,
	description font = \mdseries,
	separator sign none,
	segmentation style={solid, myexercisefg},
}
{th}

%================================
% EXAMPLE BOX
%================================

\newtcbtheorem[use counter=tcb@cnt@Theorem]{Example}{Exemple}
{%
	colback = myexamplebg
	,breakable
	,colframe = myexamplefr
	,coltitle = myexampleti
	,boxrule = 1pt
	,sharp corners
	,detach title
	,before upper=\tcbtitle\par\smallskip
	,fonttitle = \bfseries
	,description font = \mdseries
	,separator sign none
	,description delimiters parenthesis
}
{ex}

%================================
% DEFINITION BOX
%================================

\newtcbtheorem[use counter=tcb@cnt@Theorem]{Definition}{Définition}{enhanced,
	before skip=2mm,after skip=2mm, colback=red!5,colframe=red!80!black,boxrule=0.5mm,
	attach boxed title to top left={xshift=1cm,yshift*=1mm-\tcboxedtitleheight}, varwidth boxed title*=-3cm,
	boxed title style={frame code={
					\path[fill=tcbcolback]
					([yshift=-1mm,xshift=-1mm]frame.north west)
					arc[start angle=0,end angle=180,radius=1mm]
					([yshift=-1mm,xshift=1mm]frame.north east)
					arc[start angle=180,end angle=0,radius=1mm];
					\path[left color=tcbcolback!60!black,right color=tcbcolback!60!black,
						middle color=tcbcolback!80!black]
					([xshift=-2mm]frame.north west) -- ([xshift=2mm]frame.north east)
					[rounded corners=1mm]-- ([xshift=1mm,yshift=-1mm]frame.north east)
					-- (frame.south east) -- (frame.south west)
					-- ([xshift=-1mm,yshift=-1mm]frame.north west)
					[sharp corners]-- cycle;
				},interior engine=empty,
		},
	fonttitle=\bfseries,
	title={#2},#1}{def}

%================================
% Solution BOX
%================================

\makeatletter
\newtcbtheorem[use counter=tcb@cnt@Theorem]{question}{Question}{enhanced,
	breakable,
	colback=white,
	colframe=myb!80!black,
	attach boxed title to top left={yshift*=-\tcboxedtitleheight},
	fonttitle=\bfseries,
	title={#2},
	boxed title size=title,
	boxed title style={%
			sharp corners,
			rounded corners=northwest,
			colback=tcbcolframe,
			boxrule=0pt,
		},
	underlay boxed title={%
			\path[fill=tcbcolframe] (title.south west)--(title.south east)
			to[out=0, in=180] ([xshift=5mm]title.east)--
			(title.center-|frame.east)
			[rounded corners=\kvtcb@arc] |-
			(frame.north) -| cycle;
		},
	#1
}{def}
\makeatother

%================================
% SOLUTION BOX
%================================

\makeatletter
\newtcolorbox{solution}{enhanced,
	breakable,
	colback=white,
	colframe=myg!80!black,
	attach boxed title to top left={yshift*=-\tcboxedtitleheight},
	title=Solution,
	boxed title size=title,
	boxed title style={%
			sharp corners,
			rounded corners=northwest,
			colback=tcbcolframe,
			boxrule=0pt,
		},
	underlay boxed title={%
			\path[fill=tcbcolframe] (title.south west)--(title.south east)
			to[out=0, in=180] ([xshift=5mm]title.east)--
			(title.center-|frame.east)
			[rounded corners=\kvtcb@arc] |-
			(frame.north) -| cycle;
		},
}
\makeatother

%================================
% Question BOX
%================================

\makeatletter
\newtcbtheorem[use counter=tcb@cnt@Theorem]{qstion}{Question}{enhanced,
	breakable,
	colback=white,
	colframe=mygr,
	attach boxed title to top left={yshift*=-\tcboxedtitleheight},
	fonttitle=\bfseries,
	title={#2},
	boxed title size=title,
	boxed title style={%
			sharp corners,
			rounded corners=northwest,
			colback=tcbcolframe,
			boxrule=0pt,
		},
	underlay boxed title={%
			\path[fill=tcbcolframe] (title.south west)--(title.south east)
			to[out=0, in=180] ([xshift=5mm]title.east)--
			(title.center-|frame.east)
			[rounded corners=\kvtcb@arc] |-
			(frame.north) -| cycle;
		},
	#1
}{def}
\makeatother

\newtcbtheorem[number within=chapter]{wconc}{Wrong Concept}{
	breakable,
	enhanced,
	colback=white,
	colframe=myr,
	arc=0pt,
	outer arc=0pt,
	fonttitle=\bfseries\sffamily\large,
	colbacktitle=myr,
	attach boxed title to top left={},
	boxed title style={
			enhanced,
			skin=enhancedfirst jigsaw,
			arc=3pt,
			bottom=0pt,
			interior style={fill=myr}
		},
	#1
}{def}



%================================
% NOTE BOX
%================================

\usetikzlibrary{arrows,calc,shadows.blur}
\tcbuselibrary{skins}
\newtcolorbox{note}[1][]{%
	enhanced jigsaw,
	colback=gray!20!white,%
	colframe=gray!80!black,
	size=small,
	boxrule=1pt,
	title=\colorbox{white!100}{\textbf{ Remarque }},
	halign title=flush center,
	coltitle=black,
	breakable,
	drop shadow=black!50!white,
	attach boxed title to top left={xshift=1cm,yshift=-\tcboxedtitleheight/2,yshifttext=-\tcboxedtitleheight/2},
	minipage boxed title=2.6cm,
	boxed title style={%
			colback=white,
			size=fbox,
			boxrule=1pt,
			boxsep=2pt,
			underlay={%
					\coordinate (dotA) at ($(interior.west) + (-0.5pt,0)$);
					\coordinate (dotB) at ($(interior.east) + (0.5pt,0)$);
					\begin{scope}
						\clip (interior.north west) rectangle ([xshift=3ex]interior.east);
						\filldraw [white, blur shadow={shadow opacity=60, shadow yshift=-.75ex}, rounded corners=2pt] (interior.north west) rectangle (interior.south east);
					\end{scope}
					\begin{scope}[gray!80!black]
						\fill (dotA) circle (2pt);
						\fill (dotB) circle (2pt);
					\end{scope}
				},
		},
	#1,
}

%================================
% STRATÉGIE BOX
%================================

\usetikzlibrary{arrows,calc,shadows.blur}
\tcbuselibrary{skins}
\newtcolorbox{strategy}[1][]{%
	enhanced jigsaw,
	colback=myb!20!white,%
	colframe=gray!80!black,
	size=small,
	boxrule=1pt,
	title=\colorbox{white!100}{\textbf{ Stratégie }},
	halign title=flush center,
	coltitle=black,
	breakable,
	drop shadow=black!50!white,
	attach boxed title to top left={xshift=1cm,yshift=-\tcboxedtitleheight/2,yshifttext=-\tcboxedtitleheight/2},
	minipage boxed title=2.5cm,
	boxed title style={%
			colback=white,
			size=fbox,
			boxrule=1pt,
			boxsep=2pt,
			underlay={%
					\coordinate (dotA) at ($(interior.west) + (-0.5pt,0)$);
					\coordinate (dotB) at ($(interior.east) + (0.5pt,0)$);
					\begin{scope}
						\clip (interior.north west) rectangle ([xshift=3ex]interior.east);
						\filldraw [white, blur shadow={shadow opacity=60, shadow yshift=-.75ex}, rounded corners=2pt] (interior.north west) rectangle (interior.south east);
					\end{scope}
					\begin{scope}[gray!80!black]
						\fill (dotA) circle (2pt);
						\fill (dotB) circle (2pt);
					\end{scope}
				},
		},
	#1,
}

%================================
% MÉTHODE BOX
%================================

\usetikzlibrary{arrows,calc,shadows.blur}
\tcbuselibrary{skins}
\newtcolorbox{methode}[1][]{%
	enhanced jigsaw,
	colback=white,%
	colframe=gray!80!black,
	size=small,
	boxrule=1pt,
	title=\textbf{Méthode},
	halign title=flush center,
	coltitle=black,
	breakable,
	drop shadow=black!50!white,
	attach boxed title to top left={xshift=1cm,yshift=-\tcboxedtitleheight/2,yshifttext=-\tcboxedtitleheight/2},
	minipage boxed title=2.5cm,
	boxed title style={%
			colback=white,
			size=fbox,
			boxrule=1pt,
			boxsep=2pt,
			underlay={%
					\coordinate (dotA) at ($(interior.west) + (-0.5pt,0)$);
					\coordinate (dotB) at ($(interior.east) + (0.5pt,0)$);
					\begin{scope}
						\clip (interior.north west) rectangle ([xshift=3ex]interior.east);
						\filldraw [white, blur shadow={shadow opacity=60, shadow yshift=-.75ex}, rounded corners=2pt] (interior.north west) rectangle (interior.south east);
					\end{scope}
					\begin{scope}[gray!80!black]
						\fill (dotA) circle (2pt);
						\fill (dotB) circle (2pt);
					\end{scope}
				},
		},
	#1,
}

%%%%%%%%%%%%%%%%%%%%%%%%%%%%%%%%%%%%%%%%%%%
% TABLE OF CONTENTS
%%%%%%%%%%%%%%%%%%%%%%%%%%%%%%%%%%%%%%%%%%%

\usepackage{tikz}

\definecolor{doc}{RGB}{0,60,110}
\usepackage{titletoc}
\contentsmargin{0cm}
\titlecontents{chapter}[3.7pc]
{\addvspace{30pt}%
	\begin{tikzpicture}[remember picture, overlay]%
		\draw[fill=doc!60,draw=doc!60] (-7,-.1) rectangle (-0.2,.6);%
		\pgftext[left,x=-3.5cm,y=0.2cm]{\color{white}\Large\sc\bfseries Chapitre\ \thecontentslabel};%
	\end{tikzpicture}\color{doc!60}\large\sc\bfseries}%
{}
{}
{\;\titlerule\;\large\sc\bfseries Page \thecontentspage
	\begin{tikzpicture}[remember picture, overlay]
		\draw[fill=doc!60,draw=doc!60] (2pt,0) rectangle (4,0.1pt);
	\end{tikzpicture}}%
\titlecontents{section}[3.7pc]
{\addvspace{2pt}}
{\contentslabel[\thecontentslabel]{2pc}}
{}
{\hfill\small \thecontentspage}
[]
\titlecontents*{subsection}[3.7pc]
{\addvspace{-1pt}\small}
{}
{}
{\ --- \small\thecontentspage}
[ \textbullet\ ][]

\makeatletter
\renewcommand{\tableofcontents}{%
	\chapter*{%
	  \vspace*{-20\p@}%
	  \begin{tikzpicture}[remember picture, overlay]%
		  \pgftext[right,x=15cm,y=0.2cm]{\color{doc!60}\Huge\sc\bfseries \contentsname};%
		  \draw[fill=doc!60,draw=doc!60] (13,-.75) rectangle (20,1);%
		  \clip (13,-.75) rectangle (20,1);
		  \pgftext[right,x=15cm,y=0.2cm]{\color{white}\Huge\sc\bfseries \contentsname};%
	  \end{tikzpicture}}%
	\@starttoc{toc}}
\makeatother


%%%%%%%%%%%%%%%%%%%%%%%%%%%%%%%%%%%%%%%%%%%
% MINTED FOR PYTHON ALGORITHMS
%%%%%%%%%%%%%%%%%%%%%%%%%%%%%%%%%%%%%%%%%%%

\usepackage{tcolorbox}
\tcbuselibrary{minted,breakable,xparse,skins}
\definecolor{bg}{gray}{0.95}
\DeclareTCBListing{mintedbox}{O{}m!O{}}{%
  breakable=true,
  listing engine=minted,
  listing only,
  minted language=#2,
  minted style=default,
  minted options={%
    linenos,
    gobble=0,
    breaklines=true,
    breakafter=,,
    fontsize=\small,
    numbersep=8pt,
    #1},
  boxsep=0pt,
  left skip=0pt,
  right skip=0pt,
  left=25pt,
  right=0pt,
  top=3pt,
  bottom=3pt,
  arc=5pt,
  leftrule=0pt,
  rightrule=0pt,
  bottomrule=2pt,
  toprule=2pt,
  colback=bg,
  colframe=orange!70,
  enhanced,
  overlay={%
    \begin{tcbclipinterior}
    \fill[orange!20!white] (frame.south west) rectangle ([xshift=20pt]frame.north west);
    \end{tcbclipinterior}},
  #3}
  
  
 % for braces
\usetikzlibrary{decorations.pathreplacing}


\SetDate[25/12/2025]
\reversemarginpar
\setlength{\marginparsep}{.5cm}

\begin{document}
\pagestyle{fancy}
\fancyhead[L]{Seconde}
\fancyhead[C]{\textbf{Évaluation blanche — Évolutions}}
\fancyhead[R]{\today}

\null\vspace{-30pt}
Consignes particulières : 
\begin{itemize}[label=$\bullet$]
	\item 
	La calculatrice est {autorisée}.
	\item
	Sauf mention du contraire, toutes les réponses doivent être justifiées.
	%\item
	%L'exercice \ref{exe:prop-fond} peut être entièrement fait sur la feuille d'évaluation.
	\item
	L'évaluation fait 2 pages. La somme des points est \total{points}.
\end{itemize}

\marginpar{[pts]}
\hrule

\exe{4}{
	Exprimer les nombre suivants sous la forme $10^x$ pour un entier $x\in\Z$.
	
	\begin{multicols}{4}
	\begin{enumerate}
		\item $10^2 \times 10^{10}$
		\item $\left(10^7\right)^2$
		\item $\dfrac{10^{12}}{10^{15}}$
		\item $\dfrac{10^{-7}}{10^{8}}$
		\item $\dfrac{10^{0}}{10^{-12}}$
		\item $\dfrac{1}{10^{-6}}$
		\item $\dfrac{10^{32}}{10^{-16}}$
		\item $\left(\dfrac1{10^5}\right)^3$
	\end{enumerate}
	\end{multicols}

}{exe:pow}{
	\begin{multicols}{4}
	\begin{enumerate}
		\item $10^{12}$
		\item $10^{14}$
		\item $10^{-3}$
		\item $10^{-15}$
		\item $10^{12}$
		\item $10^{6}$
		\item $10^{48}$
		\item $10^{-15}$
	\end{enumerate}
	\end{multicols}
}

\exe{4, difficulty=1}{
	On estime que tous les milliers d'années après la mort d'un organisme, le nombre d'atomes de carbone 14 diminue de $11\%$.
	Ajourd'hui on mesure $2$ millions d'atomes de carbone 14.

	%Répondre aux questions suivantes en arrondissant à $10^{-3}$.
	\begin{enumerate}
		\item 
		\begin{enumerate}[label=\alph*)]
			\item Donner le  coefficient multiplicateur associé à une diminution de 11\%.
			\item Combien d'atomes de carbone 14 restera-t-il après 4 000 années ?
			%\item À partir de combien de milliers d'années ne restera-t-il que 100 000 atomes de carbone 14 ? Une réponse entière est attendue.
		\end{enumerate}
		\item 
		\begin{enumerate}[label=\alph*)]
			\item Donner l'évolution réciproque d'une diminution de 11\% et le coefficient multiplicateur associé.
			\item Combien d'atomes de carbones 14 y avait-il il y a 10 000 ans ?
		\end{enumerate}
	\end{enumerate}
}{exe:C14}{
	\begin{enumerate}
		\item 
		\begin{enumerate}[label=\alph*)]
			\item 
			$CM = 1 - \frac{11}{100} = 0,89$.
			\item 
			On applique quatre diminutions de 11\% à 2 pour trouver
				\[ 2 \times 0,89^4 \approx 1,25. \]
			Il ne restera donc que 1,25 millions d'atomes de carbone 14 dans 4 000 ans.
		\end{enumerate}
		\item 
		\begin{enumerate}[label=\alph*)]
			\item
			$CM^{-1} = 0,89^{-1} \approx 1,12$.
			L'évolution réciproque est donc une augmentation de 12\%.
			\item 
			On applique 10 évolutions réciproque à 2 pour trouver
				\[ 2 \times 0,89^{-10} \approx 6,41. \]
			Il y avait donc 6,41 millions d'atomes de carbone 14 il y a 10 000 ans.
		\end{enumerate}
	\end{enumerate}
}

\exe{3, difficulty=1}{
	En vue des soldes, un magasin augmente frauduleusement ses prix avant d'appliquer une remise de 42\%.
	On compare les prix initiaux avant augmentation aux prix finaux après application des deux évolutions.
	
	Répondre aux questions suivantes en arrondissant les valeurs au centième.
	Un coefficient multiplicateur est d'abord attendu, puis un taux d'évolution.
	\begin{enumerate}
		\item 
		Quelle augmentation faut-il effectuer pour que les prix ne changent pas, avant et après évolutions ?
		\item
		Quelle augmentation faut-il effectuer pour que, au final, les prix aient augmenté de $10\%$ ?
		\item
		Quelle augmentation faut-il effectuer pour que, au final, les prix aient diminué de $22\%$ ?
	\end{enumerate}

}{exe:prix}{
	La remise de 42\% est associé au coefficient multiplicateur $CM = 1 - \frac{42}{100} = 0,58$.
	\begin{enumerate}
		\item 
		$CM^{-1} = 0,58^{-1} \approx 1,72$.
		Il faut donc effectuer une augmentation de 72\%.
		\item
		Notons $c$ le coefficient multiplicateur associé à l'augmentation recherchée.
		Celui-ci vérifie $0,58 \times c = 1,1$, d'après le théorème des évolutions successives.
		D'où $c = \frac{1,1}{0,58} \approx 1,89$.
		Il faut donc effectuer une augmentation de 89\%.
		\item
		Notons $c$ le coefficient multiplicateur associé à l'augmentation recherchée.
		Celui-ci vérifie $0,58 \times c = 0,78$, d'après le théorème des évolutions successives.
		D'où $c = \frac{0,78}{0,58} \approx 1,34$.
		Il faut donc effectuer une augmentation de 34\%.
	\end{enumerate}
}

\exe{2, difficulty=1}{
	Dans un lycée, 20\% des élèves de Terminale sont inscrits en filière STMG.
	En outre, 8\% des élèves de Terminale sont des filles inscrites en STMG.
	
	Quelle est la proportion de filles parmis les STMG ?
}{exe:prop-prop}{
	Notons $p$ la proportion recherchée.
	La proportions de filles inscrites en STMG peut être vu comme la proportion d'élèves en STMG multiplié par la proportion $p$ de filles parmis les STMG.
	En effet, les proportions de sous-populations se multiplient entre elles.
	
	D'où $0,08 = 0,2 \times p$, et donc $p = \frac{0,08}{0,2} = 0,4 = 40\%$.
}

\exe{2}{
	Considérons $E = \bigset{1 ; 2 ; 3 ; \dots ; 35}$ l'ensemble des élèves de la classe.
	Soit $F \subseteq E$ un sous-ensemble de $E$.
	\begin{enumerate}
		\item
		Donner le cardinal de l'ensemble $E$.
		\item
		Est-il possible que la proportion d'éléments de $E$ appartenant à $F$ soit $\frac12$ ?
		\item ($\star\star$)
		Quelles sont les proportions $\frac1{d}$ possibles d'éléments de $E$ appartenant à $F$ ? $d\in\N$ est un entier naturel non nul.
	\end{enumerate}
}{exe:prop-1d}{
	\begin{enumerate}
		\item
		$|E| = 35$.
		\item
		Supposons que $F$ existe vérifiant
			\[ \dfrac{|F|}{35} = \dfrac12. \]
		Il suivrait alors que $|F| = \frac{35}2 = 17,5$, ce qui est impossible car un cardinal est toujours un entier naturel.
		\item
		Imposons
			\begin{align*}
				\dfrac{|F|}{35} = \dfrac1d && \iff && |F| = \dfrac{35}d.
			\end{align*}
		Cette égalité n'est possible dans les entiers naturels que si la fraction $\frac{35}{d}$ est un entier, est donc si $d$ divise 35.
		Les seuls diviseurs de 35 étant 1, 5, 7, et 35, ce sont les seules possibilités pour $d$.
	\end{enumerate}
}

\newpage

\exe{3}{
	Vrai ou faux ? Cocher la case correspondante.
	\vspace{-30pt}
	\begin{center}
	\def\arraystretch{1.5}
	\setlength\tabcolsep{15pt}
	\begin{tabular}{c c c}
		\hspace{10cm} & Vrai & Faux \\
		\thead{Une augmentation de 20\% suivie d'une \\ diminution de 20\% n'a aucun d'effet} & $\square$ & $\square$  \\
		{20\% de 730 est 146} & $\square$ & $\square$  \\
		\thead{Une multiplication par 1,5 correspond  \\ à une augmentation de 150\%} & $\square$ & $\square$  \\
		\thead{L'évolution réciproque d'une diminution  \\ de 80\% est une augmentation de 400\%} & $\square$ & $\square$  \\
		\thead{Prendre la moitié de la moitié \\ c'est diminuer de 75\%} & $\square$ & $\square$  \\
		\thead{Une augmentation suivie d'une diminution \\ aboutit toujours à une diminution} & $\square$ & $\square$ 
	\end{tabular}
	\end{center}
}{exe:qcm}{
	\begin{center}
	\def\arraystretch{1.5}
	\setlength\tabcolsep{15pt}
	\begin{tabular}{c c c}
		\hspace{10cm} & Vrai & Faux \\
		\thead{Une augmentation de 20\% suivie d'une \\ diminution de 20\% n'a aucun d'effet} & & $\times$  \\
		{20\% de 730 est 146} & $\times$ &   \\
		\thead{Une multiplication par 1,5 correspond  \\ à une augmentation de 150\%} &  & $\times$  \\
		\thead{L'évolution réciproque d'une diminution  \\ de 80\% est une augmentation de 400\%} & $\times$ &  \\
		\thead{Prendre la moitié de la moitié \\ c'est diminuer de 75\%} & $\times$ &  \\
		\thead{Une augmentation suivie d'une diminution \\ aboutit toujours à une diminution} & & $\times$ 
	\end{tabular}
	\end{center}
}



\exe{3, difficulty=1}{
	La caisse automatique du Carrefour de Morangis a mal été programmée : à chaque fois qu'un article est ajouté, une remise de 30\% est appliquée au panier tout entier.
	
	Un client décide de profiter de cette erreur pour payer le moins cher possible un objet qui coûte initialement 5€.
	Pour cela, il rescanne le même objet plusieurs fois à la caisse, ajoutant à chaque fois 5€, mais diminuant le total de 30\%.
	
	Par exemple, après un scan de l'objet, le panier vaut 5€ avant remise et 3,5€ après réduction de 30\%.
	\begin{enumerate}
		\item
		Donner le coefficient multiplicateur associé à une diminution de 30\%.
		\item
		Vérifier qu'après un scan, le panier final est bien de 3,5€.
		\item
		Après deux scans de l'objet, le panier vaut 10€ avant remises.
		Quel est sa valeur finale après deux diminutions successives de 30\% ?
		\item
		Faire idem avec trois scans : donner la valeur du panier avant réduction, puis après application des trois diminutions successives.
		\item $(\star)$
		À partir de combien de scans la valeur finale du panier est inférieure à 2,5€ ?
		
		\emph{Indication : il n'en faut pas plus de 10.}
	\end{enumerate}
		
}{exe:2}{
	\begin{enumerate}
		\item
		$CM = 1 - \frac{30}{100} = 0,7$.
		\item
		$5 \times 0,7 = 3,5$.
		\item
		$10 \times 0,7^2 = 4,9$.
		\item
		Faire idem avec trois scans : donner la valeur du panier avant réduction, puis après application des trois diminutions successives.
		Après 3 scans, le panier vaut $3\times5 = 15$€ avant réduction.
		Après 3 réductions de 30\%, celui-ci vaut $15 \times 0,7^3 = 5,145$€ (Carrefour arrondit à 5,14€).
		\item $(\star)$
		Après $n$ scans, le prix de panier avant réduction est $5n$.
		Après application des $n$ réductions, le prix final est de $5n \times0,7^n$.
		En choisissant des valeurs de $n$ jusqu'à obtenir un prix final inférieur à 2,50€, on trouve $n=8$ car après 7 scans, $35 \times 0,7^7 \approx 2,88$, et après 8 scans, $40\times0,7^8 \approx 2,31$.
		
		Il semble d'abord que le prix final augmente en scannant l'objet, mais la croissance linéaire ($5n$) se fait dominer rapidement par la décroissance exponentielle ($0,7^n$).
	\end{enumerate}
}

%%%%%%%%%%%

\newpage
\fancyhead[C]{\textbf{Solutions}}
\shipoutAnswer
	
\end{document}

%!TEX encoding = UTF8
%!TEX root =notes.tex


%%%%%%%%%%%%%%%%%%%%%%%%%%%%%%%%%
% PACKAGE IMPORTS
%%%%%%%%%%%%%%%%%%%%%%%%%%%%%%%%%


\usepackage[french]{babel}

\usepackage[tmargin=2cm,rmargin=1in,lmargin=1in,margin=0.85in,bmargin=2cm,footskip=.2in]{geometry}
\usepackage{amsmath,amsfonts,amsthm,amssymb,mathtools}
\usepackage[varbb]{newpxmath}
\usepackage{xfrac}
\usepackage[makeroom]{cancel}
\usepackage{mathtools}
\usepackage{bookmark}
\usepackage{enumitem}
\usepackage{hyperref,theoremref}
\hypersetup{
	pdftitle={Assignment},
	colorlinks=true, linkcolor=doc!90,
	bookmarksnumbered=true,
	bookmarksopen=true
}
\usepackage[most,many,breakable]{tcolorbox}
\usepackage{xcolor}
\usepackage{varwidth}
\usepackage{varwidth}
\usepackage{etoolbox}
%\usepackage{authblk}
\usepackage{nameref}
\usepackage{multicol,array}
\usepackage{tikz-cd}
\usepackage[ruled,vlined,linesnumbered]{algorithm2e}
\usepackage{comment} % enables the use of multi-line comments (\ifx \fi) 
\usepackage{import}
\usepackage{xifthen}
\usepackage{pdfpages}
\usepackage{transparent}


\newcommand\mycommfont[1]{\footnotesize\ttfamily\textcolor{blue}{#1}}
\SetCommentSty{mycommfont}
\newcommand{\incfig}[1]{%
    \def\svgwidth{\columnwidth}
    \import{./figures/}{#1.pdf_tex}
}

\usepackage{tikzsymbols}
%\renewcommand\qedsymbol{$\Laughey$}


%\usepackage{import}
%\usepackage{xifthen}
%\usepackage{pdfpages}
%\usepackage{transparent}


%%%%%%%%%%%%%%%%%%%%%%%%%%%%%%
% SELF MADE COLORS
%%%%%%%%%%%%%%%%%%%%%%%%%%%%%%



\definecolor{myg}{RGB}{56, 140, 70}
\definecolor{myb}{RGB}{45, 111, 177}
\definecolor{myr}{RGB}{199, 68, 64}
\definecolor{mytheorembg}{HTML}{F2F2F9}
\definecolor{mytheoremfr}{HTML}{00007B}
\definecolor{mylenmabg}{HTML}{FFFAF8}
\definecolor{mylenmafr}{HTML}{983b0f}
\definecolor{mypropbg}{HTML}{f2fbfc}
\definecolor{mypropfr}{HTML}{191971}
\definecolor{myexamplebg}{HTML}{F2FBF8}
\definecolor{myexamplefr}{HTML}{88D6D1}
\definecolor{myexampleti}{HTML}{2A7F7F}
\definecolor{mydefinitbg}{HTML}{E5E5FF}
\definecolor{mydefinitfr}{HTML}{3F3FA3}
\definecolor{notesgreen}{RGB}{0,162,0}
\definecolor{myp}{RGB}{197, 92, 212}
\definecolor{mygr}{HTML}{2C3338}
\definecolor{myred}{RGB}{127,0,0}
\definecolor{myyellow}{RGB}{169,121,69}
\definecolor{myexercisebg}{HTML}{F2FBF8}
\definecolor{myexercisefg}{HTML}{88D6D1}


%%%%%%%%%%%%%%%%%%%%%%%%%%%%
% TCOLORBOX SETUPS
%%%%%%%%%%%%%%%%%%%%%%%%%%%%

\setlength{\parindent}{1cm}
%================================
% THEOREM BOX
%================================

\tcbuselibrary{theorems,skins,hooks}
\newtcbtheorem[number within=chapter]{Theorem}{Théorème}
{%
	enhanced,
	breakable,
	colback = mytheorembg,
	frame hidden,
	boxrule = 0sp,
	borderline west = {2pt}{0pt}{mytheoremfr},
	sharp corners,
	detach title,
	before upper = \tcbtitle\par\smallskip,
	coltitle = mytheoremfr,
	fonttitle = \bfseries\sffamily,
	description font = \mdseries,
	separator sign none,
	segmentation style={solid, mytheoremfr},
}
{th}


\tcbuselibrary{theorems,skins,hooks}
\newtcolorbox{Theoremcon}
{%
	enhanced
	,breakable
	,colback = mytheorembg
	,frame hidden
	,boxrule = 0sp
	,borderline west = {2pt}{0pt}{mytheoremfr}
	,sharp corners
	,description font = \mdseries
	,separator sign none
}

%================================
% Corollery
%================================
\tcbuselibrary{theorems,skins,hooks}
\newtcbtheorem[use counter=tcb@cnt@Theorem]{Corollary}{Corollaire}
{%
	enhanced
	,breakable
	,colback = myp!10
	,frame hidden
	,boxrule = 0sp
	,borderline west = {2pt}{0pt}{myp!85!black}
	,sharp corners
	,detach title
	,before upper = \tcbtitle\par\smallskip
	,coltitle = myp!85!black
	,fonttitle = \bfseries\sffamily
	,description font = \mdseries
	,separator sign none
	,segmentation style={solid, myp!85!black}
}
{th}

%================================
% LENMA
%================================

\tcbuselibrary{theorems,skins,hooks}
\newtcbtheorem[use counter=tcb@cnt@Theorem]{Lemma}{Lemme}
{%
	enhanced,
	breakable,
	colback = mylenmabg,
	frame hidden,
	boxrule = 0sp,
	borderline west = {2pt}{0pt}{mylenmafr},
	sharp corners,
	detach title,
	before upper = \tcbtitle\par\smallskip,
	coltitle = mylenmafr,
	fonttitle = \bfseries\sffamily,
	description font = \mdseries,
	separator sign none,
	segmentation style={solid, mylenmafr},
}
{th}


%================================
% PROPOSITION
%================================

\tcbuselibrary{theorems,skins,hooks}
\newtcbtheorem[use counter=tcb@cnt@Theorem]{Prop}{Proposition}
{%
	enhanced,
	breakable,
	colback = mypropbg,
	frame hidden,
	boxrule = 0sp,
	borderline west = {2pt}{0pt}{mypropfr},
	sharp corners,
	detach title,
	before upper = \tcbtitle\par\smallskip,
	coltitle = mypropfr,
	fonttitle = \bfseries\sffamily,
	description font = \mdseries,
	separator sign none,
	segmentation style={solid, mypropfr},
}
{th}


%================================
% CLAIM
%================================

\tcbuselibrary{theorems,skins,hooks}
\newtcbtheorem[use counter=tcb@cnt@Theorem]{claim}{Claim}
{%
	enhanced
	,breakable
	,colback = myg!10
	,frame hidden
	,boxrule = 0sp
	,borderline west = {2pt}{0pt}{myg}
	,sharp corners
	,detach title
	,before upper = \tcbtitle\par\smallskip
	,coltitle = myg!85!black
	,fonttitle = \bfseries\sffamily
	,description font = \mdseries
	,separator sign none
	,segmentation style={solid, myg!85!black}
}
{th}



%================================
% Exercise
%================================

\tcbuselibrary{theorems,skins,hooks}
\newtcbtheorem[use counter=tcb@cnt@Theorem]{Exercise}{Exercice}
{%
	enhanced,
	breakable,
	colback = myexercisebg,
	frame hidden,
	boxrule = 0sp,
	borderline west = {2pt}{0pt}{myexercisefg},
	sharp corners,
	detach title,
	before upper = \tcbtitle\par\smallskip,
	coltitle = myexercisefg,
	fonttitle = \bfseries\sffamily,
	description font = \mdseries,
	separator sign none,
	segmentation style={solid, myexercisefg},
}
{th}

%================================
% EXAMPLE BOX
%================================

\newtcbtheorem[use counter=tcb@cnt@Theorem]{Example}{Exemple}
{%
	colback = myexamplebg
	,breakable
	,colframe = myexamplefr
	,coltitle = myexampleti
	,boxrule = 1pt
	,sharp corners
	,detach title
	,before upper=\tcbtitle\par\smallskip
	,fonttitle = \bfseries
	,description font = \mdseries
	,separator sign none
	,description delimiters parenthesis
}
{ex}

%================================
% DEFINITION BOX
%================================

\newtcbtheorem[use counter=tcb@cnt@Theorem]{Definition}{Définition}{enhanced,
	before skip=2mm,after skip=2mm, colback=red!5,colframe=red!80!black,boxrule=0.5mm,
	attach boxed title to top left={xshift=1cm,yshift*=1mm-\tcboxedtitleheight}, varwidth boxed title*=-3cm,
	boxed title style={frame code={
					\path[fill=tcbcolback]
					([yshift=-1mm,xshift=-1mm]frame.north west)
					arc[start angle=0,end angle=180,radius=1mm]
					([yshift=-1mm,xshift=1mm]frame.north east)
					arc[start angle=180,end angle=0,radius=1mm];
					\path[left color=tcbcolback!60!black,right color=tcbcolback!60!black,
						middle color=tcbcolback!80!black]
					([xshift=-2mm]frame.north west) -- ([xshift=2mm]frame.north east)
					[rounded corners=1mm]-- ([xshift=1mm,yshift=-1mm]frame.north east)
					-- (frame.south east) -- (frame.south west)
					-- ([xshift=-1mm,yshift=-1mm]frame.north west)
					[sharp corners]-- cycle;
				},interior engine=empty,
		},
	fonttitle=\bfseries,
	title={#2},#1}{def}

%================================
% Solution BOX
%================================

\makeatletter
\newtcbtheorem[use counter=tcb@cnt@Theorem]{question}{Question}{enhanced,
	breakable,
	colback=white,
	colframe=myb!80!black,
	attach boxed title to top left={yshift*=-\tcboxedtitleheight},
	fonttitle=\bfseries,
	title={#2},
	boxed title size=title,
	boxed title style={%
			sharp corners,
			rounded corners=northwest,
			colback=tcbcolframe,
			boxrule=0pt,
		},
	underlay boxed title={%
			\path[fill=tcbcolframe] (title.south west)--(title.south east)
			to[out=0, in=180] ([xshift=5mm]title.east)--
			(title.center-|frame.east)
			[rounded corners=\kvtcb@arc] |-
			(frame.north) -| cycle;
		},
	#1
}{def}
\makeatother

%================================
% SOLUTION BOX
%================================

\makeatletter
\newtcolorbox{solution}{enhanced,
	breakable,
	colback=white,
	colframe=myg!80!black,
	attach boxed title to top left={yshift*=-\tcboxedtitleheight},
	title=Solution,
	boxed title size=title,
	boxed title style={%
			sharp corners,
			rounded corners=northwest,
			colback=tcbcolframe,
			boxrule=0pt,
		},
	underlay boxed title={%
			\path[fill=tcbcolframe] (title.south west)--(title.south east)
			to[out=0, in=180] ([xshift=5mm]title.east)--
			(title.center-|frame.east)
			[rounded corners=\kvtcb@arc] |-
			(frame.north) -| cycle;
		},
}
\makeatother

%================================
% Question BOX
%================================

\makeatletter
\newtcbtheorem[use counter=tcb@cnt@Theorem]{qstion}{Question}{enhanced,
	breakable,
	colback=white,
	colframe=mygr,
	attach boxed title to top left={yshift*=-\tcboxedtitleheight},
	fonttitle=\bfseries,
	title={#2},
	boxed title size=title,
	boxed title style={%
			sharp corners,
			rounded corners=northwest,
			colback=tcbcolframe,
			boxrule=0pt,
		},
	underlay boxed title={%
			\path[fill=tcbcolframe] (title.south west)--(title.south east)
			to[out=0, in=180] ([xshift=5mm]title.east)--
			(title.center-|frame.east)
			[rounded corners=\kvtcb@arc] |-
			(frame.north) -| cycle;
		},
	#1
}{def}
\makeatother

\newtcbtheorem[number within=chapter]{wconc}{Wrong Concept}{
	breakable,
	enhanced,
	colback=white,
	colframe=myr,
	arc=0pt,
	outer arc=0pt,
	fonttitle=\bfseries\sffamily\large,
	colbacktitle=myr,
	attach boxed title to top left={},
	boxed title style={
			enhanced,
			skin=enhancedfirst jigsaw,
			arc=3pt,
			bottom=0pt,
			interior style={fill=myr}
		},
	#1
}{def}



%================================
% NOTE BOX
%================================

\usetikzlibrary{arrows,calc,shadows.blur}
\tcbuselibrary{skins}
\newtcolorbox{note}[1][]{%
	enhanced jigsaw,
	colback=gray!20!white,%
	colframe=gray!80!black,
	size=small,
	boxrule=1pt,
	title=\colorbox{white!100}{\textbf{ Remarque }},
	halign title=flush center,
	coltitle=black,
	breakable,
	drop shadow=black!50!white,
	attach boxed title to top left={xshift=1cm,yshift=-\tcboxedtitleheight/2,yshifttext=-\tcboxedtitleheight/2},
	minipage boxed title=2.6cm,
	boxed title style={%
			colback=white,
			size=fbox,
			boxrule=1pt,
			boxsep=2pt,
			underlay={%
					\coordinate (dotA) at ($(interior.west) + (-0.5pt,0)$);
					\coordinate (dotB) at ($(interior.east) + (0.5pt,0)$);
					\begin{scope}
						\clip (interior.north west) rectangle ([xshift=3ex]interior.east);
						\filldraw [white, blur shadow={shadow opacity=60, shadow yshift=-.75ex}, rounded corners=2pt] (interior.north west) rectangle (interior.south east);
					\end{scope}
					\begin{scope}[gray!80!black]
						\fill (dotA) circle (2pt);
						\fill (dotB) circle (2pt);
					\end{scope}
				},
		},
	#1,
}

%================================
% STRATÉGIE BOX
%================================

\usetikzlibrary{arrows,calc,shadows.blur}
\tcbuselibrary{skins}
\newtcolorbox{strategy}[1][]{%
	enhanced jigsaw,
	colback=myb!20!white,%
	colframe=gray!80!black,
	size=small,
	boxrule=1pt,
	title=\colorbox{white!100}{\textbf{ Stratégie }},
	halign title=flush center,
	coltitle=black,
	breakable,
	drop shadow=black!50!white,
	attach boxed title to top left={xshift=1cm,yshift=-\tcboxedtitleheight/2,yshifttext=-\tcboxedtitleheight/2},
	minipage boxed title=2.5cm,
	boxed title style={%
			colback=white,
			size=fbox,
			boxrule=1pt,
			boxsep=2pt,
			underlay={%
					\coordinate (dotA) at ($(interior.west) + (-0.5pt,0)$);
					\coordinate (dotB) at ($(interior.east) + (0.5pt,0)$);
					\begin{scope}
						\clip (interior.north west) rectangle ([xshift=3ex]interior.east);
						\filldraw [white, blur shadow={shadow opacity=60, shadow yshift=-.75ex}, rounded corners=2pt] (interior.north west) rectangle (interior.south east);
					\end{scope}
					\begin{scope}[gray!80!black]
						\fill (dotA) circle (2pt);
						\fill (dotB) circle (2pt);
					\end{scope}
				},
		},
	#1,
}

%================================
% MÉTHODE BOX
%================================

\usetikzlibrary{arrows,calc,shadows.blur}
\tcbuselibrary{skins}
\newtcolorbox{methode}[1][]{%
	enhanced jigsaw,
	colback=white,%
	colframe=gray!80!black,
	size=small,
	boxrule=1pt,
	title=\textbf{Méthode},
	halign title=flush center,
	coltitle=black,
	breakable,
	drop shadow=black!50!white,
	attach boxed title to top left={xshift=1cm,yshift=-\tcboxedtitleheight/2,yshifttext=-\tcboxedtitleheight/2},
	minipage boxed title=2.5cm,
	boxed title style={%
			colback=white,
			size=fbox,
			boxrule=1pt,
			boxsep=2pt,
			underlay={%
					\coordinate (dotA) at ($(interior.west) + (-0.5pt,0)$);
					\coordinate (dotB) at ($(interior.east) + (0.5pt,0)$);
					\begin{scope}
						\clip (interior.north west) rectangle ([xshift=3ex]interior.east);
						\filldraw [white, blur shadow={shadow opacity=60, shadow yshift=-.75ex}, rounded corners=2pt] (interior.north west) rectangle (interior.south east);
					\end{scope}
					\begin{scope}[gray!80!black]
						\fill (dotA) circle (2pt);
						\fill (dotB) circle (2pt);
					\end{scope}
				},
		},
	#1,
}

%%%%%%%%%%%%%%%%%%%%%%%%%%%%%%%%%%%%%%%%%%%
% TABLE OF CONTENTS
%%%%%%%%%%%%%%%%%%%%%%%%%%%%%%%%%%%%%%%%%%%

\usepackage{tikz}

\definecolor{doc}{RGB}{0,60,110}
\usepackage{titletoc}
\contentsmargin{0cm}
\titlecontents{chapter}[3.7pc]
{\addvspace{30pt}%
	\begin{tikzpicture}[remember picture, overlay]%
		\draw[fill=doc!60,draw=doc!60] (-7,-.1) rectangle (-0.2,.6);%
		\pgftext[left,x=-3.5cm,y=0.2cm]{\color{white}\Large\sc\bfseries Chapitre\ \thecontentslabel};%
	\end{tikzpicture}\color{doc!60}\large\sc\bfseries}%
{}
{}
{\;\titlerule\;\large\sc\bfseries Page \thecontentspage
	\begin{tikzpicture}[remember picture, overlay]
		\draw[fill=doc!60,draw=doc!60] (2pt,0) rectangle (4,0.1pt);
	\end{tikzpicture}}%
\titlecontents{section}[3.7pc]
{\addvspace{2pt}}
{\contentslabel[\thecontentslabel]{2pc}}
{}
{\hfill\small \thecontentspage}
[]
\titlecontents*{subsection}[3.7pc]
{\addvspace{-1pt}\small}
{}
{}
{\ --- \small\thecontentspage}
[ \textbullet\ ][]

\makeatletter
\renewcommand{\tableofcontents}{%
	\chapter*{%
	  \vspace*{-20\p@}%
	  \begin{tikzpicture}[remember picture, overlay]%
		  \pgftext[right,x=15cm,y=0.2cm]{\color{doc!60}\Huge\sc\bfseries \contentsname};%
		  \draw[fill=doc!60,draw=doc!60] (13,-.75) rectangle (20,1);%
		  \clip (13,-.75) rectangle (20,1);
		  \pgftext[right,x=15cm,y=0.2cm]{\color{white}\Huge\sc\bfseries \contentsname};%
	  \end{tikzpicture}}%
	\@starttoc{toc}}
\makeatother


%%%%%%%%%%%%%%%%%%%%%%%%%%%%%%%%%%%%%%%%%%%
% MINTED FOR PYTHON ALGORITHMS
%%%%%%%%%%%%%%%%%%%%%%%%%%%%%%%%%%%%%%%%%%%

\usepackage{tcolorbox}
\tcbuselibrary{minted,breakable,xparse,skins}
\definecolor{bg}{gray}{0.95}
\DeclareTCBListing{mintedbox}{O{}m!O{}}{%
  breakable=true,
  listing engine=minted,
  listing only,
  minted language=#2,
  minted style=default,
  minted options={%
    linenos,
    gobble=0,
    breaklines=true,
    breakafter=,,
    fontsize=\small,
    numbersep=8pt,
    #1},
  boxsep=0pt,
  left skip=0pt,
  right skip=0pt,
  left=25pt,
  right=0pt,
  top=3pt,
  bottom=3pt,
  arc=5pt,
  leftrule=0pt,
  rightrule=0pt,
  bottomrule=2pt,
  toprule=2pt,
  colback=bg,
  colframe=orange!70,
  enhanced,
  overlay={%
    \begin{tcbclipinterior}
    \fill[orange!20!white] (frame.south west) rectangle ([xshift=20pt]frame.north west);
    \end{tcbclipinterior}},
  #3}
  
  
 % for braces
\usetikzlibrary{decorations.pathreplacing}

\usepackage{minted}

\SetDate[07/01/2026]
\reversemarginpar
\setlength{\marginparsep}{.5cm}

\begin{document}
\pagestyle{fancy}
\fancyhead[L]{Seconde}
\fancyhead[C]{\textbf{Évaluation — Évolutions}}
\fancyhead[R]{\today}

\null\vspace{-30pt}
Consignes particulières : 
\begin{itemize}[label=$\bullet$]
	\item 
	La calculatrice est {autorisée}.
	\item
	Sauf mention du contraire, toutes les réponses doivent être justifiées.
	\item
	Écrire son nom avant de rendre son sujet.
	\item
	L'évaluation fait 2 pages. La somme des points est \total{points}.
\end{itemize}

\marginpar{[pts]}
\hrule

\exe{4}{
	On rappelle que $2^{10} \approx 10^3$. C'est son ordre de grandeur.
	
	\begin{enumerate}
		\item
		Montrer que $2^{352} = \bigl(2^{10}\bigr)^{35} \times 4$.
		\item
		En déduire une approximation de l'ordre de grandeur de $2^{352}$.
		\item
		Montrer que $2^{-389} = \bigl(2^{10}\bigr)^{-39} \times 2$.
		\item
		En déduire une approximation de l'ordre de grandeur de $2^{-389}$.
	\end{enumerate}
}{exe:3}{
	\begin{enumerate}
		\item
			\[ \bigl(2^{10}\bigr)^{35} \times 4 = 2^{10 \times 35} \times 2^2 = 2^{350 + 2} = 2^{352}. \]
		\item
			\[ 2^{352} \approx  \bigl(10^3\bigr)^{35} \times 4 = 4\times10^{105}. \]
		\item
			\[ \bigl(2^{10}\bigr)^{-39} \times 2 = 2^{10 \times (-39)} \times 2^1 = 2^{-390 + 1} = 2^{-389}. \]
		\item
			\[ 2^{-389} \approx \bigl(10^3\bigr)^{-39} \times 2 = 2 \times 10^{-117}. \]
	\end{enumerate}
}


\exe{4, difficulty=1}{
	Après ingestion de 500mg de paracétamol, celui-ci est décomposé dans l'organisme au cours du temps.
	Chaque heure, $10\%$ de la quantité de médicament encore présente dans l'organisme est absorbée.
	\begin{center}
\setlength\tabcolsep{20pt}
	\begin{tabular}{|c|c|c|c|c|c|}\hline
		Heure & 0 & 1 & 2 & 3 & 4 \\ \hline
		Quantité de médicament restante (mg) & 500 & 450 & & & \\ \hline
	\end{tabular}
	\end{center}

	\begin{enumerate}
		\item Donner le coefficient multiplicateur correspondant à une diminution de 10\%.
		\item Vérifier les premières valeurs du tableau ci-dessus et le compléter.
		\item Quelle quantité de médicament sera encore présente dans l'organisme après un jour (24 heures) en milligrammes ?
		Arrondir à l'unité.
	\end{enumerate}
}{exe:paracetamol}{
	\begin{center}
	\begin{tabular}{|c|c|c|c|c|c|}\hline
		Heure & 0 & 1 & 2 & 3 & 4 \\ \hline
		Quantité de médicament restante (mg) & 500 & 450 & 405 & 364,5 & 328,05 \\ \hline
	\end{tabular}
	\end{center}

	\begin{enumerate}
		\item
		Un objet à 100€ passe à 90€ lors d'une diminution de 10\% : le coefficient multiplicateur est donc égal à $\frac{90}{100} = 0,9$.
		En général, pour une diminution de $P\%$, il faut multiplier par $1 - \frac{P}{100}$.
		\item
		On multiplie par $0,9$ pour passer d'un terme à l'autre du tableau.
		\item 
			\[ 500 \times 0,9^{24} \approx 40 \]
		Il reste donc 40mg de médicament encore présents dans l'organisme après 24 heures.
	\end{enumerate}
}



\exe{6}{
	Vrai ou faux ? Cocher la case correspondante et \textbf{\underline{justifier la réponse}}.
	%\vspace{-30pt}
	\begin{center}
	\def\arraystretch{1.5}
	\setlength\tabcolsep{15pt}
	\begin{tabular}{c c c}
		\hspace{10cm} & Vrai & Faux \\
		$1 < 90\%$ & $\square$ & $\square$  \\
		$120\% = \frac65$ & $\square$ & $\square$  \\
		Multiplier une valeur par $\frac78$ l'augmente & $\square$ & $\square$  \\
		Deux augmentations de 20\% équivaut à une augmentation de 40\% & $\square$ & $\square$  \\
		L'inverse d'un nombre supérieur à 1 est inférieur à 1 & $\square$ & $\square$  \\
		Diminuer une valeur de 30\% équivaut à la multiplier par 0,3 & $\square$ & $\square$
	\end{tabular}
	\end{center}
}{exe:qcm}{
		\underline{$1 < 90\%$} \\
		 Faux car $90\% = 0,9$ qui n'est pas strictement supérieur à 1.
		
		\underline{$120\% = \frac65$} \\ 
		Vrai car $\frac65 = 1,2 = \frac{120}{100} = 120\%$.

		\underline{Multiplier une valeur par $\frac78$ l'augmente} \\ 
		Faux car $\frac78$ est strictement inférieur à 1.
		
		\underline{Deux augmentations de 20\% équivaut à une augmentation de 40\%} \\
		 Faux car $1,2 \times 1,2 = 1,44$, qui correspond à une augmentation de $44\%$.
		
		\underline{L'inverse d'un nombre supérieur à 1 est inférieur à 1} \\ 
		Vrai. Un nombre supérieur à 1 peut être vu comme le coefficient multiplicateur associé à une augmentation.
		Son inverse, lui, correspond à l'évolution réciproque qui est nécessairement une diminution.
		Son inverse est donc un nombre positif inférieur à 1.
		
		\underline{Diminuer une valeur de 30\% équivaut à la multiplier par 0,3} \\
		 Faux. D'après le cours, il faut multiplier par $0,7$.
}

\newpage


\exe{7, difficulty=2}{
	Un client dispose de deux bons de réduction : un bon réduisant le prix du panier de 15€ ; l'autre réduisant le prix du panier de 25\%.
	Comme les bons ne sont pas cumulables, le client doit décider quand l'un est plus avantageux que l'autre.
	
	Pour un prix $x\in\R$ en euros, posons $f(x)$ le prix réduit de 15€ et $g(x)$ le prix réduit de $25\%$.
	
	\begin{enumerate}
		\item
		Justifier que
			\begin{align*}
				f(x) = x-15, && \et && g(x) = 0,75x.
			\end{align*}
		\item \label{q:2}
		Tracer \underline{soigneusement} les courbes représentatives de $f$ et $g$ dans le repère figure \ref{fig:1a}.
		\item
		Trouver le $x\in\R$ réel vérifiant $f(x) = g(x)$ de deux façons différentes :
			\begin{enumerate}[label=\alph*), leftmargin=40pt]
				\item graphiquement à l'aide des courbes $\C_f$ et $\C_g$ tracées à la question \ref{q:2}.
				\item algébriquement en posant puis en résolvant l'équation.
			\end{enumerate}
		\item \label{q:4}
		Créer un protocole de décision d'utilisation du premier ou du deuxième bon en fonction du prix du panier avant réduction.
		Il s'agit de répondre aux questions suivantes : 
			\begin{enumerate}[label=\roman*), leftmargin=60pt]
				\item quand est-il plus avantageux de choisir le premier bon ? 
				\item quand les deux bons sont-ils équivalents ?
				\item quand est-il plus avantageux de choisir le second bon ? 
			\end{enumerate}
		\item
		Compléter les espaces vides (\_\_\_\_ aux lignes 2 ; 4 ; 6) de la fonction Python figure \ref{fig:1b} résumant le protocole de décision de la question \ref{q:4}.
	\end{enumerate}
}{exe:2}{
	\begin{enumerate}
		\item
		Pour $f$, diminuer le prix $x$ de 15 correspond trivialement à faire $x-15$.
		
		Pour $g$, diminuer le prix $x$ de 25\% équivaut à le multiplier par $1-\frac{25}{100} = 0,75$.
		\item 
		Voir figure \ref{fig:1sol}.
		\item
			\begin{enumerate}[label=\alph*), leftmargin=40pt]
				\item 
				Graphiquement, on regarde l'abscisse du point d'intersection des courbes $\C_f$ et $\C_g$, qui est environ $x\approx60$.
				\item 
				Algébriquement, on résoud $x-15 = 0,75x$ pour $x\in\R$.
				\begin{align*}
					x-15 &= 0,75x \\
					x-0,75x - 15 &= 0 \\
					0,25x &= 15 \\
					x &= \dfrac{15}{0,25} = 60
				\end{align*}
			\end{enumerate}
		\item
		D'après la courbe figure \ref{fig:1asol}, $f(x) < g(x)$ pour $x\in[20;60[$, ce qui signifie que le premier bon est plus avantageux lorsque le prix du panier est strictement inférieur à 60€.
		
		Comme $f(60) = g(60)$, les deux bons sont équivalents lorsque le prix est égal à 60€.
		
		Finalement, $f(x) > g(x)$ pour $x\in]60;100]$, ce qui signifie que le deuxième bon est plus avantageux pour les paniers strictement supérieurs à 60€.
		\item
		Voir figure \ref{fig:1sol}.
	\end{enumerate}
	
	\underline{Discussion} \\
	La réduction de 25\% soustrait un quart du prix.
	Il est donc cohérent que le deuxième bon soit plus avantageux au bout d'un moment.
	Pour obtenir la solution plus rapidement, on aurait pû se poser la question : quand est-ce que 25\% du prix vaut 15€ ? $0,25 x = 15 \iff x = 60$.
}

\begin{figure}%[h!]
\begin{subfigure}{.4\textwidth}
	\begin{center}
	\begin{tikzpicture}[>=stealth]
		\begin{axis}[xmin = 20, xmax=100, ymin=0, ymax=100, axis x line=middle, axis y line=middle, axis line style=->, grid=both, extra x ticks ={0}, extra y ticks = {0}, extra x ticks = {20},]
			\addplot[transparent, very thick, -] expression[domain=0:100, samples=2]{x-15} node[above left, pos=.8] {$\C_f$};
			\addplot[transparent, very thick, -] expression[domain=0:100, samples=2]{.75*x}  node[below right, pos=.8] {$\C_g$};
		\end{axis}
	\end{tikzpicture}
	\end{center}
	\caption{Courbes représentatives de $f$ et $g$.}
	\label{fig:1a}
\end{subfigure}
\hfill
\begin{subfigure}{.4\textwidth}
	\begin{center}
	\python{choix}
	\end{center}
	\caption{La fonction Python \texttt{choix} renvoie une décision sous forme de chaîne de caractères selon la valeur \texttt{prix\_panier} qui lui est donnée.}
	\label{fig:1b}
\end{subfigure}
	\caption{Figure de l'exercice \ref{exe:2}.}
	\label{fig:1}
\end{figure}





%%%%%%%%%%%

\newpage
\fancyhead[C]{\textbf{Solutions}}
\shipoutAnswer



\begin{figure}[h!]
\begin{subfigure}{.4\textwidth}
	\begin{center}
	\begin{tikzpicture}[>=stealth]
		\begin{axis}[xmin = 20, xmax=100, ymin=0, ymax=100, axis x line=middle, axis y line=middle, axis line style=->, grid=both, extra x ticks ={0}, extra y ticks = {0}, extra x ticks = {20},]
			\addplot[BLUE_E, very thick, -] expression[domain=20:100, samples=2]{x-15} node[above left, pos=.8] {$\C_f$};
			\addplot[RED_E, very thick, -] expression[domain=20:100, samples=2]{.75*x}  node[below right, pos=.8] {$\C_g$};
		\end{axis}
	\end{tikzpicture}
	\end{center}
	\caption{Courbes représentatives de $f$ et $g$.}
	\label{fig:1asol}
\end{subfigure}
\hfill
\begin{subfigure}{.4\textwidth}
	\begin{center}
	\python{choixsol}
	\end{center}
	\caption{La fonction Python \texttt{choix} renvoie une décision sous forme de chaîne de caractères selon la valeur \texttt{prix\_panier} qui lui est donnée.}
	\label{fig:1bsol}
\end{subfigure}
	\caption{Figure de l'exercice \ref{exe:2}.}
	\label{fig:1sol}
\end{figure}


\end{document}

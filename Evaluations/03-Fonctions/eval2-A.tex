%!TEX encoding = UTF8
%!TEX root =notes.tex


%%%%%%%%%%%%%%%%%%%%%%%%%%%%%%%%%
% PACKAGE IMPORTS
%%%%%%%%%%%%%%%%%%%%%%%%%%%%%%%%%


\usepackage[french]{babel}

\usepackage[tmargin=2cm,rmargin=1in,lmargin=1in,margin=0.85in,bmargin=2cm,footskip=.2in]{geometry}
\usepackage{amsmath,amsfonts,amsthm,amssymb,mathtools}
\usepackage[varbb]{newpxmath}
\usepackage{xfrac}
\usepackage[makeroom]{cancel}
\usepackage{mathtools}
\usepackage{bookmark}
\usepackage{enumitem}
\usepackage{hyperref,theoremref}
\hypersetup{
	pdftitle={Assignment},
	colorlinks=true, linkcolor=doc!90,
	bookmarksnumbered=true,
	bookmarksopen=true
}
\usepackage[most,many,breakable]{tcolorbox}
\usepackage{xcolor}
\usepackage{varwidth}
\usepackage{varwidth}
\usepackage{etoolbox}
%\usepackage{authblk}
\usepackage{nameref}
\usepackage{multicol,array}
\usepackage{tikz-cd}
\usepackage[ruled,vlined,linesnumbered]{algorithm2e}
\usepackage{comment} % enables the use of multi-line comments (\ifx \fi) 
\usepackage{import}
\usepackage{xifthen}
\usepackage{pdfpages}
\usepackage{transparent}


\newcommand\mycommfont[1]{\footnotesize\ttfamily\textcolor{blue}{#1}}
\SetCommentSty{mycommfont}
\newcommand{\incfig}[1]{%
    \def\svgwidth{\columnwidth}
    \import{./figures/}{#1.pdf_tex}
}

\usepackage{tikzsymbols}
%\renewcommand\qedsymbol{$\Laughey$}


%\usepackage{import}
%\usepackage{xifthen}
%\usepackage{pdfpages}
%\usepackage{transparent}


%%%%%%%%%%%%%%%%%%%%%%%%%%%%%%
% SELF MADE COLORS
%%%%%%%%%%%%%%%%%%%%%%%%%%%%%%



\definecolor{myg}{RGB}{56, 140, 70}
\definecolor{myb}{RGB}{45, 111, 177}
\definecolor{myr}{RGB}{199, 68, 64}
\definecolor{mytheorembg}{HTML}{F2F2F9}
\definecolor{mytheoremfr}{HTML}{00007B}
\definecolor{mylenmabg}{HTML}{FFFAF8}
\definecolor{mylenmafr}{HTML}{983b0f}
\definecolor{mypropbg}{HTML}{f2fbfc}
\definecolor{mypropfr}{HTML}{191971}
\definecolor{myexamplebg}{HTML}{F2FBF8}
\definecolor{myexamplefr}{HTML}{88D6D1}
\definecolor{myexampleti}{HTML}{2A7F7F}
\definecolor{mydefinitbg}{HTML}{E5E5FF}
\definecolor{mydefinitfr}{HTML}{3F3FA3}
\definecolor{notesgreen}{RGB}{0,162,0}
\definecolor{myp}{RGB}{197, 92, 212}
\definecolor{mygr}{HTML}{2C3338}
\definecolor{myred}{RGB}{127,0,0}
\definecolor{myyellow}{RGB}{169,121,69}
\definecolor{myexercisebg}{HTML}{F2FBF8}
\definecolor{myexercisefg}{HTML}{88D6D1}


%%%%%%%%%%%%%%%%%%%%%%%%%%%%
% TCOLORBOX SETUPS
%%%%%%%%%%%%%%%%%%%%%%%%%%%%

\setlength{\parindent}{1cm}
%================================
% THEOREM BOX
%================================

\tcbuselibrary{theorems,skins,hooks}
\newtcbtheorem[number within=chapter]{Theorem}{Théorème}
{%
	enhanced,
	breakable,
	colback = mytheorembg,
	frame hidden,
	boxrule = 0sp,
	borderline west = {2pt}{0pt}{mytheoremfr},
	sharp corners,
	detach title,
	before upper = \tcbtitle\par\smallskip,
	coltitle = mytheoremfr,
	fonttitle = \bfseries\sffamily,
	description font = \mdseries,
	separator sign none,
	segmentation style={solid, mytheoremfr},
}
{th}


\tcbuselibrary{theorems,skins,hooks}
\newtcolorbox{Theoremcon}
{%
	enhanced
	,breakable
	,colback = mytheorembg
	,frame hidden
	,boxrule = 0sp
	,borderline west = {2pt}{0pt}{mytheoremfr}
	,sharp corners
	,description font = \mdseries
	,separator sign none
}

%================================
% Corollery
%================================
\tcbuselibrary{theorems,skins,hooks}
\newtcbtheorem[use counter=tcb@cnt@Theorem]{Corollary}{Corollaire}
{%
	enhanced
	,breakable
	,colback = myp!10
	,frame hidden
	,boxrule = 0sp
	,borderline west = {2pt}{0pt}{myp!85!black}
	,sharp corners
	,detach title
	,before upper = \tcbtitle\par\smallskip
	,coltitle = myp!85!black
	,fonttitle = \bfseries\sffamily
	,description font = \mdseries
	,separator sign none
	,segmentation style={solid, myp!85!black}
}
{th}

%================================
% LENMA
%================================

\tcbuselibrary{theorems,skins,hooks}
\newtcbtheorem[use counter=tcb@cnt@Theorem]{Lemma}{Lemme}
{%
	enhanced,
	breakable,
	colback = mylenmabg,
	frame hidden,
	boxrule = 0sp,
	borderline west = {2pt}{0pt}{mylenmafr},
	sharp corners,
	detach title,
	before upper = \tcbtitle\par\smallskip,
	coltitle = mylenmafr,
	fonttitle = \bfseries\sffamily,
	description font = \mdseries,
	separator sign none,
	segmentation style={solid, mylenmafr},
}
{th}


%================================
% PROPOSITION
%================================

\tcbuselibrary{theorems,skins,hooks}
\newtcbtheorem[use counter=tcb@cnt@Theorem]{Prop}{Proposition}
{%
	enhanced,
	breakable,
	colback = mypropbg,
	frame hidden,
	boxrule = 0sp,
	borderline west = {2pt}{0pt}{mypropfr},
	sharp corners,
	detach title,
	before upper = \tcbtitle\par\smallskip,
	coltitle = mypropfr,
	fonttitle = \bfseries\sffamily,
	description font = \mdseries,
	separator sign none,
	segmentation style={solid, mypropfr},
}
{th}


%================================
% CLAIM
%================================

\tcbuselibrary{theorems,skins,hooks}
\newtcbtheorem[use counter=tcb@cnt@Theorem]{claim}{Claim}
{%
	enhanced
	,breakable
	,colback = myg!10
	,frame hidden
	,boxrule = 0sp
	,borderline west = {2pt}{0pt}{myg}
	,sharp corners
	,detach title
	,before upper = \tcbtitle\par\smallskip
	,coltitle = myg!85!black
	,fonttitle = \bfseries\sffamily
	,description font = \mdseries
	,separator sign none
	,segmentation style={solid, myg!85!black}
}
{th}



%================================
% Exercise
%================================

\tcbuselibrary{theorems,skins,hooks}
\newtcbtheorem[use counter=tcb@cnt@Theorem]{Exercise}{Exercice}
{%
	enhanced,
	breakable,
	colback = myexercisebg,
	frame hidden,
	boxrule = 0sp,
	borderline west = {2pt}{0pt}{myexercisefg},
	sharp corners,
	detach title,
	before upper = \tcbtitle\par\smallskip,
	coltitle = myexercisefg,
	fonttitle = \bfseries\sffamily,
	description font = \mdseries,
	separator sign none,
	segmentation style={solid, myexercisefg},
}
{th}

%================================
% EXAMPLE BOX
%================================

\newtcbtheorem[use counter=tcb@cnt@Theorem]{Example}{Exemple}
{%
	colback = myexamplebg
	,breakable
	,colframe = myexamplefr
	,coltitle = myexampleti
	,boxrule = 1pt
	,sharp corners
	,detach title
	,before upper=\tcbtitle\par\smallskip
	,fonttitle = \bfseries
	,description font = \mdseries
	,separator sign none
	,description delimiters parenthesis
}
{ex}

%================================
% DEFINITION BOX
%================================

\newtcbtheorem[use counter=tcb@cnt@Theorem]{Definition}{Définition}{enhanced,
	before skip=2mm,after skip=2mm, colback=red!5,colframe=red!80!black,boxrule=0.5mm,
	attach boxed title to top left={xshift=1cm,yshift*=1mm-\tcboxedtitleheight}, varwidth boxed title*=-3cm,
	boxed title style={frame code={
					\path[fill=tcbcolback]
					([yshift=-1mm,xshift=-1mm]frame.north west)
					arc[start angle=0,end angle=180,radius=1mm]
					([yshift=-1mm,xshift=1mm]frame.north east)
					arc[start angle=180,end angle=0,radius=1mm];
					\path[left color=tcbcolback!60!black,right color=tcbcolback!60!black,
						middle color=tcbcolback!80!black]
					([xshift=-2mm]frame.north west) -- ([xshift=2mm]frame.north east)
					[rounded corners=1mm]-- ([xshift=1mm,yshift=-1mm]frame.north east)
					-- (frame.south east) -- (frame.south west)
					-- ([xshift=-1mm,yshift=-1mm]frame.north west)
					[sharp corners]-- cycle;
				},interior engine=empty,
		},
	fonttitle=\bfseries,
	title={#2},#1}{def}

%================================
% Solution BOX
%================================

\makeatletter
\newtcbtheorem[use counter=tcb@cnt@Theorem]{question}{Question}{enhanced,
	breakable,
	colback=white,
	colframe=myb!80!black,
	attach boxed title to top left={yshift*=-\tcboxedtitleheight},
	fonttitle=\bfseries,
	title={#2},
	boxed title size=title,
	boxed title style={%
			sharp corners,
			rounded corners=northwest,
			colback=tcbcolframe,
			boxrule=0pt,
		},
	underlay boxed title={%
			\path[fill=tcbcolframe] (title.south west)--(title.south east)
			to[out=0, in=180] ([xshift=5mm]title.east)--
			(title.center-|frame.east)
			[rounded corners=\kvtcb@arc] |-
			(frame.north) -| cycle;
		},
	#1
}{def}
\makeatother

%================================
% SOLUTION BOX
%================================

\makeatletter
\newtcolorbox{solution}{enhanced,
	breakable,
	colback=white,
	colframe=myg!80!black,
	attach boxed title to top left={yshift*=-\tcboxedtitleheight},
	title=Solution,
	boxed title size=title,
	boxed title style={%
			sharp corners,
			rounded corners=northwest,
			colback=tcbcolframe,
			boxrule=0pt,
		},
	underlay boxed title={%
			\path[fill=tcbcolframe] (title.south west)--(title.south east)
			to[out=0, in=180] ([xshift=5mm]title.east)--
			(title.center-|frame.east)
			[rounded corners=\kvtcb@arc] |-
			(frame.north) -| cycle;
		},
}
\makeatother

%================================
% Question BOX
%================================

\makeatletter
\newtcbtheorem[use counter=tcb@cnt@Theorem]{qstion}{Question}{enhanced,
	breakable,
	colback=white,
	colframe=mygr,
	attach boxed title to top left={yshift*=-\tcboxedtitleheight},
	fonttitle=\bfseries,
	title={#2},
	boxed title size=title,
	boxed title style={%
			sharp corners,
			rounded corners=northwest,
			colback=tcbcolframe,
			boxrule=0pt,
		},
	underlay boxed title={%
			\path[fill=tcbcolframe] (title.south west)--(title.south east)
			to[out=0, in=180] ([xshift=5mm]title.east)--
			(title.center-|frame.east)
			[rounded corners=\kvtcb@arc] |-
			(frame.north) -| cycle;
		},
	#1
}{def}
\makeatother

\newtcbtheorem[number within=chapter]{wconc}{Wrong Concept}{
	breakable,
	enhanced,
	colback=white,
	colframe=myr,
	arc=0pt,
	outer arc=0pt,
	fonttitle=\bfseries\sffamily\large,
	colbacktitle=myr,
	attach boxed title to top left={},
	boxed title style={
			enhanced,
			skin=enhancedfirst jigsaw,
			arc=3pt,
			bottom=0pt,
			interior style={fill=myr}
		},
	#1
}{def}



%================================
% NOTE BOX
%================================

\usetikzlibrary{arrows,calc,shadows.blur}
\tcbuselibrary{skins}
\newtcolorbox{note}[1][]{%
	enhanced jigsaw,
	colback=gray!20!white,%
	colframe=gray!80!black,
	size=small,
	boxrule=1pt,
	title=\colorbox{white!100}{\textbf{ Remarque }},
	halign title=flush center,
	coltitle=black,
	breakable,
	drop shadow=black!50!white,
	attach boxed title to top left={xshift=1cm,yshift=-\tcboxedtitleheight/2,yshifttext=-\tcboxedtitleheight/2},
	minipage boxed title=2.6cm,
	boxed title style={%
			colback=white,
			size=fbox,
			boxrule=1pt,
			boxsep=2pt,
			underlay={%
					\coordinate (dotA) at ($(interior.west) + (-0.5pt,0)$);
					\coordinate (dotB) at ($(interior.east) + (0.5pt,0)$);
					\begin{scope}
						\clip (interior.north west) rectangle ([xshift=3ex]interior.east);
						\filldraw [white, blur shadow={shadow opacity=60, shadow yshift=-.75ex}, rounded corners=2pt] (interior.north west) rectangle (interior.south east);
					\end{scope}
					\begin{scope}[gray!80!black]
						\fill (dotA) circle (2pt);
						\fill (dotB) circle (2pt);
					\end{scope}
				},
		},
	#1,
}

%================================
% STRATÉGIE BOX
%================================

\usetikzlibrary{arrows,calc,shadows.blur}
\tcbuselibrary{skins}
\newtcolorbox{strategy}[1][]{%
	enhanced jigsaw,
	colback=myb!20!white,%
	colframe=gray!80!black,
	size=small,
	boxrule=1pt,
	title=\colorbox{white!100}{\textbf{ Stratégie }},
	halign title=flush center,
	coltitle=black,
	breakable,
	drop shadow=black!50!white,
	attach boxed title to top left={xshift=1cm,yshift=-\tcboxedtitleheight/2,yshifttext=-\tcboxedtitleheight/2},
	minipage boxed title=2.5cm,
	boxed title style={%
			colback=white,
			size=fbox,
			boxrule=1pt,
			boxsep=2pt,
			underlay={%
					\coordinate (dotA) at ($(interior.west) + (-0.5pt,0)$);
					\coordinate (dotB) at ($(interior.east) + (0.5pt,0)$);
					\begin{scope}
						\clip (interior.north west) rectangle ([xshift=3ex]interior.east);
						\filldraw [white, blur shadow={shadow opacity=60, shadow yshift=-.75ex}, rounded corners=2pt] (interior.north west) rectangle (interior.south east);
					\end{scope}
					\begin{scope}[gray!80!black]
						\fill (dotA) circle (2pt);
						\fill (dotB) circle (2pt);
					\end{scope}
				},
		},
	#1,
}

%================================
% MÉTHODE BOX
%================================

\usetikzlibrary{arrows,calc,shadows.blur}
\tcbuselibrary{skins}
\newtcolorbox{methode}[1][]{%
	enhanced jigsaw,
	colback=white,%
	colframe=gray!80!black,
	size=small,
	boxrule=1pt,
	title=\textbf{Méthode},
	halign title=flush center,
	coltitle=black,
	breakable,
	drop shadow=black!50!white,
	attach boxed title to top left={xshift=1cm,yshift=-\tcboxedtitleheight/2,yshifttext=-\tcboxedtitleheight/2},
	minipage boxed title=2.5cm,
	boxed title style={%
			colback=white,
			size=fbox,
			boxrule=1pt,
			boxsep=2pt,
			underlay={%
					\coordinate (dotA) at ($(interior.west) + (-0.5pt,0)$);
					\coordinate (dotB) at ($(interior.east) + (0.5pt,0)$);
					\begin{scope}
						\clip (interior.north west) rectangle ([xshift=3ex]interior.east);
						\filldraw [white, blur shadow={shadow opacity=60, shadow yshift=-.75ex}, rounded corners=2pt] (interior.north west) rectangle (interior.south east);
					\end{scope}
					\begin{scope}[gray!80!black]
						\fill (dotA) circle (2pt);
						\fill (dotB) circle (2pt);
					\end{scope}
				},
		},
	#1,
}

%%%%%%%%%%%%%%%%%%%%%%%%%%%%%%%%%%%%%%%%%%%
% TABLE OF CONTENTS
%%%%%%%%%%%%%%%%%%%%%%%%%%%%%%%%%%%%%%%%%%%

\usepackage{tikz}

\definecolor{doc}{RGB}{0,60,110}
\usepackage{titletoc}
\contentsmargin{0cm}
\titlecontents{chapter}[3.7pc]
{\addvspace{30pt}%
	\begin{tikzpicture}[remember picture, overlay]%
		\draw[fill=doc!60,draw=doc!60] (-7,-.1) rectangle (-0.2,.6);%
		\pgftext[left,x=-3.5cm,y=0.2cm]{\color{white}\Large\sc\bfseries Chapitre\ \thecontentslabel};%
	\end{tikzpicture}\color{doc!60}\large\sc\bfseries}%
{}
{}
{\;\titlerule\;\large\sc\bfseries Page \thecontentspage
	\begin{tikzpicture}[remember picture, overlay]
		\draw[fill=doc!60,draw=doc!60] (2pt,0) rectangle (4,0.1pt);
	\end{tikzpicture}}%
\titlecontents{section}[3.7pc]
{\addvspace{2pt}}
{\contentslabel[\thecontentslabel]{2pc}}
{}
{\hfill\small \thecontentspage}
[]
\titlecontents*{subsection}[3.7pc]
{\addvspace{-1pt}\small}
{}
{}
{\ --- \small\thecontentspage}
[ \textbullet\ ][]

\makeatletter
\renewcommand{\tableofcontents}{%
	\chapter*{%
	  \vspace*{-20\p@}%
	  \begin{tikzpicture}[remember picture, overlay]%
		  \pgftext[right,x=15cm,y=0.2cm]{\color{doc!60}\Huge\sc\bfseries \contentsname};%
		  \draw[fill=doc!60,draw=doc!60] (13,-.75) rectangle (20,1);%
		  \clip (13,-.75) rectangle (20,1);
		  \pgftext[right,x=15cm,y=0.2cm]{\color{white}\Huge\sc\bfseries \contentsname};%
	  \end{tikzpicture}}%
	\@starttoc{toc}}
\makeatother


%%%%%%%%%%%%%%%%%%%%%%%%%%%%%%%%%%%%%%%%%%%
% MINTED FOR PYTHON ALGORITHMS
%%%%%%%%%%%%%%%%%%%%%%%%%%%%%%%%%%%%%%%%%%%

\usepackage{tcolorbox}
\tcbuselibrary{minted,breakable,xparse,skins}
\definecolor{bg}{gray}{0.95}
\DeclareTCBListing{mintedbox}{O{}m!O{}}{%
  breakable=true,
  listing engine=minted,
  listing only,
  minted language=#2,
  minted style=default,
  minted options={%
    linenos,
    gobble=0,
    breaklines=true,
    breakafter=,,
    fontsize=\small,
    numbersep=8pt,
    #1},
  boxsep=0pt,
  left skip=0pt,
  right skip=0pt,
  left=25pt,
  right=0pt,
  top=3pt,
  bottom=3pt,
  arc=5pt,
  leftrule=0pt,
  rightrule=0pt,
  bottomrule=2pt,
  toprule=2pt,
  colback=bg,
  colframe=orange!70,
  enhanced,
  overlay={%
    \begin{tcbclipinterior}
    \fill[orange!20!white] (frame.south west) rectangle ([xshift=20pt]frame.north west);
    \end{tcbclipinterior}},
  #3}
  
  
 % for braces
\usetikzlibrary{decorations.pathreplacing}


\SetDate[02/12/2025]
\reversemarginpar
\setlength{\marginparsep}{.5cm}

\begin{document}
\pagestyle{fancy}
\fancyhead[L]{Seconde}
\fancyhead[C]{\textbf{Évaluation — Fonctions}}
\fancyhead[R]{\today}

\null\vspace{-30pt}
%\marginpar{A}
Consignes particulières : 
\begin{itemize}[label=$\bullet$]
	\item 
	La calculatrice est {interdite}.
	%\item
	%Les sujets d'évaluation sont individuels. Écrire son nom avant de rendre son sujet.
	\item
	Les exercices \ref{exe:prop-fond} et \ref{exe:graph} peuvent être entièrement faits sur la feuille d'évaluation. Écrire son nom avant de rendre son sujet.
	\item
	L'évaluation fait 2 pages. La somme des points est \total{points}.
\end{itemize}

\marginpar{[pts]}
\hrule

\begin{thm}[propriété fondamentale]\label{thm:1}
	Soit $f$ une fonction sur un domaine $\D$ et $(x;y)$ un point du plan avec $x\in\D$.
	Alors
		\begin{align*}
			(x ; y) \in \C_f && \iff && \underline{\qquad} = \underline{\qquad\qquad}.
		\end{align*}
\end{thm}

\exe{2}{
	Compléter le théorème \ref{thm:1} vu en cours.
}{exe:prop-fond}{
		\begin{align*}
			(x ; y) \in \C_f && \iff && y = f(x).
		\end{align*}
}

\exe{4, difficulty=1}{
	Tracer les courbes représentatives des fonctions $f(x) = x^2$ et $g(x) = (x+3)^2$ dans le repère figure \ref{fig:1}. 
	Il n'est pas nécessaire de justifier.
	
	Quel est le domaine d'étude $\D$ des fonctions ici ?
}{exe:graph}{
	Le domaine $\D$ est $[-8 ; 5]$ ici.

	\begin{center}
	\begin{tikzpicture}[>=stealth]
		\begin{axis}[xmin = -8, xmax=5, ymin=-5.1, ymax=20.1, axis x line=middle, axis y line=middle, axis line style=->, grid=both, clip=true, x=1cm]
			\addplot[no marks, BLUE_E, very thick, -] expression[domain=-8:5, samples=50]{x^2}
			node[pos=.6, right]{$\mathcal{C}_f$};
			\addplot[no marks, RED_E, very thick, -] expression[domain=-8:5, samples=50]{(x+3)^2}
			node[pos=.17, right]{$\mathcal{C}_g$};
		\end{axis}
	\end{tikzpicture}
	\end{center}
}

\begin{figure}[h!]
	\centering
	\begin{tikzpicture}[>=stealth]
		\begin{axis}[xmin = -8, xmax=5, ymin=-3.1, ymax=20.1, axis x line=middle, axis y line=middle, axis line style=->, grid=both, clip=true, x=1cm]
			\addplot[no marks, BLUE_E, very thick, -, transparent] expression[domain=-8:5, samples=50]{x^2}
			node[pos=.6, right]{$\mathcal{C}_f$};
			\addplot[no marks, RED_E, very thick, -, transparent] expression[domain=-8:5, samples=50]{(x+3)^2}
			node[pos=.5, right]{$\mathcal{C}_g$};
		\end{axis}
	\end{tikzpicture}
	\caption{Repère de l'exercice \ref{exe:graph}.}
	\label{fig:1}
\end{figure}

\exe{2}{
	Quelle est la différence entre $\bigset{ 0 ; 1 }$ et $[0 ; 1]$ ? 
	Décrire avec des mots ce que les deux notations désignent.
}{exe:diff-notations}{
	L'ensemble $\bigset{ 0 ; 1 }$ est constitué de seulement deux éléments : 0, et 1.
	
	L'intervalle $[0 ;1]$ correspond à l'ensemble de tous les nombres entre 0 et 1 inclus.
}


\exe{4, difficulty=1}{
	Soient
	\begin{align*}
		f(x) = (x+1)(x+6) && \et && g(x) = (x-2)(x+4)
	\end{align*}
	deux fonctions sur $\R$.
	
	\begin{enumerate}
		\item En justifiant les calculs, développer et réduire les expressions algébriques de $f$ et de $g$ pour trouver $f(x) = x^2 + 7x + 6$ et $g(x) = x^2 + 2x - 8$.
		\item Montrer que l'équation $f(x) = g(x)$ est équivalente à $5x+14=0$.
		\item Résoudre $5x+14=0$ pour $x\in\R$.
		\item À quoi correspond \underline{graphiquement} le réel $x$ qui vérifie l'équation $f(x) = g(x)$ ?
	\end{enumerate}
}{exe:image-selon-forme}{
	\begin{enumerate}
		\item
		Par double distributivité.
		\item
		Comme $f(x) = g(x) \iff f(x) - g(x) = 0$, on a bien
			\[ x^2  + 7x + 6 - (x^2 + 2x - 8) = 0 \iff x^2 - x^2 + 7x - 2x + 6 + 8 = 0 \iff 5x + 14 = 0. \]
		\item
			\[ 5x+14 = 0 \iff 5x = -14 \iff x = -\dfrac{14}5. \]
		\item
		Graphiquement, $x = -\frac{14}5 = -2,8$ est l'abscisse du point d'intersection de $\C_f$ et $\C_g$.
		Notons d'ailleurs qu'il n'existe qu'un seul point d'intersection des courbes.
	\end{enumerate}
}

\newpage

\exe{2, difficulty=0}{
	Considérons la fonction $f$ définie algébriquement sur $\R$ par
		\[ f(x) = \dfrac13 - \dfrac{x}2. \]
	Pour chaque point suivant, déterminer s'il appartient à $\C_f$ ou non.
	
	\begin{multicols}{2}
	\begin{enumerate}[label=\roman*)]
		\item $\left(1 ;-\dfrac16\right)$
		\item $\left(-\dfrac23 ; 0\right)$
	\end{enumerate}
	\end{multicols}

}{exe:Cf}{
	On utilise le théorème \ref{thm:1} qui demande de vérifier si $y=f(x)$.
	\begin{multicols}{2}
	\begin{enumerate}[label=\roman*)]
		\item $\left(-1 ;-\dfrac16\right) \in \C_f$
		\item $\left(-\dfrac23 ; 0\right) \not\in \C_f$
	\end{enumerate}
	\end{multicols}


}

\exe{4, difficulty=1}{
	Considérons la représentation graphique suivante d'une fonction $f$ définie sur \mbox{$\D = [-3 ; 2,3]$}.
	
	\begin{center}
		\begin{tikzpicture}[>=stealth]
			\begin{axis}[xmin = -3, xmax=2.3, ymin=-2.1, ymax=5.1, axis x line=middle, axis y line=middle, axis line style=->, grid=both,
			grid style = {opacity=.5},
			x=2.1cm,
			xtick={-3, -2, ..., 2},
			y=18pt,
			clip=true,
			ytick distance = 1,
			]
				\addplot[no marks, BLUE_E, very thick, -] expression[domain=-4:3, samples=200]{x^3 /3 - 2*x +2-2*sin(180*x)}node[pos=.52, above=5pt]{$\C_f$};
			\end{axis}
		\end{tikzpicture}
	\end{center}
	\begin{enumerate}
		\item
		Donner approximativement les images de $-0,5$ et de 1,5 par $f$. 
		\item
		Énumérer approximativement les antécédents de 5 et de 1 par $f$.
		\item 
		Exprimer aproximativement l'ensemble $\bigset{ x \in \D \tq f(x) \geq 1 }$ sous forme d'union d'intervalles.
	\end{enumerate}
}{exe:deg3}{
	\begin{enumerate}
		\item
		$f(-0,5) \approx 4,8$ et $f(1,5) \approx 2,1$.
		\item
		L'antécédent de 5 est environ -0,6.
		En effet, $f(-0,6) \approx 5$
		
		Les antécédents de 1 sont environ -2,8 ; 0,1 ; 1,2 ; et 1,9.
		En effet, $f(-2,8) \approx f(0,1) \approx f(1,2) \approx f(1,9) \approx 1$.
		\item 
		
		\[ \bigset{ x \in \D \tq f(x) \geq 1 } \approx [-2,8 ; 0,1 ] \cup [ 1,2 ; 1,9 ]. \]
	\end{enumerate}
}


\exe{2, difficulty=1}{
	Soient $a, b \in \R$ deux nombres réels.
	
	Les deux questions suivantes peuvent être traitées séparément.
	\begin{enumerate}
		\item
		Développer et réduire l'expression $(a+b)^2$.
		\item
		Montrer que $(a+b)^2$ n'est pas égal à $a^2 + b^2$ en général.
	\end{enumerate}
}{exe:freshmans-dream}{
	\begin{enumerate}
		\item
			\begin{align*}
				(a+b)^2 &= (a+b)(a+b) \\
						&= a^2 + ab + ba + b^2 \\
						&= a^2 + b^2 + 2ab
			\end{align*}
		où on a utilisé que $ab = ba$ à la dernière égalité car la multiplication est commutative dans $\R$.
		\item
		En choisissant des valeurs de $a$ et de $b$ au hasard, on a de grandes chances de tomber sur un contre-exemple.
		En fait, d'après la question 1,
			\[ (a+b)^2 = a^2 + b^2 \iff ab = 0, \]
		si et seulement si $a$ ou $b$ est nul.
		
		Prendre $(a ; b) = (1 ; 2)$ donne $(a+b)^2 = 3^2 = 9$ d'une part, et $a^2 + b^2 = 1 + 4 = 5$ d'autre part, ce qui montre que $(a+b)^2 \neq a^2 + b^2$ en général.
	\end{enumerate}
}



\exemulticols{2}{
	Considérons l'ensemble de points $E$ représenté dans le repère ci-contre.
	
	\vspace{20pt}
	
	$E$ peut-il être la courbe représentative d'une fonction $f$ ?
	Justifier rigoureusement.
	
	\vfill\null
}{
	\begin{center}
		\begin{tikzpicture}[>=stealth]
			\begin{axis}[xmin = 0, xmax=5, ymin=0, ymax=5, axis x line=middle, axis y line=middle, axis line style=->, grid=both,
			grid style = {opacity=.5},
			extra x ticks = {0},
			extra y ticks = {0},
			]
				\addplot[no marks, BLUE_E, very thick, -, no markers, smooth, tension=2] coordinates{
					(0,1)
					(1,.5)
					(1,1)
					(2,3)
					(3,2)
					(2,2.25)
					(1.5,3)
					(3,4)
					(4,1)
					(5,1.5)
				}
				node[pos=.7, above=20pt]{$\boldsymbol{E}$};
			\end{axis}
		\end{tikzpicture}
	\end{center}
}{exe:Cf-not-Cf}{
	L'ensemble $E$ ne peut pas être courbe représentative d'une fonction $f$, car il existe plusieurs points de même abscisse (par exemple $(2 ; 2,2), (2 ; 3), (2 ; 4,4)$ environ).
	En effet, la courbe représentative d'une fonction est construite à partir de points $\bigl(x ; f(x)\bigr)$ : chaque abscisse $x$ donne naissance à un unique point, et chaque point correspond à une unique abscisse.
}

%%%%%%%%%%%

\newpage
\fancyhead[C]{\textbf{Solutions}}
\shipoutAnswer
	
\end{document}

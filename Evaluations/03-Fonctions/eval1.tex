% DYSLEXIA SWITCH
\newif\ifdys
		
				% ENABLE or DISABLE font change
				% use XeLaTeX if true
				\dystrue
				\dysfalse


\ifdys

\documentclass[a4paper, 14pt]{extarticle}
\usepackage{amsmath,amsfonts,amsthm,amssymb,mathtools}

\tracinglostchars=3 % Report an error if a font does not have a symbol.
\usepackage{fontspec}
\usepackage{unicode-math}
\defaultfontfeatures{ Ligatures=TeX,
                      Scale=MatchUppercase }

\setmainfont{OpenDyslexic}[Scale=1.0]
\setmathfont{Fira Math} % Or maybe try KPMath-Sans?
\setmathfont{OpenDyslexic Italic}[range=it/{Latin,latin}]
\setmathfont{OpenDyslexic}[range=up/{Latin,latin,num}]

\else

\documentclass[a4paper, 12pt]{extarticle}

\usepackage[utf8x]{inputenc}
%fonts
\usepackage{amsmath,amsfonts,amsthm,amssymb,mathtools}
% comment below to default to computer modern
\usepackage{libertinus,libertinust1math}

\fi


\usepackage[french]{babel}
\usepackage[
a4paper,
margin=2cm,
nomarginpar,% We don't want any margin paragraphs
]{geometry}
\usepackage{icomma}

\usepackage{fancyhdr}
\usepackage{array}
\usepackage{hyperref}

\usepackage{multicol, enumerate}
\newcolumntype{P}[1]{>{\centering\arraybackslash}p{#1}}


\usepackage{stackengine}
\newcommand\xrowht[2][0]{\addstackgap[.5\dimexpr#2\relax]{\vphantom{#1}}}

% theorems

\theoremstyle{plain}
\newtheorem{theorem}{Th\'eor\`eme}
\newtheorem*{sol}{Solution}
\theoremstyle{definition}
\newtheorem{ex}{Exercice}
\newtheorem*{rpl}{Rappel}
\newtheorem{enigme}{Énigme}

% corps
\usepackage{calrsfs}
\newcommand{\C}{\mathcal{C}}
\newcommand{\R}{\mathbb{R}}
\newcommand{\Rnn}{\mathbb{R}^{2n}}
\newcommand{\Z}{\mathbb{Z}}
\newcommand{\N}{\mathbb{N}}
\newcommand{\Q}{\mathbb{Q}}

% variance
\newcommand{\Var}[1]{\text{Var}(#1)}

% domain
\newcommand{\D}{\mathcal{D}}


% date
\usepackage{advdate}
\AdvanceDate[0]


% plots
\usepackage{pgfplots}

% table line break
\usepackage{makecell}
%tablestuff
\def\arraystretch{2}
\setlength\tabcolsep{15pt}

%subfigures
\usepackage{subcaption}

\definecolor{myg}{RGB}{56, 140, 70}
\definecolor{myb}{RGB}{45, 111, 177}
\definecolor{myr}{RGB}{199, 68, 64}

% fake sections with no title to move around the merged pdf
\newcommand{\fakesection}[1]{%
  \par\refstepcounter{section}% Increase section counter
  \sectionmark{#1}% Add section mark (header)
  \addcontentsline{toc}{section}{\protect\numberline{\thesection}#1}% Add section to ToC
  % Add more content here, if needed.
}


% SOLUTION SWITCH
\newif\ifsolutions
				\solutionstrue
				%\solutionsfalse

\ifsolutions
	\newcommand{\exe}[2]{
		\begin{ex} #1  \end{ex}
		\begin{sol} #2 \end{sol}
	}
\else
	\newcommand{\exe}[2]{
		\begin{ex} #1  \end{ex}
	}
	
\fi


% tableaux var, signe
\usepackage{tkz-tab}


%pinfty minfty
\newcommand{\pinfty}{{+}\infty}
\newcommand{\minfty}{{-}\infty}

\begin{document}


\SetDate[10/10/2025]
\reversemarginpar
\setlength{\marginparsep}{.5cm}

\begin{document}
\pagestyle{fancy}
\fancyhead[L]{Seconde}
\fancyhead[C]{\textbf{Évaluation blanche — Fonctions}}
\fancyhead[R]{\today}

\null\vspace{-30pt}
Consignes particulières : 
\begin{itemize}[label=$\bullet$]
	\item 
	La calculatrice est {interdite}.
	\item
	Les sujets d'évaluation sont individuels. Écrire son nom avant de rendre son sujet.
	\item
	L'exercice \ref{exe:prop-fond} peut être entièrement fait sur la feuille d'évaluation.
	\item
	L'évaluation fait 2 pages. La somme des points est \total{points}.
\end{itemize}

\marginpar{[pts]}
\hrule

\begin{thm}[propriété fondamentale]\label{thm:1}
	Soit $f$ une fonction réelle sur un domaine $\D$ et $(x;y)$ un point du plan avec $x\in\D$.
	Alors
		\begin{align*}
			(x ; y) \in \C_f && \iff && \underline{\qquad} = \underline{\qquad\qquad}.
		\end{align*}
\end{thm}

\exe{2}{
	Compléter le théorème \ref{thm:1} vu en cours.
}{exe:prop-fond}{
	\[ y = f(x). \]
}

\exe{4, difficulty=1}{
	Considérons la courbe $\C_f$ graphiquement ci-dessous, figure \ref{fig:1}.
	Dans cet exercice, on souhaite étudier la courbe représentative de $g(x) = f(x) + 2$.
	
	\begin{enumerate}
		\item Remplir le tableau de valeurs de $f$ et de $g$ ci-dessous, figure \ref{fig:2}.
		\item Grapher $\C_g$ dans le même repère que $\C_f$.
	\end{enumerate}
	
}{exe:f-parentes-sum}{
	todo
}

\begin{figure}[h!]
	\begin{center}
	\begin{tikzpicture}[scale=1.1]
		\begin{axis}[xmin = -10, xmax=7, ymin=-4.25, ymax=3.25, axis x line=middle, axis y line=middle, axis line style=->, grid=both,
		ytick={-4,-3,...,4},
		x=20pt,
	    	]
		\addplot[no marks, myb, -, very thick] expression[domain=-10:7, samples=50]{.01*(x+6)*(x+3)*(x-5)}
		node[pos=.68, below]{$\mathcal{C}_f$};
		\end{axis}
	\end{tikzpicture}
	\end{center}
	\caption{Courbe représentative de l'exercice \ref{exe:f-parentes-sum}.}
	\label{fig:1}
\end{figure}

\begin{figure}[h!]
	\begin{center}
	\def\arraystretch{1.5}
	\setlength\tabcolsep{20pt}
	\begin{tabular}{|c|c|c|c|c|c|c|c|c|c|}\hline
		$x$ & -10 & -8 & &&&&& & 6 \\ \hline
		$f(x)$ &  &  &  & &&&&& \\ \hline
		$g(x)$ &  &  &  & &&&&& \\ \hline
	\end{tabular} 
	\end{center}
	\caption{Tableau de valeurs de l'exercice \ref{exe:f-parentes-sum}.}
	\label{fig:2}
\end{figure}

\newpage

\exe{4}{
	Un fonction $f$ admet le tableau de valeurs suivant.
		\begin{center}
		\def\arraystretch{1.2}
		\setlength\tabcolsep{20pt}
		\begin{tabular}{|c|c|c|c|c|}\hline
			$x$ & 0 & -2 & 1 & -1 \\ \hline
			$f(x)$ & 1 & 0 & 0 & 1 \\ \hline
		\end{tabular}
		\end{center}
	Parmis les expressions algébriques suivantes, lesquelles peuvent et lesquelles ne peuvent pas correspondre à $f(x)$ ?
	Justifier.
		\begin{multicols}{4}
		\begin{enumerate}[label=\roman*)]
			\item $1-x$
			\item $1+\dfrac{x}2$
			\item $\dfrac{1-x}2$
			\item $\dfrac{-x^2 - x + 2}2$
		\end{enumerate}
		\end{multicols}
}{exe:f4}{
	todo
}


\exe{3, difficulty=1}{
	Soit $f(x) = (x+10)(x-15)$, une fonction sur $\R$.
	\begin{enumerate}
		\item Développer l'expression pour vérifier que $f(x) = x^2 - 5x - 150$.
		\item Calculer les images de 0 ; de -10 ; et de 16 par $f$ avec la forme adéquate.
		\item Comment appelle-t-on les deux différentes formes de $f$ ?
	\end{enumerate}
}{exe:image-selon-forme}{
	todo
}

\exe{2}{
	Considérons l'expression $f(x)$ suivante, définie lorsque $x$ est non nul.
		\[ f(x) = \dfrac{2}{3x} \]
	Pour chaque point suivant, déterminer s'il appartient à $\C_f$ ou non.
	
	\begin{multicols}{2}
	\begin{enumerate}[label=\roman*)]
		%\item $\left(2; 1\right)$
		\item $\left(-1 ;-\dfrac23\right)$
		%\item $\left(\dfrac12 ; \dfrac13\right)$
		\item $\left(-\dfrac23 ; -1\right)$
	\end{enumerate}
	\end{multicols}

}{exe:Cf}{

	\begin{multicols}{2}
	\begin{enumerate}[label=\roman*)]
		%\item $\left(2; 1\right) \not\in \C_f$.
		\item $\left(-1 ;-\dfrac23\right) \in \C_f$
		%\item $\left(\dfrac12 ; \dfrac13\right) \not\in \C_f$
		\item $\left(-\dfrac23 ; -1\right) \in \C_f$
	\end{enumerate}
	\end{multicols}


}

\exe{4, difficulty=1}{
	Considérons la représentation graphique suivante d'une fonction $f$ définie sur \mbox{$\D = ]{-}3,4 ; 2,3[$}.
	
	\begin{center}
		\begin{tikzpicture}[>=stealth]
			\begin{axis}[xmin = -3.4, xmax=2.3, ymin=-2.1, ymax=5.1, axis x line=middle, axis y line=middle, axis line style=->, grid=both,
			grid style = {opacity=.5},
			x=2.1cm,
			xtick={-3, -2, ..., 2},
			y=18pt,
			clip=true,
			]
				\addplot[no marks, BLUE_E, very thick, -] expression[domain=-4:3, samples=200]{x^3 /3 - 2*x +3}node[pos=.52, above=5pt]{$\C_f$};
			\end{axis}
		\end{tikzpicture}
	\end{center}
	\begin{enumerate}
		\item
		Donner approximativement les images de -1 et de 2 par $f$. 
		\item
		Énumérer approximativement les antécédents de 2 et de 4 par $f$.
		\item 
		Exprimer aproximativement l'ensemble $\{ x \in \D \tq f(x) \leq 4 \}$ sous forme d'union d'intervalles.
	\end{enumerate}
}{exe:deg3}{
	todo
}

\exe{2}{
	Quelle est la différence entre $(0 ; 1)$ et $[0 ; 1]$ ? 
	Décrire, avec des mots, ce que les deux notations désignent.
}{exe:diff-notations}{
	Le couple de coordonnées $(0;1)$ correspond à un point dans le plan cartésien.
	
	L'intervalle $[0 ;1]$ correspond à l'ensembles des nombres entre 0 inclus et 1 inclus.
}

%%%%%%%%%%%

\newpage
\fancyhead[C]{\textbf{Solutions}}
\shipoutAnswer
	
\end{document}

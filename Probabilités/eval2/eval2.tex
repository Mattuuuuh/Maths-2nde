				% ENABLE or DISABLE font change
				% use XeLaTeX if true
\newif\ifdys
				\dystrue
				\dysfalse

\newif\ifsolutions
				\solutionstrue
				%\solutionsfalse

%!TEX encoding = UTF8
%!TEX root =notes.tex


%%%%%%%%%%%%%%%%%%%%%%%%%%%%%%%%%
% PACKAGE IMPORTS
%%%%%%%%%%%%%%%%%%%%%%%%%%%%%%%%%


\usepackage[french]{babel}

\usepackage[tmargin=2cm,rmargin=1in,lmargin=1in,margin=0.85in,bmargin=2cm,footskip=.2in]{geometry}
\usepackage{amsmath,amsfonts,amsthm,amssymb,mathtools}
\usepackage[varbb]{newpxmath}
\usepackage{xfrac}
\usepackage[makeroom]{cancel}
\usepackage{mathtools}
\usepackage{bookmark}
\usepackage{enumitem}
\usepackage{hyperref,theoremref}
\hypersetup{
	pdftitle={Assignment},
	colorlinks=true, linkcolor=doc!90,
	bookmarksnumbered=true,
	bookmarksopen=true
}
\usepackage[most,many,breakable]{tcolorbox}
\usepackage{xcolor}
\usepackage{varwidth}
\usepackage{varwidth}
\usepackage{etoolbox}
%\usepackage{authblk}
\usepackage{nameref}
\usepackage{multicol,array}
\usepackage{tikz-cd}
\usepackage[ruled,vlined,linesnumbered]{algorithm2e}
\usepackage{comment} % enables the use of multi-line comments (\ifx \fi) 
\usepackage{import}
\usepackage{xifthen}
\usepackage{pdfpages}
\usepackage{transparent}


\newcommand\mycommfont[1]{\footnotesize\ttfamily\textcolor{blue}{#1}}
\SetCommentSty{mycommfont}
\newcommand{\incfig}[1]{%
    \def\svgwidth{\columnwidth}
    \import{./figures/}{#1.pdf_tex}
}

\usepackage{tikzsymbols}
%\renewcommand\qedsymbol{$\Laughey$}


%\usepackage{import}
%\usepackage{xifthen}
%\usepackage{pdfpages}
%\usepackage{transparent}


%%%%%%%%%%%%%%%%%%%%%%%%%%%%%%
% SELF MADE COLORS
%%%%%%%%%%%%%%%%%%%%%%%%%%%%%%



\definecolor{myg}{RGB}{56, 140, 70}
\definecolor{myb}{RGB}{45, 111, 177}
\definecolor{myr}{RGB}{199, 68, 64}
\definecolor{mytheorembg}{HTML}{F2F2F9}
\definecolor{mytheoremfr}{HTML}{00007B}
\definecolor{mylenmabg}{HTML}{FFFAF8}
\definecolor{mylenmafr}{HTML}{983b0f}
\definecolor{mypropbg}{HTML}{f2fbfc}
\definecolor{mypropfr}{HTML}{191971}
\definecolor{myexamplebg}{HTML}{F2FBF8}
\definecolor{myexamplefr}{HTML}{88D6D1}
\definecolor{myexampleti}{HTML}{2A7F7F}
\definecolor{mydefinitbg}{HTML}{E5E5FF}
\definecolor{mydefinitfr}{HTML}{3F3FA3}
\definecolor{notesgreen}{RGB}{0,162,0}
\definecolor{myp}{RGB}{197, 92, 212}
\definecolor{mygr}{HTML}{2C3338}
\definecolor{myred}{RGB}{127,0,0}
\definecolor{myyellow}{RGB}{169,121,69}
\definecolor{myexercisebg}{HTML}{F2FBF8}
\definecolor{myexercisefg}{HTML}{88D6D1}


%%%%%%%%%%%%%%%%%%%%%%%%%%%%
% TCOLORBOX SETUPS
%%%%%%%%%%%%%%%%%%%%%%%%%%%%

\setlength{\parindent}{1cm}
%================================
% THEOREM BOX
%================================

\tcbuselibrary{theorems,skins,hooks}
\newtcbtheorem[number within=chapter]{Theorem}{Théorème}
{%
	enhanced,
	breakable,
	colback = mytheorembg,
	frame hidden,
	boxrule = 0sp,
	borderline west = {2pt}{0pt}{mytheoremfr},
	sharp corners,
	detach title,
	before upper = \tcbtitle\par\smallskip,
	coltitle = mytheoremfr,
	fonttitle = \bfseries\sffamily,
	description font = \mdseries,
	separator sign none,
	segmentation style={solid, mytheoremfr},
}
{th}


\tcbuselibrary{theorems,skins,hooks}
\newtcolorbox{Theoremcon}
{%
	enhanced
	,breakable
	,colback = mytheorembg
	,frame hidden
	,boxrule = 0sp
	,borderline west = {2pt}{0pt}{mytheoremfr}
	,sharp corners
	,description font = \mdseries
	,separator sign none
}

%================================
% Corollery
%================================
\tcbuselibrary{theorems,skins,hooks}
\newtcbtheorem[use counter=tcb@cnt@Theorem]{Corollary}{Corollaire}
{%
	enhanced
	,breakable
	,colback = myp!10
	,frame hidden
	,boxrule = 0sp
	,borderline west = {2pt}{0pt}{myp!85!black}
	,sharp corners
	,detach title
	,before upper = \tcbtitle\par\smallskip
	,coltitle = myp!85!black
	,fonttitle = \bfseries\sffamily
	,description font = \mdseries
	,separator sign none
	,segmentation style={solid, myp!85!black}
}
{th}

%================================
% LENMA
%================================

\tcbuselibrary{theorems,skins,hooks}
\newtcbtheorem[use counter=tcb@cnt@Theorem]{Lemma}{Lemme}
{%
	enhanced,
	breakable,
	colback = mylenmabg,
	frame hidden,
	boxrule = 0sp,
	borderline west = {2pt}{0pt}{mylenmafr},
	sharp corners,
	detach title,
	before upper = \tcbtitle\par\smallskip,
	coltitle = mylenmafr,
	fonttitle = \bfseries\sffamily,
	description font = \mdseries,
	separator sign none,
	segmentation style={solid, mylenmafr},
}
{th}


%================================
% PROPOSITION
%================================

\tcbuselibrary{theorems,skins,hooks}
\newtcbtheorem[use counter=tcb@cnt@Theorem]{Prop}{Proposition}
{%
	enhanced,
	breakable,
	colback = mypropbg,
	frame hidden,
	boxrule = 0sp,
	borderline west = {2pt}{0pt}{mypropfr},
	sharp corners,
	detach title,
	before upper = \tcbtitle\par\smallskip,
	coltitle = mypropfr,
	fonttitle = \bfseries\sffamily,
	description font = \mdseries,
	separator sign none,
	segmentation style={solid, mypropfr},
}
{th}


%================================
% CLAIM
%================================

\tcbuselibrary{theorems,skins,hooks}
\newtcbtheorem[use counter=tcb@cnt@Theorem]{claim}{Claim}
{%
	enhanced
	,breakable
	,colback = myg!10
	,frame hidden
	,boxrule = 0sp
	,borderline west = {2pt}{0pt}{myg}
	,sharp corners
	,detach title
	,before upper = \tcbtitle\par\smallskip
	,coltitle = myg!85!black
	,fonttitle = \bfseries\sffamily
	,description font = \mdseries
	,separator sign none
	,segmentation style={solid, myg!85!black}
}
{th}



%================================
% Exercise
%================================

\tcbuselibrary{theorems,skins,hooks}
\newtcbtheorem[use counter=tcb@cnt@Theorem]{Exercise}{Exercice}
{%
	enhanced,
	breakable,
	colback = myexercisebg,
	frame hidden,
	boxrule = 0sp,
	borderline west = {2pt}{0pt}{myexercisefg},
	sharp corners,
	detach title,
	before upper = \tcbtitle\par\smallskip,
	coltitle = myexercisefg,
	fonttitle = \bfseries\sffamily,
	description font = \mdseries,
	separator sign none,
	segmentation style={solid, myexercisefg},
}
{th}

%================================
% EXAMPLE BOX
%================================

\newtcbtheorem[use counter=tcb@cnt@Theorem]{Example}{Exemple}
{%
	colback = myexamplebg
	,breakable
	,colframe = myexamplefr
	,coltitle = myexampleti
	,boxrule = 1pt
	,sharp corners
	,detach title
	,before upper=\tcbtitle\par\smallskip
	,fonttitle = \bfseries
	,description font = \mdseries
	,separator sign none
	,description delimiters parenthesis
}
{ex}

%================================
% DEFINITION BOX
%================================

\newtcbtheorem[use counter=tcb@cnt@Theorem]{Definition}{Définition}{enhanced,
	before skip=2mm,after skip=2mm, colback=red!5,colframe=red!80!black,boxrule=0.5mm,
	attach boxed title to top left={xshift=1cm,yshift*=1mm-\tcboxedtitleheight}, varwidth boxed title*=-3cm,
	boxed title style={frame code={
					\path[fill=tcbcolback]
					([yshift=-1mm,xshift=-1mm]frame.north west)
					arc[start angle=0,end angle=180,radius=1mm]
					([yshift=-1mm,xshift=1mm]frame.north east)
					arc[start angle=180,end angle=0,radius=1mm];
					\path[left color=tcbcolback!60!black,right color=tcbcolback!60!black,
						middle color=tcbcolback!80!black]
					([xshift=-2mm]frame.north west) -- ([xshift=2mm]frame.north east)
					[rounded corners=1mm]-- ([xshift=1mm,yshift=-1mm]frame.north east)
					-- (frame.south east) -- (frame.south west)
					-- ([xshift=-1mm,yshift=-1mm]frame.north west)
					[sharp corners]-- cycle;
				},interior engine=empty,
		},
	fonttitle=\bfseries,
	title={#2},#1}{def}

%================================
% Solution BOX
%================================

\makeatletter
\newtcbtheorem[use counter=tcb@cnt@Theorem]{question}{Question}{enhanced,
	breakable,
	colback=white,
	colframe=myb!80!black,
	attach boxed title to top left={yshift*=-\tcboxedtitleheight},
	fonttitle=\bfseries,
	title={#2},
	boxed title size=title,
	boxed title style={%
			sharp corners,
			rounded corners=northwest,
			colback=tcbcolframe,
			boxrule=0pt,
		},
	underlay boxed title={%
			\path[fill=tcbcolframe] (title.south west)--(title.south east)
			to[out=0, in=180] ([xshift=5mm]title.east)--
			(title.center-|frame.east)
			[rounded corners=\kvtcb@arc] |-
			(frame.north) -| cycle;
		},
	#1
}{def}
\makeatother

%================================
% SOLUTION BOX
%================================

\makeatletter
\newtcolorbox{solution}{enhanced,
	breakable,
	colback=white,
	colframe=myg!80!black,
	attach boxed title to top left={yshift*=-\tcboxedtitleheight},
	title=Solution,
	boxed title size=title,
	boxed title style={%
			sharp corners,
			rounded corners=northwest,
			colback=tcbcolframe,
			boxrule=0pt,
		},
	underlay boxed title={%
			\path[fill=tcbcolframe] (title.south west)--(title.south east)
			to[out=0, in=180] ([xshift=5mm]title.east)--
			(title.center-|frame.east)
			[rounded corners=\kvtcb@arc] |-
			(frame.north) -| cycle;
		},
}
\makeatother

%================================
% Question BOX
%================================

\makeatletter
\newtcbtheorem[use counter=tcb@cnt@Theorem]{qstion}{Question}{enhanced,
	breakable,
	colback=white,
	colframe=mygr,
	attach boxed title to top left={yshift*=-\tcboxedtitleheight},
	fonttitle=\bfseries,
	title={#2},
	boxed title size=title,
	boxed title style={%
			sharp corners,
			rounded corners=northwest,
			colback=tcbcolframe,
			boxrule=0pt,
		},
	underlay boxed title={%
			\path[fill=tcbcolframe] (title.south west)--(title.south east)
			to[out=0, in=180] ([xshift=5mm]title.east)--
			(title.center-|frame.east)
			[rounded corners=\kvtcb@arc] |-
			(frame.north) -| cycle;
		},
	#1
}{def}
\makeatother

\newtcbtheorem[number within=chapter]{wconc}{Wrong Concept}{
	breakable,
	enhanced,
	colback=white,
	colframe=myr,
	arc=0pt,
	outer arc=0pt,
	fonttitle=\bfseries\sffamily\large,
	colbacktitle=myr,
	attach boxed title to top left={},
	boxed title style={
			enhanced,
			skin=enhancedfirst jigsaw,
			arc=3pt,
			bottom=0pt,
			interior style={fill=myr}
		},
	#1
}{def}



%================================
% NOTE BOX
%================================

\usetikzlibrary{arrows,calc,shadows.blur}
\tcbuselibrary{skins}
\newtcolorbox{note}[1][]{%
	enhanced jigsaw,
	colback=gray!20!white,%
	colframe=gray!80!black,
	size=small,
	boxrule=1pt,
	title=\colorbox{white!100}{\textbf{ Remarque }},
	halign title=flush center,
	coltitle=black,
	breakable,
	drop shadow=black!50!white,
	attach boxed title to top left={xshift=1cm,yshift=-\tcboxedtitleheight/2,yshifttext=-\tcboxedtitleheight/2},
	minipage boxed title=2.6cm,
	boxed title style={%
			colback=white,
			size=fbox,
			boxrule=1pt,
			boxsep=2pt,
			underlay={%
					\coordinate (dotA) at ($(interior.west) + (-0.5pt,0)$);
					\coordinate (dotB) at ($(interior.east) + (0.5pt,0)$);
					\begin{scope}
						\clip (interior.north west) rectangle ([xshift=3ex]interior.east);
						\filldraw [white, blur shadow={shadow opacity=60, shadow yshift=-.75ex}, rounded corners=2pt] (interior.north west) rectangle (interior.south east);
					\end{scope}
					\begin{scope}[gray!80!black]
						\fill (dotA) circle (2pt);
						\fill (dotB) circle (2pt);
					\end{scope}
				},
		},
	#1,
}

%================================
% STRATÉGIE BOX
%================================

\usetikzlibrary{arrows,calc,shadows.blur}
\tcbuselibrary{skins}
\newtcolorbox{strategy}[1][]{%
	enhanced jigsaw,
	colback=myb!20!white,%
	colframe=gray!80!black,
	size=small,
	boxrule=1pt,
	title=\colorbox{white!100}{\textbf{ Stratégie }},
	halign title=flush center,
	coltitle=black,
	breakable,
	drop shadow=black!50!white,
	attach boxed title to top left={xshift=1cm,yshift=-\tcboxedtitleheight/2,yshifttext=-\tcboxedtitleheight/2},
	minipage boxed title=2.5cm,
	boxed title style={%
			colback=white,
			size=fbox,
			boxrule=1pt,
			boxsep=2pt,
			underlay={%
					\coordinate (dotA) at ($(interior.west) + (-0.5pt,0)$);
					\coordinate (dotB) at ($(interior.east) + (0.5pt,0)$);
					\begin{scope}
						\clip (interior.north west) rectangle ([xshift=3ex]interior.east);
						\filldraw [white, blur shadow={shadow opacity=60, shadow yshift=-.75ex}, rounded corners=2pt] (interior.north west) rectangle (interior.south east);
					\end{scope}
					\begin{scope}[gray!80!black]
						\fill (dotA) circle (2pt);
						\fill (dotB) circle (2pt);
					\end{scope}
				},
		},
	#1,
}

%================================
% MÉTHODE BOX
%================================

\usetikzlibrary{arrows,calc,shadows.blur}
\tcbuselibrary{skins}
\newtcolorbox{methode}[1][]{%
	enhanced jigsaw,
	colback=white,%
	colframe=gray!80!black,
	size=small,
	boxrule=1pt,
	title=\textbf{Méthode},
	halign title=flush center,
	coltitle=black,
	breakable,
	drop shadow=black!50!white,
	attach boxed title to top left={xshift=1cm,yshift=-\tcboxedtitleheight/2,yshifttext=-\tcboxedtitleheight/2},
	minipage boxed title=2.5cm,
	boxed title style={%
			colback=white,
			size=fbox,
			boxrule=1pt,
			boxsep=2pt,
			underlay={%
					\coordinate (dotA) at ($(interior.west) + (-0.5pt,0)$);
					\coordinate (dotB) at ($(interior.east) + (0.5pt,0)$);
					\begin{scope}
						\clip (interior.north west) rectangle ([xshift=3ex]interior.east);
						\filldraw [white, blur shadow={shadow opacity=60, shadow yshift=-.75ex}, rounded corners=2pt] (interior.north west) rectangle (interior.south east);
					\end{scope}
					\begin{scope}[gray!80!black]
						\fill (dotA) circle (2pt);
						\fill (dotB) circle (2pt);
					\end{scope}
				},
		},
	#1,
}

%%%%%%%%%%%%%%%%%%%%%%%%%%%%%%%%%%%%%%%%%%%
% TABLE OF CONTENTS
%%%%%%%%%%%%%%%%%%%%%%%%%%%%%%%%%%%%%%%%%%%

\usepackage{tikz}

\definecolor{doc}{RGB}{0,60,110}
\usepackage{titletoc}
\contentsmargin{0cm}
\titlecontents{chapter}[3.7pc]
{\addvspace{30pt}%
	\begin{tikzpicture}[remember picture, overlay]%
		\draw[fill=doc!60,draw=doc!60] (-7,-.1) rectangle (-0.2,.6);%
		\pgftext[left,x=-3.5cm,y=0.2cm]{\color{white}\Large\sc\bfseries Chapitre\ \thecontentslabel};%
	\end{tikzpicture}\color{doc!60}\large\sc\bfseries}%
{}
{}
{\;\titlerule\;\large\sc\bfseries Page \thecontentspage
	\begin{tikzpicture}[remember picture, overlay]
		\draw[fill=doc!60,draw=doc!60] (2pt,0) rectangle (4,0.1pt);
	\end{tikzpicture}}%
\titlecontents{section}[3.7pc]
{\addvspace{2pt}}
{\contentslabel[\thecontentslabel]{2pc}}
{}
{\hfill\small \thecontentspage}
[]
\titlecontents*{subsection}[3.7pc]
{\addvspace{-1pt}\small}
{}
{}
{\ --- \small\thecontentspage}
[ \textbullet\ ][]

\makeatletter
\renewcommand{\tableofcontents}{%
	\chapter*{%
	  \vspace*{-20\p@}%
	  \begin{tikzpicture}[remember picture, overlay]%
		  \pgftext[right,x=15cm,y=0.2cm]{\color{doc!60}\Huge\sc\bfseries \contentsname};%
		  \draw[fill=doc!60,draw=doc!60] (13,-.75) rectangle (20,1);%
		  \clip (13,-.75) rectangle (20,1);
		  \pgftext[right,x=15cm,y=0.2cm]{\color{white}\Huge\sc\bfseries \contentsname};%
	  \end{tikzpicture}}%
	\@starttoc{toc}}
\makeatother


%%%%%%%%%%%%%%%%%%%%%%%%%%%%%%%%%%%%%%%%%%%
% MINTED FOR PYTHON ALGORITHMS
%%%%%%%%%%%%%%%%%%%%%%%%%%%%%%%%%%%%%%%%%%%

\usepackage{tcolorbox}
\tcbuselibrary{minted,breakable,xparse,skins}
\definecolor{bg}{gray}{0.95}
\DeclareTCBListing{mintedbox}{O{}m!O{}}{%
  breakable=true,
  listing engine=minted,
  listing only,
  minted language=#2,
  minted style=default,
  minted options={%
    linenos,
    gobble=0,
    breaklines=true,
    breakafter=,,
    fontsize=\small,
    numbersep=8pt,
    #1},
  boxsep=0pt,
  left skip=0pt,
  right skip=0pt,
  left=25pt,
  right=0pt,
  top=3pt,
  bottom=3pt,
  arc=5pt,
  leftrule=0pt,
  rightrule=0pt,
  bottomrule=2pt,
  toprule=2pt,
  colback=bg,
  colframe=orange!70,
  enhanced,
  overlay={%
    \begin{tcbclipinterior}
    \fill[orange!20!white] (frame.south west) rectangle ([xshift=20pt]frame.north west);
    \end{tcbclipinterior}},
  #3}
  
  
 % for braces
\usetikzlibrary{decorations.pathreplacing}


\AdvanceDate[2]

\begin{document}
\pagestyle{fancy}
\fancyhead[L]{Seconde 13}
\fancyhead[C]{\textbf{Évaluation blanche : Probabilités \ifsolutions -- Solutions  \fi}}
\fancyhead[R]{\today}


%\exe{
%	Démontrer, à l'aide d'un ou plusieurs diagrammes de Venn, la formule d'inclusion-exclusion
%		\[ P(A\cup B) = P(A) + P(B) - P(A\cap B). \]
%}{}

\exe{
	On considère un univers
		\[ \Omega = \{ -3 ; -2 ; 1 ; 4; 5; 7 ; 13 \}, \]
	ainsi que deux sous-ensembles 
		\begin{align*}
			A = \{ -3 ; 1 ; 5 \} && \text{ et } && B = \{ 1 ; 4 ; 5 ; 7 \}.
		\end{align*}
	\begin{enumerate}
		\item Donner $\overline{A}$, le complémentaire de $A$ dans l'univers $\Omega$.
		\item Vérifier l'égalité
			\[ |A\cup B| = |A| + |B| - |A\cap B| \]
		en calculant le membre de gauche puis le membre de droite et en comparant les valeurs obtenues.
	\end{enumerate}

}{
	\begin{enumerate}
		\item Le complémentaire de $A$ dans $\Omega$ correspond à l'ensemble $\overline{A}$ qui contient tous les éléments de $\Omega$ sauf ceux appartenant à $A$. On note aussi $\Omega \setminus A$ ou $\Omega - A$.
			\[ \overline{A} = \{ -2 ; 4; 7; 13 \}. \]
		\item Le cardinal d'un ensemble est égal au nombre d'éléments distincts lui appartenant.
			À gauche on calcule donc
				\[ |A \cup B| = | \{ -3 ; 1; 5 ;4 ; 7 \} | = 5. \]
			À droite on trouve
				\[ |A| + |B| - |A\cap B| = 3 + 4 - | \{ 1 ; 5 \} | = 3 + 4 - 2 =  5. \]
	\end{enumerate}

}


\exe{
	Démontrer, à l'aide d'un ou plusieurs diagrammes de Venn, la relation d'ensembles
		\[ A = \left(A \cap B \right) \cup \left(A \cap \overline{B} \right). \]
}{
	À l'aide d'un diagramme de Venn on reconnait que $A \cap \overline{B}$ correspond à tous les éléments qui appartiennent à $A$ mais pas à $B$.
	En prenant l'union avec les éléments qui appartiennent à $A$ et à $B$ (l'intersection $A \cap B$), on obtient exactement $A$.
}


\exe{
	On jette simultanément deux D$6$, dés à $6$ faces numérotées de $1$ à $6$, et on calcule la somme des faces du dessus.
	Les dés sont supposés bien équilibrés.
	
	\begin{enumerate}
		\item Donner l'univers $\Omega$ de l'expérience.
		\item Calculer la probabilité d'obtenir exactement $4$.
		\item Est-ce une situation d'équiprobabilité ?
		\item[4($\star$).] Montrer que $P(k) = P(14-k)$ pour tout $k \in \Omega$.
	\end{enumerate}
}{
	\begin{enumerate}
		\item $\Omega = \{2 ; 3 ; 4 ; \dots ; 12 \} = \{ s \in \N \text{ tq. } 2 \leq s \leq 12 \}$.
		\item On différencie les dés pour compter les façons d'obtenir $4$ avant de regrouper les probabilités.
			Il y a $3$ façons différentes : $1+3 = 3+1 = 2+2$.
			Chaque issue, lorsque les dés sont distingués, a pour probabilité $\dfrac16 \times \dfrac16 = \dfrac1{36}$ (faire un arbre de probabilité si besoin).
			D'où $P(4) = 3 \times \dfrac1{36}  = \dfrac1{12}$.
		\item Toutes les issues n'ont pas la même probabilité car, si c'était le cas, toutes les probabilités seraient $\dfrac1{|\Omega|} = \dfrac1{11}$.
		\item[4($\star$).] 
		Pour $k=4$, les paires de dés de somme $10$ sont exactement $6+4 = 4+6 = 5+5$, à comparer avec les paires dont la somme fait $4$.
		
		L'ensemble des paires de résultats dont la somme fait $k$ est en correspondance avec celui des paires de résultats dont la somme fait $14-k$.
		Une paire $(a;b)$ du premier ensemble correspond à une paire $(7-a ; 7-b)$ du deuxième.
		Ces deux ensembles ont donc le même nombre d'éléments et $P(k) = P(14-k)$.
	\end{enumerate}
}

\exe{
	On lance une pièce de monnaie bien équilibrée.
	\begin{enumerate}[label=$\bullet$]
		\item Si elle tombe sur pile, on extrait au hasard une boule dans l'urne 1 qui contient 1 boule noire, 3 boules blanches et 4 boules rouges
		\item Sinon, on extrait une boule dans l'urne 2 qui contient 4 boules noires et 3 boules blanches.
	\end{enumerate}

	Créer un arbre de probabilité pour cette situation et répondre aux questions suivantes.
	\begin{enumerate}
		\item Quelle est la probabilité d'obtenir une boule rouge ?
		\item Quelle est la probabilité d'obtenir une boule noire ou une boule blanche ?
	\end{enumerate}

}
{
	\begin{enumerate}
		\item 
			Il y a une seule façon d'obtenir rouge : d'abord obtenir pile, puis tirer rouge.
			La probabilité est $P(R) = \dfrac12 \cdot \dfrac48 = \dfrac14.$
		\item 
			L'événement est complémentaire à celui de la question $1$.
			D'où
				\[ P( N \cup B) = 1 - P(R) = \dfrac34. \]
	\end{enumerate}

}


\exe{	
	Dans une urne opaque se trouvent un nombre inconnu de boules rouges, bleues, et vertes.
	On tire aléatoirement une boule de l'urne, on note sa couleur, et on la remet dans l'urne.
	Les résultats sont décrits dans le tableau suivant.
	\begin{center}
	\begin{tabular}{|c|c|c|c|} \hline
		Couleur & Rouge & Bleu & Vert \\ \hline
		Nombre de tirages & 64~444 & 42~962 & 35~802 \\ \hline
		Fréquence & & & \\ \hline
	\end{tabular}
	\end{center}
	
	Compléter le tableau en arrondissant à $10^{-2}$ près.
	Modéliser la réalité en définissant un univers et une loi de probabilité, et l'utiliser pour répondre aux questions suivantes.
	\begin{enumerate}
		\item Quelle est la probabilité d'obtenir exactement deux boules bleues en deux tirages ?
		\item Quelle est la probabilité d'obtenir exactement deux boules rouges en trois tirages ?
		\item Quelle est la probabilité d'obtenir au moins une boule bleue en $10$ tirages ?
	\end{enumerate}
}{
	On définit une loi de probabilité $P$ sur l'univers $\Omega = \{ R ; B ; V \}$ vérifiant
		\begin{align*}
			P(R) = 0,45 && P(B) = 0,3 && P(V) = 0,25.
		\end{align*}
	On vérifie bien sûr que la somme fasse $1$ avant d'avancer notre modèle.
	
	\begin{enumerate}
		\item Un arbre binaire de profondeur $2$ avec événements $R$ et $\overline{R}$ peut aider.
			\[ P ( \text{\og deux rouges en deux tirages \fg} ) = 0,45 \cdot 0,45 =0,2025. \]
		\item On prolonge l'arbre précédent pour qu'il ait profondeur $3$. On distingue trois chemins pour obtenir $2$ rouges en $3$ tirages, chacun de probabilité $0,45 \times 0,45 \times 0,55 = 0,111375$.
			Par suite,
			\[ P ( \text{\og deux rouges en trois tirages \fg} ) = 3 \times 0,111375 =0,334125. \]
		\item On considère l'événement complémentaire, qui est de n'obtenir aucune boule bleue en $10$ tirages.
			Un arbre binaire de profondeur $10$ (se l'imaginer devrait suffir) avec événements $B$ et $\overline{B}$ donne
			\[ P ( \text{\og aucun bleu en $10$ tirages \fg} ) = 0,7^{10} \approx 0,0282. \]
			Le passage au complémentaire donne
			\[ P ( \text{\og au moins un bleu en $10$ tirages \fg} ) = 1-0,7^{10} \approx 0,9717. \]
	\end{enumerate}

}


\newpage

\subsection*{Bonus}


\exe{
	Une expérience aléatoire à deux épreuves d'univers $\{A ; B ; C ; D\}$ admet un arbre de probabilité binaire comme ci-dessous, où $t\in\R$ est un paramètre réel encore inconnu.

	\begin{center}
	\begin{tikzpicture}
		% depth 1
		\foreach \i in {-3, 3}
		\draw[-, thick, black] (0,0) node {$\bullet$} -- (\i,-2);
		% depth 2
		\foreach \i in {-3, 3} \foreach \j in {-1, 1}
			\draw[-, thick, black] (\i,-2) node {$\bullet$} -- (\i+\j,-4) node {$\bullet$};
		
		% probas
		\draw (-1.5,-1) node[above left] {$-t$};
		\draw (1.5,-1) node[above right] {$1+t$};
		
		\draw (-3.5,-3) node[above left] {$-t$};
		\draw (-2.5,-3) node[above right] {$1+t$};
		
		\draw (2.5,-3) node[above left] {$-t$};
		\draw (3.5,-3) node[above right] {$1+t$};
		
		% issues
		\draw (-4,-4) node[below] {A};
		\draw (2,-4) node[below] {B};
		\draw (-2,-4) node[below] {C};
		\draw (4,-4) node[below] {D};
	\end{tikzpicture}
	\end{center}
	
	\begin{enumerate}
		\item Montrer qu'on a forcément 
			\[ -1 \leq t \leq 0. \]
		\item Déterminer le paramètre $t$ tel que
			\[ P(D) = \frac19. \]
	\end{enumerate}
}{
	

	\begin{enumerate}
		\item Un probabilité est nécessairement entre $0$ et $1$, donc
			\[ 0 \leq -t \leq 1, \]
		et multiplier par un nombre négatif inverse l'ordre des inégalités :
			\[ -1 \leq t \leq 0. \]
		\item On a la suite d'égalités suivante.
			\begin{align*}
				P(D) &= \dfrac19 \\
				(1+t)^2 &= \dfrac19 \\
				| 1 + t | &= \sqrt{\dfrac19} = \dfrac{\sqrt{1}}{\sqrt{9}} = \dfrac13,
			\end{align*}
		où à la dernière ligne on a utilisé que $\sqrt{x^2} = |x|$ et que $\sqrt{\dfrac{a}{b}} = \dfrac{\sqrt{a}}{\sqrt{b}}$ pour $b$ non nul.
		Les propriétés des racines sont décrites sur un feuille dédiée (semaine 9 décembre sur Éléa).
		
		On continue avec
			\begin{align*}
				1+t = \dfrac13 && \text{ou} && 1+t = -\dfrac13 \\
				t = -\dfrac23 && \text{ou} && t = -\dfrac43
			\end{align*}
		Pour revoir la résolution des équations du type $E^2 = a$, revoir la feuille Fonctions 3 (semaine du 25 novembre sur Éléa).
		
		La première question implique qu'on a nécessairement $t=-\dfrac23$.
		On s'assurera que les probabilités de l'arbre ont bien un sens (ce sont des nombres entre $0$ et $1$).
	\end{enumerate}

}

\end{document}

% DYSLEXIA SWITCH
\newif\ifdys
		
				% ENABLE or DISABLE font change
				% use XeLaTeX if true
				\dystrue
				\dysfalse


\ifdys

\documentclass[a4paper, 14pt]{extarticle}
\usepackage{amsmath,amsfonts,amsthm,amssymb,mathtools}

\tracinglostchars=3 % Report an error if a font does not have a symbol.
\usepackage{fontspec}
\usepackage{unicode-math}
\defaultfontfeatures{ Ligatures=TeX,
                      Scale=MatchUppercase }

\setmainfont{OpenDyslexic}[Scale=1.0]
\setmathfont{Fira Math} % Or maybe try KPMath-Sans?
\setmathfont{OpenDyslexic Italic}[range=it/{Latin,latin}]
\setmathfont{OpenDyslexic}[range=up/{Latin,latin,num}]

\else

\documentclass[a4paper, 12pt]{extarticle}

\usepackage[utf8x]{inputenc}
\usepackage{lmodern,textcomp}
\usepackage{amsmath,amsfonts,amsthm,amssymb,mathtools}

\fi


\usepackage[french]{babel}
\usepackage[
a4paper,
margin=2cm,
nomarginpar,% We don't want any margin paragraphs
]{geometry}
\usepackage{icomma}

\usepackage{fancyhdr}
\usepackage{array}

\usepackage{multicol, enumerate}
\newcolumntype{P}[1]{>{\centering\arraybackslash}p{#1}}


\usepackage{stackengine}
\newcommand\xrowht[2][0]{\addstackgap[.5\dimexpr#2\relax]{\vphantom{#1}}}

% theorems

\theoremstyle{plain}
\newtheorem{theorem}{Th\'eor\`eme}
\newtheorem*{theorem*}{Th\'eor\`eme}
\newtheorem*{sol}{Solution}
\theoremstyle{definition}
\newtheorem{ex}{Exercice}

% corps
\newcommand{\C}{\mathcal{C}}
\newcommand{\R}{\mathbb{R}}
\newcommand{\Rnn}{\mathbb{R}^{2n}}
\newcommand{\Z}{\mathbb{Z}}
\newcommand{\N}{\mathbb{N}}
\newcommand{\Q}{\mathbb{Q}}

% domain
\newcommand{\D}{\mathcal{D}}


% date
\usepackage{advdate}
\AdvanceDate[0]


% plots
\usepackage{pgfplots}

% for calligraphic C
\usepackage{calrsfs}

% SOLUTION SWITCH
\newif\ifsolutions
				\solutionstrue
				\solutionsfalse

\ifsolutions
	\newcommand{\exe}[2]{
		\begin{ex} #1  \end{ex}
		\begin{sol} #2 \end{sol}
	}
\else
	\newcommand{\exe}[2]{
		\begin{ex} #1 \end{ex} 
	}
	
\fi

\begin{document}
\pagestyle{fancy}
\fancyhead[L]{Seconde 13}
\fancyhead[C]{\textbf{Probabilités 1 \ifsolutions -- Solutions  \fi}}
\fancyhead[R]{\today}

\exe{
	SS
}
{AAA}


\subsection*{Exercices supplémentaires}

\exe{
	Soit $\Omega = \{ 0 ; 1; \dots ; 499; 500\} \subseteq \N$ et $A, B \subseteq \Omega$ définis par
		\begin{align*}
			A = \{ n \in \Omega \text{ tq. } 2|n \}, && B = \{ n \in \Omega \text{ tq. } 5|n\}.
		\end{align*}
	\begin{enumerate}
		\item
		Montrer qu'un nombre est multiple de $2$ \textbf{et} de $5$ si et seulement s'il est multiple de $10$.
		\item
		Donner $|A|$, le nombre d'entiers naturels inférieurs ou égaux à $500$ qui sont multiples de $2$.
		\item
		Donner $|B|$, le nombre d'entiers naturels inférieurs ou égaux à $500$ qui sont multiples de $5$.
		\item
		Donner $|A \cap B|$, le nombre d'entiers naturels inférieurs ou égaux à $500$ qui sont multiples de $10$.
		\item
		En déduire $|A \cup B|$, le nombre d'entiers naturels inférieurs ou égaux à $500$ qui sont multiples de $2$ \textbf{ou} de $5$.
	\end{enumerate}

}{}

\end{document}

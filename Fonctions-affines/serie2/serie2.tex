				% ENABLE or DISABLE font change
				% use XeLaTeX if true
\newif\ifdys
				\dystrue
				\dysfalse

\newif\ifsolutions
				\solutionstrue
				\solutionsfalse

% DYSLEXIA SWITCH
\newif\ifdys
		
				% ENABLE or DISABLE font change
				% use XeLaTeX if true
				\dystrue
				\dysfalse


\ifdys

\documentclass[a4paper, 14pt]{extarticle}
\usepackage{amsmath,amsfonts,amsthm,amssymb,mathtools}

\tracinglostchars=3 % Report an error if a font does not have a symbol.
\usepackage{fontspec}
\usepackage{unicode-math}
\defaultfontfeatures{ Ligatures=TeX,
                      Scale=MatchUppercase }

\setmainfont{OpenDyslexic}[Scale=1.0]
\setmathfont{Fira Math} % Or maybe try KPMath-Sans?
\setmathfont{OpenDyslexic Italic}[range=it/{Latin,latin}]
\setmathfont{OpenDyslexic}[range=up/{Latin,latin,num}]

\else

\documentclass[a4paper, 12pt]{extarticle}

\usepackage[utf8x]{inputenc}
%fonts
\usepackage{amsmath,amsfonts,amsthm,amssymb,mathtools}
% comment below to default to computer modern
\usepackage{libertinus,libertinust1math}

\fi


\usepackage[french]{babel}
\usepackage[
a4paper,
margin=2cm,
nomarginpar,% We don't want any margin paragraphs
]{geometry}
\usepackage{icomma}

\usepackage{fancyhdr}
\usepackage{array}
\usepackage{hyperref}

\usepackage{multicol, enumerate}
\newcolumntype{P}[1]{>{\centering\arraybackslash}p{#1}}


\usepackage{stackengine}
\newcommand\xrowht[2][0]{\addstackgap[.5\dimexpr#2\relax]{\vphantom{#1}}}

% theorems

\theoremstyle{plain}
\newtheorem{theorem}{Th\'eor\`eme}
\newtheorem*{sol}{Solution}
\theoremstyle{definition}
\newtheorem{ex}{Exercice}
\newtheorem*{rpl}{Rappel}
\newtheorem{enigme}{Énigme}

% corps
\usepackage{calrsfs}
\newcommand{\C}{\mathcal{C}}
\newcommand{\R}{\mathbb{R}}
\newcommand{\Rnn}{\mathbb{R}^{2n}}
\newcommand{\Z}{\mathbb{Z}}
\newcommand{\N}{\mathbb{N}}
\newcommand{\Q}{\mathbb{Q}}

% variance
\newcommand{\Var}[1]{\text{Var}(#1)}

% domain
\newcommand{\D}{\mathcal{D}}


% date
\usepackage{advdate}
\AdvanceDate[0]


% plots
\usepackage{pgfplots}

% table line break
\usepackage{makecell}
%tablestuff
\def\arraystretch{2}
\setlength\tabcolsep{15pt}

%subfigures
\usepackage{subcaption}

\definecolor{myg}{RGB}{56, 140, 70}
\definecolor{myb}{RGB}{45, 111, 177}
\definecolor{myr}{RGB}{199, 68, 64}

% fake sections with no title to move around the merged pdf
\newcommand{\fakesection}[1]{%
  \par\refstepcounter{section}% Increase section counter
  \sectionmark{#1}% Add section mark (header)
  \addcontentsline{toc}{section}{\protect\numberline{\thesection}#1}% Add section to ToC
  % Add more content here, if needed.
}


% SOLUTION SWITCH
\newif\ifsolutions
				\solutionstrue
				%\solutionsfalse

\ifsolutions
	\newcommand{\exe}[2]{
		\begin{ex} #1  \end{ex}
		\begin{sol} #2 \end{sol}
	}
\else
	\newcommand{\exe}[2]{
		\begin{ex} #1  \end{ex}
	}
	
\fi


% tableaux var, signe
\usepackage{tkz-tab}


%pinfty minfty
\newcommand{\pinfty}{{+}\infty}
\newcommand{\minfty}{{-}\infty}

\begin{document}


\AdvanceDate[0]

\begin{document}
\pagestyle{fancy}
\fancyhead[L]{Seconde 13}
\fancyhead[C]{\textbf{Fonctions affines 2 \ifsolutions -- Solutions  \fi}}
\fancyhead[R]{\today}




\exe{
	L'échelle en degrés Fahrenheit (°F) d'une température se déduit de l'échelle en degrés Celsius (°C) par une fonction affine.
	On fait expérimentalement les mesures suivantes.
		\begin{center}
		\begin{tabular}{|c|c|c|c|c|}\hline
			Température °C & 100 & -10 & 10 & -50 \\ \hline
			Température °F & 212 & 14 & 50 & -58 \\ \hline
		\end{tabular}
		\end{center}
	
	\begin{enumerate}
		\item 
		En notant $a, b\in\R$ les paramètres réels tels que 
			\[ F(C) = aC + b, \]
		montrer que $a = \dfrac95$ et  $b=32$.
		
		\item
		Quand il fait $90$°F, quelle est la température en degrés Celsius ?
		
		\item 
		Pour quelle(s) température(s) les mesures en degrés Fahrenheit et Celsius sont-elles égales ?
		
		\item
		La température $K$ en degrés Kelvin (°K) est liée à $C$ par la relation
			\[ K(C) = C + 273,15. \]
		Exprimer algébriquement la température $K$ en fonction de la température $F$.
	\end{enumerate}
	
}{}

\exe{
    Achille dispute une course avec une tortue. On suppose que les deux participants avancent à vitesse constante. Posons $v = 60$ mètres par minute la vitesse d'Achille.
    La tortue, elle, avance 10 fois moins vite qu’Achille. 
    Achille décide donc de lui laisser généreusement 10 minutes d’avance.

    
	\begin{enumerate}
        \item Calculer la distance parcourue par la tortue après les 10 minutes laissées gracieusement par Achille.
        \item \`A la minute $10+t$, où $t\geq0$, on appelle $A(t)$ et $T(t)$ les distances parcourues respectivement par Achille et la tortue.
        Montrer que
            \begin{align*}
                A(t) = 60 t, && \text{ et } && T(t) = 60 + 6t.
            \end{align*}
        \item Combien de temps Achille met-il pour rattraper la tortue ?
        \item Quelle est la distance parcourue par les deux participants au moment où ils se croisent ?
        \item [5($\star$).] Supposons qu'on ne connaisse pas la vitesse exacte d'Achille mais qu'on sache tout de même qu'elle est strictement positive et que la tortue avance 10 fois moins vite que celui-ci.
        Montrer qu'Achille rattrape la tortue en $\dfrac{10}{9}$ minutes, indépendamment de la vitesse d'Achille.
	\end{enumerate}
	
}{}


\exe{[Intersection]
	Pour chacune des paires de fonctions affines $f,g$ sur $\R$, calculer l'intersection des droites $\C_f \cap \C_g$.
	
	\begin{multicols}{2}
	\begin{enumerate}
		\item $f(x) = 2x + 1, g(x) = -x+1$.
		\item $f(x) = -x + 1, g(x) = 2x + 1$.
		\item $f(x) = 3+7x, g(x) = 2$.
		\item $f(x) = 9-2x, g(x) = 17-x$.
		\item $f(x) = 2x+1, g(x) = 2x+1$.
		\item $f(x) = 2x+1, g(x) = 2x+2$.
	\end{enumerate}
	\end{multicols}
}{

	\begin{enumerate}
		\item $f(x) = 2x + 1, g(x) = -x+1$ .
			On cherche un point $P(x_P, y_P) \in \R^2$ vérifiant les deux équations
				\begin{align*}
					y_P = f(x_P) && \text{ et } && y_P = g(x_P)
				\end{align*}
			Comme bien sûr $y_P = y_P$, on peut résoudre l'équation
				\[ f(x_P) = g(x_P) \]
			pour trouver $x_P$.
			Ici, on pose calmement
				\begin{align*}
					f(x_P) &= g(x_P) \\
					2x_P + 1 &= -x_P + 1 \\
					3x_P &= 0 \\
					x_P &= 0,
				\end{align*}
			ce qui nous fournit $x_P = 0$.
			Reste plus qu'à utiliser que $y_P = f(x_P) = g(x_P)$ pour calculer $y_P$.
			
			D'une part, 
				\[ f(x_P) = f(0) = 2\cdot0 + 1 = 1. \]
			D'autre part, on aurait pû calculer
				\[ g(x_P) = g(0) = -0 + 1 = 1. \]
			Rien de surprenant à ce qu'on trouve la même valeur $y_P = 1$, car $x_P$ a été trouvé vérifiant $f(x_P) = g(x_P)$.
			
			En conclusion, $\C_f \cap \C_g = \{ (0;1) \}$.
			
		\item 
			Similairement,
			\begin{align*}
				f(x_P) &= g(x_P) \\
				-x_P + 1 & = 2x_P + 1 \\
				x_P &= 0,
			\end{align*}
			et $y_P = f(0) = 1$.
			Sans surprise, $(0;1)$ est à nouveau le point d'intersection de $\C_f$ et de $\C_g$ car se sont les mêmes fonctions que ci-dessus mais échangées.
		
		\item 
			Similairement,
			\begin{align*}
				f(x_P) &= g(x_P) \\
				3 + 7x_P & = 2 \\
				7x_P &= -1 \\
				x_P &= \dfrac{-1}7,
			\end{align*}
			\[
				y_P = f(x_P) = 3+7\cdot \dfrac{-1}7  = 2.
			\]
			Il aurait été encore plus facile de calculer $g(x_P) = 2$ à la place !
			
			En conclusion, $\C_f \cap \C_g = \{ \left(-\dfrac72 ; 2\right) \}$.
		
		\item
			Similairement,
			\begin{align*}
				f(x_P) &= g(x_P) \\
				9-2x_P & = 17-x_P \\
				-8 &= x_P,
			\end{align*}
			\[
				y_P = f(x_P) = 9-2 \cdot(-8) = 25.
			\]
		
			En conclusion, $\C_f \cap \C_g = \{ (-8;25) \}$.
		
		\item 			
		Similairement,
			\begin{align*}
				f(x_P) &= g(x_P) \\
				2x_P + 1 & = 2x_P + 1 \\
				0 &= 0,
			\end{align*}
			cette équation étant évidemment vraie, tous les $x_P \in \R$ vérifient l'équation.
			\[
				y_P = f(x_P) = 2x_P + 1.
			\]
			N'importe quel $(x_P ; 2x_P+1) \in \R$ appartient donc à l'intersection des droites.
			
			En conclusion, $\C_f \cap \C_g = \{ (x_P ; 2x_P+1) \text{ tq. } x_P \in \R \} = \{ (x_P ; y_P) \in \R^2 \text{ tq. } y_P = 2x_P + 1 \} = \C_f = \C_g$.
			Il n'est pas surprenant qu'on trouve que toute la droite appartienne à l'intersection, car $f$ et $g$ sont identiques : tous les points de la droite $\C_f$ sont des points d'intersection.
		
		
		\item 
			Similairement,
			\begin{align*}
				f(x_P) &= g(x_P) \\
				2x_P + 1 & = 2x_P + 2 \\
				1 &= 2,
			\end{align*}
			cette équation étant évidemment fausse, aucun $x_P \in \R$ ne vérifie l'équation.
			
			En conclusion, $\C_f \cap \C_g = \emptyset$, l'ensemble vide.
			En fait, les droites $\C_f$ et $\C_g$ sont parallèles car elles ont le même coefficient directeur $2$.
			Or les droites ne sont pas égales car leurs ordonnées à l'origine sont différentes, donc il n'y a aucun point d'intersection.
	\end{enumerate}

}

\newpage

\exe{[Parallélisme]
	Pour chacun des couples de point $P$ et fonction affine $f$ sur $\R$, trouver la fonction affine $g$ telle que $\C_f$ et $\C_g$ soient parallèles et que $P \in \C_g$.
	
	
	\begin{multicols}{2}
	\begin{enumerate}
		\item $P=(1;2)$ et $f(x) = 2x+1$.
		\item $P= (-2; 4)$ et $f(x) = 7-x$.
		\item $P=(-4^{12};-1,7)$ et $f(x) = 1600$.
		\item $P=(21;-50,5)$ et $f(x) = -3x + 5$.
	\end{enumerate}
	\end{multicols}
}{
	On pose $a,b\in\R$ les paramètres de la fonction affine $g(x)=ax+b$, $x\in\R$.
	\begin{enumerate}
		\item 
			L'information de parallélisme donne immédiatement $a=2$.
			D'où $g(x) = 2x+b$ pour tout $x\in\R$.
			
			Reste à trouver $b$ avec l'information d'appartenance.
			Celle-ci est équivalente à la contrainte
				\begin{align*}
					2 &= g(1) \\
					2 &= 2\cdot1 + b \\
					b &= 0
				\end{align*}
			D'où $g(x) = 2x$ ($x\in\R$) est la fonction affine recherchée.
		
		
		\item 
			L'information de parallélisme donne immédiatement $a=-1$.
			D'où $g(x) = -x+b$ pour tout $x\in\R$.
			
			Reste à trouver $b$ avec l'information d'appartenance.
			Celle-ci est équivalente à la contrainte
				\begin{align*}
					2 &= g(1) \\
					2 &= -1 + b \\
					b &= 3
				\end{align*}
			D'où $g(x) = -x+3$ ($x\in\R$) est la fonction affine recherchée.
			
		\item 
			L'information de parallélisme donne immédiatement $a=0$.
			D'où $g(x) =b$ pour tout $x\in\R$.
			
			Reste à trouver $b$ avec l'information d'appartenance.
			Celle-ci est équivalente à la contrainte
				\begin{align*}
					2 &= g(1) \\
					2 &=  b
				\end{align*}
			D'où $g(x) = 2$ ($x\in\R$) est la fonction affine recherchée.
			
		\item 
			L'information de parallélisme donne immédiatement $a=-3$.
			D'où $g(x) = -3x+b$ pour tout $x\in\R$.
			
			Reste à trouver $b$ avec l'information d'appartenance.
			Celle-ci est équivalente à la contrainte
				\begin{align*}
					2 &= g(1) \\
					2 &= -3\cdot1 + b \\
					b &= 5
				\end{align*}
			D'où $g(x) = -3x+5$ ($x\in\R$) est la fonction affine recherchée.
	\end{enumerate}

}

\subsection*{Entraînement}

\exe{[Interpolation]
	Pour chacune des paires de points $A,B$ suivantes, calculer les paramètres de la fonction affine $f$ telle que sa courbe représentative passe par $A$ et $B$.
	
	\begin{multicols}{2}
	\begin{enumerate}
		
		    \item $A(1; 3), B(4; 5)$
		
		    \item $A(0; -2), B(3; 4)$
		
		    \item $A(-3; -1), B(-1; 2)$
		
		    \item $A(2; 7), B(5; 7)$
		
		    \item $A(-4; 6), B(2; -3)$
		\item $A(x_A;y_A), B(x_B; y_B)$

	\end{enumerate}
	\end{multicols}
}
{
	On utilise la convention $a, b \in \R$ pour désigner respectivement le coefficient directeur et l'ordonnée à l'origine de la fonction affine $f$.
	$f(x) = ax+b$ pour tout $x\in\R$.

	\begin{enumerate}
		 \item 
		    \begin{align*}
		        a &= \frac{3-5}{1-4} = \frac{2}{3}, & b &= \frac{1 \cdot 5 - 4 \cdot 3}{1 - 4} = -\dfrac73.
		    \end{align*}
		
		    \item 
		    \begin{align*}
		        a &= \frac{-2 - 4}{0-3} = 2, & b &= \frac{0 \cdot 4 - 3 \cdot (-2)}{0 - 3} = -2.
		    \end{align*}
		    
		    \item
		    \begin{align*}
		        a &= \frac{-1 - 2}{-3 - (-1)} = \frac{3}{2}, & b &= \frac{-3 \cdot 2 - (-1) \cdot (-1)}{-3 - (-1)} = \dfrac72.
		    \end{align*}
		
		    \item
		    \begin{align*}
		        a &= \frac{7 - 7}{2 - 5} = 0, & b &= \frac{2 \cdot 7 - 5 \cdot 7}{2 - 5} = 7.
		    \end{align*}
		
		    \item
		    \begin{align*}
		        a &= \frac{6-(-3)}{-4 - 2} = -\frac{3}{2}, & b &= \frac{-4 \cdot (-3) - 2 \cdot 6}{-4 - 2} = 0.
		    \end{align*}
		\item 
			D'après le cours,
			\begin{align*}
				a = \dfrac{y_A - y_B}{x_A - x_B}, && b = \dfrac{x_A y_B - x_B y_A}{x_A - x_B}.
			\end{align*}

	\end{enumerate}


}

\exe{[Intersection]
	Pour chacune des paires de fonctions affines $f, g$, calculer l'intersection des droites $\C_f \cap \C_g$.
	
	\begin{multicols}{2}
	\begin{enumerate}
		\item $f(x) = 3-2x, g(x) = -x+1$ $(x\in\R)$.
		\item $f(x) = -7x -2, g(x) = 4-2x$ $(x\in\R)$.
		\item $f(x) = 3+7x, g(x) = 7x+3$ $(x\in\R)$.
		\item $f(x) = x, g(x) = 3$ $(x\in\R)$.
		\item $f(x) = -123, g(x) = 2x+1$ $(x\in\R)$.
		\item $f(x) = 284, g(x) = -\dfrac13$ $(x\in\R)$.
	\end{enumerate}
	\end{multicols}
}{

	\begin{enumerate}
		\item 
			On cherche un point $P(x_P, y_P) \in \R^2$ vérifiant les deux équations
				\begin{align*}
					y_P = f(x_P) && \text{ et } && y_P = g(x_P)
				\end{align*}
			Comme bien sûr $y_P = y_P$, on peut résoudre l'équation
				\[ f(x_P) = g(x_P) \]
			pour trouver $x_P$.
			Ici, on pose calmement
				\begin{align*}
					f(x_P) &= g(x_P) \\
					3 - 2x_P &= -x_P + 1 \\
					x_P &= 2,
				\end{align*}
			ce qui nous fournit $x_P = 2$.
			Reste plus qu'à utiliser que $y_P = f(x_P) = g(x_P)$ pour calculer $y_P$.
			
			D'une part, 
				\[ f(x_P) = f(2) = 3-2\cdot2 = -1. \]
			D'autre part, on aurait pû calculer
				\[ g(x_P) = g(2) = -2 + 1 = -1. \]
			Rien de surprenant à ce qu'on trouve la même valeur $y_P = -1$, car $x_P$ a été trouvé vérifiant $f(x_P) = g(x_P)$.
			
			En conclusion, $\C_f \cap \C_g = \{ (2;-1) \}$.
			
		\item 
			Similairement,
			\begin{align*}
				f(x_P) &= g(x_P) \\
				-7x_P -2 & = 4-2x_P \\
				5x_P &= -6 \\
				x_P &= -\dfrac65,
			\end{align*}
			et $y_P = g\left(-\dfrac65\right) = 4-2\left(-\dfrac65\right) = 4+\dfrac{12}5 = \dfrac{32}5$.
			
			Donc $\C_f \cap \C_g = \left\{ \left(-\dfrac65;\dfrac{32}5 \right) \right\}$.
		
		
		\item 
			On remarque que $f=g$ et donc que les droites associées aux fonctions sont confondues.			
			Donc $\C_f \cap \C_g = \C_f = \C_g$.
			
		\item
			On pose
				\begin{align*}
				f(x_P) &= g(x_P) \\
				x_P & = 3,
			\end{align*}
			ce qui donne ensuite $y_P = f(3) = 3 = g(3)$.
			
			Donc $\C_f \cap \C_g = \left\{ \left( 3; 3 \right) \right\}$.
			
			
		\item
			On pose
				\begin{align*}
				f(x_P) &= g(x_P) \\
				-123 & = 2x_P+1 \\
				x_P = -62,
			\end{align*}
			ce qui donne ensuite $y_P = f(-62) = -123$.
			
			Donc $\C_f \cap \C_g = \left\{ \left( -62; -123 \right) \right\}$.
			
			
		\item
			On pose
				\begin{align*}
				f(x_P) &= g(x_P) \\
				284 & = -\dfrac13,
			\end{align*}
			équation évidemment fausse.
			Il n'existe donc aucun $P \in \C_f \cap \C_g$. 
			En d'autres termes, $\C_f \cap \C_g = \emptyset$, l'ensemble vide.
			
			En fait, les deux fonctions sont constantes et donc en particulier parallèles (les coefficients directeurs sont tous les deux égaux à $0$).
			Or comme les ordonnées à l'origine des fonctions ne sont pas les mêmes, le théorème du cours nous donne directement le résultat.
	\end{enumerate}

}

\exe{[Parallélisme]
	Pour chacun des couples de point $P$ et fonction affine $f$, trouver la fonction affine $g$ telle que $\C_f$ et $\C_g$ soient parallèles et que $P \in \C_g$.
	
	
	\begin{multicols}{2}
	\begin{enumerate}
		\item $P=(-1;-2)$ et $f(x) = \dfrac13 - 8x$ $(x\in\R)$.
		\item $P=(-3;-4)$ et $f(x) = \dfrac13 x - 7^{8^9}$ $(x\in\R)$.
		\item $P=(-5;-6)$ et $f(x) = -\dfrac{8}{11}x$ $(x\in\R)$.
		\item $P=(7;8)$ et $f(x) = -7^{6^{5}}$ $(x\in\R)$.
	\end{enumerate}
	\end{multicols}
}{
	On pose $a,b\in\R$ les paramètres de la fonction affine $g(x)=ax+b$, $x\in\R$.
	\begin{enumerate}
		\item 
			L'information de parallélisme donne immédiatement $a=-8$.
			D'où $g(x) = -8x+b$ pour tout $x\in\R$.
			
			Reste à trouver $b$ avec l'information d'appartenance.
			Celle-ci est équivalente à la contrainte
				\begin{align*}
					-2 &= g(-1) \\
					-2 &= -8\cdot(-1) + b \\
					b &= -10
				\end{align*}
			D'où $g(x) = -8x -10$ ($x\in\R$) est la fonction affine recherchée.
		
		
		\item 
			L'information de parallélisme donne immédiatement $a=\dfrac13$.
			D'où $g(x) = \dfrac13x+b$ pour tout $x\in\R$.
			
			Reste à trouver $b$ avec l'information d'appartenance.
			Celle-ci est équivalente à la contrainte
				\begin{align*}
					-4 &= g(-3) \\
					-4 &= -1 + b \\
					b &= -3
				\end{align*}
			D'où $g(x) = \dfrac13x-3$ ($x\in\R$) est la fonction affine recherchée.
			
		\item 
			L'information de parallélisme donne immédiatement $a=-\dfrac8{11}$.
			D'où $g(x)=-\dfrac8{11}x + b$ pour tout $x\in\R$.
			
			Reste à trouver $b$ avec l'information d'appartenance.
			Celle-ci est équivalente à la contrainte
				\begin{align*}
					-6 &= g(-5) \\
					-6 &=  \dfrac{40}{11} + b \\
					b &= -\dfrac{106}{11}
				\end{align*}
			D'où $g(x) = -\dfrac8{11}x -\dfrac{106}{11}$ ($x\in\R$) est la fonction affine recherchée.
			
		\item 
			L'information de parallélisme donne immédiatement $a=0$.
			D'où $g(x) = b$ pour tout $x\in\R$.
			
			Reste à trouver $b$ avec l'information d'appartenance.
			Celle-ci est équivalente à la contrainte
				\begin{align*}
					8 &= g(7) \\
					b &= 8
				\end{align*}
			D'où $g(x) = 8$ ($x\in\R$) est la fonction affine recherchée.
	\end{enumerate}

}


\exe{[Lecture graphique]

	Déterminer les paramètres des fonction affines $f, g, h$ dont les courbes sont représentées ci-dessous sur l'intervalle $[-9;9]$.

	\begin{center}
		\begin{tikzpicture}[>=stealth, scale=1.5]
		\begin{axis}[xmin = -9, xmax=9, ymin=-10, ymax=8, axis x line=middle, axis y line=middle, axis line style=<->, xlabel={}, ylabel={}, xtick = {-10, -8, ..., 8, 10}, ytick = {-10, -8, ..., 8, 10}, grid=both]
		
			\addplot[myr, thick, domain =-9:9, samples=2] {-3*x/8 + 3}  node[pos=.2, above=3pt] {$(\mathcal{C}_f)$};
			\addplot[myg, thick, domain =-9:9, samples=2] {3*x/4-4}  node[pos=.2, above=5pt] {$(\mathcal{C}_g)$};
			\addplot[black, thick, domain =-9:9, samples=2] {2*x/7 + 2}  node[pos=.9, above=1pt] {$(\mathcal{C}_h)$};
		
			
		\end{axis}
	\end{tikzpicture}
	\end{center}
}{
	Les fonctions sont données, pour tout $x\in\R$, par
	\begin{align*}
			f(x) = -\dfrac13 x + 1, &&
			g(x) = \dfrac32 x - 2, &&
			h(x) = -\dfrac16 x + 5.
	\end{align*}
}


\end{document}


%\exe{[Lecture graphique]
%
%	Déterminer les paramètres (coefficient directeur $a$ et ordonnée à l'origine $b$) des fonction affines $f,g,h$ dont les courbes sont représentées ci-dessous.
%	
%		\begin{center}
%		\newcommand{\scale}{1.095}
%		\begin{tikzpicture}[>=stealth, scale=\scale]
%		\begin{axis}[xmin = -10, xmax=10, ymin=-10, ymax=10, axis x line=middle, axis y line=middle, axis line style=<->, xlabel={}, ylabel={}, xtick = {-10, -8, ..., 8, 10}, ytick = {-10, -8, ..., 8, 10}, grid=both]
%		
%			\addplot[myr, thick, domain =-9:9, samples=2] {-x}  node[above=6pt] {$(\mathcal{C}_f)$};
%			\addplot[myr, thick, dotted, domain =-10:-9, samples=2] {-x} ;
%			\addplot[myr, thick, dotted, domain =9:10, samples=2] {-x};
%		
%		
%			\addplot[myg, thick, domain =-9:9, samples=2] {x/2+1}  node[below=6pt] {$(\mathcal{C}_g)$};
%			\addplot[myg, thick, dotted, domain =-10:-9, samples=2] {x/2+1} ;
%			\addplot[myg, thick, dotted, domain =9:10, samples=2] {x/2+1};
%		
%		
%			\addplot[black, thick, domain =-9:9, samples=2] {7}  node[pos=.7, above] {$(\mathcal{C}_h)$};
%			\addplot[black, thick, dotted, domain =-10:-9, samples=2] {7} ;
%			\addplot[black, thick, dotted, domain =9:10, samples=2] {7};
%		
%			
%		\end{axis}
%		\end{tikzpicture}
%		\begin{tikzpicture}[>=stealth, scale=\scale]
%		\begin{axis}[xmin = -10, xmax=10, ymin=-10, ymax=10, axis x line=middle, axis y line=middle, axis line style=<->, xlabel={}, ylabel={}, xtick = {-10, -8, ..., 8, 10}, ytick = {-10, -8, ..., 8, 10}, grid=both]
%		
%			\addplot[myr, thick, domain =-9:9, samples=2] {-x/3 + 1}  node[pos=.9, below=5pt] {$(\mathcal{C}_F)$};
%			\addplot[myr, thick, dotted, domain =-10:-9, samples=2] {-x/3 + 1} ;
%			\addplot[myr, thick, dotted, domain =9:10, samples=2] {-x/3 + 1};
%		
%		
%			\addplot[myg, thick, domain =-5:7, samples=2] {3*x/2-2}  node[pos=.2, left] {$(\mathcal{C}_G)$};
%			\addplot[myg, thick, dotted, domain =-6:-5, samples=2] {3*x/2-2} ;
%			\addplot[myg, thick, dotted, domain =7:8, samples=2] {3*x/2-2};
%		
%		
%			\addplot[black, thick, domain =-9:9, samples=2] {-x/6 + 5}  node[pos=.2, above] {$(\mathcal{C}_H)$};
%			\addplot[black, thick, dotted, domain =-10:-9, samples=2] {-x/6 + 5} ;
%			\addplot[black, thick, dotted, domain =9:10, samples=2] {-x/6 + 5};
%		
%			
%		\end{axis}
%	\end{tikzpicture}
%	\end{center}
%}{
%	Pour la fonction $f(x) = ax+b (x\in\R)$, on peut par exemple choisir deux points lui appartenant et interpoler linéairement à partir de ces deux points.
%	Par unicité de la droite passant par deux points, l'interpolation trouvée doit nécessairement être $f$.
%	Prenons $A(-4;4)$ et $B(2;-2)$ et calculons les paramètres de la droite $(AB) = \C_f$.
%	
%	\begin{align*}
%		a = \dfrac{4 - (-2)}{-4 - 2} = -1, && b = \dfrac{(-4)\cdot(-2) - 4\cdot2}{-4-2} = 0.
%	\end{align*}
%	
%	\hrule\vspace{10pt}
%	
%	Pour la fonction $g(x) = ax+b (x\in\R)$, on peut faire idem en choisissant de bons points pour faciliter les calculs.
%	Par exemple, $A(0;1)$ et $B(-2;0)$ sont pratiques car il y a beaucoup de zéros qui annulent les expressions.
%	En fait, le point $A$ donne directement l'ordonnée à l'origine $b=1$, mais on peut calculer à l'aide du théorème du cours nonobstant.
%	\begin{align*}
%		a = \dfrac{1-0}{0-(-2)} = \dfrac12, && b = \dfrac{0 \cdot0 - 1 \cdot (-2)}{0-(-2)} = 1.
%	\end{align*}
%	
%	\hrule\vspace{10pt}
%	
%	Pour la fonction $h(x) = ax+b (x\in\R)$, on procède par parallélisme et appartenance.
%	La fonction est constante, donc $a=0$.
%	N'importe quel point lui appartenant a pour ordonnée $b=7$, ce qui conclut.
%
%	\hrule\vspace{10pt}
%
%	Les fonctions sont données, pour tout $x\in\R$, par
%	\begin{align*}
%			F(x) = -\dfrac13 x + 1, &&
%			G(x) = \dfrac32 x - 2, &&
%			G(x) = -\dfrac16 x + 5.
%	\end{align*}
%}
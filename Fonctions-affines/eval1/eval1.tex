				% ENABLE or DISABLE font change
				% use XeLaTeX if true
\newif\ifdys
				\dystrue
				\dysfalse

\newif\ifsolutions
				\solutionstrue
				\solutionsfalse

% DYSLEXIA SWITCH
\newif\ifdys
		
				% ENABLE or DISABLE font change
				% use XeLaTeX if true
				\dystrue
				\dysfalse


\ifdys

\documentclass[a4paper, 14pt]{extarticle}
\usepackage{amsmath,amsfonts,amsthm,amssymb,mathtools}

\tracinglostchars=3 % Report an error if a font does not have a symbol.
\usepackage{fontspec}
\usepackage{unicode-math}
\defaultfontfeatures{ Ligatures=TeX,
                      Scale=MatchUppercase }

\setmainfont{OpenDyslexic}[Scale=1.0]
\setmathfont{Fira Math} % Or maybe try KPMath-Sans?
\setmathfont{OpenDyslexic Italic}[range=it/{Latin,latin}]
\setmathfont{OpenDyslexic}[range=up/{Latin,latin,num}]

\else

\documentclass[a4paper, 12pt]{extarticle}

\usepackage[utf8x]{inputenc}
%fonts
\usepackage{amsmath,amsfonts,amsthm,amssymb,mathtools}
% comment below to default to computer modern
\usepackage{libertinus,libertinust1math}

\fi


\usepackage[french]{babel}
\usepackage[
a4paper,
margin=2cm,
nomarginpar,% We don't want any margin paragraphs
]{geometry}
\usepackage{icomma}

\usepackage{fancyhdr}
\usepackage{array}
\usepackage{hyperref}

\usepackage{multicol, enumerate}
\newcolumntype{P}[1]{>{\centering\arraybackslash}p{#1}}


\usepackage{stackengine}
\newcommand\xrowht[2][0]{\addstackgap[.5\dimexpr#2\relax]{\vphantom{#1}}}

% theorems

\theoremstyle{plain}
\newtheorem{theorem}{Th\'eor\`eme}
\newtheorem*{sol}{Solution}
\theoremstyle{definition}
\newtheorem{ex}{Exercice}
\newtheorem*{rpl}{Rappel}
\newtheorem{enigme}{Énigme}

% corps
\usepackage{calrsfs}
\newcommand{\C}{\mathcal{C}}
\newcommand{\R}{\mathbb{R}}
\newcommand{\Rnn}{\mathbb{R}^{2n}}
\newcommand{\Z}{\mathbb{Z}}
\newcommand{\N}{\mathbb{N}}
\newcommand{\Q}{\mathbb{Q}}

% variance
\newcommand{\Var}[1]{\text{Var}(#1)}

% domain
\newcommand{\D}{\mathcal{D}}


% date
\usepackage{advdate}
\AdvanceDate[0]


% plots
\usepackage{pgfplots}

% table line break
\usepackage{makecell}
%tablestuff
\def\arraystretch{2}
\setlength\tabcolsep{15pt}

%subfigures
\usepackage{subcaption}

\definecolor{myg}{RGB}{56, 140, 70}
\definecolor{myb}{RGB}{45, 111, 177}
\definecolor{myr}{RGB}{199, 68, 64}

% fake sections with no title to move around the merged pdf
\newcommand{\fakesection}[1]{%
  \par\refstepcounter{section}% Increase section counter
  \sectionmark{#1}% Add section mark (header)
  \addcontentsline{toc}{section}{\protect\numberline{\thesection}#1}% Add section to ToC
  % Add more content here, if needed.
}


% SOLUTION SWITCH
\newif\ifsolutions
				\solutionstrue
				%\solutionsfalse

\ifsolutions
	\newcommand{\exe}[2]{
		\begin{ex} #1  \end{ex}
		\begin{sol} #2 \end{sol}
	}
\else
	\newcommand{\exe}[2]{
		\begin{ex} #1  \end{ex}
	}
	
\fi


% tableaux var, signe
\usepackage{tkz-tab}


%pinfty minfty
\newcommand{\pinfty}{{+}\infty}
\newcommand{\minfty}{{-}\infty}

\begin{document}


\AdvanceDate[0]

\begin{document}
\pagestyle{fancy}
\fancyhead[L]{Seconde 13}
\fancyhead[C]{\textbf{Évaluation blanche -- Fonctions affines \ifsolutions -- Solutions  \fi}}
\fancyhead[R]{\today}

%\begin{theorem}[label=thm:1]{}{}
%	Soit $f(x) = ax + b$ une fonction affine sur $\R$.
%	Considérons $A(x_A ; y_A)$ et $B(x_B ; y_B)$ deux points distincts appartenant à $\C_f$.
%	
%	Alors le coefficient directeur $a$ est égal à
%	\vspace{5pt}
%		\[ a = .......................................... \]
%	\,
%\end{theorem}

\begin{theorem}[label=thm:1]{Propriété fondamentale}{}

		Soit $f : \D \rightarrow \R$ une fonction réelle sur un domaine $\D$ et $(x;y)$ un point du plan avec $x\in\D$.
		Alors
			\begin{align*}
				(x ; y) \in \C_f && \iff && \ifsolutions \color{red} y=f(x) \else \underline{\qquad\qquad\qquad} \fi.
			\end{align*}
\end{theorem}

\exe{[2pts]
	Compléter le théorème \ref{thm:1} vu en classe.
}{}

\exe{[4pts]
	Pour chaque fonction affine  sur $\R$ suivante, déterminer son coefficient directeur $a$ et son ordonnée à l'origine $b$.
	\begin{multicols}{2}
	\begin{enumerate}
		\item $f(x) = 7$
		\item $f(x) = 2- \dfrac23 x $
		\item $f(x) = x$
		\item $f(x) = 0$
	\end{enumerate}
	\end{multicols}
}{
	On utilise par exemple que $f(0) = b$ et que $f(1)-f(0) = a$.

	\begin{enumerate}
		\item 
			\begin{align*}
				a = 0 && b=7
			\end{align*}
		\item 
			\begin{align*}
				a = -\dfrac23 && b=2
			\end{align*}
		\item 
			\begin{align*}
				a = 1 && b=0
			\end{align*}
		\item 
			\begin{align*}
				a = 0 && b=0
			\end{align*}
	\end{enumerate}

}

%\exe{
%	Grapher fonction affine à partir de deux points pour estimer $b$.
%}{}

%\exe{
%	Interpolation affine.
%}{}

\exe{[5pts]
	Déterminer les paramètres (coefficient directeur $a$ et ordonnée à l'origine $b$) des fonction affines $f,g,h$ dont les courbes sont représentées ci-dessous.
	Les droites $\C_f$ et $\C_h$ sont parallèles.

	\begin{center}
		\begin{tikzpicture}[>=stealth, scale=1.5]
		\begin{axis}[xmin = -10, xmax=10, ymin=-10, ymax=10, axis x line=middle, axis y line=middle, axis line style=<->, xlabel={}, ylabel={}, xtick = {-10, -8, ..., 8, 10}, ytick = {-10, -8, ..., 8, 10}, grid=both]
		
			\addplot[red, thick, domain =-9:9, samples=2] {-x/2 + 2}  node[below=6pt] {$\mathcal{C}_f$};
			\addplot[red, thick, dotted, domain =-10:-9, samples=2] {-x/2 + 2} ;
			\addplot[red, thick, dotted, domain =9:10, samples=2] {-x/2 + 2};
		
		
			\addplot[olive, thick, domain =-5:7, samples=2] {3*x/2-2}  node[pos=.2, left=5pt] {$\mathcal{C}_g$};
			\addplot[olive, thick, dotted, domain =-6:-5, samples=2] {3*x/2-2} ;
			\addplot[olive, thick, dotted, domain =7:8, samples=2] {3*x/2-2};
		
		
			\addplot[black, thick, domain =-3:9, samples=2] {-x/2 + 8}  node[above=1pt, pos=.35] {$\mathcal{C}_h$};
			\addplot[black, thick, dotted, domain =-4:-3, samples=2] {-x/2 + 8} ;
			\addplot[black, thick, dotted, domain =9:10, samples=2] {-x/2 + 8};
		
			
		\end{axis}
	\end{tikzpicture}
	\end{center}
	
	
}{

	\begin{align*}
		f(x) &= -\dfrac12 x + 2 \\
		g(x) &= \dfrac32 x - 2 \\
		h(x) &= -\dfrac12 x + 8
	\end{align*}

}

\exe{[6pts]
	Pour chacune des paires de fonctions affines $f, g$ sur $\R$, calculer $\C_f \cap \C_g$.
	
	\begin{multicols}{2}
	\begin{enumerate}
		\item $f(x) = 7x -2, g(x) = -x+3$.
		\item $f(x) = -x + 3, g(x) = 7x -2$.
		\item $f(x) = 3, g(x) = -x$.
		\item $f(x) = -2x+1, g(x) = 1-x$.
		\item $f(x) = 2, g(x) = 4$.
		\item $f(x) = -x+1, g(x) = 1-x$.
	\end{enumerate}
	\end{multicols}
}{
	
	\begin{enumerate}
		\item 
			\[ \C_f \cap \C_g = \left\{ \left( \dfrac58 ;  \dfrac{19}8 \right) \right\}. \]
		\item 
			\[ \C_f \cap \C_g = \left\{ \left( \dfrac58 ;  \dfrac{19}8 \right) \right\}. \]
		\item 
			\[ \C_f \cap \C_g = \left\{ \left( -3 ; 3 \right) \right\}. \]
		\item 
			\[ \C_f \cap \C_g = \left\{ \left( 0 ; 1 \right) \right\}. \]
		\item 
			\[ \C_f \cap \C_g = \emptyset. \]
		\item 
			\[ \C_f \cap \C_g = \C_f = \C_g. \]
	\end{enumerate}


}

\ifsolutions
\else
\newpage
\fi

\exe{[3pts]
	Soit $f$ la fonction affine sur $[{-3};3]$ donnée par
		\[ f(x) = -\frac23 x + 3 \qquad \text{ pour tout } x\in[-3 ; 3]. \]
	\begin{enumerate}
		\item
		Déterminer l'expression algébrique de la fonction affine $g$ telle que $\C_g$ soit parallèle à $\C_f$ et passe par $(0;-1)$.
		\item
		Déterminer l'expression algébrique de la fonction affine $h$ telle que $\C_h$ soit parallèle à $\C_g$ et passe par $(-3;1)$.
	\end{enumerate}
}{
	\begin{enumerate}
		\item
			\[ g(x) = -\dfrac23 x - 1. \]
		\item
			\[ h(x) = -\dfrac23 x - 1. \]
	\end{enumerate}


}

\subsection*{Bonus (1pt)}

\exe{
	Trouver l'ensemble des solutions $x\in\R$ de l'équation
		\[ |2x+1| = |1 - x|. \]
}{
	Comme la valeur absolue supprime le signe, on a $4$ possibilités de signes qui donnent deux équations distinctes à résoudre.
		\begin{align*}
			2x+1 = 1 - x && \text{ ou } && -(2x+1) = 1 -x && \text{ ou } && 
			2x+1 = -(1-x) && \text{ ou } && -(2x+1) = -(1-x)
		\end{align*}
	On en déduit que $\{0 ; -2\}$ est l'ensemble des solutions.
}

%\exe{
%	Démontrer le théorème \ref{thm:1} après avoir justifié que $x_A \neq x_B$.
%}{}

%\exe{
%	Deux inconnues $a, b \in \R$ vérifient les deux équations suivantes. Déterminer $a$ et $b$.
%		\[ \begin{cases} 2a + 3b = 3, \\ 4a - b = 2. \end{cases} \]
%}{}

\end{document}
				% ENABLE or DISABLE font change
				% use XeLaTeX if true
\newif\ifdys
				\dystrue
				\dysfalse

\newif\ifsolutions
				\solutionstrue
				\solutionsfalse

%!TEX encoding = UTF8
%!TEX root =notes.tex


%%%%%%%%%%%%%%%%%%%%%%%%%%%%%%%%%
% PACKAGE IMPORTS
%%%%%%%%%%%%%%%%%%%%%%%%%%%%%%%%%


\usepackage[french]{babel}

\usepackage[tmargin=2cm,rmargin=1in,lmargin=1in,margin=0.85in,bmargin=2cm,footskip=.2in]{geometry}
\usepackage{amsmath,amsfonts,amsthm,amssymb,mathtools}
\usepackage[varbb]{newpxmath}
\usepackage{xfrac}
\usepackage[makeroom]{cancel}
\usepackage{mathtools}
\usepackage{bookmark}
\usepackage{enumitem}
\usepackage{hyperref,theoremref}
\hypersetup{
	pdftitle={Assignment},
	colorlinks=true, linkcolor=doc!90,
	bookmarksnumbered=true,
	bookmarksopen=true
}
\usepackage[most,many,breakable]{tcolorbox}
\usepackage{xcolor}
\usepackage{varwidth}
\usepackage{varwidth}
\usepackage{etoolbox}
%\usepackage{authblk}
\usepackage{nameref}
\usepackage{multicol,array}
\usepackage{tikz-cd}
\usepackage[ruled,vlined,linesnumbered]{algorithm2e}
\usepackage{comment} % enables the use of multi-line comments (\ifx \fi) 
\usepackage{import}
\usepackage{xifthen}
\usepackage{pdfpages}
\usepackage{transparent}


\newcommand\mycommfont[1]{\footnotesize\ttfamily\textcolor{blue}{#1}}
\SetCommentSty{mycommfont}
\newcommand{\incfig}[1]{%
    \def\svgwidth{\columnwidth}
    \import{./figures/}{#1.pdf_tex}
}

\usepackage{tikzsymbols}
%\renewcommand\qedsymbol{$\Laughey$}


%\usepackage{import}
%\usepackage{xifthen}
%\usepackage{pdfpages}
%\usepackage{transparent}


%%%%%%%%%%%%%%%%%%%%%%%%%%%%%%
% SELF MADE COLORS
%%%%%%%%%%%%%%%%%%%%%%%%%%%%%%



\definecolor{myg}{RGB}{56, 140, 70}
\definecolor{myb}{RGB}{45, 111, 177}
\definecolor{myr}{RGB}{199, 68, 64}
\definecolor{mytheorembg}{HTML}{F2F2F9}
\definecolor{mytheoremfr}{HTML}{00007B}
\definecolor{mylenmabg}{HTML}{FFFAF8}
\definecolor{mylenmafr}{HTML}{983b0f}
\definecolor{mypropbg}{HTML}{f2fbfc}
\definecolor{mypropfr}{HTML}{191971}
\definecolor{myexamplebg}{HTML}{F2FBF8}
\definecolor{myexamplefr}{HTML}{88D6D1}
\definecolor{myexampleti}{HTML}{2A7F7F}
\definecolor{mydefinitbg}{HTML}{E5E5FF}
\definecolor{mydefinitfr}{HTML}{3F3FA3}
\definecolor{notesgreen}{RGB}{0,162,0}
\definecolor{myp}{RGB}{197, 92, 212}
\definecolor{mygr}{HTML}{2C3338}
\definecolor{myred}{RGB}{127,0,0}
\definecolor{myyellow}{RGB}{169,121,69}
\definecolor{myexercisebg}{HTML}{F2FBF8}
\definecolor{myexercisefg}{HTML}{88D6D1}


%%%%%%%%%%%%%%%%%%%%%%%%%%%%
% TCOLORBOX SETUPS
%%%%%%%%%%%%%%%%%%%%%%%%%%%%

\setlength{\parindent}{1cm}
%================================
% THEOREM BOX
%================================

\tcbuselibrary{theorems,skins,hooks}
\newtcbtheorem[number within=chapter]{Theorem}{Théorème}
{%
	enhanced,
	breakable,
	colback = mytheorembg,
	frame hidden,
	boxrule = 0sp,
	borderline west = {2pt}{0pt}{mytheoremfr},
	sharp corners,
	detach title,
	before upper = \tcbtitle\par\smallskip,
	coltitle = mytheoremfr,
	fonttitle = \bfseries\sffamily,
	description font = \mdseries,
	separator sign none,
	segmentation style={solid, mytheoremfr},
}
{th}


\tcbuselibrary{theorems,skins,hooks}
\newtcolorbox{Theoremcon}
{%
	enhanced
	,breakable
	,colback = mytheorembg
	,frame hidden
	,boxrule = 0sp
	,borderline west = {2pt}{0pt}{mytheoremfr}
	,sharp corners
	,description font = \mdseries
	,separator sign none
}

%================================
% Corollery
%================================
\tcbuselibrary{theorems,skins,hooks}
\newtcbtheorem[use counter=tcb@cnt@Theorem]{Corollary}{Corollaire}
{%
	enhanced
	,breakable
	,colback = myp!10
	,frame hidden
	,boxrule = 0sp
	,borderline west = {2pt}{0pt}{myp!85!black}
	,sharp corners
	,detach title
	,before upper = \tcbtitle\par\smallskip
	,coltitle = myp!85!black
	,fonttitle = \bfseries\sffamily
	,description font = \mdseries
	,separator sign none
	,segmentation style={solid, myp!85!black}
}
{th}

%================================
% LENMA
%================================

\tcbuselibrary{theorems,skins,hooks}
\newtcbtheorem[use counter=tcb@cnt@Theorem]{Lemma}{Lemme}
{%
	enhanced,
	breakable,
	colback = mylenmabg,
	frame hidden,
	boxrule = 0sp,
	borderline west = {2pt}{0pt}{mylenmafr},
	sharp corners,
	detach title,
	before upper = \tcbtitle\par\smallskip,
	coltitle = mylenmafr,
	fonttitle = \bfseries\sffamily,
	description font = \mdseries,
	separator sign none,
	segmentation style={solid, mylenmafr},
}
{th}


%================================
% PROPOSITION
%================================

\tcbuselibrary{theorems,skins,hooks}
\newtcbtheorem[use counter=tcb@cnt@Theorem]{Prop}{Proposition}
{%
	enhanced,
	breakable,
	colback = mypropbg,
	frame hidden,
	boxrule = 0sp,
	borderline west = {2pt}{0pt}{mypropfr},
	sharp corners,
	detach title,
	before upper = \tcbtitle\par\smallskip,
	coltitle = mypropfr,
	fonttitle = \bfseries\sffamily,
	description font = \mdseries,
	separator sign none,
	segmentation style={solid, mypropfr},
}
{th}


%================================
% CLAIM
%================================

\tcbuselibrary{theorems,skins,hooks}
\newtcbtheorem[use counter=tcb@cnt@Theorem]{claim}{Claim}
{%
	enhanced
	,breakable
	,colback = myg!10
	,frame hidden
	,boxrule = 0sp
	,borderline west = {2pt}{0pt}{myg}
	,sharp corners
	,detach title
	,before upper = \tcbtitle\par\smallskip
	,coltitle = myg!85!black
	,fonttitle = \bfseries\sffamily
	,description font = \mdseries
	,separator sign none
	,segmentation style={solid, myg!85!black}
}
{th}



%================================
% Exercise
%================================

\tcbuselibrary{theorems,skins,hooks}
\newtcbtheorem[use counter=tcb@cnt@Theorem]{Exercise}{Exercice}
{%
	enhanced,
	breakable,
	colback = myexercisebg,
	frame hidden,
	boxrule = 0sp,
	borderline west = {2pt}{0pt}{myexercisefg},
	sharp corners,
	detach title,
	before upper = \tcbtitle\par\smallskip,
	coltitle = myexercisefg,
	fonttitle = \bfseries\sffamily,
	description font = \mdseries,
	separator sign none,
	segmentation style={solid, myexercisefg},
}
{th}

%================================
% EXAMPLE BOX
%================================

\newtcbtheorem[use counter=tcb@cnt@Theorem]{Example}{Exemple}
{%
	colback = myexamplebg
	,breakable
	,colframe = myexamplefr
	,coltitle = myexampleti
	,boxrule = 1pt
	,sharp corners
	,detach title
	,before upper=\tcbtitle\par\smallskip
	,fonttitle = \bfseries
	,description font = \mdseries
	,separator sign none
	,description delimiters parenthesis
}
{ex}

%================================
% DEFINITION BOX
%================================

\newtcbtheorem[use counter=tcb@cnt@Theorem]{Definition}{Définition}{enhanced,
	before skip=2mm,after skip=2mm, colback=red!5,colframe=red!80!black,boxrule=0.5mm,
	attach boxed title to top left={xshift=1cm,yshift*=1mm-\tcboxedtitleheight}, varwidth boxed title*=-3cm,
	boxed title style={frame code={
					\path[fill=tcbcolback]
					([yshift=-1mm,xshift=-1mm]frame.north west)
					arc[start angle=0,end angle=180,radius=1mm]
					([yshift=-1mm,xshift=1mm]frame.north east)
					arc[start angle=180,end angle=0,radius=1mm];
					\path[left color=tcbcolback!60!black,right color=tcbcolback!60!black,
						middle color=tcbcolback!80!black]
					([xshift=-2mm]frame.north west) -- ([xshift=2mm]frame.north east)
					[rounded corners=1mm]-- ([xshift=1mm,yshift=-1mm]frame.north east)
					-- (frame.south east) -- (frame.south west)
					-- ([xshift=-1mm,yshift=-1mm]frame.north west)
					[sharp corners]-- cycle;
				},interior engine=empty,
		},
	fonttitle=\bfseries,
	title={#2},#1}{def}

%================================
% Solution BOX
%================================

\makeatletter
\newtcbtheorem[use counter=tcb@cnt@Theorem]{question}{Question}{enhanced,
	breakable,
	colback=white,
	colframe=myb!80!black,
	attach boxed title to top left={yshift*=-\tcboxedtitleheight},
	fonttitle=\bfseries,
	title={#2},
	boxed title size=title,
	boxed title style={%
			sharp corners,
			rounded corners=northwest,
			colback=tcbcolframe,
			boxrule=0pt,
		},
	underlay boxed title={%
			\path[fill=tcbcolframe] (title.south west)--(title.south east)
			to[out=0, in=180] ([xshift=5mm]title.east)--
			(title.center-|frame.east)
			[rounded corners=\kvtcb@arc] |-
			(frame.north) -| cycle;
		},
	#1
}{def}
\makeatother

%================================
% SOLUTION BOX
%================================

\makeatletter
\newtcolorbox{solution}{enhanced,
	breakable,
	colback=white,
	colframe=myg!80!black,
	attach boxed title to top left={yshift*=-\tcboxedtitleheight},
	title=Solution,
	boxed title size=title,
	boxed title style={%
			sharp corners,
			rounded corners=northwest,
			colback=tcbcolframe,
			boxrule=0pt,
		},
	underlay boxed title={%
			\path[fill=tcbcolframe] (title.south west)--(title.south east)
			to[out=0, in=180] ([xshift=5mm]title.east)--
			(title.center-|frame.east)
			[rounded corners=\kvtcb@arc] |-
			(frame.north) -| cycle;
		},
}
\makeatother

%================================
% Question BOX
%================================

\makeatletter
\newtcbtheorem[use counter=tcb@cnt@Theorem]{qstion}{Question}{enhanced,
	breakable,
	colback=white,
	colframe=mygr,
	attach boxed title to top left={yshift*=-\tcboxedtitleheight},
	fonttitle=\bfseries,
	title={#2},
	boxed title size=title,
	boxed title style={%
			sharp corners,
			rounded corners=northwest,
			colback=tcbcolframe,
			boxrule=0pt,
		},
	underlay boxed title={%
			\path[fill=tcbcolframe] (title.south west)--(title.south east)
			to[out=0, in=180] ([xshift=5mm]title.east)--
			(title.center-|frame.east)
			[rounded corners=\kvtcb@arc] |-
			(frame.north) -| cycle;
		},
	#1
}{def}
\makeatother

\newtcbtheorem[number within=chapter]{wconc}{Wrong Concept}{
	breakable,
	enhanced,
	colback=white,
	colframe=myr,
	arc=0pt,
	outer arc=0pt,
	fonttitle=\bfseries\sffamily\large,
	colbacktitle=myr,
	attach boxed title to top left={},
	boxed title style={
			enhanced,
			skin=enhancedfirst jigsaw,
			arc=3pt,
			bottom=0pt,
			interior style={fill=myr}
		},
	#1
}{def}



%================================
% NOTE BOX
%================================

\usetikzlibrary{arrows,calc,shadows.blur}
\tcbuselibrary{skins}
\newtcolorbox{note}[1][]{%
	enhanced jigsaw,
	colback=gray!20!white,%
	colframe=gray!80!black,
	size=small,
	boxrule=1pt,
	title=\colorbox{white!100}{\textbf{ Remarque }},
	halign title=flush center,
	coltitle=black,
	breakable,
	drop shadow=black!50!white,
	attach boxed title to top left={xshift=1cm,yshift=-\tcboxedtitleheight/2,yshifttext=-\tcboxedtitleheight/2},
	minipage boxed title=2.6cm,
	boxed title style={%
			colback=white,
			size=fbox,
			boxrule=1pt,
			boxsep=2pt,
			underlay={%
					\coordinate (dotA) at ($(interior.west) + (-0.5pt,0)$);
					\coordinate (dotB) at ($(interior.east) + (0.5pt,0)$);
					\begin{scope}
						\clip (interior.north west) rectangle ([xshift=3ex]interior.east);
						\filldraw [white, blur shadow={shadow opacity=60, shadow yshift=-.75ex}, rounded corners=2pt] (interior.north west) rectangle (interior.south east);
					\end{scope}
					\begin{scope}[gray!80!black]
						\fill (dotA) circle (2pt);
						\fill (dotB) circle (2pt);
					\end{scope}
				},
		},
	#1,
}

%================================
% STRATÉGIE BOX
%================================

\usetikzlibrary{arrows,calc,shadows.blur}
\tcbuselibrary{skins}
\newtcolorbox{strategy}[1][]{%
	enhanced jigsaw,
	colback=myb!20!white,%
	colframe=gray!80!black,
	size=small,
	boxrule=1pt,
	title=\colorbox{white!100}{\textbf{ Stratégie }},
	halign title=flush center,
	coltitle=black,
	breakable,
	drop shadow=black!50!white,
	attach boxed title to top left={xshift=1cm,yshift=-\tcboxedtitleheight/2,yshifttext=-\tcboxedtitleheight/2},
	minipage boxed title=2.5cm,
	boxed title style={%
			colback=white,
			size=fbox,
			boxrule=1pt,
			boxsep=2pt,
			underlay={%
					\coordinate (dotA) at ($(interior.west) + (-0.5pt,0)$);
					\coordinate (dotB) at ($(interior.east) + (0.5pt,0)$);
					\begin{scope}
						\clip (interior.north west) rectangle ([xshift=3ex]interior.east);
						\filldraw [white, blur shadow={shadow opacity=60, shadow yshift=-.75ex}, rounded corners=2pt] (interior.north west) rectangle (interior.south east);
					\end{scope}
					\begin{scope}[gray!80!black]
						\fill (dotA) circle (2pt);
						\fill (dotB) circle (2pt);
					\end{scope}
				},
		},
	#1,
}

%================================
% MÉTHODE BOX
%================================

\usetikzlibrary{arrows,calc,shadows.blur}
\tcbuselibrary{skins}
\newtcolorbox{methode}[1][]{%
	enhanced jigsaw,
	colback=white,%
	colframe=gray!80!black,
	size=small,
	boxrule=1pt,
	title=\textbf{Méthode},
	halign title=flush center,
	coltitle=black,
	breakable,
	drop shadow=black!50!white,
	attach boxed title to top left={xshift=1cm,yshift=-\tcboxedtitleheight/2,yshifttext=-\tcboxedtitleheight/2},
	minipage boxed title=2.5cm,
	boxed title style={%
			colback=white,
			size=fbox,
			boxrule=1pt,
			boxsep=2pt,
			underlay={%
					\coordinate (dotA) at ($(interior.west) + (-0.5pt,0)$);
					\coordinate (dotB) at ($(interior.east) + (0.5pt,0)$);
					\begin{scope}
						\clip (interior.north west) rectangle ([xshift=3ex]interior.east);
						\filldraw [white, blur shadow={shadow opacity=60, shadow yshift=-.75ex}, rounded corners=2pt] (interior.north west) rectangle (interior.south east);
					\end{scope}
					\begin{scope}[gray!80!black]
						\fill (dotA) circle (2pt);
						\fill (dotB) circle (2pt);
					\end{scope}
				},
		},
	#1,
}

%%%%%%%%%%%%%%%%%%%%%%%%%%%%%%%%%%%%%%%%%%%
% TABLE OF CONTENTS
%%%%%%%%%%%%%%%%%%%%%%%%%%%%%%%%%%%%%%%%%%%

\usepackage{tikz}

\definecolor{doc}{RGB}{0,60,110}
\usepackage{titletoc}
\contentsmargin{0cm}
\titlecontents{chapter}[3.7pc]
{\addvspace{30pt}%
	\begin{tikzpicture}[remember picture, overlay]%
		\draw[fill=doc!60,draw=doc!60] (-7,-.1) rectangle (-0.2,.6);%
		\pgftext[left,x=-3.5cm,y=0.2cm]{\color{white}\Large\sc\bfseries Chapitre\ \thecontentslabel};%
	\end{tikzpicture}\color{doc!60}\large\sc\bfseries}%
{}
{}
{\;\titlerule\;\large\sc\bfseries Page \thecontentspage
	\begin{tikzpicture}[remember picture, overlay]
		\draw[fill=doc!60,draw=doc!60] (2pt,0) rectangle (4,0.1pt);
	\end{tikzpicture}}%
\titlecontents{section}[3.7pc]
{\addvspace{2pt}}
{\contentslabel[\thecontentslabel]{2pc}}
{}
{\hfill\small \thecontentspage}
[]
\titlecontents*{subsection}[3.7pc]
{\addvspace{-1pt}\small}
{}
{}
{\ --- \small\thecontentspage}
[ \textbullet\ ][]

\makeatletter
\renewcommand{\tableofcontents}{%
	\chapter*{%
	  \vspace*{-20\p@}%
	  \begin{tikzpicture}[remember picture, overlay]%
		  \pgftext[right,x=15cm,y=0.2cm]{\color{doc!60}\Huge\sc\bfseries \contentsname};%
		  \draw[fill=doc!60,draw=doc!60] (13,-.75) rectangle (20,1);%
		  \clip (13,-.75) rectangle (20,1);
		  \pgftext[right,x=15cm,y=0.2cm]{\color{white}\Huge\sc\bfseries \contentsname};%
	  \end{tikzpicture}}%
	\@starttoc{toc}}
\makeatother


%%%%%%%%%%%%%%%%%%%%%%%%%%%%%%%%%%%%%%%%%%%
% MINTED FOR PYTHON ALGORITHMS
%%%%%%%%%%%%%%%%%%%%%%%%%%%%%%%%%%%%%%%%%%%

\usepackage{tcolorbox}
\tcbuselibrary{minted,breakable,xparse,skins}
\definecolor{bg}{gray}{0.95}
\DeclareTCBListing{mintedbox}{O{}m!O{}}{%
  breakable=true,
  listing engine=minted,
  listing only,
  minted language=#2,
  minted style=default,
  minted options={%
    linenos,
    gobble=0,
    breaklines=true,
    breakafter=,,
    fontsize=\small,
    numbersep=8pt,
    #1},
  boxsep=0pt,
  left skip=0pt,
  right skip=0pt,
  left=25pt,
  right=0pt,
  top=3pt,
  bottom=3pt,
  arc=5pt,
  leftrule=0pt,
  rightrule=0pt,
  bottomrule=2pt,
  toprule=2pt,
  colback=bg,
  colframe=orange!70,
  enhanced,
  overlay={%
    \begin{tcbclipinterior}
    \fill[orange!20!white] (frame.south west) rectangle ([xshift=20pt]frame.north west);
    \end{tcbclipinterior}},
  #3}
  
  
 % for braces
\usetikzlibrary{decorations.pathreplacing}


\AdvanceDate[0]

\begin{document}
\pagestyle{fancy}
\fancyhead[L]{Seconde 13}
\fancyhead[C]{\textbf{Fonctions affines 1\ifsolutions -- Solutions  \fi}}
\fancyhead[R]{\today}


\exe{
	Donner $4$ points appartenant à la courbe représentative de chaque fonction affine et l'esquisser dans un repère de domaine $\D = [-3 ; 3]$.
	\begin{multicols}{2}
	\begin{enumerate}
		\item $f(x) = 1 + 2x$
		\item $g(x) = 2 + 2x$
		\item $h(x) = - x$
		\item $F(x) = 1 - x$
		\item $G(x) = -\frac12$
		\item $H(x) = 2 - \frac13x$
	\end{enumerate}
	\end{multicols}
}{}

\exe{
	Pour chaque fonction affine sur $\R$ suivante, déterminer son coefficient directeur $a$ et son ordonnée à l'origine $b$.
	\begin{multicols}{2}
	\begin{enumerate}
		\item $f(x) = 2x + 1$
		\item $f(x) = 1 + 2x$
		\item $f(x) = - x$
		\item $f(x) = -42$
		\item $f(x) = \sqrt{3} \cdot x + 2$
		\item $f(x) = 2 - \sqrt{5} \cdot x$
		\item $f(x) = 1 - x$
		\item $f(x) = 0$
	\end{enumerate}
	\end{multicols}
}{}


\exe{\label{ex:3} \, \\
		\begin{center}
		\ifdys
		\newcommand{\scale}{1.04}
		\else
		\newcommand{\scale}{1.095}
		\fi
		\begin{tikzpicture}[>=stealth, scale=\scale]
		\begin{axis}[xmin = -10, xmax=10, ymin=-10, ymax=10, axis x line=middle, axis y line=middle, axis line style=<->, xlabel={}, ylabel={}, xtick = {-10, -8, ..., 8, 10}, ytick = {-10, -8, ..., 8, 10}, grid=both]
		
			\addplot[myr, thick, domain =-9:9, samples=2] {-x}  node[pos=.9, above=6pt] {$(\mathcal{C}_f)$};
			\addplot[myr, thick, dotted, domain =-10:-9, samples=2] {-x} ;
			\addplot[myr, thick, dotted, domain =9:10, samples=2] {-x};
		
		
			\addplot[myg, thick, domain =-9:9, samples=2] {x/2+1}  node[pos=.9, below=6pt] {$(\mathcal{C}_g)$};
			\addplot[myg, thick, dotted, domain =-10:-9, samples=2] {x/2+1} ;
			\addplot[myg, thick, dotted, domain =9:10, samples=2] {x/2+1};
		
		
			\addplot[black, thick, domain =-9:9, samples=2] {7}  node[pos=.7, above] {$(\mathcal{C}_h)$};
			\addplot[black, thick, dotted, domain =-10:-9, samples=2] {7} ;
			\addplot[black, thick, dotted, domain =9:10, samples=2] {7};
		
			
		\end{axis}
		\end{tikzpicture}
		\begin{tikzpicture}[>=stealth, scale=\scale]
		\begin{axis}[xmin = -10, xmax=10, ymin=-10, ymax=10, axis x line=middle, axis y line=middle, axis line style=<->, xlabel={}, ylabel={}, xtick = {-10, -8, ..., 8, 10}, ytick = {-10, -8, ..., 8, 10}, grid=both]
		
			\addplot[myr, thick, domain =-9:9, samples=2] {-x/3 + 1}  node[pos=.9, below=5pt] {$(\mathcal{C}_F)$};
			\addplot[myr, thick, dotted, domain =-10:-9, samples=2] {-x/3 + 1} ;
			\addplot[myr, thick, dotted, domain =9:10, samples=2] {-x/3 + 1};
		
		
			\addplot[myg, thick, domain =-5:7, samples=2] {3*x/2-2}  node[pos=.2, left] {$(\mathcal{C}_G)$};
			\addplot[myg, thick, dotted, domain =-6:-5, samples=2] {3*x/2-2} ;
			\addplot[myg, thick, dotted, domain =7:8, samples=2] {3*x/2-2};
		
		
			\addplot[black, thick, domain =-9:9, samples=2] {-x/6 + 5}  node[pos=.2, above] {$(\mathcal{C}_H)$};
			\addplot[black, thick, dotted, domain =-10:-9, samples=2] {-x/6 + 5} ;
			\addplot[black, thick, dotted, domain =9:10, samples=2] {-x/6 + 5};
		
			
		\end{axis}
	\end{tikzpicture}
	\end{center}
	
	\begin{enumerate}
		\item Pour chacune des droites ci-dessus, donner $2$ points lui appartenant, dont un d'abscisse nulle.
		\item En déduire l'ordonnée à l'origine $b$ de chacune des fonctions affines $f, g, h, F, G,$ et $H$.
		%\item Donner le signe (strictement positif, strictement négatif, ou nul) du coefficient directeur $a$ de chacune des fonctions affines $f, g, h, F, G,$ et $H$.
	\end{enumerate}
}{}

\newpage

\exe{
	\begin{multicols}{2}
	Soit $f(x) = ax+b$ la fonction affine donnée graphiquement pour $x\in[-0,6 ; 1]$ ci-contre.
	
	Déterminer les paramètres $a$ et $b$.
	\vfill
	
	\begin{center}
	\begin{tikzpicture}[>=stealth, scale=1]
	\begin{axis}[xmin = -1.2, xmax=1, ymin=-4, ymax=4, axis x line=middle, axis y line=middle, axis line style=<->, xlabel={}, ylabel={}, xtick = {-4, -3, ..., 4}, ytick = {-4, -3, ..., 4}, grid=both]
		
		% (d)
		\addplot[myb, thick, domain =-0.55:.9, samples=2] {-1+5*x}  node[pos = .8, above=2pt] {$\C_f$};
		\addplot[myb, thick, dotted, domain =-.6:-.55, samples=2] {-1+5*x} ;
		\addplot[myb, thick, dotted, domain =.9:1, samples=2] {-1+5*x};
		
		% 2 points
		\addplot[black, mark=*, mark size = 1] (.5,1.5) node[below] {$(0,5 ; 1,5)$};
		\addplot[black, mark=*, mark size = 1] (-.5,-3.5);
		\addplot[black] (-.7,-2.7) node{$(-0,5 ; -3,5)$};
	\end{axis}
	\end{tikzpicture}
	\end{center}
	\end{multicols}
}{}

\exe{
	\begin{multicols}{2}
	Soit $g(x) = ax+b$ la fonction affine donnée graphiquement pour $x\in[-1 ; 3]$ ci-contre.
	
	Déterminer les paramètres $a$ et $b$.
	\vfill
	
		\begin{center}
		\begin{tikzpicture}[>=stealth, scale=1]
		\begin{axis}[xmin = -1, xmax=3, ymin=-4, ymax=4, axis x line=middle, axis y line=middle, axis line style=<->, xlabel={}, ylabel={}, xtick = {-4, -3, ..., 4}, ytick = {-4, -3, ..., 4}, grid=both]
			
			% (d)
			\addplot[myb, thick, domain =-0.8:2.6, samples=2] {2-2*x}  node[above=3pt] {$\C_g$};
			\addplot[myb, thick, dotted, domain =-1:-.8, samples=2] {2-2*x} ;
			\addplot[myb, thick, dotted, domain =2.6:3, samples=2] {2-2*x};
			
			% (0,b)
			\addplot[black, mark=*, mark size = 1] (0,2) node[above right] {$(0;2)$};
			
			% (1, b+a)
			\addplot[black, mark=*, mark size = 1] (1,0) node[above right] {$(1;0)$};
			
			% (2, b+2a)
			\addplot[black, mark=*, mark size = 1] (2,-2) node[above right] {$(2;-2)$};
		\end{axis}
		\end{tikzpicture}
		\end{center}
	\end{multicols}
}{}

\exe{
		Calculer le coefficient directeur $a$ de chacune des fonctions affines $f, g, h, F, G,$ et $H$ de l'exercice \ref{ex:3}.
}{}

%\exe{
%	Soit $f(x) = ax+b$ une fonction affine sur $\R$ où $a,b\in\R$ sont des paramètres réels.
%	\begin{enumerate}
%		\item Montrer que, pour tout $x\in\R$,
%			\[ f(x+2) - f(x) = 2a. \]
%		\item En déduire le coefficient directeur $a$ de chacune des fonctions affines $f, g,$ et $h$ de l'exercice \ref{ex:3}.
%	\end{enumerate}
%}{}

%\exe{[Interpolation]
%	Soient $(1;2)$ et $(4;-4)$ deux points du plan qui appartiennent à une droite $\C_f$, où $f$ est affine.
%	Posons $f(x) = ax+b$ où $x$ parcourt $\R$ et où $a, b\in\R$ sont des paramètres à déterminer.
%	
%	\begin{enumerate}
%		\item Tracer la droite $\C_f$ dans un repère à l'aide des deux points donnés. En déduire les valeurs de $a$ et $b$ graphiquement.
%		\item Montrer, à l'aide de la propriété fondamentale, que les paramètres $a, b$ vérifient
%			\[ \begin{cases} 2 = a + b, \\ -4 = 4a + b. \end{cases} \]
%		\item Trouver $a$ et $b$ algébriquement et les comparer aux valeurs obtenues graphiquement.
%	\end{enumerate}
%}{}

\exe{[Interpolation]
	Pour chacune des paires de points $A, B$ suivantes, calculer les paramètres ($a$ et $b$) de l'unique fonction affine $f$ telle que $A, B \in \C_f$.
	
	\begin{multicols}{2}
	\begin{enumerate}
		\item $A(1;2), B(4;-4)$.
		\item $A(2;8), B(4;7)$.
		\item $A(4;7), B(2;8)$.
		\item $A(-3; -3), B(-2; -1)$.
		\item $A(2;5), B(-10; 5)$.
		\item $A(-3;4), B(12;-11)$.
	\end{enumerate}
	\end{multicols}

}{}


\end{document}

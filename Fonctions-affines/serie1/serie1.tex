				% ENABLE or DISABLE font change
				% use XeLaTeX if true
\newif\ifdys
				\dystrue
				\dysfalse

\newif\ifsolutions
				\solutionstrue
				\solutionsfalse

% DYSLEXIA SWITCH
\newif\ifdys
		
				% ENABLE or DISABLE font change
				% use XeLaTeX if true
				\dystrue
				\dysfalse


\ifdys

\documentclass[a4paper, 14pt]{extarticle}
\usepackage{amsmath,amsfonts,amsthm,amssymb,mathtools}

\tracinglostchars=3 % Report an error if a font does not have a symbol.
\usepackage{fontspec}
\usepackage{unicode-math}
\defaultfontfeatures{ Ligatures=TeX,
                      Scale=MatchUppercase }

\setmainfont{OpenDyslexic}[Scale=1.0]
\setmathfont{Fira Math} % Or maybe try KPMath-Sans?
\setmathfont{OpenDyslexic Italic}[range=it/{Latin,latin}]
\setmathfont{OpenDyslexic}[range=up/{Latin,latin,num}]

\else

\documentclass[a4paper, 12pt]{extarticle}

\usepackage[utf8x]{inputenc}
%fonts
\usepackage{amsmath,amsfonts,amsthm,amssymb,mathtools}
% comment below to default to computer modern
\usepackage{libertinus,libertinust1math}

\fi


\usepackage[french]{babel}
\usepackage[
a4paper,
margin=2cm,
nomarginpar,% We don't want any margin paragraphs
]{geometry}
\usepackage{icomma}

\usepackage{fancyhdr}
\usepackage{array}
\usepackage{hyperref}

\usepackage{multicol, enumerate}
\newcolumntype{P}[1]{>{\centering\arraybackslash}p{#1}}


\usepackage{stackengine}
\newcommand\xrowht[2][0]{\addstackgap[.5\dimexpr#2\relax]{\vphantom{#1}}}

% theorems

\theoremstyle{plain}
\newtheorem{theorem}{Th\'eor\`eme}
\newtheorem*{sol}{Solution}
\theoremstyle{definition}
\newtheorem{ex}{Exercice}
\newtheorem*{rpl}{Rappel}
\newtheorem{enigme}{Énigme}

% corps
\usepackage{calrsfs}
\newcommand{\C}{\mathcal{C}}
\newcommand{\R}{\mathbb{R}}
\newcommand{\Rnn}{\mathbb{R}^{2n}}
\newcommand{\Z}{\mathbb{Z}}
\newcommand{\N}{\mathbb{N}}
\newcommand{\Q}{\mathbb{Q}}

% variance
\newcommand{\Var}[1]{\text{Var}(#1)}

% domain
\newcommand{\D}{\mathcal{D}}


% date
\usepackage{advdate}
\AdvanceDate[0]


% plots
\usepackage{pgfplots}

% table line break
\usepackage{makecell}
%tablestuff
\def\arraystretch{2}
\setlength\tabcolsep{15pt}

%subfigures
\usepackage{subcaption}

\definecolor{myg}{RGB}{56, 140, 70}
\definecolor{myb}{RGB}{45, 111, 177}
\definecolor{myr}{RGB}{199, 68, 64}

% fake sections with no title to move around the merged pdf
\newcommand{\fakesection}[1]{%
  \par\refstepcounter{section}% Increase section counter
  \sectionmark{#1}% Add section mark (header)
  \addcontentsline{toc}{section}{\protect\numberline{\thesection}#1}% Add section to ToC
  % Add more content here, if needed.
}


% SOLUTION SWITCH
\newif\ifsolutions
				\solutionstrue
				%\solutionsfalse

\ifsolutions
	\newcommand{\exe}[2]{
		\begin{ex} #1  \end{ex}
		\begin{sol} #2 \end{sol}
	}
\else
	\newcommand{\exe}[2]{
		\begin{ex} #1  \end{ex}
	}
	
\fi


% tableaux var, signe
\usepackage{tkz-tab}


%pinfty minfty
\newcommand{\pinfty}{{+}\infty}
\newcommand{\minfty}{{-}\infty}

\begin{document}


\AdvanceDate[0]

\begin{document}
\pagestyle{fancy}
\fancyhead[L]{Seconde 13}
\fancyhead[C]{\textbf{Fonctions affines 1\ifsolutions -- Solutions  \fi}}
\fancyhead[R]{\today}

\exe{
	Pour chaque fonction affine sur $\R$ suivante, déterminer son coefficient directeur $a$ et son ordonnée à l'origine $b$.
	\begin{multicols}{2}
	\begin{enumerate}
		\item $f(x) = 2x + 1$
		\item $f(x) = 1 + 2x$
		\item $f(x) = - x$
		\item $f(x) = -42$
		\item $f(x) = 10x + 2$
		\item $f(x) = 2 + 10x$
		\item $f(x) = 1 - x$
		\item $f(x) = 0$
	\end{enumerate}
	\end{multicols}
}{}

\exe{
	Donner $2$ points appartenant à la courbe représentative de chaque fonction affine et l'esquisser dans un repère de domaine $\D = [-3 ; 3]$.
	\begin{multicols}{2}
	\begin{enumerate}
		\item $f(x) = 1 + 2x$
		\item $g(x) = 2 + 2x$
		\item $h(x) = - x$
		\item $F(x) = 1 - x$
		\item $G(x) = -\frac12$
		\item $H(x) = 2 - \frac13x$
	\end{enumerate}
	\end{multicols}
}{}

\exe{\label{ex:3} \, \\
		\begin{center}
		\newcommand{\scale}{1.095}
		\begin{tikzpicture}[>=stealth, scale=\scale]
		\begin{axis}[xmin = -10, xmax=10, ymin=-10, ymax=10, axis x line=middle, axis y line=middle, axis line style=<->, xlabel={}, ylabel={}, xtick = {-10, -8, ..., 8, 10}, ytick = {-10, -8, ..., 8, 10}, grid=both]
		
			\addplot[myr, thick, domain =-9:9, samples=2] {-x}  node[above=6pt] {$(\mathcal{C}_f)$};
			\addplot[myr, thick, dotted, domain =-10:-9, samples=2] {-x} ;
			\addplot[myr, thick, dotted, domain =9:10, samples=2] {-x};
		
		
			\addplot[myg, thick, domain =-9:9, samples=2] {x/2+1}  node[below=6pt] {$(\mathcal{C}_g)$};
			\addplot[myg, thick, dotted, domain =-10:-9, samples=2] {x/2+1} ;
			\addplot[myg, thick, dotted, domain =9:10, samples=2] {x/2+1};
		
		
			\addplot[black, thick, domain =-9:9, samples=2] {7}  node[pos=.7, above] {$(\mathcal{C}_h)$};
			\addplot[black, thick, dotted, domain =-10:-9, samples=2] {7} ;
			\addplot[black, thick, dotted, domain =9:10, samples=2] {7};
		
			
		\end{axis}
		\end{tikzpicture}
		\begin{tikzpicture}[>=stealth, scale=\scale]
		\begin{axis}[xmin = -10, xmax=10, ymin=-10, ymax=10, axis x line=middle, axis y line=middle, axis line style=<->, xlabel={}, ylabel={}, xtick = {-10, -8, ..., 8, 10}, ytick = {-10, -8, ..., 8, 10}, grid=both]
		
			\addplot[myr, thick, domain =-9:9, samples=2] {-x/3 + 1}  node[pos=.9, below=5pt] {$(\mathcal{C}_F)$};
			\addplot[myr, thick, dotted, domain =-10:-9, samples=2] {-x/3 + 1} ;
			\addplot[myr, thick, dotted, domain =9:10, samples=2] {-x/3 + 1};
		
		
			\addplot[myg, thick, domain =-5:7, samples=2] {3*x/2-2}  node[pos=.2, left] {$(\mathcal{C}_G)$};
			\addplot[myg, thick, dotted, domain =-6:-5, samples=2] {3*x/2-2} ;
			\addplot[myg, thick, dotted, domain =7:8, samples=2] {3*x/2-2};
		
		
			\addplot[black, thick, domain =-9:9, samples=2] {-x/6 + 5}  node[pos=.2, above] {$(\mathcal{C}_H)$};
			\addplot[black, thick, dotted, domain =-10:-9, samples=2] {-x/6 + 5} ;
			\addplot[black, thick, dotted, domain =9:10, samples=2] {-x/6 + 5};
		
			
		\end{axis}
	\end{tikzpicture}
	\end{center}
	
	\begin{enumerate}
		\item Donner $2$ points appartenant à chaque courbe représentative, dont un d'abscisse nulle.
		\item En déduire l'ordonnée à l'origine $b$ de chacune des fonctions affines $f, g, h, F, G,$ et $H$.
		\item Donner le signe (strictement positif, strictement négatif, ou nul) du coefficient directeur $a$ de chacune des fonctions affines $f, g, h, F, G,$ et $H$.
	\end{enumerate}
}{}

\exe{
	Soit $f(x) = ax+b$ une fonction affine sur $\R$ où $a,b\in\R$ sont des paramètres réels.
	\begin{enumerate}
		\item Montrer que, pour tout $x\in\R$,
			\[ f(x+2) - f(x) = 2a. \]
		\item En déduire le coefficient directeur $a$ de chacune des fonctions affines $f, g, h, F, G,$ et $H$ de l'exercice \ref{ex:3}.
	\end{enumerate}
}{}

\newpage

\exe{[Interpolation]
	Soient $(1;2)$ et $(4;-4)$ deux points du plan qui appartiennent à une droite $\C_f$, où $f$ est affine.
	Posons $f(x) = ax+b$ où $x$ parcourt $\R$ et où $a, b\in\R$ sont des paramètres à déterminer.
	
	\begin{enumerate}
		\item Tracer la droite $\C_f$ dans un repère à l'aide des deux points donnés. En déduire les valeurs de $a$ et $b$ graphiquement.
		\item Montrer, à l'aide de la propriété fondamentale, que les paramètres $a, b$ vérifient
			\[ \begin{cases} 2 = a + b, \\ -4 = 4a + b. \end{cases} \]
		\item Trouver $a$ et $b$ algébriquement et comparer aux valeurs obtenues graphiquement.
	\end{enumerate}
}{}

\exe{[Interpolation]
	Pour chacune des paires de points $A, B$ suivantes, calculer les paramètres ($a$ et $b$) de la fonction affine $f$ telle que $A, B \in \C_f$.
	
	\begin{multicols}{2}
	\begin{enumerate}
		\item $A(1;2), B(4;-4)$.
		\item $A(2;8), B(4;7)$.
		\item $A(4;7), B(2;8)$.
		\item $A(-3; -3), B(-2; -1)$.
		\item $A(2;5), B(-10; 5)$.
		\item $A(-3;4), B(12;-11)$.
	\end{enumerate}
	\end{multicols}

}{}


\end{document}

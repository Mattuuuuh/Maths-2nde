%% INPUT PREAMBLE.TEX
%% THEN INPUT VARS_{i}.ADR
%% THEN RUN THIS
%% DYSLEXIA SWITCH
\newif\ifdys
		
				% ENABLE or DISABLE font change
				% use XeLaTeX if true
				\dystrue
				\dysfalse


\ifdys

\documentclass[a4paper, 14pt]{extarticle}
\usepackage{amsmath,amsfonts,amsthm,amssymb,mathtools}

\tracinglostchars=3 % Report an error if a font does not have a symbol.
\usepackage{fontspec}
\usepackage{unicode-math}
\defaultfontfeatures{ Ligatures=TeX,
                      Scale=MatchUppercase }

\setmainfont{OpenDyslexic}[Scale=1.0]
\setmathfont{Fira Math} % Or maybe try KPMath-Sans?
\setmathfont{OpenDyslexic Italic}[range=it/{Latin,latin}]
\setmathfont{OpenDyslexic}[range=up/{Latin,latin,num}]

\else

\documentclass[a4paper, 12pt]{extarticle}

\usepackage[utf8x]{inputenc}
%fonts
\usepackage{amsmath,amsfonts,amsthm,amssymb,mathtools}
% comment below to default to computer modern
\usepackage{libertinus,libertinust1math}

\fi


\usepackage[french]{babel}
\usepackage[
a4paper,
margin=2cm,
nomarginpar,% We don't want any margin paragraphs
]{geometry}
\usepackage{icomma}

\usepackage{fancyhdr}
\usepackage{array}
\usepackage{hyperref}

\usepackage{multicol, enumerate}
\newcolumntype{P}[1]{>{\centering\arraybackslash}p{#1}}


\usepackage{stackengine}
\newcommand\xrowht[2][0]{\addstackgap[.5\dimexpr#2\relax]{\vphantom{#1}}}

% theorems

\theoremstyle{plain}
\newtheorem{theorem}{Th\'eor\`eme}
\newtheorem*{sol}{Solution}
\theoremstyle{definition}
\newtheorem{ex}{Exercice}
\newtheorem*{rpl}{Rappel}
\newtheorem{enigme}{Énigme}

% corps
\usepackage{calrsfs}
\newcommand{\C}{\mathcal{C}}
\newcommand{\R}{\mathbb{R}}
\newcommand{\Rnn}{\mathbb{R}^{2n}}
\newcommand{\Z}{\mathbb{Z}}
\newcommand{\N}{\mathbb{N}}
\newcommand{\Q}{\mathbb{Q}}

% variance
\newcommand{\Var}[1]{\text{Var}(#1)}

% domain
\newcommand{\D}{\mathcal{D}}


% date
\usepackage{advdate}
\AdvanceDate[0]


% plots
\usepackage{pgfplots}

% table line break
\usepackage{makecell}
%tablestuff
\def\arraystretch{2}
\setlength\tabcolsep{15pt}

%subfigures
\usepackage{subcaption}

\definecolor{myg}{RGB}{56, 140, 70}
\definecolor{myb}{RGB}{45, 111, 177}
\definecolor{myr}{RGB}{199, 68, 64}

% fake sections with no title to move around the merged pdf
\newcommand{\fakesection}[1]{%
  \par\refstepcounter{section}% Increase section counter
  \sectionmark{#1}% Add section mark (header)
  \addcontentsline{toc}{section}{\protect\numberline{\thesection}#1}% Add section to ToC
  % Add more content here, if needed.
}


% SOLUTION SWITCH
\newif\ifsolutions
				\solutionstrue
				%\solutionsfalse

\ifsolutions
	\newcommand{\exe}[2]{
		\begin{ex} #1  \end{ex}
		\begin{sol} #2 \end{sol}
	}
\else
	\newcommand{\exe}[2]{
		\begin{ex} #1  \end{ex}
	}
	
\fi


% tableaux var, signe
\usepackage{tkz-tab}


%pinfty minfty
\newcommand{\pinfty}{{+}\infty}
\newcommand{\minfty}{{-}\infty}

\begin{document}
\input{adr/vars_44284.adr}

\pagestyle{fancy}
\fancyhead[L]{Seconde 13}
\fancyhead[C]{\textbf{Devoir Maison 3 -- \seed \ifsolutions \, -- Solutions  \fi}}
\fancyhead[R]{\today}

\begin{rpl}
	Pour une fonction affine $f(x) = ax+b$, on parle de $\C_f$ comme la \og {droite d'équation} $y=ax+b$ \fg, et on note
		\[ \C_f : y = ax+b. \]
\end{rpl}

\exe{
	On considère trois droites, données par les équations suivantes.
		\begin{align*}
			(d) : y= \aI x  \bI, && (e) : y = \aII x  \bII, && (f) : y = \aIII x  \bIII.
		\end{align*}
	\begin{enumerate}
		\item Tracer les droites $(d), (e),$ et $(f)$ dans un repère de domaine $\D = [\xmin; \xmax]$.
		\item Calculer $(d)\cap (e)$, $(d) \cap (f)$, et $(e)\cap (f)$. Ces trois points d'intersection sont les sommets du triangle $T$.
		\item Calculer la longueur de chacun des côtés de $T$. Que dire de $T$ ?
	\end{enumerate}
}{}


\exe{\label{ex:2}
	Considérons une fonction linéaire $f(x) = \slopeI x$.
	On dit que $f$ est \emph{linéaire} car son ordonnée à l'origine $b$ est nulle — c'est donc un cas particulier de fonction affine.
	
	On définit la \emph{pente} de $f$ comme étant son coefficient directeur sous forme de pourcentage, soit $\slopeIpercent\%$.
	C'est le pourcentage qu'on voit écrit sur les panneaux A16 de signalisation annonçant une descente dangeureuse.
	
	Pour chacune des questions suivantes, arrondir les valeurs à $10^{-2}$ près et les pourcentages au dixième le plus proche.
	\begin{enumerate}
		\item Montrer que $\C_f$ passe par l'origine $O(0;0)$ du repère.
		\item Trouver le point $A$ d'abscisse $\Ax$ appartenant à $\C_f$ .
		\item Dans un repère, tracer le triangle de sommets $O, A, B(\Ax;0)$ et calculer l'angle $\alpha = \widehat{AOB}$.
		\item Un ingénieur détermine qu'une route ne doit pas dépasser une pente de $\slopeIIpercent\%$ pour qu'elle soit accessible pour les poids lourds.
		Quelle est l'ordonnée maximale d'un point $C$ d'abscisse $\Ax$ si la pente de la fonction linéaire passant par $C$ ne peut pas dépasser $\slopeIIpercent\%$ ?
		\item
		Calculer l'angle $\beta = \widehat{COB}$ dans ce cas.
		\item 
		À quel angle $\gamma$ correspond une pente de $\slopepercent\%$ ?
		\item 
		À quel pourcentage de pente correspond un angle de $\angle$° ?
	\end{enumerate}
	
}{}

\begin{figure}[h!]
	\begin{subfigure}{0.65\textwidth}
	\centering
	\begin{tikzpicture}[>=stealth, scale=1.3]
	\begin{axis}[xmin = -.8, xmax=10, ymin=-.8, ymax=8, axis x line=middle, axis y line=middle, axis line style=<->, xlabel={}, ylabel={},ticks = none, grid=none]
	
		\draw[thick, black] (axis cs: 8, 0) -- (axis cs: 8,5);
		\draw[thick, black] (axis cs: 0, 0) -- (axis cs: 8,5) node[pos = .5, above, sloped] {\huge\faTruck};
		\draw (axis cs: 0, 0) node[above left] {$O$};
		\draw (axis cs: 8, 0) node[below] {$B$};
		\draw (axis cs: 8, 5) node[above left] {$A$};
		\addplot[thick, myg, domain = .86:1] {sqrt(1-x^2) } node[right, pos=.6] {$\alpha$};
		\addplot[dashed, thick, black, domain = -1:10] { 5*x/8 } node[above, pos=.95] {$\C_f$};
		
		
		\draw (axis cs: 8, 3) node[right] {$C$};
		\draw[thick, dotted, black] (axis cs: 0, 0) -- (axis cs: 8,3);
		\addplot[thick, myr, domain = 1.75:2] {sqrt(1-(x-1)^2) } node[right, pos=.5] {$\beta$};
	\end{axis}
	\end{tikzpicture}
	\caption{Schéma simplifié, non à l'échelle, de l'exercice \ref{ex:2}.}
	\end{subfigure}
	%\hspace{2cm}
	\begin{subfigure}{0.3\textwidth}
	\centering
	\includesvg[scale=.2]{France_road_sign_A16.svg}
	\caption{Exemple de panneau A16.}
	\end{subfigure}
\end{figure}



\end{document}

%% INPUT PREAMBLE.TEX
%% THEN INPUT VARS_{i}.ADR
%% THEN RUN THIS
% DYSLEXIA SWITCH
\newif\ifdys
		
				% ENABLE or DISABLE font change
				% use XeLaTeX if true
				\dystrue
				\dysfalse


\ifdys

\documentclass[a4paper, 14pt]{extarticle}
\usepackage{amsmath,amsfonts,amsthm,amssymb,mathtools}

\tracinglostchars=3 % Report an error if a font does not have a symbol.
\usepackage{fontspec}
\usepackage{unicode-math}
\defaultfontfeatures{ Ligatures=TeX,
                      Scale=MatchUppercase }

\setmainfont{OpenDyslexic}[Scale=1.0]
\setmathfont{Fira Math} % Or maybe try KPMath-Sans?
\setmathfont{OpenDyslexic Italic}[range=it/{Latin,latin}]
\setmathfont{OpenDyslexic}[range=up/{Latin,latin,num}]

\else

\documentclass[a4paper, 12pt]{extarticle}

\usepackage[utf8x]{inputenc}
%fonts
\usepackage{amsmath,amsfonts,amsthm,amssymb,mathtools}
% comment below to default to computer modern
\usepackage{libertinus,libertinust1math}

\fi


\usepackage[french]{babel}
\usepackage[
a4paper,
margin=2cm,
nomarginpar,% We don't want any margin paragraphs
]{geometry}
\usepackage{icomma}

\usepackage{fancyhdr}
\usepackage{array}
\usepackage{hyperref}

\usepackage{multicol, enumerate}
\newcolumntype{P}[1]{>{\centering\arraybackslash}p{#1}}


\usepackage{stackengine}
\newcommand\xrowht[2][0]{\addstackgap[.5\dimexpr#2\relax]{\vphantom{#1}}}

% theorems

\theoremstyle{plain}
\newtheorem{theorem}{Th\'eor\`eme}
\newtheorem*{sol}{Solution}
\theoremstyle{definition}
\newtheorem{ex}{Exercice}
\newtheorem*{rpl}{Rappel}
\newtheorem{enigme}{Énigme}

% corps
\usepackage{calrsfs}
\newcommand{\C}{\mathcal{C}}
\newcommand{\R}{\mathbb{R}}
\newcommand{\Rnn}{\mathbb{R}^{2n}}
\newcommand{\Z}{\mathbb{Z}}
\newcommand{\N}{\mathbb{N}}
\newcommand{\Q}{\mathbb{Q}}

% variance
\newcommand{\Var}[1]{\text{Var}(#1)}

% domain
\newcommand{\D}{\mathcal{D}}


% date
\usepackage{advdate}
\AdvanceDate[0]


% plots
\usepackage{pgfplots}

% table line break
\usepackage{makecell}
%tablestuff
\def\arraystretch{2}
\setlength\tabcolsep{15pt}

%subfigures
\usepackage{subcaption}

\definecolor{myg}{RGB}{56, 140, 70}
\definecolor{myb}{RGB}{45, 111, 177}
\definecolor{myr}{RGB}{199, 68, 64}

% fake sections with no title to move around the merged pdf
\newcommand{\fakesection}[1]{%
  \par\refstepcounter{section}% Increase section counter
  \sectionmark{#1}% Add section mark (header)
  \addcontentsline{toc}{section}{\protect\numberline{\thesection}#1}% Add section to ToC
  % Add more content here, if needed.
}


% SOLUTION SWITCH
\newif\ifsolutions
				\solutionstrue
				%\solutionsfalse

\ifsolutions
	\newcommand{\exe}[2]{
		\begin{ex} #1  \end{ex}
		\begin{sol} #2 \end{sol}
	}
\else
	\newcommand{\exe}[2]{
		\begin{ex} #1  \end{ex}
	}
	
\fi


% tableaux var, signe
\usepackage{tkz-tab}


%pinfty minfty
\newcommand{\pinfty}{{+}\infty}
\newcommand{\minfty}{{-}\infty}

\begin{document}
\input{adr/vars_44284.adr}

\pagestyle{fancy}
\fancyhead[L]{Seconde 13}
\fancyhead[C]{\textbf{Devoir Maison 3 -- \seed \ifsolutions \, -- Solutions  \fi}}
\fancyhead[R]{\today}

\exe{
	On considère trois fonctions affines données par, pour tout $x\in\R$,
		\begin{align*}
			f(x) = \dfrac43 (x-1) + 2, && g(x) = \dfrac{12}5 (x-1) + 2, && h(x) = \dfrac{-4}7 (x-1) + \dfrac{24}{7
			} + 2.
		\end{align*}
	\begin{enumerate}
		\item Tracer les droites $\C_f, \C_g,$ et $\C_h$ dans un repère de domaine $\D = [-3; 5]$.
		\item Calculer $\C_f \cap \C_g$, $\C_f \cap \C_h$, et $\C_g \cap \C_h$. Ces trois points d'intersection sont les sommets du triangle $T$.
		\item Calculer la longueur de chacun des côtés de $T$. Que dire de $T$ ?
	\end{enumerate}
}{}

\exe{
	Pour chaque paire de point $P$ et de fonction $f$, déterminer si $P\in\C_f$ ou non.
	
	\begin{multicols}{2}
	\begin{enumerate}
		\item $P = (0;1)$ et $f(x) = (3x- 1)^2$
		\item $P = (0;1)$ et $f(x) = (3x- 1)^3$
		\item $P = (-6;5)$ et $f(x) = \dfrac23 x + 7$
		\item $P = (-2;2)$ et $f(x) = 4-x$
	\end{enumerate}
	\end{multicols}
}{}

%\exe{
%	Pour chacune des paires de fonctions $f, g$ sur $\R$, calculer $\C_f \cap \C_g$.
%	
%	\begin{multicols}{2}
%	\begin{enumerate}
%		\item $f(x) = 7x -2, g(x) = -x+3$.
%		\item $f(x) = -x + 3, g(x) = 7x -2$.
%		\item $f(x) = 3, g(x) = -x$.
%		\item $f(x) = -2x+1, g(x) = 1-x$.
%		\item $f(x) = 2, g(x) = 4$.
%		\item $f(x) = -x+1, g(x) = 1-x$.
%	\end{enumerate}
%	\end{multicols}
%}{}

%\exe{
%	Montrer que l'identité
%		\[ x^2 = \a x \]
%	ne peut pas être vraie pour tout $x\in\R$.
%}{}
%
%\exe{[$\star$]
%	Montrer que, pour $m, n\in\N$, l'identité
%		\[ x^m = x^n \]
%	ne peut être vraie pour tout $x\in\R$ que lorsque $m=n$.
%}{}
%
%
%\exe{
%	Montrer que l'identité
%		\[ \sqrt{x^2} = x \]
%	ne peut pas être vraie pour tout $x\in\R$, puis que $\sqrt{x^2} = |x|$ pour tout $x\in\R$.
%}{}

\exe{
	Développer et réduire les expressions suivantes pour obtenir une expression de la forme
		\[ ax^2 + bx + c, \]
	où $a, b, c \in\R$ sont des nombres réels.
	
	\begin{multicols}{2}
	\begin{enumerate}
		\item $x + 2x + 3x$
		\item $(1+3x) - (8 - 2x)$
		\item $x - 2(2-x)$
		\item $x(3-x) + 8(x+1)$
		\item $(x+3)(x-1) - 2x + 1$
		\item $(x+1)^2 + 4x - 4$
	\end{enumerate}
	\end{multicols}
}{}

\end{document}

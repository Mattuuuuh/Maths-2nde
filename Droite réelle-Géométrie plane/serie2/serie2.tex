% DYSLEXIA SWITCH
\newif\ifdys
		
				% ENABLE or DISABLE font change
				% use XeLaTeX if true
				\dystrue
				\dysfalse


\ifdys

\documentclass[a4paper, 14pt]{extarticle}
\usepackage{amsmath,amsfonts,amsthm,amssymb,mathtools}

\tracinglostchars=3 % Report an error if a font does not have a symbol.
\usepackage{fontspec}
\usepackage{unicode-math}
\defaultfontfeatures{ Ligatures=TeX,
                      Scale=MatchUppercase }

\setmainfont{OpenDyslexic}[Scale=1.0]
\setmathfont{Fira Math} % Or maybe try KPMath-Sans?
\setmathfont{OpenDyslexic Italic}[range=it/{Latin,latin}]
\setmathfont{OpenDyslexic}[range=up/{Latin,latin,num}]

\else

\documentclass[a4paper, 12pt]{extarticle}
\usepackage{amsmath,amsfonts,amsthm,amssymb,mathtools}

\fi


\usepackage[french]{babel}
\usepackage[
a4paper,
margin=2cm,
nomarginpar,% We don't want any margin paragraphs
]{geometry}
\usepackage{fancyhdr}
\usepackage{array}

\usepackage{multicol, enumitem}
\newcolumntype{P}[1]{>{\centering\arraybackslash}p{#1}}


\usepackage{stackengine}
\newcommand\xrowht[2][0]{\addstackgap[.5\dimexpr#2\relax]{\vphantom{#1}}}

% theorems

\theoremstyle{plain}
\newtheorem{theorem}{Th\'eor\`eme}
\newtheorem*{sol}{Solution}
\theoremstyle{definition}
\newtheorem{ex}{Exercice}

% corps
\newcommand{\C}{\mathbb{C}}
\newcommand{\R}{\mathbb{R}}
\newcommand{\Rnn}{\mathbb{R}^{2n}}
\newcommand{\Z}{\mathbb{Z}}
\newcommand{\N}{\mathbb{N}}
\newcommand{\Q}{\mathbb{Q}}

% domain
\newcommand{\D}{\mathbb{D}}


% date
\usepackage{advdate}
\AdvanceDate[2]


% plots
%\usepackage{pgfplots}
\usepackage{tikz}


% SOLUTION SWITCH
\newif\ifsolutions
				\solutionstrue
				\solutionsfalse

\ifsolutions
	\newcommand{\exe}[2]{
		\begin{ex} #1  \end{ex}
		\begin{sol} #2 \end{sol}
	}
\else
	\newcommand{\exe}[1]{
		\begin{ex} #1  \end{ex}
	}
	
\fi

\begin{document}
\pagestyle{fancy}
\fancyhead[L]{Seconde 13}
\fancyhead[C]{\textbf{ Exercices : inégalités et valeurs absolues \ifsolutions -- Solutions  \fi}}
\fancyhead[R]{\today}

\exe{\, \\
	\begin{multicols}{2}
	
	
	ABCD est un rectangle tel que $AB = 8$ et $AD= 10$.
	M est un point du segment $[AD]$ et $N$ est le point de $[BC]$ tel que $ABNM$ est un rectangle.
	
	On pose $x=AM$.	
	\begin{enumerate}
		\item À quel intervalle appartient $x$ ?
		\item Pour quelle(s) valeur(s) de $x$ l'aire de $ABNM$ est-elle supérieure ou égale à celle du triangle $NDC$ ?
	\end{enumerate}
	
	
	\begin{center}
	\begin{tikzpicture}[scale=2]
		\draw[black,thick] (0,0) node [below, left] {$A$} -- (2,0) node [below, right] {$B$};
		\draw[black,thick] (0,0) -- (0,3) node [above, left] {$D$};
		\draw[black,thick] (2,0) -- (2,3) node [above, right] {$C$};
		\draw[black,thick] (0,3) -- (2,3);
		
		\draw[black, thick] (0,1) node[left] {$M$} -- (2,1) node[right] {$N$};
	\end{tikzpicture}
	\end{center}
	\end{multicols}
}
{}


\exe{ \hspace{1cm} \\
	\begin{multicols}{2}
		\begin{center}
		\begin{tikzpicture}[scale=0.8]
		% real line
		\draw[black, thick] (0,0) -- (7,0);
		\draw[black,thick] (7,0) -- (7,7);
		\draw[black,thick] (7,7) -- (2,7);
		\draw[black,thick] (2,7)--(2,2);
		\draw[black, thick](2,2)--(0,2);
		\draw[black,thick](0,2)--(0,0);
		
		\draw[black,thick, dotted](2,0)--(2,2);
		
		\draw[<->, thick] (2,-.2) -- (7,-.2) node [midway, below] {$5$} ;
		\draw[<->, thick] (0,-.2) -- (2,-.2) node [midway, below] {$x$} ;
		\draw[<->, thick] (-.2,0) -- (-.2,2) node [midway, left] {$x$} ;
		\draw[<->, thick] (7.2,0) -- (7.2,7) node [midway, right] {$12$} ;
	\end{tikzpicture}
	\end{center}
	
	Considérons la figure ci-contre. La longueur du côté du carré de gauche doit rester inférieure à la longueur du rectangle de droite.
	\begin{enumerate}
		\item À quel intervalle appartient $x$ ?
		\item Donner l'ensemble des valeurs de $x$ pour que le périmètre soit supérieur ou égal à $50$.
	\end{enumerate}
	
	\end{multicols}
}
{}

\newpage

\exe{
	Le gardien d'un zoo souhaite créer un enclos rectangulaire pour ses renards polaires. 
	La largeur de son enclos est de $30$m et un puits se situe à $70$m du bord gauche de l'enclos.
	
	Celui-ci souhaite déterminer la longueur $x$ de l'enclos avec les restrictions suivantes.
	
	\begin{enumerate}[leftmargin=2cm, label=\roman*)]
		\item Le puits, noté par une étoile ($*$), doit être situé à moins de $10$ mètres du bord droit de l'enclos.
		\item Le périmètre de l'enclos ne doit pas dépasser $210$m.
	\end{enumerate}
	
	\begin{enumerate}
		\item Écrire les deux contraintes sous forme d'inégalités que la longueur $x$ vérifie.
		\item À quel intervalle appartient $x$ ?
		\item À quel intervalle appartient alors l'aire totale ? Quel $x$ choisir pour maximiser la superficie de l'enclos et donc le bien-être animal ?
	\end{enumerate}
	
	\begin{center}
	\begin{tikzpicture}[scale=.8]
		\draw[black, thick] (0,0) -- (6,0);
		\draw[black,thick,dotted](6,0) -- (10,0);
		\draw[black, thick] (0,0) -- (0,5);
		\draw[<->, black, thick] (-.2, 0) -- (-.2, 5) node[midway, left] {$30$};
		\draw[black,thick, dotted] (10,0) -- (10,5);
		%\draw[black,thick] (6,0) -- (6,5);
		\draw[black, thick] (0,5) -- (6,5);
		\draw[black,thick, dotted] (6,5)--(10,5);
		
		\draw (8,2.5) node {$*$};
		\draw[<->, black,thick] (.2,2.5) -- (7.8,2.5) node [midway, above] {$70$};
		
		\draw[<->, black, thick] (0,-.2) -- (10,-.2) node [midway, below] {$x$};
	
	\end{tikzpicture}
	\end{center}
	
}
{}



\exe{
	Un soigneur animalier souhaite créer un enclos rectangulaire pour ses addax.
	La largeur de son enclos est de $20$m. Un rocher situe à $40$m du bord gauche de l'enclos, et une prise électrique se situe à $35$m du bord gauche.
	
	Celui-ci souhaite déterminer longueur $x$ de l'enclos avec les restrictions suivantes.
	\begin{enumerate}[leftmargin=2cm, label=\roman*)]
		\item Le côté droit de l'enclos doit être situé à plus de $3$ mètres du rocher, noté par un carré ($\square$).
		\item Le côté droit de l'enclos doit être situé à moins de $10$ mètres de la prise électrique, notée par une étoile ($*$).
	\end{enumerate}
	
	\begin{enumerate}
		\item Écrire les deux contraintes sous forme d'intervalles auxquels la longueur $x$ appartient.
		\item À quel ensemble appartient $x$ ? Une union d'intervalles est attendue.
		\item Quel $x$ choisir pour minimiser le périmètre et donc le coût de l'enclos ?
	\end{enumerate}
	
	\begin{center}
	\begin{tikzpicture}[scale=.8]
		\draw[black, thick] (0,0) -- (6,0);
		\draw[black,thick,dotted](6,0) -- (10,0);
		\draw[black, thick] (0,0) -- (0,5);
		\draw[<->, black, thick] (-.2, 0) -- (-.2, 5) node[midway, left] {$20$};
		\draw[black,thick, dotted] (10,0) -- (10,5);
		\draw[black, thick] (0,5) -- (6,5);
		\draw[black,thick, dotted] (6,5)--(10,5);
		
		\draw (8,3.75) node {$\square$};
		\draw[<->, black,thick] (.2,3.75) -- (7.8,3.75) node [midway, above] {$40$};
		
		\draw (6,1.25) node {$*$};
		\draw[<->, black,thick] (.2,1.25) -- (5.8,1.25) node [midway, above] {$35$};
		
		\draw[<->, black, thick] (0,-.2) -- (10,-.2) node [midway, below] {$x$};
	
	\end{tikzpicture}
	\end{center}

}
{}





\end{document}
% DYSLEXIA SWITCH
\newif\ifdys
		
				% ENABLE or DISABLE font change
				% use XeLaTeX if true
				\dystrue
				\dysfalse


\ifdys

\documentclass[a4paper, 14pt]{extarticle}
\usepackage{amsmath,amsfonts,amsthm,amssymb,mathtools}

\tracinglostchars=3 % Report an error if a font does not have a symbol.
\usepackage{fontspec}
\usepackage{unicode-math}
\defaultfontfeatures{ Ligatures=TeX,
                      Scale=MatchUppercase }

\setmainfont{OpenDyslexic}[Scale=1.0]
\setmathfont{Fira Math} % Or maybe try KPMath-Sans?
\setmathfont{OpenDyslexic Italic}[range=it/{Latin,latin}]
\setmathfont{OpenDyslexic}[range=up/{Latin,latin,num}]

\else

\documentclass[a4paper, 12pt]{extarticle}
\usepackage{amsmath,amsfonts,amsthm,amssymb,mathtools}

\fi


\usepackage[french]{babel}
\usepackage[
a4paper,
margin=2cm,
nomarginpar,% We don't want any margin paragraphs
]{geometry}
\usepackage{fancyhdr}
\usepackage{array}

\usepackage{multicol, enumerate}
\newcolumntype{P}[1]{>{\centering\arraybackslash}p{#1}}


\usepackage{stackengine}
\newcommand\xrowht[2][0]{\addstackgap[.5\dimexpr#2\relax]{\vphantom{#1}}}

% theorems

\theoremstyle{plain}
\newtheorem{theorem}{Th\'eor\`eme}
\newtheorem*{sol}{Solution}
\theoremstyle{definition}
\newtheorem{ex}{Exercice}

% corps
\newcommand{\C}{\mathbb{C}}
\newcommand{\R}{\mathbb{R}}
\newcommand{\Rnn}{\mathbb{R}^{2n}}
\newcommand{\Z}{\mathbb{Z}}
\newcommand{\N}{\mathbb{N}}
\newcommand{\Q}{\mathbb{Q}}

% domain
\newcommand{\D}{\mathbb{D}}


% date
\usepackage{advdate}
\AdvanceDate[1]


% plots
\usepackage{pgfplots}


% SOLUTION SWITCH
\newif\ifsolutions
				\solutionstrue
				\solutionsfalse

\ifsolutions
	\newcommand{\exe}[2]{
		\begin{ex} #1  \end{ex}
		\begin{sol} #2 \end{sol}
	}
\else
	\newcommand{\exe}[2]{
		\begin{ex} #1  \end{ex}
	}
	
\fi

\begin{document}
\pagestyle{fancy}
\fancyhead[L]{Seconde 13}
\fancyhead[C]{\textbf{ Géométrie : droite et plan \ifsolutions -- Solutions  \fi}}
\fancyhead[R]{\today}

\subsection*{Droite réelle}

\exe{
	Calculer le milieu et la longueur de chaque intervalle suivant.
	\begin{multicols}{2}
	\begin{enumerate}
		\item $[-1 ; 1]$
		\item $]{-1} ; 1[$
		\item $[-14{,}7001 ; -14{,699}]$
		\item $[-3{,}3 ; -\dfrac23 [$
		\item $\{ x \in \R \text{ tq. } -\dfrac27 > x \geq -\dfrac5{21} \}$
		\item $\{ x \in \R \text{ tq. } 2 \leq 2x + 1 \leq 10 \}$
	\end{enumerate}
	\end{multicols}
}{

}

\subsection*{Plan cartésien\footnote{De René Descartes, mathématicien, physicien, et philosophe français (1596--1650).} }

\exe{
	Représenter dans un repère les sommets $U(2; 3)$, $V(-1;-2)$, $W(-2;2)$ d'un triangle.
	
	\begin{enumerate}
		\item Calculer le milieu de chaque côté du triangle et les représenter dans le repère.
		\item Calculer la longueur de chaque segment. Que dire du triangle ?
	\end{enumerate}
}{

}

\exe{\label{ex:2}
	Représenter dans un repère l'origine $O$ ainsi que les sommets $E(4; 2)$, $W( 1 ; 3)$, et $N = E + W$.
	\begin{enumerate}
		\item Calculer le milieu des segments $[ON]$ et $[EW]$. Que dire du quadrilatère ${OWNE}$ ?
		\item Calculer la longueur des segments $[OE], [EN], [NW],$ et $[NO]$.
		\item Représenter dans un repère les quadrilatères de sommets $O$, $\kappa E$, $\kappa W$, et $\kappa N$ pour $\kappa = -\dfrac12$ et $\kappa = \dfrac32$.
	\end{enumerate}
}{}

\exe{
	Considérons les points $a(0;-2), b(-1;2)$ ainsi que les points $c(3; -x)$ et $d(3; 3x - 2)$ dépendant d'un réel $x\in\R$.
	
	\begin{enumerate}
		\item Pour quel réel $x\in\R$ les points $c$ et $d$ sont-ils confondus ?
		\item Donner l'intervalle des réels $x\in\R$ tels que le point $d$ est au-dessus du point $c$ graphiquement.
		\item Lorsque $d$ est au-dessus de $c$, pour quel réel $x\in\R$ le quadrilatère $abcd$ est-il un parallélogramme ?
	\end{enumerate}
}{}

\exe{
	Soient $A(-1; 3)$, $B(3; 0)$, et $x\in\R$ un réel. Posons $\tilde{A} = xA$ et $\tilde{B} = xB$.
	
	\begin{enumerate}
		\item Donner l'ensemble des $x\in\R$ tels que la longueur du segment $[\tilde{A} \tilde{B}]$ est égale à $15$.
		\item Représenter les points $\tilde{A}$ et $\tilde{B}$ dans un repère pour chacun des $x$ trouvés.
	\end{enumerate}
}

\newpage
\subsection*{Exercices supplémentaires}

\exe{
	Calculer le milieu et la longueur de chaque intervalle suivant.
	\begin{multicols}{2}
	\begin{enumerate}
		\item $[-4 ; 12]$
		\item $]{-7}{,}7 ; -7{,}6[$
		\item $[-\dfrac49 ; \dfrac23 [$
		\item $\{ x \in \R \text{ tq. } -\dfrac{8}{17} > x \geq -\dfrac9{17} \}$
		\item $\{ x \in \R \text{ tq. } \dfrac{9}{10} \leq 2x + 1 \leq \dfrac{5}{2} \}$
	\end{enumerate}
	\end{multicols}
}

\exe{
	Soient $A, B$ deux points distincts et non nuls du plan.
	
	Montrer que, pour tout réel $k \in \R$ non nul, le quadrilatère dont les sommets sont $O, kA, kB$, et $kA + kB$ est un parallélogramme.
}{

}

\exe{
	Soient $A$ et $B$ deux points du plan, et $\alpha \in \R$ un réel.
	Posons $A' = \alpha A$ et $B' = \alpha B$.
	
	Exprimer la longueur du segment $[A'B']$ en fonction de la longueur du segment $[AB]$.
}


\end{document}
% DYSLEXIA SWITCH
\newif\ifdys
		
				% ENABLE or DISABLE font change
				% use XeLaTeX if true
				\dystrue
				\dysfalse


\ifdys

\documentclass[a4paper, 14pt]{extarticle}
\usepackage{amsmath,amsfonts,amsthm,amssymb,mathtools}

\tracinglostchars=3 % Report an error if a font does not have a symbol.
\usepackage{fontspec}
\usepackage{unicode-math}
\defaultfontfeatures{ Ligatures=TeX,
                      Scale=MatchUppercase }

\setmainfont{OpenDyslexic}[Scale=1.0]
\setmathfont{Fira Math} % Or maybe try KPMath-Sans?
\setmathfont{OpenDyslexic Italic}[range=it/{Latin,latin}]
\setmathfont{OpenDyslexic}[range=up/{Latin,latin,num}]

\else

\documentclass[a4paper, 14pt]{extarticle}
\usepackage{amsmath,amsfonts,amsthm,amssymb,mathtools}

\fi


\usepackage[french]{babel}
\usepackage[
a4paper,
margin=2cm,
nomarginpar,% We don't want any margin paragraphs
]{geometry}
\usepackage{fancyhdr}
\usepackage{array}


\usepackage{multicol, enumerate}
\usepackage{amsmath,amsfonts,amsthm,amssymb,mathtools}
\newcolumntype{P}[1]{>{\centering\arraybackslash}p{#1}}


\usepackage{stackengine}
\newcommand\xrowht[2][0]{\addstackgap[.5\dimexpr#2\relax]{\vphantom{#1}}}

% theorems

\theoremstyle{plain}
\newtheorem{theorem}{Th\'eor\`eme}
\newtheorem*{sol}{Solution}
\theoremstyle{definition}
\newtheorem{ex}{Exercice}
\newtheorem*{definition}{Définition}

% corps
\newcommand{\C}{\mathbb{C}}
\newcommand{\R}{\mathbb{R}}
\newcommand{\Rnn}{\mathbb{R}^{2n}}
\newcommand{\Z}{\mathbb{Z}}
\newcommand{\N}{\mathbb{N}}
\newcommand{\Q}{\mathbb{Q}}

% domain
\newcommand{\D}{\mathbb{D}}

% plots
\usepackage{pgfplots}


% date
\usepackage{advdate}
\AdvanceDate[0]

% SOLUTION SWITCH
\newif\ifsolutions
				\solutionstrue
				%\solutionsfalse

\ifsolutions
	\newcommand{\exe}[2]{
		\begin{ex} #1  \end{ex}
		\begin{sol} #2 \end{sol}
	}
\else
	\newcommand{\exe}[2]{
		\begin{ex} #1  \end{ex}
	}
	
\fi

\begin{document}
\pagestyle{fancy}
\fancyhead[L]{Seconde 13}
\fancyhead[C]{\textbf{Unions et intersections \ifsolutions -- Solutions \fi}}
\fancyhead[R]{\today}

\begin{center}
	\begin{tikzpicture}
		
		% real line
		\draw[<->, thick] (-7,0) node[below] {$-\infty$} -- (7,0) node[below] {$+\infty$};
		
		\foreach \x in {-3,...,3}
			\draw[-] (1.3*\x,-.1) node[below] {$\x$} -- (1.3*\x,.1) node{};
		
	
	\end{tikzpicture}
\end{center}

\begin{definition}[Intersection, union]
	Pour deux ensembles $E, F$ on définit les ensembles suivants.
		\begin{enumerate}
			\item $E \cap F$ : l'intersection des deux ensembles.
			
			Un élément appartient à $E \cap F$ dès qu'il appartient à $E$ \textbf{et} à $F$.
			\item $E \cup F$ : l'union des deux ensembles.
			
			Un élément appartient à $E \cup F$ dès qu'il appartient à $E$ \textbf{ou} à $F$.
		\end{enumerate}
\end{definition}

\exe{
	Écrire les ensembles finis suivants.
	\begin{multicols}{2}
	\begin{enumerate}
		\item $\mathcal{D}_7 \cap \mathcal{D}_8$
		\item $\mathcal{D}_4 \cup \mathcal{D}_6$
		\item $\{1; 3; -4; -3\} \cap \{ -3 ; 4 ; 10\}$
		\item $\{1 ; 3 ; -4 ; -3 \} \cup \{ -3 ; 4 ; 10\}$
	\end{enumerate}
	\end{multicols}
}{

	\begin{multicols}{2}
	\begin{enumerate}
		\item $\{1\}$.
		\item $\{1; 2\}$
		\item $\{-3 ; 4\}$
		\item $\{-4 ; -3; 1; 3 ; 4 ; 10 \}$
	\end{enumerate}
	\end{multicols}

}

\exe{
	Écrire les intervalles suivants sous forme d'ensembles.
	\begin{enumerate}[i)]\itemsep10pt
		\item $]{-}4 ; {+}\infty [ = \{ x \in \R \ \text{tq.} \ x > -4 \}$
		\item $[5 ; {+}\infty [ = $
		\item $]{-}\infty ; -\pi ] = $
		\item $]{-}\infty ; -2[ \cap [-12 ; {+}\infty[ = $
		\item $]{-} \infty ; 5] \cap ]-8 ; {+}\infty [ = $
	\end{enumerate}
}{


	\begin{enumerate}[i)]
		\item $]{-}4 ; {+}\infty [ = \{ x \in \R \ \text{tq.} \ x > -4 \}$
		\item $[5 ; {+}\infty [ = \{ x \in \R \ \text{tq.} \ x \geq 5 \}$
		\item $]{-}\infty ; -\pi ] =  \{ x \in \R \ \text{tq.} \ \leq -\pi \}$
		\item $]{-}\infty ; -2[ \cap [-12 ; {+}\infty[ = \{ x \in \R \ \text{tq.} \ -12 \leq x < -2 \}$
		\item $]{-} \infty ; 5] \cap ]-8 ; {+}\infty [ = \{ x \in \R \ \text{tq.} \ -8 < x \leq 5 \}$
	\end{enumerate}

}

\exe{
	Pour chaque couples d'intervalles $I,J$, exprimer $I \cap J$ et $I \cup J$ sous forme d'intervalle.
	%\begin{multicols}{2}
	\begin{enumerate}\itemsep10pt
		\item $I = [-4 ; 2]$, et $J = [-1 ; 6[$
		\item $ I = ] {-}\infty ; 3 [$, et $J =[-2 ; {+}\infty[$ 
		\item $I = ]{-}\infty ; 7]$, et $J =]{-}\infty ; 4]$
		\item $I = ]{-}\infty ; -1]$, et $J = [-1 ; {+}\infty [$
		\item $I = [3 ; {+}\infty [$, et $J = [-5 ; {+}\infty [$
		\item $I = \{ x \in \R \ \text{tq. } x > 3 \}$, et $J = \{ x \in \R \ \text{tq. } x \leq 9 \}$
		\item $I = \{ x \in \R \ \text{tq. } 2 \leq x \leq 12 \}$, et $J = \{ x \in \R \ \text{tq. } 10 > x \geq -1 \}$
	\end{enumerate}
	%\end{multicols}
}{

	\begin{enumerate}\itemsep10pt
		\item $I\cap J = [{-}1; 2]$ et $I \cup J = [{-}4; 6[$
		\item $I\cap J = [{-}2; 3[$ et $I \cup J = \R$
		\item $I\cap J = ]{-}\infty; 4]$ et $I \cup J =  ]{-}\infty; 7]$
		\item $I\cap J = \{-1\} $ et $I \cup J = \R$
		\item $I\cap J [3; {+}\infty[= $ et $I \cup J = [{-}5; {+}\infty[$
		\item $I\cap J = ]3;9]$ et $I \cup J = \R$
		\item $I\cap J = [2 ; 10[$ et $I \cup J = [{-}1; 12]$
	\end{enumerate}


}

\end{document}
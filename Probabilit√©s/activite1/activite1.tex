\documentclass[12pt]{extarticle}
\usepackage[french]{babel}
\usepackage[
a5paper,
landscape,
margin=2cm,
nomarginpar,% We don't want any margin paragraphs
]{geometry}
\usepackage{libertinus,libertinust1math}
\usepackage{amsmath,amsthm,amssymb,mathtools}
\usepackage{fancyhdr}

\theoremstyle{definition}
\newtheorem{ex}{Exercice}
\newcommand{\exe}[2]{
	\begin{ex} #1  \end{ex}
}

% 2xa5 on a4
\usepackage{pgfpages}                                 % <— load the package
  \pgfpagesuselayout{2 on 1}[a4paper,border shrink=5mm] % <— set options

\usepackage{atbegshi}
  % duplicate the content at shipout time
  \AtBeginShipout{
    \pgfpagesshipoutlogicalpage{1}\copy\AtBeginShipoutBox
    \pgfpagesshipoutlogicalpage{2}\box\AtBeginShipoutBox
    \pgfshipoutphysicalpage
  }



\begin{document}
\pagestyle{fancy}
\fancyhead[L]{Seconde 13}
\fancyhead[C]{\textbf{Probabilités 0}}
\fancyhead[R]{\today}

\exe{
	On lance une pièce de monnaie bien équilibrée.
	\begin{enumerate}
		\item Décrire les issues possibles de l'expérience.
		\item Donner la probabilité de chacune des issues.
		\item Après $10~000$ lancers, combien de fois \og pile \fg ~peut-on s'attendre à avoir eu ?
	\end{enumerate}
}{}

\exe{
	On lance un D$6$ équilibré, dé à $6$ faces, puis on lit le numéro de la face du dessus.
	\begin{enumerate}
		\item Décrire les issues possibles de l'expérience.
		\item Donner la probabilité de chacune des issues.
		\item Après $60~000$ lancers, combien de $6$ peut-on s'attendre à avoir eu ?
		\item Donner la probabilité d'obtenir un nombre pair.
		\item Donner la probabilité d'obtenir un multiple de $3$.
	\end{enumerate}
}{}

\exe{
	Une urne contient $3$ boules rouges et $2$ boules noires. On tire une boule au hasard et on note sa couleur.
	\begin{enumerate}
		\item Décrire les issues possibles de l'expérience.
		\item Donner la probabilité de chacune des issues.
		\item Après $50~000$ tirages avec remise, combien de boules noires peut-on s'attendre à avoir eu ?
	\end{enumerate}
}{}

\end{document}

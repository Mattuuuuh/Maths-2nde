%!TEX encoding = UTF8
%!TEX root =notes.tex


%%%%%%%%%%%%%%%%%%%%%%%%%%%%%%%%%
% PACKAGE IMPORTS
%%%%%%%%%%%%%%%%%%%%%%%%%%%%%%%%%


\usepackage[french]{babel}

\usepackage[tmargin=2cm,rmargin=1in,lmargin=1in,margin=0.85in,bmargin=2cm,footskip=.2in]{geometry}
\usepackage{amsmath,amsfonts,amsthm,amssymb,mathtools}
\usepackage[varbb]{newpxmath}
\usepackage{xfrac}
\usepackage[makeroom]{cancel}
\usepackage{mathtools}
\usepackage{bookmark}
\usepackage{enumitem}
\usepackage{hyperref,theoremref}
\hypersetup{
	pdftitle={Assignment},
	colorlinks=true, linkcolor=doc!90,
	bookmarksnumbered=true,
	bookmarksopen=true
}
\usepackage[most,many,breakable]{tcolorbox}
\usepackage{xcolor}
\usepackage{varwidth}
\usepackage{varwidth}
\usepackage{etoolbox}
%\usepackage{authblk}
\usepackage{nameref}
\usepackage{multicol,array}
\usepackage{tikz-cd}
\usepackage[ruled,vlined,linesnumbered]{algorithm2e}
\usepackage{comment} % enables the use of multi-line comments (\ifx \fi) 
\usepackage{import}
\usepackage{xifthen}
\usepackage{pdfpages}
\usepackage{transparent}


\newcommand\mycommfont[1]{\footnotesize\ttfamily\textcolor{blue}{#1}}
\SetCommentSty{mycommfont}
\newcommand{\incfig}[1]{%
    \def\svgwidth{\columnwidth}
    \import{./figures/}{#1.pdf_tex}
}

\usepackage{tikzsymbols}
%\renewcommand\qedsymbol{$\Laughey$}


%\usepackage{import}
%\usepackage{xifthen}
%\usepackage{pdfpages}
%\usepackage{transparent}


%%%%%%%%%%%%%%%%%%%%%%%%%%%%%%
% SELF MADE COLORS
%%%%%%%%%%%%%%%%%%%%%%%%%%%%%%



\definecolor{myg}{RGB}{56, 140, 70}
\definecolor{myb}{RGB}{45, 111, 177}
\definecolor{myr}{RGB}{199, 68, 64}
\definecolor{mytheorembg}{HTML}{F2F2F9}
\definecolor{mytheoremfr}{HTML}{00007B}
\definecolor{mylenmabg}{HTML}{FFFAF8}
\definecolor{mylenmafr}{HTML}{983b0f}
\definecolor{mypropbg}{HTML}{f2fbfc}
\definecolor{mypropfr}{HTML}{191971}
\definecolor{myexamplebg}{HTML}{F2FBF8}
\definecolor{myexamplefr}{HTML}{88D6D1}
\definecolor{myexampleti}{HTML}{2A7F7F}
\definecolor{mydefinitbg}{HTML}{E5E5FF}
\definecolor{mydefinitfr}{HTML}{3F3FA3}
\definecolor{notesgreen}{RGB}{0,162,0}
\definecolor{myp}{RGB}{197, 92, 212}
\definecolor{mygr}{HTML}{2C3338}
\definecolor{myred}{RGB}{127,0,0}
\definecolor{myyellow}{RGB}{169,121,69}
\definecolor{myexercisebg}{HTML}{F2FBF8}
\definecolor{myexercisefg}{HTML}{88D6D1}


%%%%%%%%%%%%%%%%%%%%%%%%%%%%
% TCOLORBOX SETUPS
%%%%%%%%%%%%%%%%%%%%%%%%%%%%

\setlength{\parindent}{1cm}
%================================
% THEOREM BOX
%================================

\tcbuselibrary{theorems,skins,hooks}
\newtcbtheorem[number within=chapter]{Theorem}{Théorème}
{%
	enhanced,
	breakable,
	colback = mytheorembg,
	frame hidden,
	boxrule = 0sp,
	borderline west = {2pt}{0pt}{mytheoremfr},
	sharp corners,
	detach title,
	before upper = \tcbtitle\par\smallskip,
	coltitle = mytheoremfr,
	fonttitle = \bfseries\sffamily,
	description font = \mdseries,
	separator sign none,
	segmentation style={solid, mytheoremfr},
}
{th}


\tcbuselibrary{theorems,skins,hooks}
\newtcolorbox{Theoremcon}
{%
	enhanced
	,breakable
	,colback = mytheorembg
	,frame hidden
	,boxrule = 0sp
	,borderline west = {2pt}{0pt}{mytheoremfr}
	,sharp corners
	,description font = \mdseries
	,separator sign none
}

%================================
% Corollery
%================================
\tcbuselibrary{theorems,skins,hooks}
\newtcbtheorem[use counter=tcb@cnt@Theorem]{Corollary}{Corollaire}
{%
	enhanced
	,breakable
	,colback = myp!10
	,frame hidden
	,boxrule = 0sp
	,borderline west = {2pt}{0pt}{myp!85!black}
	,sharp corners
	,detach title
	,before upper = \tcbtitle\par\smallskip
	,coltitle = myp!85!black
	,fonttitle = \bfseries\sffamily
	,description font = \mdseries
	,separator sign none
	,segmentation style={solid, myp!85!black}
}
{th}

%================================
% LENMA
%================================

\tcbuselibrary{theorems,skins,hooks}
\newtcbtheorem[use counter=tcb@cnt@Theorem]{Lemma}{Lemme}
{%
	enhanced,
	breakable,
	colback = mylenmabg,
	frame hidden,
	boxrule = 0sp,
	borderline west = {2pt}{0pt}{mylenmafr},
	sharp corners,
	detach title,
	before upper = \tcbtitle\par\smallskip,
	coltitle = mylenmafr,
	fonttitle = \bfseries\sffamily,
	description font = \mdseries,
	separator sign none,
	segmentation style={solid, mylenmafr},
}
{th}


%================================
% PROPOSITION
%================================

\tcbuselibrary{theorems,skins,hooks}
\newtcbtheorem[use counter=tcb@cnt@Theorem]{Prop}{Proposition}
{%
	enhanced,
	breakable,
	colback = mypropbg,
	frame hidden,
	boxrule = 0sp,
	borderline west = {2pt}{0pt}{mypropfr},
	sharp corners,
	detach title,
	before upper = \tcbtitle\par\smallskip,
	coltitle = mypropfr,
	fonttitle = \bfseries\sffamily,
	description font = \mdseries,
	separator sign none,
	segmentation style={solid, mypropfr},
}
{th}


%================================
% CLAIM
%================================

\tcbuselibrary{theorems,skins,hooks}
\newtcbtheorem[use counter=tcb@cnt@Theorem]{claim}{Claim}
{%
	enhanced
	,breakable
	,colback = myg!10
	,frame hidden
	,boxrule = 0sp
	,borderline west = {2pt}{0pt}{myg}
	,sharp corners
	,detach title
	,before upper = \tcbtitle\par\smallskip
	,coltitle = myg!85!black
	,fonttitle = \bfseries\sffamily
	,description font = \mdseries
	,separator sign none
	,segmentation style={solid, myg!85!black}
}
{th}



%================================
% Exercise
%================================

\tcbuselibrary{theorems,skins,hooks}
\newtcbtheorem[use counter=tcb@cnt@Theorem]{Exercise}{Exercice}
{%
	enhanced,
	breakable,
	colback = myexercisebg,
	frame hidden,
	boxrule = 0sp,
	borderline west = {2pt}{0pt}{myexercisefg},
	sharp corners,
	detach title,
	before upper = \tcbtitle\par\smallskip,
	coltitle = myexercisefg,
	fonttitle = \bfseries\sffamily,
	description font = \mdseries,
	separator sign none,
	segmentation style={solid, myexercisefg},
}
{th}

%================================
% EXAMPLE BOX
%================================

\newtcbtheorem[use counter=tcb@cnt@Theorem]{Example}{Exemple}
{%
	colback = myexamplebg
	,breakable
	,colframe = myexamplefr
	,coltitle = myexampleti
	,boxrule = 1pt
	,sharp corners
	,detach title
	,before upper=\tcbtitle\par\smallskip
	,fonttitle = \bfseries
	,description font = \mdseries
	,separator sign none
	,description delimiters parenthesis
}
{ex}

%================================
% DEFINITION BOX
%================================

\newtcbtheorem[use counter=tcb@cnt@Theorem]{Definition}{Définition}{enhanced,
	before skip=2mm,after skip=2mm, colback=red!5,colframe=red!80!black,boxrule=0.5mm,
	attach boxed title to top left={xshift=1cm,yshift*=1mm-\tcboxedtitleheight}, varwidth boxed title*=-3cm,
	boxed title style={frame code={
					\path[fill=tcbcolback]
					([yshift=-1mm,xshift=-1mm]frame.north west)
					arc[start angle=0,end angle=180,radius=1mm]
					([yshift=-1mm,xshift=1mm]frame.north east)
					arc[start angle=180,end angle=0,radius=1mm];
					\path[left color=tcbcolback!60!black,right color=tcbcolback!60!black,
						middle color=tcbcolback!80!black]
					([xshift=-2mm]frame.north west) -- ([xshift=2mm]frame.north east)
					[rounded corners=1mm]-- ([xshift=1mm,yshift=-1mm]frame.north east)
					-- (frame.south east) -- (frame.south west)
					-- ([xshift=-1mm,yshift=-1mm]frame.north west)
					[sharp corners]-- cycle;
				},interior engine=empty,
		},
	fonttitle=\bfseries,
	title={#2},#1}{def}

%================================
% Solution BOX
%================================

\makeatletter
\newtcbtheorem[use counter=tcb@cnt@Theorem]{question}{Question}{enhanced,
	breakable,
	colback=white,
	colframe=myb!80!black,
	attach boxed title to top left={yshift*=-\tcboxedtitleheight},
	fonttitle=\bfseries,
	title={#2},
	boxed title size=title,
	boxed title style={%
			sharp corners,
			rounded corners=northwest,
			colback=tcbcolframe,
			boxrule=0pt,
		},
	underlay boxed title={%
			\path[fill=tcbcolframe] (title.south west)--(title.south east)
			to[out=0, in=180] ([xshift=5mm]title.east)--
			(title.center-|frame.east)
			[rounded corners=\kvtcb@arc] |-
			(frame.north) -| cycle;
		},
	#1
}{def}
\makeatother

%================================
% SOLUTION BOX
%================================

\makeatletter
\newtcolorbox{solution}{enhanced,
	breakable,
	colback=white,
	colframe=myg!80!black,
	attach boxed title to top left={yshift*=-\tcboxedtitleheight},
	title=Solution,
	boxed title size=title,
	boxed title style={%
			sharp corners,
			rounded corners=northwest,
			colback=tcbcolframe,
			boxrule=0pt,
		},
	underlay boxed title={%
			\path[fill=tcbcolframe] (title.south west)--(title.south east)
			to[out=0, in=180] ([xshift=5mm]title.east)--
			(title.center-|frame.east)
			[rounded corners=\kvtcb@arc] |-
			(frame.north) -| cycle;
		},
}
\makeatother

%================================
% Question BOX
%================================

\makeatletter
\newtcbtheorem[use counter=tcb@cnt@Theorem]{qstion}{Question}{enhanced,
	breakable,
	colback=white,
	colframe=mygr,
	attach boxed title to top left={yshift*=-\tcboxedtitleheight},
	fonttitle=\bfseries,
	title={#2},
	boxed title size=title,
	boxed title style={%
			sharp corners,
			rounded corners=northwest,
			colback=tcbcolframe,
			boxrule=0pt,
		},
	underlay boxed title={%
			\path[fill=tcbcolframe] (title.south west)--(title.south east)
			to[out=0, in=180] ([xshift=5mm]title.east)--
			(title.center-|frame.east)
			[rounded corners=\kvtcb@arc] |-
			(frame.north) -| cycle;
		},
	#1
}{def}
\makeatother

\newtcbtheorem[number within=chapter]{wconc}{Wrong Concept}{
	breakable,
	enhanced,
	colback=white,
	colframe=myr,
	arc=0pt,
	outer arc=0pt,
	fonttitle=\bfseries\sffamily\large,
	colbacktitle=myr,
	attach boxed title to top left={},
	boxed title style={
			enhanced,
			skin=enhancedfirst jigsaw,
			arc=3pt,
			bottom=0pt,
			interior style={fill=myr}
		},
	#1
}{def}



%================================
% NOTE BOX
%================================

\usetikzlibrary{arrows,calc,shadows.blur}
\tcbuselibrary{skins}
\newtcolorbox{note}[1][]{%
	enhanced jigsaw,
	colback=gray!20!white,%
	colframe=gray!80!black,
	size=small,
	boxrule=1pt,
	title=\colorbox{white!100}{\textbf{ Remarque }},
	halign title=flush center,
	coltitle=black,
	breakable,
	drop shadow=black!50!white,
	attach boxed title to top left={xshift=1cm,yshift=-\tcboxedtitleheight/2,yshifttext=-\tcboxedtitleheight/2},
	minipage boxed title=2.6cm,
	boxed title style={%
			colback=white,
			size=fbox,
			boxrule=1pt,
			boxsep=2pt,
			underlay={%
					\coordinate (dotA) at ($(interior.west) + (-0.5pt,0)$);
					\coordinate (dotB) at ($(interior.east) + (0.5pt,0)$);
					\begin{scope}
						\clip (interior.north west) rectangle ([xshift=3ex]interior.east);
						\filldraw [white, blur shadow={shadow opacity=60, shadow yshift=-.75ex}, rounded corners=2pt] (interior.north west) rectangle (interior.south east);
					\end{scope}
					\begin{scope}[gray!80!black]
						\fill (dotA) circle (2pt);
						\fill (dotB) circle (2pt);
					\end{scope}
				},
		},
	#1,
}

%================================
% STRATÉGIE BOX
%================================

\usetikzlibrary{arrows,calc,shadows.blur}
\tcbuselibrary{skins}
\newtcolorbox{strategy}[1][]{%
	enhanced jigsaw,
	colback=myb!20!white,%
	colframe=gray!80!black,
	size=small,
	boxrule=1pt,
	title=\colorbox{white!100}{\textbf{ Stratégie }},
	halign title=flush center,
	coltitle=black,
	breakable,
	drop shadow=black!50!white,
	attach boxed title to top left={xshift=1cm,yshift=-\tcboxedtitleheight/2,yshifttext=-\tcboxedtitleheight/2},
	minipage boxed title=2.5cm,
	boxed title style={%
			colback=white,
			size=fbox,
			boxrule=1pt,
			boxsep=2pt,
			underlay={%
					\coordinate (dotA) at ($(interior.west) + (-0.5pt,0)$);
					\coordinate (dotB) at ($(interior.east) + (0.5pt,0)$);
					\begin{scope}
						\clip (interior.north west) rectangle ([xshift=3ex]interior.east);
						\filldraw [white, blur shadow={shadow opacity=60, shadow yshift=-.75ex}, rounded corners=2pt] (interior.north west) rectangle (interior.south east);
					\end{scope}
					\begin{scope}[gray!80!black]
						\fill (dotA) circle (2pt);
						\fill (dotB) circle (2pt);
					\end{scope}
				},
		},
	#1,
}

%================================
% MÉTHODE BOX
%================================

\usetikzlibrary{arrows,calc,shadows.blur}
\tcbuselibrary{skins}
\newtcolorbox{methode}[1][]{%
	enhanced jigsaw,
	colback=white,%
	colframe=gray!80!black,
	size=small,
	boxrule=1pt,
	title=\textbf{Méthode},
	halign title=flush center,
	coltitle=black,
	breakable,
	drop shadow=black!50!white,
	attach boxed title to top left={xshift=1cm,yshift=-\tcboxedtitleheight/2,yshifttext=-\tcboxedtitleheight/2},
	minipage boxed title=2.5cm,
	boxed title style={%
			colback=white,
			size=fbox,
			boxrule=1pt,
			boxsep=2pt,
			underlay={%
					\coordinate (dotA) at ($(interior.west) + (-0.5pt,0)$);
					\coordinate (dotB) at ($(interior.east) + (0.5pt,0)$);
					\begin{scope}
						\clip (interior.north west) rectangle ([xshift=3ex]interior.east);
						\filldraw [white, blur shadow={shadow opacity=60, shadow yshift=-.75ex}, rounded corners=2pt] (interior.north west) rectangle (interior.south east);
					\end{scope}
					\begin{scope}[gray!80!black]
						\fill (dotA) circle (2pt);
						\fill (dotB) circle (2pt);
					\end{scope}
				},
		},
	#1,
}

%%%%%%%%%%%%%%%%%%%%%%%%%%%%%%%%%%%%%%%%%%%
% TABLE OF CONTENTS
%%%%%%%%%%%%%%%%%%%%%%%%%%%%%%%%%%%%%%%%%%%

\usepackage{tikz}

\definecolor{doc}{RGB}{0,60,110}
\usepackage{titletoc}
\contentsmargin{0cm}
\titlecontents{chapter}[3.7pc]
{\addvspace{30pt}%
	\begin{tikzpicture}[remember picture, overlay]%
		\draw[fill=doc!60,draw=doc!60] (-7,-.1) rectangle (-0.2,.6);%
		\pgftext[left,x=-3.5cm,y=0.2cm]{\color{white}\Large\sc\bfseries Chapitre\ \thecontentslabel};%
	\end{tikzpicture}\color{doc!60}\large\sc\bfseries}%
{}
{}
{\;\titlerule\;\large\sc\bfseries Page \thecontentspage
	\begin{tikzpicture}[remember picture, overlay]
		\draw[fill=doc!60,draw=doc!60] (2pt,0) rectangle (4,0.1pt);
	\end{tikzpicture}}%
\titlecontents{section}[3.7pc]
{\addvspace{2pt}}
{\contentslabel[\thecontentslabel]{2pc}}
{}
{\hfill\small \thecontentspage}
[]
\titlecontents*{subsection}[3.7pc]
{\addvspace{-1pt}\small}
{}
{}
{\ --- \small\thecontentspage}
[ \textbullet\ ][]

\makeatletter
\renewcommand{\tableofcontents}{%
	\chapter*{%
	  \vspace*{-20\p@}%
	  \begin{tikzpicture}[remember picture, overlay]%
		  \pgftext[right,x=15cm,y=0.2cm]{\color{doc!60}\Huge\sc\bfseries \contentsname};%
		  \draw[fill=doc!60,draw=doc!60] (13,-.75) rectangle (20,1);%
		  \clip (13,-.75) rectangle (20,1);
		  \pgftext[right,x=15cm,y=0.2cm]{\color{white}\Huge\sc\bfseries \contentsname};%
	  \end{tikzpicture}}%
	\@starttoc{toc}}
\makeatother


%%%%%%%%%%%%%%%%%%%%%%%%%%%%%%%%%%%%%%%%%%%
% MINTED FOR PYTHON ALGORITHMS
%%%%%%%%%%%%%%%%%%%%%%%%%%%%%%%%%%%%%%%%%%%

\usepackage{tcolorbox}
\tcbuselibrary{minted,breakable,xparse,skins}
\definecolor{bg}{gray}{0.95}
\DeclareTCBListing{mintedbox}{O{}m!O{}}{%
  breakable=true,
  listing engine=minted,
  listing only,
  minted language=#2,
  minted style=default,
  minted options={%
    linenos,
    gobble=0,
    breaklines=true,
    breakafter=,,
    fontsize=\small,
    numbersep=8pt,
    #1},
  boxsep=0pt,
  left skip=0pt,
  right skip=0pt,
  left=25pt,
  right=0pt,
  top=3pt,
  bottom=3pt,
  arc=5pt,
  leftrule=0pt,
  rightrule=0pt,
  bottomrule=2pt,
  toprule=2pt,
  colback=bg,
  colframe=orange!70,
  enhanced,
  overlay={%
    \begin{tcbclipinterior}
    \fill[orange!20!white] (frame.south west) rectangle ([xshift=20pt]frame.north west);
    \end{tcbclipinterior}},
  #3}
  
  
 % for braces
\usetikzlibrary{decorations.pathreplacing}


\SetDate[13/01/2026]

\begin{document}
\pagestyle{fancy}
\fancyhead[L]{Seconde}
\fancyhead[C]{\textbf{Approfondissements --- Fonction carré}}
\fancyhead[R]{\today}

%%%%%%%%%%%%%%%%%%%
%%%%%%%%%%%%%%%%%%%
%%%%%%%%%%%%%%%%%%%

\exe{, difficulty=1}{
	Construire, en posant des nombres réels $a, b, c\in\R$, une fonction de la forme
			\[ f(x) = ax^2 + bx + c \]
	telle que $-3$ soit le maximum de $f$ sur $\R$, atteint en $x^\star = -1$.
}{exe:1}{
	On souhaite une fonction $f$ dont le maximum est $-3$.
	On considère donc une fonction $f(x)$ qui soustrait toujours quelque chose de positif à $-3$, par exemple
		\[ F(x) = -3 - x^2. \]
	Le problème est que le maximum est n'est pas atteint en $-1$ mais en $0$ ici, il faut donc mettre au carré une expression qui s'annule en $-1$.
	Comme vu en cours, $x+1$ est une telle fonction, et 
		\[ f(x) = -3 - (x+1)^2 = 3 - (x^2 + 2x + 1) = 3 - x^2 - 2x - 1 = -x^2 - 2x + 2 \]
	fonctionne très bien !
	Notons qu'on aurait pû aussi choisir
		\begin{align*}
			g(x) = -3 -2(x+1)^2,&& \text{ ou }&& h(x) = -3 - 160(x+1)^2.
		\end{align*}
	La plus simple étant $f(x) = -x^2 - 2x + 2$, on pose $a=-1, b=-2,$ et $c=2$.
}


\exe{}{
	Grapher la fonction carré $f(x)=1+x^2$ sur le domaine $\D = [-4 ; 3]$.
	
	En déduire que $f$ admet son minimum 1 en $x^\star = 0$ et son maximum 17 en $x^\star=-4$.
}{exe:10}{
	 Sur GeoGebra, entrer \texttt{y=x**2} puis \texttt{x=-4} et \texttt{x=3} pour définir les bornes (droites verticales).
}

\exe{}{
	Soit $f(x) = 13 + 4(x-3)^2$. 
	Montrer que $f$ atteint son minimum en $x^\star = 3$.
	Montrer que si $\frac{a+b}2 = 3$, alors $f(a) = f(b)$.
	Graphiquement, si $a$ et $b$ sont symétriques l'un de l'autre par rapport à 3, et donc la droite verticale $x=3$ est un axe de symétre de $\C_f$ (à vérifier sur GeoGebra).
}{exe:11}{
	Si $\frac{a+b}2 = 3$, alors $a+b = 6$, et donc $b=6-a$.
	Calculons donc $f(b) = f(6-a)$ et voyons qu'on obtient bien $f(a)$.
	
	\begin{align*}
		f(6-a) &= 13 + 4\left[ (6-a)-3 \right]^2, \\
				&= 13 + 4 (3-a)^2, \\
				&= 13 + 4(a-3)^2 = f(a),
	\end{align*}
	où l'avant-dernière égalité découle du fait que le carré est pair et donc $(3-a)^2 = (a-3)^2$. 
}

\exe{, difficulty=1}{
	L'exercice  vise à montrer que plusieurs antécédents peuvent réaliser un extremum en général.
	Étudions la fonction $g$ définie algébriquement sur $\R$ par
		\[ g(x) =11 + \dfrac{1}{100}\left(\dfrac{x}5+3\right)^2(x-5)^2. \]
	\begin{enumerate}
		\item 
		Montrer que $\left(\dfrac{x}5+3\right)^2(x-5)^2 = \left[\left(\dfrac{x}5+3\right)(x-5)\right]^2$.
	
		\item
		En déduire l'extremum de $g$ ainsi que les antécédents qui le réalisent.
		
		\item 
		Calculer les images des antécédents trouvés pour vérifier qu'ils réalisent bien l'extremum.
	
		\item
		Grapher $\C_g$ à l'aide de Geogebra en lui donnant l'expression \texttt{y=11+1/100*(x/5+3)**2 * (x-5)**2} pour vérifier ses réponses (la double multiplication désigne la puissance en programmation).
	\end{enumerate}
}{exe:3}{
	$g$ est de la forme étudiée en classe avec $\frac1{100} \geq 0$, donc $g$ admet un minimum qui vaut 11.
	
	Le produit $\left(\dfrac{x}5+3\right)(x-5)$ est nul si et seulement si un des facteurs est nul.
	L'équation $\dfrac{x}5+3$ fournit le premier antécédent $x=-15$, et l'équation $x-5 = 0$ donne $x=5$.

	En conclusion, $g(x)$ atteint son minimum 11 pour $x\in \bigset{ -15 ; 5}$.
}


\exe{, difficulty=2}{
	Le but de l'exercice est de montrer que les fonctions du second degré vues au lycée ne peuvent pas admettre à la fois un maximum et un minimum sur $\R$ tout entier.

	Justifier qu'un polynôme du second degré
		\[ ax^2 + bx + c \]
	avec $a\neq0$ 
		\begin{enumerate}[label=\roman*)]
			\item 
			n'admet pas de maximum si $a > 0$
			\item
			n'admet pas de minimum si $a < 0$
		\end{enumerate}
	Ce résultat ce généralise pour les polynômes de degré quelconques ($ax^d + bx^{d-1} + ...$), selon le signe de $a$ et la parité de $d$.
	Si $d$ est pair, les règle sont les mêmes qu'avec $d=2$.
	Si $d$ est impair, $x^d$ n'admet ni minimum ni maximum (voir $d=1$ et $d=3$ pour se donner une intuition).
	En conclusion, \underline{la plus grande puissance de $x$ domine les autres}.
}{exe:minmax}{
	Quitte à multiplier l'expression par $-1$, qui transforme un maximum en un minimum (à vérifier !),
	supposons $a>0$.
	
	Si $ax^2 + bx + c \leq M$ pour tout $x\in\R$, alors
	$ ax^2 \leq M - bx -c$, et donc, pour $x>0$, diviser par $ax$ (qui est positif) donne
		\[ x \leq \dfrac{M}{ax} - \dfrac{b}{a} - \dfrac{c}{ax}, \]
	valable pour tout $x>0$.
	Cependant, par inspection, le membre de gauche, $x$, peut prendre des valeurs aussi grandes que voulues.
	Cependant, le membre de droite est borné supérieurement : diviser par $x$ de plus en plus grand diminue la valeur.
	En effet, $\frac{1}{x}$ s'approche de zéro lorsque $x$ grandit.
}



\exe{}{
	Montrer que $\min -f(x) = - \max f(x)$.
}{exe:minismax}{
	Montrer que l'argument du minimum de $-f(x)$ est égal à l'argument du maximum de $f(x)$ et conclure.
} 

\exe{}{
	Montrer que si $x^\star = \arg\max f(x)$, alors $g(x) = f(x+x^\star)$ atteint son maximum en 0.
	
	Graphiquement, nous décalons la courbe $\C_f$ vers la gauche ou la droite de sorte à fixer l'extremum en 0.
	C'est pour ça que les fonctions $2+x^2$ et $2+(x-3)^2$ sont les translatées l'une de l'autre (voir GeoGebra).
	
	Cette simple remarque permet par exemple de commencer une démonstration par « sans perte de généralité, $x^\star = 0$ », ce qui peut parfois simplifier les notations.
}{exe:argmaxtozero}{
	$g(0) = f(x^\star) \leq f(x + x^\star) = g(x)$ pour tout $x\in\R$.
} 


\exe{, difficulty=1}{
	L'exercice vise à montrer que certaines fonctions n'admettent pas de minimum alors qu'elles sont toujours supérieures à une certaine valeur.
	Considérons la fonction
		\[ f(x) = \dfrac{1}{1+x^2}. \]
	Remarquons que le dénominateur varie ; il faut donc faire attention à ce qu'il ne soit jamais nul.
	\begin{enumerate}
		\item
		Justifier que $f(x)$ est bien définie pour tout $x\in\R$.
		C'est-à-dire montrer que la fonction n'essaye jamais de diviser par zéro.
		\item
		Montrer rigoureusement que $f(x) \leq 1$ pour tout $x\in\R$ avec égalité uniquement en $x=0$.
		\item
		Montrer que $f(x) \geq c \geq 0$ ne peut être vrai pour tout $x\in\R$ que pour $c=0$.
		Le seul candidat possible de minimum est donc $0$.
		\item
		Montrer que $f$ n'admet pas de minimum sur $\R$.
	\end{enumerate}
	La plus grande borne inférieure est appelée l'\emph{infimum}.
	Il se distingue du minimum par le fait qu'il n'est pas nécessairement atteint.
}{exe:5}{
	\begin{enumerate}
		\item
		$1+x^2 \geq 1$ pour tout $x\in\R$.
		Le dénominateur n'est donc jamais nul.
		\item
		En divisant $1+x^2 \geq 1$ par $1+x^2$, qui est positif (et donc ne change pas le sens de l'inégalité), on obtient bien $f(x) \leq 1$ pour tout $x\in\R$.
		L'inégalité est égalité lorsque $1+x^2 = 1$, c'est-à-dire uniquement quand $x=0$.
		\item
		Supposons $f(x) \geq c > 0$ pour tout $x\in\R$.
		En multipliant par $(1+x^2)$ et en divisant par $c$, deux quantités positives (et donc qui ne changent pas le sens de l'inégalité), on obtient
			\[ \dfrac1c \geq 1+x^2 \]
		pour tout $x\in\R$.
		Or le membre de droite explose pour $x$ de plus en plus grand.
		Il ne peut donc pas être borné supérieurement pas une certaine constante (même très grande si $c$ est très petit).
		\item
		$f(x) \geq 0$ pour tout $x\in\R$ est le seul candidat possible de minimum, car n'importe quelle valeur $c$ strictement supérieure ne fonctionne pas.
		Or, $f(x)$ ne vaut jamais zéro, car $f(x) = 0$ est équivalente à $1=0$, en multipliant par $1+x^2$, quantité non nulle.
	\end{enumerate}
}

\exe{, difficulty=2}{
	Le but de l'exercice est d'exhiber une fonction qui admet à la fois un minimum et un maximum.

	 À l'aide de Geogebra, grapher la fonction
		\[ f(x) = \dfrac{1}{1+(x+1)^2} - \dfrac1{1+(x-1)^2}. \]
	Changer les numérateurs pour jouer avec la (non) parité de $f$.
	\begin{enumerate}
		\item
		Montrer que $f(-1) > 0$ et $f(1) < 0$.
		\item
		Montrer que $f$ est impaire, c'est-à-dire que $f(-x) = -f(x)$ pour tout $x\in\R$.
		\item
		Justifier que $f(x)$ s'approche de 0 lorsque $x$ devient très grand (positif ou négatif).
		\item
		En conclure qu'on peut restreindre la recherche d'un maximum et d'un minimum aux antécédents $x$ relativement petits, et donc que $f$ admet à la fois un minimum et un maximum.
	\end{enumerate}
	On utilise sans le nommer le \href{https://fr.wikipedia.org/wiki/Th\%C3\%A9or\%C3\%A8me\_des\_valeurs\_extr\%C3\%AAmes}{\emph{théorème des valeurs extrêmes}} qui nécessite l'hypothèse que $f$ soit continue (propriété admise au lycée).
}{exe:minandmax}{
	
	2. Exercice un peu \emph{boring} qui utilise que $(-x+1)^2 = (x-1)^2$ et que $(-x-1)^2 = (x+1)^2$ par parité de la fonction carré.

	4. Comme $f$ s'évanouit vers l'infini, $f(x)$ est toujours plus petite que $f(-1)$ et toujours plus grande que $f(-1)$ pour $x$ suffisamment grand (positif ou négatif).
	Les extrema ne sont donc pas atteints en l'infini, comme à l'exercice \ref{exe:5} : ils sont donc bien un maximum et un minimum.
}


\exe{, difficulty=3}{
	Considérons une fonction réelle $f$ sur $\bigset{0 ; 1 ; 2 ; \dots ; N}$.
	C'est-à-dire que $f(x)\in\R$ est uniquement définie en $x=0 ; 1 ; 2 ; \dots ; N$.
	
	On dit que $f$ est \emph{harmonique} dès que
		\[ f(x) = \dfrac{f(x-1) + f(x+1)}{2} \]
	pour tout $x \in \bigset{1 ; 2 ; 3 ; \dots ; N-1}$.
	
	Les quatre questions sont indépendantes.
	\begin{enumerate}
		\item 
		Montrer que la fonction $f(x) = 1$ est harmonique, puis donner son minimum et son maximum.
		\item 
		Montrer que la fonction $f(x) = 3x+5$ est harmonique, puis donner son minimum et son maximum.
		\item
		Montrer que si $f$ et $g$ sont harmoniques, alors $f+g$ est harmonique.
		L'opération $+$ entre fonctions se définit sans surprise par :
			\[ (f+g)(x) = f(x) + g(x). \]
		\item	
		Montrer que si $f$ est harmonique et qu'elle atteint son minimum ou son maximum en un certain $x^\star \in \bigset{1 ; 2 ; 3 ; \dots ; N-1}$, alors $f$ est constante.
	\end{enumerate}
}{exe:olympiade}{
	Exercice d'olympiades !
	Pour le point 4 : voir $\frac{f(x-1) + f(x+1)}{2}$ comme une moyenne.
}




%%%%%%%%%%%%%%%%%%%
%%%%%%%%%%%%%%%%%%%
%%%%%%%%%%%%%%%%%%%

\newpage
\fancyhead[C]{\textbf{Solutions}}
\shipoutAnswer

\end{document}


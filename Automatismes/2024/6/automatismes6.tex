\documentclass[14pt]{beamer}
\usepackage[french]{babel}


\usetheme{CambridgeUS}
\usecolortheme{rose}
\beamertemplatenavigationsymbolsempty

\usepackage{array}
\usepackage{amsmath,amsfonts,amsthm,amssymb,mathtools}
\newcolumntype{P}[1]{>{\centering\arraybackslash}p{#1}}


\usepackage{stackengine}
\newcommand\xrowht[2][0]{\addstackgap[.5\dimexpr#2\relax]{\vphantom{#1}}}


% corps
\newcommand{\C}{\mathbb{C}}
\newcommand{\R}{\mathbb{R}}
\newcommand{\Rnn}{\mathbb{R}^{2n}}
\newcommand{\Z}{\mathbb{Z}}
\newcommand{\N}{\mathbb{N}}
\newcommand{\Q}{\mathbb{Q}}

% domain
\newcommand{\D}{\mathbb{D}}


% date
\usepackage{advdate}
\AdvanceDate[1]

\begin{document}

\section{Automatismes}

\subsection{Inégalités}

\begin{frame}

\centering \huge
Automatismes

\end{frame}


\begin{frame}{1}
	Écrire l'ensemble suivant sous forme d'intervalle.
		\[ \{ x \in \R \text{ tq. } 7 \leq x < 18 \} \]
	
\end{frame}

\begin{frame}{2}

	Écrire l'ensemble suivant sous forme d'intervalle.
		\[ \left\{ x \in \R \text{ tq. } {-}7 \leq  5x - 12 \leq 8 \right\} \]

\end{frame}

\subsection{Milieux et longueurs}

\begin{frame}{3}

	Donner le milieu du segment $[AB]$ où
	
		\begin{align*}
			A = ({-}1 ; 4) &&  \text{ et } && B = (3 ; 7).
		\end{align*}


\end{frame}


\begin{frame}{4}

	Donner la longueur de l'intervalle $\left[ {-}6 ; 14 \right]$.

\end{frame}





\end{document}
\documentclass[14pt]{beamer}
\usepackage[french]{babel}

\usetheme{CambridgeUS}
\usecolortheme{rose}
\beamertemplatenavigationsymbolsempty


\usepackage{libertinus}
\usepackage{amsmath,amsfonts,amsthm,amssymb,mathtools}
\usepackage{array}
\newcolumntype{P}[1]{>{\centering\arraybackslash}p{#1}}


\usepackage{stackengine}
\newcommand\xrowht[2][0]{\addstackgap[.5\dimexpr#2\relax]{\vphantom{#1}}}


% corps
\usepackage{calrsfs}
\newcommand{\C}{\mathcal{C}}
\newcommand{\R}{\mathbb{R}}
\newcommand{\Rnn}{\mathbb{R}^{2n}}
\newcommand{\Z}{\mathbb{Z}}
\newcommand{\N}{\mathbb{N}}
\newcommand{\Q}{\mathbb{Q}}

% domain
\newcommand{\D}{\mathbb{D}}


% date
\usepackage{advdate}
\AdvanceDate[0]

%plots
\usepackage{pgfplots, subcaption}
\definecolor{myg}{RGB}{56, 140, 70}
\definecolor{myb}{RGB}{45, 111, 177}
\definecolor{myr}{RGB}{199, 68, 64}

%boxes
\usepackage[most]{tcolorbox}
\usepackage{multicol}

%icomma
\usepackage{icomma}

%https://osl.ugr.es/CTAN/macros/latex/contrib/tcolorbox/tcolorbox.pdf
\newtcolorbox{mybox}[3][]
{
  colframe = #2!25,
  colback  = #2!10,
  coltitle = #2!20!black,  
  halign title=flush center, 
  title    = {#3},
  #1,
}
\begin{document}

\section{Automatismes}

\begin{frame}

\centering \huge
Automatismes

\end{frame}

\subsection{Équations linéaires}

\begin{frame}{1}
	Donner le $x\in\R, x\neq0,$ vérifiant l'équation suivante.
	
		\begin{mybox}{red}{A}
		\begin{center}
			\[ \dfrac3x = \dfrac12. \]
		\end{center}
		\end{mybox}
		\begin{mybox}{green}{B}
		\begin{center}
			\[ \dfrac8x  =\dfrac{\sqrt{3}}2. \]
		\end{center}
		\end{mybox}

\end{frame}


\begin{frame}{2}
	Donner le $x\in\R, x\neq0,$ vérifiant l'équation suivante.
	
		\begin{mybox}{red}{A}
		\begin{center}
			\[ \dfrac2x =\sqrt{3}. \]
		\end{center}
		\end{mybox}
		\begin{mybox}{green}{B}
		\begin{center}
			\[ \dfrac7x  =\dfrac12. \]
		\end{center}
		\end{mybox}
\end{frame}

\subsection{Trigonométrie}


\begin{frame}{3}
	Calculer la longueur \textbf{exacte} de $AC$.
	
		\begin{tcbraster}[raster columns=2,raster equal height,nobeforeafter,raster column skip=1cm]
		\begin{mybox}{red}{A}
		\begin{center}
		\begin{tikzpicture}[scale=1]
			\draw[-, thick, black] (0,0) -- (3,0) node[midway, below] {$8$};
			\draw[-, thick, black] (3,0) -- (3,2);
			\draw[-, thick, black] (3,2) -- (0,0);
			
			\node[black, left] at  (0,0) {$A$};
			\node[black, below] at  (3,0) {$B$};
			\node[black, above] at  (3,2) {$C$};
			% angle droit
			\draw[-, thick, black] (3,.3)-- (2.7,.3);
			\draw[-, thick, black] (2.7,.3)-- (2.7,0);
			
			% angle
			\draw [black,thick,domain=0:35] plot ({.5*cos(\x)}, {.5*sin(\x)})
			node[left=5pt, right=10pt] {$30°$};
		\end{tikzpicture}
		\end{center}
		\end{mybox}
		\begin{mybox}{green}{B}
		\begin{center}
		\begin{tikzpicture}[scale=1]
			\draw[-, thick, black] (0,0) -- (3,0) node[midway, below] {$3$};
			\draw[-, thick, black] (3,0) -- (3,2);
			\draw[-, thick, black] (3,2) -- (0,0);
			
			\node[black, left] at  (0,0) {$A$};
			\node[black, below] at  (3,0) {$B$};
			\node[black, above] at  (3,2) {$C$};
			% angle droit
			\draw[-, thick, black] (3,.3)-- (2.7,.3);
			\draw[-, thick, black] (2.7,.3)-- (2.7,0);
			
			% angle
			\draw [black,thick,domain=215:270] plot ({3+.5*cos(\x)}, {2+.5*sin(\x)})
			node[below=5pt, left=-1pt] {$30°$};
		\end{tikzpicture}
		\end{center}
		\end{mybox}
		\end{tcbraster}
		
			$\sin(30°) = \dfrac12,$
			$\cos(30°) = \dfrac{\sqrt{3}}2,$ et $\tan(30°) = \dfrac1{\sqrt{3}}.$
\end{frame}

\begin{frame}{4}
	Calculer la longueur \textbf{exacte} de $AB$.
	
		\begin{tcbraster}[raster columns=2,raster equal height,nobeforeafter,raster column skip=1cm]
		\begin{mybox}{red}{A}
		\begin{center}
		\begin{tikzpicture}[scale=1]
			\draw[-, thick, black] (0,0) -- (3,0);
			\draw[-, thick, black] (3,0) -- (3,2) node[midway, right] {$7$} ;
			\draw[-, thick, black] (3,2) -- (0,0);
			
			\node[black, left] at  (0,0) {$A$};
			\node[black, below] at  (3,0) {$B$};
			\node[black, above] at  (3,2) {$C$};
			% angle droit
			\draw[-, thick, black] (3,.3)-- (2.7,.3);
			\draw[-, thick, black] (2.7,.3)-- (2.7,0);
			
			% angle
			\draw [black,thick,domain=215:270] plot ({3+.5*cos(\x)}, {2+.5*sin(\x)})
			node[below=5pt, left=-1pt] {$60°$};
		\end{tikzpicture}
		\end{center}
		\end{mybox}
		\begin{mybox}{green}{B}
		\begin{center}
		\begin{tikzpicture}[scale=1]
			\draw[-, thick, black] (0,0) -- (3,0);
			\draw[-, thick, black] (3,0) -- (3,2)  node[midway, right] {$2$} ;
			\draw[-, thick, black] (3,2) -- (0,0);
			
			\node[black, left] at  (0,0) {$A$};
			\node[black, below] at  (3,0) {$B$};
			\node[black, above] at  (3,2) {$C$};
			% angle droit
			\draw[-, thick, black] (3,.3)-- (2.7,.3);
			\draw[-, thick, black] (2.7,.3)-- (2.7,0);
			
			% angle
			\draw [black,thick,domain=0:35] plot ({.5*cos(\x)}, {.5*sin(\x)})
			node[left=5pt, right=10pt] {$60°$};
		\end{tikzpicture}
		\end{center}
		\end{mybox}
		\end{tcbraster}
		
			$\cos(60°) = \dfrac12,$
			$\sin(60°) = \dfrac{\sqrt{3}}2,$ et $\tan(60°) =\sqrt{3}.$
\end{frame}

\end{document}
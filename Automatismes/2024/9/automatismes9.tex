\documentclass[14pt]{beamer}
\usepackage[french]{babel}

\usetheme{CambridgeUS}
\usecolortheme{rose}
\beamertemplatenavigationsymbolsempty


\usepackage{libertinus}
\usepackage{amsmath,amsfonts,amsthm,amssymb,mathtools}
\usepackage{array}
\newcolumntype{P}[1]{>{\centering\arraybackslash}p{#1}}


\usepackage{stackengine}
\newcommand\xrowht[2][0]{\addstackgap[.5\dimexpr#2\relax]{\vphantom{#1}}}


% corps
\usepackage{calrsfs}
\newcommand{\C}{\mathcal{C}}
\newcommand{\R}{\mathbb{R}}
\newcommand{\Rnn}{\mathbb{R}^{2n}}
\newcommand{\Z}{\mathbb{Z}}
\newcommand{\N}{\mathbb{N}}
\newcommand{\Q}{\mathbb{Q}}

% domain
\newcommand{\D}{\mathbb{D}}


% date
\usepackage{advdate}
\AdvanceDate[0]

%plots
\usepackage{pgfplots, subcaption}
\definecolor{myg}{RGB}{56, 140, 70}
\definecolor{myb}{RGB}{45, 111, 177}
\definecolor{myr}{RGB}{199, 68, 64}

%boxes
\usepackage[most]{tcolorbox}
\usepackage{multicol}

%icomma
\usepackage{icomma}

%https://osl.ugr.es/CTAN/macros/latex/contrib/tcolorbox/tcolorbox.pdf
\newtcolorbox{mybox}[3][]
{
  colframe = #2!25,
  colback  = #2!10,
  coltitle = #2!20!black,  
  halign title=flush center, 
  title    = {#3},
  #1,
}
\begin{document}

\section{Automatismes}

\begin{frame}

\centering \huge
Automatismes

\end{frame}

\subsection{Appartenances}

\begin{frame}{1}
	Pour $f:\R\rightarrow\R$ donnée algébriquement, déterminer si $P \in \C_f$ ou non.
	
		\begin{mybox}{red}{A}
		\begin{center}
			\begin{align*} f(x) = 2x+1, && P=(-1;1). \end{align*}
		\end{center}
		\end{mybox}
		\begin{mybox}{green}{B}
		\begin{center}
			\begin{align*} f(x) = -3x-1, && P=(0;-1). \end{align*}
		\end{center}
		\end{mybox}

\end{frame}


\begin{frame}{2}
	Pour $g:\R\rightarrow\R$ donnée algébriquement, déterminer si $P \in \C_g$ ou non.
	
		\begin{mybox}{red}{A}
		\begin{center}
			\begin{align*} g(x) = x^2 - 1, && P=(-1;-2). \end{align*}
		\end{center}
		\end{mybox}
		\begin{mybox}{green}{B}
		\begin{center}
			\begin{align*} g(x) = \dfrac3x, && P=(-3; -1). \end{align*}
		\end{center}
		\end{mybox}

\end{frame}

\subsection{Lecture graphique}


\begin{frame}{3}
	Est-ce que $(0;-3)$ appartient à la courbe ?
		\begin{center}
		\begin{tikzpicture}[>=stealth, scale=1]
			\begin{axis}[xmin = -3.4, xmax=2.3, ymin=-5.1, ymax=5.1, axis x line=middle, axis y line=middle, axis line style=->, grid=both, ytick={-4,-3, ..., 3, 4}]
				\addplot[no marks, myr, -, thick] expression[domain=-4:3, samples=100]{x^3 /3 - 2*x +3}
				node[pos=.5, above=5pt]{$A$};
				\addplot[no marks, myg, -, thick] expression[domain=-2:2.1, samples=100]{2*x^2 - x -1}
				node[pos=.4, left=5pt]{$B$};
			\end{axis}
		\end{tikzpicture}
		\end{center}

\end{frame}



\begin{frame}{4}
	Est-ce que $(1;0)$ appartient à la courbe ?
		\begin{center}
		\begin{tikzpicture}[>=stealth, scale=1]
			\begin{axis}[xmin = -3.4, xmax=2.3, ymin=-5.1, ymax=5.1, axis x line=middle, axis y line=middle, axis line style=->, grid=both, ytick={-4,-3, ..., 3, 4}]
				\addplot[no marks, myr, -, thick] expression[domain=-4:3, samples=100]{x^3 /3 - 2*x +3}
				node[pos=.5, above=5pt]{$A$};
				\addplot[no marks, myg, -, thick] expression[domain=-2:2.1, samples=100]{2*x^2 - x -1}
				node[pos=.4, left=5pt]{$B$};
			\end{axis}
		\end{tikzpicture}
		\end{center}

\end{frame}

\end{document}
% DYSLEXIA SWITCH
\newif\ifdys
		
				% ENABLE or DISABLE font change
				% use XeLaTeX if true
				\dystrue
				\dysfalse


\ifdys

\documentclass[a4paper, 14pt]{extarticle}
\usepackage{amsmath,amsfonts,amsthm,amssymb,mathtools}

\tracinglostchars=3 % Report an error if a font does not have a symbol.
\usepackage{fontspec}
\usepackage{unicode-math}
\defaultfontfeatures{ Ligatures=TeX,
                      Scale=MatchUppercase }

\setmainfont{OpenDyslexic}[Scale=1.0]
\setmathfont{Fira Math} % Or maybe try KPMath-Sans?
\setmathfont{OpenDyslexic Italic}[range=it/{Latin,latin}]
\setmathfont{OpenDyslexic}[range=up/{Latin,latin,num}]

\else

\documentclass[a4paper, 12pt]{extarticle}

\usepackage[utf8x]{inputenc}
%fonts
\usepackage{amsmath,amsfonts,amsthm,amssymb,mathtools}
% comment below to default to computer modern
\usepackage{libertinus,libertinust1math}

\fi


\usepackage[french]{babel}
\usepackage[
a4paper,
margin=2cm,
nomarginpar,% We don't want any margin paragraphs
]{geometry}
\usepackage{icomma}

\usepackage{fancyhdr}
\usepackage{array}
\usepackage{hyperref}

\usepackage{multicol, enumerate}
\newcolumntype{P}[1]{>{\centering\arraybackslash}p{#1}}


\usepackage{stackengine}
\newcommand\xrowht[2][0]{\addstackgap[.5\dimexpr#2\relax]{\vphantom{#1}}}

% theorems

\theoremstyle{plain}
\newtheorem{theorem}{Th\'eor\`eme}
\newtheorem*{sol}{Solution}
\theoremstyle{definition}
\newtheorem{ex}{Exercice}
\newtheorem*{rpl}{Rappel}
\newtheorem{enigme}{Énigme}

% corps
\usepackage{calrsfs}
\newcommand{\C}{\mathcal{C}}
\newcommand{\R}{\mathbb{R}}
\newcommand{\Rnn}{\mathbb{R}^{2n}}
\newcommand{\Z}{\mathbb{Z}}
\newcommand{\N}{\mathbb{N}}
\newcommand{\Q}{\mathbb{Q}}

% variance
\newcommand{\Var}[1]{\text{Var}(#1)}

% domain
\newcommand{\D}{\mathcal{D}}


% date
\usepackage{advdate}
\AdvanceDate[0]


% plots
\usepackage{pgfplots}

% table line break
\usepackage{makecell}
%tablestuff
\def\arraystretch{2}
\setlength\tabcolsep{15pt}

%subfigures
\usepackage{subcaption}

\definecolor{myg}{RGB}{56, 140, 70}
\definecolor{myb}{RGB}{45, 111, 177}
\definecolor{myr}{RGB}{199, 68, 64}

% fake sections with no title to move around the merged pdf
\newcommand{\fakesection}[1]{%
  \par\refstepcounter{section}% Increase section counter
  \sectionmark{#1}% Add section mark (header)
  \addcontentsline{toc}{section}{\protect\numberline{\thesection}#1}% Add section to ToC
  % Add more content here, if needed.
}


% SOLUTION SWITCH
\newif\ifsolutions
				\solutionstrue
				%\solutionsfalse

\ifsolutions
	\newcommand{\exe}[2]{
		\begin{ex} #1  \end{ex}
		\begin{sol} #2 \end{sol}
	}
\else
	\newcommand{\exe}[2]{
		\begin{ex} #1  \end{ex}
	}
	
\fi


% tableaux var, signe
\usepackage{tkz-tab}


%pinfty minfty
\newcommand{\pinfty}{{+}\infty}
\newcommand{\minfty}{{-}\infty}

\begin{document}


\pagestyle{fancy}
\fancyhead[L]{Seconde 13}
\fancyhead[C]{\textbf{Entraînement : fonctions, calcul littéral, équations \ifsolutions \, -- Solutions  \fi}}
\fancyhead[R]{\today}

\exe{[Images]
	Soit 
		\[ f(x) = 3x^2 - 2x + 2 - \dfrac4x, \]
	fonction sur $]{-\infty}; 0[ \cup ]0 ; {+\infty}[$.
	
	Justifier que $f$ n'est pas définie en $0$, et calculer les images suivantes.
	
	\begin{multicols}{2}
	\begin{enumerate}
		\item $f(1)$
		\item $f(4)$
		\item $f(-1)$
		\item $f(-4)$
		\item $f(-8)$
		\item $f(8)$
		\item $f(3)$
		\item $f(-2)$
	\end{enumerate}
	\end{multicols}


}{}

\exe{[Propriété fondamentale]
	Pour chaque paire de point $P$ et de fonction $f$, déterminer si $P\in\C_f$ ou non.
	
	Pour chaque fonction, trouver aussi un point $Q\neq P$ qui appartient à $\C_f$.
	
	\begin{multicols}{2}
	\begin{enumerate}
		\item $P = (0;1)$ et $f(x) = (3x- 1)^2$
		\item $P = (0;1)$ et $f(x) = (3x- 1)^3$
		\item $P = (-6;5)$ et $f(x) = \dfrac23 x + 7$
		\item $P = (-2;2)$ et $f(x) = 4-x$
		\item $P = (-2;2)$ et $f(x) = x^3 -2x + 8$
		\item $P = (-1;2)$ et $f(x) = x^4 - \dfrac1x$
		\item $P = \left(-2;\dfrac29\right)$ et $f(x) = \left( \dfrac{x}3 \right)^2 - 7$
		\item $P = \left(5 ; \sqrt{5} -3 \right)$ et $f(x) =\dfrac{x}{\sqrt{5}} - 3$
	\end{enumerate}
	\end{multicols}
}{}

\exe{[Équations linéaires]
	Pour chacune des paires de fonctions $f, g$ sur $\R$, calculer $\C_f \cap \C_g$.
	
	\begin{multicols}{2}
	\begin{enumerate}
		\item $f(x) = 7x -2, g(x) = -x+3$.
		\item $f(x) = -x + 3, g(x) = 7x -2$.
		\item $f(x) = 3, g(x) = -x$.
		\item $f(x) = -2x+1, g(x) = 1-x$.
		\item $f(x) = 2, g(x) = 4$.
		\item $f(x) = -x+1, g(x) = 1-x$.
	\end{enumerate}
	\end{multicols}
}{}


\exe{[Développement-réduction]
	Développer et réduire les expressions suivantes pour obtenir une expression de la forme
		\[ ax^2 + bx + c, \]
	où $a, b, c \in\R$ sont des nombres réels (possiblement nuls).
	
	\begin{multicols}{2}
	\begin{enumerate}
		\item $x + 2x + 3x$
		\item $(1+3x) - (8 - 2x)$
		\item $x - 2(2-x)$
		\item $x(3-x) + 8(x+1)$
		\item $(x+3)(x-1) - 2x + 1$
		\item $(x+1)^2 + 4x - 4$
		\item $(3x-1)^2 - 9x^2 - 4$
		\item $x(3x-1)  + 3x^2 +12$
	\end{enumerate}
	\end{multicols}
}{}

\exe{[Équations avec racines]
	Résoudre les équations suivantes pour $x\in\R$ non nul. Exprimer $x$ sous la forme $q \sqrt{r}$ avec $q\in\Q$ rationnel et $r\in\N$ le plus petit possible.
	\begin{multicols}{2}
	\begin{enumerate}
		\item $\dfrac3x = \dfrac73.$
		\item $ \dfrac7x  =\dfrac1x + 2$
		\item $\dfrac{16}x  =\dfrac{\sqrt{3}}2$
		\item $-\dfrac2x =4\sqrt{3}$
		\item $\dfrac9{-x}  =\dfrac{\sqrt{6}}5$
		\item $-\dfrac3x =\dfrac57\sqrt{7}$
		\item $-\dfrac9{2x}  =\dfrac{\sqrt{11}}3$
		\item $\dfrac2{7x} =-\sqrt{14}$
	\end{enumerate}
	\end{multicols}
}{}

\newpage


\exe{[Identités remarquables]
	Développer les expressions algébriques suivantes pour obtenir une expression de la forme
		\[ ax^2 + bx + c, \]
	où $a, b, c \in\R$ sont des nombres réels (possiblement nuls).
		\begin{multicols}{2}
		\begin{enumerate}
			\item $f(x) = (1+x)^2$
			\item $g(x) = (x-3)^2$
			\item $h(x) = (3-x)^2$
			\item $F(x) = (3 + 2x)^2$
			\item $G(x) = (3x - 7)^2$
			\item $H(x) = (-7x - 2)^2$
		\end{enumerate}
		\end{multicols}
}{
	\begin{align*}
		f(x) &= (1+x)^2 = 1^2 + 2(1)(x) + x^2 = 1 + 2x + x^2 \\
		g(x) &= (x-3)^2 = x^2 - 2(x)(3) + 3^2 = x^2 - 6x + 9 \\
		h(x) &= (3-x)^2 = 3^2 - 2(3)(x) + x^2 = 9 - 6x + x^2 \\
		F(x) &= (3 + 2x)^2 = 3^2 + 2(3)(2x) + (2x)^2 = 9 + 12x + 4x^2 \\
		G(x) &= (3x - 7)^2 = (3x)^2 - 2(3x)(7) + 7^2 = 9x^2 - 42x + 49 \\
		H(x) &= (-7x - 2)^2 = (7x + 2)^2 = (7x)^2 + 2(7x)(2) + 2^2 = 49x^2 + 28x + 4
	\end{align*}

}

\exe{[Racine et valeur absolue]
	Montrer que l'identité
		\[ \sqrt{x^2} = x \]
	ne peut pas être vraie pour tout $x\in\R$, puis qu'on a en fait
		\[\sqrt{x^2} = |x| \]
	pour tout $x\in\R$.
}{}

\exe{[Racines et valeurs absolues]
	Pour chaque équation suivante, donner l'ensemble des solutions $x\in\R$ réelles.
		\begin{multicols}{2}
		\begin{enumerate}
			\item $|x+1| = 4$
			\item $(x+1)^2 = 16$
			\item $|2x-1| = 3$
			\item $(2x-1)^2 = 9$
			\item $(3-3x)^2 = 0$
			\item $(x-1)^2 = 5$
			\item $(x+3)^2 = -1$
			\item $16x^2 = 64$
		\end{enumerate}
		\end{multicols}
}{

	\begin{enumerate}
		\item 
		\begin{align*}
			x^2 + 2x + 1 &= 16 \\
			(x+1)^2 &= 16 \\
			|x+1| &= 4 \\
			x+1=4 \qquad & \text{ou} \qquad x+1=-4 \\
			x=3 \qquad & \text{ou} \qquad x=-5
		\end{align*}
		\item
		\begin{align*}
			4x^2 - 4x + 1 &= 9 \\
			(2x-1)^2 &= 9 \\
			|2x-1| &= 3 \\
			2x-1=3 \qquad & \text{ou} \qquad 2x-1=-3 \\
			x=2 \qquad & \text{ou} \qquad x=-1
		\end{align*}
		 
		\item
		\begin{align*}
			9 - 18x + 9x^2 &= 0 \\
			(3-3x)^2 &= 0 \\
			|3-3x| &= 0 \\
			3-3x &= 0 \\
			x&=1
		\end{align*}
		\item 
		\begin{align*}
			x^2 - 2x + 1 &= 5 \\
			(x-1)^2 &= 5 \\
			|x-1| &= \sqrt{5} \\
			x-1=\sqrt{5} \qquad & \text{ou} \qquad x-1=-\sqrt{5} \\
			x=1+\sqrt{5} \qquad & \text{ou} \qquad x=1-\sqrt{5}
		\end{align*}
		\item 
		\begin{align*}
			x^2 +6x + 9 &= -1 \\
			(x+3)^2 &= -1 
		\end{align*}
		Il n'y a pas de solutions car un carré est toujours positif ou nul !
		\item 
		\begin{align*}
			16x^2 - 64 = 0 \\
			(4x+8)(4x-8) &= 0 \\
			4x+8=0 \qquad & \text{ou} \qquad 4x-8=0 \\
			x=-2 \qquad & \text{ou} \qquad x=2
		\end{align*}
		où on a utilisé qu'un produit est nul si est seulement si un des facteurs est nul.
	\end{enumerate}
}


\exe{[Puissances de $x$ non miscibles]
	Montrer que l'identité
		\[ x^2 = x \]
	ne peut pas être vraie pour tout $x\in\R$.
}{}

\exe{[Puissances de $x$ non miscibles, suite]
	Montrer que, pour $m, n\in\N$ non nuls, l'identité
		\[ x^m = x^n \]
	ne peut être vraie pour tout $x\in\R$ que lorsque $m=n$.
}{}



\end{document}

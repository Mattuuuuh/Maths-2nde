\documentclass[14pt]{beamer}
\usepackage[french]{babel}


\usetheme{CambridgeUS}
\usecolortheme{rose}
\beamertemplatenavigationsymbolsempty

\usepackage{array}
\usepackage{amsmath,amsfonts,amsthm,amssymb,mathtools}
\newcolumntype{P}[1]{>{\centering\arraybackslash}p{#1}}


\usepackage{stackengine}
\newcommand\xrowht[2][0]{\addstackgap[.5\dimexpr#2\relax]{\vphantom{#1}}}


% corps
\newcommand{\C}{\mathbb{C}}
\newcommand{\R}{\mathbb{R}}
\newcommand{\Rnn}{\mathbb{R}^{2n}}
\newcommand{\Z}{\mathbb{Z}}
\newcommand{\N}{\mathbb{N}}
\newcommand{\Q}{\mathbb{Q}}

% domain
\newcommand{\D}{\mathbb{D}}


% date
\usepackage{advdate}
\AdvanceDate[1]

\begin{document}

\section{Automatismes}

\begin{frame}

\centering \huge
Automatismes

\end{frame}

\subsection{Proportions}

\begin{frame}{1}
	\centering
	Calculer $140\%$ de $125$.
\end{frame}

\begin{frame}{2}
	$7$ élèves sur les $14$ ayant répondu au sondage jugent le contenu du cours trop difficile.
	
	\vfill
	Écrire la proportion d'élèves
	\begin{enumerate}
		\item ayant participé au sondage ; et
		\item jugeant le contenu du cours trop difficile.
	\end{enumerate}
	
\end{frame}

\subsection{Sous-population}

\begin{frame}{3}
	\centering
	Calculer $50\%$ de $80\%$ de $300$.
\end{frame}

\begin{frame}{4}
	$77\%$ des élèves ayant répondu au sondage jugent le rythme trop soutenu.
	$38\%$ des élèves ont répondu au sondage.
	
	\vfill
	Écrire approximativement la proportion d'élèves de la classe jugeant le rythme trop soutenu.
	
\end{frame}






\end{document}
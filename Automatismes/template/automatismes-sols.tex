% DYSLEXIA SWITCH
\newif\ifdys
		
				% ENABLE or DISABLE font change
				% use XeLaTeX if true
				\dystrue
				\dysfalse


\ifdys

\documentclass[a4paper, 14pt]{extarticle}
\usepackage{amsmath,amsfonts,amsthm,amssymb,mathtools}

\tracinglostchars=3 % Report an error if a font does not have a symbol.
\usepackage{fontspec}
\usepackage{unicode-math}
\defaultfontfeatures{ Ligatures=TeX,
                      Scale=MatchUppercase }

\setmainfont{OpenDyslexic}[Scale=1.0]
\setmathfont{Fira Math} % Or maybe try KPMath-Sans?
\setmathfont{OpenDyslexic Italic}[range=it/{Latin,latin}]
\setmathfont{OpenDyslexic}[range=up/{Latin,latin,num}]

\else

\documentclass[a4paper, 12pt]{extarticle}

\usepackage[utf8x]{inputenc}
%fonts
\usepackage{amsmath,amsfonts,amsthm,amssymb,mathtools}
% comment below to default to computer modern
\usepackage{libertinus,libertinust1math}

\fi


\usepackage[french]{babel}
\usepackage[
a4paper,
margin=2cm,
nomarginpar,% We don't want any margin paragraphs
]{geometry}
\usepackage{icomma}

\usepackage{fancyhdr}
\usepackage{array}
\usepackage{hyperref}

\usepackage{multicol, enumerate}
\newcolumntype{P}[1]{>{\centering\arraybackslash}p{#1}}


\usepackage{stackengine}
\newcommand\xrowht[2][0]{\addstackgap[.5\dimexpr#2\relax]{\vphantom{#1}}}

% theorems

\theoremstyle{plain}
\newtheorem{theorem}{Th\'eor\`eme}
\newtheorem*{sol}{Solution}
\theoremstyle{definition}
\newtheorem{ex}{Exercice}
\newtheorem*{rpl}{Rappel}
\newtheorem{enigme}{Énigme}

% corps
\usepackage{calrsfs}
\newcommand{\C}{\mathcal{C}}
\newcommand{\R}{\mathbb{R}}
\newcommand{\Rnn}{\mathbb{R}^{2n}}
\newcommand{\Z}{\mathbb{Z}}
\newcommand{\N}{\mathbb{N}}
\newcommand{\Q}{\mathbb{Q}}

% variance
\newcommand{\Var}[1]{\text{Var}(#1)}

% domain
\newcommand{\D}{\mathcal{D}}


% date
\usepackage{advdate}
\AdvanceDate[0]


% plots
\usepackage{pgfplots}

% table line break
\usepackage{makecell}
%tablestuff
\def\arraystretch{2}
\setlength\tabcolsep{15pt}

%subfigures
\usepackage{subcaption}

\definecolor{myg}{RGB}{56, 140, 70}
\definecolor{myb}{RGB}{45, 111, 177}
\definecolor{myr}{RGB}{199, 68, 64}

% fake sections with no title to move around the merged pdf
\newcommand{\fakesection}[1]{%
  \par\refstepcounter{section}% Increase section counter
  \sectionmark{#1}% Add section mark (header)
  \addcontentsline{toc}{section}{\protect\numberline{\thesection}#1}% Add section to ToC
  % Add more content here, if needed.
}


% SOLUTION SWITCH
\newif\ifsolutions
				\solutionstrue
				%\solutionsfalse

\ifsolutions
	\newcommand{\exe}[2]{
		\begin{ex} #1  \end{ex}
		\begin{sol} #2 \end{sol}
	}
\else
	\newcommand{\exe}[2]{
		\begin{ex} #1  \end{ex}
	}
	
\fi


% tableaux var, signe
\usepackage{tkz-tab}


%pinfty minfty
\newcommand{\pinfty}{{+}\infty}
\newcommand{\minfty}{{-}\infty}

\begin{document}

\usepackage[answerdelayed]{exercise}
\AdvanceDate[3]

\begin{document}

\section{Automatismes}

\begin{frame}

\centering \huge
Automatismes

\large
Calculatrice interdite

\end{frame}

\subsection{Propriétés des puissances}

\framedelayed{
	Écrire le nombre suivant sous forme $a^b$, où $a, b \in \Z$ sont des entiers relatifs.
	\boxAB{
		$\dfrac{13^{15}}{13^{9}}$
	}{
		$\dfrac{11^{17}}{11^9}$.
	}
}{
	\boxAB{
		$\dfrac{13^{15}}{13^{9}} = \pause 13^{15-9} = \pause 13^6.$
	}{
		$\dfrac{11^{17}}{11^9} = $
	}
}

\framedelayed{
	Écrire le nombre suivant sous forme $a^b$, où $a, b \in \Z$ sont des entiers relatifs.
	\boxAB{
		$\bigl(2^6\bigr)^{-5}$
	}{
		$\bigl(2^7\bigr)^{-4}$
	}
}{

test

}

\subsection{Encadrement}

\framedelayed{
	Donner l'amplitude de l'encadrement suivant sous forme de puissance de $10$.
	\boxAB{
		$318,615  < 318,621702 < 318,625$
	}{
		$610,3075 < 610,3081702 <  610,3085$
	}
}{

test

}

\subsection{Diochotomie}

\framedelayed{
	On considère une fonction $f$ croissante sur $\R$, et $x_0$ l'unique nombre vérifiant $f(x_0) = 0$.
	
	On sait en outre que $x_0$ est encadré par
		\[ 6 < x_0 < 8. \]
	\boxAB{
		Donner le signe de $f(8)$.
	}{
		Donner le signe de $f(6)$.
	}
}{

test

}


\section{Solutions}
\begin{frame}
\centering \huge
Solutions
\end{frame}

\shipoutAnswer

\end{document}
\documentclass[14pt]{beamer}
\usepackage[french]{babel}


\usetheme{CambridgeUS}
\usecolortheme{rose}
\beamertemplatenavigationsymbolsempty

\usepackage{array}
\usepackage{amsmath,amsfonts,amsthm,amssymb,mathtools}
\newcolumntype{P}[1]{>{\centering\arraybackslash}p{#1}}


\usepackage{stackengine}
\newcommand\xrowht[2][0]{\addstackgap[.5\dimexpr#2\relax]{\vphantom{#1}}}


% corps
\newcommand{\C}{\mathbb{C}}
\newcommand{\R}{\mathbb{R}}
\newcommand{\Rnn}{\mathbb{R}^{2n}}
\newcommand{\Z}{\mathbb{Z}}
\newcommand{\N}{\mathbb{N}}
\newcommand{\Q}{\mathbb{Q}}

% domain
\newcommand{\D}{\mathbb{D}}


% date
\usepackage{advdate}
\AdvanceDate[1]

%plots
\usepackage{pgfplots, subcaption}
\definecolor{myg}{RGB}{56, 140, 70}
\definecolor{myb}{RGB}{45, 111, 177}
\definecolor{myr}{RGB}{199, 68, 64}

\begin{document}

\section{Automatismes}

\begin{frame}

\centering \huge
Automatismes

\end{frame}

\subsection{Moyennes}

\begin{frame}{1}
	Donner la moyenne exacte $\overline{X}$ de la série statistique
		\[ X= (0 ; -1 ; 12 ; -11 ; -11 ; 3 ; 0). \]
\end{frame}

\begin{frame}{2}
	Calculer la moyenne de la série statistique suivante.
	
	\vspace{1cm}
	\centering
	\begin{tabular}{|c|c|c|c|c|}\hline
	Valeur & -2,5 & -4,5 & 15,1 & 9,825  \\ \hline
	Effectif & 3 & 2 & 10 & 50 \\ \hline
	\end{tabular}
\end{frame}

\subsection{Variance}

\begin{frame}{3}
	Calculer la variance $\text{Var}(X)$ de la série statistique $X$ suivante de moyenne $\overline{X} = 5$.
	
	\vspace{1cm}
	\centering
	\begin{tabular}{|c|c|c|c|}\hline
	Valeur & 2,5 & 4,5 & 5,85 \\ \hline
	Effectif & 3 & 2 & 10\\ \hline
	\end{tabular}	
	
\end{frame}

\begin{frame}{4}
	\centering
	Vrai ou faux : l'écart type de la deuxième série statistique est plus grand que celui de la première.
	
\begin{figure}[t!]
  \centering
  \begin{subfigure}[b]{.45\textwidth}
    \centering
  \begin{tikzpicture}[scale=.6]
    \begin{axis}[
    	ymin=0,
        %ymin=0, ymax=8,
        %minor y tick num = 3,
        %xmin = 0, xmax=20,
        %xtick = {2, 4, ..., 18},
        area style,
        xlabel = {Valeur},
        ylabel = {Effectif}
      ]
      \addplot+[ybar interval,mark=no, draw=myg, fill=myg!50] plot coordinates {
        (5,6) (6,3) (7,0) (8,3) (9,2) (10,10) (11,0)
      };
    \end{axis} 
  \end{tikzpicture}
  \caption{Première série.}
  \label{fig:a}
  \end{subfigure}
  \hfill
  % NOT LINE BREAK!!
  \begin{subfigure}[b]{.45\textwidth}
    \centering
  \begin{tikzpicture}[scale=.6]
    \begin{axis}[
    	ymin=0,
        %ymin=0, ymax=8,
        %minor y tick num = 3,
        %xmin = 0, xmax=20,
        %xtick = {2, 4, ..., 18},
        area style,
        xlabel = {Valeur},
        ylabel = {Effectif}
      ]
      \addplot+[ybar interval,mark=no, draw=myb, fill=myb!50] plot coordinates {
        (22,1) (23,5) (24,7) (25,10) (26,8) (27,3) (28,0)
      };
    \end{axis} 
  \end{tikzpicture}
  \caption{Deuxième série.}
  \label{fig:b}
  \end{subfigure}
  
\end{figure}

	
\end{frame}


\end{document}
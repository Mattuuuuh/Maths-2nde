\documentclass[14pt]{beamer}
\usepackage[french]{babel}


\usetheme{CambridgeUS}
\usecolortheme{rose}
\beamertemplatenavigationsymbolsempty

\usepackage{array}
\usepackage{amsmath,amsfonts,amsthm,amssymb,mathtools}
\newcolumntype{P}[1]{>{\centering\arraybackslash}p{#1}}


\usepackage{stackengine}
\newcommand\xrowht[2][0]{\addstackgap[.5\dimexpr#2\relax]{\vphantom{#1}}}


% corps
\newcommand{\C}{\mathbb{C}}
\newcommand{\R}{\mathbb{R}}
\newcommand{\Rnn}{\mathbb{R}^{2n}}
\newcommand{\Z}{\mathbb{Z}}
\newcommand{\N}{\mathbb{N}}
\newcommand{\Q}{\mathbb{Q}}

% domain
\newcommand{\D}{\mathbb{D}}


% date
\usepackage{advdate}
\AdvanceDate[1]

\begin{document}

\section{Automatismes}

\subsection{Vrai ou faux}

\begin{frame}

\centering \huge
Automatismes

\end{frame}


\begin{frame}{1}
	Calculer.
	\[ \frac{\ \ 3\ \ }{\dfrac43} - \dfrac{11}{2} = ? \]
	
\end{frame}

\begin{frame}{2}

	Écrire sous forme d'intervalle.
		\[ \{ x \in \R \text{ tq. } -1 \leq 3x +4 \leq 1 \} = \ ? \]

\end{frame}

\begin{frame}{3}

	Écrire sous forme d'intervalle.
	\[ ]{-}\infty ; {-}2 [ \cap \{ x \in \R \text{ tq. } 2 \geq x \} = \ ? \]

\end{frame}


\begin{frame}{4}

	Donner la valeur (sans les barres de valeur absolue).
		\[ \left| -\dfrac{11}3 + \dfrac12 \right| = ?  \]

\end{frame}





\end{document}
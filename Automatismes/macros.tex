%!TEX encoding = UTF8
%!TEX root = 0-notes.tex

%%%%%%%%%%%%%%%%%%%%%%%%%%%%%%
% SELF MADE COMMANDS
%%%%%%%%%%%%%%%%%%%%%%%%%%%%%%

%%

% deliminators
\DeclarePairedDelimiter{\abs}{\lvert}{\rvert}
%\DeclarePairedDelimiter{\norm}{\lVert}{\rVert}

\DeclarePairedDelimiter{\ceil}{\lceil}{\rceil}
\DeclarePairedDelimiter{\floor}{\lfloor}{\rfloor}
\DeclarePairedDelimiter{\round}{\lfloor}{\rceil}

\newsavebox\diffdbox
\newcommand{\slantedromand}{{\mathpalette\makesl{d}}}
\providecommand*{\pdv}[3][]{\frac{\partial^{#1}#2}{\partial#3^{#1}}}
%  - others
\DeclareMathOperator{\Lap}{\mathcal{L}}
\DeclareMathOperator{\Var}{Var} % variance
\DeclareMathOperator{\Cov}{Cov} % covariance

% Since the amsthm package isn't loaded

% I prefer the slanted \leq
\let\oldleq\leq % save them in case they're every wanted
\let\oldgeq\geq
\renewcommand{\leq}{\leqslant}
\renewcommand{\geq}{\geqslant}

% tel que
\newcommand{\tqs}{\text{ tels que }}
\newcommand{\tq}{\text{ tq. }}
\newcommand{\et}{\text{ et }}
\newcommand{\ou}{\text{ ou }}
\newcommand{\pourtout}{\text{ pour tout }}
\newcommand{\sct}{\text{ sachant }}

% Lois
\newcommand{\Bern}{\text{Bern}}
\newcommand{\Binom}{\text{Binom}}

% ensemble avec bigl et bigr
\newcommand{\bigset}[1]{\bigl\{ #1 \bigr\}}
\newcommand{\Bigset}[1]{\Bigl\{ #1 \Bigr\}}
\newcommand{\bigpar}[1]{\bigl( #1 \bigr)}
\newcommand{\Bigpar}[1]{\Bigl( #1 \Bigr)}

% PLUS INFTY AND MINUS INFTY WITH NO SPACE
\newcommand{\pinfty}{{+}\infty}
\newcommand{\minfty}{{-}\infty}

% vecteur flèche
\renewcommand{\vec}[1]{\overrightarrow{#1}}

% vecteur pmatrix
\newcommand{\pvec}[2]{\begin{pmatrix} #1 \\ #2 \end{pmatrix}}

% vecteur norme
\newcommand{\norm}[1]{\left\Vert #1 \right\Vert}

% point plan
\newcommand{\point}[3]{
	#1\left(#2 ; #3 \right)
}

% \smash avant \underline pour coller la ligne au mot
\let\oldunderline\underline
\renewcommand{\underline}[1]{\oldunderline{\smash{#1}}}

% emph + index
\newcommand{\emphindex}[1]{\emph{#1}\index{#1}}

% tableau croisé
\newcommand{\tableaucroise}[4]{
\begin{tabular}{|c|c|c|c|}
	\cline{2-4}
	\multicolumn{1}{c|}{} & #1 \\ \hline
	#2 \\ \hline
	#3  \\ \hline
	#4  \\ \hline
\end{tabular}
}

% python minted
\newcommand{\python}[1]{
\inputminted[
		linenos,
		gobble=0,
		breaklines=true, % otherwise it breaks for no apparent reason?
		breakafter=,,
		fontsize=\small,
		numbersep=5pt,
		tabsize=4, % tab ident = 4 spaces
		fontfamily=helvetica, %important pour les signes <, >
		breaklines=false,
]{python}{#1.py}
}

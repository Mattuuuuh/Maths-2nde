\documentclass[14pt]{beamer}
\usepackage[french]{babel}


\usetheme{CambridgeUS}
\usecolortheme{rose}
\beamertemplatenavigationsymbolsempty

\usepackage{array}
\usepackage{amsmath,amsfonts,amsthm,amssymb,mathtools}
\newcolumntype{P}[1]{>{\centering\arraybackslash}p{#1}}


\usepackage{stackengine}
\newcommand\xrowht[2][0]{\addstackgap[.5\dimexpr#2\relax]{\vphantom{#1}}}


% corps
\newcommand{\C}{\mathbb{C}}
\newcommand{\R}{\mathbb{R}}
\newcommand{\Rnn}{\mathbb{R}^{2n}}
\newcommand{\Z}{\mathbb{Z}}
\newcommand{\N}{\mathbb{N}}
\newcommand{\Q}{\mathbb{Q}}

% domain
\newcommand{\D}{\mathbb{D}}


% date
\usepackage{advdate}
\AdvanceDate[2]

\begin{document}

\section{Automatismes}


\subsection{Diviseurs, premiers, parité}

\begin{frame}

	\begin{itemize} \itemsep2em 
		\item Écrire la décomposition en produit de facteurs premiers de $132$.
		\item Écrire l'ensemble des diviseurs de $48$.
		\item $49$ est-il premier ? $51$ est-il premier ?
		\item Vrai ou faux : ajouter un nombre pair à un autre nombre ne change pas sa parité.
		\item Vrai ou faux : multiplier un nombre par un nombre pair donne toujours un nombre impair.
	\end{itemize}

\end{frame}


\subsection{Fractions}

\begin{frame}

	\begin{itemize} \itemsep2em 
		\item $1 - \dfrac12 = $
		\item $\dfrac78 + \dfrac{9}{40} = $
		\item $\dfrac{-3}{\dfrac54} = $
		\item $\dfrac{63}{20} \times \dfrac{15}{14} = $
	\end{itemize}


\end{frame}

%\subsection{Calcul littéral}
%
%\begin{frame}
%
%	\begin{itemize} \itemsep2em 
%		\item $(3x -1) + (8x + 4)$
%		\item $(2x + 5) - (2x + 7)$
%		\item $5(-x - 3) + 7(3x + 2)$
%		\item $\dfrac16 (x + 12) + \dfrac12 (x - 2)$
%		\item $(3x)^2 + (2x)^2$
%	\end{itemize}
%
%\end{frame}


\end{document}
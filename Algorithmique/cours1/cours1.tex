				% ENABLE or DISABLE font change
				% use XeLaTeX if true
\newif\ifdys
				\dystrue
				\dysfalse

\newif\ifsolutions
				\solutionstrue
				\solutionsfalse

% DYSLEXIA SWITCH
\newif\ifdys
		
				% ENABLE or DISABLE font change
				% use XeLaTeX if true
				\dystrue
				\dysfalse


\ifdys

\documentclass[a4paper, 14pt]{extarticle}
\usepackage{amsmath,amsfonts,amsthm,amssymb,mathtools}

\tracinglostchars=3 % Report an error if a font does not have a symbol.
\usepackage{fontspec}
\usepackage{unicode-math}
\defaultfontfeatures{ Ligatures=TeX,
                      Scale=MatchUppercase }

\setmainfont{OpenDyslexic}[Scale=1.0]
\setmathfont{Fira Math} % Or maybe try KPMath-Sans?
\setmathfont{OpenDyslexic Italic}[range=it/{Latin,latin}]
\setmathfont{OpenDyslexic}[range=up/{Latin,latin,num}]

\else

\documentclass[a4paper, 12pt]{extarticle}

\usepackage[utf8x]{inputenc}
%fonts
\usepackage{amsmath,amsfonts,amsthm,amssymb,mathtools}
% comment below to default to computer modern
\usepackage{libertinus,libertinust1math}

\fi


\usepackage[french]{babel}
\usepackage[
a4paper,
margin=2cm,
nomarginpar,% We don't want any margin paragraphs
]{geometry}
\usepackage{icomma}

\usepackage{fancyhdr}
\usepackage{array}
\usepackage{hyperref}

\usepackage{multicol, enumerate}
\newcolumntype{P}[1]{>{\centering\arraybackslash}p{#1}}


\usepackage{stackengine}
\newcommand\xrowht[2][0]{\addstackgap[.5\dimexpr#2\relax]{\vphantom{#1}}}

% theorems

\theoremstyle{plain}
\newtheorem{theorem}{Th\'eor\`eme}
\newtheorem*{sol}{Solution}
\theoremstyle{definition}
\newtheorem{ex}{Exercice}
\newtheorem*{rpl}{Rappel}
\newtheorem{enigme}{Énigme}

% corps
\usepackage{calrsfs}
\newcommand{\C}{\mathcal{C}}
\newcommand{\R}{\mathbb{R}}
\newcommand{\Rnn}{\mathbb{R}^{2n}}
\newcommand{\Z}{\mathbb{Z}}
\newcommand{\N}{\mathbb{N}}
\newcommand{\Q}{\mathbb{Q}}

% variance
\newcommand{\Var}[1]{\text{Var}(#1)}

% domain
\newcommand{\D}{\mathcal{D}}


% date
\usepackage{advdate}
\AdvanceDate[0]


% plots
\usepackage{pgfplots}

% table line break
\usepackage{makecell}
%tablestuff
\def\arraystretch{2}
\setlength\tabcolsep{15pt}

%subfigures
\usepackage{subcaption}

\definecolor{myg}{RGB}{56, 140, 70}
\definecolor{myb}{RGB}{45, 111, 177}
\definecolor{myr}{RGB}{199, 68, 64}

% fake sections with no title to move around the merged pdf
\newcommand{\fakesection}[1]{%
  \par\refstepcounter{section}% Increase section counter
  \sectionmark{#1}% Add section mark (header)
  \addcontentsline{toc}{section}{\protect\numberline{\thesection}#1}% Add section to ToC
  % Add more content here, if needed.
}


% SOLUTION SWITCH
\newif\ifsolutions
				\solutionstrue
				%\solutionsfalse

\ifsolutions
	\newcommand{\exe}[2]{
		\begin{ex} #1  \end{ex}
		\begin{sol} #2 \end{sol}
	}
\else
	\newcommand{\exe}[2]{
		\begin{ex} #1  \end{ex}
	}
	
\fi


% tableaux var, signe
\usepackage{tkz-tab}


%pinfty minfty
\newcommand{\pinfty}{{+}\infty}
\newcommand{\minfty}{{-}\infty}

\begin{document}


\AdvanceDate[0]

\begin{document}
\pagestyle{fancy}
\fancyhead[L]{Seconde 13}
\fancyhead[C]{\textbf{Algorithmique : Glossaire \ifsolutions \, -- Solutions  \fi}}
\fancyhead[R]{\today}

\subsection*{Variables : affectation, changement de type}

\begin{mintedbox}{python}
print('Hello, World!')
a = 3
b=5
print(a) # imprime 3
a = b
print(a) # imprime 5

c = 1.2
d = int(c) # int(c) est la partie entière de b
print(d) # imprime 1
\end{mintedbox}

\subsection*{Opérations : somme, produit, puissance}

\begin{mintedbox}{python}
a, b = 5, -3 # version raccourcie de a=5 et b=-3

# sommes et différences
print(a+3) # imprime 8
print(-b) # imprime 3

# produits et quotients
print(b*b) # imprime 9
print(a/b) # imprime -1.6666666666666667
print(int(a/b)) # imprime -1

# puissances
print(2**5) # imprime 32 : 2 puissance 5
print(2e5) # imprime 200000.0 : 2 fois 10 puissance 5
print(1e-3) # imprime 0.001 : 1 fois 10 puissance -3
\end{mintedbox}

\subsection*{Conditions et portes logiques OR, AND}

\begin{mintedbox}{python}
a, b = 3, 4
COND = (a < b)
print(COND) # imprime True

if COND: # si COND, alors
	print('COND est vraie')
else: # sinon,
	print('COND est fausse')

# si a > b OU 2a > b
if a > b or 2*a >= b:
	print('a est strictement supérieur à b ou 2a est supérieur ou égal à b')

# si a < b ET 2a > b
if a < b and 2*a > b:
	print('a < b < 2a')
\end{mintedbox}

\subsection*{Boucles \texttt{for} et \texttt{while}}

\begin{mintedbox}{python}
# range(a, b) va de a à b-1
for k in range(1, 10):
	print(k)
	if (k>=7):
		print(k, ' est supérieur ou égal à 7')

N=0		
# tant que 
while N <= 4:
	print(N, ' est inférieur ou égal à 4')
	N = N+1
print(N)
\end{mintedbox}

\subsection*{Fonctions : nom, argument, retour, et récursivité}

\begin{mintedbox}{python}
# 'nomdefonction' est une fonction qui prend uniquement 'a' comme argument
def nomdefonction(a):
	return a+1

a = 3
b = nomdefonction(a)
print(b) # imprime 4

# fonction 'somme' qui prend deux arguments
def somme(a,b):
	return a+b

print(somme(4,12)) # imprime 16

def recursif(n):
	if n<=0: # condition de sortie
		return 0
	
	return 1 + recursif(n-1) # argument modifié qui se dirige vers la condition de sortie

print(recursif(5)) # imprime 5
print(recursif(12)) # imprime 12
\end{mintedbox}

\end{document}

				% ENABLE or DISABLE font change
				% use XeLaTeX if true
\newif\ifdys
				\dystrue
				\dysfalse

\newif\ifsolutions
				\solutionstrue
				\solutionsfalse

% DYSLEXIA SWITCH
\newif\ifdys
		
				% ENABLE or DISABLE font change
				% use XeLaTeX if true
				\dystrue
				\dysfalse


\ifdys

\documentclass[a4paper, 14pt]{extarticle}
\usepackage{amsmath,amsfonts,amsthm,amssymb,mathtools}

\tracinglostchars=3 % Report an error if a font does not have a symbol.
\usepackage{fontspec}
\usepackage{unicode-math}
\defaultfontfeatures{ Ligatures=TeX,
                      Scale=MatchUppercase }

\setmainfont{OpenDyslexic}[Scale=1.0]
\setmathfont{Fira Math} % Or maybe try KPMath-Sans?
\setmathfont{OpenDyslexic Italic}[range=it/{Latin,latin}]
\setmathfont{OpenDyslexic}[range=up/{Latin,latin,num}]

\else

\documentclass[a4paper, 12pt]{extarticle}

\usepackage[utf8x]{inputenc}
%fonts
\usepackage{amsmath,amsfonts,amsthm,amssymb,mathtools}
% comment below to default to computer modern
\usepackage{libertinus,libertinust1math}

\fi


\usepackage[french]{babel}
\usepackage[
a4paper,
margin=2cm,
nomarginpar,% We don't want any margin paragraphs
]{geometry}
\usepackage{icomma}

\usepackage{fancyhdr}
\usepackage{array}
\usepackage{hyperref}

\usepackage{multicol, enumerate}
\newcolumntype{P}[1]{>{\centering\arraybackslash}p{#1}}


\usepackage{stackengine}
\newcommand\xrowht[2][0]{\addstackgap[.5\dimexpr#2\relax]{\vphantom{#1}}}

% theorems

\theoremstyle{plain}
\newtheorem{theorem}{Th\'eor\`eme}
\newtheorem*{sol}{Solution}
\theoremstyle{definition}
\newtheorem{ex}{Exercice}
\newtheorem*{rpl}{Rappel}
\newtheorem{enigme}{Énigme}

% corps
\usepackage{calrsfs}
\newcommand{\C}{\mathcal{C}}
\newcommand{\R}{\mathbb{R}}
\newcommand{\Rnn}{\mathbb{R}^{2n}}
\newcommand{\Z}{\mathbb{Z}}
\newcommand{\N}{\mathbb{N}}
\newcommand{\Q}{\mathbb{Q}}

% variance
\newcommand{\Var}[1]{\text{Var}(#1)}

% domain
\newcommand{\D}{\mathcal{D}}


% date
\usepackage{advdate}
\AdvanceDate[0]


% plots
\usepackage{pgfplots}

% table line break
\usepackage{makecell}
%tablestuff
\def\arraystretch{2}
\setlength\tabcolsep{15pt}

%subfigures
\usepackage{subcaption}

\definecolor{myg}{RGB}{56, 140, 70}
\definecolor{myb}{RGB}{45, 111, 177}
\definecolor{myr}{RGB}{199, 68, 64}

% fake sections with no title to move around the merged pdf
\newcommand{\fakesection}[1]{%
  \par\refstepcounter{section}% Increase section counter
  \sectionmark{#1}% Add section mark (header)
  \addcontentsline{toc}{section}{\protect\numberline{\thesection}#1}% Add section to ToC
  % Add more content here, if needed.
}


% SOLUTION SWITCH
\newif\ifsolutions
				\solutionstrue
				%\solutionsfalse

\ifsolutions
	\newcommand{\exe}[2]{
		\begin{ex} #1  \end{ex}
		\begin{sol} #2 \end{sol}
	}
\else
	\newcommand{\exe}[2]{
		\begin{ex} #1  \end{ex}
	}
	
\fi


% tableaux var, signe
\usepackage{tkz-tab}


%pinfty minfty
\newcommand{\pinfty}{{+}\infty}
\newcommand{\minfty}{{-}\infty}

\begin{document}


\AdvanceDate[1]

\begin{document}
\pagestyle{fancy}
\fancyhead[L]{Seconde 13}
\fancyhead[C]{\textbf{Algorithmique 6 : encadrements et notation scientifique \ifsolutions \, -- Solutions  \fi}}
\fancyhead[R]{\today}

\begin{proprietes*}{puissances de 10}
	Soit $n\in\N$ un entier naturel. Comme
		\begin{enumerate}[label=$\bullet$]
			\item multiplier par $10$ décale la virgule d'une position vers la droite ; et
			\item diviser par $10$ décale la virgule d'une position vers la gauche ;
		\end{enumerate}
	on a nécessairement
		\begin{flalign*}
			&&
			10^n = 1\underbrace{00\dots00}_{\ifsolutions \text{$n$ fois} \fi}
			&&
			\text{et}
			&&
			10^{-n} = \dfrac1{10^n} = \underbrace{0,00\dots00}_{\ifsolutions \text{$n$ fois} \fi}1.
			&&
		\end{flalign*}
\end{proprietes*}

\exe{\label{ex:1}
	Sans calculatrice, donner la valeur numérique des fractions suivantes.
	
	\begin{multicols}{3}
	\begin{enumerate}[label=\roman*)]
		\item $\dfrac1{10}$
		\item $\dfrac1{5}$
		\item $\dfrac3{10^{5}}$
		\item $\dfrac7{20}$
		\item $\dfrac{395}{50}$
		\item $\dfrac{11}{200}$
	\end{enumerate}
	\end{multicols}

}{}

\begin{definition*}{notation scientifique}
	Soit $x\in\R$ un nombre réel \emph{décimal}, c'est-à-dire que $x$ s'écrit avec un nombre fini de chiffres après la virgule.
	Alors $x$ s'écrit
		\[ x = a \times 10^n, \]
	pour un certain nombre décimal $a \in [1 ; 10[$, et $n\in\Z$ entier relatif, l'ordre de grandeur de $x$.
\end{definition*}

\exe{
	Sans calculatrice,  écrire les nombres de l'exercice \ref{ex:1} en notation scientifique.
}{}

\exe{
	Sans calculatrice, écrire les nombres suivants en notation scientifique.
		
	\begin{multicols}{3}
	\begin{enumerate}[label=\alph*)]
		\item $201$
		\item $10$
		\item $123 400 000$
		\item $0,8$
		\item $0,000327$
		\item $0,0090001$
	\end{enumerate}
	\end{multicols}
}{}


\begin{definition*}{encadrement}
	On dit que $a$ et $b$ \emph{encadrent} le nombre réel $x\in\R$ si $a < x < b$.
	\begin{enumerate}[label=$\bullet$]
		\item $b-a$ est l'\emph{amplitude} de l'encadrement.
		\item L'encadrement est à $10^{n}$ près si son amplitude est égale à $10^{n}$ (pour $n\in\Z$ entier relatif).
	\end{enumerate}
\end{definition*}

\exe{[Vrai ou faux]
	L'encadrement 
	
	\def\arraystretch{2}
	\setlength\tabcolsep{15pt}
	\begin{tabular}{c c c}
		\hspace{10cm} & Vrai & Faux \\
		$2,6 < \sqrt{7} < 2,8$ est à $10^{-1}$ près & $\square$ & $\square$  \\
		$3,14 < \pi < 3,15$ est à $10^{-2}$ près & $\square$ & $\square$  \\
		$-4,473 < -2\sqrt5 < -4,472$ est à $10^{-3}$ près & $\square$ & $\square$  \\
		$3,3 \times 10^{-4} < 3,3931 \times 10^{-4} < 3,4 \times 10^{-4}$ est à $10^{-5}$ près & $\square$ & $\square$  \\
	\end{tabular}
}{}

\newpage

\exe{
	Encadrer les nombres suivants à l'amplitude demandée.
	\begin{multicols}{2}
	\begin{enumerate}[label=\alph*)]
		\item $9 854,698 \times 10^3$ à $10^4$ près.
		\item $36,05 \times 10^{-4}$ à $10^{-1}$ près.
		\item $-31,45$ à $10^{-1}$ près.
		\item $-0,0125$ à $10^{-4}$ près.
	\end{enumerate}
	\end{multicols}
}{}

\exe{
	On donne l'encadrement du nombre d'or $\phi =  \dfrac{1+\sqrt5}2$ suivant.
		\[ 1,61803 < \dfrac{1+\sqrt5}2 < 1,618035 \]
	\begin{enumerate}
		\item Donner l'amplitude de l'encadrement en notation scientifique.
		\item Trouver un encadrement de $\sqrt{5}$ et donner son amplitude.
	\end{enumerate}
}{}

% pour échantillonnage je pense
% \hrule
%
%\exe{
%	Donner les ensembles suivants sous forme d'intervalle.
%		\begin{multicols}{2}
%		\begin{enumerate}
%			\item $ B_{0;1} = \bigl\{ x \in \R \text{ tq. } |x| \leq 1 \bigr\}$
%			\item $ B_{0;2} = \bigl\{ x \in \R \text{ tq. } |x| \leq 2 \bigr\} $
%			\item $ B_{0;10^{-2}} = \bigl\{ x \in \R \text{ tq. } |x| \leq 10^{-2} \bigr\} $
%			\item $ B_{5;1} = \bigl\{ x \in \R \text{ tq. } |x-5| \leq 1 \bigr\} $
%			\item $ B_{-3; 2} = \bigl\{ x \in \R \text{ tq. } |x+3| \leq 1 \bigr\}$
%			\item $ B_{-10; 10^{-1}} = \bigl\{ x \in \R \text{ tq. } |x+10| \leq 10^{-1} \bigr\}$
%		\end{enumerate}
%		\end{multicols}
%}{}
%
%\begin{theoreme*}{intervalle symétrique}
%	Soit $c \in \R$ un réel, et $r \in \R_+$ un réel positif ou nul. 
%	Alors l'ensemble
%		\[ B_{c, r} = \bigl\{ x \in \R \text{ tq. } |x - c| \leq r \bigr\} \]
%	est égal à l'intervalle contenant tous les nombres réels à distance inférieure à $r$ de $c$.
%		\[ B_{c,r} = [c -r ; c+r] \]
%\end{theoreme*}
%
%\exe{
%	
%}{}

\end{document}

				% ENABLE or DISABLE font change
				% use XeLaTeX if true
\newif\ifdys
				\dystrue
				\dysfalse

\newif\ifsolutions
				\solutionstrue
				\solutionsfalse

% DYSLEXIA SWITCH
\newif\ifdys
		
				% ENABLE or DISABLE font change
				% use XeLaTeX if true
				\dystrue
				\dysfalse


\ifdys

\documentclass[a4paper, 14pt]{extarticle}
\usepackage{amsmath,amsfonts,amsthm,amssymb,mathtools}

\tracinglostchars=3 % Report an error if a font does not have a symbol.
\usepackage{fontspec}
\usepackage{unicode-math}
\defaultfontfeatures{ Ligatures=TeX,
                      Scale=MatchUppercase }

\setmainfont{OpenDyslexic}[Scale=1.0]
\setmathfont{Fira Math} % Or maybe try KPMath-Sans?
\setmathfont{OpenDyslexic Italic}[range=it/{Latin,latin}]
\setmathfont{OpenDyslexic}[range=up/{Latin,latin,num}]

\else

\documentclass[a4paper, 12pt]{extarticle}

\usepackage[utf8x]{inputenc}
%fonts
\usepackage{amsmath,amsfonts,amsthm,amssymb,mathtools}
% comment below to default to computer modern
\usepackage{libertinus,libertinust1math}

\fi


\usepackage[french]{babel}
\usepackage[
a4paper,
margin=2cm,
nomarginpar,% We don't want any margin paragraphs
]{geometry}
\usepackage{icomma}

\usepackage{fancyhdr}
\usepackage{array}
\usepackage{hyperref}

\usepackage{multicol, enumerate}
\newcolumntype{P}[1]{>{\centering\arraybackslash}p{#1}}


\usepackage{stackengine}
\newcommand\xrowht[2][0]{\addstackgap[.5\dimexpr#2\relax]{\vphantom{#1}}}

% theorems

\theoremstyle{plain}
\newtheorem{theorem}{Th\'eor\`eme}
\newtheorem*{sol}{Solution}
\theoremstyle{definition}
\newtheorem{ex}{Exercice}
\newtheorem*{rpl}{Rappel}
\newtheorem{enigme}{Énigme}

% corps
\usepackage{calrsfs}
\newcommand{\C}{\mathcal{C}}
\newcommand{\R}{\mathbb{R}}
\newcommand{\Rnn}{\mathbb{R}^{2n}}
\newcommand{\Z}{\mathbb{Z}}
\newcommand{\N}{\mathbb{N}}
\newcommand{\Q}{\mathbb{Q}}

% variance
\newcommand{\Var}[1]{\text{Var}(#1)}

% domain
\newcommand{\D}{\mathcal{D}}


% date
\usepackage{advdate}
\AdvanceDate[0]


% plots
\usepackage{pgfplots}

% table line break
\usepackage{makecell}
%tablestuff
\def\arraystretch{2}
\setlength\tabcolsep{15pt}

%subfigures
\usepackage{subcaption}

\definecolor{myg}{RGB}{56, 140, 70}
\definecolor{myb}{RGB}{45, 111, 177}
\definecolor{myr}{RGB}{199, 68, 64}

% fake sections with no title to move around the merged pdf
\newcommand{\fakesection}[1]{%
  \par\refstepcounter{section}% Increase section counter
  \sectionmark{#1}% Add section mark (header)
  \addcontentsline{toc}{section}{\protect\numberline{\thesection}#1}% Add section to ToC
  % Add more content here, if needed.
}


% SOLUTION SWITCH
\newif\ifsolutions
				\solutionstrue
				%\solutionsfalse

\ifsolutions
	\newcommand{\exe}[2]{
		\begin{ex} #1  \end{ex}
		\begin{sol} #2 \end{sol}
	}
\else
	\newcommand{\exe}[2]{
		\begin{ex} #1  \end{ex}
	}
	
\fi


% tableaux var, signe
\usepackage{tkz-tab}


%pinfty minfty
\newcommand{\pinfty}{{+}\infty}
\newcommand{\minfty}{{-}\infty}

\begin{document}


\AdvanceDate[0]

\begin{document}
\pagestyle{fancy}
\fancyhead[L]{Seconde 13}
\fancyhead[C]{\textbf{Algorithmique : complexité et temps d'execution \ifsolutions \, -- Solutions  \fi}}
\fancyhead[R]{\today}

\exe{
	On considère différentes fonctions de la figure \ref{fig:1} qui servent à calculer la somme
		\[ S = 1 + 2 + \dots + (N-1) + N, \]
	pour un $N\in\N$ entier naturel donné.
	
	Implémenter (compléter?) les algorithmes et répondre aux questions suivantes.
	
	\begin{enumerate}
		\item Appeler les fonctions \texttt{somme(N)} et \texttt{sommeclose(N)} pour $N=0, 1, 5, 10, 2^{10}, 2^{25}, 2^{28}$.
		Que dire sur les valeurs retournées ? Que dire sur le temps d'execution ?
		\item Combien d'opérations arithmétiques sont nécessaires pour que \texttt{somme(N)} termine ? 
		\item Combien d'opérations arithmétiques sont nécessaires pour que \texttt{somme\_close(N)} termine ? 
	\end{enumerate}
}{}

\begin{figure}[h!]
\begin{subfigure}{.5\textwidth}
\begin{mintedbox}{python}
def somme(N):
	S = 0
	for i in range(1,N+1):
		S = S+i
	return S
\end{mintedbox}
\caption{Algorithme 1}
\end{subfigure}
\begin{subfigure}{.5\textwidth}
\begin{mintedbox}{python}
def sommeclose(N):
	return N*(N+1)/2
\end{mintedbox}
\caption{Algorithme 2}
\end{subfigure}
\caption{Algorithmes de calcul de somme $S = 1 + \dots N$.}
\label{fig:1}
\end{figure}

\exe{
	On considère différentes fonctions de la figure \ref{fig:2} qui servent à calculer la somme
		\[ S = 1^3 + 2^3 + \dots + (N-1)^3 + N^3, \]
	pour un $N\in\N$ entier naturel donné.
	
	Implémenter (compléter?) les algorithmes et répondre aux questions suivantes.
	
	\begin{enumerate}
		\item Appeler les fonctions \texttt{somme3(N)} et \texttt{somme3close(N)} pour $N=1, 2, 5, 9, 2^{10}, 2^{23}, 2^{25}$.
		Que dire sur les valeurs retournées ? Que dire sur le temps d'execution ?
		\item Combien d'opérations arithmétiques sont nécessaires pour que \texttt{somme3(N)} termine ? 
		\item Combien d'opérations arithmétiques sont nécessaires pour que \texttt{somme3\_close(N)} termine ? 
	\end{enumerate}
}{}

\begin{figure}[h!]
\begin{subfigure}{.5\textwidth}
\begin{mintedbox}{python}
def somme3(N):
	S = 0
	for i in range(1,N+1):
		S = S+i**3
	return S
\end{mintedbox}
\caption{Algorithme 1}
\label{fig:2a}
\end{subfigure}
\begin{subfigure}{.5\textwidth}
\begin{mintedbox}{python}
def somme3_close(N):
	return (N*(N+1)/2)**2
\end{mintedbox}
\caption{Algorithme 2}
\end{subfigure}
\caption{Algorithmes de calcul de somme $S = 1^3 + \dots N^3$.}
\label{fig:2}
\end{figure}

\newpage

\exe{[Exponentiation rapide]
	Le but de l'exercice est de calculer $q^N$ le plus vite possible, où $q, N\in\N$ sont deux entiers naturels.
	On se restreint aux opérations $+$ et $\times$ afin de pouvoir compter les opérations arithmétique de base (la puissance n'en est pas une !).
	
	\begin{enumerate}
		\item Créer une fonction \texttt{puissance(q,N)} qui retourne $q^N$ en utilisant uniquement la multiplication.
		\item Combien de multiplications sont nécessaires pour  \texttt{puissance(q,N)} termine ?
		\item Considérons l'algorithme \texttt{fastexp}. Que retourne l'appel \texttt{fastexp(q, 1)} ? et \texttt{fastexp(q, 2)},  \texttt{fastexp(q, 3)} ?
		\item En déduire que si $N=2^n$, alors  \texttt{fastexp(q, n)} retourne $q^N$ en $n$ opérations arithmétiques.
		\item Comparer le nombre d'opérations arithmétiques nécessaires au calcul de $q^{1024}$ en utilisant \texttt{puissance(q,1024)} versus \texttt{fastexp(q, 10)}
	\end{enumerate}
}{}

% fast exponentiation
\begin{mintedbox}{python}
def fastexp(q, n):
	r = q
	for i in range(1,n+1):
		r = r**2
	return r
\end{mintedbox}

% SOMME(N) RECURSIVE
%\begin{mintedbox}{python}
%def somme_recursive(N):
%	if N==0:
%		return 0
%	return N+somme_recursive(N-1)
%\end{mintedbox}

\end{document}

				% ENABLE or DISABLE font change
				% use XeLaTeX if true
\newif\ifdys
				\dystrue
				\dysfalse

\newif\ifsolutions
				\solutionstrue
				\solutionsfalse

% DYSLEXIA SWITCH
\newif\ifdys
		
				% ENABLE or DISABLE font change
				% use XeLaTeX if true
				\dystrue
				\dysfalse


\ifdys

\documentclass[a4paper, 14pt]{extarticle}
\usepackage{amsmath,amsfonts,amsthm,amssymb,mathtools}

\tracinglostchars=3 % Report an error if a font does not have a symbol.
\usepackage{fontspec}
\usepackage{unicode-math}
\defaultfontfeatures{ Ligatures=TeX,
                      Scale=MatchUppercase }

\setmainfont{OpenDyslexic}[Scale=1.0]
\setmathfont{Fira Math} % Or maybe try KPMath-Sans?
\setmathfont{OpenDyslexic Italic}[range=it/{Latin,latin}]
\setmathfont{OpenDyslexic}[range=up/{Latin,latin,num}]

\else

\documentclass[a4paper, 12pt]{extarticle}

\usepackage[utf8x]{inputenc}
%fonts
\usepackage{amsmath,amsfonts,amsthm,amssymb,mathtools}
% comment below to default to computer modern
\usepackage{libertinus,libertinust1math}

\fi


\usepackage[french]{babel}
\usepackage[
a4paper,
margin=2cm,
nomarginpar,% We don't want any margin paragraphs
]{geometry}
\usepackage{icomma}

\usepackage{fancyhdr}
\usepackage{array}
\usepackage{hyperref}

\usepackage{multicol, enumerate}
\newcolumntype{P}[1]{>{\centering\arraybackslash}p{#1}}


\usepackage{stackengine}
\newcommand\xrowht[2][0]{\addstackgap[.5\dimexpr#2\relax]{\vphantom{#1}}}

% theorems

\theoremstyle{plain}
\newtheorem{theorem}{Th\'eor\`eme}
\newtheorem*{sol}{Solution}
\theoremstyle{definition}
\newtheorem{ex}{Exercice}
\newtheorem*{rpl}{Rappel}
\newtheorem{enigme}{Énigme}

% corps
\usepackage{calrsfs}
\newcommand{\C}{\mathcal{C}}
\newcommand{\R}{\mathbb{R}}
\newcommand{\Rnn}{\mathbb{R}^{2n}}
\newcommand{\Z}{\mathbb{Z}}
\newcommand{\N}{\mathbb{N}}
\newcommand{\Q}{\mathbb{Q}}

% variance
\newcommand{\Var}[1]{\text{Var}(#1)}

% domain
\newcommand{\D}{\mathcal{D}}


% date
\usepackage{advdate}
\AdvanceDate[0]


% plots
\usepackage{pgfplots}

% table line break
\usepackage{makecell}
%tablestuff
\def\arraystretch{2}
\setlength\tabcolsep{15pt}

%subfigures
\usepackage{subcaption}

\definecolor{myg}{RGB}{56, 140, 70}
\definecolor{myb}{RGB}{45, 111, 177}
\definecolor{myr}{RGB}{199, 68, 64}

% fake sections with no title to move around the merged pdf
\newcommand{\fakesection}[1]{%
  \par\refstepcounter{section}% Increase section counter
  \sectionmark{#1}% Add section mark (header)
  \addcontentsline{toc}{section}{\protect\numberline{\thesection}#1}% Add section to ToC
  % Add more content here, if needed.
}


% SOLUTION SWITCH
\newif\ifsolutions
				\solutionstrue
				%\solutionsfalse

\ifsolutions
	\newcommand{\exe}[2]{
		\begin{ex} #1  \end{ex}
		\begin{sol} #2 \end{sol}
	}
\else
	\newcommand{\exe}[2]{
		\begin{ex} #1  \end{ex}
	}
	
\fi


% tableaux var, signe
\usepackage{tkz-tab}


%pinfty minfty
\newcommand{\pinfty}{{+}\infty}
\newcommand{\minfty}{{-}\infty}

\begin{document}


\AdvanceDate[0]

\begin{document}
\pagestyle{fancy}
\fancyhead[L]{Seconde 13}
\fancyhead[C]{\textbf{Vecteurs $\star$ \ifsolutions -- Solutions  \fi}}
\fancyhead[R]{\today}

\exe{[$\star$] \label{ex:1}
	Considérons $A(x_A ; y_A)$ et $B(x_B;y_B)$ deux points différents de $O(0;0)$, l'origine du plan.
	Le théorème de Pythagore énonce que le triangle $OAB$ est rectangle en $O$ si et seulement si
		\[ OA^2 + OB^2 = AB^2. \]
	Démontrer que cette condition est équivalente à
		\[ x_A \cdot x_B + y_A \cdot y_B = 0. \]
}{}

\exe{[$\star$]
	%On supposera $a \cdot a' \neq0$, c'est-à-dire que ni $\C_f$ ni $\C_g$ n'est horizontale.
	
	\begin{enumerate}		
		
		\item
		Esquisser les courbes des fonctions $f(x) = 2x -3$ et $g(x) = 1 -\dfrac12 x$ sur $[0 ; 4]$.
		
		\item 
		Soient $f(x) = ax$ et $g(x)=a'x$ deux fonctions affines sur $\R$ passant par l'origine.
		Montrer que $(1 ; a) \in \C_f$ et que $(1 ;a') \in \C_g$.
		
		\item
		Montrer que $\C_f$ et $\C_g$ sont perpendiculaires si et seulement si $1 + a a' =0$ à l'aide de l'exercice \ref{ex:1}
		
		\item
		Conclure que, pour deux fonctions affines $F$ et $G$ quelconques (ne passant pas forcément par l'origine), les droites $\C_F$ et $\C_G$ sont perpendiculaires si et seulement si le produit des coefficients directeurs vaut $-1$.
	\end{enumerate}
}{}

\exe{[$\star$]
	Considérons un point $B = (0;b)$ du plan, et un vecteur $A = (1;a)$.
	
	Montrer que l'ensemble 
		\[ E = \{ B + \lambda \cdot A \text{ où $\lambda$ parcourt $\R$} \} \]
	est égal à $\C_f$, où $f(x) = ax+b$, définie sur $\R$.
}{}

\exe{[$\star\star$]
	On considère deux points $A = (3 ; -2)$ et $B = (-2 ; 5)$.
	Le but de l'exercice est de construire le projeté orthogonal $P$ de $B$ sur la droite $(OA)$, ce qui nous permettra de recréer le devoir maison 2.
	
	\begin{enumerate}
		\item
		Dessiner les vecteurs $\vec{OA}$ et $\vec{OB}$.
		\item
		Tracer la droite donnée par l'ensemble des points $B + \lambda \cdot \vec{OA}$, où $\lambda$ parcourt $\R$.
		\item
		Dessiner le point $\tilde{P}$ de la droite de la question précédente vérifiant que le triangle $OB\tilde{P}$ est rectangle en $\tilde{P}$.
		\item
		En utilisant l'exercice $\ref{ex:1}$, montrer que $\tilde{P} = B - \lambda \cdot A$, pour le $\lambda$ vérifiant
			\[ \text{$\vec{O\tilde{P}}$ est perpendiculaire à $\vec{OA}$} \qquad \iff \qquad -16 -13 \lambda = 0. \]
		\item
		Vérifier que $P = \lambda \cdot A$ en démontrant que le triangle $BPA$ est rectangle en $P$ à l'aide de la réciproque de Pythagore.
	\end{enumerate}

}{}


\exe{[$\star$]
	Considérons $A, A', B, B'$ quatre points tels que $x_A \neq x_B$ et $x_{A'} \neq x_{B'}$.
	
	Montrer que les droites $(AB)$ et $(A'B')$ sont parallèles si et seulement si les vecteurs $\vec{AB}$ et $\vec{A'B'}$ sont colinéaires.
}{}

\exe{[$\star\star$]
	Montrer que $\det(u, v) = 0$ si et seulement si les vecteur $u$ et $v$ sont colinéaires.
}{}

\exe{[$\star\star$]
	Soit $A, B, C$ trois points quelconques.
	
	Montrer que $\vec{AB}$ et $\vec{AC}$ sont colinéaires si et seulement si les points $A, B, C$ sont alignés.
}{}


\end{document}

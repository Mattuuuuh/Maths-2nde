				% ENABLE or DISABLE font change
				% use XeLaTeX if true
\newif\ifdys
				\dystrue
				\dysfalse

\newif\ifsolutions
				\solutionstrue
				\solutionsfalse

%!TEX encoding = UTF8
%!TEX root =notes.tex


%%%%%%%%%%%%%%%%%%%%%%%%%%%%%%%%%
% PACKAGE IMPORTS
%%%%%%%%%%%%%%%%%%%%%%%%%%%%%%%%%


\usepackage[french]{babel}

\usepackage[tmargin=2cm,rmargin=1in,lmargin=1in,margin=0.85in,bmargin=2cm,footskip=.2in]{geometry}
\usepackage{amsmath,amsfonts,amsthm,amssymb,mathtools}
\usepackage[varbb]{newpxmath}
\usepackage{xfrac}
\usepackage[makeroom]{cancel}
\usepackage{mathtools}
\usepackage{bookmark}
\usepackage{enumitem}
\usepackage{hyperref,theoremref}
\hypersetup{
	pdftitle={Assignment},
	colorlinks=true, linkcolor=doc!90,
	bookmarksnumbered=true,
	bookmarksopen=true
}
\usepackage[most,many,breakable]{tcolorbox}
\usepackage{xcolor}
\usepackage{varwidth}
\usepackage{varwidth}
\usepackage{etoolbox}
%\usepackage{authblk}
\usepackage{nameref}
\usepackage{multicol,array}
\usepackage{tikz-cd}
\usepackage[ruled,vlined,linesnumbered]{algorithm2e}
\usepackage{comment} % enables the use of multi-line comments (\ifx \fi) 
\usepackage{import}
\usepackage{xifthen}
\usepackage{pdfpages}
\usepackage{transparent}


\newcommand\mycommfont[1]{\footnotesize\ttfamily\textcolor{blue}{#1}}
\SetCommentSty{mycommfont}
\newcommand{\incfig}[1]{%
    \def\svgwidth{\columnwidth}
    \import{./figures/}{#1.pdf_tex}
}

\usepackage{tikzsymbols}
%\renewcommand\qedsymbol{$\Laughey$}


%\usepackage{import}
%\usepackage{xifthen}
%\usepackage{pdfpages}
%\usepackage{transparent}


%%%%%%%%%%%%%%%%%%%%%%%%%%%%%%
% SELF MADE COLORS
%%%%%%%%%%%%%%%%%%%%%%%%%%%%%%



\definecolor{myg}{RGB}{56, 140, 70}
\definecolor{myb}{RGB}{45, 111, 177}
\definecolor{myr}{RGB}{199, 68, 64}
\definecolor{mytheorembg}{HTML}{F2F2F9}
\definecolor{mytheoremfr}{HTML}{00007B}
\definecolor{mylenmabg}{HTML}{FFFAF8}
\definecolor{mylenmafr}{HTML}{983b0f}
\definecolor{mypropbg}{HTML}{f2fbfc}
\definecolor{mypropfr}{HTML}{191971}
\definecolor{myexamplebg}{HTML}{F2FBF8}
\definecolor{myexamplefr}{HTML}{88D6D1}
\definecolor{myexampleti}{HTML}{2A7F7F}
\definecolor{mydefinitbg}{HTML}{E5E5FF}
\definecolor{mydefinitfr}{HTML}{3F3FA3}
\definecolor{notesgreen}{RGB}{0,162,0}
\definecolor{myp}{RGB}{197, 92, 212}
\definecolor{mygr}{HTML}{2C3338}
\definecolor{myred}{RGB}{127,0,0}
\definecolor{myyellow}{RGB}{169,121,69}
\definecolor{myexercisebg}{HTML}{F2FBF8}
\definecolor{myexercisefg}{HTML}{88D6D1}


%%%%%%%%%%%%%%%%%%%%%%%%%%%%
% TCOLORBOX SETUPS
%%%%%%%%%%%%%%%%%%%%%%%%%%%%

\setlength{\parindent}{1cm}
%================================
% THEOREM BOX
%================================

\tcbuselibrary{theorems,skins,hooks}
\newtcbtheorem[number within=chapter]{Theorem}{Théorème}
{%
	enhanced,
	breakable,
	colback = mytheorembg,
	frame hidden,
	boxrule = 0sp,
	borderline west = {2pt}{0pt}{mytheoremfr},
	sharp corners,
	detach title,
	before upper = \tcbtitle\par\smallskip,
	coltitle = mytheoremfr,
	fonttitle = \bfseries\sffamily,
	description font = \mdseries,
	separator sign none,
	segmentation style={solid, mytheoremfr},
}
{th}


\tcbuselibrary{theorems,skins,hooks}
\newtcolorbox{Theoremcon}
{%
	enhanced
	,breakable
	,colback = mytheorembg
	,frame hidden
	,boxrule = 0sp
	,borderline west = {2pt}{0pt}{mytheoremfr}
	,sharp corners
	,description font = \mdseries
	,separator sign none
}

%================================
% Corollery
%================================
\tcbuselibrary{theorems,skins,hooks}
\newtcbtheorem[use counter=tcb@cnt@Theorem]{Corollary}{Corollaire}
{%
	enhanced
	,breakable
	,colback = myp!10
	,frame hidden
	,boxrule = 0sp
	,borderline west = {2pt}{0pt}{myp!85!black}
	,sharp corners
	,detach title
	,before upper = \tcbtitle\par\smallskip
	,coltitle = myp!85!black
	,fonttitle = \bfseries\sffamily
	,description font = \mdseries
	,separator sign none
	,segmentation style={solid, myp!85!black}
}
{th}

%================================
% LENMA
%================================

\tcbuselibrary{theorems,skins,hooks}
\newtcbtheorem[use counter=tcb@cnt@Theorem]{Lemma}{Lemme}
{%
	enhanced,
	breakable,
	colback = mylenmabg,
	frame hidden,
	boxrule = 0sp,
	borderline west = {2pt}{0pt}{mylenmafr},
	sharp corners,
	detach title,
	before upper = \tcbtitle\par\smallskip,
	coltitle = mylenmafr,
	fonttitle = \bfseries\sffamily,
	description font = \mdseries,
	separator sign none,
	segmentation style={solid, mylenmafr},
}
{th}


%================================
% PROPOSITION
%================================

\tcbuselibrary{theorems,skins,hooks}
\newtcbtheorem[use counter=tcb@cnt@Theorem]{Prop}{Proposition}
{%
	enhanced,
	breakable,
	colback = mypropbg,
	frame hidden,
	boxrule = 0sp,
	borderline west = {2pt}{0pt}{mypropfr},
	sharp corners,
	detach title,
	before upper = \tcbtitle\par\smallskip,
	coltitle = mypropfr,
	fonttitle = \bfseries\sffamily,
	description font = \mdseries,
	separator sign none,
	segmentation style={solid, mypropfr},
}
{th}


%================================
% CLAIM
%================================

\tcbuselibrary{theorems,skins,hooks}
\newtcbtheorem[use counter=tcb@cnt@Theorem]{claim}{Claim}
{%
	enhanced
	,breakable
	,colback = myg!10
	,frame hidden
	,boxrule = 0sp
	,borderline west = {2pt}{0pt}{myg}
	,sharp corners
	,detach title
	,before upper = \tcbtitle\par\smallskip
	,coltitle = myg!85!black
	,fonttitle = \bfseries\sffamily
	,description font = \mdseries
	,separator sign none
	,segmentation style={solid, myg!85!black}
}
{th}



%================================
% Exercise
%================================

\tcbuselibrary{theorems,skins,hooks}
\newtcbtheorem[use counter=tcb@cnt@Theorem]{Exercise}{Exercice}
{%
	enhanced,
	breakable,
	colback = myexercisebg,
	frame hidden,
	boxrule = 0sp,
	borderline west = {2pt}{0pt}{myexercisefg},
	sharp corners,
	detach title,
	before upper = \tcbtitle\par\smallskip,
	coltitle = myexercisefg,
	fonttitle = \bfseries\sffamily,
	description font = \mdseries,
	separator sign none,
	segmentation style={solid, myexercisefg},
}
{th}

%================================
% EXAMPLE BOX
%================================

\newtcbtheorem[use counter=tcb@cnt@Theorem]{Example}{Exemple}
{%
	colback = myexamplebg
	,breakable
	,colframe = myexamplefr
	,coltitle = myexampleti
	,boxrule = 1pt
	,sharp corners
	,detach title
	,before upper=\tcbtitle\par\smallskip
	,fonttitle = \bfseries
	,description font = \mdseries
	,separator sign none
	,description delimiters parenthesis
}
{ex}

%================================
% DEFINITION BOX
%================================

\newtcbtheorem[use counter=tcb@cnt@Theorem]{Definition}{Définition}{enhanced,
	before skip=2mm,after skip=2mm, colback=red!5,colframe=red!80!black,boxrule=0.5mm,
	attach boxed title to top left={xshift=1cm,yshift*=1mm-\tcboxedtitleheight}, varwidth boxed title*=-3cm,
	boxed title style={frame code={
					\path[fill=tcbcolback]
					([yshift=-1mm,xshift=-1mm]frame.north west)
					arc[start angle=0,end angle=180,radius=1mm]
					([yshift=-1mm,xshift=1mm]frame.north east)
					arc[start angle=180,end angle=0,radius=1mm];
					\path[left color=tcbcolback!60!black,right color=tcbcolback!60!black,
						middle color=tcbcolback!80!black]
					([xshift=-2mm]frame.north west) -- ([xshift=2mm]frame.north east)
					[rounded corners=1mm]-- ([xshift=1mm,yshift=-1mm]frame.north east)
					-- (frame.south east) -- (frame.south west)
					-- ([xshift=-1mm,yshift=-1mm]frame.north west)
					[sharp corners]-- cycle;
				},interior engine=empty,
		},
	fonttitle=\bfseries,
	title={#2},#1}{def}

%================================
% Solution BOX
%================================

\makeatletter
\newtcbtheorem[use counter=tcb@cnt@Theorem]{question}{Question}{enhanced,
	breakable,
	colback=white,
	colframe=myb!80!black,
	attach boxed title to top left={yshift*=-\tcboxedtitleheight},
	fonttitle=\bfseries,
	title={#2},
	boxed title size=title,
	boxed title style={%
			sharp corners,
			rounded corners=northwest,
			colback=tcbcolframe,
			boxrule=0pt,
		},
	underlay boxed title={%
			\path[fill=tcbcolframe] (title.south west)--(title.south east)
			to[out=0, in=180] ([xshift=5mm]title.east)--
			(title.center-|frame.east)
			[rounded corners=\kvtcb@arc] |-
			(frame.north) -| cycle;
		},
	#1
}{def}
\makeatother

%================================
% SOLUTION BOX
%================================

\makeatletter
\newtcolorbox{solution}{enhanced,
	breakable,
	colback=white,
	colframe=myg!80!black,
	attach boxed title to top left={yshift*=-\tcboxedtitleheight},
	title=Solution,
	boxed title size=title,
	boxed title style={%
			sharp corners,
			rounded corners=northwest,
			colback=tcbcolframe,
			boxrule=0pt,
		},
	underlay boxed title={%
			\path[fill=tcbcolframe] (title.south west)--(title.south east)
			to[out=0, in=180] ([xshift=5mm]title.east)--
			(title.center-|frame.east)
			[rounded corners=\kvtcb@arc] |-
			(frame.north) -| cycle;
		},
}
\makeatother

%================================
% Question BOX
%================================

\makeatletter
\newtcbtheorem[use counter=tcb@cnt@Theorem]{qstion}{Question}{enhanced,
	breakable,
	colback=white,
	colframe=mygr,
	attach boxed title to top left={yshift*=-\tcboxedtitleheight},
	fonttitle=\bfseries,
	title={#2},
	boxed title size=title,
	boxed title style={%
			sharp corners,
			rounded corners=northwest,
			colback=tcbcolframe,
			boxrule=0pt,
		},
	underlay boxed title={%
			\path[fill=tcbcolframe] (title.south west)--(title.south east)
			to[out=0, in=180] ([xshift=5mm]title.east)--
			(title.center-|frame.east)
			[rounded corners=\kvtcb@arc] |-
			(frame.north) -| cycle;
		},
	#1
}{def}
\makeatother

\newtcbtheorem[number within=chapter]{wconc}{Wrong Concept}{
	breakable,
	enhanced,
	colback=white,
	colframe=myr,
	arc=0pt,
	outer arc=0pt,
	fonttitle=\bfseries\sffamily\large,
	colbacktitle=myr,
	attach boxed title to top left={},
	boxed title style={
			enhanced,
			skin=enhancedfirst jigsaw,
			arc=3pt,
			bottom=0pt,
			interior style={fill=myr}
		},
	#1
}{def}



%================================
% NOTE BOX
%================================

\usetikzlibrary{arrows,calc,shadows.blur}
\tcbuselibrary{skins}
\newtcolorbox{note}[1][]{%
	enhanced jigsaw,
	colback=gray!20!white,%
	colframe=gray!80!black,
	size=small,
	boxrule=1pt,
	title=\colorbox{white!100}{\textbf{ Remarque }},
	halign title=flush center,
	coltitle=black,
	breakable,
	drop shadow=black!50!white,
	attach boxed title to top left={xshift=1cm,yshift=-\tcboxedtitleheight/2,yshifttext=-\tcboxedtitleheight/2},
	minipage boxed title=2.6cm,
	boxed title style={%
			colback=white,
			size=fbox,
			boxrule=1pt,
			boxsep=2pt,
			underlay={%
					\coordinate (dotA) at ($(interior.west) + (-0.5pt,0)$);
					\coordinate (dotB) at ($(interior.east) + (0.5pt,0)$);
					\begin{scope}
						\clip (interior.north west) rectangle ([xshift=3ex]interior.east);
						\filldraw [white, blur shadow={shadow opacity=60, shadow yshift=-.75ex}, rounded corners=2pt] (interior.north west) rectangle (interior.south east);
					\end{scope}
					\begin{scope}[gray!80!black]
						\fill (dotA) circle (2pt);
						\fill (dotB) circle (2pt);
					\end{scope}
				},
		},
	#1,
}

%================================
% STRATÉGIE BOX
%================================

\usetikzlibrary{arrows,calc,shadows.blur}
\tcbuselibrary{skins}
\newtcolorbox{strategy}[1][]{%
	enhanced jigsaw,
	colback=myb!20!white,%
	colframe=gray!80!black,
	size=small,
	boxrule=1pt,
	title=\colorbox{white!100}{\textbf{ Stratégie }},
	halign title=flush center,
	coltitle=black,
	breakable,
	drop shadow=black!50!white,
	attach boxed title to top left={xshift=1cm,yshift=-\tcboxedtitleheight/2,yshifttext=-\tcboxedtitleheight/2},
	minipage boxed title=2.5cm,
	boxed title style={%
			colback=white,
			size=fbox,
			boxrule=1pt,
			boxsep=2pt,
			underlay={%
					\coordinate (dotA) at ($(interior.west) + (-0.5pt,0)$);
					\coordinate (dotB) at ($(interior.east) + (0.5pt,0)$);
					\begin{scope}
						\clip (interior.north west) rectangle ([xshift=3ex]interior.east);
						\filldraw [white, blur shadow={shadow opacity=60, shadow yshift=-.75ex}, rounded corners=2pt] (interior.north west) rectangle (interior.south east);
					\end{scope}
					\begin{scope}[gray!80!black]
						\fill (dotA) circle (2pt);
						\fill (dotB) circle (2pt);
					\end{scope}
				},
		},
	#1,
}

%================================
% MÉTHODE BOX
%================================

\usetikzlibrary{arrows,calc,shadows.blur}
\tcbuselibrary{skins}
\newtcolorbox{methode}[1][]{%
	enhanced jigsaw,
	colback=white,%
	colframe=gray!80!black,
	size=small,
	boxrule=1pt,
	title=\textbf{Méthode},
	halign title=flush center,
	coltitle=black,
	breakable,
	drop shadow=black!50!white,
	attach boxed title to top left={xshift=1cm,yshift=-\tcboxedtitleheight/2,yshifttext=-\tcboxedtitleheight/2},
	minipage boxed title=2.5cm,
	boxed title style={%
			colback=white,
			size=fbox,
			boxrule=1pt,
			boxsep=2pt,
			underlay={%
					\coordinate (dotA) at ($(interior.west) + (-0.5pt,0)$);
					\coordinate (dotB) at ($(interior.east) + (0.5pt,0)$);
					\begin{scope}
						\clip (interior.north west) rectangle ([xshift=3ex]interior.east);
						\filldraw [white, blur shadow={shadow opacity=60, shadow yshift=-.75ex}, rounded corners=2pt] (interior.north west) rectangle (interior.south east);
					\end{scope}
					\begin{scope}[gray!80!black]
						\fill (dotA) circle (2pt);
						\fill (dotB) circle (2pt);
					\end{scope}
				},
		},
	#1,
}

%%%%%%%%%%%%%%%%%%%%%%%%%%%%%%%%%%%%%%%%%%%
% TABLE OF CONTENTS
%%%%%%%%%%%%%%%%%%%%%%%%%%%%%%%%%%%%%%%%%%%

\usepackage{tikz}

\definecolor{doc}{RGB}{0,60,110}
\usepackage{titletoc}
\contentsmargin{0cm}
\titlecontents{chapter}[3.7pc]
{\addvspace{30pt}%
	\begin{tikzpicture}[remember picture, overlay]%
		\draw[fill=doc!60,draw=doc!60] (-7,-.1) rectangle (-0.2,.6);%
		\pgftext[left,x=-3.5cm,y=0.2cm]{\color{white}\Large\sc\bfseries Chapitre\ \thecontentslabel};%
	\end{tikzpicture}\color{doc!60}\large\sc\bfseries}%
{}
{}
{\;\titlerule\;\large\sc\bfseries Page \thecontentspage
	\begin{tikzpicture}[remember picture, overlay]
		\draw[fill=doc!60,draw=doc!60] (2pt,0) rectangle (4,0.1pt);
	\end{tikzpicture}}%
\titlecontents{section}[3.7pc]
{\addvspace{2pt}}
{\contentslabel[\thecontentslabel]{2pc}}
{}
{\hfill\small \thecontentspage}
[]
\titlecontents*{subsection}[3.7pc]
{\addvspace{-1pt}\small}
{}
{}
{\ --- \small\thecontentspage}
[ \textbullet\ ][]

\makeatletter
\renewcommand{\tableofcontents}{%
	\chapter*{%
	  \vspace*{-20\p@}%
	  \begin{tikzpicture}[remember picture, overlay]%
		  \pgftext[right,x=15cm,y=0.2cm]{\color{doc!60}\Huge\sc\bfseries \contentsname};%
		  \draw[fill=doc!60,draw=doc!60] (13,-.75) rectangle (20,1);%
		  \clip (13,-.75) rectangle (20,1);
		  \pgftext[right,x=15cm,y=0.2cm]{\color{white}\Huge\sc\bfseries \contentsname};%
	  \end{tikzpicture}}%
	\@starttoc{toc}}
\makeatother


%%%%%%%%%%%%%%%%%%%%%%%%%%%%%%%%%%%%%%%%%%%
% MINTED FOR PYTHON ALGORITHMS
%%%%%%%%%%%%%%%%%%%%%%%%%%%%%%%%%%%%%%%%%%%

\usepackage{tcolorbox}
\tcbuselibrary{minted,breakable,xparse,skins}
\definecolor{bg}{gray}{0.95}
\DeclareTCBListing{mintedbox}{O{}m!O{}}{%
  breakable=true,
  listing engine=minted,
  listing only,
  minted language=#2,
  minted style=default,
  minted options={%
    linenos,
    gobble=0,
    breaklines=true,
    breakafter=,,
    fontsize=\small,
    numbersep=8pt,
    #1},
  boxsep=0pt,
  left skip=0pt,
  right skip=0pt,
  left=25pt,
  right=0pt,
  top=3pt,
  bottom=3pt,
  arc=5pt,
  leftrule=0pt,
  rightrule=0pt,
  bottomrule=2pt,
  toprule=2pt,
  colback=bg,
  colframe=orange!70,
  enhanced,
  overlay={%
    \begin{tcbclipinterior}
    \fill[orange!20!white] (frame.south west) rectangle ([xshift=20pt]frame.north west);
    \end{tcbclipinterior}},
  #3}
  
  
 % for braces
\usetikzlibrary{decorations.pathreplacing}


\AdvanceDate[2]

\begin{document}
\pagestyle{fancy}
\fancyhead[L]{Seconde 13}
\fancyhead[C]{\textbf{Évaluation blanche -- Vecteurs \ifsolutions -- Solutions  \fi}}
\fancyhead[R]{\today}

\exe{[Calcul]
	Soient $A(3;-1), B(-1; -5),$ et $C(3 ; 4)$.
	\begin{multicols}{2}
	\begin{enumerate}
		\item Calculer $\vec{AB}, \vec{BA}$, et $\vec{CA}$.
		\item Calculer $\norm{\vec{AB}}, \norm{\vec{BA}}$, $\norm{\vec{AC}}$.
		\item Calculer $3 \vec{AB} + 3 \vec{CA}$ et $3 \vec{CB}$.
		\item Calculer $\norm{-4 \vec{AB}}$ et $\norm{-\dfrac1{13} \vec{CA}}$.
	\end{enumerate}
	\end{multicols}

}{

	\begin{enumerate}
		\item 
			\begin{align*}
				\vec{AB} = B - A = \pvec{-4}{-4} && \vec{BA} = - \vec{AB} = \pvec44 && \vec{CA} = A - C = \pvec0{-5}.
			\end{align*}
		\item
			\begin{align*}
				\norm{\vec{AB}} &= \sqrt{(-4)^2 + (-4)^2} = 4\sqrt{2} \\ \norm{\vec{BA}} &= \norm{- \vec{AB}} = \norm{\vec{AB}} = 4\sqrt{2} \\ \norm{\vec{CA}} &= \sqrt{(-5)^2} = |-5| = 5.
			\end{align*}
		\item
			On utilise la relation de Chasle $\vec{CB} = \vec{CA} + \vec{AB}$ pour obtenir immédiatement le résultat.
			\begin{align*}
				3 \vec{AB} + 3 \vec{CA} &= 3 \pvec{-4}{-4} + 3 \pvec0{-5} = \pvec{-12}{-27} \\
				3 \vec{CB} &= 3 \left( \vec{CA} + \vec{AB} \right) =  3 \vec{AB} + 3 \vec{CA} = \pvec{-12}{-27}
			\end{align*}
		\item
			\begin{gather*}
				\norm{-4 \vec{AB}} = |-4| \cdot \norm{\vec{AB}} = 4\norm{\vec{AB}} = 16\sqrt{2}  \\
				\norm{-\dfrac1{13} \vec{CA}} = \dfrac1{13} \norm{\vec{CA}} = \dfrac5{13}
			\end{gather*}
			
	\end{enumerate}

}

\exe{[Représentation graphique]
	Construire, dans le plan vierge de droite, les sommes 
		\[u+v+w \qquad \text{ et } \qquad \dfrac12u - 2v - w,\]
	où les vecteurs $u, v, w$ sont donnés dans le plan de gauche ci-dessous.
	
	%\begin{center}
		\begin{tikzpicture}[>=stealth, scale=1]
		\begin{axis}[xmin = -10, xmax=10, ymin=-10, ymax=10, axis x line=none, axis y line=none, axis line style=<->, xlabel={}, ylabel={}, ticks = none]
			\draw[very thick, ->, myg] (axis cs:-3,-4) -- (axis cs: 0,0) node[above] {$u$};
			\draw[very thick, ->, myr] (axis cs:-2,0) -- (axis cs: -5,6) node[above] {$v$};
			\draw[very thick, ->, myb] (axis cs:0,3) -- (axis cs: 8,-2) node[above] {$w$};
		\end{axis}
		\end{tikzpicture}
		\vline
		\ifsolutions		
		\begin{tikzpicture}[>=stealth, scale=1]
		\begin{axis}[xmin = -10, xmax=20, ymin=-15, ymax=10, axis x line=none, axis y line=none, axis line style=<->, xlabel={}, ylabel={}, ticks = none]
			\draw[very thick, ->, black, dotted] (axis cs:-3,-4) -- (axis cs: 0,0) node[pos=.5, above left] {$u$};
			\draw[very thick, ->, black, dotted] (axis cs:-2+2,0) -- (axis cs: -5+2,6) node[pos=.5, left] {$v$};
			\draw[very thick, ->, black, dotted] (axis cs:0-3,3+3) -- (axis cs: 8-3,-2+3) node[pos=.5, above] {$w$};
			
			\draw[very thick, ->, red] (axis cs:-3,-4) -- (axis cs: 8-3,-2+3) node[pos=.5, below right] {$u+v+w$};
			
			
			
			\draw[very thick, ->, black, dotted] (axis cs:-3+15,-4) -- (axis cs:-1.5+15, -2) node[pos=.5, above left] {$\frac12u$};
			\draw[very thick, ->, black, dotted] (axis cs:-2+2-1.5+15, 0-2) -- (axis cs: 4+2-1.5+15, -12-2) node[pos=.5, above right] {$-2v$};
			\draw[very thick, ->, black, dotted] (axis cs:0+4+2-1.5+15,3-3-12-2) -- (axis cs: -8+4+2-1.5+15, 8-3-12-2) node[pos=.5, below left] {$-w$};
			
			\draw[very thick, ->, green](axis cs:-3+15,-4) -- (axis cs: -8+4+2-1.5+15, 8-3-12-2) node[pos=.8, left] {$\frac12u-2v-w$};
		\end{axis}
		\end{tikzpicture}
		\fi
	%\end{center}
}{
	On met les vecteurs bout à bout pour créer la somme.
	La multiplication par un scalaire ne change pas la direction mais multiplie la norme et peut changer le sens si celui-ci est négatif.
}


\exe{[Vrai ou faux]
	Pour chacune des propositions suivantes, montrer qu'elle est toujours vraie ou trouver un contre-exemple.
	\begin{enumerate}
		%\item $\vec{AB} = \vec{CD} \iff \vec{AC} = \vec{BD}$.
		\item Soient $u, v$ deux vecteurs tels que $v = 5u$.
		Alors $\det(u,v) = 0$.
		\item Si $\vec{AB}$ et $\vec{CD}$ sont colinéaires, alors les points $A, B, C$, et $D$ sont alignés.
		\item Pour $u, v$ deux vecteurs, on a $\det(u, v) = -\det(v, u)$.
		\item Si $\norm{v} = \sqrt{5}$, alors $v = \pvec{2}{1}$.
	\end{enumerate}
}{

	\begin{enumerate}
		%\item $\vec{AB} = \vec{CD} \iff \vec{AC} = \vec{BD}$.
		\item 
		C'est vrai : d'après le cours, si $u$ et $v$ sont colinéaires, alors $\det(u, v) = 0$.
		On peut aussi le démontrer par le calcul.
		
		\item 
		C'est faux en général : on sait que $(AB)$ et $(CD)$ sont parallèles, mais pas si les droites sont confondues.
		On pourra donc construire un contre-exemple comme $A(0;0), B(1;0)$ et $C(1;0), D(1;1)$.
		
		\item 
		C'est vrai par le calcul. Soit $u = \pvec{a}{b}$ et $v=\pvec{c}{d}$.
		Alors $\det(u, v) = ad - bc$ et $\det(v, u) = cb - ad$.
		
		\item 
		C'est faux car la norme d'un vecteur ne peut pas donner ses coordonnées (sauf si la norme est nulle).
		
		Par exemple, comme $\norm{v} = \norm{-v}$, on a que $v = \pvec{-2}{-1}$ est un contre-exemple.
		
		On pourrait aussi échanger les coordonnées pour avoir $v = \pvec12$ qui donne un autre contre-exemple.
	\end{enumerate}
}

\exe{[Parallélisme]

	Soit $u = \pvec{4}{2}$ et $v = \pvec{1}{3}$ deux vecteurs.
	Représenter dans un repère les points
		\begin{align*}
			A(-3;-1), && B = A+u, && C = A+u+v, && \text{ et } && D=A+v,
		\end{align*}
	et répondre aux questions suivantes.
	\begin{enumerate}
		\item Que dire du quadrilatère $ABCD$ visuellement ?
		\item Montrer que $(AB)$ et $(CD)$ sont parallèles puis que $(AD)$ et $(BC)$ sont parallèles.
		\item Démontrer de façon générale cette propriété pour $A, u, v$ un point et deux vecteurs quelconques.
	\end{enumerate}
}{
	\begin{enumerate}
		\item $ABCD$ semble être un parallélogramme car ses cotés opposés sont parallèles deux à deux.
		\item On calcule $\vec{AB} = B-A = u$ et $\vec{CD} = D-C = A+v - (A+u+v) = -u$ pour voir qu'ils sont coléinaires : les droites $(AB)$ et $(CD)$ sont donc parallèles.
		
		Idem avec $\vec{AD} = D - A = v$ et $\vec{BC} = C - B = A + u + v - (A + u) = v$, ce qui conclut similairement. 
		\item 
		La démonstration employée ci-dessus ne dépend aucunement des coordonnées de $A, u$, ou $v$.
		En mathématiques, il est souvent plus facile de calculer en gardant les lettres qu'en les remplaçant avec des nombres.
	\end{enumerate}
	
}

\ifsolutions
\else
\newpage
\fi

\exe{[Vecteur directeur]
	On considère un point $A(5 ; -12)$ et un vecteur $v = \pvec{2}{-1}$.
	Soit $(d)$ la droite passant par $A$ et dirigée par $v$.
	
	\begin{enumerate}
		\item
		Donner $4$ points distincts appartenant à $(d)$.
		\item
		Trouver la fonction affine $f$ telle que $(d) = \C_f$.
	\end{enumerate}
}{
	
	\begin{enumerate}
		\item
		D'après le cours, $A + k \cdot v$ appartient à la droite pour n'importe quel $k\in\R$.
		On prendra donc $k=0$ pour retrouver $A$, $k=1, 2, 3$ pour avoir $B(7;-13), C(9;-13)$, et $D(11;-14)$.
		
		\item
		On cherche $a$ et $b$ de l'expression algébrique de $f$ : $f(x) = ax+b$.
		
		$v$ est colinéaire au vecteur $\pvec{1}{-\frac12}$, ce qui nous donne immédiatement $a=-\dfrac12$ d'après le cours.
		On pourrait utiliser l'appartenance d'un point à $\C_f$ pour trouver $b$, mais on peut à la place créer un point d'abscisse nulle en utilisant le raisonnement de la question 1.
		
		En effet, $A + k \cdot v$ appartient toujours à la droite.
		Le $k\in\R$ donnant un point d'abscisse nulle vérifie donc $5 + 2k = 0 \iff k = -\dfrac52.$
		
		Il suit que $A - \dfrac52v = \left(0 ; -12 + \dfrac52 \right) = \left(0 ; \dfrac{-19}2 \right).$
		Par conséquent, $b = \dfrac{-19}2$, et 
			\[ f(x) = -\dfrac12 x - \dfrac{19}2. \]
	\end{enumerate}

}

\exe{[Quadrilatère]
	Considérons quatre points $A(-4; 1), B(-3 ; -3), C(4; -4), D(3; 1)$.
	
	Le quadralitère $ABCD$ est-il un parallélogramme ?
	Si non, donner un point $\tilde{D}$ tel que $ABC\tilde{D}$ en soit un.
}{
	Il s'agit de vérifier si $\vec{AB}$ et $\vec{CD}$ sont colinéaires et si $\vec{BC}$ et $\vec{AD}$ le sont aussi.
	On calcule
		\[ \vec{AB} = B-A = \pvec{1}{-4}, \qquad \qquad \vec{CD} = D - C = \pvec{-1}{5}. \]
	Ces deux vecteurs ne sont pas colinéaires car s'ils l'étaient, on aurait un $k \in \R$ tel que
		\[ \vec{AB} = k \vec{CD}, \]
	ce qui donnerait $k = -1$ en étudiant la première coordonnée, et $k=-\dfrac45$ en regardant la deuxième.
	Ça n'est bien sûr pas possible.
	
	En faisant un dessin, on remarque que le point $\tilde{D}$ doit nécessairement vérifier
		\[ \tilde{D} + \vec{AB} = C \qquad \iff \qquad \tilde{D} = C - \vec{AB} = (4 - 1 ; -4 - (-4) ) = (3 ;0). \]
	Par construction, $\vec{AB} = C- \tilde{D} = \vec{\tilde{D}C}$, qui montre que deux côtés opposés sont parallèles.
	Le parallélisme des deux autres côtés se fait de la même manière.
}

\exe{[Milieu]
	Soient $A, B, C$ trois points tels que $B$ soit le milieu du segment $[AC]$.
	Faire un dessin puis montrer que $\vec{AB} = \vec{BC}$.
}{
	La formule du milieu du cours
		\[ B = \dfrac{A+C}2 \]
	est équivalente à
		\[ 2B = A + C \qquad \iff \qquad B-A = C-B \qquad \iff \qquad \vec{AB} = \vec{BC}. \]
}

\exe{[Parallélisme]
	Soient les points $A(3;2), B(-3 ; 7), C(-2; -3), D(4;-6)$.
	
	Les droites $(AB)$ et $(CD)$ sont-elles parallèles ?
	
	Si non, donner un point $\tilde{B}$ tel que les droites $(A\tilde{B})$ et $(CD)$ soient parallèles.
}{
	Il s'agit de vérifier si $\vec{AB}$ et $\vec{CD}$ sont colinéaires.
	On calcule
		\[ \vec{AB} = B-A = \pvec{-6}{5}, \qquad \qquad \vec{CD} = D - C = \pvec{6}{-3}. \]
	Les vecteurs ne sont pas colinéaires : s'il existait un $k\in\R$ tel que $\vec{AB} = k \vec{CD}$, on aurait nécessairement $k=-1$ par la première coordonnée, et $k=-\dfrac53 \neq -1$ par le seconde.
	Les droites ne sont donc pas parallèles.
	
	Pour que $(A\tilde{B})$ et $(CD)$ soient parallèles, elles doivent partager un vecteur directeur.
	Par exemple, le vecteur $\vec{CD}$ convient.
	
	On peut alors choisir $\tilde{B}$ parmis les $A+k \cdot \vec{CD}$ où $k\in\R$ est un scalaire quelconque.
	En prenant $k=-\dfrac13$ pour montrer qu'on aime les fractions, on peut définir $\tilde{B} = (1 ; 3)$.
	
	On vérifiera que $\vec{A\tilde{B}} = \pvec{-2}{1}$, qui est bien colinéaire à $\vec{CD} = \pvec{6}{-3}$.
}


\end{document}
% DYSLEXIA SWITCH
\newif\ifdys
		
				% ENABLE or DISABLE font change
				% use XeLaTeX if true
				%\dystrue
				\dysfalse


\ifdys

\documentclass[a4paper, 14pt]{extarticle}
\usepackage{amsmath,amsfonts,amsthm,amssymb,mathtools}

\tracinglostchars=3 % Report an error if a font does not have a symbol.
\usepackage{fontspec}
\usepackage{unicode-math}
\defaultfontfeatures{ Ligatures=TeX,
                      Scale=MatchUppercase }

\setmainfont{OpenDyslexic}[Scale=1.0]
\setmathfont{Fira Math} % Or maybe try KPMath-Sans?
\setmathfont{OpenDyslexic Italic}[range=it/{Latin,latin}]
\setmathfont{OpenDyslexic}[range=up/{Latin,latin,num}]

\else

\documentclass[a4paper, 12pt]{extarticle}
\usepackage{amsmath,amsfonts,amsthm,amssymb,mathtools}

\fi


\usepackage[french]{babel}
\usepackage[
a4paper,
margin=2cm,
nomarginpar,% We don't want any margin paragraphs
]{geometry}
\usepackage{fancyhdr}
\usepackage{array}

\usepackage{multicol, enumerate}
\newcolumntype{P}[1]{>{\centering\arraybackslash}p{#1}}


\usepackage{stackengine}
\newcommand\xrowht[2][0]{\addstackgap[.5\dimexpr#2\relax]{\vphantom{#1}}}

% theorems

\theoremstyle{plain}
\newtheorem{theorem}{Th\'eor\`eme}
\newtheorem*{sol}{Solution}
\theoremstyle{definition}
\newtheorem{ex}{Exercice}

% corps
\newcommand{\C}{\mathbb{C}}
\newcommand{\R}{\mathbb{R}}
\newcommand{\Rnn}{\mathbb{R}^{2n}}
\newcommand{\Z}{\mathbb{Z}}
\newcommand{\N}{\mathbb{N}}
\newcommand{\Q}{\mathbb{Q}}

% domain
\newcommand{\D}{\mathbb{D}}


% date
\usepackage{advdate}
\AdvanceDate[1]


% plots
\usepackage{pgfplots}


% SOLUTION SWITCH
\newif\ifsolutions
				\solutionstrue
				\solutionsfalse

\ifsolutions
	\newcommand{\exe}[2]{
		\begin{ex} #1  \end{ex}
		\begin{sol} #2 \end{sol}
	}
\else
	\newcommand{\exe}[2]{
		\begin{ex} #1  \end{ex}
	}
	
\fi

\begin{document}
\pagestyle{fancy}
\fancyhead[L]{Seconde 13}
\fancyhead[C]{\textbf{ Droite réelle \ifsolutions -- Solutions  \fi}}
\fancyhead[R]{\today}

\subsection*{Intervalles}

\begin{center}
	
	\begin{tikzpicture}
		
		% real line
		\draw[<->, thick] (-7,0) node[below] {$-\infty$} -- (7,0) node[below] {$+\infty$};
		
		\foreach \x in {-3,...,3}
			\draw[-] (1.3*\x,-.1) node[below] {$\x$} -- (1.3*\x,.1) node{};
		
	
	\end{tikzpicture}
\end{center}

\exe{\label{ex:1}


	Représenter sur la droite réelle les intervalles suivants. Donner la taille des intervalles bornés.
	\begin{multicols}{2}
	\begin{enumerate}
		\item $[-1 ; 1]$
		\item $[-3 ; 1]$
		\item $[-1 ; 1] \cap [-3 ; 1]$
		\item $]3 ; +\infty [$
		\item $] - \infty ; 2] \cap [-3 ; -2]$
		\item $]- \infty ; 4 [ \cup [2 ; +\infty[$
	\end{enumerate}
	\end{multicols}
	
	
}
{}


\begin{center}
	\begin{tikzpicture}
		
		% real line
		\draw[<->, thick] (-7,0) node[below] {$-\infty$} -- (7,0) node[below] {$+\infty$};
		
		\foreach \x in {-3,...,3}
			\draw[-] (1.3*\x,-.1) node[below] {$\x$} -- (1.3*\x,.1) node{};
		
	
	\end{tikzpicture}
\end{center}


\exe{
	Écrire chaque intervalle de l'exercice \ref{ex:1} en termes d'inégalités qu'un réel $x \in \R$ vérifie dès qu'il appartient à celui-ci.
}
{}

\exe{
	Vrai ou faux ?
	
	\begin{multicols}{2}
	\begin{enumerate}
		\item $[-1; 3 [ \subset [-2 ; 4]$
		\item $] -6{,}2 ; 3] \subset [-6{,}1 ; 4[$
		\item $\{ -1; 3{,}4 \} \subset ]-\infty; -1] \cup [2; 5[$
		\item $ 3 \in [-3 ; 3[ \cup ]3, 6]$
	\end{enumerate}
	\end{multicols}
}


\exe{
	À quels intervalles le réel $x \in \R$ appartient-il s'il vérifie les inégalités suivantes ?
	\begin{multicols}{2}
	\begin{enumerate}
		\item $x \geq 0$
		\item $x > 1$
		\item $x \leq -3$
		\item $-3 < x \leq 8$
		\item $2x + 1 \leq 4$
		\item $-x > 0$
		\item $-3x \leq 1$
		\item $-4 \leq 2x - 1 < 9$
	\end{enumerate}
	\end{multicols}
}
{}

\subsection*{Valeur absolue}

\ex{
	Écrire les nombres suivants sans les barres de valeur absolue.
	\begin{multicols}{3}
	\begin{enumerate}[a)]
		\item $|-7|$
		\item $|8|$
		\item $|-13-8|$
		\item $|\pi - 4|$
		\item $|-5+3| + |-7+4|$
		\item $|\sqrt{2} - \sqrt{3}|$
	\end{enumerate}
	\end{multicols}
}
{}




\exe{
	À quels intervalles le réel $x \in \R$ appartient-il s'il vérifie les inégalités suivantes ?
	\begin{multicols}{2}
	\begin{enumerate}
		\item $| x - 2 | \leq 1$
		\item $| 2 + x| > 4$
		\item $| 3 + 2x | \geq 1$
		\item $| 4 - 3x | < 5$
	\end{enumerate}
	\end{multicols}
}
{}



\newpage
\subsection*{Exercices supplémentaires}

\exe{\label{ex:2}
	Écrire les unions et intersections suivantes sous forme d'intervalle.
	
	
	\begin{multicols}{2}
	\begin{enumerate}
		\item $[-1,1] \cap [-3, 1]$
		\item $]-\infty, 2] \cap [-3, 1]$
		\item $]- \infty, 4 [ \cup [2, +\infty[$
		\item $] -3 ; 4 [ \cap [-5 ; 4]$
		\item $]-2 ; 4 [  \cup [4 ; 8[$
		\item $ ]- \infty ; 1 [ \cap ]-1 ; +\infty [$
	\end{enumerate}
	\end{multicols}
}
{}


\exe{
	Écrire chaque intervalle de l'exercice \ref{ex:2} en termes d'inégalités qu'un réel $x \in \R$ vérifie dès qu'il appartient à celui-ci.
}
{}


\exe{
	À quels intervalles le réel $x \in \R$ appartient-il s'il vérifie les inégalités suivantes ?
	\begin{multicols}{2}
	\begin{enumerate}
		\item $x > -2$
		\item $x  < 2$
		\item $-8 \leq x \leq 10$
		\item $1 - x \leq 4$
		\item $ 11 > 3 - 4x  $
		\item $12 \geq 3x \geq -9$
		\item $15 > 4 - 2x \geq -3$
	\end{enumerate}
	\end{multicols}
}
{}


\exe{
	Trouver la ou les valeurs de $x \in \R$ vérifiant
	\begin{multicols}{2}
	\begin{enumerate}
		\item $|x - 1| = 3$
		\item $|2x-3| = 7$
		\item $|10-x| \leq 2$
		\item $|3 - 4x| \geq 2$
		\item $|x-2| = |x-4|$
		\item $| 3x + 8| = |4 - 2x |$
	\end{enumerate}
	\end{multicols}
}
{}

\exe{
	Montrer l'inégalité triangulaire : pour tout $x,y \in \R$, on a
		\[ |x -  y| \leq |x| + | y|. \]
		
	En déduire que pour tout $x,y,z \in \R$, on a
		\[ |x - z| \leq |x-y| + |y - z|. \]
}
{}


\end{document}
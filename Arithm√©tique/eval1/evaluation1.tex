% DYSLEXIA SWITCH
\newif\ifdys
		
				% ENABLE or DISABLE font change
				% use XeLaTeX if true
				\dystrue
				\dysfalse


\ifdys

\documentclass[a4paper, 14pt]{extarticle}
\usepackage{amsmath,amsfonts,amsthm,amssymb,mathtools}

\tracinglostchars=3 % Report an error if a font does not have a symbol.
\usepackage{fontspec}
\usepackage{unicode-math}
\defaultfontfeatures{ Ligatures=TeX,
                      Scale=MatchUppercase }

\setmainfont{OpenDyslexic}[Scale=1.0]
\setmathfont{Fira Math} % Or maybe try KPMath-Sans?
\setmathfont{OpenDyslexic Italic}[range=it/{Latin,latin}]
\setmathfont{OpenDyslexic}[range=up/{Latin,latin,num}]

\else
% NO DYSLEXIA

\documentclass[a4paper, 14pt]{extarticle}
\usepackage{amsmath,amsfonts,amsthm,amssymb,mathtools}

\fi

\usepackage[french]{babel}
\usepackage[
a4paper,
margin=2cm,
nomarginpar,% We don't want any margin paragraphs
]{geometry}
\usepackage{fancyhdr}
\usepackage{array}

\usepackage{multicol, enumerate}
\usepackage{amsmath,amsfonts,amsthm,amssymb,mathtools}
\newcolumntype{P}[1]{>{\centering\arraybackslash}p{#1}}


\usepackage{stackengine}
\newcommand\xrowht[2][0]{\addstackgap[.5\dimexpr#2\relax]{\vphantom{#1}}}

% theorems

\theoremstyle{plain}
\newtheorem{theorem}{Th\'eor\`eme}
\newtheorem{Sol}{Solution}
\newtheorem*{Sol*}{Solution}
\theoremstyle{definition}
\newtheorem{ex}{Exercice}

% corps
\newcommand{\C}{\mathbb{C}}
\newcommand{\R}{\mathbb{R}}
\newcommand{\Rnn}{\mathbb{R}^{2n}}
\newcommand{\Z}{\mathbb{Z}}
\newcommand{\N}{\mathbb{N}}
\newcommand{\Q}{\mathbb{Q}}

% domain
\newcommand{\D}{\mathbb{D}}


% DATE 
\usepackage{advdate}
\AdvanceDate[1]



\begin{document}
\pagestyle{fancy}
\fancyhead[L]{Seconde 13 }
\fancyhead[C]{\textbf{Évaluation 1}}
\fancyhead[R]{\today }
		% REMOVE PAGE NUMBERING 
		%\cfoot{} 

\begin{center}
\begin{tabular}{ | p{.3\linewidth} | p{.5\linewidth} | p{.1\linewidth} |  } 
  \hline\xrowht{20pt}
  Nom et prénom  & Appréciation & Note \\ \xrowht{40pt}
  & & \\ \hline
\end{tabular}
\end{center}


\subsection*{Multiples et diviseurs}

	\noindent
	\textbf{Question 1.} (2pts) Pour $d, n \in \N$ deux entiers naturels non nuls, donner la définition mathématique de $d | n$.
	\vspace{4cm}
	
	\noindent
	\textbf{Question 2.} (1pt) Donner l'ensemble des multiples de $12$ inférieurs ou égaux à $100$.
	\vspace{3cm}
	
	\noindent
	\textbf{Question 3.} (2pts) Donner l'ensemble des diviseurs de $40$.
	\vspace{3cm}
	
\subsection*{Nombres premiers}


	\noindent
	\textbf{Question 4.} (4pts) Décomposer les nombres suivants en produit de facteurs premiers.
			\begin{enumerate}[i)]
				\item 72 =
				\item[]
				\item 135 =
				\item[]
				\item $44 \times 66$ = 
				\item[] 
				\item $(14)^3 \times (24)^4$ = 
			\end{enumerate}
			

	\noindent
	\textbf{Question 5.} (2pts) Donner l'ensemble des nombres premiers inférieurs ou égaux à $20$.
	\vspace{3cm}

\subsection*{Coprimalité}


	
	\noindent
	\textbf{Question 6.} (2pts) Écrire les fractions suivantes sous forme irréductible.
			\begin{enumerate}[i)]
				\item $\dfrac{72}{135}$ =
				\item[] \vspace{.5cm}
				\item $\dfrac{44}{32} \times  \dfrac{100}{14}$ = 
			\end{enumerate}
	\vspace{1cm}

	\noindent
	\textbf{Question 7.} (2pts) Pour $a, b \in \N$ non nuls, donner la définition mathématique de \og $a$ et $b$ sont premiers entre eux \fg.
	\vspace{7cm}
	
	\noindent
	\textbf{Bonus.} (2pts) Soit $n\in \N$. Montrer que si $n$ est pair, alors $n^3$ est pair.
	\vspace{8cm}


\end{document}
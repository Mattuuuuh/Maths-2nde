\documentclass[a4paper, 14pt]{extarticle}
\usepackage[french]{babel}
\usepackage[
a4paper,
margin=2cm,
nomarginpar,% We don't want any margin paragraphs
]{geometry}
\usepackage{fancyhdr}
\usepackage{array}

\usepackage{multicol, enumerate}
\usepackage{amsmath,amsfonts,amsthm,amssymb,mathtools}
\newcolumntype{P}[1]{>{\centering\arraybackslash}p{#1}}


\usepackage{stackengine}
\newcommand\xrowht[2][0]{\addstackgap[.5\dimexpr#2\relax]{\vphantom{#1}}}

% theorems

\theoremstyle{plain}
\newtheorem{theorem}{Th\'eor\`eme}
\newtheorem{Sol}{Solution}
\newtheorem*{Sol*}{Solution}
\theoremstyle{definition}
\newtheorem{ex}{Exercice}
\newtheorem*{definition}{Définition}

% corps
\newcommand{\C}{\mathbb{C}}
\newcommand{\R}{\mathbb{R}}
\newcommand{\Rnn}{\mathbb{R}^{2n}}
\newcommand{\Z}{\mathbb{Z}}
\newcommand{\N}{\mathbb{N}}
\newcommand{\Q}{\mathbb{Q}}

% domain
\newcommand{\D}{\mathbb{D}}


% date
\usepackage{advdate}
\AdvanceDate[1]

% dys
% 
\newif\ifdys
		
% ENABLE or DISABLE font change
% use XeLaTeX if true
%\dystrue
\dysfalse

\ifdys

\usepackage{fontspec}
\usepackage{unicode-math}
\setmainfont{OpenDyslexic}
\setmathfont{OpenDyslexic}

\fi

% bibliography 

\usepackage{biblatex} %Imports biblatex package
\addbibresource{scholar.bib} %Import the bibliography file

\begin{document}
\pagestyle{fancy}
\fancyhead[L]{Seconde 13}
\fancyhead[C]{\textbf{Activité en groupe 2}}
\fancyhead[R]{\today}

%\subsection*{Parité}


\subsection*{Calendriers julien\footnote{De Jules César (né en 100 av. J.-C., mort en 44 av. J.-C.).} et grégorien\footnote{Du pape Grégoire XIII (1502-1585).}}

La Terre complète son tour autour du Soleil en environ $365{,}2422$ jours. 
Un calendrier ne peut donc pas contenir le même nombre de jours ($365$) tous les ans si celui-ci doit être synchronisé avec les saisons.

En $46$ av. J.-C., Jules César introduit alors la règle de l'année bissextile.

\begin{ex}[Calendrier julien]
	\begin{enumerate}
		\item[]
		\item
		Trouver un entier naturel $a \in \N$ non nul tel que
			\[ 365{,}2422 \approx 365 + \dfrac1a.\]

		\item
		En déduire la règle d'année bissextile du calendrier julien.
		
		\item Au bout de $1600$ ans, de combien de jours le calendrier est-il en retard ?
	\end{enumerate}
\end{ex}

En $1582$, plus de $1600$ ans après l'introduction de la règle d'année bissextile, le calendrier julien prend donc une avance notoire.
Le pape Grégoire XIII promulgue alors une réforme du calendrier, ainsi créant le calendrier grégorien encore utilisé aujourd'hui.

Afin de corriger le problème de dérive du calendrier, une nouvelle restriction est ajoutée à la définition d'année bissextile.

\begin{ex}[Calendrier grégorien]
	\begin{enumerate}
		\item[]
		\item
		Trouver un entier naturel $b \in \N$ tel que
			\[ 365{,}2422 \approx 365 + \dfrac14 - \dfrac{b}{400}.\]

		\item  En déduire la règle d'année bissextile du calendrier grégorien.
		
		\item Au bout de combien d'années le calendrier est-il en retard d'un jour ?
	\end{enumerate}
\end{ex}

\begin{ex}[Un nouveau calendrier ?]

	\begin{enumerate}
		\item[]
		\item
		Trouver un entier naturel $c \in \N$ tel que
			\[ 365{,}2422 \approx 365 + \dfrac14 - \dfrac{3}{400} - \dfrac{c}{4000}.\]

		\item  Créer une nouvelle règle d'année bissextile et calculer le nombre d'années avant que votre calendrier soit en retard d'un jour.
	\end{enumerate}

\end{ex}

\vfill
\nocite{*}
\printbibliography[title={Référence}]

\end{document}
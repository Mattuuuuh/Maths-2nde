\documentclass[a4paper, 14pt]{extarticle}
\usepackage[french]{babel}
\usepackage[
a4paper,
margin=2cm,
nomarginpar,% We don't want any margin paragraphs
]{geometry}
\usepackage{fancyhdr}
\usepackage{array}

\usepackage{multicol, enumerate}
\usepackage{amsmath,amsfonts,amsthm,amssymb,mathtools}
\newcolumntype{P}[1]{>{\centering\arraybackslash}p{#1}}


\usepackage{stackengine}
\newcommand\xrowht[2][0]{\addstackgap[.5\dimexpr#2\relax]{\vphantom{#1}}}

% theorems

\theoremstyle{plain}
\newtheorem{theorem}{Th\'eor\`eme}
\newtheorem{Sol}{Solution}
\newtheorem*{Sol*}{Solution}
\theoremstyle{definition}
\newtheorem{ex}{Exercice}
\newtheorem*{definition}{Définition}

% corps
\newcommand{\C}{\mathbb{C}}
\newcommand{\R}{\mathbb{R}}
\newcommand{\Rnn}{\mathbb{R}^{2n}}
\newcommand{\Z}{\mathbb{Z}}
\newcommand{\N}{\mathbb{N}}
\newcommand{\Q}{\mathbb{Q}}

% domain
\newcommand{\D}{\mathbb{D}}


% date
\usepackage{advdate}
\AdvanceDate[1]

% dys
% 
\newif\ifdys
		
% ENABLE or DISABLE font change
% use XeLaTeX if true
%\dystrue
\dysfalse

\ifdys

\usepackage{fontspec}
\usepackage{unicode-math}
\setmainfont{OpenDyslexic}
\setmathfont{OpenDyslexic}

\fi

\begin{document}
\pagestyle{fancy}
\fancyhead[L]{Seconde 13}
\fancyhead[C]{\textbf{Activité en groupe 1}}
\fancyhead[R]{\today}

%\subsection*{Parité}


\subsection*{Nombres pairs, nombres impairs}

\begin{ex}
	Barrer les nombres impairs.
	
	\begin{center}
	\begin{tabular}{ | P{.05\linewidth} | P{.05\linewidth} | P{.05\linewidth} | P{.05\linewidth} | P{.05\linewidth} | P{.05\linewidth} | P{.05\linewidth} | P{.05\linewidth} | P{.05\linewidth} | P{.05\linewidth} |    } 
	  \hline\xrowht{10pt}
	  0 & 1 & 2 & 3 & 4 & 5 & 6 & 7 & 8 & 9  \\ \hline \xrowht{10pt}
	  10 & 11 & 12 & 13 & 14 & 15 & 16 & 17 & 18 & 19  \\ \hline
	\end{tabular}
	\end{center}
\end{ex}

\begin{definition}
	Pour un entier naturel $n\in\N$, on dit que $n$ est \textbf{pair} ou \textbf{multiple de} $ \boldsymbol{2}$  si et seulement si
	
	\vspace{5pt}
	\framebox(450,100){} 

\end{definition}

\begin{ex}

	Soit un entier naturel $a\in\N$.	
	 Les entiers suivants sont-ils toujours pairs ? Justifier.
	\begin{multicols}{2}
	\begin{enumerate}[(1)]
		\item $2a + 4$
		\item $2a + 1$
		\item $3a$
		\item  $3(3a + 1) - (a-1)$
	\end{enumerate}
	\end{multicols}
\end{ex}

\begin{definition}
	Pour un entier naturel $n\in\N$, on dit que $n$ est \textbf{impair} si et seulement si 
	
	\vspace{5pt}
	\framebox(450,100){} 
\end{definition}

\begin{ex}
	Soit un entier naturel $m\in\N$. Les entiers suivants sont-ils toujours pairs ? impairs ? Justifier.
	
	\begin{multicols}{2}
	\begin{enumerate}[(a)]
		\item $2m - 1$
		\item $2m + 3$
		\item $4 (3m - 1)$
		\item $ m(m + 1) $
		\item $7(4m + 1)  - 3(2m + 2)$
		\item $m^2$
	\end{enumerate}
	\end{multicols}
\end{ex}

\hrule

\begin{ex}
	Soit  un entier naturel $n\in\N$ quelconque. Montrer que l'entier
		\[ (2n + 1)^2 - 1 \]
	est toujours pair.
	
	\textit{Rappel : on a l'identité remarquable $a^2 - b^2 = (a+b)(a-b)$.}
\end{ex}


\end{document}
% DYSLEXIA SWITCH
\newif\ifdys
		
				% ENABLE or DISABLE font change
				% use XeLaTeX if true
				\dystrue
				\dysfalse


\ifdys

\documentclass[a4paper, 14pt]{extarticle}
\usepackage{amsmath,amsfonts,amsthm,amssymb,mathtools}

\tracinglostchars=3 % Report an error if a font does not have a symbol.
\usepackage{fontspec}
\usepackage{unicode-math}
\defaultfontfeatures{ Ligatures=TeX,
                      Scale=MatchUppercase }

\setmainfont{OpenDyslexic}[Scale=1.0]
\setmathfont{Fira Math} % Or maybe try KPMath-Sans?
\setmathfont{OpenDyslexic Italic}[range=it/{Latin,latin}]
\setmathfont{OpenDyslexic}[range=up/{Latin,latin,num}]

\else

\documentclass[a4paper, 12pt]{extarticle}

\usepackage[utf8x]{inputenc}
\usepackage{lmodern,textcomp}
\usepackage{amsmath,amsfonts,amsthm,amssymb,mathtools}

\fi


\usepackage[french]{babel}
\usepackage[
a4paper,
margin=2cm,
nomarginpar,% We don't want any margin paragraphs
]{geometry}
\usepackage{icomma}

\usepackage{fancyhdr}
\usepackage{array}

\usepackage{multicol, enumerate}
\newcolumntype{P}[1]{>{\centering\arraybackslash}p{#1}}


\usepackage{stackengine}
\newcommand\xrowht[2][0]{\addstackgap[.5\dimexpr#2\relax]{\vphantom{#1}}}

% theorems

\theoremstyle{plain}
\newtheorem{theorem}{Th\'eor\`eme}
\newtheorem*{sol}{Solution}
\theoremstyle{definition}
\newtheorem{ex}{Exercice}

% corps
\newcommand{\C}{\mathbb{C}}
\newcommand{\R}{\mathbb{R}}
\newcommand{\Rnn}{\mathbb{R}^{2n}}
\newcommand{\Z}{\mathbb{Z}}
\newcommand{\N}{\mathbb{N}}
\newcommand{\Q}{\mathbb{Q}}

% domain
\newcommand{\D}{\mathbb{D}}


% date
\usepackage{advdate}
\AdvanceDate[1]


% plots
\usepackage{pgfplots}


% SOLUTION SWITCH
\newif\ifsolutions
				\solutionstrue
				\solutionsfalse

\ifsolutions
	\newcommand{\exe}[2]{
		\begin{ex} #1  \end{ex}
		\begin{sol} #2 \end{sol}
	}
\else
	\newcommand{\exe}[2]{
		\begin{ex} #1  \end{ex}
	}
	
\fi

\begin{document}
\pagestyle{fancy}
\fancyhead[L]{Seconde 13}
\fancyhead[C]{\textbf{Statistiques 1\ifsolutions -- Solutions  \fi}}
\fancyhead[R]{\today}

\subsection*{Sous-populations}

\exe{
        Une classe de Seconde comprend $25$ filles pour $9$ garçons.
        Calculer le pourcentage de filles et de garçons dans la classe.
}{}

\exe{
   En sachant que les $16 600$ espèces de fourmis constituent environ $1{,}3\%$ du total des espèces d'insectes répertoriées sur Terre, estimer le nombre total d'espèces d'insectes. 
}{}


\exe{
  En 2023 en France, $13\%$ des espèces (faune et flore) sont considérées comme menacées à l'échelle mondiale (catégories ``danger critique'' à ``vulnérable'' de l'UICN).
  Parmis celles-ci, $23\%$ sont en danger critique.
  
  Calculer le pourcentage d'espèces en danger critique par rapport au nombre total d'espèces.
}{}

\subsection*{Évolution}

\exe{
  Calculer sans calculatrice les pourcentages suivants.
  \begin{multicols}{2}
    \begin{enumerate}
    \item $75\%$ de $60$
    \item $60\%$ de $75$
    \item $72\%$ de $25$
    \item $68\%$ de $20$
    \item $125\%$ de $40$
    \item $40\%$ de $125$
    \end{enumerate}
  \end{multicols}
}{
  \begin{multicols}{2}
    \begin{enumerate}
    \item $\dfrac34 \cdot 60 = 3 \dot \dfrac{60}4 = 3 \cdot 15 = 45$
    \item $45$
    \item $\dfrac14 \cdot 72 = 18$
    \item $\dfrac15 \cdot 68 = \dfrac{136}{10} = 13,6$
    \item $40 + \dfrac14 \cdot 40 = 50$
    \item $50$
    \end{enumerate}
  \end{multicols}
}

\exe{
  Approximer sans calculatrice les pourcentages suivants.
  \begin{multicols}{2}
    \begin{enumerate}
    \item $33\%$ de $150$
    \item $166\%$ de $180$
    \item $11\%$ de $90$
    \item $89\%$ de $81$
    \item $16,6\%$ de $18$
    \item $83,4\%$ de $36$
    \end{enumerate}
  \end{multicols}
}{
  \begin{multicols}{2}
    \begin{enumerate}
    \item $\approx \dfrac13 \cdot 150 = 50$
    \item $\approx 180 + \dfrac23 \cdot 180 = 180 + 120 = 300$
    \item $\approx \dfrac19 \cdot 90 = 10$
    \item $\approx 81 - \dfrac19 \cdot 81 = 81 - 9 = 72$
    \item $\approx \dfrac16 \cdot 18 = 3$
    \item $\approx 36 - \dfrac16 \cdot 36 = 36 - 6 = 30$
    \end{enumerate}
  \end{multicols}
}

\exe{
  On estime la biomasse totale des fourmis sur Terre à $12$ millions de tonnes.
  Ceci représenterait $20\%$ de la biomasse humaine.

  Estimer la biomasse totale des humains sur Terre en tonnes.
}{}

\exe{
  Une jeune femme dépose $10$€ à la banque. Celle-ci lui promet un taux d'intérêt à l'année de $3\%$.
  Ainsi, après la première année, il y aura $1{,}03 \cdot 10 = 10{,}3$€ sur son compte.
  La deuxième année, il y aura $1{,}03 \cdot 10{,}35 = 10{,}609$€, etc...

  À l'aide de la calculatrice, répondre aux questions suivantes.
  \begin{enumerate}
  \item Combien d'argent aura-t-elle après $5$ ans ?
  \item Combien d'argent aura-t-elle après $50$ ans ?
  \item Combien d'argent y aura-t-il sur son compte après $1000$ ans ?
  \end{enumerate}
}
{}

\exe{
  Considérons deux tailleurs, l'un à $250$€ et l'autre à $360$€.
  \begin{enumerate}
  \item Quelle augmentation de prix faut-il appliquer au premier tailleur pour qu'il ait le prix du second ?
  \item Quel rabais faut-il appliquer au second tailleur pour qu'il ait le prix du premier ?
  \end{enumerate}
}{}

\exe{
  Si on augmente le prix d'un objet de $150\%$, quel rabais faut-il appliquer pour retrouver le prix initial de l'objet ?
}{}

\newpage
\subsection*{Exercices supplémentaires}

\exe{
        On considère l'ensemble $E= \{1; 2; \dots ; n-1 ; n\}$ dépendant d'un entier naturel $n\geq1$.
        On pose $F = \{k \in E \text{ tq. } 2 | k\}$, l'ensemble des éléments pairs de $E$.

        A-t-on toujours $\dfrac{|F|}{|E|} = \dfrac12$ ? Donner l'ensemble des entiers $n\geq1$ pour lesquels l'égalité est vraie.
}

\exe{
  Si on augmente le prix d'un objet de $100p\%$ ($p\geq0$ réel), quel rabais faut-il appliquer (en fonction de $p$) pour retrouver le prix initial de l'objet ?
}{}

\ex{
  En $2023$, le prix moyen du gaz naturel facturé aux ménages français s'élève à $115$€ par MWh, toutes taxes comprises (TTC).
  En $2022$, le prix était de $96$€.

  Calculer le pourcentage d'augmentation du prix entre l'année $2022$ et l'année $2023$.
}{}

\exe{
  En $2022$ en France, la consommation de gaz naturel s'établit à $463$ TWh.
  En $2021$, celle-ci s'élvait plutôt à $475{,}85$ TWh.

  Calculer le pourcentage de diminution de la consommation entre l'année $2021$ et l'année $2022$.
}{}


\exe{
    Soit l'ensemble d'entiers $E=\{n; n+1; \dots; m-1 ; m\}$ dépendant de deux entiers relatifs $n<m$.
    Calculer $|E|$ en fonction de $n$ et $m$.
    Vérifier la formule avec $n=-1$ et $m=1$ en sachant que $|\{-1 ; 0 ; 1 \}| = 3$.
}{}

\end{document}

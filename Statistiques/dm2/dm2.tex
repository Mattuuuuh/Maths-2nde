%% INPUT PREAMBLE.TEX
%% THEN INPUT VARS_{i}.ADR
%% THEN RUN THIS
%% DYSLEXIA SWITCH
\newif\ifdys
		
				% ENABLE or DISABLE font change
				% use XeLaTeX if true
				\dystrue
				\dysfalse


\ifdys

\documentclass[a4paper, 14pt]{extarticle}
\usepackage{amsmath,amsfonts,amsthm,amssymb,mathtools}

\tracinglostchars=3 % Report an error if a font does not have a symbol.
\usepackage{fontspec}
\usepackage{unicode-math}
\defaultfontfeatures{ Ligatures=TeX,
                      Scale=MatchUppercase }

\setmainfont{OpenDyslexic}[Scale=1.0]
\setmathfont{Fira Math} % Or maybe try KPMath-Sans?
\setmathfont{OpenDyslexic Italic}[range=it/{Latin,latin}]
\setmathfont{OpenDyslexic}[range=up/{Latin,latin,num}]

\else

\documentclass[a4paper, 12pt]{extarticle}

\usepackage[utf8x]{inputenc}
%fonts
\usepackage{amsmath,amsfonts,amsthm,amssymb,mathtools}
% comment below to default to computer modern
\usepackage{libertinus,libertinust1math}

\fi


\usepackage[french]{babel}
\usepackage[
a4paper,
margin=2cm,
nomarginpar,% We don't want any margin paragraphs
]{geometry}
\usepackage{icomma}

\usepackage{fancyhdr}
\usepackage{array}
\usepackage{hyperref}

\usepackage{multicol, enumerate}
\newcolumntype{P}[1]{>{\centering\arraybackslash}p{#1}}


\usepackage{stackengine}
\newcommand\xrowht[2][0]{\addstackgap[.5\dimexpr#2\relax]{\vphantom{#1}}}

% theorems

\theoremstyle{plain}
\newtheorem{theorem}{Th\'eor\`eme}
\newtheorem*{sol}{Solution}
\theoremstyle{definition}
\newtheorem{ex}{Exercice}
\newtheorem*{rpl}{Rappel}
\newtheorem{enigme}{Énigme}

% corps
\usepackage{calrsfs}
\newcommand{\C}{\mathcal{C}}
\newcommand{\R}{\mathbb{R}}
\newcommand{\Rnn}{\mathbb{R}^{2n}}
\newcommand{\Z}{\mathbb{Z}}
\newcommand{\N}{\mathbb{N}}
\newcommand{\Q}{\mathbb{Q}}

% variance
\newcommand{\Var}[1]{\text{Var}(#1)}

% domain
\newcommand{\D}{\mathcal{D}}


% date
\usepackage{advdate}
\AdvanceDate[0]


% plots
\usepackage{pgfplots}

% table line break
\usepackage{makecell}
%tablestuff
\def\arraystretch{2}
\setlength\tabcolsep{15pt}

%subfigures
\usepackage{subcaption}

\definecolor{myg}{RGB}{56, 140, 70}
\definecolor{myb}{RGB}{45, 111, 177}
\definecolor{myr}{RGB}{199, 68, 64}

% fake sections with no title to move around the merged pdf
\newcommand{\fakesection}[1]{%
  \par\refstepcounter{section}% Increase section counter
  \sectionmark{#1}% Add section mark (header)
  \addcontentsline{toc}{section}{\protect\numberline{\thesection}#1}% Add section to ToC
  % Add more content here, if needed.
}


% SOLUTION SWITCH
\newif\ifsolutions
				\solutionstrue
				%\solutionsfalse

\ifsolutions
	\newcommand{\exe}[2]{
		\begin{ex} #1  \end{ex}
		\begin{sol} #2 \end{sol}
	}
\else
	\newcommand{\exe}[2]{
		\begin{ex} #1  \end{ex}
	}
	
\fi


% tableaux var, signe
\usepackage{tkz-tab}


%pinfty minfty
\newcommand{\pinfty}{{+}\infty}
\newcommand{\minfty}{{-}\infty}

\begin{document}
\input{vars_TEST.adr}

\pagestyle{fancy}
\fancyhead[L]{Seconde 13}
\fancyhead[C]{\textbf{Devoir Maison 1 -- \seed \ifsolutions -- Solutions  \fi}}
\fancyhead[R]{\today}


\exe{
	On considère la série statistique $X$ suivante, dépendente de deux entiers naturels $a,b\in\N$.
	
		\begin{center}
		\begin{tabular}{|c|c|c|c|}\hline
		Valeur   & \kmp & \kpq & \nval \\ \hline
		Effectif & $a$ & $b$ & \n \\ \hline
		\end{tabular}
		\end{center}
		
	\begin{enumerate}
		\item
		Montrer que la condition $\overline{X} = \k$ est équivalente à la relation
			\[ - \psd \cdot a + \qsd \cdot b = 1. \]
		\item
		En déduire que $\psd$ divise $\qsd \cdot b - 1$ et trouver le plus petit entier naturel $b_0\in\N$ vérifiant
			\[ \psd | \left( \qsd\cdot b_0 - 1 \right). \]
		\item
		Posons 
			\[ a_0 = \dfrac{\qsd \cdot b_0 - 1}\psd. \]
		Montrer que pour le couple d'entiers $(a;b) = (a_0 ; b_0)$, la série $X$ est de moyenne $\k$.
		\item
		Montrer que pour tout $n\in\N$, la série $X$ associée au couple
			\[ (a;b) = (a_0 ; b_0) + n \cdot (\qsd ; \psd) \]
		est de moyenne $\k$.
		\item
		Représenter les points $(a;b)$ dans un repère pour $n=0 ; 1 ; 2; 3$. 
		Que dire sur les points ?
	\end{enumerate}
}
{}

\exe{
	On considère la série statistique $X$ suivante, dépendente de deux entiers naturels $a,b\in\N$.
	
		\begin{center}
		\begin{tabular}{|c|c|c|c|}\hline
		Valeur   & \kmpB & \kpqB & \nvalB \\ \hline
		Effectif & $a$ & $b$ & \nB \\ \hline
		\end{tabular}
		\end{center}
		
	\begin{enumerate}
		\item
		Montrer que la condition $\overline{X} = \k$ est équivalente à la relation
			\[ - \pB \cdot a + \qB \cdot b = \dmnB. \]
		\item
		Trouver le plus grand diviseur commun à $\p$ et $\q$ et montrer qu'il divise nécessairement 
			\[ - \pB \cdot a + \qB \cdot b.\]
		\item
		Conclure par contradiction qu'il n'existe pas d'entiers naturels $a,b\in\N$ tels que la série $X$ ait une moyenne égale à $\k$.
		\item 
		Trouver un moyenne possible et deux couples $(a,b)$ distincts qui la réalisent.
		Y a-t-il une infinité de tels couples ?
	\end{enumerate}
}
{}

\newpage 

\exe{
	Écrire un tableau Valeur/Effectif pour chaque histogramme de la figure \ref{fig:hist}.
	On assignera la valeur moyenne à chaque élément d'une classe.
	Par exemple, de l'histogramme \ref{fig:a} on lit $\EFFCONCa$ notes de valeur $\CONCa,5$.
	
	Ensuite, pour chaque série obtenue, calculer
		\begin{multicols}{2}
		\begin{enumerate}[i)]
			\item La moyenne ;
			\item L'écart type ;
			\item La médiane ;
			\item Le premier quartile ;
			\item Le troisième quartile ; et
			\item L'écart interquartile.
		\end{enumerate}
		\end{multicols}
		
	Comparer les séries statistiques en s'appuyant sur les valeurs calculées.
}

\begin{figure}[t]
  \centering
  \begin{subfigure}[b]{.45\textwidth}
    \centering
  \begin{tikzpicture}[scale=1]
    \begin{axis}[
    	ymin=0,
        %ymin=0, ymax=8,
        %minor y tick num = 3,
        xmin = 0, xmax=20,
        xtick = {2, 4, ..., 18},
        area style,
        xlabel = {Note sur $20$},
        ylabel = {Effectif}
      ]
      \addplot+[ybar interval,mark=no, draw=myg, fill=myg!50] plot coordinates {
        (\CONCa, \EFFCONCa) (\CONCb, \EFFCONCb) (\CONCc, \EFFCONCc) (\CONCd, \EFFCONCd) (\CONCd+1, 0)
      };
    \end{axis} 
  \end{tikzpicture}
  \caption{Première série.}
  \label{fig:a}
  \end{subfigure}
  \hfill
  % NOT LINE BREAK!!
  \begin{subfigure}[b]{.45\textwidth}
    \centering
  \begin{tikzpicture}[scale=1]
    \begin{axis}[
    	ymin=0,
        %ymin=0, ymax=8,
        %minor y tick num = 3,
        xmin = 0, xmax=20,
        xtick = {2, 4, ..., 18},
        area style,
        xlabel = {Note sur $20$},
        ylabel = {Effectif}
      ]
      \addplot+[ybar interval,mark=no, draw=myb, fill=myb!50] plot coordinates {
        (\DISPa, \EFFDISPa) (\DISPb, \EFFDISPb) (\DISPc, \EFFDISPc) (\DISPc+1, 0) (\DISPd, \EFFDISPd) (\DISPe, \EFFDISPe) (\DISPe+1, 0)
      };
    \end{axis} 
  \end{tikzpicture}
  \caption{Deuxième série.}
  \label{fig:b}
  \end{subfigure}
  
  
  \caption{Histogrammes de notes (min $0$, max $20$). Les classes sont de la forme $[k;k+1[, k\in\N$. 
  La colonne entre $12$ et $13$ compte donc toutes les notes appartenant à $[12;13[$.}
  \label{fig:hist}
\end{figure}


\end{document}

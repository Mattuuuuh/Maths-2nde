% DYSLEXIA SWITCH
\newif\ifdys
		
				% ENABLE or DISABLE font change
				% use XeLaTeX if true
				\dystrue
				\dysfalse


\ifdys

\documentclass[a4paper, 14pt]{extarticle}
\usepackage{amsmath,amsfonts,amsthm,amssymb,mathtools}

\tracinglostchars=3 % Report an error if a font does not have a symbol.
\usepackage{fontspec}
\usepackage{unicode-math}
\defaultfontfeatures{ Ligatures=TeX,
                      Scale=MatchUppercase }

\setmainfont{OpenDyslexic}[Scale=1.0]
\setmathfont{Fira Math} % Or maybe try KPMath-Sans?
\setmathfont{OpenDyslexic Italic}[range=it/{Latin,latin}]
\setmathfont{OpenDyslexic}[range=up/{Latin,latin,num}]

\else

\documentclass[a4paper, 12pt]{extarticle}

\usepackage[utf8x]{inputenc}
\usepackage{lmodern,textcomp}
\usepackage{amsmath,amsfonts,amsthm,amssymb,mathtools}

\fi


\usepackage[french]{babel}
\usepackage[
a4paper,
margin=2cm,
nomarginpar,% We don't want any margin paragraphs
]{geometry}
\usepackage{icomma}

\usepackage{fancyhdr}
\usepackage{array}
\usepackage{hyperref}

\usepackage{multicol, enumerate}
\newcolumntype{P}[1]{>{\centering\arraybackslash}p{#1}}


\usepackage{stackengine}
\newcommand\xrowht[2][0]{\addstackgap[.5\dimexpr#2\relax]{\vphantom{#1}}}

% theorems

\theoremstyle{plain}
\newtheorem{theorem}{Th\'eor\`eme}
\newtheorem*{sol}{Solution}
\theoremstyle{definition}
\newtheorem{ex}{Exercice}

% corps
\newcommand{\C}{\mathbb{C}}
\newcommand{\R}{\mathbb{R}}
\newcommand{\Rnn}{\mathbb{R}^{2n}}
\newcommand{\Z}{\mathbb{Z}}
\newcommand{\N}{\mathbb{N}}
\newcommand{\Q}{\mathbb{Q}}

% domain
\newcommand{\D}{\mathbb{D}}


% date
\usepackage{advdate}
\AdvanceDate[1]


% plots
\usepackage{pgfplots}

% table line break
\usepackage{makecell}

%subfigures
\usepackage{subcaption}

\definecolor{myg}{RGB}{56, 140, 70}
\definecolor{myb}{RGB}{45, 111, 177}
\definecolor{myr}{RGB}{199, 68, 64}


% SOLUTION SWITCH
\newif\ifsolutions
				\solutionstrue
				\solutionsfalse

\ifsolutions
	\newcommand{\exe}[2]{
		\begin{ex} #1  \end{ex}
		\begin{sol} #2 \end{sol}
	}
\else
	\newcommand{\exe}[2]{
		\begin{ex} #1  \end{ex}
	}
	
\fi

\begin{document}
\pagestyle{fancy}
\fancyhead[L]{Seconde 13}
\fancyhead[C]{\textbf{Statistiques 3 \ifsolutions -- Solutions  \fi}}
\fancyhead[R]{\today}

\exe{
	Calculer la moyenne de chaque série statistique suivante.
		\begin{multicols}{2}
		\begin{enumerate}
			\item $(0; 20)$
			\item $(0; 0; 20)$
			\item $(0; 20; 20)$
			\item $(-10; 23 ; -10 ; -23 ; 10 ; 10)$
			\item $(-50 ; -100 ; -23 ; -31; -50 ; -50)$
			\item $(0; 1;2;3; \dots; 18; 19; 20)$
		\end{enumerate}
		\end{multicols}
}{}

\exe{\label{ex:33}
	Pour chaque série statistique suivante, calculer la moyenne.
		\begin{multicols}{2}
		\begin{enumerate}
			\item 
				\begin{tabular}{|c|c|c|c|c|}\hline
				Valeur   & 9 & 13 & 11 & 7 \\ \hline
				Effectif & 5 & 12 & 5 & 12 \\ \hline
				\end{tabular}
				
			\item 
				\begin{tabular}{|c|c|c|c|c|}\hline
				Valeur   & 2 & 13 & 18 & 7 \\ \hline
				Effectif & 5 & 12 & 5 & 12 \\ \hline
				\end{tabular}
				
			\item 
				\begin{tabular}{|c|c|c|c|c|}\hline
				Valeur   & 0 & 20 & 2 & 5 \\ \hline
				Effectif & 5 & 14 & 10 & 2  \\ \hline
				\end{tabular}

			\item 
				\begin{tabular}{|c|c|c|c|}\hline
				Valeur   & 0 & 20 & 10 \\ \hline
				Effectif & 17 & 17 & 1 \\ \hline
				\end{tabular}


		\end{enumerate}
		\end{multicols}
}{}




\exe{
	Pour chaque série statistique suivante, remplir la ligne \og Fréquence \fg \, et calculer la moyenne.
		\begin{multicols}{2}
		\begin{enumerate}
			\item 
				\begin{tabular}{|c|c|c|}\hline
				Valeur   & 0 & 20 \\ \hline
				Effectif & 1 & 2  \\ \hline
				Fréquence & &  \\ \hline
				\end{tabular}
				
			\item 
				\begin{tabular}{|c|c|c|}\hline
				Valeur   & 0 & 20 \\ \hline
				Effectif & 2 & 1  \\ \hline
				Fréquence & &  \\ \hline
				\end{tabular}
				
			\item 
				\begin{tabular}{|c|c|c|c|c|}\hline
				Valeur   & 11 & 13 & 9 & 7 \\ \hline
				Effectif & 2 & 1 & 5 & 10 \\ \hline
				Fréquence & &&&  \\ \hline
				\end{tabular}
				
			\item 
				\begin{tabular}{|c|c|c|c|c|}\hline
				Valeur   & -23 & -1 & 1 & 23 \\ \hline
				Effectif & 67 & 12 & 4 & 17 \\ \hline
				Fréquence & &&&  \\ \hline
				\end{tabular}

		\end{enumerate}
		\end{multicols}
}{}



\exe{
	Un élève reçoit $5$ notes listées ainsi : $(4,25 ; 10; 18,5 ; 15,5 ; 11,25)$.
	À ces notes sont associés $5$ coefficients, donnés dans le même ordre par $(0,5 ; 1,4 ; 0,1 ; 0,5; 1,1)$.
	
	Calculer la moyenne pondérée des notes.
}{}


\exe{
	Calculer et comparer les écart types des séries de l'exercice \ref{ex:33}.
}{}

\exe{
	\begin{multicols}{2}
	On considère la série ci-contre tirée de l'exercice \ref{ex:33}.
	
		\begin{center}
		\begin{tabular}{|c|c|c|c|}\hline
		Valeur   & 0 & 20 & 10 \\ \hline
		Effectif & 17 & 17 & 1 \\ \hline
		\end{tabular}
		\end{center}
	\end{multicols}
		
	Modifier l'effectif de la valeur $10$ en $10, 20, 50, 100$, et calculer la moyenne et l'écart type de la série obtenue.
	Comment évolue $\sigma$ ? La série devient-elle plus ou moins concentrée autour de sa moyenne ?
}{}

\begin{figure}[t!]
  \centering
  \begin{subfigure}[b]{.45\textwidth}
    \centering
  \begin{tikzpicture}[scale=1]
    \begin{axis}[
        ymin=0, ymax=8,
        %minor y tick num = 3,
        xtick = {2, 4, ..., 16, 18},
        area style,
        xlabel = {Note sur $20$},
        ylabel = {Effectif}
      ]
      \addplot+[ybar interval,mark=no, draw=myg, fill=myg!50] plot coordinates {
        (2,1) (3,0) (4,0) (5,1) (6,1) (7,3) (8,1) (9,5) (10,1) (11,0) (12,3) (13,4) (14,3) (15,3) (16,5) (17,1) (18,2) (19,0)
      };
    \end{axis} 
  \end{tikzpicture}
  \caption{Évaluation ``Droite réelle''.}
  \label{fig:a}
  \end{subfigure}
  \hfill
  % NOT LINE BREAK!!
  \begin{subfigure}[b]{.45\textwidth}
    \centering
  \begin{tikzpicture}[scale=1]
    \begin{axis}[
        ymin=0, ymax=8,
        %minor y tick num = 3,
        %xtick = {4, 5, ..., 18, 19},
        xtick = {2, 4, ..., 16, 18},
        area style,
        xlabel = {Note sur $20$},
        ylabel = {Effectif}
      ]
      \addplot+[ybar interval,mark=no, draw=myb, fill=myb!50] plot coordinates {
        (4,1) (5,0) (6,1) (7,0) (8,2) (9,1) (10,4) (11,2) (12,7) (13,7) (14,3) (15,0) (16,2) (17,2) (18,1) (19,0)
      };
    \end{axis} 
  \end{tikzpicture}
  \caption{Évaluation ``Droite et plan''.}
  \end{subfigure}
    \begin{subfigure}[b]{.45\textwidth}
    \centering
  \begin{tikzpicture}[scale=1]
    \begin{axis}[
        ymin=0, ymax=8,
        %minor y tick num = 3,
        xtick = {2, 4, ..., 16, 18},
        area style,
        xlabel = {Note sur $20$},
        ylabel = {Effectif}
      ]
      \addplot+[ybar interval,mark=no, draw=myr, fill=myr!50] plot coordinates {
      		(2,1) (3,0) (8,2) (9,0) (10, 1) (11,4) (12,1) (13,4) (14,4) (15,7) (16,7) (17,1) (18,2) (19,0)
      };
    \end{axis} 
  \end{tikzpicture}
  \caption{Automatismes 1 à 5.}
  \end{subfigure}
  
  
  \caption{Histogrammes de notes (min $0$, max $20$). Les classes sont de la forme $[k;k+1[, k\in\N$. 
  La colonne entre $12$ et $13$ compte donc toutes les notes appartenant à $[12;13[$.
  \\
  Lecture : $5$ notes de l'évaluation ``Droite réelle'' appartiennent à l'intervalle $[9;10[$.}
  \label{fig:hist}
\end{figure}

\exe{
	Écrire un tableau Valeur/Effectif pour chaque histogramme de la figure \ref{fig:hist} (au dos).
	On assignera la valeur moyenne $k+\frac12$ à chaque élément d'une classe $[k; k+1[, k\in\N$.
	Par exemple, de l'histogramme \ref{fig:a} on lit $4$ notes de valeur $16,5$.
	
	Ensuite, pour chaque série obtenue, calculer
		\begin{multicols}{2}
		\begin{enumerate}[i)]
			\item La moyenne ;
			\item L'écart type ;
			\item La médiane ;
			\item Le premier quartile ;
			\item Le troisième quartile ; et
			\item L'écart interquartile.
		\end{enumerate}
		\end{multicols}
}{}

\exe{
	Pour quel(s) $N\in\N$ la moyenne de la série suivante est-elle $10$ ? Comparer avec la série de l'exercice \ref{ex:33}.
	
		\begin{center}
		\begin{tabular}{|c|c|c|c|c|}\hline
		Valeur   & 0 & 20 & 2 & 5 \\ \hline
		Effectif & 5 & 14 & N & 2  \\ \hline
		\end{tabular}
		\end{center}
}{}

\end{document}

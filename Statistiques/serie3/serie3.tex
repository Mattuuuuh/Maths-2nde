% DYSLEXIA SWITCH
\newif\ifdys
		
				% ENABLE or DISABLE font change
				% use XeLaTeX if true
				\dystrue
				\dysfalse


\ifdys

\documentclass[a4paper, 14pt]{extarticle}
\usepackage{amsmath,amsfonts,amsthm,amssymb,mathtools}

\tracinglostchars=3 % Report an error if a font does not have a symbol.
\usepackage{fontspec}
\usepackage{unicode-math}
\defaultfontfeatures{ Ligatures=TeX,
                      Scale=MatchUppercase }

\setmainfont{OpenDyslexic}[Scale=1.0]
\setmathfont{Fira Math} % Or maybe try KPMath-Sans?
\setmathfont{OpenDyslexic Italic}[range=it/{Latin,latin}]
\setmathfont{OpenDyslexic}[range=up/{Latin,latin,num}]

\else

\documentclass[a4paper, 12pt]{extarticle}

\usepackage[utf8x]{inputenc}
\usepackage{lmodern,textcomp}
\usepackage{amsmath,amsfonts,amsthm,amssymb,mathtools}

\fi


\usepackage[french]{babel}
\usepackage[
a4paper,
margin=2cm,
nomarginpar,% We don't want any margin paragraphs
]{geometry}
\usepackage{icomma}

\usepackage{fancyhdr}
\usepackage{array}
\usepackage{hyperref}

\usepackage{multicol, enumerate}
\newcolumntype{P}[1]{>{\centering\arraybackslash}p{#1}}


\usepackage{stackengine}
\newcommand\xrowht[2][0]{\addstackgap[.5\dimexpr#2\relax]{\vphantom{#1}}}

% theorems

\theoremstyle{plain}
\newtheorem{theorem}{Th\'eor\`eme}
\newtheorem*{sol}{Solution}
\theoremstyle{definition}
\newtheorem{ex}{Exercice}

% corps
\newcommand{\C}{\mathbb{C}}
\newcommand{\R}{\mathbb{R}}
\newcommand{\Rnn}{\mathbb{R}^{2n}}
\newcommand{\Z}{\mathbb{Z}}
\newcommand{\N}{\mathbb{N}}
\newcommand{\Q}{\mathbb{Q}}

% variance
\newcommand{\Var}[1]{\text{Var}(#1)}

% domain
\newcommand{\D}{\mathbb{D}}


% date
\usepackage{advdate}
\AdvanceDate[1]


% plots
\usepackage{pgfplots}

% table line break
\usepackage{makecell}

%subfigures
\usepackage{subcaption}

\definecolor{myg}{RGB}{56, 140, 70}
\definecolor{myb}{RGB}{45, 111, 177}
\definecolor{myr}{RGB}{199, 68, 64}


% SOLUTION SWITCH
\newif\ifsolutions
				\solutionstrue
				%\solutionsfalse

\ifsolutions
	\newcommand{\exe}[2]{
		\begin{ex} #1  \end{ex}
		\begin{sol} #2 \end{sol}
	}
\else
	\newcommand{\exe}[2]{
		\begin{ex} #1  \end{ex}
	}
	
\fi

\begin{document}
\pagestyle{fancy}
\fancyhead[L]{Seconde 13}
\fancyhead[C]{\textbf{Statistiques 3 \ifsolutions -- Solutions  \fi}}
\fancyhead[R]{\today}

\exe{
	Calculer la moyenne de chaque série statistique suivante.
		\begin{multicols}{2}
		\begin{enumerate}
			\item $(0; 20)$
			\item $(0; 0; 20)$
			\item $(0; 20; 20)$
			\item $(-10; 23 ; -10 ; -23 ; 10 ; 10)$
			\item $(-50 ; -100 ; -23 ; -31; -50 ; -50)$
			\item $(0; 1;2;3; \dots; 18; 19; 20)$
		\end{enumerate}
		\end{multicols}
}{

		\begin{enumerate}
			\item $\overline{X} = \dfrac{0+20}{2} = 10$
			\item $\overline{X} = \dfrac{0+0+20}{3} = \dfrac{20}3 \approx 6,67$
			\item $\overline{X} = \dfrac{0+20+20}{3} = \dfrac{40}3 \approx 13,33$
			\item $\overline{X} = \dfrac{-10+23-10-23+10+10}{6} = 0$. La série est symétrique par rapport à $0$ (chaque valeur a son opposé du même effectif).
			\item $\overline{X} = \dfrac{-50-100-23-31-50-50}{6} = -\dfrac{304}6 = -\dfrac{152}3 \approx -50,67$
			\item $\overline{X} = \dfrac{0+ 1+2+3+ \dots+ 18+ 19+ 20}{21} = 10 $. La série est symétrique par rapport à $10$.
		\end{enumerate}
		
		Le dernier exercice se généralise pour donner une formule très utile.
		La série statistique $X=(0; 1 ; 2; 3 ; \dots; 2n-1 ; 2n)$ est symétrique par rapport à $n$ pour n'importe quel entier naturel $n\in\N$.
		Sa moyenne est donc de $n$, ce qui est équivalent à dire
			\begin{align*}
				\dfrac{0 + 1 + \dots + 2n-1 + 2n}{2n+1} &= n \\
				0 + 1 + \dots + 2n-1 + 2n &= n\cdot (2n+1)
			\end{align*}
		Vérification : pour $n=10$, on retrouve $10\cdot 21 = 210$ pour la somme des entiers naturels de $0$ à $20$.
		Vérifiez la formule obtenue pour $n=2; 3; 5$ pour se rassurer encore davantage !
}

\exe{\label{ex:33}
	Pour chaque série statistique suivante, calculer la moyenne.
		\begin{multicols}{2}
		\begin{enumerate}
			\item 
				\begin{tabular}{|c|c|c|c|c|}\hline
				Valeur   & 9 & 13 & 11 & 7 \\ \hline
				Effectif & 5 & 12 & 5 & 12 \\ \hline
				\end{tabular}
				
			\item 
				\begin{tabular}{|c|c|c|c|c|}\hline
				Valeur   & 2 & 13 & 18 & 7 \\ \hline
				Effectif & 5 & 12 & 5 & 12 \\ \hline
				\end{tabular}
				
			\item 
				\begin{tabular}{|c|c|c|c|c|}\hline
				Valeur   & 0 & 20 & 2 & 5 \\ \hline
				Effectif & 5 & 14 & 10 & 2  \\ \hline
				\end{tabular}

			\item 
				\begin{tabular}{|c|c|c|c|}\hline
				Valeur   & 0 & 20 & 10 \\ \hline
				Effectif & 17 & 17 & 1 \\ \hline
				\end{tabular}


		\end{enumerate}
		\end{multicols}
}{
		\begin{enumerate}
			\item 
				$\overline{X} = \dfrac{9\cdot5 + 13\cdot12 + 11\cdot5 + 7\cdot12}{5+12+5+12} = 10$
				
			\item 
				$\overline{X} = \dfrac{2\cdot5 + 13\cdot12 + 18\cdot5 + 7\cdot12}{5+12+5+12} = 10$
				
			\item 
				$\overline{X} = \dfrac{0\cdot5 + 20\cdot14 + 2\cdot10 + 5\cdot2}{5+14+10+2} = 10$

			\item 
				$\overline{X} = \dfrac{0\cdot17 + 20\cdot17 + 10\cdot1}{17+17+1} = 10$

		\end{enumerate}

}




\exe{
	Pour chaque série statistique suivante, remplir la ligne \og Fréquence \fg \, et calculer la moyenne.
		\begin{multicols}{2}
		\begin{enumerate}
			\item 
				\begin{tabular}{|c|c|c|}\hline
				Valeur   & 0 & 20 \\ \hline
				Effectif & 1 & 2  \\ \hline
				Fréquence & &  \\ \hline
				\end{tabular}
				
			\item 
				\begin{tabular}{|c|c|c|}\hline
				Valeur   & 0 & 20 \\ \hline
				Effectif & 2 & 1  \\ \hline
				Fréquence & &  \\ \hline
				\end{tabular}
				
			\item 
				\begin{tabular}{|c|c|c|c|c|}\hline
				Valeur   & 11 & 13 & 9 & 7 \\ \hline
				Effectif & 2 & 1 & 5 & 10 \\ \hline
				Fréquence & &&&  \\ \hline
				\end{tabular}
				
			\item 
				\begin{tabular}{|c|c|c|c|c|}\hline
				Valeur   & -23 & -1 & 1 & 23 \\ \hline
				Effectif & 67 & 12 & 4 & 17 \\ \hline
				Fréquence & &&&  \\ \hline
				\end{tabular}

		\end{enumerate}
		\end{multicols}
}{
		\begin{enumerate}
			\item 
				La somme des effectifs $N$ est donnée par $N = 1+2 = 3$.
				Pour trouver la fréquence d'une valeur, on divise son effectif par $N$.
				
				\begin{tabular}{|c|c|c|}\hline
				Valeur   & 0 & 20 \\ \hline
				Effectif & 1 & 2  \\ \hline
				Fréquence & $\frac13$ & $\frac23$  \\ \hline
				\end{tabular}
				
				La moyenne est la somme des produits Valeur $\times$ Fréquence.
				D'où $\overline{X} = 0 \cdot \frac13 + 20 \cdot \frac23 =\dfrac{40}3 \approx 13,33$, qu'on comparera à l'exercice $1$.
				
			\item 
				La somme des effectifs $N$ est donnée par $N = 1+2 = 3$.
				Pour trouver la fréquence d'une valeur, on divise son effectif par $N$.
				
				\begin{tabular}{|c|c|c|}\hline
				Valeur   & 0 & 20 \\ \hline
				Effectif & 2 & 1  \\ \hline
				Fréquence & $\frac23$ & $\frac13$ \\ \hline
				\end{tabular}
				
				D'où $\overline{X} = 0 \cdot \frac23 + 20 \cdot \frac13 =\dfrac{20}3 \approx 6,67$, qu'on comparera à l'exercice $1$.
				
			\item 
				La somme des effectifs $N$ est donnée par $N = 2+1+5+10 = 18$.
				Pour trouver la fréquence d'une valeur, on divise son effectif par $N$.
				
				\begin{tabular}{|c|c|c|c|c|}\hline
				Valeur   & 11 & 13 & 9 & 7 \\ \hline
				Effectif & 2 & 1 & 5 & 10 \\ \hline
				Fréquence & $\frac2{18}$ & $\frac1{18}$ & $\frac5{18}$ & $\frac{10}{18}$  \\ \hline
				\end{tabular}
				
				D'où $\overline{X} = 11\cdot\frac2{18} + 13\cdot \frac1{18} + 9\cdot \frac5{18} 7\cdot\frac{10}{18} = \dfrac{150}{18} \approx 8,33 $.
				
			\item 
				La somme des effectifs $N$ est donnée par $N = 67+12+4+17 = 100$.
				Pour trouver la fréquence d'une valeur, on divise son effectif par $N$.
				
				\begin{tabular}{|c|c|c|c|c|}\hline
				Valeur   & -23 & -1 & 1 & 23 \\ \hline
				Effectif & 67 & 12 & 4 & 17 \\ \hline
				Fréquence & $\frac{67}{100}$ & $\frac{12}{100}$ & $\frac{4}{100}$ & $\frac{17}{100}$ \\ \hline
				\end{tabular}
				
				D'où $\overline{X} = (-23) \cdot \frac{67}{100} + (-1)\cdot \frac{12}{100} + 1\cdot \frac{4}{100} + 23\cdot \frac{17}{100} = -11,58$.

		\end{enumerate}


}



\exe{
	Un élève reçoit $5$ notes listées ainsi : $(4,25 ; 10; 18,5 ; 15,5 ; 11,25)$.
	À ces notes sont associés $5$ coefficients, donnés dans le même ordre par $(0,5 ; 1,4 ; 0,1 ; 0,5; 1,1)$.
	
	Calculer la moyenne pondérée des notes.
}{
	Il y a deux manières de faire.
	D'une part, on peux calculer la somme des produits Note $\times$ Coefficient puis diviser par la somme des coefficients.
		\[ \dfrac{4,25\cdot0,5 + 10\cdot1,4 + 18,5\cdot0,1 + 15,5\cdot 0,5 + 11,25\cdot1,1}{0,5+1,4+0,1+0,5+1,1} = \dfrac{38,1}{3,6}  \approx 10,58. \]
	D'autre part, on peut calculer le poids de chaque note en normalisant les coefficients, c'est-à-dire en divisant chaque coefficient par leur somme.
	
	$p_1 = \dfrac{0,5}{3,6}$, $p_2 = \dfrac{1,4}{3,6}$, $p_3 = \dfrac{0,1}{3,6}$, $p_4 = \dfrac{0,5}{3,6}$, et $p_5 = \dfrac{1,1}{3,6}$, puis
		\[ 4,25 \cdot p_1 + 10\cdot p_2 + 18,5 \cdot p_3 + 15,5 \cdot p_4 + 11,25\cdot p_5 = 10,58. \]

	La première méthode est plus facile à employer mais elle découle en fait de la seconde : une moyenne est toujours une combinaison convexe de valeurs car les poids somment toujours à $1$.
}


\exe{
	La moyenne des salaires mensuels d'une entreprise est de $1860$€.
	L'inflation désigne une augmentation générale des prix. Ainsi, $6\%$ d'inflation signifie que, en moyenne, les prix augmentent de $6\%$.
	Réciproquement et afin de pouvoir comparer les pouvoirs d'achats, on peut fixer les prix et voir l'inflation comme une diminution du salaire (en l'occurrence de $1-\frac{1}{1,06} \approx 5,6 \%$).
	
	L'entreprise, après une année où l'inflation se mesurait à $6\%$ décide d'ajouter des primes aux salaires : tous les employés recevront $50$€ mensuellement en plus. Ces $50$€ sont, eux aussi, soumis à l'inflation.
	
	Quelle est la nouvelle moyenne des salaires mensuels de l'entreprise ?

}{
	Notons $X$ la série statistique des salaires avant inflation.
	On utilise ici la linéarité de la moyenne.
	
	La prime s'ajoute après l'inflation, donc la moyenne après inflation et prime s'élève à $\overline{X+50} = \overline{X} + 50$.
	
	Enfin, afin de corriger l'inflation et d'après le texte, on multiplie tous les nouveaux salaires par $(1,06)^{-1}$.
	La nouvelle moyenne est donc donnée par 
		\[ \overline{\frac{1}{1,06} \cdot \left( X + 50 \right) } = \frac{1}{1,06} \cdot \left( \overline{X} + 50 \right) = \frac{1}{1,06} \cdot (1860 + 50) \approx 1801.89. \]
	À \og euro fixe \fg \, et en moyenne, la prime de permet pas de compenser l'inflation.
}


\exe{
	Calculer et comparer les écart types des séries de l'exercice \ref{ex:33}.
}{
	On calcule d'abord la variance puis l'écart type.
	D'après le cours, $\Var{X} = \overline{(X - \overline{X})^2}$ et $\sigma(X) = \sqrt{\Var{X}}$.
	
	\begin{enumerate}
		\item 
			\[ \Var{X} = \dfrac{5\cdot(9-10)^2 + 12 \cdot(13-10)^2 + 5\cdot(11-10)^2 + 12 \cdot(7-10)^2}{5+12+5+12} = \dfrac{226}{34}. \]
		
		D'où $\sigma(X) = \sqrt{\dfrac{226}{34}} \approx 2,578$.
		
		\item  
			\[\Var{X} = \dfrac{5\cdot(2-10)^2 + 12 \cdot(13-10)^2 + 5\cdot(18-10)^2 + 12 \cdot(7-10)^2}{5+12+5+12} = \dfrac{856}{34}. \]
		
		D'où $\sigma(X) \approx 5,02$.
		
		\item  
			\[\Var{X} = \dfrac{5\cdot(0-10)^2 + 14 \cdot(20-10)^2 + 10\cdot(2-10)^2 + 2 \cdot(5-10)^2}{5+14+10+2} = \dfrac{2590}{31} . \]
		
		D'où $\sigma(X) \approx 9,14.$
		
		\item  
			\[\Var{X} = \dfrac{17\cdot(0-10)^2 + 17 \cdot(20-10)^2 + 1\cdot(10-10)^2 }{17+17+1} = \dfrac{3400}{35}. \]
		
		D'où $\sigma(X) \approx 9,86$.
	\end{enumerate}

	Les écarts types augmentent car les séries statistiques sont de moins en moins concentrées autour de leur moyenne $10$.
}

\exe{
	\begin{multicols}{2}
	On considère la série ci-contre tirée de l'exercice \ref{ex:33}.
	
		\begin{center}
		\begin{tabular}{|c|c|c|c|}\hline
		Valeur   & 0 & 20 & 10 \\ \hline
		Effectif & 17 & 17 & 1 \\ \hline
		\end{tabular}
		\end{center}
	\end{multicols}
		
	Modifier l'effectif de la valeur $10$ en $10, 20, 50, 100$, et calculer la moyenne et l'écart type de la série obtenue.
	Comment évolue $\sigma$ ? La série devient-elle plus ou moins concentrée autour de sa moyenne ?
}{
	Remplaçons l'effectif de $10$ par $k$ pour obtenir directement une formule générale qu'on pourra évaluer en $k=10, 20, 50, 100$.

	D'abord, la moyenne ne change pas :
		\[ \overline{X} = \dfrac{0\cdot17 + 20\cdot17 + 10\cdot k}{17+17+k} = \dfrac{10\cdot(34 + k)}{34+k} = 10, \]
	ce qui n'est pas surprenant car ajouter une valeur moyenne ne change pas la moyenne générale.

	On a ainsi 
		\[ \Var{X} = \dfrac{17\cdot(0-10)^2+17\cdot(20-10)^2 + k\cdot(10-10)^2}{17+17+k} = \dfrac{3400}{34+k}. \]
		
	Pour $k=1$, on retrouve l'écart type $\sigma(X) = \sqrt{\dfrac{3400}{35}} \approx 9,86$ de l'exercice précédent.
	Les autres valeurs sont listées dans le tableau ci-dessous.
	
	\begin{center}
	\begin{tabular}{|c|c|c|c|c|c|}\hline
		$k$ & 1 & 10 & 20 & 50 & 100 \\ \hline
		$\sigma(X) \approx$ & 9,86 & 8,79 & 7,93 & 6,36 & 5,04 \\ \hline
	\end{tabular}
	\end{center}
	
	On remarque que $\sigma$ diminue quand $k$ augmente.
	En effet, plus le dénominateur de la fraction $\Var{X} = \frac{3400}{34+k}$ augmente, plus la variance diminue, et donc plus l'écart type diminue aussi.
	
	La série statistique devient de plus en plus concentrée autour de sa moyenne car il y a de plus en plus de valeurs égales à celle-ci.
}

\exe{
	Écrire un tableau Valeur/Effectif pour chaque histogramme de la figure \ref{fig:hist} (au dos).
	On assignera la valeur moyenne $k+\frac12$ à chaque élément d'une classe $[k; k+1[, k\in\N$.
	Par exemple, de l'histogramme \ref{fig:a} on lit $4$ notes de valeur $16,5$.
	
	Ensuite, pour chaque série obtenue, calculer
		\begin{multicols}{2}
		\begin{enumerate}[i)]
			\item La moyenne ;
			\item L'écart type ;
			\item La médiane ;
			\item Le premier quartile ;
			\item Le troisième quartile ; et
			\item L'écart interquartile.
		\end{enumerate}
		\end{multicols}
}{
	On traite la série statistique $X$ décrite par le premier histogramme \ref{fig:a}.
	On vérifie les résultats à l'aide d'un tableur : 
	voir le fichier joint \texttt{hist(a).csv}.
	
		\begin{center}
		\begin{tabular}{|c|c|c|c|c|c|c|c|c|c|c|c|c|c|c|}\hline
		Note & 2,5 & 5,5 & 6,5 &7,5 & 8,5 & 9,5 & 10,5 & 12,5 & 13,5 & 14,5 & 15,5 & 16,5 & 17,5 & 18,5 \\ \hline
		Effectif & 1 & 1 & 1 & 3 & 1 & 5 & 1 & 3 & 4 & 3 & 3 & 5 & 1 & 2 \\ \hline
		\end{tabular}
		\end{center}
		
	L'effectif total est donné par $N= 1+1+1+3+1+5+1+3+4+3+3+5+1+2 = 34$
	
		\begin{enumerate}[i)]
			\item
				\[ \overline{X} = \dfrac{2,5+5,5+6,5+7,5\cdot3 + \dots  + 16,5\cdot5 + 17,5+18,5\cdot2}{34} = \dfrac{422}{34} \approx 12,411 \]
			\item 
				\[ \sigma(X) \approx 4,06. \]
			\item 
				$N=34$ est pair donc prend la moyenne des valeur de rang $17$ et $18$ en vérifiant que la série soit bien rangée par ordre croissant.
				Comme $1+1+1+3+1+5+1+3 = 16$, ces deux valeurs sont $13,5$ et la médiane est donc $13,5$.
			\item 
				On prend le rang immédiatement supérieur ou égal à $N/4 = 8,5$, c'est-à-dire $9$.
				Le premier quartile est donc donné par $Q_1 = 9,5$.
			\item
				On prend le rang immédiatement supérieur ou égal à $3N/4 = 25,5$, c'est-à-dire $26$.
				Le troisième quartile est donc donné par $Q_3 = 15,5$. 
			\item
				L'écart interquartile est calculé, par définition, en posant
					\[ |Q_1 - Q_3| = |9,5 - 15,5| = 6. \]
				
		\end{enumerate}
}

%\subsection*{Exercices supplémentaires}

\exe{$(\star)$
	Pour quel(s) $N\in\N$ la moyenne de la série suivante est-elle $10$ ? Comparer avec la série de l'exercice \ref{ex:33}.
	
		\begin{center}
		\begin{tabular}{|c|c|c|c|c|}\hline
		Valeur   & 0 & 20 & 2 & 5 \\ \hline
		Effectif & 5 & 14 & N & 2  \\ \hline
		\end{tabular}
		\end{center}
}{
	La contrainte $\overline{X} = 10$ se traduit par l'égalité suivante à résoudre pour $N\in\N$.
	
		\begin{align*}
			\overline{X} &= 10 \\
			\dfrac{5\cdot0 + 14\cdot20 + N\cdot2 + 2\cdot5}{5+14+N+2} &= 10 \\
			0  + 280 + 2\cdot N + 10 &= 10 \cdot(21 + N) \\
			290 + 2 \cdot N &= 210 + 10 \cdot N \\
			80 &= 8 \cdot N \\
			N &= 10		
		\end{align*}

	$10$ est bien un entier naturel, donc $N=10$ est l'unique solution.
}

\exe{$(\star)$
	Est-il possible que la série suivante ait une moyenne de $10$ pour un certain $N\in\N$ ? Si oui, lequel ?
	
		\begin{center}
		\begin{tabular}{|c|c|c|c|c|}\hline
		Valeur   & 23 & 12 & 2 & 7 \\ \hline
		Effectif & 5 & 14 & N & 2  \\ \hline
		\end{tabular}
		\end{center}
}{
	La contrainte $\overline{X} = 10$ se traduit par l'égalité suivante à résoudre pour $N\in\N$.
	
		\begin{align*}
			\overline{X} &= 10 \\
			\dfrac{5\cdot23 + 14\cdot12 + N\cdot2 + 2\cdot7}{5+14+N+2} &= 10 \\
			115  + 168 + 2\cdot N + 14 &= 10 \cdot(21 + N) \\
			297 + 2 \cdot N &= 210 + 10 \cdot N \\
			87 &= 8 \cdot N \\
			N &= \dfrac{87}8		
		\end{align*}

	L'unique solution candidate est donc $\dfrac{87}{8}$ qui n'est pas un entier naturel.
	Donc il n'est pas possible que cette série statistique ait une moyenne de $10$, peu importe l'effectif $N\in\N$.
	
	En prenant une valeur approchée, par exemple $N=11$, la moyenne est de $9,96$, par exactement $10$ !
}



\begin{figure}[t!]
  \centering
  \begin{subfigure}[b]{.45\textwidth}
    \centering
  \begin{tikzpicture}[scale=1]
    \begin{axis}[
        ymin=0, ymax=8,
        %minor y tick num = 3,
        xtick = {2, 4, ..., 16, 18},
        area style,
        xlabel = {Note sur $20$},
        ylabel = {Effectif}
      ]
      \addplot+[ybar interval,mark=no, draw=myg, fill=myg!50] plot coordinates {
        (2,1) (3,0) (4,0) (5,1) (6,1) (7,3) (8,1) (9,5) (10,1) (11,0) (12,3) (13,4) (14,3) (15,3) (16,5) (17,1) (18,2) (19,0)
      };
    \end{axis} 
  \end{tikzpicture}
  \caption{Évaluation ``Droite réelle''.}
  \label{fig:a}
  \end{subfigure}
  \hfill
  % NOT LINE BREAK!!
  \begin{subfigure}[b]{.45\textwidth}
    \centering
  \begin{tikzpicture}[scale=1]
    \begin{axis}[
        ymin=0, ymax=8,
        %minor y tick num = 3,
        %xtick = {4, 5, ..., 18, 19},
        xtick = {2, 4, ..., 16, 18},
        area style,
        xlabel = {Note sur $20$},
        ylabel = {Effectif}
      ]
      \addplot+[ybar interval,mark=no, draw=myb, fill=myb!50] plot coordinates {
        (4,1) (5,0) (6,1) (7,0) (8,2) (9,1) (10,4) (11,2) (12,7) (13,7) (14,3) (15,0) (16,2) (17,2) (18,1) (19,0)
      };
    \end{axis} 
  \end{tikzpicture}
  \caption{Évaluation ``Droite et plan''.}
  \end{subfigure}
    \begin{subfigure}[b]{.45\textwidth}
    \centering
  \begin{tikzpicture}[scale=1]
    \begin{axis}[
        ymin=0, ymax=8,
        %minor y tick num = 3,
        xtick = {2, 4, ..., 16, 18},
        area style,
        xlabel = {Note sur $20$},
        ylabel = {Effectif}
      ]
      \addplot+[ybar interval,mark=no, draw=myr, fill=myr!50] plot coordinates {
      		(2,1) (3,0) (8,2) (9,0) (10, 1) (11,4) (12,1) (13,4) (14,4) (15,7) (16,7) (17,1) (18,2) (19,0)
      };
    \end{axis} 
  \end{tikzpicture}
  \caption{Automatismes 1 à 5.}
  \end{subfigure}
  
  
  \caption{Histogrammes de notes (min $0$, max $20$). Les classes sont de la forme $[k;k+1[, k\in\N$. 
  La colonne entre $12$ et $13$ compte donc toutes les notes appartenant à $[12;13[$.
  \\
  Lecture : $5$ notes de l'évaluation ``Droite réelle'' appartiennent à l'intervalle $[9;10[$.}
  \label{fig:hist}
\end{figure}



\end{document}

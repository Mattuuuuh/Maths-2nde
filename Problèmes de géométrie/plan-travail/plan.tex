\documentclass[12pt]{beamer}
\usepackage[french]{babel}

\usetheme{CambridgeUS}
\usecolortheme{rose}
\beamertemplatenavigationsymbolsempty


%boxes
\usepackage[most]{tcolorbox}
\usepackage{multicol}


\usepackage{libertinus}
\usepackage{amsmath,amsfonts,amsthm,amssymb,mathtools}
\usepackage{array}
\newcolumntype{P}[1]{>{\centering\arraybackslash}p{#1}}


\usepackage{stackengine}
\newcommand\xrowht[2][0]{\addstackgap[.5\dimexpr#2\relax]{\vphantom{#1}}}


% corps
\usepackage{calrsfs}
\newcommand{\C}{\mathcal{C}}
\newcommand{\R}{\mathbb{R}}
\newcommand{\Rnn}{\mathbb{R}^{2n}}
\newcommand{\Z}{\mathbb{Z}}
\newcommand{\N}{\mathbb{N}}
\newcommand{\Q}{\mathbb{Q}}

% domain
\newcommand{\D}{\mathbb{D}}


% date
\usepackage{advdate}
\AdvanceDate[0]

%plots
\usepackage{pgfplots, subcaption}
\definecolor{myg}{RGB}{56, 140, 70}
\definecolor{myb}{RGB}{45, 111, 177}
\definecolor{myr}{RGB}{199, 68, 64}

%icomma
\usepackage{icomma}
\begin{document}

\begin{frame}{Répartition des exercices}

\begin{center}
\begin{tabular}{|P{.2\textwidth}|P{.17\textwidth}|P{.2\textwidth}|P{.28\textwidth}|}\hline
		& Application & Entraînement & Approfondissement \\ \hline
	Projeté orthogonal & 1, 2, 3 & 4, 5 & \\ \hline
	Calcul de hauteur, d'aire, de volume & 1, 3 & 2 & 4 \\ \hline
	Trigonométrie & 1, 2 & 3, 4, 5 & 6 \\ \hline
	Fonctions et optimisation & 1 & 2 & \\ \hline
\end{tabular}
\end{center}

\end{frame}

\end{document}
% DYSLEXIA SWITCH
\newif\ifdys
		
				% ENABLE or DISABLE font change
				% use XeLaTeX if true
				\dystrue
				\dysfalse


\ifdys

\documentclass[a4paper, 14pt]{extarticle}
\usepackage{amsmath,amsfonts,amsthm,amssymb,mathtools}

\tracinglostchars=3 % Report an error if a font does not have a symbol.
\usepackage{fontspec}
\usepackage{unicode-math}
\defaultfontfeatures{ Ligatures=TeX,
                      Scale=MatchUppercase }

\setmainfont{OpenDyslexic}[Scale=1.0]
\setmathfont{Fira Math} % Or maybe try KPMath-Sans?
\setmathfont{OpenDyslexic Italic}[range=it/{Latin,latin}]
\setmathfont{OpenDyslexic}[range=up/{Latin,latin,num}]

\else

\documentclass[a4paper, 12pt]{extarticle}

\usepackage[utf8x]{inputenc}
\usepackage{lmodern,textcomp}
\usepackage{amsmath,amsfonts,amsthm,amssymb,mathtools}

\fi


\usepackage[french]{babel}
\usepackage[
a4paper,
margin=2cm,
nomarginpar,% We don't want any margin paragraphs
]{geometry}
\usepackage{icomma}

\usepackage{fancyhdr}
\usepackage{array}

\usepackage{multicol, enumerate}
\newcolumntype{P}[1]{>{\centering\arraybackslash}p{#1}}


\usepackage{stackengine}
\newcommand\xrowht[2][0]{\addstackgap[.5\dimexpr#2\relax]{\vphantom{#1}}}

% theorems

\theoremstyle{plain}
\newtheorem{theorem}{Th\'eor\`eme}
\newtheorem*{theorem*}{Th\'eor\`eme}
\newtheorem*{sol}{Solution}
\theoremstyle{definition}
\newtheorem{ex}{Exercice}

% corps
\newcommand{\C}{\mathcal{C}}
\newcommand{\R}{\mathbb{R}}
\newcommand{\Rnn}{\mathbb{R}^{2n}}
\newcommand{\Z}{\mathbb{Z}}
\newcommand{\N}{\mathbb{N}}
\newcommand{\Q}{\mathbb{Q}}

% domain
\newcommand{\D}{\mathcal{D}}


% date
\usepackage{advdate}
\AdvanceDate[0]


% plots
\usepackage{pgfplots}

% for calligraphic C
\usepackage{calrsfs}

% SOLUTION SWITCH
\newif\ifsolutions
				\solutionstrue
				\solutionsfalse

\ifsolutions
	\newcommand{\exe}[2]{
		\begin{ex} #1  \end{ex}
		\begin{sol} #2 \end{sol}
	}
\else
	\newcommand{\exe}[2]{
		\begin{ex} #1  \end{ex}
	}
	
\fi

\begin{document}
\pagestyle{fancy}
\fancyhead[L]{Seconde 13}
\fancyhead[C]{\textbf{Projeté orthogonal  \ifsolutions -- Solutions  \fi}}
\fancyhead[R]{\today}

\exe{
	Tracer une droite quelconque et un point n'appartenant pas à cette droite.
	À l'aide uniquement d'un compas, tracer le projeté orthogonal du point sur la droite.
}
{}

\exe{ \,
	\begin{multicols}{2}
	\begin{center}
	\begin{tikzpicture}[scale=.8]
		\draw[-, thick, black] (0,0) -- (8, 3) node[below]{$(d)$};
		\draw[-, thick, black] (0,3) -- (9, 1) node[below]{$(d')$};
	
		\node[black] at (1, 3/8) {$\bullet$};
		\node[black, above] at (1, 3/8) {$A$};
		
		\node[black] at (1.5, 24/9) {$\bullet$};
		\node[black, above] at  (1.5, 24/9) {$B$};
	\end{tikzpicture}
	\end{center}
	
	Dans la figure ci-dessous, tracer un point $C$ tel que $A$ soit le projeté orthogonal de $C$ sur $(d)$ et que $B$ soit le projeté orthogonal de $C$ sur $(d')$.
	\end{multicols}
}
{}


\exe{
	Dans la figure ci-dessous, tracer
		\begin{enumerate}
			\item le point $B$, projeté orthogonal de $A$ sur $(d')$ ;
			\item le point $C$, projeté orthogonal de $B$ sur $(d)$ ; et
			\item le point $D$, projeté orthogonal de $C$ sur $(d')$.
		\end{enumerate}

	\begin{center}
	\begin{tikzpicture}[scale=.8]
		\draw[-, thick, black] (0,0) -- (8, 1) node[below]{$(d)$};
		\draw[-, thick, black] (0,1) -- (8, -2) node[below]{$(d')$};
	
		\node[black] at (7, 7/8) {$\bullet$};
		\node[black, above] at  (7, 7/8)  {$A$};
	\end{tikzpicture}
	\end{center}
		
	Que dire des droites $(AB)$ et $(CD)$ ? Justifier.
}
{}

\exe{
	Un homme est dans une salle pentagonale comme schématisé ci-dessous.
	Celui-ci souhaite toucher les murs $[AB], [CD], [AE], [BC],$ et $[ED]$ dans cet ordre en effectuant le trajet le plus court à chaque déplacement.
	Tracer ses déplacements sur la figure si son point de départ est le point $M$.
	
	\begin{center}
	\begin{tikzpicture}
		\draw[-, thick, black] (0,0) node{$\bullet$} -- (2.8, -1.2) node{$\bullet$};
		\draw[-, thick, black] (2.8,-1.2) node{$\bullet$} -- (4, 1.5) node{$\bullet$};
		\draw[-, thick, black] (4, 1.5) node{$\bullet$} -- (1.95,3) node{$\bullet$};
		\draw[-, thick, black] (1.95,3) node{$\bullet$} -- (.3,2.3)  node{$\bullet$};
		\draw[-, thick, black] (.3,2.3) node{$\bullet$} -- (0,0) node{$\bullet$};
		
		\node[black, below] at  (0,0)  {$A$};
		\node[black, right] at  (2.8,-1.2)  {$B$};
		\node[black, right] at  (4, 1.5)  {$C$};
		\node[black, above] at  (1.95,3)  {$D$};
		\node[black, left] at (.3,2.3)  {$E$};
		
		
		\node[black, above] at (1.2, 1) {$M$};
		\node[black] at (1.2, 1) {$\times$};
	
	\end{tikzpicture}
	\end{center}
		
}{}


\exe{
	On considère une droite $(d)$ et les points $A$ et $B$ distincts n'appartenant pas à la droite $(d)$.
	On note $A'$ et $B'$ les projetés orthogonaux respectifs des points $A$ et $B$ sur la droite $(d)$.
	
	Calculer la longueur $AB$ sachant que $AA' =1, BB' = 5$, et $A'B' = 4$. 
	Distinguer deux cas selon que les points sont ou non du même côté de $(d)$.

}{}


\end{document}

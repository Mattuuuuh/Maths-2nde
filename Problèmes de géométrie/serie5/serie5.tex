				% ENABLE or DISABLE font change
				% use XeLaTeX if true
\newif\ifdys
				\dystrue
				\dysfalse

\newif\ifsolutions
				\solutionstrue
				\solutionsfalse

% DYSLEXIA SWITCH
\newif\ifdys
		
				% ENABLE or DISABLE font change
				% use XeLaTeX if true
				\dystrue
				\dysfalse


\ifdys

\documentclass[a4paper, 14pt]{extarticle}
\usepackage{amsmath,amsfonts,amsthm,amssymb,mathtools}

\tracinglostchars=3 % Report an error if a font does not have a symbol.
\usepackage{fontspec}
\usepackage{unicode-math}
\defaultfontfeatures{ Ligatures=TeX,
                      Scale=MatchUppercase }

\setmainfont{OpenDyslexic}[Scale=1.0]
\setmathfont{Fira Math} % Or maybe try KPMath-Sans?
\setmathfont{OpenDyslexic Italic}[range=it/{Latin,latin}]
\setmathfont{OpenDyslexic}[range=up/{Latin,latin,num}]

\else

\documentclass[a4paper, 12pt]{extarticle}

\usepackage[utf8x]{inputenc}
%fonts
\usepackage{amsmath,amsfonts,amsthm,amssymb,mathtools}
% comment below to default to computer modern
\usepackage{libertinus,libertinust1math}

\fi


\usepackage[french]{babel}
\usepackage[
a4paper,
margin=2cm,
nomarginpar,% We don't want any margin paragraphs
]{geometry}
\usepackage{icomma}

\usepackage{fancyhdr}
\usepackage{array}
\usepackage{hyperref}

\usepackage{multicol, enumerate}
\newcolumntype{P}[1]{>{\centering\arraybackslash}p{#1}}


\usepackage{stackengine}
\newcommand\xrowht[2][0]{\addstackgap[.5\dimexpr#2\relax]{\vphantom{#1}}}

% theorems

\theoremstyle{plain}
\newtheorem{theorem}{Th\'eor\`eme}
\newtheorem*{sol}{Solution}
\theoremstyle{definition}
\newtheorem{ex}{Exercice}
\newtheorem*{rpl}{Rappel}
\newtheorem{enigme}{Énigme}

% corps
\usepackage{calrsfs}
\newcommand{\C}{\mathcal{C}}
\newcommand{\R}{\mathbb{R}}
\newcommand{\Rnn}{\mathbb{R}^{2n}}
\newcommand{\Z}{\mathbb{Z}}
\newcommand{\N}{\mathbb{N}}
\newcommand{\Q}{\mathbb{Q}}

% variance
\newcommand{\Var}[1]{\text{Var}(#1)}

% domain
\newcommand{\D}{\mathcal{D}}


% date
\usepackage{advdate}
\AdvanceDate[0]


% plots
\usepackage{pgfplots}

% table line break
\usepackage{makecell}
%tablestuff
\def\arraystretch{2}
\setlength\tabcolsep{15pt}

%subfigures
\usepackage{subcaption}

\definecolor{myg}{RGB}{56, 140, 70}
\definecolor{myb}{RGB}{45, 111, 177}
\definecolor{myr}{RGB}{199, 68, 64}

% fake sections with no title to move around the merged pdf
\newcommand{\fakesection}[1]{%
  \par\refstepcounter{section}% Increase section counter
  \sectionmark{#1}% Add section mark (header)
  \addcontentsline{toc}{section}{\protect\numberline{\thesection}#1}% Add section to ToC
  % Add more content here, if needed.
}


% SOLUTION SWITCH
\newif\ifsolutions
				\solutionstrue
				%\solutionsfalse

\ifsolutions
	\newcommand{\exe}[2]{
		\begin{ex} #1  \end{ex}
		\begin{sol} #2 \end{sol}
	}
\else
	\newcommand{\exe}[2]{
		\begin{ex} #1  \end{ex}
	}
	
\fi


% tableaux var, signe
\usepackage{tkz-tab}


%pinfty minfty
\newcommand{\pinfty}{{+}\infty}
\newcommand{\minfty}{{-}\infty}

\begin{document}


\AdvanceDate[0]

\begin{document}
\pagestyle{fancy}
\fancyhead[L]{Seconde 13}
\fancyhead[C]{\textbf{Racines carrées \ifsolutions -- Solutions  \fi}}
\fancyhead[R]{\today}

\begin{definition*}{Racine carrée}
	Soit $a \in \R, a\geq0$ un réel positif ou nul.
	La quantité $\sqrt{a}$ est l'unique nombre réel positif ou nul vérifiant
		\[ \sqrt{a}^2 = a. \]
\end{definition*}

\exe{
	Compléter les pointillés.
	\begin{multicols}{2}
	\begin{enumerate}
		\item $\sqrt{25} = \dots$
		\item $\sqrt{81} = \dots$
		\item $\sqrt{121} = \dots$
		\item $\sqrt{\quad\dots\quad} = 25$
		\item $\sqrt{\quad\dots\quad} = 12$
		\item $\sqrt{\quad\dots\quad} = 10^3$
	\end{enumerate}
	\end{multicols}
}
{}

\exe{
	Calculer les racines carrées suivantes.
	\begin{multicols}{2}
	\begin{enumerate}
		\item $\sqrt{7^2}$
		\item $\sqrt{17}^2$
		\item $\sqrt{(-9)^2}$
		\item $\sqrt{10^4}$
		\item $\left(-\sqrt{4}\right)^2$
		\item $-\sqrt{15^2}$
	\end{enumerate}
	\end{multicols}
}
{}

\exe{
	Donner un encadrement des nombres à l'unité.
		\begin{multicols}{2}
		\begin{enumerate}
			\item $\sqrt{43}$
			\item $\sqrt{70,8}$
			\item $\sqrt{\dfrac{61}7}$
			\item $\sqrt{101,204}$
		\end{enumerate}
		\end{multicols}
}{}

\begin{theorem*}{Propriétés de la racine carrée}
	Soient $a,b \in\R$ deux réels positifs, $b\neq0$.
	Alors
		\begin{align*}
			\sqrt{a \cdot b} = \sqrt{a} \cdot \sqrt{b}, && \text{ et } && \sqrt{\dfrac{a}{b}} = \dfrac{\sqrt{a}}{\sqrt{b}}.
		\end{align*}
\end{theorem*}
	
\exe{
	Calculer les produits suivants.
	\begin{multicols}{2}
	\begin{enumerate}
		\item $\sqrt{169} \cdot \sqrt{81}$
		\item $\sqrt{169 \cdot 81}$
		\item $\sqrt{0,16} \cdot \sqrt{900}$
		\item $\sqrt{0,16 \cdot 900}$
	\end{enumerate}
	\end{multicols}
}{}

\exe{
	Écrire les nombres suivants sous la forme $a\sqrt{b}$ où $a\in\N$ et $b\in\N$ est le plus petit entier possible.

	\begin{multicols}{2}
	\begin{enumerate}
		\item $\sqrt{12}$
		\item $\sqrt{150}$
		\item $5\sqrt{96}$
		\item $2\sqrt{300}$
		\item $\dfrac{12}{\sqrt{3}}$
		\item $\dfrac{18}{\sqrt{6}}$
	\end{enumerate}
	\end{multicols}
}{}


\exe{
	Écrire les nombres suivants sous la forme $\sqrt{a}$ où $a\in\N$.
	\begin{multicols}{2}
	\begin{enumerate}
		\item $3\sqrt{2}$
		\item $50\sqrt{0,5}$
	\end{enumerate}
	\end{multicols}
}{}

\ifdys
\else
\newpage
\fi

\exe{
	Calculer les fractions suivantes.
	\begin{multicols}{2}
	\begin{enumerate}
		\item $\dfrac{\sqrt{64}}{\sqrt{4}}$
		\item $\sqrt{\dfrac{64}4}$
		\item $\dfrac{\sqrt{0,81}}{\sqrt{0,09}}$
		\item $\sqrt{\dfrac{0,81}{0,09}}$
	\end{enumerate}
	\end{multicols}
}{}

\exe{
	Résoudre les équations suivantes pour $x\in\R$. Exprimer $x$ sous la forme $q \sqrt{b}$ avec $q\in\Q$ rationnel et $b\in\N$ le plus petit possible.
	\begin{multicols}{2}
	\begin{enumerate}
		\item $\dfrac3x = \dfrac12.$
		\item $ \dfrac7x  =\dfrac12$
		\item $\dfrac8x  =\dfrac{\sqrt{3}}2$
		\item $\dfrac2x =\sqrt{3}$
		\item $\dfrac9x  =\dfrac{\sqrt{6}}5$
		\item $-\dfrac3x =\sqrt{7}$
		\item $-\dfrac7{2x}  =\dfrac{\sqrt{11}}3$
		\item $\dfrac3{5x} =-\sqrt{5}$
	\end{enumerate}
	\end{multicols}
}{}

\exe{
	Calculer la longueur \textbf{exacte} de tous les côtés des triangles rectangles suivants.


	\begin{multicols}{2}
	\begin{tikzpicture}[scale=1]
		\draw[-, thick, black] (0,0) -- (3,0) node[midway, below] {$14$};
		\draw[-, thick, black] (3,0) -- (3,2);
		\draw[-, thick, black] (3,2) -- (0,0);
		
		\node[black, left] at  (0,0) {$A$};
		\node[black, below] at  (3,0) {$B$};
		\node[black, above] at  (3,2) {$C$};
		% angle droit
		\draw[-, thick, black] (3,.3)-- (2.7,.3);
		\draw[-, thick, black] (2.7,.3)-- (2.7,0);
		
		% angle
		\draw [black,thick,domain=0:35] plot ({.5*cos(\x)}, {.5*sin(\x)})
		node[left=5pt, right=10pt] {$30$°};
	\end{tikzpicture}
	
	
	\begin{tikzpicture}[scale=1]
		\draw[-, thick, black] (0,0) -- (3,0) node[midway, below] {$13$};
		\draw[-, thick, black] (3,0) -- (3,2);
		\draw[-, thick, black] (3,2) -- (0,0);
		
		\node[black, left] at  (0,0) {$A$};
		\node[black, below] at  (3,0) {$B$};
		\node[black, above] at  (3,2) {$C$};
		% angle droit
		\draw[-, thick, black] (3,.3)-- (2.7,.3);
		\draw[-, thick, black] (2.7,.3)-- (2.7,0);
		
		% angle
		\draw [black,thick,domain=215:270] plot ({3+.5*cos(\x)}, {2+.5*sin(\x)})
		node[below=5pt, left=-1pt] {$30$°};
	\end{tikzpicture}
	

	\begin{tikzpicture}[scale=1]
		\draw[-, thick, black] (0,0) -- (3,0);
		\draw[-, thick, black] (3,0) -- (3,2) node[midway, right] {$2\sqrt{6}$} ;
		\draw[-, thick, black] (3,2) -- (0,0);
		
		\node[black, left] at  (0,0) {$A$};
		\node[black, below] at  (3,0) {$B$};
		\node[black, above] at  (3,2) {$C$};
		% angle droit
		\draw[-, thick, black] (3,.3)-- (2.7,.3);
		\draw[-, thick, black] (2.7,.3)-- (2.7,0);
		
		% angle
		\draw [black,thick,domain=215:270] plot ({3+.5*cos(\x)}, {2+.5*sin(\x)})
		node[below=5pt, left=-1pt] {$60$°};
	\end{tikzpicture}



	\begin{tikzpicture}[scale=1]
		\draw[-, thick, black] (0,0) -- (3,0);
		\draw[-, thick, black] (3,0) -- (3,2)  node[midway, right] {$6\sqrt{15}$} ;
		\draw[-, thick, black] (3,2) -- (0,0);
		
		\node[black, left] at  (0,0) {$A$};
		\node[black, below] at  (3,0) {$B$};
		\node[black, above] at  (3,2) {$C$};
		% angle droit
		\draw[-, thick, black] (3,.3)-- (2.7,.3);
		\draw[-, thick, black] (2.7,.3)-- (2.7,0);
		
		% angle
		\draw [black,thick,domain=0:35] plot ({.5*cos(\x)}, {.5*sin(\x)})
		node[left=5pt, right=10pt] {$60$°};
	\end{tikzpicture}
	\end{multicols}
	\emph{Données : }	
		\[ \sin(30^\circ) = \dfrac12, \qquad \cos(30^\circ) = \dfrac{\sqrt{3}}2, \qquad \tan(30^\circ) = \dfrac1{\sqrt{3}}. \]
	
		\[ \cos(60^\circ) = \dfrac12, \qquad \sin(60^\circ) = \dfrac{\sqrt{3}}2, \qquad \tan(60^\circ) =\sqrt{3}.\]
}{}

\end{document}

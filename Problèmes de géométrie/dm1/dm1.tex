%% INPUT PREAMBLE.TEX
%% THEN INPUT VARS_{i}.ADR
%% THEN RUN THIS
%% DYSLEXIA SWITCH
\newif\ifdys
		
				% ENABLE or DISABLE font change
				% use XeLaTeX if true
				\dystrue
				\dysfalse


\ifdys

\documentclass[a4paper, 14pt]{extarticle}
\usepackage{amsmath,amsfonts,amsthm,amssymb,mathtools}

\tracinglostchars=3 % Report an error if a font does not have a symbol.
\usepackage{fontspec}
\usepackage{unicode-math}
\defaultfontfeatures{ Ligatures=TeX,
                      Scale=MatchUppercase }

\setmainfont{OpenDyslexic}[Scale=1.0]
\setmathfont{Fira Math} % Or maybe try KPMath-Sans?
\setmathfont{OpenDyslexic Italic}[range=it/{Latin,latin}]
\setmathfont{OpenDyslexic}[range=up/{Latin,latin,num}]

\else

\documentclass[a4paper, 12pt]{extarticle}

\usepackage[utf8x]{inputenc}
%fonts
\usepackage{amsmath,amsfonts,amsthm,amssymb,mathtools}
% comment below to default to computer modern
\usepackage{libertinus,libertinust1math}

\fi


\usepackage[french]{babel}
\usepackage[
a4paper,
margin=2cm,
nomarginpar,% We don't want any margin paragraphs
]{geometry}
\usepackage{icomma}

\usepackage{fancyhdr}
\usepackage{array}
\usepackage{hyperref}

\usepackage{multicol, enumerate}
\newcolumntype{P}[1]{>{\centering\arraybackslash}p{#1}}


\usepackage{stackengine}
\newcommand\xrowht[2][0]{\addstackgap[.5\dimexpr#2\relax]{\vphantom{#1}}}

% theorems

\theoremstyle{plain}
\newtheorem{theorem}{Th\'eor\`eme}
\newtheorem*{sol}{Solution}
\theoremstyle{definition}
\newtheorem{ex}{Exercice}
\newtheorem*{rpl}{Rappel}
\newtheorem{enigme}{Énigme}

% corps
\usepackage{calrsfs}
\newcommand{\C}{\mathcal{C}}
\newcommand{\R}{\mathbb{R}}
\newcommand{\Rnn}{\mathbb{R}^{2n}}
\newcommand{\Z}{\mathbb{Z}}
\newcommand{\N}{\mathbb{N}}
\newcommand{\Q}{\mathbb{Q}}

% variance
\newcommand{\Var}[1]{\text{Var}(#1)}

% domain
\newcommand{\D}{\mathcal{D}}


% date
\usepackage{advdate}
\AdvanceDate[0]


% plots
\usepackage{pgfplots}

% table line break
\usepackage{makecell}
%tablestuff
\def\arraystretch{2}
\setlength\tabcolsep{15pt}

%subfigures
\usepackage{subcaption}

\definecolor{myg}{RGB}{56, 140, 70}
\definecolor{myb}{RGB}{45, 111, 177}
\definecolor{myr}{RGB}{199, 68, 64}

% fake sections with no title to move around the merged pdf
\newcommand{\fakesection}[1]{%
  \par\refstepcounter{section}% Increase section counter
  \sectionmark{#1}% Add section mark (header)
  \addcontentsline{toc}{section}{\protect\numberline{\thesection}#1}% Add section to ToC
  % Add more content here, if needed.
}


% SOLUTION SWITCH
\newif\ifsolutions
				\solutionstrue
				%\solutionsfalse

\ifsolutions
	\newcommand{\exe}[2]{
		\begin{ex} #1  \end{ex}
		\begin{sol} #2 \end{sol}
	}
\else
	\newcommand{\exe}[2]{
		\begin{ex} #1  \end{ex}
	}
	
\fi


% tableaux var, signe
\usepackage{tkz-tab}


%pinfty minfty
\newcommand{\pinfty}{{+}\infty}
\newcommand{\minfty}{{-}\infty}

\begin{document}
\input{adr/vars_12345.adr}

\pagestyle{fancy}
\fancyhead[L]{Seconde 13}
\fancyhead[C]{\textbf{Devoir Maison 2 -- \seed \ifsolutions \, -- Solutions  \fi}}
\fancyhead[R]{\today}

Dans un repère d'origine $O$, on considère les trois points suivants, dépendant d'un nombre réel $x \in [0;\xmax]$.
	\begin{align*}
		A  = x \cdot (\xA;\yA) && B = (\xmax-x)\cdot(\xB ; \yB) && P = \LAMBDA \cdot (\xmax-x)\cdot (\xA;\yA).
	\end{align*}

On admettra que les points $O, P$, et $A$ sont alignés (ils sont multiples d'un même point).
	
	
	
%\begin{figure}[h!]
%
%	\begin{center}
%	\begin{tikzpicture}[>=stealth, scale=.8]
%		\begin{axis}[xmin = -9, xmax=18, ymin=0, ymax=45, axis x line=middle, axis y line=middle, axis line style=-]
%			\addplot[black, mark=*, mark size = 1, thick] (8,12) node[above] {$A$};
%			\addplot[black, mark=*, mark size = 1, thick] (0,0) node[above] {$O$};
%			\addplot[black, mark=*, mark size = 1, thick] (-6,30) node[above] {$B$};
%			\addplot[black, mark=*, mark size = 1, thick] (12,18) node[above] {$P$};
%		\end{axis}
%	\end{tikzpicture}
%	\end{center}
%	\caption{aaa}
%	
%\end{figure}
	
\exe{
	Donner les coordonnées des points $A, B,$ et $P$ lorsque $x=\xfirst$ et lorsque $x=\xsecond$.
	
	Tracer ces points dans deux repères qui contiennent l'origine $O$.
}{}
\exe{
	 À l'aide de la formule de la longueur de segment vue en cours, montrer que
	 	\begin{align*}
	 		OB^2 = \Bnormsq\cdot |\xmax-x|^2, && \text{ et } && OP^2 = \Pnormsq\cdot |\xmax-x|^2.
		\end{align*}
}{}
\exe{
	Similairement, montrer que
	 	\[ BP^2 = \BPnormsq \cdot |\xmax-x|^2.\] 
}{}
\exe{
	 Démontrer que le triangle $OBP$ est rectangle en $P$ à l'aide de la réciproque du théorème de Pythagore.
}{}
\exe{
	 En déduire que $P$ est le projeté orthogonal de $B$ sur $(OA)$ et donc que $BP$ est la hauteur du triangle $OAB$ de base $OA$.
}{}
\exe{
	 Démontrer que l'aire $\mathcal{A}(x)$ du triangle $OAB$ est donnée par, pour $x\in[0;\xmax]$,
		\[ \mathcal{A}(x) = \prodovertwo \cdot |\xmax-x| \cdot |x| =  \prodovertwo(\xmax-x)x. \]
}{}
\exe{
	 Esquisser la courbe $\C_\mathcal{A}$ de l'aire $\mathcal{A}$, fonction réelle sur $[0;\xmax]$.
}{}
\exe{
	 À l'aide de $\C_\mathcal{A}$, estimer la valeur du $x^\star \in [0;\xmax]$ telle que $A(x^\star)$ soit maximale.
}{}
\exe{
	 Montrer que, pour tout $x\in[0;\xmax]$, on a l'identité
		\[ \mathcal{A}(x) =  \BETA - \prodovertwo\left(x-\ALPHA\right)^2. \]
}{}
\exe{
	 En déduire que, pour tout $x\in[0;\xmax]$,
		\[ \mathcal{A}(x) \leq  \BETA , \]
	et donc que $\mathcal{A}$ atteint son maximum en $x^\star=\ALPHA$.
}{}


\end{document}

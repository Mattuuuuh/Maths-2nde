%% INPUT PREAMBLE.TEX
%% THEN INPUT VARS_{i}.ADR
%% THEN RUN THIS
%%!TEX encoding = UTF8
%!TEX root =notes.tex


%%%%%%%%%%%%%%%%%%%%%%%%%%%%%%%%%
% PACKAGE IMPORTS
%%%%%%%%%%%%%%%%%%%%%%%%%%%%%%%%%


\usepackage[french]{babel}

\usepackage[tmargin=2cm,rmargin=1in,lmargin=1in,margin=0.85in,bmargin=2cm,footskip=.2in]{geometry}
\usepackage{amsmath,amsfonts,amsthm,amssymb,mathtools}
\usepackage[varbb]{newpxmath}
\usepackage{xfrac}
\usepackage[makeroom]{cancel}
\usepackage{mathtools}
\usepackage{bookmark}
\usepackage{enumitem}
\usepackage{hyperref,theoremref}
\hypersetup{
	pdftitle={Assignment},
	colorlinks=true, linkcolor=doc!90,
	bookmarksnumbered=true,
	bookmarksopen=true
}
\usepackage[most,many,breakable]{tcolorbox}
\usepackage{xcolor}
\usepackage{varwidth}
\usepackage{varwidth}
\usepackage{etoolbox}
%\usepackage{authblk}
\usepackage{nameref}
\usepackage{multicol,array}
\usepackage{tikz-cd}
\usepackage[ruled,vlined,linesnumbered]{algorithm2e}
\usepackage{comment} % enables the use of multi-line comments (\ifx \fi) 
\usepackage{import}
\usepackage{xifthen}
\usepackage{pdfpages}
\usepackage{transparent}


\newcommand\mycommfont[1]{\footnotesize\ttfamily\textcolor{blue}{#1}}
\SetCommentSty{mycommfont}
\newcommand{\incfig}[1]{%
    \def\svgwidth{\columnwidth}
    \import{./figures/}{#1.pdf_tex}
}

\usepackage{tikzsymbols}
%\renewcommand\qedsymbol{$\Laughey$}


%\usepackage{import}
%\usepackage{xifthen}
%\usepackage{pdfpages}
%\usepackage{transparent}


%%%%%%%%%%%%%%%%%%%%%%%%%%%%%%
% SELF MADE COLORS
%%%%%%%%%%%%%%%%%%%%%%%%%%%%%%



\definecolor{myg}{RGB}{56, 140, 70}
\definecolor{myb}{RGB}{45, 111, 177}
\definecolor{myr}{RGB}{199, 68, 64}
\definecolor{mytheorembg}{HTML}{F2F2F9}
\definecolor{mytheoremfr}{HTML}{00007B}
\definecolor{mylenmabg}{HTML}{FFFAF8}
\definecolor{mylenmafr}{HTML}{983b0f}
\definecolor{mypropbg}{HTML}{f2fbfc}
\definecolor{mypropfr}{HTML}{191971}
\definecolor{myexamplebg}{HTML}{F2FBF8}
\definecolor{myexamplefr}{HTML}{88D6D1}
\definecolor{myexampleti}{HTML}{2A7F7F}
\definecolor{mydefinitbg}{HTML}{E5E5FF}
\definecolor{mydefinitfr}{HTML}{3F3FA3}
\definecolor{notesgreen}{RGB}{0,162,0}
\definecolor{myp}{RGB}{197, 92, 212}
\definecolor{mygr}{HTML}{2C3338}
\definecolor{myred}{RGB}{127,0,0}
\definecolor{myyellow}{RGB}{169,121,69}
\definecolor{myexercisebg}{HTML}{F2FBF8}
\definecolor{myexercisefg}{HTML}{88D6D1}


%%%%%%%%%%%%%%%%%%%%%%%%%%%%
% TCOLORBOX SETUPS
%%%%%%%%%%%%%%%%%%%%%%%%%%%%

\setlength{\parindent}{1cm}
%================================
% THEOREM BOX
%================================

\tcbuselibrary{theorems,skins,hooks}
\newtcbtheorem[number within=chapter]{Theorem}{Théorème}
{%
	enhanced,
	breakable,
	colback = mytheorembg,
	frame hidden,
	boxrule = 0sp,
	borderline west = {2pt}{0pt}{mytheoremfr},
	sharp corners,
	detach title,
	before upper = \tcbtitle\par\smallskip,
	coltitle = mytheoremfr,
	fonttitle = \bfseries\sffamily,
	description font = \mdseries,
	separator sign none,
	segmentation style={solid, mytheoremfr},
}
{th}


\tcbuselibrary{theorems,skins,hooks}
\newtcolorbox{Theoremcon}
{%
	enhanced
	,breakable
	,colback = mytheorembg
	,frame hidden
	,boxrule = 0sp
	,borderline west = {2pt}{0pt}{mytheoremfr}
	,sharp corners
	,description font = \mdseries
	,separator sign none
}

%================================
% Corollery
%================================
\tcbuselibrary{theorems,skins,hooks}
\newtcbtheorem[use counter=tcb@cnt@Theorem]{Corollary}{Corollaire}
{%
	enhanced
	,breakable
	,colback = myp!10
	,frame hidden
	,boxrule = 0sp
	,borderline west = {2pt}{0pt}{myp!85!black}
	,sharp corners
	,detach title
	,before upper = \tcbtitle\par\smallskip
	,coltitle = myp!85!black
	,fonttitle = \bfseries\sffamily
	,description font = \mdseries
	,separator sign none
	,segmentation style={solid, myp!85!black}
}
{th}

%================================
% LENMA
%================================

\tcbuselibrary{theorems,skins,hooks}
\newtcbtheorem[use counter=tcb@cnt@Theorem]{Lemma}{Lemme}
{%
	enhanced,
	breakable,
	colback = mylenmabg,
	frame hidden,
	boxrule = 0sp,
	borderline west = {2pt}{0pt}{mylenmafr},
	sharp corners,
	detach title,
	before upper = \tcbtitle\par\smallskip,
	coltitle = mylenmafr,
	fonttitle = \bfseries\sffamily,
	description font = \mdseries,
	separator sign none,
	segmentation style={solid, mylenmafr},
}
{th}


%================================
% PROPOSITION
%================================

\tcbuselibrary{theorems,skins,hooks}
\newtcbtheorem[use counter=tcb@cnt@Theorem]{Prop}{Proposition}
{%
	enhanced,
	breakable,
	colback = mypropbg,
	frame hidden,
	boxrule = 0sp,
	borderline west = {2pt}{0pt}{mypropfr},
	sharp corners,
	detach title,
	before upper = \tcbtitle\par\smallskip,
	coltitle = mypropfr,
	fonttitle = \bfseries\sffamily,
	description font = \mdseries,
	separator sign none,
	segmentation style={solid, mypropfr},
}
{th}


%================================
% CLAIM
%================================

\tcbuselibrary{theorems,skins,hooks}
\newtcbtheorem[use counter=tcb@cnt@Theorem]{claim}{Claim}
{%
	enhanced
	,breakable
	,colback = myg!10
	,frame hidden
	,boxrule = 0sp
	,borderline west = {2pt}{0pt}{myg}
	,sharp corners
	,detach title
	,before upper = \tcbtitle\par\smallskip
	,coltitle = myg!85!black
	,fonttitle = \bfseries\sffamily
	,description font = \mdseries
	,separator sign none
	,segmentation style={solid, myg!85!black}
}
{th}



%================================
% Exercise
%================================

\tcbuselibrary{theorems,skins,hooks}
\newtcbtheorem[use counter=tcb@cnt@Theorem]{Exercise}{Exercice}
{%
	enhanced,
	breakable,
	colback = myexercisebg,
	frame hidden,
	boxrule = 0sp,
	borderline west = {2pt}{0pt}{myexercisefg},
	sharp corners,
	detach title,
	before upper = \tcbtitle\par\smallskip,
	coltitle = myexercisefg,
	fonttitle = \bfseries\sffamily,
	description font = \mdseries,
	separator sign none,
	segmentation style={solid, myexercisefg},
}
{th}

%================================
% EXAMPLE BOX
%================================

\newtcbtheorem[use counter=tcb@cnt@Theorem]{Example}{Exemple}
{%
	colback = myexamplebg
	,breakable
	,colframe = myexamplefr
	,coltitle = myexampleti
	,boxrule = 1pt
	,sharp corners
	,detach title
	,before upper=\tcbtitle\par\smallskip
	,fonttitle = \bfseries
	,description font = \mdseries
	,separator sign none
	,description delimiters parenthesis
}
{ex}

%================================
% DEFINITION BOX
%================================

\newtcbtheorem[use counter=tcb@cnt@Theorem]{Definition}{Définition}{enhanced,
	before skip=2mm,after skip=2mm, colback=red!5,colframe=red!80!black,boxrule=0.5mm,
	attach boxed title to top left={xshift=1cm,yshift*=1mm-\tcboxedtitleheight}, varwidth boxed title*=-3cm,
	boxed title style={frame code={
					\path[fill=tcbcolback]
					([yshift=-1mm,xshift=-1mm]frame.north west)
					arc[start angle=0,end angle=180,radius=1mm]
					([yshift=-1mm,xshift=1mm]frame.north east)
					arc[start angle=180,end angle=0,radius=1mm];
					\path[left color=tcbcolback!60!black,right color=tcbcolback!60!black,
						middle color=tcbcolback!80!black]
					([xshift=-2mm]frame.north west) -- ([xshift=2mm]frame.north east)
					[rounded corners=1mm]-- ([xshift=1mm,yshift=-1mm]frame.north east)
					-- (frame.south east) -- (frame.south west)
					-- ([xshift=-1mm,yshift=-1mm]frame.north west)
					[sharp corners]-- cycle;
				},interior engine=empty,
		},
	fonttitle=\bfseries,
	title={#2},#1}{def}

%================================
% Solution BOX
%================================

\makeatletter
\newtcbtheorem[use counter=tcb@cnt@Theorem]{question}{Question}{enhanced,
	breakable,
	colback=white,
	colframe=myb!80!black,
	attach boxed title to top left={yshift*=-\tcboxedtitleheight},
	fonttitle=\bfseries,
	title={#2},
	boxed title size=title,
	boxed title style={%
			sharp corners,
			rounded corners=northwest,
			colback=tcbcolframe,
			boxrule=0pt,
		},
	underlay boxed title={%
			\path[fill=tcbcolframe] (title.south west)--(title.south east)
			to[out=0, in=180] ([xshift=5mm]title.east)--
			(title.center-|frame.east)
			[rounded corners=\kvtcb@arc] |-
			(frame.north) -| cycle;
		},
	#1
}{def}
\makeatother

%================================
% SOLUTION BOX
%================================

\makeatletter
\newtcolorbox{solution}{enhanced,
	breakable,
	colback=white,
	colframe=myg!80!black,
	attach boxed title to top left={yshift*=-\tcboxedtitleheight},
	title=Solution,
	boxed title size=title,
	boxed title style={%
			sharp corners,
			rounded corners=northwest,
			colback=tcbcolframe,
			boxrule=0pt,
		},
	underlay boxed title={%
			\path[fill=tcbcolframe] (title.south west)--(title.south east)
			to[out=0, in=180] ([xshift=5mm]title.east)--
			(title.center-|frame.east)
			[rounded corners=\kvtcb@arc] |-
			(frame.north) -| cycle;
		},
}
\makeatother

%================================
% Question BOX
%================================

\makeatletter
\newtcbtheorem[use counter=tcb@cnt@Theorem]{qstion}{Question}{enhanced,
	breakable,
	colback=white,
	colframe=mygr,
	attach boxed title to top left={yshift*=-\tcboxedtitleheight},
	fonttitle=\bfseries,
	title={#2},
	boxed title size=title,
	boxed title style={%
			sharp corners,
			rounded corners=northwest,
			colback=tcbcolframe,
			boxrule=0pt,
		},
	underlay boxed title={%
			\path[fill=tcbcolframe] (title.south west)--(title.south east)
			to[out=0, in=180] ([xshift=5mm]title.east)--
			(title.center-|frame.east)
			[rounded corners=\kvtcb@arc] |-
			(frame.north) -| cycle;
		},
	#1
}{def}
\makeatother

\newtcbtheorem[number within=chapter]{wconc}{Wrong Concept}{
	breakable,
	enhanced,
	colback=white,
	colframe=myr,
	arc=0pt,
	outer arc=0pt,
	fonttitle=\bfseries\sffamily\large,
	colbacktitle=myr,
	attach boxed title to top left={},
	boxed title style={
			enhanced,
			skin=enhancedfirst jigsaw,
			arc=3pt,
			bottom=0pt,
			interior style={fill=myr}
		},
	#1
}{def}



%================================
% NOTE BOX
%================================

\usetikzlibrary{arrows,calc,shadows.blur}
\tcbuselibrary{skins}
\newtcolorbox{note}[1][]{%
	enhanced jigsaw,
	colback=gray!20!white,%
	colframe=gray!80!black,
	size=small,
	boxrule=1pt,
	title=\colorbox{white!100}{\textbf{ Remarque }},
	halign title=flush center,
	coltitle=black,
	breakable,
	drop shadow=black!50!white,
	attach boxed title to top left={xshift=1cm,yshift=-\tcboxedtitleheight/2,yshifttext=-\tcboxedtitleheight/2},
	minipage boxed title=2.6cm,
	boxed title style={%
			colback=white,
			size=fbox,
			boxrule=1pt,
			boxsep=2pt,
			underlay={%
					\coordinate (dotA) at ($(interior.west) + (-0.5pt,0)$);
					\coordinate (dotB) at ($(interior.east) + (0.5pt,0)$);
					\begin{scope}
						\clip (interior.north west) rectangle ([xshift=3ex]interior.east);
						\filldraw [white, blur shadow={shadow opacity=60, shadow yshift=-.75ex}, rounded corners=2pt] (interior.north west) rectangle (interior.south east);
					\end{scope}
					\begin{scope}[gray!80!black]
						\fill (dotA) circle (2pt);
						\fill (dotB) circle (2pt);
					\end{scope}
				},
		},
	#1,
}

%================================
% STRATÉGIE BOX
%================================

\usetikzlibrary{arrows,calc,shadows.blur}
\tcbuselibrary{skins}
\newtcolorbox{strategy}[1][]{%
	enhanced jigsaw,
	colback=myb!20!white,%
	colframe=gray!80!black,
	size=small,
	boxrule=1pt,
	title=\colorbox{white!100}{\textbf{ Stratégie }},
	halign title=flush center,
	coltitle=black,
	breakable,
	drop shadow=black!50!white,
	attach boxed title to top left={xshift=1cm,yshift=-\tcboxedtitleheight/2,yshifttext=-\tcboxedtitleheight/2},
	minipage boxed title=2.5cm,
	boxed title style={%
			colback=white,
			size=fbox,
			boxrule=1pt,
			boxsep=2pt,
			underlay={%
					\coordinate (dotA) at ($(interior.west) + (-0.5pt,0)$);
					\coordinate (dotB) at ($(interior.east) + (0.5pt,0)$);
					\begin{scope}
						\clip (interior.north west) rectangle ([xshift=3ex]interior.east);
						\filldraw [white, blur shadow={shadow opacity=60, shadow yshift=-.75ex}, rounded corners=2pt] (interior.north west) rectangle (interior.south east);
					\end{scope}
					\begin{scope}[gray!80!black]
						\fill (dotA) circle (2pt);
						\fill (dotB) circle (2pt);
					\end{scope}
				},
		},
	#1,
}

%================================
% MÉTHODE BOX
%================================

\usetikzlibrary{arrows,calc,shadows.blur}
\tcbuselibrary{skins}
\newtcolorbox{methode}[1][]{%
	enhanced jigsaw,
	colback=white,%
	colframe=gray!80!black,
	size=small,
	boxrule=1pt,
	title=\textbf{Méthode},
	halign title=flush center,
	coltitle=black,
	breakable,
	drop shadow=black!50!white,
	attach boxed title to top left={xshift=1cm,yshift=-\tcboxedtitleheight/2,yshifttext=-\tcboxedtitleheight/2},
	minipage boxed title=2.5cm,
	boxed title style={%
			colback=white,
			size=fbox,
			boxrule=1pt,
			boxsep=2pt,
			underlay={%
					\coordinate (dotA) at ($(interior.west) + (-0.5pt,0)$);
					\coordinate (dotB) at ($(interior.east) + (0.5pt,0)$);
					\begin{scope}
						\clip (interior.north west) rectangle ([xshift=3ex]interior.east);
						\filldraw [white, blur shadow={shadow opacity=60, shadow yshift=-.75ex}, rounded corners=2pt] (interior.north west) rectangle (interior.south east);
					\end{scope}
					\begin{scope}[gray!80!black]
						\fill (dotA) circle (2pt);
						\fill (dotB) circle (2pt);
					\end{scope}
				},
		},
	#1,
}

%%%%%%%%%%%%%%%%%%%%%%%%%%%%%%%%%%%%%%%%%%%
% TABLE OF CONTENTS
%%%%%%%%%%%%%%%%%%%%%%%%%%%%%%%%%%%%%%%%%%%

\usepackage{tikz}

\definecolor{doc}{RGB}{0,60,110}
\usepackage{titletoc}
\contentsmargin{0cm}
\titlecontents{chapter}[3.7pc]
{\addvspace{30pt}%
	\begin{tikzpicture}[remember picture, overlay]%
		\draw[fill=doc!60,draw=doc!60] (-7,-.1) rectangle (-0.2,.6);%
		\pgftext[left,x=-3.5cm,y=0.2cm]{\color{white}\Large\sc\bfseries Chapitre\ \thecontentslabel};%
	\end{tikzpicture}\color{doc!60}\large\sc\bfseries}%
{}
{}
{\;\titlerule\;\large\sc\bfseries Page \thecontentspage
	\begin{tikzpicture}[remember picture, overlay]
		\draw[fill=doc!60,draw=doc!60] (2pt,0) rectangle (4,0.1pt);
	\end{tikzpicture}}%
\titlecontents{section}[3.7pc]
{\addvspace{2pt}}
{\contentslabel[\thecontentslabel]{2pc}}
{}
{\hfill\small \thecontentspage}
[]
\titlecontents*{subsection}[3.7pc]
{\addvspace{-1pt}\small}
{}
{}
{\ --- \small\thecontentspage}
[ \textbullet\ ][]

\makeatletter
\renewcommand{\tableofcontents}{%
	\chapter*{%
	  \vspace*{-20\p@}%
	  \begin{tikzpicture}[remember picture, overlay]%
		  \pgftext[right,x=15cm,y=0.2cm]{\color{doc!60}\Huge\sc\bfseries \contentsname};%
		  \draw[fill=doc!60,draw=doc!60] (13,-.75) rectangle (20,1);%
		  \clip (13,-.75) rectangle (20,1);
		  \pgftext[right,x=15cm,y=0.2cm]{\color{white}\Huge\sc\bfseries \contentsname};%
	  \end{tikzpicture}}%
	\@starttoc{toc}}
\makeatother


%%%%%%%%%%%%%%%%%%%%%%%%%%%%%%%%%%%%%%%%%%%
% MINTED FOR PYTHON ALGORITHMS
%%%%%%%%%%%%%%%%%%%%%%%%%%%%%%%%%%%%%%%%%%%

\usepackage{tcolorbox}
\tcbuselibrary{minted,breakable,xparse,skins}
\definecolor{bg}{gray}{0.95}
\DeclareTCBListing{mintedbox}{O{}m!O{}}{%
  breakable=true,
  listing engine=minted,
  listing only,
  minted language=#2,
  minted style=default,
  minted options={%
    linenos,
    gobble=0,
    breaklines=true,
    breakafter=,,
    fontsize=\small,
    numbersep=8pt,
    #1},
  boxsep=0pt,
  left skip=0pt,
  right skip=0pt,
  left=25pt,
  right=0pt,
  top=3pt,
  bottom=3pt,
  arc=5pt,
  leftrule=0pt,
  rightrule=0pt,
  bottomrule=2pt,
  toprule=2pt,
  colback=bg,
  colframe=orange!70,
  enhanced,
  overlay={%
    \begin{tcbclipinterior}
    \fill[orange!20!white] (frame.south west) rectangle ([xshift=20pt]frame.north west);
    \end{tcbclipinterior}},
  #3}
  
  
 % for braces
\usetikzlibrary{decorations.pathreplacing}
\input{adr/vars_12345.adr}

\pagestyle{fancy}
\fancyhead[L]{Seconde 13}
\fancyhead[C]{\textbf{Devoir Maison 2 -- \seed \ifsolutions \, -- Solutions  \fi}}
\fancyhead[R]{\today}

Dans un repère d'origine $O$, on considère les trois points suivants, dépendant d'un nombre réel $x \in [0;\xmax]$.
	\begin{align*}
		A  = x \cdot (\xA;\yA) && B = (\xmax-x)\cdot(\xB ; \yB) && P = \LAMBDA \cdot (\xmax-x)\cdot (\xA;\yA).
	\end{align*}

On admettra que les points $O, P$, et $A$ sont alignés (ils sont multiples d'un même point).
	
	
\exe{
	Donner les coordonnées des points $A, B,$ et $P$ lorsque $x=\xfirst$ et lorsque $x=\xsecond$.
	
	Tracer ces points dans deux repères qui contiennent l'origine $O$.
}{
	On trace le triangle $OBA$ et le projeté orthogonal $P$ du sommet $B$ sur $(OA)$ lorsque $x=\xfirst$ puis $x=\xsecond$.

	\begin{figure}[h!]
		\begin{center}
		\begin{tikzpicture}[>=stealth, scale=1]
			\begin{axis}[xmin = \xlow, xmax=\xhigh, ymin=\ylow, ymax=\yhigh, axis x line=middle, axis y line=middle, axis line style=-]
				\addplot[black, mark=*, mark size = 1, thick] (\xAfirst, \yAfirst) node[above] {$A$};
				\addplot[black, mark=*, mark size = 1, thick] (0,0) node[above right] {$O$};
				\addplot[black, mark=*, mark size = 1, thick] (\xBfirst, \yBfirst) node[above] {$B$};
				\addplot[black, mark=*, mark size = 1, thick] (\xPfirst, \yPfirst) node[above] {$P$};
				
				% triangle OBA
				\draw (axis cs:0,0) -- (axis cs:\xBfirst,\yBfirst);
				\draw (axis cs:\xAfirst, \yAfirst) -- (axis cs:\xBfirst,\yBfirst);
				\draw (axis cs:0,0) -- (axis cs:\xAfirst, \yAfirst);
				
				% height P projected onto (OA)
				\draw[dotted] (axis cs:0,0) -- (axis cs:\xPfirst, \yPfirst);
				\draw[dotted] (axis cs:\xBfirst, \yBfirst) -- (axis cs:\xPfirst, \yPfirst);
			\end{axis}
		\end{tikzpicture}
		\end{center}
		\caption{Lorsque $x=\xfirst$.}
	\end{figure}
	\begin{figure}[h!]
	\begin{center}
	\begin{tikzpicture}[>=stealth, scale=1]
		\begin{axis}[xmin = \xlow, xmax=\xhigh, ymin=\ylow, ymax=\yhigh, axis x line=middle, axis y line=middle, axis line style=-]
			\addplot[black, mark=*, mark size = 1, thick] (\xAsecond, \yAsecond) node[above] {$A$};
			\addplot[black, mark=*, mark size = 1, thick] (0,0) node[above right] {$O$};
			\addplot[black, mark=*, mark size = 1, thick] (\xBsecond, \yBsecond) node[above] {$B$};
			\addplot[black, mark=*, mark size = 1, thick] (\xPsecond, \yPsecond) node[above] {$P$};
			
			% triangle OBA
			\draw (axis cs:0,0) -- (axis cs:\xBsecond,\yBsecond);
			\draw (axis cs:\xAsecond, \yAsecond) -- (axis cs:\xBsecond,\yBsecond);
			\draw (axis cs:0,0) -- (axis cs:\xAsecond, \yAsecond);
			
			% height P projected onto (OA)
			\draw[dotted] (axis cs:0,0) -- (axis cs:\xPsecond, \yPsecond);
			\draw[dotted] (axis cs:\xBsecond, \yBsecond) -- (axis cs:\xPsecond, \yPsecond);
		\end{axis}
	\end{tikzpicture}
	\end{center}
	\caption{Lorsque $x=\xsecond$.}
	\end{figure}
	
}

\exe{
	 À l'aide de la formule de la longueur de segment vue en cours, montrer que
	 	\begin{align*}
	 		OB^2 = \Bnormsq\cdot |\xmax-x|^2, && \text{ et } && OP^2 = \Pnormsq\cdot |\xmax-x|^2.
		\end{align*}
}{
	En général, pour connaître la longueur $AB^2$, le cours nous donne la formule
		\[ AB^2 = |x_A - x_B|^2 + |y_A - y_B|^2. \]
	Remarquons que les valeurs absolues peuvent disparaître lorsque mises au carré : $|x|^2 = x^2$ pour tout $x\in\R$ réel.
	De plus en général, on a $(ab)^2 = a^2 b^2$, ce qui permet de distribuer le carré.
	
	En particulier lorsqu'un des point est nul, on somme simplement le carré des coordonées du deuxième point.
	Ainsi,
		\begin{align*}
			OB^2 &= x_B^2 + y_B^2 \\
					&= \left[ (\xmax-x)\cdot(\xB) \right]^2 + \left[ (\xmax-x)\cdot(\yB) \right]^2 \\
					&= (\xmax-x)^2 \cdot \left[ (\xB)^2 + (\yB)^2 \right] \\
					&= \Bnormsq\cdot |\xmax-x|^2,
		\end{align*}
	comme souhaité.
	
	Idem pour le calcul de $OP$.
		\begin{align*}
			OP^2 &= x_P^2 + y_P^2 \\
					&= \left[ \LAMBDA \cdot (\xmax-x)\cdot(\xA) \right]^2 + \left[ \LAMBDA \cdot (\xmax-x)\cdot(\yA) \right]^2 \\
					&= \left( \LAMBDA \right)^2 \cdot (\xmax-x)^2 \cdot \left[(\xA)^2 + (\yA)^2\right] \\
					&= \Pnormsq\cdot |\xmax-x|^2
		\end{align*}

}
\exe{
	Similairement, montrer que
	 	\[ BP^2 = \BPnormsq \cdot |\xmax-x|^2.\] 
}{
	On réutilise la formule de la distance en remarquant que $|\xmax-x|^2$ se factorise à nouveau.
	
	\begin{align*}
	%B = (\xmax-x)\cdot(\xB ; \yB) && P = \LAMBDA \cdot (\xmax-x)\cdot (\xA;\yA).
		BP^2 &= (x_B - x_P)^2 + (y_B - y_P)^2 \\
				&= \left[  (\xmax-x)\cdot(\xB) - \LAMBDA \cdot (\xmax-x)\cdot (\xA) \right]^2 + 
				\left[  (\xmax-x)\cdot(\yB) - \LAMBDA \cdot (\xmax-x)\cdot (\yA) \right]^2 \\
				&= (\xmax-x)^2 \cdot \left(\xB - \LAMBDA \cdot (\xA)\right)^2 
				+ (\xmax-x)^2 \cdot \left(\yB - \LAMBDA \cdot (\yA)\right)^2 \\
				&= (\xmax-x)^2 \cdot \BPnormsq
	\end{align*}
}
\exe{
	 Démontrer que le triangle $OBP$ est rectangle en $P$ à l'aide de la réciproque du théorème de Pythagore.
}{
	Il s'agit ici de vérifier que l'égalité
		\[ OB^2 = OP^2 + BP^2 \]
	tient bien.
	Le membre de gauche a été calculé comme étant
		\[ OB^2 = \Bnormsq\cdot |\xmax-x|^2, \]
	et le membre de droite comme étant
		\begin{align*}
			OP^2 + BP^2 &=  \Pnormsq\cdot |\xmax-x|^2 + \BPnormsq \cdot |\xmax-x|^2 \\
							&= \left( \Pnormsq + \BPnormsq \right) \cdot |\xmax-x|^2 \\
							&= \Bnormsq\cdot |\xmax-x|^2
		\end{align*}
	L'égalité est donc bien vérifiée quelque soit $x \in [0; \xmax]$ et le triangle $OBP$ est rectangle en $P$.
}

\exe{
	 En déduire que $P$ est le projeté orthogonal de $B$ sur $(OA)$ et donc que $BP$ est la hauteur du triangle $OAB$ de base $OA$.
}{
	D'après l'énoncé, les points $O, A,$ et $P$ sont alignés.
	Il suit donc que les droites $(OA)$ et $(OP)$ sont confondues et donc que $P$ appartient à la droite $(OA)$.
	
	De plus, le triangle $OBP$ est rectangle en $P$, donc les droites $(BP)$ et $(OP) =(OA)$ sont perpendiculaires.
	Par définition, $P$ est donc le projeté orthogonal de $B$ sur $(OA)$ car c'est le point de $(OA)$ tel que $(OA)$ et $(BP)$ sont perpendiculaires.
	
	On conclut que le triangle $OAB$ admet une base $OA$ et une hauteur $BP$ car $P$ est le projeté orthogonal de $B$ sur $(OA)$.
}

\exe{
	 Démontrer que l'aire $\mathcal{A}(x)$ du triangle $OAB$ est donnée par, pour $x\in[0;\xmax]$,
		\[ \mathcal{A}(x) = \prodovertwo \cdot |\xmax-x| \cdot |x| =  \prodovertwo(\xmax-x)x. \]
}{
	L'aire du triangle est donné par la formule
		\begin{align*}
			\text{Aire} &= \dfrac{\text{Base $\cdot$ Hauteur}}2 \\
						&= \dfrac{OA \cdot BP}2.
		\end{align*}
	Nous avons déjà calculés $BP^2$ et donc $BP$ se déduit en prenant sa racine carrée (en n'oubliant pas la propriété de la racine $\sqrt{a\cdot b} = \sqrt{a} \cdot \sqrt{b}$).
	On pourrait d'ailleurs aussi calculer le carré de l'aire puis prendre une unique racine carrée à la fin.
	
	Comme la longueur $BP$ est positive, on trouve
	\begin{align*}
		BP = |BP| = \sqrt{BP^2} = \sqrt{\BPnormsq \cdot |\xmax-x|^2} = \sqrt{\BPnormsq} \cdot |\xmax-x|.
	\end{align*}
	
	D'autre part,
		\[ OA^2 = x_A^2 + y_A^2 = x^2 \left[ (\xA)^2 + (\yA)^2 \right] =  \Anormsq x^2. \]
	D'où 
		\[ OA = \sqrt{\Anormsq} \cdot |x|, \]
	et donc on conclut que	
	\begin{align*}
		\mathcal{A}(x) &=  \dfrac{OA \cdot BP}2 \\
						&= \dfrac12 \cdot OA \cdot BP \\
						&= \dfrac12 \sqrt{\Anormsq} \cdot |x|  \cdot \sqrt{\BPnormsq} \cdot |\xmax-x| \\
						&= \dfrac12 \sqrt{\Anormsq \cdot \BPnormsq} \cdot |x| \cdot |\xmax-x| \\
						&= \prodovertwo \cdot |\xmax-x| \cdot |x|
	\end{align*}
	Finalement, comme la variable $x$ appartient à l'intervalle $[0; \xmax]$, les expressions à l'intérieur des valeurs absolues sont toujours positives et les barres de valeur absolue peuvent disparaître.
}


\exe{
	 Esquisser la courbe $\C_\mathcal{A}$ de l'aire $\mathcal{A}$, fonction réelle sur $[0;\xmax]$.
}{
	\begin{figure}[h!]
	\begin{center}
	\begin{tikzpicture}[>=stealth, scale=1]
		\begin{axis}[xmin = 0, xmax=\xmax, ymin=0, ymax=\BETAval+10, axis x line=middle, axis y line=middle, axis line style=-]
			\addplot[myb, thick, domain =0:\xmax, samples=50] {\prodovertwoval * (\xmax-x)*x}  node[pos=.5, right=15pt] {$\mathcal{C}_\mathcal{A}$};
		\end{axis}
	\end{tikzpicture}
	\end{center}
	\caption{Courbe représentative de $\mathcal{A}$ sur le domaine $[0; \xmax]$.}
	\end{figure}


}

\exe{
	 À l'aide de $\C_\mathcal{A}$, estimer la valeur du $x^\star \in [0;\xmax]$ telle que $\mathcal{A}(x^\star)$ soit maximale.
}{
	On estime la valeur maximale de $\mathcal{A}$ au pic de la courbe représentative de $\mathcal{A}$, c'est-à-dire lorsque $x^\star \approx \ALPHA$.
}


\exe{
	 Montrer que, pour tout $x\in[0;\xmax]$, on a l'identité
		\[ \mathcal{A}(x) =  \BETA - \prodovertwo\left(x-\ALPHA\right)^2. \]
}{
	On part de l'expression de droite qu'on développe pour arriver à l'expression de $\mathcal{A}(x)$ trouvée à la question 6.
	À l'aide de l'identité remarquable $(a-b)^2 = a^2 + b^2 - 2ab$, on obtient le développement suivant.
	
	\begin{align*}
		&~\quad \BETA - \prodovertwo\left(x-\ALPHA\right)^2 \\
		&= \BETA - \prodovertwo \left[ x^2 - 2 \cdot x \cdot \ALPHA + \left( \ALPHA \right)^2 \right]  \\
		&= \BETA - \prodovertwo \cdot x^2 + \prodovertwo \cdot 2 \cdot \ALPHA \cdot x - \prodovertwo \cdot \left( \ALPHA \right)^2 \\
		&= - \prodovertwo \cdot x^2 + \prodovertwo \cdot \xmax \cdot x,
	\end{align*}
	où on utilise que les constantes s'annulent car $\BETA = \prodovertwo \cdot \left( \ALPHA \right)^2$.

	On factorise le résultat par $\prodovertwo$ puis $x$ pour bien obtenir
	\begin{align*}
		\BETA - \prodovertwo\left(x-\ALPHA\right)^2 &= - \prodovertwo \cdot x^2 + \prodovertwo \cdot \xmax \cdot x \\
														&= \prodovertwo \cdot x \cdot (\xmax - x) \\
														&= \mathcal{A}(x),
	\end{align*}
	comme souhaité.
}


\exe{
	 En déduire que, pour tout $x\in[0;\xmax]$,
		\[ \mathcal{A}(x) \leq  \BETA , \]
	et donc que $\mathcal{A}$ atteint son maximum en $x^\star=\ALPHA$.
}{
	Un carré est toujours positif, donc son opposé toujours négatif.
	L'expression de $\mathcal{A}(x)$ de la question $9$ permet donc de conclure car on a, pour tout $x$ du domaine,
		\begin{align*}
			- \prodovertwo\left(x-\ALPHA\right)^2 &\leq 0 \\
			\BETA - \prodovertwo\left(x-\ALPHA\right)^2 &\leq \BETA \\
			\mathcal{A}(x) \leq \BETA.
		\end{align*}
		
	En outre, l'égalité n'est vérifiée que lorsque le carré est nul, c'est-à-dire quand l'expression à l'intérieur du carré est nulle.
	On a donc
		\begin{align*}
			\mathcal{A}(x^\star) = \BETA && \iff && \left(x^\star-\ALPHA\right)^2 = 0 && \iff && x^\star - \ALPHA = 0 && \iff x^\star = \ALPHA.
		\end{align*}
}


\end{document}

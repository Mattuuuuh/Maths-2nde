% DYSLEXIA SWITCH
\newif\ifdys
		
				% ENABLE or DISABLE font change
				% use XeLaTeX if true
				\dystrue
				\dysfalse


\ifdys

\documentclass[a4paper, 14pt]{extarticle}
\usepackage{amsmath,amsfonts,amsthm,amssymb,mathtools}

\tracinglostchars=3 % Report an error if a font does not have a symbol.
\usepackage{fontspec}
\usepackage{unicode-math}
\defaultfontfeatures{ Ligatures=TeX,
                      Scale=MatchUppercase }

\setmainfont{OpenDyslexic}[Scale=1.0]
\setmathfont{Fira Math} % Or maybe try KPMath-Sans?
\setmathfont{OpenDyslexic Italic}[range=it/{Latin,latin}]
\setmathfont{OpenDyslexic}[range=up/{Latin,latin,num}]

\else

\documentclass[a4paper, 12pt]{extarticle}

\usepackage[utf8x]{inputenc}
\usepackage{lmodern,textcomp}
\usepackage{amsmath,amsfonts,amsthm,amssymb,mathtools}

\fi


\usepackage[french]{babel}
\usepackage[
a4paper,
margin=2cm,
nomarginpar,% We don't want any margin paragraphs
]{geometry}
\usepackage{icomma}

\usepackage{fancyhdr}
\usepackage{array}

\usepackage{multicol, enumerate}
\newcolumntype{P}[1]{>{\centering\arraybackslash}p{#1}}


\usepackage{stackengine}
\newcommand\xrowht[2][0]{\addstackgap[.5\dimexpr#2\relax]{\vphantom{#1}}}

% theorems

\theoremstyle{plain}
\newtheorem{theorem}{Th\'eor\`eme}
\newtheorem*{theorem*}{Th\'eor\`eme}
\newtheorem*{sol}{Solution}
\theoremstyle{definition}
\newtheorem{ex}{Exercice}

% corps
\newcommand{\C}{\mathcal{C}}
\newcommand{\R}{\mathbb{R}}
\newcommand{\Rnn}{\mathbb{R}^{2n}}
\newcommand{\Z}{\mathbb{Z}}
\newcommand{\N}{\mathbb{N}}
\newcommand{\Q}{\mathbb{Q}}

% domain
\newcommand{\D}{\mathcal{D}}


% date
\usepackage{advdate}
\AdvanceDate[0]


% plots
\usepackage{pgfplots}

% for calligraphic C
\usepackage{calrsfs}

%degrees
\usepackage{gensymb}

% SOLUTION SWITCH
\newif\ifsolutions
				\solutionstrue
				\solutionsfalse

\ifsolutions
	\newcommand{\exe}[2]{
		\begin{ex} #1  \end{ex}
		\begin{sol} #2 \end{sol}
	}
\else
	\newcommand{\exe}[2]{
		\begin{ex} #1  \end{ex}
	}
	
\fi

\begin{document}
\pagestyle{fancy}
\fancyhead[L]{Seconde 13}
\fancyhead[C]{\textbf{Fonctions et optimisation \ifsolutions -- Solutions  \fi}}
\fancyhead[R]{\today}

\exe{
	On considère un point $P$ sur un quart de cercle de rayon $2$ et de centre $O$, l'origine d'un repère.
	On pose $\theta \in [0; 90\degree]$ l'angle entre la droite $(OP)$ et l'axe des abscisses.
	Chaque choix de $\theta$ correspond donc à un point et aussi à un rectangle comme représentés ci-dessous.
	
	\begin{center}
	\begin{tikzpicture}[>=stealth, scale=1]
		\begin{axis}[xmin = 0, xmax=2.1, ymin=0, ymax=2.1, axis x line=middle, axis y line=middle, axis line style=->, grid=both]
			\addplot[no marks, blue, -, thick] expression[domain=0:2, samples=50]{sqrt(4-x^2)};
			\addplot[black, mark=*, mark size = 1, thick] (1.25, 1.561) node[above] {$P$};
			
			
			\addplot[no marks, black, -, thick] expression[domain=0:1.25, samples=2]{1.561/1.25*x};
			
			\addplot[no marks, red, -, thick] expression[domain=0.13:.2, samples=50]{sqrt(.04-x^2)}
			node[right, pos=.2] {$\theta$};
			
			\addplot[no marks, black, -, thick] expression[domain=0:1.25, samples=2]{1.561};
			\draw[black, thick] (axis cs:1.25, 0) -- (axis cs:1.25,1.561);
		\end{axis}
	\end{tikzpicture}
	\end{center}
	
	\begin{enumerate}
		\item
		Montrer que l'aire $f(\theta)$ du rectangle est donnée par 
			\[ f(\theta) = 4 \cos(\theta) \sin(\theta). \]
		\item
		On souhaite choisir un angle $\theta \in [0; 90\degree]$ qui maximise l'aire $f(\theta)$.
		Justifier qu'on puisse se restreindre à chercher $\theta$ dans $[0 ; 45\degree]$.
		
		\item
		On admet que, pour tous les angles $\theta \in [0 ; 45\degree]$,
			\[ \cos(\theta) \sin(\theta) = \dfrac12 \sin(2\theta). \]
		Utiliser cette identité pour trouver le maximum de $f$ et l'angle $\theta$ qui le réalise.
		
		\item
		En déduire les coordonnées du point $P$ sur l'arc de cercle pour lequel l'aire est maximale.
	\end{enumerate}
}
{}

\exe{
	On considère exactement la même situation qu'à l'exercice précédent mais cette fois ci en prenant $x$ l'abscisse du point $P$ du quart de cercle comme paramètre.
	On a donc $x \in [0; 2]$.
	
	\begin{center}
	\begin{tikzpicture}[>=stealth, scale=1]
		\begin{axis}[xmin = 0, xmax=2.1, ymin=0, ymax=2.1, axis x line=middle, axis y line=middle, axis line style=->, grid=both]
			\addplot[no marks, blue, -, thick] expression[domain=0:2, samples=50]{sqrt(4-x^2)};
			\addplot[black, mark=*, mark size = 1, thick] (1.25, 1.561) node[above=5pt] {$P(x;y)$};
			
			\addplot[no marks, black, -, thick] expression[domain=0:1.25, samples=2]{1.561};
			\draw[black, thick] (axis cs:1.25, 0) -- (axis cs:1.25,1.561)
			node[ left=1pt, pos=0.05] {$x$};
		\end{axis}
	\end{tikzpicture}
	\end{center}
	
	\begin{enumerate}
		\item Montrer que l'aire du rectangle est donnée par
			\[ g(x) = x \sqrt{4-x^2}. \]
		\item
			Estimer la valeur de $x$ qui maximise $g$ à l'aide du tableau de la calculatrice.
		\item
			Déduire de l'exercice précédent que $f(x)$ est maximal en $x=\sqrt{2}$.
	\end{enumerate}
}{}

\end{document}

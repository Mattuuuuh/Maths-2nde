%!TEX encoding = UTF8
%!TEX root =notes.tex


%%%%%%%%%%%%%%%%%%%%%%%%%%%%%%%%%
% PACKAGE IMPORTS
%%%%%%%%%%%%%%%%%%%%%%%%%%%%%%%%%


\usepackage[french]{babel}

\usepackage[tmargin=2cm,rmargin=1in,lmargin=1in,margin=0.85in,bmargin=2cm,footskip=.2in]{geometry}
\usepackage{amsmath,amsfonts,amsthm,amssymb,mathtools}
\usepackage[varbb]{newpxmath}
\usepackage{xfrac}
\usepackage[makeroom]{cancel}
\usepackage{mathtools}
\usepackage{bookmark}
\usepackage{enumitem}
\usepackage{hyperref,theoremref}
\hypersetup{
	pdftitle={Assignment},
	colorlinks=true, linkcolor=doc!90,
	bookmarksnumbered=true,
	bookmarksopen=true
}
\usepackage[most,many,breakable]{tcolorbox}
\usepackage{xcolor}
\usepackage{varwidth}
\usepackage{varwidth}
\usepackage{etoolbox}
%\usepackage{authblk}
\usepackage{nameref}
\usepackage{multicol,array}
\usepackage{tikz-cd}
\usepackage[ruled,vlined,linesnumbered]{algorithm2e}
\usepackage{comment} % enables the use of multi-line comments (\ifx \fi) 
\usepackage{import}
\usepackage{xifthen}
\usepackage{pdfpages}
\usepackage{transparent}


\newcommand\mycommfont[1]{\footnotesize\ttfamily\textcolor{blue}{#1}}
\SetCommentSty{mycommfont}
\newcommand{\incfig}[1]{%
    \def\svgwidth{\columnwidth}
    \import{./figures/}{#1.pdf_tex}
}

\usepackage{tikzsymbols}
%\renewcommand\qedsymbol{$\Laughey$}


%\usepackage{import}
%\usepackage{xifthen}
%\usepackage{pdfpages}
%\usepackage{transparent}


%%%%%%%%%%%%%%%%%%%%%%%%%%%%%%
% SELF MADE COLORS
%%%%%%%%%%%%%%%%%%%%%%%%%%%%%%



\definecolor{myg}{RGB}{56, 140, 70}
\definecolor{myb}{RGB}{45, 111, 177}
\definecolor{myr}{RGB}{199, 68, 64}
\definecolor{mytheorembg}{HTML}{F2F2F9}
\definecolor{mytheoremfr}{HTML}{00007B}
\definecolor{mylenmabg}{HTML}{FFFAF8}
\definecolor{mylenmafr}{HTML}{983b0f}
\definecolor{mypropbg}{HTML}{f2fbfc}
\definecolor{mypropfr}{HTML}{191971}
\definecolor{myexamplebg}{HTML}{F2FBF8}
\definecolor{myexamplefr}{HTML}{88D6D1}
\definecolor{myexampleti}{HTML}{2A7F7F}
\definecolor{mydefinitbg}{HTML}{E5E5FF}
\definecolor{mydefinitfr}{HTML}{3F3FA3}
\definecolor{notesgreen}{RGB}{0,162,0}
\definecolor{myp}{RGB}{197, 92, 212}
\definecolor{mygr}{HTML}{2C3338}
\definecolor{myred}{RGB}{127,0,0}
\definecolor{myyellow}{RGB}{169,121,69}
\definecolor{myexercisebg}{HTML}{F2FBF8}
\definecolor{myexercisefg}{HTML}{88D6D1}


%%%%%%%%%%%%%%%%%%%%%%%%%%%%
% TCOLORBOX SETUPS
%%%%%%%%%%%%%%%%%%%%%%%%%%%%

\setlength{\parindent}{1cm}
%================================
% THEOREM BOX
%================================

\tcbuselibrary{theorems,skins,hooks}
\newtcbtheorem[number within=chapter]{Theorem}{Théorème}
{%
	enhanced,
	breakable,
	colback = mytheorembg,
	frame hidden,
	boxrule = 0sp,
	borderline west = {2pt}{0pt}{mytheoremfr},
	sharp corners,
	detach title,
	before upper = \tcbtitle\par\smallskip,
	coltitle = mytheoremfr,
	fonttitle = \bfseries\sffamily,
	description font = \mdseries,
	separator sign none,
	segmentation style={solid, mytheoremfr},
}
{th}


\tcbuselibrary{theorems,skins,hooks}
\newtcolorbox{Theoremcon}
{%
	enhanced
	,breakable
	,colback = mytheorembg
	,frame hidden
	,boxrule = 0sp
	,borderline west = {2pt}{0pt}{mytheoremfr}
	,sharp corners
	,description font = \mdseries
	,separator sign none
}

%================================
% Corollery
%================================
\tcbuselibrary{theorems,skins,hooks}
\newtcbtheorem[use counter=tcb@cnt@Theorem]{Corollary}{Corollaire}
{%
	enhanced
	,breakable
	,colback = myp!10
	,frame hidden
	,boxrule = 0sp
	,borderline west = {2pt}{0pt}{myp!85!black}
	,sharp corners
	,detach title
	,before upper = \tcbtitle\par\smallskip
	,coltitle = myp!85!black
	,fonttitle = \bfseries\sffamily
	,description font = \mdseries
	,separator sign none
	,segmentation style={solid, myp!85!black}
}
{th}

%================================
% LENMA
%================================

\tcbuselibrary{theorems,skins,hooks}
\newtcbtheorem[use counter=tcb@cnt@Theorem]{Lemma}{Lemme}
{%
	enhanced,
	breakable,
	colback = mylenmabg,
	frame hidden,
	boxrule = 0sp,
	borderline west = {2pt}{0pt}{mylenmafr},
	sharp corners,
	detach title,
	before upper = \tcbtitle\par\smallskip,
	coltitle = mylenmafr,
	fonttitle = \bfseries\sffamily,
	description font = \mdseries,
	separator sign none,
	segmentation style={solid, mylenmafr},
}
{th}


%================================
% PROPOSITION
%================================

\tcbuselibrary{theorems,skins,hooks}
\newtcbtheorem[use counter=tcb@cnt@Theorem]{Prop}{Proposition}
{%
	enhanced,
	breakable,
	colback = mypropbg,
	frame hidden,
	boxrule = 0sp,
	borderline west = {2pt}{0pt}{mypropfr},
	sharp corners,
	detach title,
	before upper = \tcbtitle\par\smallskip,
	coltitle = mypropfr,
	fonttitle = \bfseries\sffamily,
	description font = \mdseries,
	separator sign none,
	segmentation style={solid, mypropfr},
}
{th}


%================================
% CLAIM
%================================

\tcbuselibrary{theorems,skins,hooks}
\newtcbtheorem[use counter=tcb@cnt@Theorem]{claim}{Claim}
{%
	enhanced
	,breakable
	,colback = myg!10
	,frame hidden
	,boxrule = 0sp
	,borderline west = {2pt}{0pt}{myg}
	,sharp corners
	,detach title
	,before upper = \tcbtitle\par\smallskip
	,coltitle = myg!85!black
	,fonttitle = \bfseries\sffamily
	,description font = \mdseries
	,separator sign none
	,segmentation style={solid, myg!85!black}
}
{th}



%================================
% Exercise
%================================

\tcbuselibrary{theorems,skins,hooks}
\newtcbtheorem[use counter=tcb@cnt@Theorem]{Exercise}{Exercice}
{%
	enhanced,
	breakable,
	colback = myexercisebg,
	frame hidden,
	boxrule = 0sp,
	borderline west = {2pt}{0pt}{myexercisefg},
	sharp corners,
	detach title,
	before upper = \tcbtitle\par\smallskip,
	coltitle = myexercisefg,
	fonttitle = \bfseries\sffamily,
	description font = \mdseries,
	separator sign none,
	segmentation style={solid, myexercisefg},
}
{th}

%================================
% EXAMPLE BOX
%================================

\newtcbtheorem[use counter=tcb@cnt@Theorem]{Example}{Exemple}
{%
	colback = myexamplebg
	,breakable
	,colframe = myexamplefr
	,coltitle = myexampleti
	,boxrule = 1pt
	,sharp corners
	,detach title
	,before upper=\tcbtitle\par\smallskip
	,fonttitle = \bfseries
	,description font = \mdseries
	,separator sign none
	,description delimiters parenthesis
}
{ex}

%================================
% DEFINITION BOX
%================================

\newtcbtheorem[use counter=tcb@cnt@Theorem]{Definition}{Définition}{enhanced,
	before skip=2mm,after skip=2mm, colback=red!5,colframe=red!80!black,boxrule=0.5mm,
	attach boxed title to top left={xshift=1cm,yshift*=1mm-\tcboxedtitleheight}, varwidth boxed title*=-3cm,
	boxed title style={frame code={
					\path[fill=tcbcolback]
					([yshift=-1mm,xshift=-1mm]frame.north west)
					arc[start angle=0,end angle=180,radius=1mm]
					([yshift=-1mm,xshift=1mm]frame.north east)
					arc[start angle=180,end angle=0,radius=1mm];
					\path[left color=tcbcolback!60!black,right color=tcbcolback!60!black,
						middle color=tcbcolback!80!black]
					([xshift=-2mm]frame.north west) -- ([xshift=2mm]frame.north east)
					[rounded corners=1mm]-- ([xshift=1mm,yshift=-1mm]frame.north east)
					-- (frame.south east) -- (frame.south west)
					-- ([xshift=-1mm,yshift=-1mm]frame.north west)
					[sharp corners]-- cycle;
				},interior engine=empty,
		},
	fonttitle=\bfseries,
	title={#2},#1}{def}

%================================
% Solution BOX
%================================

\makeatletter
\newtcbtheorem[use counter=tcb@cnt@Theorem]{question}{Question}{enhanced,
	breakable,
	colback=white,
	colframe=myb!80!black,
	attach boxed title to top left={yshift*=-\tcboxedtitleheight},
	fonttitle=\bfseries,
	title={#2},
	boxed title size=title,
	boxed title style={%
			sharp corners,
			rounded corners=northwest,
			colback=tcbcolframe,
			boxrule=0pt,
		},
	underlay boxed title={%
			\path[fill=tcbcolframe] (title.south west)--(title.south east)
			to[out=0, in=180] ([xshift=5mm]title.east)--
			(title.center-|frame.east)
			[rounded corners=\kvtcb@arc] |-
			(frame.north) -| cycle;
		},
	#1
}{def}
\makeatother

%================================
% SOLUTION BOX
%================================

\makeatletter
\newtcolorbox{solution}{enhanced,
	breakable,
	colback=white,
	colframe=myg!80!black,
	attach boxed title to top left={yshift*=-\tcboxedtitleheight},
	title=Solution,
	boxed title size=title,
	boxed title style={%
			sharp corners,
			rounded corners=northwest,
			colback=tcbcolframe,
			boxrule=0pt,
		},
	underlay boxed title={%
			\path[fill=tcbcolframe] (title.south west)--(title.south east)
			to[out=0, in=180] ([xshift=5mm]title.east)--
			(title.center-|frame.east)
			[rounded corners=\kvtcb@arc] |-
			(frame.north) -| cycle;
		},
}
\makeatother

%================================
% Question BOX
%================================

\makeatletter
\newtcbtheorem[use counter=tcb@cnt@Theorem]{qstion}{Question}{enhanced,
	breakable,
	colback=white,
	colframe=mygr,
	attach boxed title to top left={yshift*=-\tcboxedtitleheight},
	fonttitle=\bfseries,
	title={#2},
	boxed title size=title,
	boxed title style={%
			sharp corners,
			rounded corners=northwest,
			colback=tcbcolframe,
			boxrule=0pt,
		},
	underlay boxed title={%
			\path[fill=tcbcolframe] (title.south west)--(title.south east)
			to[out=0, in=180] ([xshift=5mm]title.east)--
			(title.center-|frame.east)
			[rounded corners=\kvtcb@arc] |-
			(frame.north) -| cycle;
		},
	#1
}{def}
\makeatother

\newtcbtheorem[number within=chapter]{wconc}{Wrong Concept}{
	breakable,
	enhanced,
	colback=white,
	colframe=myr,
	arc=0pt,
	outer arc=0pt,
	fonttitle=\bfseries\sffamily\large,
	colbacktitle=myr,
	attach boxed title to top left={},
	boxed title style={
			enhanced,
			skin=enhancedfirst jigsaw,
			arc=3pt,
			bottom=0pt,
			interior style={fill=myr}
		},
	#1
}{def}



%================================
% NOTE BOX
%================================

\usetikzlibrary{arrows,calc,shadows.blur}
\tcbuselibrary{skins}
\newtcolorbox{note}[1][]{%
	enhanced jigsaw,
	colback=gray!20!white,%
	colframe=gray!80!black,
	size=small,
	boxrule=1pt,
	title=\colorbox{white!100}{\textbf{ Remarque }},
	halign title=flush center,
	coltitle=black,
	breakable,
	drop shadow=black!50!white,
	attach boxed title to top left={xshift=1cm,yshift=-\tcboxedtitleheight/2,yshifttext=-\tcboxedtitleheight/2},
	minipage boxed title=2.6cm,
	boxed title style={%
			colback=white,
			size=fbox,
			boxrule=1pt,
			boxsep=2pt,
			underlay={%
					\coordinate (dotA) at ($(interior.west) + (-0.5pt,0)$);
					\coordinate (dotB) at ($(interior.east) + (0.5pt,0)$);
					\begin{scope}
						\clip (interior.north west) rectangle ([xshift=3ex]interior.east);
						\filldraw [white, blur shadow={shadow opacity=60, shadow yshift=-.75ex}, rounded corners=2pt] (interior.north west) rectangle (interior.south east);
					\end{scope}
					\begin{scope}[gray!80!black]
						\fill (dotA) circle (2pt);
						\fill (dotB) circle (2pt);
					\end{scope}
				},
		},
	#1,
}

%================================
% STRATÉGIE BOX
%================================

\usetikzlibrary{arrows,calc,shadows.blur}
\tcbuselibrary{skins}
\newtcolorbox{strategy}[1][]{%
	enhanced jigsaw,
	colback=myb!20!white,%
	colframe=gray!80!black,
	size=small,
	boxrule=1pt,
	title=\colorbox{white!100}{\textbf{ Stratégie }},
	halign title=flush center,
	coltitle=black,
	breakable,
	drop shadow=black!50!white,
	attach boxed title to top left={xshift=1cm,yshift=-\tcboxedtitleheight/2,yshifttext=-\tcboxedtitleheight/2},
	minipage boxed title=2.5cm,
	boxed title style={%
			colback=white,
			size=fbox,
			boxrule=1pt,
			boxsep=2pt,
			underlay={%
					\coordinate (dotA) at ($(interior.west) + (-0.5pt,0)$);
					\coordinate (dotB) at ($(interior.east) + (0.5pt,0)$);
					\begin{scope}
						\clip (interior.north west) rectangle ([xshift=3ex]interior.east);
						\filldraw [white, blur shadow={shadow opacity=60, shadow yshift=-.75ex}, rounded corners=2pt] (interior.north west) rectangle (interior.south east);
					\end{scope}
					\begin{scope}[gray!80!black]
						\fill (dotA) circle (2pt);
						\fill (dotB) circle (2pt);
					\end{scope}
				},
		},
	#1,
}

%================================
% MÉTHODE BOX
%================================

\usetikzlibrary{arrows,calc,shadows.blur}
\tcbuselibrary{skins}
\newtcolorbox{methode}[1][]{%
	enhanced jigsaw,
	colback=white,%
	colframe=gray!80!black,
	size=small,
	boxrule=1pt,
	title=\textbf{Méthode},
	halign title=flush center,
	coltitle=black,
	breakable,
	drop shadow=black!50!white,
	attach boxed title to top left={xshift=1cm,yshift=-\tcboxedtitleheight/2,yshifttext=-\tcboxedtitleheight/2},
	minipage boxed title=2.5cm,
	boxed title style={%
			colback=white,
			size=fbox,
			boxrule=1pt,
			boxsep=2pt,
			underlay={%
					\coordinate (dotA) at ($(interior.west) + (-0.5pt,0)$);
					\coordinate (dotB) at ($(interior.east) + (0.5pt,0)$);
					\begin{scope}
						\clip (interior.north west) rectangle ([xshift=3ex]interior.east);
						\filldraw [white, blur shadow={shadow opacity=60, shadow yshift=-.75ex}, rounded corners=2pt] (interior.north west) rectangle (interior.south east);
					\end{scope}
					\begin{scope}[gray!80!black]
						\fill (dotA) circle (2pt);
						\fill (dotB) circle (2pt);
					\end{scope}
				},
		},
	#1,
}

%%%%%%%%%%%%%%%%%%%%%%%%%%%%%%%%%%%%%%%%%%%
% TABLE OF CONTENTS
%%%%%%%%%%%%%%%%%%%%%%%%%%%%%%%%%%%%%%%%%%%

\usepackage{tikz}

\definecolor{doc}{RGB}{0,60,110}
\usepackage{titletoc}
\contentsmargin{0cm}
\titlecontents{chapter}[3.7pc]
{\addvspace{30pt}%
	\begin{tikzpicture}[remember picture, overlay]%
		\draw[fill=doc!60,draw=doc!60] (-7,-.1) rectangle (-0.2,.6);%
		\pgftext[left,x=-3.5cm,y=0.2cm]{\color{white}\Large\sc\bfseries Chapitre\ \thecontentslabel};%
	\end{tikzpicture}\color{doc!60}\large\sc\bfseries}%
{}
{}
{\;\titlerule\;\large\sc\bfseries Page \thecontentspage
	\begin{tikzpicture}[remember picture, overlay]
		\draw[fill=doc!60,draw=doc!60] (2pt,0) rectangle (4,0.1pt);
	\end{tikzpicture}}%
\titlecontents{section}[3.7pc]
{\addvspace{2pt}}
{\contentslabel[\thecontentslabel]{2pc}}
{}
{\hfill\small \thecontentspage}
[]
\titlecontents*{subsection}[3.7pc]
{\addvspace{-1pt}\small}
{}
{}
{\ --- \small\thecontentspage}
[ \textbullet\ ][]

\makeatletter
\renewcommand{\tableofcontents}{%
	\chapter*{%
	  \vspace*{-20\p@}%
	  \begin{tikzpicture}[remember picture, overlay]%
		  \pgftext[right,x=15cm,y=0.2cm]{\color{doc!60}\Huge\sc\bfseries \contentsname};%
		  \draw[fill=doc!60,draw=doc!60] (13,-.75) rectangle (20,1);%
		  \clip (13,-.75) rectangle (20,1);
		  \pgftext[right,x=15cm,y=0.2cm]{\color{white}\Huge\sc\bfseries \contentsname};%
	  \end{tikzpicture}}%
	\@starttoc{toc}}
\makeatother


%%%%%%%%%%%%%%%%%%%%%%%%%%%%%%%%%%%%%%%%%%%
% MINTED FOR PYTHON ALGORITHMS
%%%%%%%%%%%%%%%%%%%%%%%%%%%%%%%%%%%%%%%%%%%

\usepackage{tcolorbox}
\tcbuselibrary{minted,breakable,xparse,skins}
\definecolor{bg}{gray}{0.95}
\DeclareTCBListing{mintedbox}{O{}m!O{}}{%
  breakable=true,
  listing engine=minted,
  listing only,
  minted language=#2,
  minted style=default,
  minted options={%
    linenos,
    gobble=0,
    breaklines=true,
    breakafter=,,
    fontsize=\small,
    numbersep=8pt,
    #1},
  boxsep=0pt,
  left skip=0pt,
  right skip=0pt,
  left=25pt,
  right=0pt,
  top=3pt,
  bottom=3pt,
  arc=5pt,
  leftrule=0pt,
  rightrule=0pt,
  bottomrule=2pt,
  toprule=2pt,
  colback=bg,
  colframe=orange!70,
  enhanced,
  overlay={%
    \begin{tcbclipinterior}
    \fill[orange!20!white] (frame.south west) rectangle ([xshift=20pt]frame.north west);
    \end{tcbclipinterior}},
  #3}
  
  
 % for braces
\usetikzlibrary{decorations.pathreplacing}


\SetDate[10/12/2025]

\begin{document}
\pagestyle{fancy}
\fancyhead[L]{Seconde 5}
\fancyhead[C]{\textbf{Exercices de colle}}
\fancyhead[R]{\today}

\fancyfoot[C]{Source : \href{https://www.insee.fr/fr/statistiques/4277635}{https://www.insee.fr/fr/statistiques/4277635}\\\thepage}

\exe{}{
	En 2019, 753 000 bébés sont nés en France, soit 6 000  naissances de moins qu’en 2018 ($– 0,7\%$). Le nombre de naissances baisse chaque année depuis cinq ans, mais à un rythme qui ralentit au fil des années. Alors que la baisse était de 2,4 \% en 2015, elle est passée à 1,9 \% en 2016 puis 1,8 \% en 2017, 1,4 \% en 2018 et enfin 0,7 \% en 2019. En France métropolitaine, le nombre de naissances s’établit à 714 000. Il reste plus élevé que le point bas de 1994 (711 000).
	
	\begin{center}
	\includegraphics[scale=0.55]{fig.png}
	\end{center}
	
	\begin{enumerate}
		\item Calculer, à l'aide du texte, le nombre de bébés nés en France en $2018, 2017, 2016, 2015$, et $2014$.
		Comparer avec la figure 3.
		
		\item Calculer l'évolution relative du nombre de bébés nés en France métropolitaine entre $1994$ et $2019$.
		
		\item Calculer l'évolution relative du nombre de bébés nés en France métropolitaine entre $1971$ et $1976$.
		
		\item Donner approximativement le nombre de bébés nés en France métropolitaine en $1916$ et en $1941$. Expliquer les pics de natalité.
		
		\item Calculer et comparer les évolutions relatives de natalité d'après-guerre : entre $1916$ et $1920$ contre $1941$ et $1947$.
	\end{enumerate}
}{exe:insee-naissances}{
	\begin{enumerate}
		\item Du texte on déduit les évolutions suivantes.
		
		\begin{tabular}{|c|c|c|c|c|c|}\hline
			Année & 2019 & 2018 & 2017 & 2016 & 2015 \\ \hline
			Évolution & -0,7\% & -1,4\% & -1,8\% & -1,9\% & -2,4\% \\ \hline
		\end{tabular}
		
		L'année $2018$ est en fait déjà donné par le texte à $753+6 = 759$ milliers de naissances.
		
		Pour l'année $2017$, on utilise l'évolution de $-1,4\%$.
		
		On calcule donc que, pour retourner en arrière, l'évolution réciproque de $+1,4\%$ correspond à un coefficient multiplicatif de 
			\[ 1-\dfrac{1,4}{100} = 0,983. \]
		L'évolution réciproque est alors donnée par $\dfrac{1}{0,986}$, qui multipliée à $759$ donne environ $770$.
		
		\begin{tabular}{|c|c|c|c|c|c|c|}\hline
			Année & 2019 & 2018 & 2017 & 2016 & 2015 & 2014 \\ \hline
			\makecell{Nombre de  naissances \\ (approximativement, en milliers)} & 753 & 759 & 770 & 784 & 799 & 819 \\ \hline
		\end{tabular}
		
		On remarque que les valeurs obtenues sont en dessous de celles du graphique qui n'indique $800$ mille naissances qu'en $2010$.
		C'est parce que celui-ci ne concerne que la France métropolitaine.
		
		\item 
		Le texte nous donne deux valeurs : 711 mille en 1994, et 714 mille en 2019.
		La proportion est donnée par 
			\[ \dfrac{714}{711} \approx 1,004. \]
		Comme $1,004 - 1 = 0,004 = 0,4\%$, le nombre de bébés nés en France métropolitaine a augmenté de 0,4\% entre 1994 et 2019.
		
		\item
		On lit graphiquement (et donc approximativement) les valeurs pour obtenir une proportion d'environ
			\[ \approx \dfrac{725}{880} \approx 0,82. \]
		Il s'agit d'une diminution d'environ $1-0,82 = 18\%$.
		
		\item 
		La figure nous indique environ $390$ mille naissances en 1916 et $520$ mille naissances en 1941.
		Ces valeurs sont prises en pendant la Première et la Seconde Guerre mondiale (et au moins $9$ mois après leur début !).
		
		\item Entre $1916$ et $1920$ on calcule un proportion d'environ
			\[ \dfrac{840}{390} \approx 2,15, \]
		qui correspond à une évolution de $2,15-1 = 115\%$.
		
		Entre $1941$ et $1947$ on calcule une proportion d'environ
			\[ \dfrac{880}{520} \approx 1,69, \]
		qui correspond à une évolution de $1,69 - 1 = 69\%$.
		
		Les natalités après la Seconde Guerre mondiale sont \underline{relativement} moins importante comparé à la Première Guerre mondiale. 
		Cependant, les valeurs hautes de natalités persistent dans le temps jusqu'en $1974$ — c'est le baby-boom.
		
	\end{enumerate}
}

\newpage
\fancyfoot[C]{\thepage}


\exe{}{
	En $2022$ en France, la consommation de gaz naturel s'établit à $463$ TWh.
	En $2021$, celle-ci s'élevait plutôt à $475,85$ TWh.

	Calculer le pourcentage de diminution de la consommation entre l'année $2021$ et l'année $2022$.
}{exe:evol10}{
	On calcule la proportion $\dfrac{463}{475,85} \approx 0,973 = 97,3\%$.
	Celle-ci correspond à une diminution de $2,7\%$.
}


\exe{}{
	Tracer la courbe de la fonction $f$ sur $\D=[-5;3]$ donnée algébriquement par
		\[ f(x) = 1-x. \]
	Que dire de $\C_f$ ?
}{exe:graph-droite}{		
	$\C_f$ est une droite.
	\begin{center}
	\begin{tikzpicture}[>=stealth]
		\begin{axis}[xmin = -5.5, xmax=3.5, ymin=-3.5, ymax=7.5, axis x line=middle, axis y line=middle, axis line style=->, grid=both]
			\addplot[no marks, BLUE_E, very thick, -] expression[domain=-5:3, samples=2]{1-x} 
			node[above, pos=.4] {$\C_f$};
		\end{axis}
	\end{tikzpicture}
	\end{center}
		
}

\exemulticols{}{
	Donner l'ensemble des \mbox{$x \in [-2,5 ; 2,3]$} vérifiant $f(x) = g(x)$ à l'aide des graphes de $f$ et $g$ ci-contre.
}{
	\begin{center}
	\begin{tikzpicture}[>=stealth]
		\begin{axis}[xmin = -2.5, xmax=2.3, ymin=-4.1, ymax=4.6, axis x line=middle, axis y line=middle, axis line style=->, grid=both,
			grid style = {opacity=.5},
			x=1.5cm,
			xtick={-3, -2, ..., 2},
			y=15pt,
			]
			\addplot[no marks, BLUE_E, -, very thick] expression[domain=-2.5:2.3, samples=101]{(x-1)*(x-2)*(x+2)/2}
			node[pos=.5, above]{$\mathcal{C}_f$};
			\addplot[no marks, RED_E, -,  very thick] expression[domain=-2.5:2.3, samples=101]{1/(x-2.4) + 1/(x+2.6) + 2*x}
			node[pos=.65, above]{$\mathcal{C}_g$};
		\end{axis}
	\end{tikzpicture}
	\end{center}
}{exe:f-equals-g}{
	On trouve approximativement $x\approx-2,2$ ; $x\approx0,5$ et $x \approx 2,1$.
}

\exe{}{
	Un marchand décide de changer le prix de sa marchandise de 1 000€ à 999€, prix psychologique.
	Il compare le nombre de ventes avant et après le changement de prix.
	\begin{enumerate}
		\item De quel pourcentage les ventes doivent-elles augmenter pour que le chiffre d'affaire reste inchangé ?
		\item De quel pourcentage les ventes doivent-elles augmenter pour que le chiffre d'affaire augmente de 10\% ?
	\end{enumerate}
}{exe:prix-psychologique}{
	\begin{enumerate}
		\item
		Le coefficient multiplicateur correspondant à l'évolution du prix est donné par 
			\[ \dfrac{999}{1~000} = 0,999, \]
		ce qui implique une diminution de $0,1\%$.
		
		Une évolution réciproque des ventes annulera la diminution du prix.
		Le coefficient multiplicateur réciproque étant 
			\[ \dfrac1{0,999} = \dfrac{1~000}{999} \approx 1,001, \]
		une augmentation de $\approx0,1\%$
		\item De quel pourcentage les ventes doivent-elles augmenter pour que le chiffre d'affaire augmente de 10\% ?
		Le coefficient multiplicateur $CM$ recherché vérifie
			\begin{align*}
				0,999 \times CM = 1,1 && \iff && CM = \dfrac{1,1}{0,999} \approx 1,1011.
			\end{align*}
		Ceci correspond à une augmentation de $\approx 10,11\%$.
	\end{enumerate}
}


\exe{}{
	Lors d'un payement par carte bancaire, une commission valant 1\% du montant de la transaction est versée comme frais bancaires.
	Ainsi, lorsqu'un débiteur paye 100€, le créditeur reçoit 99€, et 1€ est versé à la banque.
	\begin{enumerate}
		\item Pour un payement de 200€, quelle quantité est versée en frais bancaires ? quelle quantité le créditeur reçoit-il ?
		\item Répondre à la même question pour un payement de 202€.
		\item Obtient-on la quantité initiale après une augmentation de 1\% suivie d'une diminution de 1\% ?
		\item Quelle quantité le débiteur doit-il verser pour que le créditeur reçoive exactement 200€ ? Est-ce un nombre réel ? rationnel ? décimal ? entier ?
	\end{enumerate}
}{exe:caution-matthieu-payup}{
	\begin{enumerate}
		\item
		Lorsque le débiteur paye 200€, 1\% de cette quantité est versée en frais bancaires, soit 2€.
		Le créditeur reçoit donc 198€.
		
		\item 
		Lorsque le débiteur paye 202€, 1\% de cette quantité est versée en frais bancaires, soit 2,02€.
		Le créditeur reçoit donc 199,98€.
		
		\item 
		La réponse est non et les deux questions précédentes donnent un exemple : augmenter 200€ de 1\% donne 202€, et diminuer 202€ de 1\% donne 199,98€.
		
		\item 
		Nommons $P$ le prix payé par le débiteur.
		Après diminution de 1\%, $0,99\times P$ est reçu par le créditeur.
		Or, 
			\begin{align*}
				0,99 \times P = 200 && \iff && P = \dfrac{200}{0,99} \approx 202,02.
			\end{align*}
		Bien sûr, $P$ est réel et non entier.
		Pour reconnaître les ensembles dans lesquels $P$ appartient, il faut garder une forme exacte :
			\[ P = \dfrac{200}{0,99} = \dfrac{20~000}{99}. \]
		Il suit que $P$ est un nombre rationnel.
		
		Le chapitre d'arithmétique permettra de conclure immédiatement que $P$ n'est pas décimal, car la fraction est irréductible et que le dénominateur n'admet pas que 2 et 5 comme facteurs premiers ($99 = 11 \times 3^2$).
	\end{enumerate}
}

\exe{, difficulty=2}{
	Considérons la courbe représentative de la fonction $f(x) = x + 1$ et le point $A(2;1)$.
	Le but de l'exercice est de trouver le point de $\C_f$ le plus proche de $A$.
	\begin{enumerate}
		\item
		Grapher $\C_f$ sur le domaine $[0 ; 4]$ et placer le point $A$ dans un repère orthonormé.
		Le point $A$ appartient-il à $\C_f$ ?
		\item
		Montrer qu'un point de $\C_f$ est de la forme $(x ; x+1)$, avec $x\in\R$.
		\item
		Montrer que la distance au carré de ce point à A est égale à
			\[ 2(x-1)^2 + 2. \]
		\item
		Conclure que le point de $\C_f$ le plus proche de $A$ est $B(1 ; 2)$.
		Placer $B$ dans le repère déjà construit.
	\end{enumerate}
}{exe:proj-xp1}{
	TODO
}


%%%%%%%%%%%

\newpage
\fancyhead[C]{\textbf{Solutions}}
\fancyfoot[C]{\thepage}
\shipoutAnswer

\end{document}

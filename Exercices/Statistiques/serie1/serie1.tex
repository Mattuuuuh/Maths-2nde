% DYSLEXIA SWITCH
\newif\ifdys
		
				% ENABLE or DISABLE font change
				% use XeLaTeX if true
				\dystrue
				\dysfalse


\ifdys

\documentclass[a4paper, 14pt]{extarticle}
\usepackage{amsmath,amsfonts,amsthm,amssymb,mathtools}

\tracinglostchars=3 % Report an error if a font does not have a symbol.
\usepackage{fontspec}
\usepackage{unicode-math}
\defaultfontfeatures{ Ligatures=TeX,
                      Scale=MatchUppercase }

\setmainfont{OpenDyslexic}[Scale=1.0]
\setmathfont{Fira Math} % Or maybe try KPMath-Sans?
\setmathfont{OpenDyslexic Italic}[range=it/{Latin,latin}]
\setmathfont{OpenDyslexic}[range=up/{Latin,latin,num}]

\else

\documentclass[a4paper, 12pt]{extarticle}

\usepackage[utf8x]{inputenc}
\usepackage{lmodern,textcomp}
\usepackage{amsmath,amsfonts,amsthm,amssymb,mathtools}

\fi


\usepackage[french]{babel}
\usepackage[
a4paper,
margin=2cm,
nomarginpar,% We don't want any margin paragraphs
]{geometry}
\usepackage{icomma}

\usepackage{fancyhdr}
\usepackage{array}

\usepackage{multicol, enumerate}
\newcolumntype{P}[1]{>{\centering\arraybackslash}p{#1}}


\usepackage{stackengine}
\newcommand\xrowht[2][0]{\addstackgap[.5\dimexpr#2\relax]{\vphantom{#1}}}

% theorems

\theoremstyle{plain}
\newtheorem{theorem}{Th\'eor\`eme}
\newtheorem*{sol}{Solution}
\theoremstyle{definition}
\newtheorem{ex}{Exercice}

% corps
\newcommand{\C}{\mathbb{C}}
\newcommand{\R}{\mathbb{R}}
\newcommand{\Rnn}{\mathbb{R}^{2n}}
\newcommand{\Z}{\mathbb{Z}}
\newcommand{\N}{\mathbb{N}}
\newcommand{\Q}{\mathbb{Q}}

% domain
\newcommand{\D}{\mathbb{D}}


% date
\usepackage{advdate}
\AdvanceDate[0]


% plots
\usepackage{pgfplots}


% SOLUTION SWITCH
\newif\ifsolutions
				\solutionstrue
				\solutionsfalse

\ifsolutions
	\newcommand{\exe}[2]{
		\begin{ex} #1  \end{ex}
		\begin{sol} #2 \end{sol}
	}
\else
	\newcommand{\exe}[2]{
		\begin{ex} #1  \end{ex}
	}
	
\fi

\begin{document}
\pagestyle{fancy}
\fancyhead[L]{Seconde 13}
\fancyhead[C]{\textbf{Statistiques 1\ifsolutions -- Solutions  \fi}}
\fancyhead[R]{\today}

\subsection*{Sous-populations}

\exe{
        Une classe de Seconde comprend $25$ filles pour $9$ garçons.
        Calculer le pourcentage de filles et de garçons dans la classe.
}{
	Le nombre total d'élèves est de $25+9 = 34$.
	
	On calcule donc, pour les filles, la proportion $\dfrac{25}{34} \approx 0,73 = 73\%$.
	
	Idem pour les garçons, $\dfrac{9}{34} \approx 0,27 = 27\%$. 
	Remarquons que la somme des pourcentages est de $100\%$ car $\dfrac{25}{34} + \dfrac{9}{34} = \dfrac{34}{34} = 1 = 100\%$.
	On aurait donc pu déduire le pourcentage de garçons en calculant $100 - 73 = 27$.
}

\exe{
   En sachant que les $16 600$ espèces de fourmis constituent environ $1{,}3\%$ du total des espèces d'insectes répertoriées sur Terre, estimer le nombre total d'espèces d'insectes. 
}{
	On a $\dfrac{16 600}{\text{nombre d'espèces d'insectes}} = 1,3\% = 0,013$.
	
	Par conséquent, 
		\[ \text{nombre d'espèces d'insectes} = \dfrac{16600}{0,013} \approx 1,3 \times 10^{6}, \]
	soit environ $1,3$ millions.
	
	Remarquons que la fraction $\dfrac{16600}{0,013}$ ne donne pas un nombre entier, car le pourcentage a été approximé (\og \emph{environ} $1,3$\% \fg).
}


\exe{
  En 2023 en France, $13\%$ des espèces (faune et flore) sont considérées comme menacées à l'échelle mondiale (catégories ``danger critique'' à ``vulnérable'' de l'UICN).
  Parmis celles-ci, $23\%$ sont en danger critique.
  
  Calculer le pourcentage d'espèces en danger critique par rapport au nombre total d'espèces.
}{
	Notons $E$ l'ensemble des espèces indigènes à la France, $M$ la sous-population des espèces menacées, et $D$ la sous-population des espèces en danger critique.
	On a donc la suite d'inclusions
		\[ D \subset M \subset E. \]
	
	Le texte donne les informations
		\begin{align*}
			|M| = 0,13 \cdot |E| && \text{et} && |D| = 0,23 \cdot |M|
		\end{align*}
	Par conséquent, $|D| = 0,23 \times 0,13 \cdot |E| \approx 0,03 \cdot |E|$.
	Donc les espèces en danger critique constituent $3\%$ des espèces.

	Les proportions sont ainsi multiplicatives. Attention à ne pas naïvement multiplier les pourcentages, car $13\times 23 \approx 300$.
}

\subsection*{Évolution}

\exe{
  Calculer sans calculatrice les valeurs suivantes.
  \begin{multicols}{2}
    \begin{enumerate}
    \item $75\%$ de $60$
    \item $60\%$ de $75$
    \item $72\%$ de $25$
    \item $68\%$ de $20$
    \item $125\%$ de $40$
    \item $40\%$ de $125$
    \end{enumerate}
  \end{multicols}
}{
  \begin{multicols}{2}
    \begin{enumerate}
    \item $\dfrac34 \cdot 60 = 3 \cdot \dfrac{60}4 = 3 \cdot 15 = 45$
    \item $45$
    \item $\dfrac14 \cdot 72 = 18$
    \item $\dfrac15 \cdot 68 = \dfrac{136}{10} = 13,6$
    \item $40 + \dfrac14 \cdot 40 = 50$
    \item $50$
    \end{enumerate}
  \end{multicols}
}

\exe{
  Approximer sans calculatrice les valeurs suivantes.
  \begin{multicols}{2}
    \begin{enumerate}
    \item $33\%$ de $150$
    \item $166\%$ de $180$
    \item $11\%$ de $90$
    \item $89\%$ de $81$
    \item $16,6\%$ de $18$
    \item $83,4\%$ de $36$
    \end{enumerate}
  \end{multicols}
}{
  \begin{multicols}{2}
    \begin{enumerate}
    \item $\approx \dfrac13 \cdot 150 = 50$
    \item $\approx 180 + \dfrac23 \cdot 180 = 180 + 120 = 300$
    \item $\approx \dfrac19 \cdot 90 = 10$
    \item $\approx 81 - \dfrac19 \cdot 81 = 81 - 9 = 72$
    \item $\approx \dfrac16 \cdot 18 = 3$
    \item $\approx 36 - \dfrac16 \cdot 36 = 36 - 6 = 30$
    \end{enumerate}
  \end{multicols}
}

\exe{
  On estime la biomasse totale des fourmis sur Terre à $12$ millions de tonnes.
  Ceci serait égal à $20\%$ de la biomasse humaine.

  Estimer la biomasse totale des humains sur Terre en tonnes.
}{
	On a la relation
		\[ \dfrac{\text{biomasse des fourmis}}{\text{biomasse humaine}} = 0,2. \]
	D'où
		\[ \text{biomasse humaine} = \dfrac{12 \times 10^6}{0,2} \text{T} = 60 \times 10^6 \text{T}.\] 

}

\exe{
  Une jeune femme dépose $10$€ à la banque. Celle-ci lui promet un taux d'intérêt à l'année de $3\%$.
  Ainsi, après la première année, il y aura $1{,}03 \cdot 10 = 10{,}3$€ sur son compte.
  La deuxième année, il y aura $1{,}03 \cdot 10{,}35 = 10{,}609$€, etc...

  À l'aide de la calculatrice, répondre aux questions suivantes.
  \begin{enumerate}
  \item Combien d'argent aura-t-elle après $5$ ans ?
  \item Combien d'argent aura-t-elle après $50$ ans ?
  \item Combien d'argent y aura-t-il sur son compte après $1000$ ans ?
  \end{enumerate}
}
{
  \begin{enumerate}
  \item On multiplie $5$ fois par $1,03$, ce qui donne $1,03^5 \times 10  \approx 11,59$€.
  \item On multiplie $50$ fois par $1,03$, ce qui donne $1,03^{50} \times 10  \approx 43,84$€.
  \item On multiplie $1000$ fois par $1,03$, ce qui donne $1,03^{1000} \times 10  \approx 6,87 \times 10^{13}$€, c'est-à-dire environ $68$ billions d'euros ($1$ billion = $1000$ milliards).
  \end{enumerate}
}

\exe{
  Considérons deux tailleurs, l'un à $250$€ et l'autre à $360$€.
  \begin{enumerate}
  \item Quelle augmentation de prix faut-il appliquer au premier tailleur pour qu'il ait le prix du second ?
  \item Quel rabais faut-il appliquer au second tailleur pour qu'il ait le prix du premier ?
  \end{enumerate}
}{
  \begin{enumerate}
  \item On calcule l'évolution $\dfrac{360}{250}  = 1,44 = 144\%$. Ainsi, le deuxième tailleur vaut $144\%$ du prix du premier : une augmentation de $44\%$ est nécessaire.
  \item On calcule l'évolution $\dfrac{250}{360}  \approx 0,7 = 70\%$. Le premier tailleur vaut environ $70\%$ du prix du second : une diminution de $30\%$ est nécessaire.
  \end{enumerate}
}

\exe{
        À quelle évolution correspond une augmentation de $20\%$ suivie d'une diminution de $20\%$ ?
}{
	Augmenter une quantité $N$ de $20\%$ correspond à la multiplier par $1,2$.
	Une diminution, elle, multiplie par $0,8$.
	
	La quantité finale est donné par 
		\[ 0,8 \cdot (1,2 \cdot N) = (0,8 \cdot 1,2) \cdot N = 0,96 \cdot N, \]
	qui correspond à une diminution de $4\%$.
}

\exe{
  Si on augmente le prix d'un objet de $150\%$, quel rabais faut-il appliquer pour retrouver le prix initial de l'objet ?
}{
	Notons $P$ le prix initial de l'objet.
	Le prix augmenté vaut donc $1,5 \cdot P$.
	Pour retrouver $P$, il faut multiplier le prix augmenté par l'inverse de $1,5$, soit $1,5^{-1} = \dfrac23 \approx 0,666 = 66,6\%$.
	Ceci correspond à une diminution de $33,4\%$.
}

\subsection*{Exercices supplémentaires}

\exe{
        On considère l'ensemble $E= \{1; 2; \dots ; n-1 ; n\}$ dépendant d'un entier naturel $n\geq1$.
        On pose $F = \{k \in E \text{ tq. } 2 | k\}$, l'ensemble des éléments pairs de $E$.

        A-t-on toujours $\dfrac{|F|}{|E|} = \dfrac12$ ? Donner l'ensemble des entiers $n\geq1$ pour lesquels l'égalité est vraie.
}{
	En prenant $n=1; 2; 3; 4$, on conjecture que l'égalité est vraie si et seulement si $n$ est lui-même pair.
	
	D'une part, pour avoir $\dfrac{|F|}{|E|} = \dfrac12$, on doit nécessairement avoir $n = |E| = 2 \cdot |F|$, et donc $n$ pair.
	
	Réciproquement, si $n$ est pair, alors $n=2k$ et $F = \{ 2\times1; 2\times2; 2\times3; \cdot ; 2\times(k-1) ; 2 \times k\}$.
	Ainsi $|F| = k = \dfrac{n}2$ et l'égalité est vérifiée. 
}

\exe{
  En $2023$, le prix moyen du gaz naturel facturé aux ménages français s'élève à $115$€ par MWh, toutes taxes comprises (TTC).
  En $2022$, le prix était de $96$€.

  Calculer le pourcentage d'augmentation du prix entre l'année $2022$ et l'année $2023$.
}{
	On calcule la proportion $\dfrac{115}{96} \approx 1,20 = 120\%$.
	Celle-ci correspond à une augmentation de $20\%$.
}

\exe{
  En $2022$ en France, la consommation de gaz naturel s'établit à $463$ TWh.
  En $2021$, celle-ci s'élevait plutôt à $475{,}85$ TWh.

  Calculer le pourcentage de diminution de la consommation entre l'année $2021$ et l'année $2022$.
}{
	On calcule la proportion $\dfrac{463}{475{,}85} \approx 0,973 = 97,3\%$.
	Celle-ci correspond à une diminution de $2,7\%$.
}

\exe{
  Si on augmente le prix d'un objet de $100p\%$ ($p\geq0$ réel), quel rabais faut-il appliquer (en fonction de $p$) pour retrouver le prix initial de l'objet ?
}{
	Soit $N\geq0$ un prix quelconque, et $(1+p)\cdot N$ le prix augmenté de $100p\%$.
	
	Pour retrouver le prix original, il faut multiplier par $(1+p)^{-1} = \dfrac{1}{1+p}$.
	La diminution correspondante est donnée par
		\[ 1 - \dfrac{1}{1+p} = \dfrac{p}{1+p}. \]
	On comparera avec l'exercice 10, où $p=0,5$, et $\dfrac{p}{1+p} = \dfrac{0,5}{1,5} = \dfrac13 \approx 33,3 \%$.
}

\exe{
    Considérons $p\geq 0$ une proportion et $100p$ le pourcentage associé.
    \begin{enumerate}
    \item À quelle évolution, en fonction de $p$, correspond une augmentation de $100p\%$ suivie d'une diminution de $100p\%$ ?
    \item Quel $p$ choisir pour trouver une diminution finale de $16\%$ ?
    \end{enumerate}
}{

    \begin{enumerate}
    \item Soit $N\geq0$ une quantité. Après une augmentation de $100p\%$ puis une diminution de $100p\%$,
    		la quantité est donnée par $(1-p) \cdot (1+p) \cdot N = (1-p^2) \cdot N$.
    		Ceci correspond à une diminution de $100\left(p^2\right) \%$.
    		
    		On comparera avec l'exercice 9, où $p=0,2$ et $p^2 = 0,04 = 4\%$.
    \item On pose l'égalité suivante
    		\[ 100p^2 = 16. \]
    	Remarquons que $16$ et $100$ sont tous les deux des carrés parfaits :
    		\[ p^2 = \dfrac{16}{100} = \left( \dfrac{4}{10} \right)^2, \]
	et donc $p = \dfrac{4}{10} = 40\%$, car $p\geq 0$.
	
	Vérification : au vu de la question 1, on calcule $0,4^2 = 0,16 = 16\%$.
    \end{enumerate}

}

\exe{
    Soit l'ensemble d'entiers $E=\{n; n+1; \dots; m-1 ; m\}$ dépendant de deux entiers relatifs $n<m$.
    Calculer $|E|$ en fonction de $n$ et $m$.
    Vérifier la formule avec $n=-1$ et $m=1$ en sachant que $|\{-1 ; 0 ; 1 \}| = 3$.
}{
	On soustrait $n$ à chaque élément de $E$, ce qui ne change pas le cardinal mais a l'avantage de simplifier l'ensemble.
		\[ |E| = |\{ 0; 1 ; \dots ; m-n \}. \]
	Il y a $m-n$ entiers de $1$ à $m-n$, et donc $|E| = m-n+1$ éléments au total en n'oubliant pas $0$.

	Pour $n=-1, m=1$, on a bien $m-n+1 = (1) - (-1) + 1  = 3$.
}

\end{document}

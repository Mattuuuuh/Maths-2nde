% DYSLEXIA SWITCH
\newif\ifdys
		
				% ENABLE or DISABLE font change
				% use XeLaTeX if true
				\dystrue
				\dysfalse


\ifdys

\documentclass[a4paper, 14pt]{extarticle}
\usepackage{amsmath,amsfonts,amsthm,amssymb,mathtools}

\tracinglostchars=3 % Report an error if a font does not have a symbol.
\usepackage{fontspec}
\usepackage{unicode-math}
\defaultfontfeatures{ Ligatures=TeX,
                      Scale=MatchUppercase }

\setmainfont{OpenDyslexic}[Scale=1.0]
\setmathfont{Fira Math} % Or maybe try KPMath-Sans?
\setmathfont{OpenDyslexic Italic}[range=it/{Latin,latin}]
\setmathfont{OpenDyslexic}[range=up/{Latin,latin,num}]

\else

\documentclass[a4paper, 12pt]{extarticle}

\usepackage[utf8x]{inputenc}
\usepackage{lmodern,textcomp}
\usepackage{amsmath,amsfonts,amsthm,amssymb,mathtools}

\fi


\usepackage[french]{babel}
\usepackage[
a4paper,
margin=2cm,
nomarginpar,% We don't want any margin paragraphs
]{geometry}
\usepackage{icomma}

\usepackage{fancyhdr}
\usepackage{array}
\usepackage{hyperref}

\usepackage{multicol, enumerate}
\newcolumntype{P}[1]{>{\centering\arraybackslash}p{#1}}


\usepackage{stackengine}
\newcommand\xrowht[2][0]{\addstackgap[.5\dimexpr#2\relax]{\vphantom{#1}}}

% theorems

\theoremstyle{plain}
\newtheorem{theorem}{Th\'eor\`eme}
\newtheorem*{sol}{Solution}
\theoremstyle{definition}
\newtheorem{ex}{Exercice}

% corps
\newcommand{\C}{\mathbb{C}}
\newcommand{\R}{\mathbb{R}}
\newcommand{\Rnn}{\mathbb{R}^{2n}}
\newcommand{\Z}{\mathbb{Z}}
\newcommand{\N}{\mathbb{N}}
\newcommand{\Q}{\mathbb{Q}}

% variance
\newcommand{\Var}[1]{\text{Var}(#1)}

% domain
\newcommand{\D}{\mathbb{D}}


% date
\usepackage{advdate}
\AdvanceDate[1]


% plots
\usepackage{pgfplots}

% table line break
\usepackage{makecell}

%subfigures
\usepackage{subcaption}

\definecolor{myg}{RGB}{56, 140, 70}
\definecolor{myb}{RGB}{45, 111, 177}
\definecolor{myr}{RGB}{199, 68, 64}


% SOLUTION SWITCH
\newif\ifsolutions
				\solutionstrue
				\solutionsfalse

\ifsolutions
	\newcommand{\exe}[2]{
		\begin{ex} #1  \end{ex}
		\begin{sol} #2 \end{sol}
	}
\else
	\newcommand{\exe}[2]{
		\begin{ex} #1  \end{ex}
	}
	
\fi

\begin{document}
\pagestyle{fancy}
\fancyhead[L]{Seconde 13}
\fancyhead[C]{\textbf{Statistiques 4 $\star$ (équations diophantiennes) \ifsolutions -- Solutions  \fi}}
\fancyhead[R]{\today}


\exe{[$\star$]

	Soient $a, b \in \N$ deux entiers naturels non tous deux nuls.
	On considère la série statistique suivante, dépendant des entiers $a$ et $b$.
	
		\begin{center}
		\begin{tabular}{|c|c|c|}\hline
			Valeur   & 9 & 12 \\ \hline
			Effectif & $a$ & $b$ \\ \hline
		\end{tabular}
		\end{center}
		
	\begin{enumerate}
		\item
		Montrer que, pour les couples $(a;b) = (2;1)$ et $(a;b) = (4;2)$, la moyenne de la série est de $10$.
		\item
		Montrer que, pour tous les couples de la forme $(a;b) = \kappa\cdot(2;1)$ où $\kappa\in\N$ est un entier naturel non nul, la moyenne de la série est de $10$.
		\item
		Représenter graphiquement ces couples dans un repère en prenant $\kappa=1; 2; 3; \dots$.
		Que dire des points ?
	\end{enumerate}
}{}


\exe{[$\star$]
	Soient $a, b \in \N$ deux entiers naturels tels que $(a;b) \neq (0;0)$.
	On considère la série statistique suivante, dépendant des entiers $a$ et $b$.
	
		\begin{center}
		\begin{tabular}{|c|c|c|}\hline
			Valeur   & 7 & 12 \\ \hline
			Effectif & $a$ & $b$ \\ \hline
		\end{tabular}
		\end{center}

	\begin{enumerate}
		\item
		Montrer que, pour les couples $(a;b) = (2;3)$ et $(a;b) = (4;6)$, la moyenne de la série est de $10$.
		\item
		Montrer que, si la moyenne de la série associée à un couple $(a;b)$ est de $10$, alors le couple vérifie
			\[ 2b = 3a. \]
		\item
		En déduire que l'entier naturel $a$ est nécessairement pair.
		(On peut raisonner par contraposition : si $a$ est impair, montrer que $3a$ est aussi impair).
		\item
		En écrivant $a=2\cdot k$ où $k\in\N$ est un entier naturel non nul, démontrer que $b=3\cdot k$. 
		\item
		Déduire que tous les couples $(a;b)$ dont la série associée est de moyenne $10$ sont de la forme
			\[ (a;b) = (2;3)\cdot k, \]
		où $k\in\N$ est un entier naturel non nul.
	\end{enumerate}
}{}


\exe{[$\star$]

	Soient $a, b \in \N$ deux entiers naturels tels que $(a;b) \neq (0;0)$.
	On considère la série statistique suivante, dépendant des entiers $a$ et $b$.
		\begin{center}
		\begin{tabular}{|c|c|c|c|}\hline
			Valeur   & 11 & 8 & 13 \\ \hline
			Effectif & 1 & $a$ & $b$ \\ \hline
		\end{tabular}
		\end{center}

	\begin{enumerate}
		\item
		Montrer que, pour les couples $(a;b) = (2;1)$ et $(a;b) = (5;3)$, la moyenne de la série est de $10$.
		\item
		Montrer que, si la moyenne de la série associée à un couple $(a;b)$ est de $10$, alors le couple vérifie
			\[ 2a - 3b = 1. \]
		\item
		En déduire que l'entier naturel $b$ est nécessairement impair.
		(On peut raisonner par contradiction : si $b$ est pair, montrer que l'égalité est impossible).
		\item
		En écrivant $b=2\cdot k + 1$ où $k\in\N$ est un entier naturel, démontrer que $a = 3k + 2$. 
		\item
		Déduire que tous les couples $(a;b)$ dont la série associée est de moyenne $10$ sont de la forme
			\[ (a;b) = (3;2)\cdot k + (2;1), \]
		où $k\in\N$ est un entier naturel.
		\item
		Représenter graphiquement ces couples dans un repère en prenant $k=1; 2; 3; \dots$.
		Que dire des points ?
	\end{enumerate}
		
}{}

\end{document}

% DYSLEXIA SWITCH
\newif\ifdys
		
				% ENABLE or DISABLE font change
				% use XeLaTeX if true
				\dystrue
				\dysfalse


\ifdys

\documentclass[a4paper, 14pt]{extarticle}
\usepackage{amsmath,amsfonts,amsthm,amssymb,mathtools}

\tracinglostchars=3 % Report an error if a font does not have a symbol.
\usepackage{fontspec}
\usepackage{unicode-math}
\defaultfontfeatures{ Ligatures=TeX,
                      Scale=MatchUppercase }

\setmainfont{OpenDyslexic}[Scale=1.0]
\setmathfont{Fira Math} % Or maybe try KPMath-Sans?
\setmathfont{OpenDyslexic Italic}[range=it/{Latin,latin}]
\setmathfont{OpenDyslexic}[range=up/{Latin,latin,num}]

\else

\documentclass[a4paper, 12pt]{extarticle}

\usepackage[utf8x]{inputenc}
%fonts
\usepackage{amsmath,amsfonts,amsthm,amssymb,mathtools}
% comment below to default to computer modern
\usepackage{libertinus,libertinust1math}

\fi


\usepackage[french]{babel}
\usepackage[
a4paper,
margin=2cm,
nomarginpar,% We don't want any margin paragraphs
]{geometry}
\usepackage{icomma}

\usepackage{fancyhdr}
\usepackage{array}
\usepackage{hyperref}

\usepackage{multicol, enumerate}
\newcolumntype{P}[1]{>{\centering\arraybackslash}p{#1}}


\usepackage{stackengine}
\newcommand\xrowht[2][0]{\addstackgap[.5\dimexpr#2\relax]{\vphantom{#1}}}

% theorems

\theoremstyle{plain}
\newtheorem{theorem}{Th\'eor\`eme}
\newtheorem*{sol}{Solution}
\theoremstyle{definition}
\newtheorem{ex}{Exercice}
\newtheorem*{rpl}{Rappel}
\newtheorem{enigme}{Énigme}

% corps
\usepackage{calrsfs}
\newcommand{\C}{\mathcal{C}}
\newcommand{\R}{\mathbb{R}}
\newcommand{\Rnn}{\mathbb{R}^{2n}}
\newcommand{\Z}{\mathbb{Z}}
\newcommand{\N}{\mathbb{N}}
\newcommand{\Q}{\mathbb{Q}}

% variance
\newcommand{\Var}[1]{\text{Var}(#1)}

% domain
\newcommand{\D}{\mathcal{D}}


% date
\usepackage{advdate}
\AdvanceDate[0]


% plots
\usepackage{pgfplots}

% table line break
\usepackage{makecell}
%tablestuff
\def\arraystretch{2}
\setlength\tabcolsep{15pt}

%subfigures
\usepackage{subcaption}

\definecolor{myg}{RGB}{56, 140, 70}
\definecolor{myb}{RGB}{45, 111, 177}
\definecolor{myr}{RGB}{199, 68, 64}

% fake sections with no title to move around the merged pdf
\newcommand{\fakesection}[1]{%
  \par\refstepcounter{section}% Increase section counter
  \sectionmark{#1}% Add section mark (header)
  \addcontentsline{toc}{section}{\protect\numberline{\thesection}#1}% Add section to ToC
  % Add more content here, if needed.
}


% SOLUTION SWITCH
\newif\ifsolutions
				\solutionstrue
				%\solutionsfalse

\ifsolutions
	\newcommand{\exe}[2]{
		\begin{ex} #1  \end{ex}
		\begin{sol} #2 \end{sol}
	}
\else
	\newcommand{\exe}[2]{
		\begin{ex} #1  \end{ex}
	}
	
\fi


% tableaux var, signe
\usepackage{tkz-tab}


%pinfty minfty
\newcommand{\pinfty}{{+}\infty}
\newcommand{\minfty}{{-}\infty}

\begin{document}


\AdvanceDate[0]

\begin{document}
\pagestyle{fancy}
\fancyhead[L]{Seconde 7}
\fancyhead[C]{\textbf{Évaluation — Ensembles de nombres}}
\fancyhead[R]{\today}

Consignes particulières : 
\begin{itemize}[label=$\bullet$]
	\item 
	La calculatrice est {interdite}.
	\item 
	Les exercices \ref{exe:1} et \ref{exe:2} peuvent être faits entièrement sur la feuille d'évaluation.
\end{itemize}

\hrule

\exemulticols{}{
	Compléter les pointillés en ajoutant un signe d'inclusion ($\subseteq$ ou $\supseteq$) entre les ensembles de nombres ci-contre.
}{
	\begin{center}
	\def\arraystretch{1.5}
	\setlength\tabcolsep{10pt}
	\begin{tabular}{ccc||ccc}
		$\N$ & \dots & $\Z$ & $\Z$ & \dots & $\N$ \\
		$\Z$ & \dots & $\Q$ & $\D$ & \dots & $\Q$
	\end{tabular}
	\end{center}
}{exe:1}{
	Solutions 1.
}

\exe{, difficulty=1}{
	Donner deux ensembles $A$ et $B$ vérifiants $A \not\subseteq B$ et $B \not\subseteq A$ : $A$ n'est pas un sous-ensemble de $B$, et $B$ n'est pas un sous-ensemble de $A$.
}{exe:1.1}{
	Solution 1.1.
}

\exe{}{
	Vrai ou faux ? Cocher la case correspondante.
	\begin{center}
	\begin{tabular}{c c c}
		\hspace{10cm} & Vrai & Faux \\
		$\dfrac12 = 0,5$ & $\square$ & $\square$  \\
		$\dfrac13 = 0,33$ & $\square$ & $\square$  \\
		$\dfrac13 = 0,33333$ & $\square$ & $\square$  \\
		$1$ est un nombre rationnel & $\square$ & $\square$  \\
		Tous les nombres sont rationnels & $\square$ & $\square$  \\
	\end{tabular}
	\end{center}
}{exe:2}{
	Solutions 2.
}

\exe{}{
	Donner les éléments de chaque ensemble suivant.
	\begin{multicols}{2}
	\begin{enumerate}[label=]
		\item $A=\bigset{n \in \N \text{ tel que } 3 \leq n \leq 8 }$
		\item $B = \bigset{a \in A \tq \text{$a$ est pair}}$
		\item $C = \bigset{n \in \Z \tq -1 \leq n \leq 1}$
		\item $D = \bigset{c \in C \tq \text{$c$ est non nul}}$
	\end{enumerate} 
	\end{multicols}
}{exe:3}{
	Solutions 3.
}

\exe{, difficulty=1}{
	Exprimer les nombres suivants sous forme de fraction d'entiers.
	\begin{multicols}{2}
	\begin{enumerate}[label=]
		\item $A = 0,666...$ ($6$ se répète à l'infini)
		\item $B = 0,555...$ ($5$ se répète à l'infini)
		\item $C = 0,020202...$ ($02$ se répète à l'infini)
		\item $D = 0,1545454...$ ($54$ se répète à l'infini)
	\end{enumerate} 
	\end{multicols}
}{exe:4}{
	Solutions 4.
}

\exe{}{
	Écrire les nombres suivants en notation scientifique.
	\begin{multicols}{3}
	\begin{enumerate}[label=\alph*)]
		\item 304
		\item 100
		\item 234 500 000
		\item 0,4
		\item 0,000 438
		\item 0,003 000 2
	\end{enumerate}
	\end{multicols}
}{exe:5}{
	Solutions 5.
	
}

\exe{, difficulty=2}{
	Soit $k\in\N$ non nul. 
	On suppose qu'aucune puissance de 10 n'est dans la table de multiplication de $k$.
	Montrer que $\frac1k$ n'est pas décimal.
	\\\\
	\emph{Toute trace de recherche sera prise en compte.}
}{exe:6}{
	Solutions 6.
}

%%%%%%%%%%%%

\newpage
\fancyhead[C]{\textbf{Solutions}}
\shipoutAnswer

\end{document}

%!TEX encoding = UTF8
%!TEX root =notes.tex


%%%%%%%%%%%%%%%%%%%%%%%%%%%%%%%%%
% PACKAGE IMPORTS
%%%%%%%%%%%%%%%%%%%%%%%%%%%%%%%%%


\usepackage[french]{babel}

\usepackage[tmargin=2cm,rmargin=1in,lmargin=1in,margin=0.85in,bmargin=2cm,footskip=.2in]{geometry}
\usepackage{amsmath,amsfonts,amsthm,amssymb,mathtools}
\usepackage[varbb]{newpxmath}
\usepackage{xfrac}
\usepackage[makeroom]{cancel}
\usepackage{mathtools}
\usepackage{bookmark}
\usepackage{enumitem}
\usepackage{hyperref,theoremref}
\hypersetup{
	pdftitle={Assignment},
	colorlinks=true, linkcolor=doc!90,
	bookmarksnumbered=true,
	bookmarksopen=true
}
\usepackage[most,many,breakable]{tcolorbox}
\usepackage{xcolor}
\usepackage{varwidth}
\usepackage{varwidth}
\usepackage{etoolbox}
%\usepackage{authblk}
\usepackage{nameref}
\usepackage{multicol,array}
\usepackage{tikz-cd}
\usepackage[ruled,vlined,linesnumbered]{algorithm2e}
\usepackage{comment} % enables the use of multi-line comments (\ifx \fi) 
\usepackage{import}
\usepackage{xifthen}
\usepackage{pdfpages}
\usepackage{transparent}


\newcommand\mycommfont[1]{\footnotesize\ttfamily\textcolor{blue}{#1}}
\SetCommentSty{mycommfont}
\newcommand{\incfig}[1]{%
    \def\svgwidth{\columnwidth}
    \import{./figures/}{#1.pdf_tex}
}

\usepackage{tikzsymbols}
%\renewcommand\qedsymbol{$\Laughey$}


%\usepackage{import}
%\usepackage{xifthen}
%\usepackage{pdfpages}
%\usepackage{transparent}


%%%%%%%%%%%%%%%%%%%%%%%%%%%%%%
% SELF MADE COLORS
%%%%%%%%%%%%%%%%%%%%%%%%%%%%%%



\definecolor{myg}{RGB}{56, 140, 70}
\definecolor{myb}{RGB}{45, 111, 177}
\definecolor{myr}{RGB}{199, 68, 64}
\definecolor{mytheorembg}{HTML}{F2F2F9}
\definecolor{mytheoremfr}{HTML}{00007B}
\definecolor{mylenmabg}{HTML}{FFFAF8}
\definecolor{mylenmafr}{HTML}{983b0f}
\definecolor{mypropbg}{HTML}{f2fbfc}
\definecolor{mypropfr}{HTML}{191971}
\definecolor{myexamplebg}{HTML}{F2FBF8}
\definecolor{myexamplefr}{HTML}{88D6D1}
\definecolor{myexampleti}{HTML}{2A7F7F}
\definecolor{mydefinitbg}{HTML}{E5E5FF}
\definecolor{mydefinitfr}{HTML}{3F3FA3}
\definecolor{notesgreen}{RGB}{0,162,0}
\definecolor{myp}{RGB}{197, 92, 212}
\definecolor{mygr}{HTML}{2C3338}
\definecolor{myred}{RGB}{127,0,0}
\definecolor{myyellow}{RGB}{169,121,69}
\definecolor{myexercisebg}{HTML}{F2FBF8}
\definecolor{myexercisefg}{HTML}{88D6D1}


%%%%%%%%%%%%%%%%%%%%%%%%%%%%
% TCOLORBOX SETUPS
%%%%%%%%%%%%%%%%%%%%%%%%%%%%

\setlength{\parindent}{1cm}
%================================
% THEOREM BOX
%================================

\tcbuselibrary{theorems,skins,hooks}
\newtcbtheorem[number within=chapter]{Theorem}{Théorème}
{%
	enhanced,
	breakable,
	colback = mytheorembg,
	frame hidden,
	boxrule = 0sp,
	borderline west = {2pt}{0pt}{mytheoremfr},
	sharp corners,
	detach title,
	before upper = \tcbtitle\par\smallskip,
	coltitle = mytheoremfr,
	fonttitle = \bfseries\sffamily,
	description font = \mdseries,
	separator sign none,
	segmentation style={solid, mytheoremfr},
}
{th}


\tcbuselibrary{theorems,skins,hooks}
\newtcolorbox{Theoremcon}
{%
	enhanced
	,breakable
	,colback = mytheorembg
	,frame hidden
	,boxrule = 0sp
	,borderline west = {2pt}{0pt}{mytheoremfr}
	,sharp corners
	,description font = \mdseries
	,separator sign none
}

%================================
% Corollery
%================================
\tcbuselibrary{theorems,skins,hooks}
\newtcbtheorem[use counter=tcb@cnt@Theorem]{Corollary}{Corollaire}
{%
	enhanced
	,breakable
	,colback = myp!10
	,frame hidden
	,boxrule = 0sp
	,borderline west = {2pt}{0pt}{myp!85!black}
	,sharp corners
	,detach title
	,before upper = \tcbtitle\par\smallskip
	,coltitle = myp!85!black
	,fonttitle = \bfseries\sffamily
	,description font = \mdseries
	,separator sign none
	,segmentation style={solid, myp!85!black}
}
{th}

%================================
% LENMA
%================================

\tcbuselibrary{theorems,skins,hooks}
\newtcbtheorem[use counter=tcb@cnt@Theorem]{Lemma}{Lemme}
{%
	enhanced,
	breakable,
	colback = mylenmabg,
	frame hidden,
	boxrule = 0sp,
	borderline west = {2pt}{0pt}{mylenmafr},
	sharp corners,
	detach title,
	before upper = \tcbtitle\par\smallskip,
	coltitle = mylenmafr,
	fonttitle = \bfseries\sffamily,
	description font = \mdseries,
	separator sign none,
	segmentation style={solid, mylenmafr},
}
{th}


%================================
% PROPOSITION
%================================

\tcbuselibrary{theorems,skins,hooks}
\newtcbtheorem[use counter=tcb@cnt@Theorem]{Prop}{Proposition}
{%
	enhanced,
	breakable,
	colback = mypropbg,
	frame hidden,
	boxrule = 0sp,
	borderline west = {2pt}{0pt}{mypropfr},
	sharp corners,
	detach title,
	before upper = \tcbtitle\par\smallskip,
	coltitle = mypropfr,
	fonttitle = \bfseries\sffamily,
	description font = \mdseries,
	separator sign none,
	segmentation style={solid, mypropfr},
}
{th}


%================================
% CLAIM
%================================

\tcbuselibrary{theorems,skins,hooks}
\newtcbtheorem[use counter=tcb@cnt@Theorem]{claim}{Claim}
{%
	enhanced
	,breakable
	,colback = myg!10
	,frame hidden
	,boxrule = 0sp
	,borderline west = {2pt}{0pt}{myg}
	,sharp corners
	,detach title
	,before upper = \tcbtitle\par\smallskip
	,coltitle = myg!85!black
	,fonttitle = \bfseries\sffamily
	,description font = \mdseries
	,separator sign none
	,segmentation style={solid, myg!85!black}
}
{th}



%================================
% Exercise
%================================

\tcbuselibrary{theorems,skins,hooks}
\newtcbtheorem[use counter=tcb@cnt@Theorem]{Exercise}{Exercice}
{%
	enhanced,
	breakable,
	colback = myexercisebg,
	frame hidden,
	boxrule = 0sp,
	borderline west = {2pt}{0pt}{myexercisefg},
	sharp corners,
	detach title,
	before upper = \tcbtitle\par\smallskip,
	coltitle = myexercisefg,
	fonttitle = \bfseries\sffamily,
	description font = \mdseries,
	separator sign none,
	segmentation style={solid, myexercisefg},
}
{th}

%================================
% EXAMPLE BOX
%================================

\newtcbtheorem[use counter=tcb@cnt@Theorem]{Example}{Exemple}
{%
	colback = myexamplebg
	,breakable
	,colframe = myexamplefr
	,coltitle = myexampleti
	,boxrule = 1pt
	,sharp corners
	,detach title
	,before upper=\tcbtitle\par\smallskip
	,fonttitle = \bfseries
	,description font = \mdseries
	,separator sign none
	,description delimiters parenthesis
}
{ex}

%================================
% DEFINITION BOX
%================================

\newtcbtheorem[use counter=tcb@cnt@Theorem]{Definition}{Définition}{enhanced,
	before skip=2mm,after skip=2mm, colback=red!5,colframe=red!80!black,boxrule=0.5mm,
	attach boxed title to top left={xshift=1cm,yshift*=1mm-\tcboxedtitleheight}, varwidth boxed title*=-3cm,
	boxed title style={frame code={
					\path[fill=tcbcolback]
					([yshift=-1mm,xshift=-1mm]frame.north west)
					arc[start angle=0,end angle=180,radius=1mm]
					([yshift=-1mm,xshift=1mm]frame.north east)
					arc[start angle=180,end angle=0,radius=1mm];
					\path[left color=tcbcolback!60!black,right color=tcbcolback!60!black,
						middle color=tcbcolback!80!black]
					([xshift=-2mm]frame.north west) -- ([xshift=2mm]frame.north east)
					[rounded corners=1mm]-- ([xshift=1mm,yshift=-1mm]frame.north east)
					-- (frame.south east) -- (frame.south west)
					-- ([xshift=-1mm,yshift=-1mm]frame.north west)
					[sharp corners]-- cycle;
				},interior engine=empty,
		},
	fonttitle=\bfseries,
	title={#2},#1}{def}

%================================
% Solution BOX
%================================

\makeatletter
\newtcbtheorem[use counter=tcb@cnt@Theorem]{question}{Question}{enhanced,
	breakable,
	colback=white,
	colframe=myb!80!black,
	attach boxed title to top left={yshift*=-\tcboxedtitleheight},
	fonttitle=\bfseries,
	title={#2},
	boxed title size=title,
	boxed title style={%
			sharp corners,
			rounded corners=northwest,
			colback=tcbcolframe,
			boxrule=0pt,
		},
	underlay boxed title={%
			\path[fill=tcbcolframe] (title.south west)--(title.south east)
			to[out=0, in=180] ([xshift=5mm]title.east)--
			(title.center-|frame.east)
			[rounded corners=\kvtcb@arc] |-
			(frame.north) -| cycle;
		},
	#1
}{def}
\makeatother

%================================
% SOLUTION BOX
%================================

\makeatletter
\newtcolorbox{solution}{enhanced,
	breakable,
	colback=white,
	colframe=myg!80!black,
	attach boxed title to top left={yshift*=-\tcboxedtitleheight},
	title=Solution,
	boxed title size=title,
	boxed title style={%
			sharp corners,
			rounded corners=northwest,
			colback=tcbcolframe,
			boxrule=0pt,
		},
	underlay boxed title={%
			\path[fill=tcbcolframe] (title.south west)--(title.south east)
			to[out=0, in=180] ([xshift=5mm]title.east)--
			(title.center-|frame.east)
			[rounded corners=\kvtcb@arc] |-
			(frame.north) -| cycle;
		},
}
\makeatother

%================================
% Question BOX
%================================

\makeatletter
\newtcbtheorem[use counter=tcb@cnt@Theorem]{qstion}{Question}{enhanced,
	breakable,
	colback=white,
	colframe=mygr,
	attach boxed title to top left={yshift*=-\tcboxedtitleheight},
	fonttitle=\bfseries,
	title={#2},
	boxed title size=title,
	boxed title style={%
			sharp corners,
			rounded corners=northwest,
			colback=tcbcolframe,
			boxrule=0pt,
		},
	underlay boxed title={%
			\path[fill=tcbcolframe] (title.south west)--(title.south east)
			to[out=0, in=180] ([xshift=5mm]title.east)--
			(title.center-|frame.east)
			[rounded corners=\kvtcb@arc] |-
			(frame.north) -| cycle;
		},
	#1
}{def}
\makeatother

\newtcbtheorem[number within=chapter]{wconc}{Wrong Concept}{
	breakable,
	enhanced,
	colback=white,
	colframe=myr,
	arc=0pt,
	outer arc=0pt,
	fonttitle=\bfseries\sffamily\large,
	colbacktitle=myr,
	attach boxed title to top left={},
	boxed title style={
			enhanced,
			skin=enhancedfirst jigsaw,
			arc=3pt,
			bottom=0pt,
			interior style={fill=myr}
		},
	#1
}{def}



%================================
% NOTE BOX
%================================

\usetikzlibrary{arrows,calc,shadows.blur}
\tcbuselibrary{skins}
\newtcolorbox{note}[1][]{%
	enhanced jigsaw,
	colback=gray!20!white,%
	colframe=gray!80!black,
	size=small,
	boxrule=1pt,
	title=\colorbox{white!100}{\textbf{ Remarque }},
	halign title=flush center,
	coltitle=black,
	breakable,
	drop shadow=black!50!white,
	attach boxed title to top left={xshift=1cm,yshift=-\tcboxedtitleheight/2,yshifttext=-\tcboxedtitleheight/2},
	minipage boxed title=2.6cm,
	boxed title style={%
			colback=white,
			size=fbox,
			boxrule=1pt,
			boxsep=2pt,
			underlay={%
					\coordinate (dotA) at ($(interior.west) + (-0.5pt,0)$);
					\coordinate (dotB) at ($(interior.east) + (0.5pt,0)$);
					\begin{scope}
						\clip (interior.north west) rectangle ([xshift=3ex]interior.east);
						\filldraw [white, blur shadow={shadow opacity=60, shadow yshift=-.75ex}, rounded corners=2pt] (interior.north west) rectangle (interior.south east);
					\end{scope}
					\begin{scope}[gray!80!black]
						\fill (dotA) circle (2pt);
						\fill (dotB) circle (2pt);
					\end{scope}
				},
		},
	#1,
}

%================================
% STRATÉGIE BOX
%================================

\usetikzlibrary{arrows,calc,shadows.blur}
\tcbuselibrary{skins}
\newtcolorbox{strategy}[1][]{%
	enhanced jigsaw,
	colback=myb!20!white,%
	colframe=gray!80!black,
	size=small,
	boxrule=1pt,
	title=\colorbox{white!100}{\textbf{ Stratégie }},
	halign title=flush center,
	coltitle=black,
	breakable,
	drop shadow=black!50!white,
	attach boxed title to top left={xshift=1cm,yshift=-\tcboxedtitleheight/2,yshifttext=-\tcboxedtitleheight/2},
	minipage boxed title=2.5cm,
	boxed title style={%
			colback=white,
			size=fbox,
			boxrule=1pt,
			boxsep=2pt,
			underlay={%
					\coordinate (dotA) at ($(interior.west) + (-0.5pt,0)$);
					\coordinate (dotB) at ($(interior.east) + (0.5pt,0)$);
					\begin{scope}
						\clip (interior.north west) rectangle ([xshift=3ex]interior.east);
						\filldraw [white, blur shadow={shadow opacity=60, shadow yshift=-.75ex}, rounded corners=2pt] (interior.north west) rectangle (interior.south east);
					\end{scope}
					\begin{scope}[gray!80!black]
						\fill (dotA) circle (2pt);
						\fill (dotB) circle (2pt);
					\end{scope}
				},
		},
	#1,
}

%================================
% MÉTHODE BOX
%================================

\usetikzlibrary{arrows,calc,shadows.blur}
\tcbuselibrary{skins}
\newtcolorbox{methode}[1][]{%
	enhanced jigsaw,
	colback=white,%
	colframe=gray!80!black,
	size=small,
	boxrule=1pt,
	title=\textbf{Méthode},
	halign title=flush center,
	coltitle=black,
	breakable,
	drop shadow=black!50!white,
	attach boxed title to top left={xshift=1cm,yshift=-\tcboxedtitleheight/2,yshifttext=-\tcboxedtitleheight/2},
	minipage boxed title=2.5cm,
	boxed title style={%
			colback=white,
			size=fbox,
			boxrule=1pt,
			boxsep=2pt,
			underlay={%
					\coordinate (dotA) at ($(interior.west) + (-0.5pt,0)$);
					\coordinate (dotB) at ($(interior.east) + (0.5pt,0)$);
					\begin{scope}
						\clip (interior.north west) rectangle ([xshift=3ex]interior.east);
						\filldraw [white, blur shadow={shadow opacity=60, shadow yshift=-.75ex}, rounded corners=2pt] (interior.north west) rectangle (interior.south east);
					\end{scope}
					\begin{scope}[gray!80!black]
						\fill (dotA) circle (2pt);
						\fill (dotB) circle (2pt);
					\end{scope}
				},
		},
	#1,
}

%%%%%%%%%%%%%%%%%%%%%%%%%%%%%%%%%%%%%%%%%%%
% TABLE OF CONTENTS
%%%%%%%%%%%%%%%%%%%%%%%%%%%%%%%%%%%%%%%%%%%

\usepackage{tikz}

\definecolor{doc}{RGB}{0,60,110}
\usepackage{titletoc}
\contentsmargin{0cm}
\titlecontents{chapter}[3.7pc]
{\addvspace{30pt}%
	\begin{tikzpicture}[remember picture, overlay]%
		\draw[fill=doc!60,draw=doc!60] (-7,-.1) rectangle (-0.2,.6);%
		\pgftext[left,x=-3.5cm,y=0.2cm]{\color{white}\Large\sc\bfseries Chapitre\ \thecontentslabel};%
	\end{tikzpicture}\color{doc!60}\large\sc\bfseries}%
{}
{}
{\;\titlerule\;\large\sc\bfseries Page \thecontentspage
	\begin{tikzpicture}[remember picture, overlay]
		\draw[fill=doc!60,draw=doc!60] (2pt,0) rectangle (4,0.1pt);
	\end{tikzpicture}}%
\titlecontents{section}[3.7pc]
{\addvspace{2pt}}
{\contentslabel[\thecontentslabel]{2pc}}
{}
{\hfill\small \thecontentspage}
[]
\titlecontents*{subsection}[3.7pc]
{\addvspace{-1pt}\small}
{}
{}
{\ --- \small\thecontentspage}
[ \textbullet\ ][]

\makeatletter
\renewcommand{\tableofcontents}{%
	\chapter*{%
	  \vspace*{-20\p@}%
	  \begin{tikzpicture}[remember picture, overlay]%
		  \pgftext[right,x=15cm,y=0.2cm]{\color{doc!60}\Huge\sc\bfseries \contentsname};%
		  \draw[fill=doc!60,draw=doc!60] (13,-.75) rectangle (20,1);%
		  \clip (13,-.75) rectangle (20,1);
		  \pgftext[right,x=15cm,y=0.2cm]{\color{white}\Huge\sc\bfseries \contentsname};%
	  \end{tikzpicture}}%
	\@starttoc{toc}}
\makeatother


%%%%%%%%%%%%%%%%%%%%%%%%%%%%%%%%%%%%%%%%%%%
% MINTED FOR PYTHON ALGORITHMS
%%%%%%%%%%%%%%%%%%%%%%%%%%%%%%%%%%%%%%%%%%%

\usepackage{tcolorbox}
\tcbuselibrary{minted,breakable,xparse,skins}
\definecolor{bg}{gray}{0.95}
\DeclareTCBListing{mintedbox}{O{}m!O{}}{%
  breakable=true,
  listing engine=minted,
  listing only,
  minted language=#2,
  minted style=default,
  minted options={%
    linenos,
    gobble=0,
    breaklines=true,
    breakafter=,,
    fontsize=\small,
    numbersep=8pt,
    #1},
  boxsep=0pt,
  left skip=0pt,
  right skip=0pt,
  left=25pt,
  right=0pt,
  top=3pt,
  bottom=3pt,
  arc=5pt,
  leftrule=0pt,
  rightrule=0pt,
  bottomrule=2pt,
  toprule=2pt,
  colback=bg,
  colframe=orange!70,
  enhanced,
  overlay={%
    \begin{tcbclipinterior}
    \fill[orange!20!white] (frame.south west) rectangle ([xshift=20pt]frame.north west);
    \end{tcbclipinterior}},
  #3}
  
  
 % for braces
\usetikzlibrary{decorations.pathreplacing}

\usepackage{minted}
\SetDate[27/01/2026]

\begin{document}
\pagestyle{fancy}
\fancyhead[L]{Seconde}
\fancyhead[C]{\textbf{Arithmétique}}
\fancyhead[R]{\today}

\exe{}{
	Lorsqu'on ouvre un livre, on peut identifier deux pages : celle de gauche, et celle de droite.
	Par exemple, la page 33 d'un livre est toujours une page de droite.
	
	Est-il possible que la page 330 soit également une page de droite ?
	
	Comment savoir si la page $n$ est de gauche ou de droite ? $n \in\N$ est un entier quelconque.
}{exe:parite-pages}{
	La page 33 d'un livre est une page droite. La prochaine page, la 34, est donc de gauche, et la suivante, la 35 est de droite.
	Les nombres pairs et impairs s'alternent : toutes les pages de droites sont donc impaires, et toutes les pages de gauche sont paires.
	Il suit que la page 330 ne peut pas être une page de droite car 330 est pair.
	
	Plus généralement, la page $n$ est de droite si $n$ est impair (non divisible par 2), et de gauche si $n$ est pair (divisible par 2).
}

\exe{}{
	Donner les chiffres pairs dont l'écriture en toutes lettres contient un nombre pair de lettres.
}{exe:pair-pair}{
	Il y a 5 chiffres pairs : 0, 2, 4, 6, et 8.
	Parmis eux, zéro, deux, quatre, et huit s'écrivent avec un nombre pair de lettres.
}

\exe{}{
	Donner les chiffres pairs dont l'écriture en toutes lettres respecte l'ordre alphabétique.
}{exe:pair-alph}{
	Il y a 5 chiffres pairs : 0, 2, 4, 6, et 8.
	Seul « deux » respecte l'ordre alphabétique.
}

\exe{, difficulty=1}{
	Soit $n\in\N$ un entier naturel.
	Montrer que $\frac{n(n+1)}2$ est toujours un entier.
}{exe:nnp1}{
	Comme $n$ et $n+1$ se suivent, l'un des deux est pair.
	Leur produit est donc pair, et la division par 2 donne un entier.
}

%\exe{, difficulty=1}{
%	Montrer que le carré d'un nombre impair prend la forme $4k + 1$ avec $k\in\N$ entier.
%}{exe:oddsquare}{
%	$(2n+1)^2 = 4n^2 + 4n + 1 = 4(n^2 + n) + 1 = 4k+1$.
%}

% sympa mais osef non ?
%\exe{, difficulty=1}{
%	On cherche les solutions entières de l'équation diophantienne $3x - 2y = 1$ où $x, y\in\Z$.
%	\begin{enumerate}
%		\item Montrer que $x = 2(y-x) + 1$ et en déduire que $x$ est impair.
%		\item En écrivant $x = 2k+1$ pour un entier $k\in\Z$, montrer que $y=3k+1$. Est-ce que $y$ est toujours entier ?
%		\item Prendre $k=35$ puis $k=-40$ et trouver les solution de $3x-2y=1$ associées.
%	\end{enumerate}
%}{exe:dioph}{
%	Soient $x, y\in\Z$ une solution entière de l'équation $3x-2y=1$.
%	\begin{enumerate}
%		\item 
%			\[ 3x - 2y = 1 \iff 3x = 2y + 1 \iff x + 2x = 2y + 1 \iff x = 2y-2x+1 \iff x = 2(y-x)+1. \]
%		Comme $x$ et $y$ sont entiers, la différence $y-x$ l'est aussi.
%		$x = 2(y-x)+1$ est donc un nombre impair.
%		\item 
%		$x$ étant impair, on peut l'écrire $x = 2k+1$ pour un entier $k\in\Z$. D'où
%			\[ 3x - 2y = 1 \iff 3(2k+1) - 2y = 1 \iff 6k + 3 - 2y = 1 \iff 6k + 2 = 2y \iff y = 3k+1. \]
%		$y = 3k+1$ est toujours entier car $k\in\Z$ l'est, et donc son triple aussi.
%		\item 
%		Pour $k=35$, $x=2k+1 = 71$ et $y=3k+1 = 106$. On vérifie que $3\times71 - 2\times106 = 1$.
%		
%		Pour $k=-40$, $x=2k+1 = -79$ et $y=3k+1 = -119$. On vérifie que $3\times(-79) - 2\times(-119) = 1$.
%	\end{enumerate}
%}

\hrule

\exe{}{
	Une tablette de chocolat classique comprend 12 carrés. Quelles sont les dimensions possibles de la tablette ? On cherche le nombre de carrés en longueur et en largeur.
}{exe:chocolat0}{
	Notons $a$ et $b$ les dimensions de la tablette, entiers naturels.
	Alors le nombre de carrés est donné par $a \times b = 12$.
	Par inspection, les seules solutions entières sont $12 = 1\times12 = 2\times6 = 3\times4$.
	
	Remarquons que $12 = 2\times6 = 6\times2$, donc les possibilités sont doublées pour les couples $(a ; b)$ car 12 n'est pas un carré parfait :
		\[ (a ; b) \in \Bigset{ (1 ; 12) ; (12; 1) ; (2 ; 6) ; (6 ; 2) ; (3 ; 4) ; (4 ; 3) }. \]
	L'ensemble précédent n'est pas un ensemble de nombres mais un ensemble de couples !
}


\exe{}{
	Donner $\D_1, \D_2, \D_3, \D_4, \D_5, \D_6, \D_7, \D_8,$ et $\D_9$.
}{exe:diviseurs}{
	\begin{multicols}{2}
	$\D_1 = \{ 1 \}$
	
	$\D_2 = \{ 1 ; 2\}$
	
	$\D_3 = \{ 1 ; 3\}$
	
	$\D_4 = \{ 1 ; 4 ; 2 \}$
	
	$\D_5 = \{ 1 ; 5\}$
	
	$\D_6 = \{ 1 ; 6 ; 2 ; 3\}$
	
	$\D_7 = \{ 1 ; 7\}$
	
	$\D_8 = \{ 1 ; 8 ; 2 ; 4\}$
	
	$\D_9 = \{ 1 ; 3 ; 9\}$
	\end{multicols}
}


\exe{, difficulty=1}{
	Donner $\D_{48}$ et en déduire les entiers naturels $x\in\N$ vérifiant $x^2 - 8x= 48$.
	
	\emph{Indication : montrer d'abord que $x^2 - 8x = x(x-8)$.}
}{exe:2nd-degré-Z}{
	$\D_{48} = \bigset{1 ; 48 ; 2 ; 24 ; 3 ; 16 ; 4 ; 12 ; 6 ; 8}$.

	Soit $x\in\N$ vérifiant $x^2 - 8x = 48 \iff x(x-8) = 48$.
	Comme $x\in\N$ est entier, $x-8$ l'est aussi, et les deux nombres divisent 48.
	
	On cherche donc deux éléments de $\D_{48}$ dont le produit vaut 48 et qui sont séparés de $8$ unités.
	Par inspection, seul $4\times12 = 48$ et donc $x=12$ fonctionne, et c'est la seule solution dans $\N$ de l'équation.
	
	Par symétrie, il existe une autre solution, $x = -4$, qui appartient à $\Z$ mais pas à $\N$.
}

\hrule



\exe{}{
	Une tablette de chocolat classique comprend 11 carrés. Est-ce possible ?
}{exe:chocolat}{
	C'est possible si les dimensions sont $1\times11$ ; c'est un Toblerone !
}


\exe{}{
	Donner les 5 plus petits nombres premiers.
}{exe:premiers}{
	D'après l'exercice \ref{exe:diviseurs}, on sait que 2, 3, 5, et 7 sont premiers, et que 4, 6, 8 ne le sont pas.
	On énumère $\D_9 = \bigset{ 1 ; 9 ; 3 }$, $\D_{10} = \bigset{ 1 ; 10 ; 2 ; 5 }$, puis $\D_{11} = \bigset{ 1 ; 11 }$, ce qui nous donne le 5è premiers.
	
	\[ \P = \bigset{ 2 ; 3 ; 5 ; 7 ; 11 ; 13 ; 17 ; 19 ; 23 ; 29 ; 31 ; \dots }. \]
}


\exe{}{
	Parmis les premiers de l'exercice \ref{exe:premiers}, lesquels admettent un nombre premier de lettres lorsque écrits en toutes lettres ?
}{exe:premier-premier}{
	Parmis $\bigset{ 2 ; 3 ; 5 ; 7 ; 11 }$, seul « trois » s'écrit avec 5 lettres, le reste des éléments s'écrivant avec 4 lettres.
	Les prochains sont dix-sept et dix-neuf, qui contiennent 7 lettres chacun.
	Question ouverte : en existe-t-il une infinité ?
}



\exe{}{
	Donner la décomposition en facteurs premiers des éléments de $\D_{24}$ (excepté 1).
	Que remarque-t-on ?
}{exe:décompositions-primaires}{
	On a $\D_{24} = \bigset{ 1 ; 24 ; 2 ; 12 ; 3 ; 8 ; 4 ; 6 }$, qui se décomposent comme suit.
	\begin{multicols}{2}
		$2 = 2$ est déjà premier.
		
		$3=3$ est déjà premier.
		
		$4 = 2 \times 2 = 2^2$.
		
		$6 = 2\times3$.
		
		$8 = 2\times4 = 2\times2\times2 = 2^3$.
		
		$12 = 3 \times 4 = 2^2 \times 3$.
		
		$24 = 4 \times 6 = 2^2 \times 2 \times 3 = 2^3 \times 3$.
	\end{multicols}
}


\exe{, difficulty=1}{
	Sachant que $12^2 = 144$, est-ce que 7 divise 144 ?
}{exe:7div144}{
	Comme 7 est premier, il s'agit donc de savoir s'il appartient à la décomposition primaire 144 ou non, par unicité de la décomposition.
	Or, la décomposition de 12 est $12 = 2^2 \times 3$.
	Ainsi, la décomposition de $12^2$ est $12^2 = \bigl(2^2 \times 3 \bigr)^2 = 2^4 \times 3^2$.
	7 ne divise donc pas 144 car il n'appartient pas à sa décomposition.
}


\exe{}{
	À l'aide de la décomposition $2160 = 2^4 \times 3^3 \times 5$, écrire 2160 sous la forme $a^2 \times b$, où $a, b\in\N$ entiers et $b$ est le plus petit possible.
	
	Écrire $\sqrt{2160}$ sous la forme $a\sqrt{b}$ avec $a, b\in\N$ entier, et $b$ le plus petit possible.
}{exe:sq-factor}{
	On crée $a$ en extrayant le plus de carrés possible pour obtenir le $b$ le plus petit.
	$2160 = \bigl(2^2 \times 3\bigr)^2 \times 3 \times 5 = 12^2 \times 15$.
	
	Par conséquent, $\sqrt{2160} = \sqrt{12^2 \times 15} = 12\sqrt{15}$.
}

\exe{}{
	Exprimer 6~750 sous la forme $a^2 \times b$, où $a, b\in\N$ et $b$ est le plus petit possible. 
	
	Écrire $\sqrt{6750}$ sous la forme $a\sqrt{b}$ avec $a, b\in\N$ entier, et $b$ le plus petit possible.
}{exe:6750-factor}{
	On décompose d'abord $6~750 = 10 \times 675 = 10 \times 135 \times 5 = 10 \times 27 \times 5^2 = 2 \times 3^3 \times 5^3$.
	D'où $6~750 = \bigl(3\times5)^2 \times 2 \times3\times5 = 15^2 \times 30$.
	
	Par conséquent, $\sqrt{6750} = \sqrt{15^2 \times 30} = 15\sqrt{30}$.
}

% TODO: plus d'exes avec les puissances. Genre décomposer 15^7
\exe{}{
	Décomposer les nombres suivants en produit de facteurs premiers.
	\begin{multicols}{3}
	\begin{enumerate}[label=\roman*)]
		\item  $64 $
		\item $25$
		\item $81$
		\item  $6 \times 121 \times 64$
		\item $45 \times 125 \times 9$
		\item $10^2 \times 15^3$
	\end{enumerate}
	\end{multicols}
}{exe:old-div1}{

	\begin{multicols}{3}
	\begin{enumerate}[label=\roman*)]
		\item  $64 = 2^6 $
		\item $25 = 5^2$
		\item $81 = 9^2$
		\item  $6 \times 121 \times 64 = 2^7 \times 3 \times 11^2$
		\item $45 \times 125 \times 9 = 3^4 \times 5^4$
		\item $10^2 \times 15^3 = 2^2 \times 3^3 \times 5^5$
	\end{enumerate}
	\end{multicols}


}


\exe{}{
	On considère le nombre $n=2^6 \times 3^2 \times 7^7$.
	Quelles sont les affirmations exactes ? Justifier.
	\begin{multicols}{2}
	\begin{enumerate}[label=\alph*)]
		\item $2^3 \times 3^3$ divise $n$
		\item $2^3 \times 7^3$ divise $n$
		\item $2^3 \times 7$ divise $n$
		\item $3^2 \times 7^7$ divise $n$
	\end{enumerate}
	\end{multicols}
}{exe:old-div2}{

	\textbf{Remarque importante} : si $d$ divise $n$, on peut écrire 
		\[ n = d \times k, \]
	pour $k \in \N$ un entier naturel.
	
	En décomposant $d$ et $k$ en produit de nombres premiers, on trouve la décomposition de $n$ (voir exercice \ref{exe:old-div1} par exemple).
	Ainsi, si un nombre premier apparaît dans la décomposition de $d$, il doit apparaître au moins autant de fois dans la décomposition de $n$.
	

	\begin{enumerate}[label=\alph*)]
		\item $2^3 \times 3^3$ ne divise pas $n$ car $3$ n'apparaît que $2$ fois dans la décomposition de $n$.
		\item $2^3 \times 7^3$ divise $n$ car $n = (2^3 \times 7^3) \times (2^3 \times 3^2 \times 7^4)$.
		\item $2^3 \times 7$ divise $n$ car $n = (2^3 \times 7) \times ( 2^3 \times 3^2 \times 7^6)$.
		\item $3^2 \times 7^7$ divise $n$ car $n=(3^2 \times 7^7) \times 2^6 $.
	\end{enumerate}
}

% à faire en classe, ils sauront pas (trop nuls)
%\exe{}{
%	Soit $n \in \N$ un entier naturel.
%	\begin{enumerate}
%		\item Écrire la décomposition en produit de facteurs premiers de $10^n$.
%		\item Démontrer par l'absurde que le rationnel $\frac{5}{12} \in \Q$ n'est pas un nombre décimal.
%		
%		\emph{Rappel : $\D = \Bigset{ \dfrac{a}{10^n} \tq a\in\Z, n \in\N }$, les nombres décimaux.}
%	\end{enumerate}
%}{exe:old-div3}{
%	\begin{enumerate}
%		\item $10^n = 2^n \times 5^n$.
%		\item On reprend la preuve par l'absurde vue en cours jusqu'à trouver deux entiers $a\in\Z$, $n\in\N$ tels que
%			\[ \dfrac{a}{10^n} = \dfrac{5}{12}. \]
%		En manipulant l'égalité on trouve
%			\[ 3 \times 4 a = 5 \times 10^n = 2^n 5^{n+1}. \]
%		Ainsi le premier $3$ apparaît à gauche mais pas à droite, ce qui contredit l'unicité du théorème de décomposition vu en cours (théorème fondamental de l'arithmétique).
%	\end{enumerate}
%}

%\hrule

\newpage

\exe{}{
	\begin{enumerate}
		\item Donner les diviseurs \emph{communs} à $6$ et à $15$.
		\item Rendre la fraction $\dfrac{6}{15}$ irréductible.
	\end{enumerate}
}{exe:old-div4}{

	\begin{enumerate}
		\item $\mathcal{D}_6 \cap \mathcal{D}_{15} = \{ 1, 3 \}$.
		\item $\dfrac{6}{15} = \dfrac{3 \times 2}{3 \times 5} = \dfrac25$, irréductible car $2$ et $5$ sont premiers entre eux. 
	\end{enumerate}
	
}


\exe{}{
	\begin{enumerate}
		\item Donner les multiples \emph{communs}  à $10$ et $35$ inférieurs ou égaux à $100$.
		\item Calculer la somme $\dfrac{3}{10} + \dfrac{6}{35}$ puis la rendre irréductible.
	\end{enumerate}
}{exe:old-div5}{
	\begin{enumerate}
		\item $\mathcal{M}_{10} \cap \mathcal{M}_{35} = \{ 0, 70 \}$.
		\item  $\dfrac{3}{10} + \dfrac{6}{35} = \dfrac{21}{70} + \dfrac{12}{70} = \dfrac{33}{70}$, irréductible car $33$ et $70$ sont premiers entre eux.
	\end{enumerate}
	
	\textbf{Remarque importante} : un diviseur de $33 = 3 \times 11$ admet uniquement $3$ et $11$ dans sa décomposition en produit de nombres premiers.
	Idem pour $70 = 2 \times 5 \times 7$, un diviseur admet uniquement $2, 5$, et $7$ dans sa décomposition.
	
	Donc le seul diviseur commun à $33$ et $70$ est $1$.
}

\exe{}{
	Exprimer chaque valeur suivante sous forme de fraction irréductible.
	\begin{multicols}{4}
	\begin{enumerate}[label=\alph*), itemsep=10pt]
		\item $1-\dfrac{4}{44}$
		\item $\dfrac{15}{27}$
		\item $\dfrac{8}{3} + 6$
		\item $\dfrac{65}{5^2}$
		\item $\dfrac{6}{32}\times\dfrac{2}{3}$
		\item $\dfrac{38}{20}-3$
		\item $\left(\dfrac{10}{44}\right)^{-1}$
		\item $\dfrac{6}{18}\times2 + 1$
		\item $\left(\dfrac{66}{99}\right)^2$
		\item $\dfrac{23}{18} + \dfrac12$
		\item $\dfrac{34}{20}$
		\item $2 - \dfrac{9}{5}$
	\end{enumerate}
	\end{multicols}
}{exe:frac-irr}{
	\begin{multicols}{4}
	\begin{enumerate}[label=\alph*), itemsep=10pt]
		\item $\dfrac{10}{11}$
		\item $\dfrac{5}{9}$
		\item $\dfrac{26}{3}$
		\item $\dfrac{13}{5}$
		\item $\dfrac{1}{18}$
		\item $\dfrac{-11}{10}$
		\item $\dfrac{22}{5}$
		\item $\dfrac{5}{3}$
		\item $\dfrac{4}{9}$
		\item $\dfrac{16}{9}$
		\item $\dfrac{17}{10}$
		\item $\dfrac{1}{5}$
	\end{enumerate}
	\end{multicols}
}

%\hrule

\subsection*{Exercices d'approfondissement}

% pour l'éval ?
%\exe{, difficulty=1}{
%	On cherche à résoudre l'équation cubique suivante, où $x\in\Z$ est entier :
%		\[ x^3 + 3x^2 - x - 108 = 0. \]
%	\begin{enumerate}
%		\item
%		Montrer que l'équation est équivalente à 
%			\[  (x+1)(x-1)(x+3) = 105. \]
%		\item	
%		Donner la décomposition en facteurs premiers de 105.
%		\item
%		En déduire toutes les valeurs $x\in\Z$ solutions de l'équation originale.
%	\end{enumerate}
%}{exe:div105}{
%	todo
%}

%
%\exe{, difficulty=2}{
%	Le but de cet exercice est de trouver une forme close pour la somme des $n$ premiers entiers naturels $1+2+3+4+\cdots + n$, nombre appelé \emph{nombre triangulaire}.
%	\begin{enumerate}
%		%\item
%		%Quelles hypothèses poser sur $n$ ? Dans quel ensemble de nombres appartient-il ?
%		\item
%		Montrer que la différence de deux carrés parfaits consécutifs est un nombre impair.
%		\item
%		Montrer qu'en ajoutant les différences $1^2-0^2$, $2^2 - 1^2, 3^2 - 2^2, \dots, (n+1)^2 - n^2$, on obtient $(n+1)^2$.
%		\item
%		En déduire que
%			\[ 1 + 3 + 5 + \cdots + (2n+1) = (n+1)^2. \]
%		\item
%		Conclure que 
%			\[ 1 + 2 + 3 + \cdots + n = \dfrac{n(n+1)}2. \]
%		\emph{Indication : soustraire $n+1$ des deux côtés de l'équation obtenue à la question 3.}
%	\end{enumerate}
%}{exe:star1}{
%	
%	\begin{enumerate}
%		\item
%			\[ (n+1)^2 - n^2 = n^2 + 2n + 1 - n^2 = 2n+1. \]
%		\item
%		En ajoutant les différences l'une après l'autre, on remarque que chaque nouveau terme annule la somme partielle qui devient un carré parfait.
%		En réordonnant les termes, on effectue en fait la somme
%			\[ 1^2 + 2^2 +3^2 + \cdots + (n+1)^2 - 0^2 - 1^2 - 2^2 - \cdots - n^2 = (n+1)^2 - 0^2. \]
%		\item
%		Chaque différence est un nouveau nombre impair d'après la première question : $(k+1)^2 - k^2 = 2k+1$, avec $k$ allant de 0 à $n$.
%		\item
%		En soustrayant 1 à chaque terme de la somme de gauche, nous soustrayons $n+1$ au total (les termes allant de $2\times0+1$ à $2n+1$, il y en a autant que d'éléments de $\bigset{0 ; 1 ; \cdots ; n}$, soit $n+1$).
%		Par conséquent,
%			\[ 0 + 2 + 4 + 6 + \cdots + 2n = (n+1)^2 - (n+1) = (n+1)(n+1-1) = n(n+1). \]
%		Diviser par deux conclut.
%	\end{enumerate}
%
%	\underline{Discussion}
%	
%	La somme de différences successives s'appelle \emph{somme téléscopique}.
%	
%	En considérant les différences successives de cubes, on peut montrer que
%		\[ 1 + 4 + 9 + \cdots + n^2 = \dfrac{n(n+1)(2n+1)}6. \]
%	On obtient d'ailleurs que $n(n+1)(2n+1)$ est toujours un multiple de 6 (résultat qu'on peut démontrer de façon plus simple en étudiant la parité et la divisibilité par 3).
%	
%	Plus généralement, la somme des $k^d$ de $k=1$ à $n$ donne une expression commençant par $\frac1{d+1}n^{d+1}$.
%	Sommer fait gagner un ordre ; phrase dont on se rappelera lors de l'étude de l'intégration ! (\emph{cf}. théorème fondamental de l'analyse)
%}


\exe{, difficulty=2}{
	On appelle \emph{nombre de Mersenne}\footnotemark 
	un nombre prenant la forme $2^n - 1$ avec $n\in\N$.
	\begin{enumerate}
		\item
		Donner les 15 plus petits nombres de Mersenne.
		\item
		Parmis ces nombres, lesquels sont premiers ? L'utilisation d'un ordinateur est pertinente.
		\item
		Que dire des exposants $n$ lorsque $2^n -1$ est premier ?
		\item
		Montrer que pour tout $q\in\R$ et $d\geq1$ entier,
			\[ (q-1)\left(1+q+q^2 + q^3 + \cdots + q^{d-1}\right) = q^d - 1. \]
		\item
		En déduire que si $n$ n'est pas premier, alors $2^n -1$ n'est pas premier non plus. 
	\end{enumerate}
}{exe:Mersenne}{
	À rendre pour possibilité de points bonus à la prochaine évaluation.
	Ni correction ni points ne seront attribués à un travail suspect ou ne démontrant pas une bonne compréhension de l'exercice.
	
	Voici un programme Python permettant de trouver les diviseurs d'un nombre.
	\python{diviseurs}
}
\footnotetext{Marin Mersenne (1588-1648), physicien, mathématicien, musicologue et philosophe français.}

\exe{, difficulty=2}{
	Ce exercice traite du \emph{théorème des deux carrés} de Fermat\footnotemark.
	On dit que $n\in\N$ est somme de deux carrés si $n=a^2 + b^2$ avec $a, b \in \N$ entiers.
	\begin{enumerate}
		\item 
		Montrer que 2 et 4 sont somme de deux carrés, mais que 3 et 7 ne le sont pas.
		%\item 
		%Montrer qu'un nombre de la forme $4k+3$ où $k\in\N$ ne peut pas être somme de deux carrés.
		%
		%\emph{Indication : justifier qu'un entier est nécessairement de la forme $4k, 4k+1, 4k+2,$ ou $4k+3$ et montrer que seules les deux premières formes sont possibles pour son carré.}
		\item\label{q:3}
		Développer $(ax - by)^2 + (ay + bx)^2$ et en déduire que si $m$ et $n$ sont somme de deux carrés, alors le produit $mn$ aussi.
		\item\label{q:4}
		Décomposer 85 en facteurs premiers, écrire chaque facteur comme somme de deux carrés, et en déduire l'écriture de 85 comme somme de deux carrés à l'aide de la question \ref{q:3}.
		\item
		Décomposer $12~240 = a^2 \times b$ avec $a, b$ entiers et $b$ le plus petit possible.
		L'écrire comme somme de deux carrés à l'aide des questions \ref{q:3} et \ref{q:4}.
	\end{enumerate}
}{exe:star2}{
	À rendre pour possibilité de points bonus à la prochaine évaluation.
	Ni correction ni points ne seront attribués à un travail suspect ou ne démontrant pas une bonne compréhension de l'exercice.
	
	\underline{Discussion}
	
	%La question 2 deviendra évidente lors de l'étude de l'arithmétique modulaire (programme de Terminale).
	%Essentiellement, les opérations arithmétiques restent cohérentes lorsqu'on étudie uniquement les restes après division euclidienne par 4. 
	%En particulier, le reste du carré est le carré du reste (propriété à montrer !).
	%Les seuls restes possibles pour un carré sont donc $0^2 = 0, 1^2 = 1, 2^2 = 4,$ de reste $0$ et $3^2 = 9$, de reste 1. 
	%On voit donc que les seuls restes 0 et 1 sont possibles.
	
	Nous avons montré que si chaque premier $p$ divisant $n$ et apparaissant dans sa décomposition avec une puissance impaire est somme de deux carrés, alors $n$ est somme de deux carrés (les puissances paires étant déjà des carrés).
	
	Le théorème des deux carrés de Fermat énonce que la direction inverse est également vraie.
	De plus, il énonce que si un premier $p$ n'est pas de la forme $4k+3$ (donc $p=2$ ou $p=4k+1$), alors $p$ est somme de deux carrés.
}
\footnotetext{Pierre de Fermat (1607 - 1665), magistrat et mathématicien français.}



%%%%%%%%%%%

\newpage
\fancyhead[C]{\textbf{Solutions}}
\shipoutAnswer

\end{document}

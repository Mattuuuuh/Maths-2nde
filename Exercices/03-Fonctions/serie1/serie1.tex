% DYSLEXIA SWITCH
\newif\ifdys
		
				% ENABLE or DISABLE font change
				% use XeLaTeX if true
				\dystrue
				\dysfalse


\ifdys

\documentclass[a4paper, 14pt]{extarticle}
\usepackage{amsmath,amsfonts,amsthm,amssymb,mathtools}

\tracinglostchars=3 % Report an error if a font does not have a symbol.
\usepackage{fontspec}
\usepackage{unicode-math}
\defaultfontfeatures{ Ligatures=TeX,
                      Scale=MatchUppercase }

\setmainfont{OpenDyslexic}[Scale=1.0]
\setmathfont{Fira Math} % Or maybe try KPMath-Sans?
\setmathfont{OpenDyslexic Italic}[range=it/{Latin,latin}]
\setmathfont{OpenDyslexic}[range=up/{Latin,latin,num}]

\else

\documentclass[a4paper, 12pt]{extarticle}

\usepackage[utf8x]{inputenc}
%fonts
\usepackage{amsmath,amsfonts,amsthm,amssymb,mathtools}
% comment below to default to computer modern
\usepackage{libertinus,libertinust1math}

\fi


\usepackage[french]{babel}
\usepackage[
a4paper,
margin=2cm,
nomarginpar,% We don't want any margin paragraphs
]{geometry}
\usepackage{icomma}

\usepackage{fancyhdr}
\usepackage{array}
\usepackage{hyperref}

\usepackage{multicol, enumerate}
\newcolumntype{P}[1]{>{\centering\arraybackslash}p{#1}}


\usepackage{stackengine}
\newcommand\xrowht[2][0]{\addstackgap[.5\dimexpr#2\relax]{\vphantom{#1}}}

% theorems

\theoremstyle{plain}
\newtheorem{theorem}{Th\'eor\`eme}
\newtheorem*{sol}{Solution}
\theoremstyle{definition}
\newtheorem{ex}{Exercice}
\newtheorem*{rpl}{Rappel}
\newtheorem{enigme}{Énigme}

% corps
\usepackage{calrsfs}
\newcommand{\C}{\mathcal{C}}
\newcommand{\R}{\mathbb{R}}
\newcommand{\Rnn}{\mathbb{R}^{2n}}
\newcommand{\Z}{\mathbb{Z}}
\newcommand{\N}{\mathbb{N}}
\newcommand{\Q}{\mathbb{Q}}

% variance
\newcommand{\Var}[1]{\text{Var}(#1)}

% domain
\newcommand{\D}{\mathcal{D}}


% date
\usepackage{advdate}
\AdvanceDate[0]


% plots
\usepackage{pgfplots}

% table line break
\usepackage{makecell}
%tablestuff
\def\arraystretch{2}
\setlength\tabcolsep{15pt}

%subfigures
\usepackage{subcaption}

\definecolor{myg}{RGB}{56, 140, 70}
\definecolor{myb}{RGB}{45, 111, 177}
\definecolor{myr}{RGB}{199, 68, 64}

% fake sections with no title to move around the merged pdf
\newcommand{\fakesection}[1]{%
  \par\refstepcounter{section}% Increase section counter
  \sectionmark{#1}% Add section mark (header)
  \addcontentsline{toc}{section}{\protect\numberline{\thesection}#1}% Add section to ToC
  % Add more content here, if needed.
}


% SOLUTION SWITCH
\newif\ifsolutions
				\solutionstrue
				%\solutionsfalse

\ifsolutions
	\newcommand{\exe}[2]{
		\begin{ex} #1  \end{ex}
		\begin{sol} #2 \end{sol}
	}
\else
	\newcommand{\exe}[2]{
		\begin{ex} #1  \end{ex}
	}
	
\fi


% tableaux var, signe
\usepackage{tkz-tab}


%pinfty minfty
\newcommand{\pinfty}{{+}\infty}
\newcommand{\minfty}{{-}\infty}

\begin{document}


\SetDate[04/11/2025]

\begin{document}
\pagestyle{fancy}
\fancyhead[L]{Seconde}
\fancyhead[C]{\textbf{Fonctions 1}}
\fancyhead[R]{\today}

\exe{, difficulty=1}{
	Un lynx prend la fuite : il court à 15 mètres par seconde.
	\begin{enumerate}
		\item La distance parcourue par le lynx est-elle une fonction du temps écoulé depuis son départ ?
		\item Le temps écoulé depuis le départ du lynx est-il une fonction de la distance parcourue ?
	\end{enumerate}
}{exe:lynx}{
	\begin{enumerate}
		\item 
		Oui car à chaque temps est associé \textbf{une unique} distance.
		\item 
		Oui également, car à chaque distance est associé \textbf{un unique} temps.
	\end{enumerate}
}

\exe{, difficulty=1}{
	Un étudiant jette une balle dans les airs et mesure la hauteur de la balle tous les quarts de seconde.
	Il note ses résultats dans le tableau ci-dessous.
	\begin{center}
		\def\arraystretch{1.5}
		%\setlength\tabcolsep{20pt}
		\begin{tabular}{|c|c|c|c|c|c|c|c|}\hline
			Hauteur (cm) & 85 & 145 & 190 & 145 & 85 & 40 & 0 \\ \hline
			Temps (s) & 0 & 0,25 & 0,5 & 0,75 & 1 & 1,25 & 1,5 \\\hline
		\end{tabular}
	\end{center}
	
	\begin{enumerate}
		\item La hauteur est-elle une fonction du temps ? Justifier.
		\item Le temps est-il une fonction de la hauteur ? Justifier.
	\end{enumerate}
}{exe:f1}{
	\begin{enumerate}
		\item On peut associer \textbf{une seule} hauteur à chaque temps. La hauteur peut donc être vue comme fonction du temps.
		\item On ne peut pas associer un unique temps à chaque hauteur. Par exemple, il y a deux temps distincts pour lesquels la hauteur est de $85$cm (0s et 1s).
		Le temps ne peut donc pas être vu comme fonction de la hauteur.
	\end{enumerate}
}
	
	
\exe{, difficulty=2}{
	\begin{enumerate}
		\item
		Montrer que le rayon $R$ d'un cercle est fonction de son périmètre $P$ et écrire la fonction $\text{Rayon}(P)$ associée.
		\item
		Montrer que l'aire $A$ d'un cercle est fonction de son rayon $R$ et écrire la fonction $\text{Aire}(R)$ associée.
		\item
		En déduire que l'aire $A$ d'un cercle est fonction de son périmètre $P$ et écrire la fonction $\text{Aire}(P)$ associée.
	\end{enumerate}
}{exe:f2}{
	\begin{enumerate}
		\item
		De la relation 
			\[ P = 2\pi \cdot R, \]
		on déduit que 
			\[ R = \dfrac{1}{2\pi} \cdot P. \]
		Remarquons qu'on a inversé le membre de gauche et le membre de droite pour extraire la forme d'une fonction : une valeur de $P$ donne une unique valeur de $R$.
		
		Ainsi on peut voir $R$ comme une fonction de $P$. On note alors $R=R(P)$, qui vérifie
		\[ R(P) = \dfrac{1}{2\pi} \cdot P. \]
		
		\item 
		La formule de l'aire
			\[ A = \pi \cdot R^2 \]
		exprime directement l'aire comme fonction du rayon : à chaque rayon possible, on peut calculer une unique aire.
		On note alors
			\[ A(R) =  \pi \cdot R^2. \]
		\item
		On souhaite exprimer l'aire $A$ comme fonction du périmètre $P$.
		Or d'après les questions précédentes, l'aire $A$ est fonction du rayon $R$, et le rayon $R$ est lui-même fonction du périmètre $P$.
		
		Ainsi, une valeur de $P$ donne une unique valeur de $R$ qui donne une unique valeur de $A$.
		En suivant le raisonnement, on peut composer les fonctions trouvées ci-dessus en considérant $A(R(P))$.
		Pour ne pas surcharger les notations, appelons plutôt $\text{Aire}$ la fonction prenant un périmètre.
			\begin{align*}
			 \text{Aire}(P) &= A\bigl(R(P)\bigr) \\ &=A\left(\dfrac{1}{2\pi} \cdot P\right) \\ &= \pi \left( \dfrac{1}{2\pi} \cdot P \right)^2 \\ &= \dfrac{\pi}{4\pi^2} \cdot P^2 = \dfrac{1}{4\pi} \cdot P^2.
			 \end{align*}
	\end{enumerate}
}
	
	
\exe{}{
	Considérons la fonction $f$ donnée {algébriquement} par
	\begin{align*}
		f(x) = 3x + 1
	\end{align*}
	pour tout $x\in\R$.
	
	\begin{enumerate}
		\item
		Calculer l'image par $f$ de $0$ ; de $3,1$ ; de $\frac13$ ; de $-1$ ; de $-\frac23$.
		\item
		Donner un antécédent de $1$ par $f$.
		\item
		Déterminer tous les antécédents de $6$ par $f$.	
	\end{enumerate}
}{exe:images-antécédents}{
	\begin{enumerate}
	\item
	L'image de $x$ est donnée par $f(x)$. On calcule donc
		\begin{align*}
			f(0) &= 3\cdot0 + 1 = 1 \\
			f(3,1) &= 3\cdot3,1 + 1 = 10,3 \\
			f\left(\frac13\right) &= 3\cdot\frac13 + 1 = 1 + 1 = 2 \\
			f(-1) &= 3\cdot(-1) + 1 = -2 \\
			f\left(-\frac23\right) &= 3 \cdot \left(- \frac23\right) + 1 = -1.
		\end{align*}
	\item
	La relation
		\[ f(0) = 1 \]
	implique que l'image de $0$ est $1$, et qu'un antécédent de $1$ est $0$.
	On a donc trouvé $0$, antécédent de $1$ par $f$.
	\item
	Appelons $x$ un antécédent de $6$ par $f$.
	Autrement dit, l'image de $x$ est $6$, et donc 
		\[ f(x) = 6. \]
	En utilisant l'expression algébrique de $f$, on trouve
		\begin{align*}
			f(x) &= 6, \\
			3\cdot x + 1 &= 6, \\
			3\cdot x &= 5, \\
			x &= \dfrac53.
		\end{align*}
	Ainsi $x=\frac53$ est le seul antécédent de $6$ par $f$.
	En général, certaines fonctions admettes plusieurs antécédents.
	Voir par exemple l'exercice suivant.
	\end{enumerate}


}

\exe{}{
	Un fonction $f$ admet le tableau de valeurs suivant.
		\begin{center}
		\def\arraystretch{1.2}
		\setlength\tabcolsep{20pt}
		\begin{tabular}{|c|c|c|c|c|}\hline
			$x$ & 0 & -2 & 1 & -1 \\ \hline
			$f(x)$ & 1 & 0 & 0 & 1 \\ \hline
		\end{tabular}
		\end{center}
	Parmis les expressions algébriques suivantes, lesquelles \underline{ne peuvent pas} correspondre à $f(x)$ ?
		\begin{multicols}{4}
		\begin{enumerate}[label=\roman*)]
			\item $1-x$
			\item $1+\dfrac{x}2$
			\item $\dfrac{1-x}2$
			\item $\dfrac{-x^2 - x + 2}2$
		\end{enumerate}
		\end{multicols}
}{exe:f4}{
	Il s'agit de discriminer les fonctions possibles en utilisant les images du tableau.
		\begin{enumerate}[label=\roman*)]
			\item $1-x$ vaut bien $1$ en $x=0$, mais l'expression vaut $3$ en $x=-2$, donc ce n'est pas l'expression de $f$.
			\item  $1+\dfrac{x}2$ vaut bien $1$ en $x=0$ et $0$ en $x=-2$, mais elle vaut $1$ en $x=1$, ce n'est donc pas l'expression de $f$.
			\item $\dfrac{1-x}2$ vaut $\frac12$ en $x=0$, ce n'est donc pas l'expression de $f$.
			\item On déduit $f(x) = \dfrac{-x^2 - x + 2}2$ est la seule expression possible, qu'on vérifiera en calculant les images de $0, -2, 1,$ et $-1$.
		\end{enumerate}

	Cette exercice deviendra vite évident après avoir terminé l'étude des fonctions affines. 
	Comme les courbes représentatives des trois premières expressions sont des droites non constantes, toutes leurs images ont un unique antécédent.
	Or ici $0$ et $1$ ont deux antécédents.
}



\exe{}{
	Remplir le tableau d'images suivant.
		\begin{center}
		\def\arraystretch{1.2}
		\setlength\tabcolsep{20pt}
		\begin{tabular}{|c|c|c|c|c|}\hline
			$x$ & 0 & -2 & 1 & -1 \\ \hline
			$f(x)$ & 1 & 3 & 0 & \\ \hline
			$g(x)$ & 1 & 0 &  & -2 \\ \hline
			$f(x)+2g(x)$ &  & & 2 & 1 \\ \hline
		\end{tabular}
		\end{center}
}{exe:fonctions-QCM-trous}{
		\begin{center}
		\def\arraystretch{1.2}
		\setlength\tabcolsep{20pt}
		\begin{tabular}{|c|c|c|c|c|}\hline
			$x$ & 0 & -2 & 1 & -1 \\ \hline
			$f(x)$ & 1 & 3 & 0 & 5 \\ \hline
			$g(x)$ & 1 & 0 & 1 & -2 \\ \hline
			$f(x)+2g(x)$ & 3 & 3 & 2 & 1 \\ \hline
		\end{tabular}
		\end{center}
}

\newpage

\exe{, difficulty=1}{
	Posons $h(x) = (x+2)(x-1)$ pour tout $x\in\R$.
	Remplir le tableau de valeurs ci-dessous.
		\begin{center}
		\def\arraystretch{1.2}
		\setlength\tabcolsep{30pt}
		\begin{tabular}{|c|c|c|}\hline
			$x$ & -2 & 1 \\ \hline
			$f(x)$ & -42 & 1729 \\ \hline
			$h(x)$ &  &   \\ \hline
			$f(x)+h(x)$ &  &   \\ \hline
			$f(x)+2h(x)$ &  &   \\ \hline
		\end{tabular}
		\end{center}
}{exe:fonctions-QCM2}{
	Lors du calcul de chacune des images, au moins un facteur est nul : le produit vaut donc toujours zéro.
}

\exe{, difficulty=2}{
	Montrer qu'il existe une infinité de fonctions, toutes différentes, vérifiant le même tableau de valeurs de l'exercice \ref{exe:fonctions-QCM2}.
}{exe:fonctions-QCM3}{
	On prend par exemple $f(x) + h(x), f(x) + 2h(x), f(x) + 3h(x), \dots$.
}


\exe{, difficulty=1}{
	Soit $f(x) = (x-12)(x-13)$.
	Développer l'expression pour vérifier que $f(x) = x^2 - 25x + 156$.
	
	Calculer les images de 0 ; 1 ; 11 ; 12 ; 13; et 14 par $f$ avec la forme adéquate et sans calculatrice.
}{exe:image-selon-forme}{
	Pour $f(0)$ et $f(1)$, on préférera la forme développée $f(x) = x^2 - 25x + 156$.
	Ainsi, $f(0) = 156$ immédiatement, et $f(1) = 1 - 25 + 156 = 132$.
	
	Pour $f(11), f(12), f(13), f(14)$, la forme factorisée est préférable, car on calcule facilement que
		\begin{align*}
			f(11) &= (11-12)(11-13) = (-1)(-2) = 2 \\
			f(12) &= (12-12)(12-13) = 0 \\
			f(13) &= (13-12)(13-13) = 0 \\
			f(14) &= (14-12)(14-13) = (2)(1) =2
		\end{align*}
}


\exe{, difficulty=1}{
	Considérons les fonctions $f(x) =x^2 -6x + 9$ et $g(x)=(x-3)^2$ pour tout $x\in\R$.
	\begin{enumerate}
		\item
		Calculer les images par $f$ et $g$ de $0$ ; de $6$ ; de $3$ ; de $-\frac23$.
		\item
		Donner deux antécédents de $9$ par $f$.
		\item
		Montrer que $f(x) = g(x)$ pour tout $x\in\R$ réel.
	\end{enumerate}
}{exe:forme-canonique}{
	\begin{enumerate}
		\item
		Pour $f$, on calcule
			\begin{align*}
				f(0) &= 9, \\
				f(6) &= 6^2 - 6^2 + 9 = 9, \\
				f(3) &= 3^2 - 6\times3 + 9 = 0, \\
				f\left(-\frac23\right) &= \left(-\frac23\right)^2 - 6\left(-\frac23\right) + 9 = \frac49 + 4 + 9 = \frac49 + \frac{117}9 = \frac{121}9.
			\end{align*}
		Pour $g$, on calcule
			\begin{align*}
				g(0) &= (-3)^2 = 9, \\
				g(6) &= 3^2 = 9, \\
				g(3) &= 0^2 = 0, \\
				g\left(-\frac23\right) &= \left(-\frac23 - 3\right)^2= \left(-\dfrac{11}3\right)^2 = \frac{121}9.
			\end{align*}
		\item
		0 et 6 sont deux antécédents de 9 par $f$.
		\item
		On part toujours de la forme factorisée ($g$ ici) pour arriver vers la forme développée réduite ($f$ ici).
			\begin{align*}
				g(x) &= (x-3)^2 \\
					&= (x-3)(x-3) \\
					&= x(x-3) - 3(x-3) \\
					&= x^2 - 3x - 3x + 9 \\
					&= x^2 - 6x + 9
			\end{align*}
		Le développement de carrés parfaits deviendront immédiates après l'étude des identités remarquables.
	\end{enumerate}
}

\subsection*{Exercices supplémentaires}

\exe{}{
	Considérons la fonction $f(x) = \frac15-x$ pour tout $x\in\R$.
	
	\begin{enumerate}
		\item
		Calculer l'image par $f$ de $0$ ; de $0,2$ ; de $\frac47$ ; de $-\frac23$.
		\item
		Donner un antécédent de $0$ par $f$.
		\item
		Déterminer tous les antécédents de $5$ par $f$.	
	\end{enumerate}
}{exe:images-antécédents2}{
	\begin{enumerate}
		\item
		Calculer l'image par $f$ de $0$ ; de $0,2$ ; de $\frac47$ ; de $-\frac23$.
		\begin{align*}
			f(0) &= \frac15 \\
			f(0,2) &= \frac15 - 0,2 = 0 \\
			f\left(\frac47\right) &= \frac15 - \frac47 = \frac{7}{35} - \frac{20}{35} = \frac{-13}{35} \\
			f\left(-\frac23\right) &= \frac15 + \frac23 = \frac{3}{15} + \frac{10}{15} = \frac{13}{15}
		\end{align*}
		\item
		0,2 en est un d'après la question précédente.
		\item
		On pose $f(x) = 5$ et on résoud pour $x$.
			\begin{align*}
				f(x) &= 5 \\
				\frac15 -x &= 5 \\
				-x &= 5-\frac15 = \frac{24}{5} \\
				x &= -\frac{24}{5}
			\end{align*}
	\end{enumerate}
}



\exe{, difficulty=1}{
	Un fonction $f$ définie sur tout $\R$ admet le tableau de valeurs suivant.
		\begin{center}
		\def\arraystretch{1.2}
		\setlength\tabcolsep{20pt}
		\begin{tabular}{|c|c|c|c|c|}\hline
			$x$ & 0 & -2 & 1 & -1 \\ \hline
			$f(x)$ & 1 & 0 & 0 & 1 \\ \hline
		\end{tabular}
		\end{center}
	Parmis les expressions algébriques suivantes, lesquelles \underline{ne peuvent pas} correspondre à $f(x)$ ?
		\begin{multicols}{3}
		\begin{enumerate}[label=\roman*)]
			\item $1-x$
			\item $\dfrac{-x^2 - x + 2}2$
			\item $1+\dfrac{x}2$
			\item $\dfrac{1-x}2$
			\item $\dfrac{x^3 - 3x + 2}2$
			\item $\dfrac{-x^4 - 2x^3 + x + 2}2$
		\end{enumerate}
		\end{multicols}
}{exe:fonctions-QCM}{
		\begin{enumerate}[label=\roman*)]
			\item
			Impossible car $f(-2) = 0$.
			\item 
			Celle-ci est possible, car toutes les images correspondent au tableau.
			\item 
			Impossible car $f(1) = 0$.
			\item
			Impossible car $f(-2) = 0$.
			\item 
			Celle-ci est possible, car toutes les images correspondent au tableau.
			\item 
			Celle-ci est possible, car toutes les images correspondent au tableau.
		\end{enumerate}
}

\exe{, difficulty=2}{
	Considérons $f(x) = x^2 - 6x + 16$ pour tout $x\in\R$.
	\begin{enumerate}
		\item Montrer que $f(x) = 7 + (x-3)^2$ pour tout $x\in\R$ réel.
		\item Conclure que pour tout $x\in\R$ réel, $f(x)$ est supérieur à 7.
		\item Trouver un $x^\star\in\R$ réel vérifiant $f(x^\star) = 7$.
	\end{enumerate}
	On dira alors que $f$ atteint son minimum en $x^\star$, car $f(x) \geq f(x^\star)$ pour tout $x\in\R$.
}{exe:canonique0}{
	\begin{enumerate}
		\item 
		On part toujours de la forme factorisée.
		Par double distributivité,
			\begin{align*}
				7 + (x-3)^2 &= 7 + (x-3)(x-3) \\
							&= 7 + x^2 - 3x - 3x + 9 \\
							&= x^2 - 6x + 16 = f(x)
			\end{align*}
		\item
		Le carré étant toujours positif, $(x-3)^2$ est toujours positif, et donc $f(x) = 7 + (x-3)^2$ est toujours plus grand que 7.
		\item
		En prenant $x^\star = 3$, on a bien $f(x^\star) = 7 + (3-3)^2 = 7$.
	\end{enumerate}
}

%%%%%
%%%%% INTERVALLES
%%%%% 




%%%%%
%%%%% REP GRAPHIQUE
%%%%% 
	

%\exe{}{
%	Considérons la représentation graphique suivante d'une fonction $f$ définie sur $\D = ]{-}3,4 ; 2,3[$.
%	
%		\begin{center}
%		\begin{tikzpicture}[>=stealth]
%			\begin{axis}[xmin = -3.4, xmax=2.3, ymin=-5.1, ymax=5.1, axis x line=middle, axis y line=middle, axis line style=->, grid=both, clip=true]
%				\addplot[no marks, blue, -] expression[domain=-5:3, samples=50]{x^3 /3 - 2*x +3}
%				node[pos=.3, right]{$\mathcal{C}_f$};
%			\end{axis}
%		\end{tikzpicture}
%		\end{center}
%	\begin{enumerate}
%		\item Donner approximativement les images de $-1,5$ et de $-\frac{20}7$ par $f$.
%		\item Énumérer approximativement les antécédents de $-2$ et de $2$ par $f$.
%		\item Donner approximativement un réel qui admet exactement deux antécédents par $f$.
%		\item Si $f$ était définie sur $\R$ tout entier, serait-il toujours possible de connaître l'image de $-2$ ? Et tous les antécédents de $-2$ ?
%	\end{enumerate}
%	Supposons désormais que $f(x) = 3-2x +\frac13 x^3$ pour tout $x\in\D$ du domaine.
%	\begin{enumerate}[resume]
%		\item Vérifier à la calculatrice les réponses aux deux premières questions.
%		\item Montrer sans calculatrice que l'image par $f$ de $-3$ est $0$ et que l'image par $f$ de $0$ est $3$.
%	\end{enumerate}
%}{exe:f5}{
%	\begin{enumerate}
%		\item On détermine approximativement
%			\begin{align*}
%			f(-1,5) \approx 4,5 && f\left(-\dfrac{20}7\right) \approx f(-2,86) \approx 2
%			 \end{align*}
%		\item On cherche d'abord les antécédents de $-2$, c'est-à-dire les nombres $x$ du domaine vérifiants
%			\[ f(x) = -2. \]
%		Pour ça, on se pose \og à hauteur -2 \fg : on tracer une droite horizontale d'ordonnée $-2$ et on regarde les points d'intersection.
%			\begin{center}
%			\begin{tikzpicture}[>=stealth]
%				\begin{axis}[xmin = -3.4, xmax=2.3, ymin=-5.1, ymax=5.1, axis x line=middle, axis y line=middle, axis line style=->, grid=both, clip=true]
%					\addplot[no marks, blue, -] expression[domain=-5:3, samples=50]{x^3 /3 - 2*x +3};
%					
%					\addplot[no marks, red, -] expression[domain=-5:3, samples=2]{-2};
%				\end{axis}
%			\end{tikzpicture}
%			\end{center}
%		
%		
%		En l'occurrence, seul $x\approx -3,2$ fonctionne :
%			\[ f(-3,2) \approx -2. \]
%		Pour les antécédents de $2$, on fait idem en regardant les points de la courbe d'ordonnée $2$.
%		On trouve trois valeurs approximatives : $-2,8 ; 0,5 ; $ et $2,1$.
%		
%			\begin{center}
%			\begin{tikzpicture}[>=stealth]
%				\begin{axis}[xmin = -3.4, xmax=2.3, ymin=-5.1, ymax=5.1, axis x line=middle, axis y line=middle, axis line style=->, grid=both, clip=true]
%					\addplot[no marks, blue, -] expression[domain=-5:3, samples=50]{x^3 /3 - 2*x +3};
%					
%					\addplot[no marks, red, -] expression[domain=-5:3, samples=2]{2};
%				\end{axis}
%			\end{tikzpicture}
%			\end{center}
%		
%		\item 
%		On a plusieurs choix ici. 
%		Il faut placer une droite horizontale telle qu'elle s'intersecte exactement deux fois avec la courbe de $f$.
%		Un choix clair est $4$ (en violet ci-dessous).
%		Un choix moins clair est $1,1$, en faisant en sorte que la courbe frôle la droite horizontale qu'on place en ordonnée $1,1$ (en rouge ci-dessous).
%		On dit alors que la droite est \emph{tangente} à la courbe.
%		
%			\begin{center}
%			\begin{tikzpicture}[>=stealth]
%				\begin{axis}[xmin = -3.4, xmax=2.3, ymin=-5.1, ymax=5.1, axis x line=middle, axis y line=middle, axis line style=->, grid=both, clip=true]
%					\addplot[no marks, blue, -] expression[domain=-5:3, samples=50]{x^3 /3 - 2*x +3};
%					
%					\addplot[no marks, red, -] expression[domain=-5:3, samples=2]{1.1};
%					\addplot[no marks, violet, -] expression[domain=-5:3, samples=2]{4};
%				\end{axis}
%			\end{tikzpicture}
%			\end{center}
%		
%		\item L'image est unique et ne dépend pas du domaine (tant que celui-ci contient $-2$ !).
%		On peut donc toujours connaître $f(-2)$, même sachant qu'on connaisse pas $f$ sur $\R$ tout entier mais que sur un domaine restreint.
%		
%		Les antécédents, eux, dépendent du domaine choisi.
%		Comme on ne sait pas du tout à quoi ressemble $\C_f$ en dehors du domaine choisi, il est impossible de déterminer tous les réels $x\in\R$ antécédents de $-2$ par $f$.
%		
%		\item On calcule grâce à la calculatrice (ou sans...)
%			\begin{align*}
%				f(-1,5) = 4,875 && f\left(-\dfrac{20}7 \right) \approx 0,94.
%			\end{align*}
%		\item On calcule à la main que
%			\begin{align*}
%				f(-3) &= 3 - 2 \cdot (-3) + \dfrac13 \cdot (-3)^3, \\
%					&= 3 + 6 - 9, \\
%					&= 0.
%			\end{align*}
%		Pour l'image de $0$, remarquons que seul le terme ne dépendant pas de $x$ subsiste. 
%		On l'appelle le terme \emph{constant}, et on trouve $f(0) = 3$, comme requis.
%	\end{enumerate}
%}

%%%%%%%%%%%

\newpage
\fancyhead[C]{\textbf{Solutions}}
\shipoutAnswer

\end{document}

% DYSLEXIA SWITCH
\newif\ifdys
		
				% ENABLE or DISABLE font change
				% use XeLaTeX if true
				\dystrue
				\dysfalse


\ifdys

\documentclass[a4paper, 14pt]{extarticle}
\usepackage{amsmath,amsfonts,amsthm,amssymb,mathtools}

\tracinglostchars=3 % Report an error if a font does not have a symbol.
\usepackage{fontspec}
\usepackage{unicode-math}
\defaultfontfeatures{ Ligatures=TeX,
                      Scale=MatchUppercase }

\setmainfont{OpenDyslexic}[Scale=1.0]
\setmathfont{Fira Math} % Or maybe try KPMath-Sans?
\setmathfont{OpenDyslexic Italic}[range=it/{Latin,latin}]
\setmathfont{OpenDyslexic}[range=up/{Latin,latin,num}]

\else

\documentclass[a4paper, 12pt]{extarticle}

\usepackage[utf8x]{inputenc}
%fonts
\usepackage{amsmath,amsfonts,amsthm,amssymb,mathtools}
% comment below to default to computer modern
\usepackage{libertinus,libertinust1math}

\fi


\usepackage[french]{babel}
\usepackage[
a4paper,
margin=2cm,
nomarginpar,% We don't want any margin paragraphs
]{geometry}
\usepackage{icomma}

\usepackage{fancyhdr}
\usepackage{array}
\usepackage{hyperref}

\usepackage{multicol, enumerate}
\newcolumntype{P}[1]{>{\centering\arraybackslash}p{#1}}


\usepackage{stackengine}
\newcommand\xrowht[2][0]{\addstackgap[.5\dimexpr#2\relax]{\vphantom{#1}}}

% theorems

\theoremstyle{plain}
\newtheorem{theorem}{Th\'eor\`eme}
\newtheorem*{sol}{Solution}
\theoremstyle{definition}
\newtheorem{ex}{Exercice}
\newtheorem*{rpl}{Rappel}
\newtheorem{enigme}{Énigme}

% corps
\usepackage{calrsfs}
\newcommand{\C}{\mathcal{C}}
\newcommand{\R}{\mathbb{R}}
\newcommand{\Rnn}{\mathbb{R}^{2n}}
\newcommand{\Z}{\mathbb{Z}}
\newcommand{\N}{\mathbb{N}}
\newcommand{\Q}{\mathbb{Q}}

% variance
\newcommand{\Var}[1]{\text{Var}(#1)}

% domain
\newcommand{\D}{\mathcal{D}}


% date
\usepackage{advdate}
\AdvanceDate[0]


% plots
\usepackage{pgfplots}

% table line break
\usepackage{makecell}
%tablestuff
\def\arraystretch{2}
\setlength\tabcolsep{15pt}

%subfigures
\usepackage{subcaption}

\definecolor{myg}{RGB}{56, 140, 70}
\definecolor{myb}{RGB}{45, 111, 177}
\definecolor{myr}{RGB}{199, 68, 64}

% fake sections with no title to move around the merged pdf
\newcommand{\fakesection}[1]{%
  \par\refstepcounter{section}% Increase section counter
  \sectionmark{#1}% Add section mark (header)
  \addcontentsline{toc}{section}{\protect\numberline{\thesection}#1}% Add section to ToC
  % Add more content here, if needed.
}


% SOLUTION SWITCH
\newif\ifsolutions
				\solutionstrue
				%\solutionsfalse

\ifsolutions
	\newcommand{\exe}[2]{
		\begin{ex} #1  \end{ex}
		\begin{sol} #2 \end{sol}
	}
\else
	\newcommand{\exe}[2]{
		\begin{ex} #1  \end{ex}
	}
	
\fi


% tableaux var, signe
\usepackage{tkz-tab}


%pinfty minfty
\newcommand{\pinfty}{{+}\infty}
\newcommand{\minfty}{{-}\infty}

\begin{document}


\usepackage{minted}

%\SetDate[17/10/2025]

\begin{document}
\pagestyle{fancy}
\fancyhead[L]{Seconde}
\fancyhead[C]{\textbf{Fonctions, balayages, et comparaisons en Python}}
\fancyhead[R]{\today}

\subsection*{Définir une fonction}

\begin{multicols}{3}
	Pour définir une fonction, on utilise \texttt{def}.
	L'indentation délimite les bornes de la fonction.
	L'instruction \texttt{return} définit la valeur retournée.
	
	Un fonction peut prendre plusieurs arguments.
	
	\columnbreak
	\centering
	\begin{minipage}{.1\textwidth}
	\python{fonction}
	\end{minipage}	
	
	\columnbreak
	\centering
	\begin{minipage}{.1\textwidth}
	\python{fonction2}
	\end{minipage}
\end{multicols}

\exe{}{
	Considérons la fonction $f(x) = 2 - x$. 
	
	Que valent $f(-1)$ et $f(2)$ ? Vérifier les réponses à l'aide de Python.
}{exe:first-example}{
	td
}

\subsection*{Balayer un intervalle}

\begin{multicols}{2}
	Pour parcourir l'intervalle $[a ; b]$ en prenant $N+1$ valeurs espacées uniformément,
	on calcule
		\begin{align*} 
			x = a + \dfrac{k}N (b-a), && \text{pour} && k = 0, 1, 2, 3, \dots, N.
		\end{align*}
	L'expression \texttt{range(N+1)} correspond à l'ensemble ${\bigset{ 0 ; 1 ; 2 ; \dots ; N }}$.

	\columnbreak
	\centering
	\begin{minipage}{.2\textwidth}
	\python{balayage}
	\end{minipage}
\end{multicols}


\exe{}{
	Considérons la fonction inverse $f(x) = \frac1x$. 
	
	Calculer les images par $f$ de 101 antécédents entre -1 et 1.
	Que remarque-t-on ?
}{exe:inv}{
	\begin{multicols}{2}
	\python{inverse}
	\end{multicols}
}

\subsection*{Comparer deux valeurs}


\begin{multicols}{2}
	L'instruction \texttt{if x == y:} permet d'exécuter des instructions dès que l'égalité $x=y$ est vraie.
	
	\begin{enumerate}[label=\warning]
		\item deux signes \texttt{=} sont nécessaires pour distinguer la comparaison de l'affectation.
	\end{enumerate}
	\centering
	\begin{minipage}{.1\textwidth}
	\python{condition}
	\end{minipage}
\end{multicols}

\exe{}{
	Écarter le cas $x=0$ de la fonction inverse de l'exercice \ref{exe:inv} afin de pouvoir calculer ses images sans erreur.
}{exe:inv-with-cond}{
	\begin{multicols}{2}
	\python{inverse-condition}
	\end{multicols}
}

\subsection*{Précision}

\begin{multicols}{2}
	La précision étant finie, la comparaison peut renvoyer \texttt{False} alors que les valeurs comparées sont théoriquement égales.
	
	Pour éviter ces problèmes, il peut être pertinent de vérifier que deux valeurs sont proches plutôt qu'égales.
	On vérifiera alors, par exemple, que la distance entre $x$ et $y$ est plus petite que $0,001$ au lieu que $x = y$.
	
	L'appel \texttt{abs(x-y)} permet de calculer la distance entre $x$ et $y$.
	
	\centering
	\begin{minipage}{.2\textwidth}
	\python{precision}
	\end{minipage}
\end{multicols}

\subsection*{Exercices : longueurs}

\exe{}{
	Considérons la fonction \texttt{carre(x)} qui renvoie le carré de $x$.
	
	Que valent \texttt{carre(3)} et \texttt{carre(-3)} ? Vérifier les réponses à l'aide de Python.
}{exe:carre}{
	td
}

\exe{, difficulty=1}{
	Définir la fonction \texttt{carre\_longueur(xA, yA, xB, yB)} qui renvoient la longueur $AB^2$, où \texttt{xA, yA, xB, yB} sont les coordonées de $A$ et de $B$.
	On pourra utiliser la fonction \texttt{carre} de l'exercice \ref{exe:carre}.
	
	Que vaut \texttt{carre\_longueur(2,1,-1,-3)} ? Vérifier la réponse à l'aide de Python, en 
}{exe:carre-longueur}{
	td
}

\exe{, difficulty=2}{
	Définir la fonction \texttt{is\_isosceles(xA, xB, yA, yB)} qui renvoie \texttt{True} si le triangle $ABO$ est isocèle en $O$, et \texttt{False} sinon.
	On pourra utiliser la fonction \texttt{carre\_longueur} de l'exercice \ref{exe:carre-longueur}.
	
	Que vaut \texttt{is\_isosceles(-1, -3, 3, -1)} ? Vérifier la réponse à l'aide de Python.
}{exe:is-isosceles}{
	td
}

\exe{, difficulty=2}{
	Définir la fonction \texttt{is\_rightangled(xA, xB, yA, yB)} qui renvoie \texttt{True} si le triangle $ABO$ est rectangle en $O$, et \texttt{False} sinon.
	On pourra utiliser la fonction \texttt{carre\_longueur} de l'exercice \ref{exe:carre-longueur}.
	
	Que vaut \texttt{is\_rightangled(-1, 3, 6, 2)} ? Vérifier la réponse à l'aide de Python.
}{exe:is-rightangled}{
	td
}


\subsection*{Exercices : moyennes}


\exe{, difficulty=0}{
	Définir la fonction $\texttt{moyenne(x,y)}$ qui renvoie la moyenne de $x$ et de $y$.
	
	Que valent \texttt{moyenne(10,7)} et \texttt{moyenne(7,10)} ? Vérifier les réponses à l'aide de Python.
}{exe:moyenne}{
	td
}

\exe{, difficulty=1}{
	Définir la fonction $\texttt{moyenne\_ponderee(x,y, p)}$ qui renvoie la moyenne pondérée $p x + (1-p)y$.
	
	Que valent \texttt{moyenne\_ponderee(7, 10, 0.5)}, \texttt{moyenne\_ponderee(7, 10, 1/3)}, et \texttt{moyenne\_ponderee(7, 10, 2/3)} ? Vérifier les réponses à l'aide de Python.
}{exe:moyenne-ponderee}{
	td
}

\exe{, difficulty=2}{
	Définir la fonction $\texttt{moyenne\_coeff(x,y, c1, c2)}$ qui renvoie la moyenne pondérée par les coefficients $c_1, c_2$ suivante.
		\[ \dfrac{c_1 x + c_2 y}{c_1 + c_2}. \]
	
	Que valent \texttt{moyenne\_coeff(7, 10, 1, 1)}, \texttt{moyenne\_coeff(7, 10, 1, 2)}, et \texttt{moyenne\_coeff(7, 10, 2, 1)} ? Vérifier les réponses à l'aide de Python.
}{exe:moyenne-coeff}{
	td
}

\subsection*{Exercices : comparaisons}

\exe{, difficulty=1}{
	Balayer l'intervalle $[1 ; 2]$ et afficher tous les $x$ vérifiant $x^2 < 2$.
	
	En affinant le baylage, trouver un encadrement de $\sqrt2$ à $10^{-3}$ près.
}{exe:sqrt2}{
	td
}

\exe{, difficulty=1}{
	Considérons $F(x) = x^2$ et $G(x) = -100x^2 + 1$.
	
	Parcourir 1001 antécédents $x$ de l'intervalle $[-1 ; 1]$ et afficher tous les $x$ vérifiant $F(x) < G(x)$.
	
	En déduire, approximativement, l'ensemble $\bigset{x \in [-1 ; 1] \tq F(x) < G(x)}$ sous forme d'intervalle.
}{exe:sqrt.5}{
	\begin{multicols}{2}
	\python{sqrt.5}
	\end{multicols}
}


%%%%%%%%%%%

\newpage
\fancyhead[C]{\textbf{Solutions}}
\shipoutAnswer

\end{document}

% DYSLEXIA SWITCH
\newif\ifdys
		
				% ENABLE or DISABLE font change
				% use XeLaTeX if true
				\dystrue
				\dysfalse


\ifdys

\documentclass[a4paper, 14pt]{extarticle}
\usepackage{amsmath,amsfonts,amsthm,amssymb,mathtools}

\tracinglostchars=3 % Report an error if a font does not have a symbol.
\usepackage{fontspec}
\usepackage{unicode-math}
\defaultfontfeatures{ Ligatures=TeX,
                      Scale=MatchUppercase }

\setmainfont{OpenDyslexic}[Scale=1.0]
\setmathfont{Fira Math} % Or maybe try KPMath-Sans?
\setmathfont{OpenDyslexic Italic}[range=it/{Latin,latin}]
\setmathfont{OpenDyslexic}[range=up/{Latin,latin,num}]

\else

\documentclass[a4paper, 12pt]{extarticle}

\usepackage[utf8x]{inputenc}
%fonts
\usepackage{amsmath,amsfonts,amsthm,amssymb,mathtools}
% comment below to default to computer modern
\usepackage{libertinus,libertinust1math}

\fi


\usepackage[french]{babel}
\usepackage[
a4paper,
margin=2cm,
nomarginpar,% We don't want any margin paragraphs
]{geometry}
\usepackage{icomma}

\usepackage{fancyhdr}
\usepackage{array}
\usepackage{hyperref}

\usepackage{multicol, enumerate}
\newcolumntype{P}[1]{>{\centering\arraybackslash}p{#1}}


\usepackage{stackengine}
\newcommand\xrowht[2][0]{\addstackgap[.5\dimexpr#2\relax]{\vphantom{#1}}}

% theorems

\theoremstyle{plain}
\newtheorem{theorem}{Th\'eor\`eme}
\newtheorem*{sol}{Solution}
\theoremstyle{definition}
\newtheorem{ex}{Exercice}
\newtheorem*{rpl}{Rappel}
\newtheorem{enigme}{Énigme}

% corps
\usepackage{calrsfs}
\newcommand{\C}{\mathcal{C}}
\newcommand{\R}{\mathbb{R}}
\newcommand{\Rnn}{\mathbb{R}^{2n}}
\newcommand{\Z}{\mathbb{Z}}
\newcommand{\N}{\mathbb{N}}
\newcommand{\Q}{\mathbb{Q}}

% variance
\newcommand{\Var}[1]{\text{Var}(#1)}

% domain
\newcommand{\D}{\mathcal{D}}


% date
\usepackage{advdate}
\AdvanceDate[0]


% plots
\usepackage{pgfplots}

% table line break
\usepackage{makecell}
%tablestuff
\def\arraystretch{2}
\setlength\tabcolsep{15pt}

%subfigures
\usepackage{subcaption}

\definecolor{myg}{RGB}{56, 140, 70}
\definecolor{myb}{RGB}{45, 111, 177}
\definecolor{myr}{RGB}{199, 68, 64}

% fake sections with no title to move around the merged pdf
\newcommand{\fakesection}[1]{%
  \par\refstepcounter{section}% Increase section counter
  \sectionmark{#1}% Add section mark (header)
  \addcontentsline{toc}{section}{\protect\numberline{\thesection}#1}% Add section to ToC
  % Add more content here, if needed.
}


% SOLUTION SWITCH
\newif\ifsolutions
				\solutionstrue
				%\solutionsfalse

\ifsolutions
	\newcommand{\exe}[2]{
		\begin{ex} #1  \end{ex}
		\begin{sol} #2 \end{sol}
	}
\else
	\newcommand{\exe}[2]{
		\begin{ex} #1  \end{ex}
	}
	
\fi


% tableaux var, signe
\usepackage{tkz-tab}


%pinfty minfty
\newcommand{\pinfty}{{+}\infty}
\newcommand{\minfty}{{-}\infty}

\begin{document}


\SetDate[10/10/2025]

\begin{document}
\pagestyle{fancy}
\fancyhead[L]{Seconde}
\fancyhead[C]{\textbf{Fonctions 2}}
\fancyhead[R]{\today}

%%%%%
%%%%% INTERVALLES
%%%%% 




%%%%%
%%%%% REP GRAPHIQUE
%%%%% 
	

\exe{}{
	Considérons la fonction $f$ sur $\R$ donnée algébriquement par
		\[ f(x) = \dfrac17-x. \]
	Pour chaque point suivant, déterminer s'il appartient à $\C_f$ ou non.
	
	\begin{multicols}{3}
	\begin{enumerate}[label=\roman*)]
		\item $\left(0; \dfrac17\right)$
		\item $\left(\dfrac17 ; 0\right)$
		\item $\left(\dfrac27 ; \dfrac37\right)$
		\item $\left(-\dfrac{13}7 ; 2\right)$
		\item $\left(\dfrac67 ; 1\right)$
		\item $\left(\dfrac27 ; -\dfrac17\right)$
	\end{enumerate}
	\end{multicols}

}{exe:Cf}{
	TODO
}

\exe{}{
	Considérons deux fonctions $f, g$ sur $\D = ]-3 ; 3[$ données algébriquement par
		\begin{align*}
			f(x) = x^2 - 2x && g(x) = (x-1)^2
		\end{align*}
	
	\begin{enumerate}
		\item Esquisser les représentations graphiques de $f$ et de $g$ dans un même repère.
		\item Démontrer que $g(x) = f(x) + 1$ pour tout $x$ du domaine.
	\end{enumerate}
}{exe:id-rem-graph}{
	TODO
}

\exe{}{
	Esquisser la courbe de la fonction $f$ sur $\D=[-2; 4]$ donnée algébriquement par
		\[ f(x) = 3. \]
	Que dire de $f$ et de $\C_f$ ?
}{exe:graph-const}{
	TODO
}

\exe{}{
	Esquisser la courbe de la fonction $f$ sur $\D=[-5;3]$ donnée algébriquement par
		\[ f(x) = 1-x. \]
	Que dire de $\C_f$ ?
}{exe:graph-droite}{
	TODO
}

\exe{}{
	Esquisser la courbe de la fonction $f$ sur $\D=[3;10]$ donnée algébriquement par
		\[ f(x) = \dfrac3x + 1. \]
}{exe:graph-droite2}{
	TODO
}

\newpage

\exe{}{
	Considérons la représentation graphique suivante d'une fonction $f$ définie sur $\D = ]{-}3,4 ; 2,3[$.
	
	\begin{center}
		\begin{tikzpicture}[>=stealth]
			\begin{axis}[xmin = -3.4, xmax=2.3, ymin=-2.1, ymax=5.1, axis x line=middle, axis y line=middle, axis line style=->, grid=both,
			grid style = {opacity=.5},
			x=2cm,
			xtick={-3, -2, ..., 2},
			y=20pt,
			clip=true,
			]
				\addplot[no marks, BLUE_E, very thick, -] expression[domain=-4:3, samples=200]{x^3 /3 - 2*x +3}node[pos=.52, above=5pt]{$\C_f$};
			\end{axis}
		\end{tikzpicture}
	\end{center}
	\begin{enumerate}
		\item Donner approximativement les images de $-1,5$ et de $-\frac{20}7$ par $f$.
		\item Énumérer approximativement les antécédents de $-2$ et de $2$ par $f$.
		\item Donner approximativement un réel qui admet exactement deux antécédents par $f$.
		\item Si $f$ était définie sur $\R$ tout entier, serait-il toujours possible de connaître l'image de $-2$ ? Et tous les antécédents de $-2$ ?
	\end{enumerate}
	Supposons désormais que $f(x) = 3-2 x +\frac13 x^3$ pour tout $x\in\D$ du domaine.
	\begin{enumerate}[resume]
		\item Vérifier à la calculatrice les réponses aux deux premières questions.
		\item Montrer sans calculatrice que l'image par $f$ de $-3$ est $0$ et que l'image par $f$ de $0$ est $3$.
	\end{enumerate}
}{exe:deg3}{
	TODO
}


\exemulticols{}{
	Un fonction $f$ admet une représentation graphique ci-contre.
	Parmis les expressions algébriques suivantes, lequelles ne peuvent pas correspondre à $f(x)$ ?
		\begin{multicols}{2}
		\begin{enumerate}[label=\roman*)]
			\item $1-x$
			\item $\dfrac{-1-x}3$
			\item $\left(x+\dfrac13\right)^2$
			\item $-2x - \dfrac23$
		\end{enumerate}
		\end{multicols}
		\vfill\null
}{
	\begin{center}
		\begin{tikzpicture}[>=stealth]
			\begin{axis}[xmin = -3.1, xmax=1.1, ymin=-3.1, ymax=5.1, axis x line=middle, axis y line=middle, axis line style=->, grid=both,
			grid style = {opacity=.5},
			clip=true,
			]
				\addplot[no marks, BLUE_E, very thick, -] expression[domain=-3:2, samples=2]{-2/3 - 2*x}
				node[pos=.3, right]{$\mathcal{C}_f$};
			\end{axis}
		
		\end{tikzpicture}
	\end{center}
}{exe:expr-from-graph}{
	TODO
}

\exe{}{
	Comparer les représentations graphiques des fonctions suivantes données algébriquement.
		\begin{align*}
			f(x) = x^2, && g(x) = x^2 - 3, && h(x) = (x+4)^2.
		\end{align*}
}{exe:y-shift}{
	TODO
}

\exe{, difficulty=2}{
	Donner une fonction réelle et un domaine telle qu'une de ses images admet
		\begin{multicols}{2}
		\begin{enumerate}
			\item exactement un antécédent
			\item exactement deux antécédents
			\item exactement trois antécédents
			\item une infinité d'antécédents
		\end{enumerate}	
		\end{multicols}
}{exe:nb-antécédents}{
	TODO
}

\exe{, difficulty=2}{
	Donner graphiquement une fonction sur $\R$ non constante telle que toutes les images de $f$ admettent un nombre infini d'antécédents.
}{exe:infinité-antécédents}{
	TODO
}


%%%%%%%%%%%

\newpage
\fancyhead[C]{\textbf{Solutions}}
\shipoutAnswer

\end{document}

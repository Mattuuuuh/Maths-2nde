% DYSLEXIA SWITCH
\newif\ifdys
		
				% ENABLE or DISABLE font change
				% use XeLaTeX if true
				\dystrue
				\dysfalse


\ifdys

\documentclass[a4paper, 14pt]{extarticle}
\usepackage{amsmath,amsfonts,amsthm,amssymb,mathtools}

\tracinglostchars=3 % Report an error if a font does not have a symbol.
\usepackage{fontspec}
\usepackage{unicode-math}
\defaultfontfeatures{ Ligatures=TeX,
                      Scale=MatchUppercase }

\setmainfont{OpenDyslexic}[Scale=1.0]
\setmathfont{Fira Math} % Or maybe try KPMath-Sans?
\setmathfont{OpenDyslexic Italic}[range=it/{Latin,latin}]
\setmathfont{OpenDyslexic}[range=up/{Latin,latin,num}]

\else

\documentclass[a4paper, 12pt]{extarticle}

\usepackage[utf8x]{inputenc}
%fonts
\usepackage{amsmath,amsfonts,amsthm,amssymb,mathtools}
% comment below to default to computer modern
\usepackage{libertinus,libertinust1math}

\fi


\usepackage[french]{babel}
\usepackage[
a4paper,
margin=2cm,
nomarginpar,% We don't want any margin paragraphs
]{geometry}
\usepackage{icomma}

\usepackage{fancyhdr}
\usepackage{array}
\usepackage{hyperref}

\usepackage{multicol, enumerate}
\newcolumntype{P}[1]{>{\centering\arraybackslash}p{#1}}


\usepackage{stackengine}
\newcommand\xrowht[2][0]{\addstackgap[.5\dimexpr#2\relax]{\vphantom{#1}}}

% theorems

\theoremstyle{plain}
\newtheorem{theorem}{Th\'eor\`eme}
\newtheorem*{sol}{Solution}
\theoremstyle{definition}
\newtheorem{ex}{Exercice}
\newtheorem*{rpl}{Rappel}
\newtheorem{enigme}{Énigme}

% corps
\usepackage{calrsfs}
\newcommand{\C}{\mathcal{C}}
\newcommand{\R}{\mathbb{R}}
\newcommand{\Rnn}{\mathbb{R}^{2n}}
\newcommand{\Z}{\mathbb{Z}}
\newcommand{\N}{\mathbb{N}}
\newcommand{\Q}{\mathbb{Q}}

% variance
\newcommand{\Var}[1]{\text{Var}(#1)}

% domain
\newcommand{\D}{\mathcal{D}}


% date
\usepackage{advdate}
\AdvanceDate[0]


% plots
\usepackage{pgfplots}

% table line break
\usepackage{makecell}
%tablestuff
\def\arraystretch{2}
\setlength\tabcolsep{15pt}

%subfigures
\usepackage{subcaption}

\definecolor{myg}{RGB}{56, 140, 70}
\definecolor{myb}{RGB}{45, 111, 177}
\definecolor{myr}{RGB}{199, 68, 64}

% fake sections with no title to move around the merged pdf
\newcommand{\fakesection}[1]{%
  \par\refstepcounter{section}% Increase section counter
  \sectionmark{#1}% Add section mark (header)
  \addcontentsline{toc}{section}{\protect\numberline{\thesection}#1}% Add section to ToC
  % Add more content here, if needed.
}


% SOLUTION SWITCH
\newif\ifsolutions
				\solutionstrue
				%\solutionsfalse

\ifsolutions
	\newcommand{\exe}[2]{
		\begin{ex} #1  \end{ex}
		\begin{sol} #2 \end{sol}
	}
\else
	\newcommand{\exe}[2]{
		\begin{ex} #1  \end{ex}
	}
	
\fi


% tableaux var, signe
\usepackage{tkz-tab}


%pinfty minfty
\newcommand{\pinfty}{{+}\infty}
\newcommand{\minfty}{{-}\infty}

\begin{document}


\SetDate[10/10/2025]
\usepackage{minted}

\begin{document}
\pagestyle{fancy}
\fancyhead[L]{Seconde}
\fancyhead[C]{\textbf{Fonctions 2}}
\fancyhead[R]{\today}

% idée originale : © Benjamin Clamme
\exemulticols{}{
	On considère le programme de calcul ci-contre.
	\begin{enumerate}
		\item
		Que renvoie le programme lorsqu'on lui donne 1 ?
		\item 
		Compléter l'expression obtenue après chaque instruction du programme.
		\item
		Développer et réduire l'expression finale.
		%\item
		%Vérifier la réponse à la question \ref{q1} en calculant l'image de 1.
		\item
		Quel nombre renvoie 0 lorsque donné au programme de calcul ?
	\end{enumerate}
}{
	\begin{center}
\def\arraystretch{2}
\setlength\tabcolsep{20pt}
	\begin{tabular}{ |  c | c |  } \hline
	 Instruction & Expression \\ \hline
	 Choisir un nombre & $x$ \\ \hline 
	 Ajouter 3  &  \\ \hline
	 Diviser par 3 & \\ \hline 
	 Ajouter 1  &  \\ \hline
	\end{tabular}
	\end{center}	
}{exe:programme-calcul}{
	\begin{center}
	\begin{tabular}{ |  c | c |  } \hline
	 Instruction & Expression \\ \hline
	 Choisir un nombre & $x$ \\ \hline 
	 Ajouter 3  & $x+3$ \\ \hline
	 Diviser par 3 & $\frac13(x+3)$ \\ \hline 
	 Ajouter 1  &  $\frac13(x+3) + 1$ \\ \hline
	\end{tabular}
	\end{center}	
	À noter que les deux expressions suivantes sont égales :
		\[ \dfrac{x+3}{3} = \dfrac13 (x+3). \]	
	Posons $f(x) = \frac13(x+3) + 1$ dans la suite.
	\begin{enumerate}
		\item
		$f(1) = \frac13 \times 4 + 1 = \frac73$.
		\item[]
		\item
		\begin{align*}	
			f(x) &= \dfrac13(x+3) + 1, \\
				&= \dfrac13 x + \dfrac13 \times 3 + 1, \\
				&= \dfrac13x + 2.
		\end{align*}
		À noter que distribuer $\frac13$ revient à écrire que
			\[  \dfrac{x+3}{3} =  \dfrac{x}{3} +  \dfrac{3}{3}, \]
		qui est une propriété classique des fractions.
		\item
		On se demande quel $x$ vérifie $f(x) = 0$.
		On peut soit remonter le programme de calcul, soit résoudre l'équation algébriquement.
		Il s'avère que ces deux méthodes sont strictement identiques si on garde la forme $f(x) = \frac13(x+3) + 1$.
		
		Avec la forme développée :
			\begin{align*}
				f(x) &= 0, \\
				\dfrac13 x + 2 &= 0, \\
				\dfrac13 x &= - 2, \\
				x &= -2 \times 3 = -6.
			\end{align*}
		Avec la forme du tableau : 
			\begin{align*}
				f(x) &= 0, \\
				\dfrac13 (x+3) + 1 &= 0, \\
				\dfrac13 (x+3) &= - 1, \\
				x+3 &= -3, \\
				x &= - 6.
			\end{align*}
		Chaque opération faite permet de remonter un étape du programme de calcul : on soustrait 1, puis on multiplie par 3, et enfin on soustrait 3.
	\end{enumerate}
}

%%%%%
%%%%% INTERVALLES
%%%%% 

% volé à © Benjamin Clamme (goat)
\exe{}{
	Traduire l'appartenance de $x\in\R$ à chaque intervalle ci-dessous en termes d'inégalités :
	\begin{multicols}{2}
	\begin{enumerate}[label=(\alph*)]
		\item $I_1 = ]-3;7[$
		\item $I_2 = [4;12]$
		\item $I_3 = ]-\infty;0]$
		\item $I_4 = [-3;4[$
	\end{enumerate}
	\end{multicols}
	Tracer la droite réelle avec les intervalles $I_1$ et $I_2$ coloriés.
	Donner une condition pour que $x$ appartienne à l'intervalle $I_1$ \textbf{ET} à l'intervalle $I_2$.
	Exprimer ce résultat sous forme d'intervalle.
	%On note $I_1 \cap I_2$ l'\emph{intersection} des intervalles : ce sont les nombres qui appartiennent à la fois à $I_1$ et à $I_2$.
}{exe:intervalles}{
	\begin{enumerate}[label=(\alph*)]
		\item 
			\begin{align*}
				x \in ]-3;7[ && \iff && -3 < x < 7
			\end{align*}
		\item 
			\begin{align*}
				x \in [4;12] && \iff && 4 \leq x  \leq 12
			\end{align*}
		\item 
			\begin{align*}
				x \in ]-\infty;0] && \iff && x \leq 0
			\end{align*}
		\item 
			\begin{align*}
				x \in [-3;4[ && \iff && -3 \leq x \leq 4
			\end{align*}
	\end{enumerate}
	
	Si $x$ appartient à $I_1$ et à $I_2$, alors il vérifie à la fois
		\begin{align*}
			 -3 < x < 7 && \et && 4 \leq x  \leq 12.
		\end{align*}
	$x$ est donc supérieur ou égal à 4, car c'est la borne inférieure la plus contraignante.
	$x$ est strictement inférieur à 7, car c'est la borne supérieure la plus contraignante.
	La condition est donc équivalente à 
		\begin{align*}
			 4 \leq x < 7 && \iff && x \in [4 ; 7[.
		\end{align*}
}
%%%%%
%%%%% REP GRAPHIQUE
%%%%% 
	


%\exe{}{
%	Considérons deux fonctions $f, g$ sur $\D = ]-3 ; 3[$ données algébriquement par
%		\begin{align*}
%			f(x) = x^2 - 2x && g(x) = (x-1)^2
%		\end{align*}
%	
%	\begin{enumerate}
%		\item Esquisser les représentations graphiques de $f$ et de $g$ dans un même repère.
%		\item Démontrer que $g(x) = f(x) + 1$ pour tout $x$ du domaine.
%	\end{enumerate}
%}{exe:id-rem-graph}{
%	
%	\begin{enumerate}
%		\item On choisit plusieurs valeurs de $x\in]{-}3 ; 3[$ et on représente les points
%			\[ \bigl(x ; f(x) \bigr), \]
%		qu'on relie pour esquisser $\C_f$.
%		On fait idem pour $g$, ce qui donne le graphique ci-dessous.
%		
%		\begin{center}
%		\begin{tikzpicture}[>=stealth]
%			\begin{axis}[xmin = -3.1, xmax=3.1, ymin=-1.1, ymax=15.1, axis x line=middle, axis y line=middle, axis line style=->, grid=both]
%				\addplot[no marks, PURPLE_E, very thick, -] expression[domain=-3:3, samples=100]{x^2 -2*x}
%				node[pos=.45, below=10pt]{$\mathcal{C}_f$};
%				\addplot[no marks, BLUE_E, very thick, -] expression[domain=-3:3, samples=100]{(x-1)^2}
%				node[pos=.4, right]{$\mathcal{C}_g$};
%			\end{axis}
%		
%		\end{tikzpicture}
%		\end{center}
%		
%		
%		\item 
%		L'identité remarquable $a^2 - b^2 = (a+b)(a-b)$ donne en l'occurrence
%			\begin{align*}
%				(x-1)^2 - 1^2 &= (x-1+1) \cdot (x-1-1) \\
%								&= x \cdot (x-2) \\
%								&= x^2 - 2\cdot x = f(x).
%			\end{align*}
%			
%		On déduit alors que 
%			\[ (x-1)^2 = g(x) = f(x) + 1 = x^2 - 2\cdot x + 1 \]
%		pour tout $x\in]{-}3;3[$.
%	\end{enumerate}
%	
%}

%\exe{}{
%	Esquisser la courbe de la fonction $f$ sur $\D=[-5;3]$ donnée algébriquement par
%		\[ f(x) = 1-x. \]
%	Que dire de $\C_f$ ?
%}{exe:graph-droite}{
%	$\C_f$ est une droite et on dit que $f$ est \emph{affine}.
%		\begin{center}
%		\begin{tikzpicture}[>=stealth]
%			\begin{axis}[xmin = -2.1, xmax=4.1, ymin=-3.1, ymax=4.1, axis x line=middle, axis y line=middle, axis line style=->, grid=both]
%				\addplot[no marks, BLUE_E, very thick, -] expression[domain=-2:4, samples=2]{1-x} 
%				node[above, pos=.4] {$\C_f$};
%			\end{axis}
%		\end{tikzpicture}
%		\end{center}
%}

\exe{}{
	Esquisser la courbe de la fonction $f$ sur $\D=[3;10]$ donnée algébriquement par
		\[ f(x) = \dfrac3x + 1. \]
}{exe:graph-droite2}{
	Il faut choisir suffisamment de $x \in [3;10]$ du domaine afin de pouvoir bien tracer l'allure de la courbe.
	Attention, celle-ci n'est pas une droite !
	
		\begin{center}
		\begin{tikzpicture}[>=stealth]
			\begin{axis}[xmin = 3, xmax=10, ymin=1, ymax=2.1, axis x line=middle, axis y line=middle, axis line style=->, grid=both]
				\addplot[no marks, BLUE_E, very thick, -] expression[domain=3:10, samples=50]{3/x + 1} 
				node[above, pos=.4] {$\C_f$};
			\end{axis}
		\end{tikzpicture}
		\end{center}
}

\exe{}{
	Esquisser la courbe de la fonction $f$ sur $\D=[-2; 4]$ donnée algébriquement par
		\[ f(x) = 3. \]
	Que dire de $f$ et de $\C_f$ ?
}{exe:graph-const}{
	Pour tracer $\C_f$, on choisit des valeurs de $x$ du domaine $[{-2};4]$, on calcule $f(x)$, et on place les points $(x ; f(x))$ obtenus.
	
		\begin{center}
		\begin{tikzpicture}[>=stealth]
			\begin{axis}[xmin = -2.1, xmax=4.1, ymin=2, ymax=4, axis x line=middle, axis y line=middle, axis line style=->, grid=both]
				\addplot[no marks, BLUE_E, very thick, -] expression[domain=-2:4, samples=2]{3} 
				node[above, pos=.5] {$\C_f$};
			\end{axis}
		\end{tikzpicture}
		\end{center}
	
	On dit ici que $f$ est \emph{constante} car elle ne dépend pas de $x$.
	$\C_f$ est donc une droite horizontale.
}

\exe{}{
	Considérons la fonction $f$ sur $\R$ donnée algébriquement par
		\[ f(x) = \dfrac17-x. \]
	Pour chaque point suivant, déterminer s'il appartient à $\C_f$ ou non.
	
	\begin{multicols}{3}
	\begin{enumerate}[label=\roman*)]
		\item $\left(0; \frac17\right)$
		\item $\left(-\frac17 ; 0\right)$
		%\item $\left(\dfrac27 ; \dfrac37\right)$
		%\item $\left(-\dfrac{13}7 ; 2\right)$
		%\item $\left(\dfrac67 ; 1\right)$
		\item $\left(\frac27 ; -\frac17\right)$
	\end{enumerate}
	\end{multicols}

}{exe:Cf}{
	On rappelle la propriété fondamentale
		\begin{align*}
			(x;y) \in \C_f && \iff && y = f(x)
		\end{align*}
	Pour savoir si un point $(x;y)$ appartient à $\C_f$, il s'agit de vérifier si l'égalité $y=f(x)$ tient.
	
	\begin{enumerate}[label=\roman*)]
		\item On applique la propriété pour $x = 0, y= \frac17$.
		D'une part, $f(x) = f(0) = \frac17 - 0 = \frac17$, et d'autre part $y=\frac17$.
		On a donc bien $y=f(x)$ pour ce couple, et il appartient à $\C_f$.
			\[ \left(0; \dfrac17\right) \in \C_f. \]
		
		\item On choisit $(x;y) = \left(-\frac17 ; 0\right)$, et on compare $f(x) = f\left(-\frac17\right) = \frac27$ à $y=0$. 
		On a donc
			\[ \left(-\dfrac17 ; 0\right) \not\in \C_f. \]
		%\item On choisit $(x;y) = \left(\dfrac17 ; 0\right)$, et on compare $f(x) = f\left(\dfrac17\right) = 0$ à $y=0$. 
		On a donc bien
			\[ \left(\dfrac17 ; 0\right) \in \C_f. \]
			
		%\item On choisit $(x;y) = \left(\dfrac27 ; \dfrac37\right)$, et on compare $f(x) = f\left(\dfrac27\right) = -\dfrac17$ à $y=\dfrac37$. 
		%D'où
		%	\[ \left(\dfrac27 ; \dfrac37\right) \not\in \C_f. \]
		%	
		%\item On choisit $(x;y) = \left(-\dfrac{13}7 ; 2\right)$, et on compare $f(x) = f\left(-\dfrac{13}7\right) = 2$ à $y=2$. 
		%On a donc bien
		%	\[ \left(-\dfrac{13}7 ; 2\right) \in \C_f. \]
		%\item On choisit $(x;y) = \left(\dfrac67 ; 1\right)$, et on compare $f(x) = f\left(\dfrac67\right) = -\dfrac57$ à $y=1$. 
		%Par suite,
		%	\[ \left(\dfrac67 ; 1\right) \not\in \C_f. \]
		\item On choisit $(x;y) = \left(\frac27 ; -\frac17\right)$, et on compare $f(x) = f\left(\frac27\right) = \frac{-1}7$ à $y=-\frac17$. 
		D'où
			\[ \left(\dfrac27 ; -\dfrac17\right) \not\in \C_f. \]
	\end{enumerate}
}


\exe{, difficulty=1}{
	Utiliser le repère ci-dessous pour comparer les représentations graphiques des trois fonctions suivantes données algébriquement.
		\begin{align*}
			f(x) = x^2, && g(x) = x^2 - 3, && h(x) = (x+4)^2.
		\end{align*}
		
	\begin{center}
	\begin{tikzpicture}[>=stealth]
		\begin{axis}[xmin = -8, xmax=5, ymin=-3.1, ymax=20.1, axis x line=middle, axis y line=middle, axis line style=->, grid=both, clip=true, x=1cm,
		%y=5pt,
		]
		\end{axis}
	\end{tikzpicture}
	\end{center}
}{exe:y-shift}{
	On graphe les fonctions ci-dessous. 
	On remarque que $\C_g$ est juste en dessous de $\C_f$. C'est logique car
		\[ g(x) = f(x) - 3, \]
	donc quand on place les points pour $\C_f$ et $\C_g$, ceux de $\C_g$ sont $3$ en dessous de ceux de $\C_f$.
	
	On remarque aussi que $\C_h$ est la courbe de $\C_f$ décalée vers la gauche.
	C'est également cohérent car
		\[ h(x) = f(x+4). \]
	Pour connaître l'image de $x$ par $h$, on prend l'image de $x+4$ par $f$.
	On a donc par exemple $h(-4) = f(0), h(3,9) = f(0,1),$ etc... 
	On comprend bien pourquoi $\C_h$ est la $\C_f$ décalée de $4$ vers la gauche.
	
		\begin{center}
		\begin{tikzpicture}[>=stealth]
			\begin{axis}[xmin = -8, xmax=5, ymin=-3.1, ymax=20.1, axis x line=middle, axis y line=middle, axis line style=->, grid=both, clip=true, x=1cm]
				\addplot[no marks, BLUE_E, very thick, -] expression[domain=-8:5, samples=50]{x^2}
				node[pos=.6, right]{$\mathcal{C}_f$};
				\addplot[no marks, RED_E, very thick, -] expression[domain=-8:5, samples=50]{x^2-3}
				node[pos=.8, right]{$\mathcal{C}_g$};
				\addplot[no marks, PURPLE_E, very thick, -] expression[domain=-8:5, samples=50]{(x+4)^2}
				node[pos=.1, right]{$\mathcal{C}_h$};
			\end{axis}
		\end{tikzpicture}
		\end{center}
}

%\newpage

\exe{}{
	Considérons la représentation graphique suivante d'une fonction $f$ définie sur \mbox{$\D = ]{-}3,4 ; 2,3[$}.
	
	\begin{center}
		\begin{tikzpicture}[>=stealth]
			\begin{axis}[xmin = -3.4, xmax=2.3, ymin=-2.1, ymax=5.1, axis x line=middle, axis y line=middle, axis line style=->, grid=both,
			grid style = {opacity=.5},
			x=2.1cm,
			xtick={-3, -2, ..., 2},
			y=18pt,
			clip=true,
			]
				\addplot[no marks, BLUE_E, very thick, -] expression[domain=-4:3, samples=200]{x^3 /3 - 2*x +3}node[pos=.52, above=5pt]{$\C_f$};
			\end{axis}
		\end{tikzpicture}
	\end{center}
	\begin{enumerate}
		\item\label{q1}
		Donner approximativement les images de -1 et de 2 par $f$. 
		\item Énumérer approximativement les antécédents de -2 et de 2 par $f$.
		\item Donner approximativement un réel qui admet exactement deux antécédents par $f$.
		\item Si $f$ était définie sur $\R$ tout entier, serait-il toujours possible de connaître l'image de -2 ? Et tous les antécédents de -2 ?
	\end{enumerate}
	Supposons désormais que $f(x) = 3-2 x +\frac13 x^3$ pour tout $x\in\D$ du domaine.
	\begin{enumerate}[resume]
		\item Répondre à nouveau à la question \ref{q1} à l'aide de l'expression algébrique de $f$.
		Des valeurs exactes sont attendues.
		\item Montrer que l'image de $-3$ par $f$ est $0$ et que l'image de $0$ par $f$ est $3$.
	\end{enumerate}
}{exe:deg3}{
	\begin{enumerate}
		\item On détermine approximativement
			\begin{align*}
			f(-1) \approx 4,5 && \et && f(2) \approx 1,5.
			 \end{align*}
		\item On cherche d'abord les antécédents de $-2$, c'est-à-dire les nombres $x$ du domaine vérifiants
			\[ f(x) = -2. \]
		Pour ça, on se pose \og à hauteur -2 \fg : on tracer une droite horizontale d'ordonnée $-2$ et on regarde les points d'intersection.
			\begin{center}
			\begin{tikzpicture}[>=stealth]
				\begin{axis}[xmin = -3.4, xmax=2.3, ymin=-2.1, ymax=5.1, axis x line=middle, axis y line=middle, axis line style=->, grid=both,
				grid style = {opacity=.5},
				x=2cm,
				xtick={-3, -2, ..., 2},
				y=20pt,
				clip=true,
				]
					\addplot[no marks, BLUE_E, very thick, -] expression[domain=-5:3, samples=50]{x^3 /3 - 2*x +3};
					
					\addplot[no marks, RED_E, very thick, -] expression[domain=-5:3, samples=2]{-2};
				\end{axis}
			\end{tikzpicture}
			\end{center}
		
		
		En l'occurrence, seul $x\approx -3,2$ fonctionne :
			\[ f(-3,2) \approx -2. \]
		Pour les antécédents de $2$, on fait idem en regardant les points de la courbe d'ordonnée $2$.
		On trouve trois valeurs approximatives : $-2,8 ; 0,5 ; $ et $2,1$.
		
			\begin{center}
			\begin{tikzpicture}[>=stealth]
				\begin{axis}[xmin = -3.4, xmax=2.3, ymin=-2.1, ymax=5.1, axis x line=middle, axis y line=middle, axis line style=->, grid=both,
				grid style = {opacity=.5},
				x=2cm,
				xtick={-3, -2, ..., 2},
				y=20pt,
				clip=true,
				]
					\addplot[no marks, BLUE_E, very thick, -] expression[domain=-5:3, samples=50]{x^3 /3 - 2*x +3};
					
					\addplot[no marks, RED_E, very thick, -] expression[domain=-5:3, samples=2]{2};
				\end{axis}
			\end{tikzpicture}
			\end{center}
		
		\item 
		On a plusieurs choix ici. 
		Il faut placer une droite horizontale telle qu'elle s'intersecte exactement deux fois avec la courbe de $f$.
		Un choix clair est $4$ (en violet ci-dessous).
		Un choix moins clair est $1,1$, en faisant en sorte que la courbe frôle la droite horizontale qu'on place en ordonnée $1,1$ (en rouge ci-dessous).
		On dit alors que la droite est \emph{tangente} à la courbe.
		
			\begin{center}
			\begin{tikzpicture}[>=stealth]
				\begin{axis}[xmin = -3.4, xmax=2.3, ymin=-2.1, ymax=5.1, axis x line=middle, axis y line=middle, axis line style=->, grid=both,
				grid style = {opacity=.5},
				x=2cm,
				xtick={-3, -2, ..., 2},
				y=20pt,
				clip=true,
				]
					\addplot[no marks, BLUE_E, very thick, -] expression[domain=-5:3, samples=50]{x^3 /3 - 2*x +3};
					
					\addplot[no marks, RED_E, very thick,-] expression[domain=-5:3, samples=2]{1.1};
					\addplot[no marks, PURPLE_E, very thick, -] expression[domain=-5:3, samples=2]{4};
				\end{axis}
			\end{tikzpicture}
			\end{center}
		
		\item L'image est unique et ne dépend pas du domaine (tant que celui-ci contient $-2$ !).
		On peut donc toujours connaître $f(-2)$, même sans connaître $f$ en dehors du domaine.
		
		Les antécédents, eux, dépendent du domaine choisi.
		Comme on ne sait pas du tout à quoi ressemble $\C_f$ en dehors du domaine choisi, il est impossible de déterminer tous les réels $x\in\R$ antécédents de $-2$ par $f$.
		
		\item On calcule grâce à la calculatrice (ou sans...)
			\begin{align*}
				f(-1) = \dfrac{14}3 && f(2) =  \dfrac53.
			\end{align*}
		On peut également utiliser le programme Python ci-après (la notation \texttt{x**3} signifiant $x^3$).
	
		\begin{center}
		\python{images-f}
		\end{center}
		
		\item On calcule à la main que
			\begin{align*}
				f(-3) &= 3 - 2 \cdot (-3) + \dfrac13 \cdot (-3)^3, \\
					&= 3 + 6 - 9, \\
					&= 0.
			\end{align*}
		Pour l'image de $0$, remarquons que seul le terme ne dépendant pas de $x$ subsiste. 
		On l'appelle le terme \emph{constant}, et on trouve $f(0) = 3$, comme requis.
	\end{enumerate}
}


\exemulticols{}{
	Un fonction $f$ est représentée graphiquement ci-contre.
	Parmis les expressions algébriques suivantes, lequelles ne peuvent pas correspondre à $f(x)$ ?
		\begin{multicols}{2}
		\begin{enumerate}[label=\roman*)]
			\item $1-x$
			\item $\dfrac{-1-x}3$
			\item $\left(x+\dfrac13\right)^2$
			\item $-2x - \dfrac23$
		\end{enumerate}
		\end{multicols}
		\vfill\null
}{
	\begin{center}
		\begin{tikzpicture}[>=stealth]
			\begin{axis}[xmin = -3.1, xmax=1.1, ymin=-3.1, ymax=5.1, axis x line=middle, axis y line=middle, axis line style=->, grid=both,
			grid style = {opacity=.5},
			clip=true,
			]
				\addplot[no marks, BLUE_E, very thick, -] expression[domain=-3:2, samples=2]{-2/3 - 2*x}
				node[pos=.3, right]{$\mathcal{C}_f$};
			\end{axis}
		
		\end{tikzpicture}
	\end{center}
}{exe:expr-from-graph}{
	Clairement $f(0) < 0$ est strictement négatif.
	L'expression de $f$ ne peut donc pas être la première ni la troisième.
	
	Ensuite, si l'expression de $f$ était la deuxième, on aurait $f(-1) = 0$, ce qui n'est clairement pas le cas.
	Ainsi $f(x) = -2x - \frac23$ est le seul choix possible.
}

\exe{, difficulty=2}{
	Pour chaque propriété suivante, donner algébriquement une fonction la vérifiant.
		%\begin{multicols}{1}
		\begin{enumerate}
			\item Une image admet exactement un antécédent.
			\item Une image admet exactement deux antécédents.
			\item Une image admet exactement trois antécédents.
			\item Une image admet une infinité d'antécédents.
		\end{enumerate}	
		%\end{multicols}
}{exe:nb-antécédents}{
	\begin{enumerate}
		\item La fonction affine $f(x) = x$ donne une droite dont chaque image admet un unique antécédent.
		\item La fonction affine $f(x) = x^2$ donne une parabole. En résolvant $f(x) =1$, on peut démontrer que seuls $1$ et $-1$ sont les antécédents de $1$.
		\item Considérons $f(x) = (x-1)\cdot(x-2)\cdot(x-3)$. Résoudre $f(x)=0$ pour montrer que seuls $1 ; 2 ;$ et $3$ sont antécédents de $0$.
		\item Une fonction constante fonctionne bien. Par exemple $f(x) = 0$.
	\end{enumerate}	
}

\exe{, difficulty=2}{
	Donner graphiquement une fonction sur $\R$ non constante telle que toutes les images de $f$ admettent un nombre infini d'antécédents.
}{exe:infinité-antécédents}{
	On peut définir par exemple la fonction \emph{signe} en incluant 0 dans les positifs (traditionnellement, $\text{signe}(0) = 0$).
		\[ \text{signe}(x) = \begin{cases*} +1 & si $x \geq 0$, \\
								-1 & si $x<0$.
				\end{cases*}. \]
	Pour une fonction plus intéressante, on pourrait prendre une fonction en vague.
	La fonction sinus fonctionne bien : entrer par exemple \texttt{y=sin(x)} sur Geogebra.
}


%%%%%%%%%%%

\newpage
\fancyhead[C]{\textbf{Solutions}}
\shipoutAnswer
	
\end{document}

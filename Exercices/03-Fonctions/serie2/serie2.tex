%!TEX encoding = UTF8
%!TEX root =notes.tex


%%%%%%%%%%%%%%%%%%%%%%%%%%%%%%%%%
% PACKAGE IMPORTS
%%%%%%%%%%%%%%%%%%%%%%%%%%%%%%%%%


\usepackage[french]{babel}

\usepackage[tmargin=2cm,rmargin=1in,lmargin=1in,margin=0.85in,bmargin=2cm,footskip=.2in]{geometry}
\usepackage{amsmath,amsfonts,amsthm,amssymb,mathtools}
\usepackage[varbb]{newpxmath}
\usepackage{xfrac}
\usepackage[makeroom]{cancel}
\usepackage{mathtools}
\usepackage{bookmark}
\usepackage{enumitem}
\usepackage{hyperref,theoremref}
\hypersetup{
	pdftitle={Assignment},
	colorlinks=true, linkcolor=doc!90,
	bookmarksnumbered=true,
	bookmarksopen=true
}
\usepackage[most,many,breakable]{tcolorbox}
\usepackage{xcolor}
\usepackage{varwidth}
\usepackage{varwidth}
\usepackage{etoolbox}
%\usepackage{authblk}
\usepackage{nameref}
\usepackage{multicol,array}
\usepackage{tikz-cd}
\usepackage[ruled,vlined,linesnumbered]{algorithm2e}
\usepackage{comment} % enables the use of multi-line comments (\ifx \fi) 
\usepackage{import}
\usepackage{xifthen}
\usepackage{pdfpages}
\usepackage{transparent}


\newcommand\mycommfont[1]{\footnotesize\ttfamily\textcolor{blue}{#1}}
\SetCommentSty{mycommfont}
\newcommand{\incfig}[1]{%
    \def\svgwidth{\columnwidth}
    \import{./figures/}{#1.pdf_tex}
}

\usepackage{tikzsymbols}
%\renewcommand\qedsymbol{$\Laughey$}


%\usepackage{import}
%\usepackage{xifthen}
%\usepackage{pdfpages}
%\usepackage{transparent}


%%%%%%%%%%%%%%%%%%%%%%%%%%%%%%
% SELF MADE COLORS
%%%%%%%%%%%%%%%%%%%%%%%%%%%%%%



\definecolor{myg}{RGB}{56, 140, 70}
\definecolor{myb}{RGB}{45, 111, 177}
\definecolor{myr}{RGB}{199, 68, 64}
\definecolor{mytheorembg}{HTML}{F2F2F9}
\definecolor{mytheoremfr}{HTML}{00007B}
\definecolor{mylenmabg}{HTML}{FFFAF8}
\definecolor{mylenmafr}{HTML}{983b0f}
\definecolor{mypropbg}{HTML}{f2fbfc}
\definecolor{mypropfr}{HTML}{191971}
\definecolor{myexamplebg}{HTML}{F2FBF8}
\definecolor{myexamplefr}{HTML}{88D6D1}
\definecolor{myexampleti}{HTML}{2A7F7F}
\definecolor{mydefinitbg}{HTML}{E5E5FF}
\definecolor{mydefinitfr}{HTML}{3F3FA3}
\definecolor{notesgreen}{RGB}{0,162,0}
\definecolor{myp}{RGB}{197, 92, 212}
\definecolor{mygr}{HTML}{2C3338}
\definecolor{myred}{RGB}{127,0,0}
\definecolor{myyellow}{RGB}{169,121,69}
\definecolor{myexercisebg}{HTML}{F2FBF8}
\definecolor{myexercisefg}{HTML}{88D6D1}


%%%%%%%%%%%%%%%%%%%%%%%%%%%%
% TCOLORBOX SETUPS
%%%%%%%%%%%%%%%%%%%%%%%%%%%%

\setlength{\parindent}{1cm}
%================================
% THEOREM BOX
%================================

\tcbuselibrary{theorems,skins,hooks}
\newtcbtheorem[number within=chapter]{Theorem}{Théorème}
{%
	enhanced,
	breakable,
	colback = mytheorembg,
	frame hidden,
	boxrule = 0sp,
	borderline west = {2pt}{0pt}{mytheoremfr},
	sharp corners,
	detach title,
	before upper = \tcbtitle\par\smallskip,
	coltitle = mytheoremfr,
	fonttitle = \bfseries\sffamily,
	description font = \mdseries,
	separator sign none,
	segmentation style={solid, mytheoremfr},
}
{th}


\tcbuselibrary{theorems,skins,hooks}
\newtcolorbox{Theoremcon}
{%
	enhanced
	,breakable
	,colback = mytheorembg
	,frame hidden
	,boxrule = 0sp
	,borderline west = {2pt}{0pt}{mytheoremfr}
	,sharp corners
	,description font = \mdseries
	,separator sign none
}

%================================
% Corollery
%================================
\tcbuselibrary{theorems,skins,hooks}
\newtcbtheorem[use counter=tcb@cnt@Theorem]{Corollary}{Corollaire}
{%
	enhanced
	,breakable
	,colback = myp!10
	,frame hidden
	,boxrule = 0sp
	,borderline west = {2pt}{0pt}{myp!85!black}
	,sharp corners
	,detach title
	,before upper = \tcbtitle\par\smallskip
	,coltitle = myp!85!black
	,fonttitle = \bfseries\sffamily
	,description font = \mdseries
	,separator sign none
	,segmentation style={solid, myp!85!black}
}
{th}

%================================
% LENMA
%================================

\tcbuselibrary{theorems,skins,hooks}
\newtcbtheorem[use counter=tcb@cnt@Theorem]{Lemma}{Lemme}
{%
	enhanced,
	breakable,
	colback = mylenmabg,
	frame hidden,
	boxrule = 0sp,
	borderline west = {2pt}{0pt}{mylenmafr},
	sharp corners,
	detach title,
	before upper = \tcbtitle\par\smallskip,
	coltitle = mylenmafr,
	fonttitle = \bfseries\sffamily,
	description font = \mdseries,
	separator sign none,
	segmentation style={solid, mylenmafr},
}
{th}


%================================
% PROPOSITION
%================================

\tcbuselibrary{theorems,skins,hooks}
\newtcbtheorem[use counter=tcb@cnt@Theorem]{Prop}{Proposition}
{%
	enhanced,
	breakable,
	colback = mypropbg,
	frame hidden,
	boxrule = 0sp,
	borderline west = {2pt}{0pt}{mypropfr},
	sharp corners,
	detach title,
	before upper = \tcbtitle\par\smallskip,
	coltitle = mypropfr,
	fonttitle = \bfseries\sffamily,
	description font = \mdseries,
	separator sign none,
	segmentation style={solid, mypropfr},
}
{th}


%================================
% CLAIM
%================================

\tcbuselibrary{theorems,skins,hooks}
\newtcbtheorem[use counter=tcb@cnt@Theorem]{claim}{Claim}
{%
	enhanced
	,breakable
	,colback = myg!10
	,frame hidden
	,boxrule = 0sp
	,borderline west = {2pt}{0pt}{myg}
	,sharp corners
	,detach title
	,before upper = \tcbtitle\par\smallskip
	,coltitle = myg!85!black
	,fonttitle = \bfseries\sffamily
	,description font = \mdseries
	,separator sign none
	,segmentation style={solid, myg!85!black}
}
{th}



%================================
% Exercise
%================================

\tcbuselibrary{theorems,skins,hooks}
\newtcbtheorem[use counter=tcb@cnt@Theorem]{Exercise}{Exercice}
{%
	enhanced,
	breakable,
	colback = myexercisebg,
	frame hidden,
	boxrule = 0sp,
	borderline west = {2pt}{0pt}{myexercisefg},
	sharp corners,
	detach title,
	before upper = \tcbtitle\par\smallskip,
	coltitle = myexercisefg,
	fonttitle = \bfseries\sffamily,
	description font = \mdseries,
	separator sign none,
	segmentation style={solid, myexercisefg},
}
{th}

%================================
% EXAMPLE BOX
%================================

\newtcbtheorem[use counter=tcb@cnt@Theorem]{Example}{Exemple}
{%
	colback = myexamplebg
	,breakable
	,colframe = myexamplefr
	,coltitle = myexampleti
	,boxrule = 1pt
	,sharp corners
	,detach title
	,before upper=\tcbtitle\par\smallskip
	,fonttitle = \bfseries
	,description font = \mdseries
	,separator sign none
	,description delimiters parenthesis
}
{ex}

%================================
% DEFINITION BOX
%================================

\newtcbtheorem[use counter=tcb@cnt@Theorem]{Definition}{Définition}{enhanced,
	before skip=2mm,after skip=2mm, colback=red!5,colframe=red!80!black,boxrule=0.5mm,
	attach boxed title to top left={xshift=1cm,yshift*=1mm-\tcboxedtitleheight}, varwidth boxed title*=-3cm,
	boxed title style={frame code={
					\path[fill=tcbcolback]
					([yshift=-1mm,xshift=-1mm]frame.north west)
					arc[start angle=0,end angle=180,radius=1mm]
					([yshift=-1mm,xshift=1mm]frame.north east)
					arc[start angle=180,end angle=0,radius=1mm];
					\path[left color=tcbcolback!60!black,right color=tcbcolback!60!black,
						middle color=tcbcolback!80!black]
					([xshift=-2mm]frame.north west) -- ([xshift=2mm]frame.north east)
					[rounded corners=1mm]-- ([xshift=1mm,yshift=-1mm]frame.north east)
					-- (frame.south east) -- (frame.south west)
					-- ([xshift=-1mm,yshift=-1mm]frame.north west)
					[sharp corners]-- cycle;
				},interior engine=empty,
		},
	fonttitle=\bfseries,
	title={#2},#1}{def}

%================================
% Solution BOX
%================================

\makeatletter
\newtcbtheorem[use counter=tcb@cnt@Theorem]{question}{Question}{enhanced,
	breakable,
	colback=white,
	colframe=myb!80!black,
	attach boxed title to top left={yshift*=-\tcboxedtitleheight},
	fonttitle=\bfseries,
	title={#2},
	boxed title size=title,
	boxed title style={%
			sharp corners,
			rounded corners=northwest,
			colback=tcbcolframe,
			boxrule=0pt,
		},
	underlay boxed title={%
			\path[fill=tcbcolframe] (title.south west)--(title.south east)
			to[out=0, in=180] ([xshift=5mm]title.east)--
			(title.center-|frame.east)
			[rounded corners=\kvtcb@arc] |-
			(frame.north) -| cycle;
		},
	#1
}{def}
\makeatother

%================================
% SOLUTION BOX
%================================

\makeatletter
\newtcolorbox{solution}{enhanced,
	breakable,
	colback=white,
	colframe=myg!80!black,
	attach boxed title to top left={yshift*=-\tcboxedtitleheight},
	title=Solution,
	boxed title size=title,
	boxed title style={%
			sharp corners,
			rounded corners=northwest,
			colback=tcbcolframe,
			boxrule=0pt,
		},
	underlay boxed title={%
			\path[fill=tcbcolframe] (title.south west)--(title.south east)
			to[out=0, in=180] ([xshift=5mm]title.east)--
			(title.center-|frame.east)
			[rounded corners=\kvtcb@arc] |-
			(frame.north) -| cycle;
		},
}
\makeatother

%================================
% Question BOX
%================================

\makeatletter
\newtcbtheorem[use counter=tcb@cnt@Theorem]{qstion}{Question}{enhanced,
	breakable,
	colback=white,
	colframe=mygr,
	attach boxed title to top left={yshift*=-\tcboxedtitleheight},
	fonttitle=\bfseries,
	title={#2},
	boxed title size=title,
	boxed title style={%
			sharp corners,
			rounded corners=northwest,
			colback=tcbcolframe,
			boxrule=0pt,
		},
	underlay boxed title={%
			\path[fill=tcbcolframe] (title.south west)--(title.south east)
			to[out=0, in=180] ([xshift=5mm]title.east)--
			(title.center-|frame.east)
			[rounded corners=\kvtcb@arc] |-
			(frame.north) -| cycle;
		},
	#1
}{def}
\makeatother

\newtcbtheorem[number within=chapter]{wconc}{Wrong Concept}{
	breakable,
	enhanced,
	colback=white,
	colframe=myr,
	arc=0pt,
	outer arc=0pt,
	fonttitle=\bfseries\sffamily\large,
	colbacktitle=myr,
	attach boxed title to top left={},
	boxed title style={
			enhanced,
			skin=enhancedfirst jigsaw,
			arc=3pt,
			bottom=0pt,
			interior style={fill=myr}
		},
	#1
}{def}



%================================
% NOTE BOX
%================================

\usetikzlibrary{arrows,calc,shadows.blur}
\tcbuselibrary{skins}
\newtcolorbox{note}[1][]{%
	enhanced jigsaw,
	colback=gray!20!white,%
	colframe=gray!80!black,
	size=small,
	boxrule=1pt,
	title=\colorbox{white!100}{\textbf{ Remarque }},
	halign title=flush center,
	coltitle=black,
	breakable,
	drop shadow=black!50!white,
	attach boxed title to top left={xshift=1cm,yshift=-\tcboxedtitleheight/2,yshifttext=-\tcboxedtitleheight/2},
	minipage boxed title=2.6cm,
	boxed title style={%
			colback=white,
			size=fbox,
			boxrule=1pt,
			boxsep=2pt,
			underlay={%
					\coordinate (dotA) at ($(interior.west) + (-0.5pt,0)$);
					\coordinate (dotB) at ($(interior.east) + (0.5pt,0)$);
					\begin{scope}
						\clip (interior.north west) rectangle ([xshift=3ex]interior.east);
						\filldraw [white, blur shadow={shadow opacity=60, shadow yshift=-.75ex}, rounded corners=2pt] (interior.north west) rectangle (interior.south east);
					\end{scope}
					\begin{scope}[gray!80!black]
						\fill (dotA) circle (2pt);
						\fill (dotB) circle (2pt);
					\end{scope}
				},
		},
	#1,
}

%================================
% STRATÉGIE BOX
%================================

\usetikzlibrary{arrows,calc,shadows.blur}
\tcbuselibrary{skins}
\newtcolorbox{strategy}[1][]{%
	enhanced jigsaw,
	colback=myb!20!white,%
	colframe=gray!80!black,
	size=small,
	boxrule=1pt,
	title=\colorbox{white!100}{\textbf{ Stratégie }},
	halign title=flush center,
	coltitle=black,
	breakable,
	drop shadow=black!50!white,
	attach boxed title to top left={xshift=1cm,yshift=-\tcboxedtitleheight/2,yshifttext=-\tcboxedtitleheight/2},
	minipage boxed title=2.5cm,
	boxed title style={%
			colback=white,
			size=fbox,
			boxrule=1pt,
			boxsep=2pt,
			underlay={%
					\coordinate (dotA) at ($(interior.west) + (-0.5pt,0)$);
					\coordinate (dotB) at ($(interior.east) + (0.5pt,0)$);
					\begin{scope}
						\clip (interior.north west) rectangle ([xshift=3ex]interior.east);
						\filldraw [white, blur shadow={shadow opacity=60, shadow yshift=-.75ex}, rounded corners=2pt] (interior.north west) rectangle (interior.south east);
					\end{scope}
					\begin{scope}[gray!80!black]
						\fill (dotA) circle (2pt);
						\fill (dotB) circle (2pt);
					\end{scope}
				},
		},
	#1,
}

%================================
% MÉTHODE BOX
%================================

\usetikzlibrary{arrows,calc,shadows.blur}
\tcbuselibrary{skins}
\newtcolorbox{methode}[1][]{%
	enhanced jigsaw,
	colback=white,%
	colframe=gray!80!black,
	size=small,
	boxrule=1pt,
	title=\textbf{Méthode},
	halign title=flush center,
	coltitle=black,
	breakable,
	drop shadow=black!50!white,
	attach boxed title to top left={xshift=1cm,yshift=-\tcboxedtitleheight/2,yshifttext=-\tcboxedtitleheight/2},
	minipage boxed title=2.5cm,
	boxed title style={%
			colback=white,
			size=fbox,
			boxrule=1pt,
			boxsep=2pt,
			underlay={%
					\coordinate (dotA) at ($(interior.west) + (-0.5pt,0)$);
					\coordinate (dotB) at ($(interior.east) + (0.5pt,0)$);
					\begin{scope}
						\clip (interior.north west) rectangle ([xshift=3ex]interior.east);
						\filldraw [white, blur shadow={shadow opacity=60, shadow yshift=-.75ex}, rounded corners=2pt] (interior.north west) rectangle (interior.south east);
					\end{scope}
					\begin{scope}[gray!80!black]
						\fill (dotA) circle (2pt);
						\fill (dotB) circle (2pt);
					\end{scope}
				},
		},
	#1,
}

%%%%%%%%%%%%%%%%%%%%%%%%%%%%%%%%%%%%%%%%%%%
% TABLE OF CONTENTS
%%%%%%%%%%%%%%%%%%%%%%%%%%%%%%%%%%%%%%%%%%%

\usepackage{tikz}

\definecolor{doc}{RGB}{0,60,110}
\usepackage{titletoc}
\contentsmargin{0cm}
\titlecontents{chapter}[3.7pc]
{\addvspace{30pt}%
	\begin{tikzpicture}[remember picture, overlay]%
		\draw[fill=doc!60,draw=doc!60] (-7,-.1) rectangle (-0.2,.6);%
		\pgftext[left,x=-3.5cm,y=0.2cm]{\color{white}\Large\sc\bfseries Chapitre\ \thecontentslabel};%
	\end{tikzpicture}\color{doc!60}\large\sc\bfseries}%
{}
{}
{\;\titlerule\;\large\sc\bfseries Page \thecontentspage
	\begin{tikzpicture}[remember picture, overlay]
		\draw[fill=doc!60,draw=doc!60] (2pt,0) rectangle (4,0.1pt);
	\end{tikzpicture}}%
\titlecontents{section}[3.7pc]
{\addvspace{2pt}}
{\contentslabel[\thecontentslabel]{2pc}}
{}
{\hfill\small \thecontentspage}
[]
\titlecontents*{subsection}[3.7pc]
{\addvspace{-1pt}\small}
{}
{}
{\ --- \small\thecontentspage}
[ \textbullet\ ][]

\makeatletter
\renewcommand{\tableofcontents}{%
	\chapter*{%
	  \vspace*{-20\p@}%
	  \begin{tikzpicture}[remember picture, overlay]%
		  \pgftext[right,x=15cm,y=0.2cm]{\color{doc!60}\Huge\sc\bfseries \contentsname};%
		  \draw[fill=doc!60,draw=doc!60] (13,-.75) rectangle (20,1);%
		  \clip (13,-.75) rectangle (20,1);
		  \pgftext[right,x=15cm,y=0.2cm]{\color{white}\Huge\sc\bfseries \contentsname};%
	  \end{tikzpicture}}%
	\@starttoc{toc}}
\makeatother


%%%%%%%%%%%%%%%%%%%%%%%%%%%%%%%%%%%%%%%%%%%
% MINTED FOR PYTHON ALGORITHMS
%%%%%%%%%%%%%%%%%%%%%%%%%%%%%%%%%%%%%%%%%%%

\usepackage{tcolorbox}
\tcbuselibrary{minted,breakable,xparse,skins}
\definecolor{bg}{gray}{0.95}
\DeclareTCBListing{mintedbox}{O{}m!O{}}{%
  breakable=true,
  listing engine=minted,
  listing only,
  minted language=#2,
  minted style=default,
  minted options={%
    linenos,
    gobble=0,
    breaklines=true,
    breakafter=,,
    fontsize=\small,
    numbersep=8pt,
    #1},
  boxsep=0pt,
  left skip=0pt,
  right skip=0pt,
  left=25pt,
  right=0pt,
  top=3pt,
  bottom=3pt,
  arc=5pt,
  leftrule=0pt,
  rightrule=0pt,
  bottomrule=2pt,
  toprule=2pt,
  colback=bg,
  colframe=orange!70,
  enhanced,
  overlay={%
    \begin{tcbclipinterior}
    \fill[orange!20!white] (frame.south west) rectangle ([xshift=20pt]frame.north west);
    \end{tcbclipinterior}},
  #3}
  
  
 % for braces
\usetikzlibrary{decorations.pathreplacing}


\SetDate[12/11/2025]
\usepackage{minted}

\begin{document}
\pagestyle{fancy}
\fancyhead[L]{Seconde}
\fancyhead[C]{\textbf{Fonctions 2}}
\fancyhead[R]{\today}

% idée originale : © Benjamin Clamme
\exemulticols{}{
	On considère le programme de calcul ci-contre.
	\begin{enumerate}
		\item
		Que renvoie le programme lorsqu'on lui donne 1 ?
		\item 
		Compléter l'expression obtenue après chaque instruction du programme.
		\item
		Développer et réduire l'expression finale.
		\item
		Quel nombre donner au programme pour obtenir 0 ?
	\end{enumerate}
}{
	\begin{center}
\def\arraystretch{2}
\setlength\tabcolsep{20pt}
	\begin{tabular}{ |  c | c |  } \hline
	 Instruction & Expression \\ \hline
	 Choisir un nombre & $x$ \\ \hline 
	 Ajouter 3  &  \\ \hline
	 Diviser par 3 & \\ \hline 
	 Ajouter 1  &  \\ \hline
	\end{tabular}
	\end{center}	
}{exe:programme-calcul}{
	\begin{center}
	\begin{tabular}{ |  c | c |  } \hline
	 Instruction & Expression \\ \hline
	 Choisir un nombre & $x$ \\ \hline 
	 Ajouter 3  & $x+3$ \\ \hline
	 Diviser par 3 & $\frac13(x+3)$ \\ \hline 
	 Ajouter 1  &  $\frac13(x+3) + 1$ \\ \hline
	\end{tabular}
	\end{center}	
	À noter que les deux expressions suivantes sont égales :
		\[ \dfrac{x+3}{3} = \dfrac13 (x+3). \]	
	Posons $f(x) = \frac13(x+3) + 1$ dans la suite.
	\begin{enumerate}
		\item
		$f(1) = \frac13 \times 4 + 1 = \frac73$.
		\item[]
		\item
		\begin{align*}	
			f(x) &= \dfrac13(x+3) + 1, \\
				&= \dfrac13 x + \dfrac13 \times 3 + 1, \\
				&= \dfrac13x + 2.
		\end{align*}
		À noter que distribuer $\frac13$ revient à écrire que
			\[  \dfrac{x+3}{3} =  \dfrac{x}{3} +  \dfrac{3}{3}, \]
		qui est une propriété classique des fractions.
		\item
		On se demande quel $x$ vérifie $f(x) = 0$.
		On peut soit remonter le programme de calcul, soit résoudre l'équation algébriquement.
		Il s'avère que ces deux méthodes sont strictement identiques si on garde la forme $f(x) = \frac13(x+3) + 1$.
		
		Avec la forme développée :
			\begin{align*}
				f(x) &= 0, \\
				\dfrac13 x + 2 &= 0, \\
				\dfrac13 x &= - 2, \\
				x &= -2 \times 3 = -6.
			\end{align*}
		Avec la forme du tableau : 
			\begin{align*}
				f(x) &= 0, \\
				\dfrac13 (x+3) + 1 &= 0, \\
				\dfrac13 (x+3) &= - 1, \\
				x+3 &= -3, \\
				x &= - 6.
			\end{align*}
		Chaque opération faite permet de remonter un étape du programme de calcul : on soustrait 1, puis on multiplie par 3, et enfin on soustrait 3.
	\end{enumerate}
}

%%%%%
%%%%% INTERVALLES
%%%%% 

% for later maybe :)
% volé à © Benjamin Clamme (goat)
%\exe{}{
%	Traduire l'appartenance de $x\in\R$ à chaque intervalle ci-dessous en termes d'inégalités :
%	\begin{multicols}{2}
%	\begin{enumerate}[label=(\alph*)]
%		\item $I_1 = ]-3;7[$
%		\item $I_2 = [4;12]$
%		\item $I_3 = ]-\infty;0]$
%		\item $I_4 = [-3;4[$
%	\end{enumerate}
%	\end{multicols}
%	Tracer la droite réelle avec les intervalles $I_1$ et $I_2$ coloriés.
%	Donner une condition pour que $x$ appartienne à l'intervalle $I_1$ \textbf{ET} à l'intervalle $I_2$.
%	Exprimer ce résultat sous forme d'intervalle.
%	%On note $I_1 \cap I_2$ l'\emph{intersection} des intervalles : ce sont les nombres qui appartiennent à la fois à $I_1$ et à $I_2$.
%}{exe:intervalles}{
%	\begin{enumerate}[label=(\alph*)]
%		\item 
%			\begin{align*}
%				x \in ]-3;7[ && \iff && -3 < x < 7
%			\end{align*}
%		\item 
%			\begin{align*}
%				x \in [4;12] && \iff && 4 \leq x  \leq 12
%			\end{align*}
%		\item 
%			\begin{align*}
%				x \in ]-\infty;0] && \iff && x \leq 0
%			\end{align*}
%		\item 
%			\begin{align*}
%				x \in [-3;4[ && \iff && -3 \leq x \leq 4
%			\end{align*}
%	\end{enumerate}
%	
%	Si $x$ appartient à $I_1$ et à $I_2$, alors il vérifie à la fois
%		\begin{align*}
%			 -3 < x < 7 && \et && 4 \leq x  \leq 12.
%		\end{align*}
%	$x$ est donc supérieur ou égal à 4, car c'est la borne inférieure la plus contraignante.
%	$x$ est strictement inférieur à 7, car c'est la borne supérieure la plus contraignante.
%	La condition est donc équivalente à 
%		\begin{align*}
%			 4 \leq x < 7 && \iff && x \in [4 ; 7[.
%		\end{align*}
%}

%%%%%
%%%%% REP GRAPHIQUE
%%%%% 
	


% for later maybe :)
%\exe{}{
%	Considérons deux fonctions $f, g$ sur $\D = ]-3 ; 3[$ données algébriquement par
%		\begin{align*}
%			f(x) = x^2 - 2x && g(x) = (x-1)^2
%		\end{align*}
%	
%	\begin{enumerate}
%		\item Esquisser les représentations graphiques de $f$ et de $g$ dans un même repère.
%		\item Démontrer que $g(x) = f(x) + 1$ pour tout $x$ du domaine.
%	\end{enumerate}
%}{exe:id-rem-graph}{
%	
%	\begin{enumerate}
%		\item On choisit plusieurs valeurs de $x\in]{-}3 ; 3[$ et on représente les points
%			\[ \bigl(x ; f(x) \bigr), \]
%		qu'on relie pour esquisser $\C_f$.
%		On fait idem pour $g$, ce qui donne le graphique ci-dessous.
%		
%		\begin{center}
%		\begin{tikzpicture}[>=stealth]
%			\begin{axis}[xmin = -3.1, xmax=3.1, ymin=-1.1, ymax=15.1, axis x line=middle, axis y line=middle, axis line style=->, grid=both]
%				\addplot[no marks, PURPLE_E, very thick, -] expression[domain=-3:3, samples=100]{x^2 -2*x}
%				node[pos=.45, below=10pt]{$\mathcal{C}_f$};
%				\addplot[no marks, BLUE_E, very thick, -] expression[domain=-3:3, samples=100]{(x-1)^2}
%				node[pos=.4, right]{$\mathcal{C}_g$};
%			\end{axis}
%		
%		\end{tikzpicture}
%		\end{center}
%		
%		
%		\item 
%		L'identité remarquable $a^2 - b^2 = (a+b)(a-b)$ donne en l'occurrence
%			\begin{align*}
%				(x-1)^2 - 1^2 &= (x-1+1) \cdot (x-1-1) \\
%								&= x \cdot (x-2) \\
%								&= x^2 - 2\cdot x = f(x).
%			\end{align*}
%			
%		On déduit alors que 
%			\[ (x-1)^2 = g(x) = f(x) + 1 = x^2 - 2\cdot x + 1 \]
%		pour tout $x\in]{-}3;3[$.
%	\end{enumerate}
%	
%}

% for later maybe :)
%\exe{}{
%	Esquisser la courbe de la fonction $f$ sur $\D=[-5;3]$ donnée algébriquement par
%		\[ f(x) = 1-x. \]
%	Que dire de $\C_f$ ?
%}{exe:graph-droite}{
%	$\C_f$ est une droite et on dit que $f$ est \emph{affine}.
%		\begin{center}
%		\begin{tikzpicture}[>=stealth]
%			\begin{axis}[xmin = -2.1, xmax=4.1, ymin=-3.1, ymax=4.1, axis x line=middle, axis y line=middle, axis line style=->, grid=both]
%				\addplot[no marks, BLUE_E, very thick, -] expression[domain=-2:4, samples=2]{1-x} 
%				node[above, pos=.4] {$\C_f$};
%			\end{axis}
%		\end{tikzpicture}
%		\end{center}
%}

\exe{}{
	Esquisser la courbe de la fonction $f$ sur $\D=[-2; 4]$ donnée algébriquement par
		\[ f(x) = 3. \]
	Que dire de $f$ et de $\C_f$ ?
}{exe:graph-const}{
	Pour tracer $\C_f$, on choisit des valeurs de $x$ du domaine $[{-2};4]$, on calcule $f(x)$, et on place les points $(x ; f(x))$ obtenus.
	
		\begin{center}
		\begin{tikzpicture}[>=stealth]
			\begin{axis}[xmin = -2.1, xmax=4.1, ymin=2, ymax=4, axis x line=middle, axis y line=middle, axis line style=->, grid=both]
				\addplot[no marks, BLUE_E, very thick, -] expression[domain=-2:4, samples=2]{3} 
				node[above, pos=.5] {$\C_f$};
			\end{axis}
		\end{tikzpicture}
		\end{center}
	
	On dit ici que $f$ est \emph{constante} car elle ne dépend pas de $x$.
	$\C_f$ est donc une droite horizontale.
}

\exe{, difficulty=1}{
	Esquisser la courbe de la fonction $f$ sur $\D=[3;10]$ donnée algébriquement par
		\[ f(x) = \dfrac3x + 1. \]
}{exe:graph-droite2}{
	Il faut choisir suffisamment de $x \in [3;10]$ du domaine afin de pouvoir bien tracer l'allure de la courbe.
	Attention, celle-ci n'est pas une droite !
	
		\begin{center}
		\begin{tikzpicture}[>=stealth]
			\begin{axis}[xmin = 3, xmax=10, ymin=1, ymax=2.1, axis x line=middle, axis y line=middle, axis line style=->, grid=both]
				\addplot[no marks, BLUE_E, very thick, -] expression[domain=3:10, samples=50]{3/x + 1} 
				node[above, pos=.4] {$\C_f$};
			\end{axis}
		\end{tikzpicture}
		\end{center}
}

\exe{}{
	Considérons la fonction $f$ sur $\R$ donnée algébriquement par
		\[ f(x) = \dfrac17-x. \]
	Pour chaque point suivant, déterminer s'il appartient à $\C_f$ ou non.
	
	\begin{multicols}{3}
	\begin{enumerate}[label=\roman*)]
		\item $\left(0; \frac17\right)$
		\item $\left(-\frac17 ; 0\right)$
		\item $\left(\frac27 ; -\frac17\right)$
	\end{enumerate}
	\end{multicols}

}{exe:Cf}{
	On rappelle la propriété fondamentale
		\begin{align*}
			(x;y) \in \C_f && \iff && y = f(x)
		\end{align*}
	Pour savoir si un point $(x;y)$ appartient à $\C_f$, il s'agit de vérifier si l'égalité $y=f(x)$ tient.
	
	\begin{enumerate}[label=\roman*)]
		\item On applique la propriété pour $x = 0, y= \frac17$.
		D'une part, $f(x) = f(0) = \frac17 - 0 = \frac17$, et d'autre part $y=\frac17$.
		On a donc bien $y=f(x)$ pour ce couple, et il appartient à $\C_f$.
			\[ \left(0; \dfrac17\right) \in \C_f. \]
		
		\item On choisit $(x;y) = \left(-\frac17 ; 0\right)$, et on compare $f(x) = f\left(-\frac17\right) = \frac27$ à $y=0$. 
		On a donc
			\[ \left(-\dfrac17 ; 0\right) \not\in \C_f. \]
		On a donc bien
			\[ \left(\dfrac17 ; 0\right) \in \C_f. \]
		\item On choisit $(x;y) = \left(\frac27 ; -\frac17\right)$, et on compare $f(x) = f\left(\frac27\right) = \frac{-1}7$ à $y=-\frac17$. 
		D'où
			\[ \left(\dfrac27 ; -\dfrac17\right) \not\in \C_f. \]
	\end{enumerate}
}


\exe{, difficulty=2}{
	Utiliser le repère ci-dessous pour comparer les représentations graphiques des trois fonctions suivantes données algébriquement.
		\begin{align*}
			f(x) = x^2, && g(x) = x^2 - 3, && h(x) = (x+4)^2.
		\end{align*}
		
	\begin{center}
	\begin{tikzpicture}[>=stealth]
		\begin{axis}[xmin = -8, xmax=5, ymin=-5.1, ymax=20.1, axis x line=middle, axis y line=middle, axis line style=->, grid=both, clip=true, x=1cm]
		\end{axis}
	\end{tikzpicture}
	\end{center}
	
	\begin{enumerate}[label=(\alph*)]
		\item
		Estimer, grâces aux courbes, les solutions $x\in[-8;5]$ de l'équation $f(x) = h(x)$.
		Vérifier l'exactitude des solutions en calculant et en comparant $f(x)$ et $g(x)$.
		\item
		Faire de même pour les solutions $x\in[-8;5]$ de $g(x) = h(x)$.
		\item
		Estimer, grâces aux courbes, à quel intervalle appartiennent les $x\in[-8;5]$ vérifiant $h(x) \leq f(x)$.
	\end{enumerate}
}{exe:y-shift}{
	On graphe les fonctions ci-dessous. 
	On remarque que $\C_g$ est juste en dessous de $\C_f$. C'est logique car
		\[ g(x) = f(x) - 3, \]
	donc quand on place les points pour $\C_f$ et $\C_g$, ceux de $\C_g$ sont $3$ en dessous de ceux de $\C_f$.
	
	On remarque aussi que $\C_h$ est la courbe de $\C_f$ décalée vers la gauche.
	C'est également cohérent car
		\[ h(x) = f(x+4). \]
	Pour connaître l'image de $x$ par $h$, on prend l'image de $x+4$ par $f$.
	On a donc par exemple $h(-4) = f(0), h(3,9) = f(0,1),$ etc... 
	On comprend bien pourquoi $\C_h$ est la $\C_f$ décalée de $4$ vers la gauche.
	
		\begin{center}
		\begin{tikzpicture}[>=stealth]
			\begin{axis}[xmin = -8, xmax=5, ymin=-5.1, ymax=20.1, axis x line=middle, axis y line=middle, axis line style=->, grid=both, clip=true, x=1cm]
				\addplot[no marks, BLUE_E, very thick, -] expression[domain=-8:5, samples=50]{x^2}
				node[pos=.6, right]{$\mathcal{C}_f$};
				\addplot[no marks, RED_E, very thick, -] expression[domain=-8:5, samples=50]{x^2-3}
				node[pos=.8, right]{$\mathcal{C}_g$};
				\addplot[no marks, PURPLE_E, very thick, -] expression[domain=-8:5, samples=50]{(x+4)^2}
				node[pos=.1, right]{$\mathcal{C}_h$};
			\end{axis}
		\end{tikzpicture}
		\end{center}

	\begin{enumerate}[label=(\alph*)]
		\item
		Les $x\in[-8;5]$ vérifiant $f(x)=h(x)$ sont les abscisses des points d'intersection de $\C_f$ et $\C_h$.
		Ici, seul $x\approx -2$ fonctionne.
		On calclue $f(-2) = 4$ et $h(-2) = 4$, et on en déduite que $x=-2$ exactement.
		
		\item
		Les $x\in[-8;5]$ vérifiant $h(x)=g(x)$ sont les abscisses des points d'intersection de $\C_h$ et $\C_g$.
		Ici, seul $x\approx -2,5$ fonctionne.
		On calcule $h(-2,5) = 1,5^2 = \frac94$ et $g(-2,5) = 2,5^2 - 3 = \frac{25}4 - 3 = \frac{13}4$ et donc que la solution trouvée n'est pas exacte.
		\item
		Les $x\in[-8;5]$ vérifiant $h(x) \leq f(x)$ sont les abscisses pour lesquelles la courbe $\C_h$ est sous la courbe $\C_f$.
		Ici, on trouve approximativement $[-8 ; -2]$.
	\end{enumerate}
}

\newpage

\exe{, difficulty=2}{
	Considérons la représentation graphique suivante d'une fonction $f$ définie sur \mbox{$\D = ]{-}3,4 ; 2,3[$}.
	
	\begin{center}
		\begin{tikzpicture}[>=stealth]
			\begin{axis}[xmin = -3.4, xmax=2.3, ymin=-2.1, ymax=5.1, axis x line=middle, axis y line=middle, axis line style=->, grid=both,
			grid style = {opacity=.5},
			x=2.1cm,
			xtick={-3, -2, ..., 2},
			y=18pt,
			clip=true,
			]
				\addplot[no marks, BLUE_E, very thick, -] expression[domain=-4:3, samples=200]{x^3 /3 - 2*x +3}node[pos=.52, above=5pt]{$\C_f$};
			\end{axis}
		\end{tikzpicture}
	\end{center}
	\begin{enumerate}
		\item\label{q1}
		Donner approximativement les images de -1 et de 2 par $f$. 
		\item Énumérer approximativement les antécédents de 2 et de 4 par $f$.
		\item Exprimer aproximativement l'ensemble $\{ x \in \D \tq f(x) \geq 4 \}$ sous forme d'intervalle.
		\item Donner approximativement un réel qui admet exactement deux antécédents par $f$.
		\item Si $f$ était définie sur $\R$ tout entier, serait-il toujours possible de connaître l'image de -2 ? Et tous les antécédents de -2 ?
	\end{enumerate}
	Supposons désormais que $f(x) = 3-2 x +\frac13 x^3$ pour tout $x\in\D$ du domaine.
	\begin{enumerate}[resume]
		\item Répondre à nouveau à la question \ref{q1} à l'aide de l'expression algébrique de $f$.
		Des valeurs exactes sont attendues.
		\item Montrer que l'image de $-3$ par $f$ est $0$ et que l'image de $0$ par $f$ est $3$.
	\end{enumerate}
}{exe:deg3}{
	\begin{enumerate}
		\item On détermine approximativement
			\begin{align*}
			f(-1) \approx 4,5 && \et && f(2) \approx 1,5.
			 \end{align*}
		\item On cherche d'abord les antécédents de 2, c'est-à-dire les nombres $x$ du domaine vérifiants
			\[ f(x) = -2. \]
		Pour ça, on se pose \og à hauteur 2 \fg : on tracer une droite horizontale d'ordonnée 2 et on regarde les points d'intersection.
			\begin{center}
			\begin{tikzpicture}[>=stealth]
				\begin{axis}[xmin = -3.4, xmax=2.3, ymin=-2.1, ymax=5.1, axis x line=middle, axis y line=middle, axis line style=->, grid=both,
				grid style = {opacity=.5},
				x=2cm,
				xtick={-3, -2, ..., 2},
				y=20pt,
				clip=true,
				]
					\addplot[no marks, BLUE_E, very thick, -] expression[domain=-5:3, samples=50]{x^3 /3 - 2*x +3};
					
					\addplot[no marks, RED_E, very thick, -] expression[domain=-5:3, samples=2]{2};
				\end{axis}
			\end{tikzpicture}
			\end{center}
		
		
		En l'occurrence, seuls $x\approx -2,8 ; 0,5 ;  \et 2,1$ fonctionnent.
			
		Pour les antécédents de 4, on fait idem en regardant les points de la droite d'ordonnée 4.
			\begin{center}
			\begin{tikzpicture}[>=stealth]
				\begin{axis}[xmin = -3.4, xmax=2.3, ymin=-2.1, ymax=5.1, axis x line=middle, axis y line=middle, axis line style=->, grid=both,
				grid style = {opacity=.5},
				x=2cm,
				xtick={-3, -2, ..., 2},
				y=20pt,
				clip=true,
				]
					\addplot[no marks, BLUE_E, very thick, -] expression[domain=-5:3, samples=50]{x^3 /3 - 2*x +3};
					
					\addplot[no marks, RED_E, very thick, -] expression[domain=-5:3, samples=2]{4};
				\end{axis}
			\end{tikzpicture}
			\end{center}
		On trouve trois valeurs approximatives : $x \approx -2,2$ et $x \approx 0,5$.
		
		\item
		On regarde quels antécédents $x \in \D$ du domaine d'étude ont une image supérieure ou égale à 4.
		En reprenant le graphe de la résolution de $f(x) = 4$, on remarque que \textbf{tous} les $x$ entre -2,2 et 0,5 ont une image supérieure à 4.
		Ainsi, 
			\[ \bigset{ x \in \D \tq f(x) \geq 4 } \approx [-2,2 ; 0,5], \]
		les bornes étant incluses car l'inégalité est large.
		
		\item 
		On a plusieurs choix ici. 
		Il faut placer une droite horizontale telle qu'elle s'intersecte exactement deux fois avec la courbe de $f$.
		Un choix clair est $4$ (en violet ci-dessous).
		Un choix moins clair est $1,1$, en faisant en sorte que la courbe frôle la droite horizontale qu'on place en ordonnée $1,1$ (en rouge ci-dessous).
		On dit alors que la droite est \emph{tangente} à la courbe.
		
			\begin{center}
			\begin{tikzpicture}[>=stealth]
				\begin{axis}[xmin = -3.4, xmax=2.3, ymin=-2.1, ymax=5.1, axis x line=middle, axis y line=middle, axis line style=->, grid=both,
				grid style = {opacity=.5},
				x=2cm,
				xtick={-3, -2, ..., 2},
				y=20pt,
				clip=true,
				]
					\addplot[no marks, BLUE_E, very thick, -] expression[domain=-5:3, samples=50]{x^3 /3 - 2*x +3};
					
					\addplot[no marks, RED_E, very thick,-] expression[domain=-5:3, samples=2]{1.1};
					\addplot[no marks, PURPLE_E, very thick, -] expression[domain=-5:3, samples=2]{4};
				\end{axis}
			\end{tikzpicture}
			\end{center}
		
		\item L'image est unique et ne dépend pas du domaine (tant que celui-ci contient $-2$ !).
		On peut donc toujours connaître $f(-2)$, même sans connaître $f$ en dehors du domaine.
		
		Les antécédents, eux, dépendent du domaine choisi.
		Comme on ne sait pas du tout à quoi ressemble $\C_f$ en dehors du domaine choisi, il est impossible de déterminer tous les réels $x\in\R$ antécédents de $-2$ par $f$.
		
		\item On calcule grâce à la calculatrice (ou sans...)
			\begin{align*}
				f(-1) = \dfrac{14}3 && f(2) =  \dfrac53.
			\end{align*}
		On peut également utiliser le programme Python ci-après (la notation \texttt{x**3} signifiant $x^3$).
	
		\begin{center}
		\python{images-f}
		\end{center}
		
		\item On calcule à la main que
			\begin{align*}
				f(-3) &= 3 - 2 \cdot (-3) + \dfrac13 \cdot (-3)^3, \\
					&= 3 + 6 - 9, \\
					&= 0.
			\end{align*}
		Pour l'image de $0$, remarquons que seul le terme ne dépendant pas de $x$ subsiste. 
		On l'appelle le terme \emph{constant}, et on trouve $f(0) = 3$, comme requis.
	\end{enumerate}
}


\exemulticols{, difficulty=1}{
	Un fonction $f$ est représentée graphiquement ci-contre.
	Parmis les expressions algébriques suivantes, lequelles ne peuvent pas correspondre à $f(x)$ ?
		\begin{multicols}{2}
		\begin{enumerate}[label=\roman*)]
			\item $1-x$
			\item $\dfrac{-1-x}3$
			\item $\left(x+\dfrac13\right)^2$
			\item $-2x - \dfrac23$
		\end{enumerate}
		\end{multicols}
		\vfill\null
}{
	\begin{center}
		\begin{tikzpicture}[>=stealth]
			\begin{axis}[xmin = -3.1, xmax=1.1, ymin=-3.1, ymax=5.1, axis x line=middle, axis y line=middle, axis line style=->, grid=both,
			grid style = {opacity=.5},
			clip=true,
			ytick distance = 1,
			]
				\addplot[no marks, BLUE_E, very thick, -] expression[domain=-3:2, samples=2]{-2/3 - 2*x}
				node[pos=.3, right]{$\mathcal{C}_f$};
			\end{axis}
		
		\end{tikzpicture}
	\end{center}
}{exe:expr-from-graph}{
	Clairement $f(0) < 0$ est strictement négatif.
	L'expression de $f$ ne peut donc pas être la première ni la troisième.
	
	Ensuite, si l'expression de $f$ était la deuxième, on aurait $f(-1) = 0$, ce qui n'est clairement pas le cas.
	Ainsi $f(x) = -2x - \frac23$ est le seul choix possible.
}


\exemulticols{, difficulty=1}{
	Considérons deux fonctions, $g$ et $h$, données graphiquement sur $\D = ]{-}3,4 ; 2,3[$ ci-contre.
	
	Donner approximativement l'ensemble des nombres $x$ du domaine vérifiant les (in)équations suivantes.
	\begin{enumerate}
		\itemsep1em 
		\item $g(x) = h(x)$
		\item $g(x) \leq h(x)$
		\item $h(x) \leq g(x)$
	\end{enumerate}	
}{
	\begin{center}
	\begin{tikzpicture}[>=stealth, scale=1]
		\begin{axis}[xmin = -3.4, xmax=2.3, ymin=-4, ymax=8.1, axis x line=middle, axis y line=middle, axis line style=->, grid=both, clip=true]
		\addplot[no marks, BLUE_E, very thick, -] expression[domain=-3.4:2.3, samples=100]{x^3 /3 - 2*x +3}
		node[pos=.4, above=15pt]{$\mathcal{C}_g$};
		\addplot[no marks, RED_E, very thick, -] expression[domain=-1.5:1.5, samples=50]{-4*x^2 + 7}
		node[pos=.68, above=15pt]{$\mathcal{C}_h$};
		
		\addplot[no marks, RED_E, very thick, -] expression[domain=-3.4:-1.5, samples=50]{x-.5-.1*cos(5*x*180)};
		\addplot[no marks, RED_E, very thick, -] expression[domain=1.5:2.3, samples=50]{-2*x+1-.5*cos(5*x*180)};
		\end{axis}
	\end{tikzpicture}
	\end{center}
}{exe:eq-ineq}{
	\begin{enumerate}
		\item On cherche ici l'ensemble
			\[ \bigset{ x \in \D \text{ tels que } g(x) = h(x) }. \]
		On lit (approximativement) les abscisses des points de $\C_g \cap \C_h$, l'intersection des deux courbes :
			\[ \bigset{ x \in \D \tq g(x) = h(x) } \approx \bigset{ {-0,8}, {1,2} } \]
		\item On cherche ici l'ensemble
			\[ \bigset{ x \in \D \tq g(x) \leq h(x) }. \]
		On lit (approximativement) les abscisses lorsque les la courbe $\C_g$ est sous $\C_h$ :
			\[ \bigset{ x \in \D \tq g(x) \leq h(x) } \approx [ {-0,8} ; {1,2} ]. \]
		\item On cherche ici l'ensemble
			\[ \bigset{ x \in \D \tq h(x) \leq g(x) }. \]
		On lit (approximativement) les abscisses lorsque les la courbe $\C_h$ est sous $\C_g$.
			\[ \bigset{ x \in \D \tq g(x) \leq h(x) } \approx [{-3,4} ; {-0,8} ] \cup [ {1,2} ; {2,3} ]. \]
	\end{enumerate}
	
	Remarquons en outre les relations suivantes :
		\[ \bigset{ x \in \D \tq h(x) \leq g(x) } \cup \bigset{ x \in \D \tq h(x) \geq g(x) } = \D, \]
	ce qui n'est pas  surprenant car pour tout $x\in\D$, on a forcément $g(x) \leq h(x)$ \textbf{ou} $g(x) \geq h(x)$. 
	Ainsi, 
		\[ \bigset{ x \in \D \tq h(x) \leq g(x) \ou  h(x) \geq g(x) }  = \bigset{ x \in \D } = \D. \]
	C'est donc pour cela que les ensembles sont complémentaires par rapport à $\D$ : leur union est $\D$ tout entier.
	
	En outre, les points appartenant aux deux ensembles à la fois sont
		\begin{align*}
		 \bigset{ x \in \D \tq h(x) \leq g(x) \et h(x) \geq g(x) } \bigset{ x \in \D \tq h(x) = g(x) },
		 \end{align*}
	car $h(x) \leq g(x)$ \textbf{et} $h(x) \geq g(x)$ est équivalent à $g(x) = h(x)$ pour tout $x\in\D$.
	
	C'est donc pour ça que les bornes des intervalles étudiés sont les abscisses des points d'intersection.
	Ces points sont exactement les moments où une courbe passe au-dessus d'une autre.
}


% for later maybe :)
%\exe{, difficulty=2}{
%	Pour chaque propriété suivante, donner algébriquement une fonction la vérifiant.
%		%\begin{multicols}{1}
%		\begin{enumerate}
%			\item Une image admet exactement un antécédent.
%			\item Une image admet exactement deux antécédents.
%			\item Une image admet exactement trois antécédents.
%			\item Une image admet une infinité d'antécédents.
%		\end{enumerate}	
%		%\end{multicols}
%}{exe:nb-antécédents}{
%	\begin{enumerate}
%		\item La fonction affine $f(x) = x$ donne une droite dont chaque image admet un unique antécédent.
%		\item La fonction affine $f(x) = x^2$ donne une parabole. En résolvant $f(x) =1$, on peut démontrer que seuls $1$ et $-1$ sont les antécédents de $1$.
%		\item Considérons $f(x) = (x-1)\cdot(x-2)\cdot(x-3)$. Résoudre $f(x)=0$ pour montrer que seuls $1 ; 2 ;$ et $3$ sont antécédents de $0$.
%		\item Une fonction constante fonctionne bien. Par exemple $f(x) = 0$.
%	\end{enumerate}	
%}
%
%\exe{, difficulty=2}{
%	Donner graphiquement une fonction sur $\R$ non constante telle que toutes les images de $f$ admettent un nombre infini d'antécédents.
%}{exe:infinité-antécédents}{
%	On peut définir par exemple la fonction \emph{signe} en incluant 0 dans les positifs (traditionnellement, $\text{signe}(0) = 0$).
%		\[ \text{signe}(x) = \begin{cases*} +1 & si $x \geq 0$, \\
%								-1 & si $x<0$.
%				\end{cases*}. \]
%	Pour une fonction plus intéressante, on pourrait prendre une fonction en vague.
%	La fonction sinus fonctionne bien : entrer par exemple \texttt{y=sin(x)} sur Geogebra.
%}


%%%%%%%%%%%

\newpage
\fancyhead[C]{\textbf{Solutions}}
\shipoutAnswer
	
\end{document}

%% INPUT PREAMBLE.TEX
%% THEN INPUT VARS_{i}.ADR
%% THEN RUN THIS
% DYSLEXIA SWITCH
\newif\ifdys
		
				% ENABLE or DISABLE font change
				% use XeLaTeX if true
				\dystrue
				\dysfalse


\ifdys

\documentclass[a4paper, 14pt]{extarticle}
\usepackage{amsmath,amsfonts,amsthm,amssymb,mathtools}

\tracinglostchars=3 % Report an error if a font does not have a symbol.
\usepackage{fontspec}
\usepackage{unicode-math}
\defaultfontfeatures{ Ligatures=TeX,
                      Scale=MatchUppercase }

\setmainfont{OpenDyslexic}[Scale=1.0]
\setmathfont{Fira Math} % Or maybe try KPMath-Sans?
\setmathfont{OpenDyslexic Italic}[range=it/{Latin,latin}]
\setmathfont{OpenDyslexic}[range=up/{Latin,latin,num}]

\else

\documentclass[a4paper, 12pt]{extarticle}

\usepackage[utf8x]{inputenc}
%fonts
\usepackage{amsmath,amsfonts,amsthm,amssymb,mathtools}
% comment below to default to computer modern
\usepackage{libertinus,libertinust1math}

\fi


\usepackage[french]{babel}
\usepackage[
a4paper,
margin=2cm,
nomarginpar,% We don't want any margin paragraphs
]{geometry}
\usepackage{icomma}

\usepackage{fancyhdr}
\usepackage{array}
\usepackage{hyperref}

\usepackage{multicol, enumerate}
\newcolumntype{P}[1]{>{\centering\arraybackslash}p{#1}}


\usepackage{stackengine}
\newcommand\xrowht[2][0]{\addstackgap[.5\dimexpr#2\relax]{\vphantom{#1}}}

% theorems

\theoremstyle{plain}
\newtheorem{theorem}{Th\'eor\`eme}
\newtheorem*{sol}{Solution}
\theoremstyle{definition}
\newtheorem{ex}{Exercice}
\newtheorem*{rpl}{Rappel}
\newtheorem{enigme}{Énigme}

% corps
\usepackage{calrsfs}
\newcommand{\C}{\mathcal{C}}
\newcommand{\R}{\mathbb{R}}
\newcommand{\Rnn}{\mathbb{R}^{2n}}
\newcommand{\Z}{\mathbb{Z}}
\newcommand{\N}{\mathbb{N}}
\newcommand{\Q}{\mathbb{Q}}

% variance
\newcommand{\Var}[1]{\text{Var}(#1)}

% domain
\newcommand{\D}{\mathcal{D}}


% date
\usepackage{advdate}
\AdvanceDate[0]


% plots
\usepackage{pgfplots}

% table line break
\usepackage{makecell}
%tablestuff
\def\arraystretch{2}
\setlength\tabcolsep{15pt}

%subfigures
\usepackage{subcaption}

\definecolor{myg}{RGB}{56, 140, 70}
\definecolor{myb}{RGB}{45, 111, 177}
\definecolor{myr}{RGB}{199, 68, 64}

% fake sections with no title to move around the merged pdf
\newcommand{\fakesection}[1]{%
  \par\refstepcounter{section}% Increase section counter
  \sectionmark{#1}% Add section mark (header)
  \addcontentsline{toc}{section}{\protect\numberline{\thesection}#1}% Add section to ToC
  % Add more content here, if needed.
}


% SOLUTION SWITCH
\newif\ifsolutions
				\solutionstrue
				%\solutionsfalse

\ifsolutions
	\newcommand{\exe}[2]{
		\begin{ex} #1  \end{ex}
		\begin{sol} #2 \end{sol}
	}
\else
	\newcommand{\exe}[2]{
		\begin{ex} #1  \end{ex}
	}
	
\fi


% tableaux var, signe
\usepackage{tkz-tab}


%pinfty minfty
\newcommand{\pinfty}{{+}\infty}
\newcommand{\minfty}{{-}\infty}

\begin{document}

%\input{adr/vars_44284.adr}
\newcommand{\seed}{TEST}

\pagestyle{fancy}
\fancyhead[L]{Seconde 13}
\fancyhead[C]{\textbf{Devoir Maison 4 --- Interpolation de Lagrange --- \seed \ifsolutions \, -- Solutions  \fi}}
\fancyhead[R]{\today}

\begin{enigme}\label{enigme:1}
	Compléter la suite logique suivante.
		\begin{center}
			$3$ --- $4$ --- ?
		\end{center}
\end{enigme}

\exe{
	Le but de l'exercice est de répondre à l'énigme \ref{enigme:1} de deux façons différentes.
	On cherche deux fonctions polynomiales $f$ et $g$ telles que 
		\begin{enumerate}[(i)]
			\item $ f(1) = g(1) = 3$ ;
			\item $f(2) = g(2) = 4$ ; et
			\item $f(3) \neq g(3)$.
		\end{enumerate}
	On aura ainsi deux façons cohérentes de compléter la suite de nombres :
		\begin{center}
			$3$ --- $4$ --- $f(3)$, \hspace{3cm} et \hspace{3cm} $3$ --- $4$ --- $g(3)$.
		\end{center}
	
	\begin{enumerate}
		\item Trouver une fonction $a$ qui s'annule en $2$ et en $3$, mais pas en $1$.
		\item Trouver une fonction $b$ qui s'annule en $1$ et en $3$, mais pas en $2$.
		\item Trouver une fonction $c$ qui s'annule en $1$ et $2$, mais pas en $3$.
		\item Sans développer l'expression, montrer que la fonction
			\[ f(x) = \dfrac{3}{a(1)} a(x) + \dfrac{4}{b(2)} b(x)  + \dfrac{17}{c(3)} c(x) \]
		vérifie que
			\begin{enumerate}[(i)]
				\item $f(1) = 3$ ;
				\item $f(2) = 4$ ; et
				\item $f(3) = 17$.
			\end{enumerate}
		\item Développer et réduire l'expression de $f$.
		\item Construire une fonction $g$ telle que
			\begin{enumerate}[(i)]
				\item $g(1) = 3$ ;
				\item $g(2) = 4$ ; et
				\item $g(3) = 27$.
			\end{enumerate}
		\item Développer et réduire l'expression de $g$.
	\end{enumerate}
}{}

\subsection*{Bonus}

\begin{enigme}\label{enigme:2}
	Compléter la suite logique suivante.
		\begin{center}
			$3$ --- $4$ --- $5$ --- ?
		\end{center}
\end{enigme}

\exe{
	Répondre à l'énigme \ref{enigme:2} de deux façons différentes en donnant deux polynômes $f, g$ tels que
		\begin{enumerate}[(i)]
			\item $ f(1) = g(1) = 3$ ;
			\item $f(2) = g(2) = 4$ ;
			\item $f(3) = g(3) = 5$ ; et
			\item $f(4) \neq g(4)$.
		\end{enumerate}
	Développer et réduire les expressions de $f(x)$ et $g(x)$.
}{}

\end{document}

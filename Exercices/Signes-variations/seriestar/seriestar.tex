				% ENABLE or DISABLE font change
				% use XeLaTeX if true
\newif\ifdys
				\dystrue
				\dysfalse

\newif\ifsolutions
				\solutionstrue
				\solutionsfalse

% DYSLEXIA SWITCH
\newif\ifdys
		
				% ENABLE or DISABLE font change
				% use XeLaTeX if true
				\dystrue
				\dysfalse


\ifdys

\documentclass[a4paper, 14pt]{extarticle}
\usepackage{amsmath,amsfonts,amsthm,amssymb,mathtools}

\tracinglostchars=3 % Report an error if a font does not have a symbol.
\usepackage{fontspec}
\usepackage{unicode-math}
\defaultfontfeatures{ Ligatures=TeX,
                      Scale=MatchUppercase }

\setmainfont{OpenDyslexic}[Scale=1.0]
\setmathfont{Fira Math} % Or maybe try KPMath-Sans?
\setmathfont{OpenDyslexic Italic}[range=it/{Latin,latin}]
\setmathfont{OpenDyslexic}[range=up/{Latin,latin,num}]

\else

\documentclass[a4paper, 12pt]{extarticle}

\usepackage[utf8x]{inputenc}
%fonts
\usepackage{amsmath,amsfonts,amsthm,amssymb,mathtools}
% comment below to default to computer modern
\usepackage{libertinus,libertinust1math}

\fi


\usepackage[french]{babel}
\usepackage[
a4paper,
margin=2cm,
nomarginpar,% We don't want any margin paragraphs
]{geometry}
\usepackage{icomma}

\usepackage{fancyhdr}
\usepackage{array}
\usepackage{hyperref}

\usepackage{multicol, enumerate}
\newcolumntype{P}[1]{>{\centering\arraybackslash}p{#1}}


\usepackage{stackengine}
\newcommand\xrowht[2][0]{\addstackgap[.5\dimexpr#2\relax]{\vphantom{#1}}}

% theorems

\theoremstyle{plain}
\newtheorem{theorem}{Th\'eor\`eme}
\newtheorem*{sol}{Solution}
\theoremstyle{definition}
\newtheorem{ex}{Exercice}
\newtheorem*{rpl}{Rappel}
\newtheorem{enigme}{Énigme}

% corps
\usepackage{calrsfs}
\newcommand{\C}{\mathcal{C}}
\newcommand{\R}{\mathbb{R}}
\newcommand{\Rnn}{\mathbb{R}^{2n}}
\newcommand{\Z}{\mathbb{Z}}
\newcommand{\N}{\mathbb{N}}
\newcommand{\Q}{\mathbb{Q}}

% variance
\newcommand{\Var}[1]{\text{Var}(#1)}

% domain
\newcommand{\D}{\mathcal{D}}


% date
\usepackage{advdate}
\AdvanceDate[0]


% plots
\usepackage{pgfplots}

% table line break
\usepackage{makecell}
%tablestuff
\def\arraystretch{2}
\setlength\tabcolsep{15pt}

%subfigures
\usepackage{subcaption}

\definecolor{myg}{RGB}{56, 140, 70}
\definecolor{myb}{RGB}{45, 111, 177}
\definecolor{myr}{RGB}{199, 68, 64}

% fake sections with no title to move around the merged pdf
\newcommand{\fakesection}[1]{%
  \par\refstepcounter{section}% Increase section counter
  \sectionmark{#1}% Add section mark (header)
  \addcontentsline{toc}{section}{\protect\numberline{\thesection}#1}% Add section to ToC
  % Add more content here, if needed.
}


% SOLUTION SWITCH
\newif\ifsolutions
				\solutionstrue
				%\solutionsfalse

\ifsolutions
	\newcommand{\exe}[2]{
		\begin{ex} #1  \end{ex}
		\begin{sol} #2 \end{sol}
	}
\else
	\newcommand{\exe}[2]{
		\begin{ex} #1  \end{ex}
	}
	
\fi


% tableaux var, signe
\usepackage{tkz-tab}


%pinfty minfty
\newcommand{\pinfty}{{+}\infty}
\newcommand{\minfty}{{-}\infty}

\begin{document}


\AdvanceDate[0]

\begin{document}
\pagestyle{fancy}
\fancyhead[L]{Seconde 13}
\fancyhead[C]{\textbf{Signes et variations $\star$ \ifsolutions -- Solutions  \fi}}
\fancyhead[R]{\today}


\exe{[$\star$]
	Donner une fonction $f : \R \rightarrow \R$ telle que
		\[ \{ x \in \R \text{ tq. } f(x) > 0 \} = \left] \minfty ; -2 \right[ \cup \left] \dfrac{22}7 ; \pinfty \right[. \]
}{}

\exe{[$\star$]
	Donner une fonction $f : \R \rightarrow \R$ telle que
		\[ \{ x \in \R \text{ tq. } f(x) > 0 \} = \left] \minfty ; -1 \right[ \cup \left] \dfrac{22}7 ; 5 \right[ \cup \left] 7 ; 8 \right[. \]
}{}



\exe{[$\star$]
	Donner un polynôme $f(x) = ax^4 + bx^3 + cx^2 + dx + e$ à coefficients $a, b, c, d, e$ \underline{entiers} relatifs pas tous nuls et tel que $3, -\frac23,$ et $\sqrt{2}$ annulent $f$.
}{}

\exe{[$\star$]
	Soit $f(x) = ax^2 + bx + c$ un polynôme à coefficients $a, b, c$ entiers.
	On appelle $d\in\N$ un entier naturel qui n'est pas un carré parfait.
	\begin{enumerate}
		\item Montrer que si $f\left(\sqrt{d}\right)= 0$, alors $b=0$.
		\item En déduire que si $\sqrt{d}$ est racine de $f$, alors $-\sqrt{d}$ est nécessairement l'autre racine de $f$.
		\item Une racine du polynôme $f(x) = x^2 - x - 1$ peut-elle s'écrire $\sqrt{k}$ avec $k\in\N$ un entier non carré parfait ?
		%\item Donner des paramètres $a,b,c$ entiers avec $b\neq0$ tels que $f$ admette une racine entière.
	\end{enumerate}
}{}

\exe{
	Soit $f(x) = x^2 - x - 1$ une fonction quadratique sur $\R$.
	
	Montrer que  $f(x) = \left( x - \dfrac{1 - \sqrt{5}}2 \right)\cdot \left( x - \dfrac{1 + \sqrt{5}}2 \right)$ pour tout $x\in\R$ et en déduire les racines de $f$. On appelle la racine positive $\phi = \dfrac{1 + \sqrt{5}}2$ (lu \og phi \fg) le \emph{nombre d'or}.
	
	Vérifier que les racines trouvées annulent bien $f$ en calculant leur image par $f$.
}{}

\exe{[tiré de la série Fonctions affines $\star$]
	Considérons une fonction quadratique 
		\[ f(x) = ax^2 + bx + c, \]
	où $a, b, c\in\R$ sont trois paramètres réels.
	Supposons de surcroît qu'on connaisse deux racines distinctes de $f$, c'est-à-dire qu'on connaisse $\alpha, \beta\in\R$ tels que $\alpha\neq\beta$ et
		\[ f(\alpha) = f(\beta) = 0. \]
	\begin{enumerate}
		\item Montrer que la fonction $g$ donnée par
			\[ g(x) = f(x) - a (x-\alpha)(x-\beta) \qquad \text{ pour tout } x\in\R \]
		est affine.
		\item Montrer que $g$ admet deux racines distinctes.
		\item En déduire, par interpolation linéaire, que $g$ est constamment nulle et donc que
			\[ f(x) = a (x-\alpha)(x-\beta)  \qquad \text{ pour tout } x\in\R.  \]
	\end{enumerate}
}


%\exe{[$\star$]
%	Remplir le tableau de signes de telle sorte qu'il soit correct.
%}{}



\exe{[$\star$]
	\begin{enumerate}
		\item Montrer que si $0 < x < y$, alors $0 < x^2 < y^2$.
		\item Montrer que si $y < x < 0$, alors $y^2 > x^2 > 0$.
		\item En déduire les variations de la fonction carré $f(x) = x^2$ sur tout $\R$.
	\end{enumerate}
}{}

\exe{[$\star$]
	Montrer que pour $0< x<y$, on a
		\[ \sqrt{y}-\sqrt{x} = \dfrac{y-x}{\sqrt{y}+\sqrt{x}}, \]
	En déduire les variations de $f(x) = \sqrt{x}$ sur $[0; \pinfty[$.
}{}

\exe{[$\star$]
	Soit $f(x) = \dfrac1x$ définie sur $\D_f = \R-\{0\}$.
	\begin{enumerate}
		%\item Montrer que $f$ peut prendre des valeurs aussi grandes que souhaitées autour de $0$ :
	%pour tout $M > 0$, il existe un $|x|>0$ tel que $|f(x)| > M$.
		 \item Montrer que $f$ est toujours strictement décroissante en séparant l'étude à $]\minfty;0[$ et $]0;\pinfty[$.
		\item Faire le tableau de variations et de signes de $f$ sur $\D_f$ et esquisser $\C_f$.
	\end{enumerate}
}{}

%\exe{[$\star$]
%	Soit $f(x) = x^2$ définie sur $I \subset \R$, un intervalle borné de $\R$.
%	Montrer que $f$ atteint son maximum en l'une des deux bornes de $I$.
%}{}


\end{document}

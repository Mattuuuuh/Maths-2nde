%!TEX encoding = UTF8
%!TEX root =notes.tex


%%%%%%%%%%%%%%%%%%%%%%%%%%%%%%%%%
% PACKAGE IMPORTS
%%%%%%%%%%%%%%%%%%%%%%%%%%%%%%%%%


\usepackage[french]{babel}

\usepackage[tmargin=2cm,rmargin=1in,lmargin=1in,margin=0.85in,bmargin=2cm,footskip=.2in]{geometry}
\usepackage{amsmath,amsfonts,amsthm,amssymb,mathtools}
\usepackage[varbb]{newpxmath}
\usepackage{xfrac}
\usepackage[makeroom]{cancel}
\usepackage{mathtools}
\usepackage{bookmark}
\usepackage{enumitem}
\usepackage{hyperref,theoremref}
\hypersetup{
	pdftitle={Assignment},
	colorlinks=true, linkcolor=doc!90,
	bookmarksnumbered=true,
	bookmarksopen=true
}
\usepackage[most,many,breakable]{tcolorbox}
\usepackage{xcolor}
\usepackage{varwidth}
\usepackage{varwidth}
\usepackage{etoolbox}
%\usepackage{authblk}
\usepackage{nameref}
\usepackage{multicol,array}
\usepackage{tikz-cd}
\usepackage[ruled,vlined,linesnumbered]{algorithm2e}
\usepackage{comment} % enables the use of multi-line comments (\ifx \fi) 
\usepackage{import}
\usepackage{xifthen}
\usepackage{pdfpages}
\usepackage{transparent}


\newcommand\mycommfont[1]{\footnotesize\ttfamily\textcolor{blue}{#1}}
\SetCommentSty{mycommfont}
\newcommand{\incfig}[1]{%
    \def\svgwidth{\columnwidth}
    \import{./figures/}{#1.pdf_tex}
}

\usepackage{tikzsymbols}
%\renewcommand\qedsymbol{$\Laughey$}


%\usepackage{import}
%\usepackage{xifthen}
%\usepackage{pdfpages}
%\usepackage{transparent}


%%%%%%%%%%%%%%%%%%%%%%%%%%%%%%
% SELF MADE COLORS
%%%%%%%%%%%%%%%%%%%%%%%%%%%%%%



\definecolor{myg}{RGB}{56, 140, 70}
\definecolor{myb}{RGB}{45, 111, 177}
\definecolor{myr}{RGB}{199, 68, 64}
\definecolor{mytheorembg}{HTML}{F2F2F9}
\definecolor{mytheoremfr}{HTML}{00007B}
\definecolor{mylenmabg}{HTML}{FFFAF8}
\definecolor{mylenmafr}{HTML}{983b0f}
\definecolor{mypropbg}{HTML}{f2fbfc}
\definecolor{mypropfr}{HTML}{191971}
\definecolor{myexamplebg}{HTML}{F2FBF8}
\definecolor{myexamplefr}{HTML}{88D6D1}
\definecolor{myexampleti}{HTML}{2A7F7F}
\definecolor{mydefinitbg}{HTML}{E5E5FF}
\definecolor{mydefinitfr}{HTML}{3F3FA3}
\definecolor{notesgreen}{RGB}{0,162,0}
\definecolor{myp}{RGB}{197, 92, 212}
\definecolor{mygr}{HTML}{2C3338}
\definecolor{myred}{RGB}{127,0,0}
\definecolor{myyellow}{RGB}{169,121,69}
\definecolor{myexercisebg}{HTML}{F2FBF8}
\definecolor{myexercisefg}{HTML}{88D6D1}


%%%%%%%%%%%%%%%%%%%%%%%%%%%%
% TCOLORBOX SETUPS
%%%%%%%%%%%%%%%%%%%%%%%%%%%%

\setlength{\parindent}{1cm}
%================================
% THEOREM BOX
%================================

\tcbuselibrary{theorems,skins,hooks}
\newtcbtheorem[number within=chapter]{Theorem}{Théorème}
{%
	enhanced,
	breakable,
	colback = mytheorembg,
	frame hidden,
	boxrule = 0sp,
	borderline west = {2pt}{0pt}{mytheoremfr},
	sharp corners,
	detach title,
	before upper = \tcbtitle\par\smallskip,
	coltitle = mytheoremfr,
	fonttitle = \bfseries\sffamily,
	description font = \mdseries,
	separator sign none,
	segmentation style={solid, mytheoremfr},
}
{th}


\tcbuselibrary{theorems,skins,hooks}
\newtcolorbox{Theoremcon}
{%
	enhanced
	,breakable
	,colback = mytheorembg
	,frame hidden
	,boxrule = 0sp
	,borderline west = {2pt}{0pt}{mytheoremfr}
	,sharp corners
	,description font = \mdseries
	,separator sign none
}

%================================
% Corollery
%================================
\tcbuselibrary{theorems,skins,hooks}
\newtcbtheorem[use counter=tcb@cnt@Theorem]{Corollary}{Corollaire}
{%
	enhanced
	,breakable
	,colback = myp!10
	,frame hidden
	,boxrule = 0sp
	,borderline west = {2pt}{0pt}{myp!85!black}
	,sharp corners
	,detach title
	,before upper = \tcbtitle\par\smallskip
	,coltitle = myp!85!black
	,fonttitle = \bfseries\sffamily
	,description font = \mdseries
	,separator sign none
	,segmentation style={solid, myp!85!black}
}
{th}

%================================
% LENMA
%================================

\tcbuselibrary{theorems,skins,hooks}
\newtcbtheorem[use counter=tcb@cnt@Theorem]{Lemma}{Lemme}
{%
	enhanced,
	breakable,
	colback = mylenmabg,
	frame hidden,
	boxrule = 0sp,
	borderline west = {2pt}{0pt}{mylenmafr},
	sharp corners,
	detach title,
	before upper = \tcbtitle\par\smallskip,
	coltitle = mylenmafr,
	fonttitle = \bfseries\sffamily,
	description font = \mdseries,
	separator sign none,
	segmentation style={solid, mylenmafr},
}
{th}


%================================
% PROPOSITION
%================================

\tcbuselibrary{theorems,skins,hooks}
\newtcbtheorem[use counter=tcb@cnt@Theorem]{Prop}{Proposition}
{%
	enhanced,
	breakable,
	colback = mypropbg,
	frame hidden,
	boxrule = 0sp,
	borderline west = {2pt}{0pt}{mypropfr},
	sharp corners,
	detach title,
	before upper = \tcbtitle\par\smallskip,
	coltitle = mypropfr,
	fonttitle = \bfseries\sffamily,
	description font = \mdseries,
	separator sign none,
	segmentation style={solid, mypropfr},
}
{th}


%================================
% CLAIM
%================================

\tcbuselibrary{theorems,skins,hooks}
\newtcbtheorem[use counter=tcb@cnt@Theorem]{claim}{Claim}
{%
	enhanced
	,breakable
	,colback = myg!10
	,frame hidden
	,boxrule = 0sp
	,borderline west = {2pt}{0pt}{myg}
	,sharp corners
	,detach title
	,before upper = \tcbtitle\par\smallskip
	,coltitle = myg!85!black
	,fonttitle = \bfseries\sffamily
	,description font = \mdseries
	,separator sign none
	,segmentation style={solid, myg!85!black}
}
{th}



%================================
% Exercise
%================================

\tcbuselibrary{theorems,skins,hooks}
\newtcbtheorem[use counter=tcb@cnt@Theorem]{Exercise}{Exercice}
{%
	enhanced,
	breakable,
	colback = myexercisebg,
	frame hidden,
	boxrule = 0sp,
	borderline west = {2pt}{0pt}{myexercisefg},
	sharp corners,
	detach title,
	before upper = \tcbtitle\par\smallskip,
	coltitle = myexercisefg,
	fonttitle = \bfseries\sffamily,
	description font = \mdseries,
	separator sign none,
	segmentation style={solid, myexercisefg},
}
{th}

%================================
% EXAMPLE BOX
%================================

\newtcbtheorem[use counter=tcb@cnt@Theorem]{Example}{Exemple}
{%
	colback = myexamplebg
	,breakable
	,colframe = myexamplefr
	,coltitle = myexampleti
	,boxrule = 1pt
	,sharp corners
	,detach title
	,before upper=\tcbtitle\par\smallskip
	,fonttitle = \bfseries
	,description font = \mdseries
	,separator sign none
	,description delimiters parenthesis
}
{ex}

%================================
% DEFINITION BOX
%================================

\newtcbtheorem[use counter=tcb@cnt@Theorem]{Definition}{Définition}{enhanced,
	before skip=2mm,after skip=2mm, colback=red!5,colframe=red!80!black,boxrule=0.5mm,
	attach boxed title to top left={xshift=1cm,yshift*=1mm-\tcboxedtitleheight}, varwidth boxed title*=-3cm,
	boxed title style={frame code={
					\path[fill=tcbcolback]
					([yshift=-1mm,xshift=-1mm]frame.north west)
					arc[start angle=0,end angle=180,radius=1mm]
					([yshift=-1mm,xshift=1mm]frame.north east)
					arc[start angle=180,end angle=0,radius=1mm];
					\path[left color=tcbcolback!60!black,right color=tcbcolback!60!black,
						middle color=tcbcolback!80!black]
					([xshift=-2mm]frame.north west) -- ([xshift=2mm]frame.north east)
					[rounded corners=1mm]-- ([xshift=1mm,yshift=-1mm]frame.north east)
					-- (frame.south east) -- (frame.south west)
					-- ([xshift=-1mm,yshift=-1mm]frame.north west)
					[sharp corners]-- cycle;
				},interior engine=empty,
		},
	fonttitle=\bfseries,
	title={#2},#1}{def}

%================================
% Solution BOX
%================================

\makeatletter
\newtcbtheorem[use counter=tcb@cnt@Theorem]{question}{Question}{enhanced,
	breakable,
	colback=white,
	colframe=myb!80!black,
	attach boxed title to top left={yshift*=-\tcboxedtitleheight},
	fonttitle=\bfseries,
	title={#2},
	boxed title size=title,
	boxed title style={%
			sharp corners,
			rounded corners=northwest,
			colback=tcbcolframe,
			boxrule=0pt,
		},
	underlay boxed title={%
			\path[fill=tcbcolframe] (title.south west)--(title.south east)
			to[out=0, in=180] ([xshift=5mm]title.east)--
			(title.center-|frame.east)
			[rounded corners=\kvtcb@arc] |-
			(frame.north) -| cycle;
		},
	#1
}{def}
\makeatother

%================================
% SOLUTION BOX
%================================

\makeatletter
\newtcolorbox{solution}{enhanced,
	breakable,
	colback=white,
	colframe=myg!80!black,
	attach boxed title to top left={yshift*=-\tcboxedtitleheight},
	title=Solution,
	boxed title size=title,
	boxed title style={%
			sharp corners,
			rounded corners=northwest,
			colback=tcbcolframe,
			boxrule=0pt,
		},
	underlay boxed title={%
			\path[fill=tcbcolframe] (title.south west)--(title.south east)
			to[out=0, in=180] ([xshift=5mm]title.east)--
			(title.center-|frame.east)
			[rounded corners=\kvtcb@arc] |-
			(frame.north) -| cycle;
		},
}
\makeatother

%================================
% Question BOX
%================================

\makeatletter
\newtcbtheorem[use counter=tcb@cnt@Theorem]{qstion}{Question}{enhanced,
	breakable,
	colback=white,
	colframe=mygr,
	attach boxed title to top left={yshift*=-\tcboxedtitleheight},
	fonttitle=\bfseries,
	title={#2},
	boxed title size=title,
	boxed title style={%
			sharp corners,
			rounded corners=northwest,
			colback=tcbcolframe,
			boxrule=0pt,
		},
	underlay boxed title={%
			\path[fill=tcbcolframe] (title.south west)--(title.south east)
			to[out=0, in=180] ([xshift=5mm]title.east)--
			(title.center-|frame.east)
			[rounded corners=\kvtcb@arc] |-
			(frame.north) -| cycle;
		},
	#1
}{def}
\makeatother

\newtcbtheorem[number within=chapter]{wconc}{Wrong Concept}{
	breakable,
	enhanced,
	colback=white,
	colframe=myr,
	arc=0pt,
	outer arc=0pt,
	fonttitle=\bfseries\sffamily\large,
	colbacktitle=myr,
	attach boxed title to top left={},
	boxed title style={
			enhanced,
			skin=enhancedfirst jigsaw,
			arc=3pt,
			bottom=0pt,
			interior style={fill=myr}
		},
	#1
}{def}



%================================
% NOTE BOX
%================================

\usetikzlibrary{arrows,calc,shadows.blur}
\tcbuselibrary{skins}
\newtcolorbox{note}[1][]{%
	enhanced jigsaw,
	colback=gray!20!white,%
	colframe=gray!80!black,
	size=small,
	boxrule=1pt,
	title=\colorbox{white!100}{\textbf{ Remarque }},
	halign title=flush center,
	coltitle=black,
	breakable,
	drop shadow=black!50!white,
	attach boxed title to top left={xshift=1cm,yshift=-\tcboxedtitleheight/2,yshifttext=-\tcboxedtitleheight/2},
	minipage boxed title=2.6cm,
	boxed title style={%
			colback=white,
			size=fbox,
			boxrule=1pt,
			boxsep=2pt,
			underlay={%
					\coordinate (dotA) at ($(interior.west) + (-0.5pt,0)$);
					\coordinate (dotB) at ($(interior.east) + (0.5pt,0)$);
					\begin{scope}
						\clip (interior.north west) rectangle ([xshift=3ex]interior.east);
						\filldraw [white, blur shadow={shadow opacity=60, shadow yshift=-.75ex}, rounded corners=2pt] (interior.north west) rectangle (interior.south east);
					\end{scope}
					\begin{scope}[gray!80!black]
						\fill (dotA) circle (2pt);
						\fill (dotB) circle (2pt);
					\end{scope}
				},
		},
	#1,
}

%================================
% STRATÉGIE BOX
%================================

\usetikzlibrary{arrows,calc,shadows.blur}
\tcbuselibrary{skins}
\newtcolorbox{strategy}[1][]{%
	enhanced jigsaw,
	colback=myb!20!white,%
	colframe=gray!80!black,
	size=small,
	boxrule=1pt,
	title=\colorbox{white!100}{\textbf{ Stratégie }},
	halign title=flush center,
	coltitle=black,
	breakable,
	drop shadow=black!50!white,
	attach boxed title to top left={xshift=1cm,yshift=-\tcboxedtitleheight/2,yshifttext=-\tcboxedtitleheight/2},
	minipage boxed title=2.5cm,
	boxed title style={%
			colback=white,
			size=fbox,
			boxrule=1pt,
			boxsep=2pt,
			underlay={%
					\coordinate (dotA) at ($(interior.west) + (-0.5pt,0)$);
					\coordinate (dotB) at ($(interior.east) + (0.5pt,0)$);
					\begin{scope}
						\clip (interior.north west) rectangle ([xshift=3ex]interior.east);
						\filldraw [white, blur shadow={shadow opacity=60, shadow yshift=-.75ex}, rounded corners=2pt] (interior.north west) rectangle (interior.south east);
					\end{scope}
					\begin{scope}[gray!80!black]
						\fill (dotA) circle (2pt);
						\fill (dotB) circle (2pt);
					\end{scope}
				},
		},
	#1,
}

%================================
% MÉTHODE BOX
%================================

\usetikzlibrary{arrows,calc,shadows.blur}
\tcbuselibrary{skins}
\newtcolorbox{methode}[1][]{%
	enhanced jigsaw,
	colback=white,%
	colframe=gray!80!black,
	size=small,
	boxrule=1pt,
	title=\textbf{Méthode},
	halign title=flush center,
	coltitle=black,
	breakable,
	drop shadow=black!50!white,
	attach boxed title to top left={xshift=1cm,yshift=-\tcboxedtitleheight/2,yshifttext=-\tcboxedtitleheight/2},
	minipage boxed title=2.5cm,
	boxed title style={%
			colback=white,
			size=fbox,
			boxrule=1pt,
			boxsep=2pt,
			underlay={%
					\coordinate (dotA) at ($(interior.west) + (-0.5pt,0)$);
					\coordinate (dotB) at ($(interior.east) + (0.5pt,0)$);
					\begin{scope}
						\clip (interior.north west) rectangle ([xshift=3ex]interior.east);
						\filldraw [white, blur shadow={shadow opacity=60, shadow yshift=-.75ex}, rounded corners=2pt] (interior.north west) rectangle (interior.south east);
					\end{scope}
					\begin{scope}[gray!80!black]
						\fill (dotA) circle (2pt);
						\fill (dotB) circle (2pt);
					\end{scope}
				},
		},
	#1,
}

%%%%%%%%%%%%%%%%%%%%%%%%%%%%%%%%%%%%%%%%%%%
% TABLE OF CONTENTS
%%%%%%%%%%%%%%%%%%%%%%%%%%%%%%%%%%%%%%%%%%%

\usepackage{tikz}

\definecolor{doc}{RGB}{0,60,110}
\usepackage{titletoc}
\contentsmargin{0cm}
\titlecontents{chapter}[3.7pc]
{\addvspace{30pt}%
	\begin{tikzpicture}[remember picture, overlay]%
		\draw[fill=doc!60,draw=doc!60] (-7,-.1) rectangle (-0.2,.6);%
		\pgftext[left,x=-3.5cm,y=0.2cm]{\color{white}\Large\sc\bfseries Chapitre\ \thecontentslabel};%
	\end{tikzpicture}\color{doc!60}\large\sc\bfseries}%
{}
{}
{\;\titlerule\;\large\sc\bfseries Page \thecontentspage
	\begin{tikzpicture}[remember picture, overlay]
		\draw[fill=doc!60,draw=doc!60] (2pt,0) rectangle (4,0.1pt);
	\end{tikzpicture}}%
\titlecontents{section}[3.7pc]
{\addvspace{2pt}}
{\contentslabel[\thecontentslabel]{2pc}}
{}
{\hfill\small \thecontentspage}
[]
\titlecontents*{subsection}[3.7pc]
{\addvspace{-1pt}\small}
{}
{}
{\ --- \small\thecontentspage}
[ \textbullet\ ][]

\makeatletter
\renewcommand{\tableofcontents}{%
	\chapter*{%
	  \vspace*{-20\p@}%
	  \begin{tikzpicture}[remember picture, overlay]%
		  \pgftext[right,x=15cm,y=0.2cm]{\color{doc!60}\Huge\sc\bfseries \contentsname};%
		  \draw[fill=doc!60,draw=doc!60] (13,-.75) rectangle (20,1);%
		  \clip (13,-.75) rectangle (20,1);
		  \pgftext[right,x=15cm,y=0.2cm]{\color{white}\Huge\sc\bfseries \contentsname};%
	  \end{tikzpicture}}%
	\@starttoc{toc}}
\makeatother


%%%%%%%%%%%%%%%%%%%%%%%%%%%%%%%%%%%%%%%%%%%
% MINTED FOR PYTHON ALGORITHMS
%%%%%%%%%%%%%%%%%%%%%%%%%%%%%%%%%%%%%%%%%%%

\usepackage{tcolorbox}
\tcbuselibrary{minted,breakable,xparse,skins}
\definecolor{bg}{gray}{0.95}
\DeclareTCBListing{mintedbox}{O{}m!O{}}{%
  breakable=true,
  listing engine=minted,
  listing only,
  minted language=#2,
  minted style=default,
  minted options={%
    linenos,
    gobble=0,
    breaklines=true,
    breakafter=,,
    fontsize=\small,
    numbersep=8pt,
    #1},
  boxsep=0pt,
  left skip=0pt,
  right skip=0pt,
  left=25pt,
  right=0pt,
  top=3pt,
  bottom=3pt,
  arc=5pt,
  leftrule=0pt,
  rightrule=0pt,
  bottomrule=2pt,
  toprule=2pt,
  colback=bg,
  colframe=orange!70,
  enhanced,
  overlay={%
    \begin{tcbclipinterior}
    \fill[orange!20!white] (frame.south west) rectangle ([xshift=20pt]frame.north west);
    \end{tcbclipinterior}},
  #3}
  
  
 % for braces
\usetikzlibrary{decorations.pathreplacing}


\SetDate[06/01/2026]

\begin{document}
\pagestyle{fancy}
\fancyhead[L]{Seconde}
\fancyhead[C]{\textbf{Valeurs absolues}}
\fancyhead[R]{\today}



\exe{,difficulty=0}{
	Écrire les valeurs suivantes sans les barres de valeur absolue.
	\begin{multicols}{3}
	\begin{enumerate}[label=\roman*)]
		\item $|-7|$
		\item $|8|$
		\item $|-13-8|$
		\item $|\pi - 4|$
		\item $|-5+3| + |-7+4|$
		\item $|\sqrt{2} - \sqrt{3}|$
	\end{enumerate}
	\end{multicols}
}{exe:vabs-1}{


	\begin{multicols}{3}
	\begin{enumerate}[label=\roman*)]
		\item $7$
		\item $8$
		\item $21$
		\item $4 - \pi$
		\item $5$
		\item $\sqrt{3}- \sqrt{2}$
	\end{enumerate}
	\end{multicols}
	Le dernier point utilise que $\sqrt3 > \sqrt2$, propriété de croissance de la racine carrée qui peut être soit admise, soit démontrée généralement à l'exercice \ref{exe:sqrt-croissante}.
	Dans ce cas précis, démontrer que
		\[ \sqrt3 - \sqrt2 = \dfrac{1}{\sqrt3 + \sqrt2} > 0 \]
	conclut.

}



\exe{,difficulty=1}{
	Pour chaque contrainte suivante donner le plus grand intervalle dans lequel le nombre réel $x\in\R$ peut appartenir.
	%À quels intervalles le réel $x \in \R$ appartient-il s'il vérifie les inégalités suivantes ?
	\begin{multicols}{2}
	\begin{enumerate}
		\item $x \geq -2$
		\item $x  < 2$
		\item $-8 < x \leq 10$
		\item $1 - x \leq 4$
		\item $ 11 > 3 - 4x  $
		\item $12 \geq 3x \geq -9$
		\item $15 > 4 - 2x \geq -3$
	\end{enumerate}
	\end{multicols}
}{exe:vabs-2}{
	\begin{multicols}{2}
	\begin{enumerate}
		\item $[{-}2; {+}\infty[$
		\item $]{-}\infty ; 2[$
		\item $[-{8}; 10]$
		\item $[-3 ; {+}\infty[$
		\item $]2 ; {+}\infty[$
		\item $[-3 ; 4]$
		\item $\left] -\frac{11}2 ; \frac72 \right]$
	\end{enumerate}
	\end{multicols}

}


\exe{,difficulty=1}{
	Pour chaque contrainte suivante donner le plus grand ensemble dans lequel le nombre réel $x\in\R$ peut appartenir.
	\begin{multicols}{3}
	\begin{enumerate}
		\item $x^2 = 10$
		\item $| x - 2 | \leq 1$
		\item $| 2 + x| \leq -2$
		\item $| 3 + 2x | \leq 0$
		\item $(4 - 3x)^2 < 25$
		\item $-2x^2 = -12$
	\end{enumerate}
	\end{multicols}
}{exe:vabs-3}{

	\begin{multicols}{2}
	\begin{enumerate}
		\item $\bigset{-\sqrt{10} ; \sqrt{10} }$
		\item $[1 ; 3]$
		\item $\emptyset$, l'ensemble vide.
		\item $\bigset{ -\frac32 }$
		\item $\left] -\frac13; 3\right[$
		\item $\bigset{-\sqrt{6} ; \sqrt{6} }$
	\end{enumerate}
	\end{multicols}

}






\exe{,difficulty=2}{
	Soit $f(x) = -3 + (2x+1)^2$ définie sur $\R$.
	
	\begin{enumerate}
		%\item
		%Développer et réduire l'expression de $f$.
		\item
		Donner son extremum (minimum ou maximum) avec l'antécédent qui le réalise.
		\item
		Répondre aux questions suivantes en donnant le plus grand ensemble dans lequel le nombre réel $x\in\R$ peut appartenir.
	\end{enumerate}
	
	\begin{multicols}{2}
	\begin{enumerate}[label=\alph*), leftmargin=50pt]
		\item Quels $x$ vérifient $f(x) \leq -3$ ?
		\item Quels $x$ vérifient $f(x) \leq -4$ ?
		\item Quels $x$ vérifient $f(x) \leq 6$ ?
		\item Quels $x$ vérifient $f(x) \leq 10$ ?
		\item Quels $x$ vérifient $f(x) \leq 22$ ?
		\item Quels $x$ vérifient $f(x) > 22$ ?
	\end{enumerate}
	\end{multicols}
}{exe:vabs-5}{

	\begin{enumerate}
		%\item
		%	\[ f(x) = -3 + (2x+1)^2 = -3 + 4x^2 + 1 + 4x = 4x^2 + 4x - 2. \]
		\item
			Partons de la positivité du carré avec égalité lorsque nul.
				\begin{align*}
					(2x+1)^2 &\geq 0 && \text{avec égalité lorsque $2x+1=0$} \\
					-3 + (2x+1)^2 &\geq -3 && \text{avec égalité lorsque $2x = -1$} \\
					f(x) &\geq -3 && \text{avec égalité lorsque $x = -\dfrac12$}
				\end{align*}
			Il suit que $f$ atteint son minimum -3 en $x^\star = -\frac12$.
	\end{enumerate}
	
	\begin{enumerate}[label=\alph*)]
		\item
		Comme le minimum de $f$ est $-3$, seul $x^\star = -\frac12$ vérifie $f(x) \leq -3$.
			\[ \bigset{ x \in\R \tq f(x) \leq -3 } = \bigset{ - \dfrac12 }. \]
		\item
		Comme le minimum de $f$ est $-3$, aucun $x \in\R$ ne peut vérifier $f(x) \leq -4$.
			\[ \bigset{ x \in\R \tq f(x) \leq -4 } = \emptyset \text{ (l'ensemble vide). } \]
		\item 
		Posons $f(x) \leq 6$ et résolvons.
			\begin{align*}
				-3 + (2x+1)^2 &\leq 6 \\
				(2x+1)^2 &\leq 9 \\
				\sqrt{(2x+1)^2} &\leq \sqrt{9} \\
				|2x+1| &\leq 3
			\end{align*}
		où on a utilisé que $0 \leq x < y \implies \sqrt{x} < \sqrt{y}$.
		Pour poursuivre, il convient d'utiliser la propriété $|E| \leq a \iff -a \leq E \leq a$.
			\begin{alignat*}{2}
				-3 &\leq 2x+1 &\leq 3 \\
				-4 &\leq ~~~2x &\leq 2 \\
				-2 &\leq \quad x &\leq 1
			\end{alignat*}
		Il suit donc que
			\[ \bigset{ x \in\R \tq f(x) \leq 6 } = [-2 ; 1] \]
		On peut d'ailleurs vérifier que $f(-2) = -3 + (-3)^2 + 6$ et que $f(1) = -3 + (3)^2 + 6$, et que $f(x) \leq 6$ entre ces deux valeurs.
		\item
		Posons $f(x) \leq 10$ et résolvons.
			\begin{align*}
				-3 + (2x+1)^2 &\leq 10 \\
				(2x+1)^2 &\leq 13 \\
				\sqrt{(2x+1)^2} &\leq \sqrt{13} \\
				|2x+1| &\leq \sqrt{13}
			\end{align*}
		où on a utilisé que $0 \leq x < y \iff \sqrt{x} < \sqrt{y}$.
		Pour poursuivre, il convient d'utiliser la propriété $|E| \leq a \iff -a \leq E \leq a$.
			\begin{alignat*}{2}
				-\sqrt{13} &\leq 2x+1 &\leq& \sqrt{13} \\
				-\sqrt{13} -1  &\leq ~~~2x &\leq& \sqrt{13}-1 \\
				\dfrac{-\sqrt{13}-1}2 &\leq \quad x &\leq& \dfrac{\sqrt{13} -1}2
			\end{alignat*}
		Il suit donc que
			\[ \bigset{ x \in\R \tq f(x) \leq 10 } = \left[\dfrac{-\sqrt{13}-1}2 ; \dfrac{\sqrt{13}-1}2\right] \]
		\item 
		Posons $f(x) \leq 22$ et résolvons.
			\begin{align*}
				-3 + (2x+1)^2 &\leq 22 \\
				(2x+1)^2 &\leq 25 \\
				\sqrt{(2x+1)^2} &\leq \sqrt{25} \\
				|2x+1| &\leq 5
			\end{align*}
		où on a utilisé que $0 \leq x < y \iff \sqrt{x} < \sqrt{y}$.
		Pour poursuivre, il convient d'utiliser la propriété $|E| \leq a \iff -a \leq E \leq a$.
			\begin{alignat*}{2}
				-5 &\leq 2x+1 &\leq& 5 \\
				-6  &\leq ~~~2x &\leq& 4 \\
				-3 &\leq \quad x &\leq& 2
			\end{alignat*}
		Il suit donc que
			\[ \bigset{ x \in\R \tq f(x) \leq 22 } = \left[-3 ; 2\right] \]
		\item 
		Cet ensemble est le complémentaire à $\R$ du dernier ensemble, car on a soit $f(x) \leq 22$, soit $f(x) > 22$.
		Par conséquent,
			\[ \bigset{ x \in\R \tq f(x) \geq 22 } = \left[ \minfty ; -3\right[ \cup \left]2 ; \pinfty\right] \]
	\end{enumerate}
}

%\subsection*{Exercices supplémentaires}
\hrule

\exe{,difficulty=1}{
	Pour chaque inégalité suivante donner le plus grand ensemble dans lequel le nombre réel $x\in\R$ peut appartenir.
	\begin{multicols}{2}
	\begin{enumerate}
		\item $| x - 2 | \leq 1$
		\item $| 2 + x| > 4$
		\item $| 3 + 2x | \geq 1$
		\item $| 4 - 3x | < 5$
	\end{enumerate}
	\end{multicols}
}{exe:vabs-4}{

	\begin{multicols}{2}
	\begin{enumerate}
		\item $[1 ; 3]$
		\item $]{-}\infty ; -6[ \cup ]2 ; {+}\infty[$
		\item $]{-}\infty; -2] \cup [-1; {+}\infty[$
		\item $\left] -\frac13; 3\right[$
	\end{enumerate}
	\end{multicols}

}


\exe{,difficulty=2}{
	Trouver la ou les valeurs de $x \in \R$ vérifiant
	\begin{multicols}{2}
	\begin{enumerate}
		\item $|x - 1| = 3$
		\item $|2x-3| = 7$
		\item $|10-x| \leq 2$
		\item $|3 - 4x| \geq 2$
		\item $|x-2| = |x-4|$
		\item $| 3x + 8| = |4 - 2x |$
	\end{enumerate}
	\end{multicols}
}{exe:vabs-6}{

	\begin{multicols}{2}
	\begin{enumerate}
		\item $x \in \{-2 ; 4\}$
		\item $x \in \{5 ; -2\}$
		\item $[{-}8 ; 12]$
		\item $\left]{-}\infty ; \frac14\right] \cup \left[\frac54; {+}\infty\right[$
		\item $x = 3$
		\item $x \in \left\{-12 ; -\frac45\right\}$
	\end{enumerate}
	\end{multicols}

}

\exe{, difficulty=2}{
	Montrer le \emph{théorème de Léonie généralisé} : pour tout $x,y\in\R$, on a 
		\begin{align*}
			\min\{x,y\} = \dfrac{x+y}2 - \dfrac{|x-y|}2, && \text{ et } && \max\{x,y\} = \dfrac{x+y}2 + \dfrac{|x-y|}2.
		\end{align*}
	%\emph{$\min$ donne la plus petite valeur d'un ensemble fini et $\max$ la plus grande.}
}{exe:valeur-absolue-minmax}{
	Quels que soient $x, y\in\R$, le segment borné par $x$ et $y$ est de milieu $\dfrac{x+y}2$ et de longueur $|x-y|$.
	Ses extrémités s'expriment donc comme $c - r$ et $c+r$ où $c$ est le centre du segment et $r$ son rayon, c'est-à-dire la moitié de sa longueur.

	Algébriquement, si $x<y$, alors $\dfrac{x+y}2 - \dfrac{|x-y|}2 = \dfrac{x+y}2 - \dfrac{y-x}2 = \dfrac{(x+y)-(y-x)}2 = \dfrac{2x}2 = x = \min\{x ,y\}$.
	On peut faire de même pour le maximum puis dans le cas $x>y$ (ou se convaincre que le cas $x>y$ n'est pas à faire par symétrie des expressions).
}

\exe{, difficulty=2}{
	$x, y\geq0$. Montrer que $\sqrt{x}-\sqrt{y} = \dfrac{x-y}{\sqrt{x}+\sqrt{y}}$ et en déduire que $x < y \iff \sqrt{x} < \sqrt{y}$.
}{exe:sqrt-croissante}{
	L'identité remarquable $(a+b)(a-b) = a^2 - b^2$ implique que
		\[ \left(\sqrt{x}+\sqrt{y} \right)\left(\sqrt{x}-\sqrt{y} \right) = x - y. \]
	En divisant par $\sqrt{x}+\sqrt{y}$, on obtient l'égalité recherchée.
		\[ \sqrt{x}-\sqrt{y} = \dfrac{x-y}{\sqrt{x}+\sqrt{y}} \]
	Remarquons désormais que le membre de droite de l'égalité a le même signe que $x-y$, car $\sqrt{x} + \sqrt{y}$ est strictement positif comme somme de positifs stricts.
	Le signe de $\sqrt{x}-\sqrt{y}$ est donc le même que celui de $x-y$.
	
	En particulier, $x-y < 0 \iff \sqrt{x} - \sqrt{y} < 0$, ce qui conclut.
}

%%%%%%%%%%%

\newpage
\fancyhead[C]{\textbf{Solutions}}
\shipoutAnswer

\end{document}

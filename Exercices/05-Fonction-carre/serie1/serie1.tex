%!TEX encoding = UTF8
%!TEX root =notes.tex


%%%%%%%%%%%%%%%%%%%%%%%%%%%%%%%%%
% PACKAGE IMPORTS
%%%%%%%%%%%%%%%%%%%%%%%%%%%%%%%%%


\usepackage[french]{babel}

\usepackage[tmargin=2cm,rmargin=1in,lmargin=1in,margin=0.85in,bmargin=2cm,footskip=.2in]{geometry}
\usepackage{amsmath,amsfonts,amsthm,amssymb,mathtools}
\usepackage[varbb]{newpxmath}
\usepackage{xfrac}
\usepackage[makeroom]{cancel}
\usepackage{mathtools}
\usepackage{bookmark}
\usepackage{enumitem}
\usepackage{hyperref,theoremref}
\hypersetup{
	pdftitle={Assignment},
	colorlinks=true, linkcolor=doc!90,
	bookmarksnumbered=true,
	bookmarksopen=true
}
\usepackage[most,many,breakable]{tcolorbox}
\usepackage{xcolor}
\usepackage{varwidth}
\usepackage{varwidth}
\usepackage{etoolbox}
%\usepackage{authblk}
\usepackage{nameref}
\usepackage{multicol,array}
\usepackage{tikz-cd}
\usepackage[ruled,vlined,linesnumbered]{algorithm2e}
\usepackage{comment} % enables the use of multi-line comments (\ifx \fi) 
\usepackage{import}
\usepackage{xifthen}
\usepackage{pdfpages}
\usepackage{transparent}


\newcommand\mycommfont[1]{\footnotesize\ttfamily\textcolor{blue}{#1}}
\SetCommentSty{mycommfont}
\newcommand{\incfig}[1]{%
    \def\svgwidth{\columnwidth}
    \import{./figures/}{#1.pdf_tex}
}

\usepackage{tikzsymbols}
%\renewcommand\qedsymbol{$\Laughey$}


%\usepackage{import}
%\usepackage{xifthen}
%\usepackage{pdfpages}
%\usepackage{transparent}


%%%%%%%%%%%%%%%%%%%%%%%%%%%%%%
% SELF MADE COLORS
%%%%%%%%%%%%%%%%%%%%%%%%%%%%%%



\definecolor{myg}{RGB}{56, 140, 70}
\definecolor{myb}{RGB}{45, 111, 177}
\definecolor{myr}{RGB}{199, 68, 64}
\definecolor{mytheorembg}{HTML}{F2F2F9}
\definecolor{mytheoremfr}{HTML}{00007B}
\definecolor{mylenmabg}{HTML}{FFFAF8}
\definecolor{mylenmafr}{HTML}{983b0f}
\definecolor{mypropbg}{HTML}{f2fbfc}
\definecolor{mypropfr}{HTML}{191971}
\definecolor{myexamplebg}{HTML}{F2FBF8}
\definecolor{myexamplefr}{HTML}{88D6D1}
\definecolor{myexampleti}{HTML}{2A7F7F}
\definecolor{mydefinitbg}{HTML}{E5E5FF}
\definecolor{mydefinitfr}{HTML}{3F3FA3}
\definecolor{notesgreen}{RGB}{0,162,0}
\definecolor{myp}{RGB}{197, 92, 212}
\definecolor{mygr}{HTML}{2C3338}
\definecolor{myred}{RGB}{127,0,0}
\definecolor{myyellow}{RGB}{169,121,69}
\definecolor{myexercisebg}{HTML}{F2FBF8}
\definecolor{myexercisefg}{HTML}{88D6D1}


%%%%%%%%%%%%%%%%%%%%%%%%%%%%
% TCOLORBOX SETUPS
%%%%%%%%%%%%%%%%%%%%%%%%%%%%

\setlength{\parindent}{1cm}
%================================
% THEOREM BOX
%================================

\tcbuselibrary{theorems,skins,hooks}
\newtcbtheorem[number within=chapter]{Theorem}{Théorème}
{%
	enhanced,
	breakable,
	colback = mytheorembg,
	frame hidden,
	boxrule = 0sp,
	borderline west = {2pt}{0pt}{mytheoremfr},
	sharp corners,
	detach title,
	before upper = \tcbtitle\par\smallskip,
	coltitle = mytheoremfr,
	fonttitle = \bfseries\sffamily,
	description font = \mdseries,
	separator sign none,
	segmentation style={solid, mytheoremfr},
}
{th}


\tcbuselibrary{theorems,skins,hooks}
\newtcolorbox{Theoremcon}
{%
	enhanced
	,breakable
	,colback = mytheorembg
	,frame hidden
	,boxrule = 0sp
	,borderline west = {2pt}{0pt}{mytheoremfr}
	,sharp corners
	,description font = \mdseries
	,separator sign none
}

%================================
% Corollery
%================================
\tcbuselibrary{theorems,skins,hooks}
\newtcbtheorem[use counter=tcb@cnt@Theorem]{Corollary}{Corollaire}
{%
	enhanced
	,breakable
	,colback = myp!10
	,frame hidden
	,boxrule = 0sp
	,borderline west = {2pt}{0pt}{myp!85!black}
	,sharp corners
	,detach title
	,before upper = \tcbtitle\par\smallskip
	,coltitle = myp!85!black
	,fonttitle = \bfseries\sffamily
	,description font = \mdseries
	,separator sign none
	,segmentation style={solid, myp!85!black}
}
{th}

%================================
% LENMA
%================================

\tcbuselibrary{theorems,skins,hooks}
\newtcbtheorem[use counter=tcb@cnt@Theorem]{Lemma}{Lemme}
{%
	enhanced,
	breakable,
	colback = mylenmabg,
	frame hidden,
	boxrule = 0sp,
	borderline west = {2pt}{0pt}{mylenmafr},
	sharp corners,
	detach title,
	before upper = \tcbtitle\par\smallskip,
	coltitle = mylenmafr,
	fonttitle = \bfseries\sffamily,
	description font = \mdseries,
	separator sign none,
	segmentation style={solid, mylenmafr},
}
{th}


%================================
% PROPOSITION
%================================

\tcbuselibrary{theorems,skins,hooks}
\newtcbtheorem[use counter=tcb@cnt@Theorem]{Prop}{Proposition}
{%
	enhanced,
	breakable,
	colback = mypropbg,
	frame hidden,
	boxrule = 0sp,
	borderline west = {2pt}{0pt}{mypropfr},
	sharp corners,
	detach title,
	before upper = \tcbtitle\par\smallskip,
	coltitle = mypropfr,
	fonttitle = \bfseries\sffamily,
	description font = \mdseries,
	separator sign none,
	segmentation style={solid, mypropfr},
}
{th}


%================================
% CLAIM
%================================

\tcbuselibrary{theorems,skins,hooks}
\newtcbtheorem[use counter=tcb@cnt@Theorem]{claim}{Claim}
{%
	enhanced
	,breakable
	,colback = myg!10
	,frame hidden
	,boxrule = 0sp
	,borderline west = {2pt}{0pt}{myg}
	,sharp corners
	,detach title
	,before upper = \tcbtitle\par\smallskip
	,coltitle = myg!85!black
	,fonttitle = \bfseries\sffamily
	,description font = \mdseries
	,separator sign none
	,segmentation style={solid, myg!85!black}
}
{th}



%================================
% Exercise
%================================

\tcbuselibrary{theorems,skins,hooks}
\newtcbtheorem[use counter=tcb@cnt@Theorem]{Exercise}{Exercice}
{%
	enhanced,
	breakable,
	colback = myexercisebg,
	frame hidden,
	boxrule = 0sp,
	borderline west = {2pt}{0pt}{myexercisefg},
	sharp corners,
	detach title,
	before upper = \tcbtitle\par\smallskip,
	coltitle = myexercisefg,
	fonttitle = \bfseries\sffamily,
	description font = \mdseries,
	separator sign none,
	segmentation style={solid, myexercisefg},
}
{th}

%================================
% EXAMPLE BOX
%================================

\newtcbtheorem[use counter=tcb@cnt@Theorem]{Example}{Exemple}
{%
	colback = myexamplebg
	,breakable
	,colframe = myexamplefr
	,coltitle = myexampleti
	,boxrule = 1pt
	,sharp corners
	,detach title
	,before upper=\tcbtitle\par\smallskip
	,fonttitle = \bfseries
	,description font = \mdseries
	,separator sign none
	,description delimiters parenthesis
}
{ex}

%================================
% DEFINITION BOX
%================================

\newtcbtheorem[use counter=tcb@cnt@Theorem]{Definition}{Définition}{enhanced,
	before skip=2mm,after skip=2mm, colback=red!5,colframe=red!80!black,boxrule=0.5mm,
	attach boxed title to top left={xshift=1cm,yshift*=1mm-\tcboxedtitleheight}, varwidth boxed title*=-3cm,
	boxed title style={frame code={
					\path[fill=tcbcolback]
					([yshift=-1mm,xshift=-1mm]frame.north west)
					arc[start angle=0,end angle=180,radius=1mm]
					([yshift=-1mm,xshift=1mm]frame.north east)
					arc[start angle=180,end angle=0,radius=1mm];
					\path[left color=tcbcolback!60!black,right color=tcbcolback!60!black,
						middle color=tcbcolback!80!black]
					([xshift=-2mm]frame.north west) -- ([xshift=2mm]frame.north east)
					[rounded corners=1mm]-- ([xshift=1mm,yshift=-1mm]frame.north east)
					-- (frame.south east) -- (frame.south west)
					-- ([xshift=-1mm,yshift=-1mm]frame.north west)
					[sharp corners]-- cycle;
				},interior engine=empty,
		},
	fonttitle=\bfseries,
	title={#2},#1}{def}

%================================
% Solution BOX
%================================

\makeatletter
\newtcbtheorem[use counter=tcb@cnt@Theorem]{question}{Question}{enhanced,
	breakable,
	colback=white,
	colframe=myb!80!black,
	attach boxed title to top left={yshift*=-\tcboxedtitleheight},
	fonttitle=\bfseries,
	title={#2},
	boxed title size=title,
	boxed title style={%
			sharp corners,
			rounded corners=northwest,
			colback=tcbcolframe,
			boxrule=0pt,
		},
	underlay boxed title={%
			\path[fill=tcbcolframe] (title.south west)--(title.south east)
			to[out=0, in=180] ([xshift=5mm]title.east)--
			(title.center-|frame.east)
			[rounded corners=\kvtcb@arc] |-
			(frame.north) -| cycle;
		},
	#1
}{def}
\makeatother

%================================
% SOLUTION BOX
%================================

\makeatletter
\newtcolorbox{solution}{enhanced,
	breakable,
	colback=white,
	colframe=myg!80!black,
	attach boxed title to top left={yshift*=-\tcboxedtitleheight},
	title=Solution,
	boxed title size=title,
	boxed title style={%
			sharp corners,
			rounded corners=northwest,
			colback=tcbcolframe,
			boxrule=0pt,
		},
	underlay boxed title={%
			\path[fill=tcbcolframe] (title.south west)--(title.south east)
			to[out=0, in=180] ([xshift=5mm]title.east)--
			(title.center-|frame.east)
			[rounded corners=\kvtcb@arc] |-
			(frame.north) -| cycle;
		},
}
\makeatother

%================================
% Question BOX
%================================

\makeatletter
\newtcbtheorem[use counter=tcb@cnt@Theorem]{qstion}{Question}{enhanced,
	breakable,
	colback=white,
	colframe=mygr,
	attach boxed title to top left={yshift*=-\tcboxedtitleheight},
	fonttitle=\bfseries,
	title={#2},
	boxed title size=title,
	boxed title style={%
			sharp corners,
			rounded corners=northwest,
			colback=tcbcolframe,
			boxrule=0pt,
		},
	underlay boxed title={%
			\path[fill=tcbcolframe] (title.south west)--(title.south east)
			to[out=0, in=180] ([xshift=5mm]title.east)--
			(title.center-|frame.east)
			[rounded corners=\kvtcb@arc] |-
			(frame.north) -| cycle;
		},
	#1
}{def}
\makeatother

\newtcbtheorem[number within=chapter]{wconc}{Wrong Concept}{
	breakable,
	enhanced,
	colback=white,
	colframe=myr,
	arc=0pt,
	outer arc=0pt,
	fonttitle=\bfseries\sffamily\large,
	colbacktitle=myr,
	attach boxed title to top left={},
	boxed title style={
			enhanced,
			skin=enhancedfirst jigsaw,
			arc=3pt,
			bottom=0pt,
			interior style={fill=myr}
		},
	#1
}{def}



%================================
% NOTE BOX
%================================

\usetikzlibrary{arrows,calc,shadows.blur}
\tcbuselibrary{skins}
\newtcolorbox{note}[1][]{%
	enhanced jigsaw,
	colback=gray!20!white,%
	colframe=gray!80!black,
	size=small,
	boxrule=1pt,
	title=\colorbox{white!100}{\textbf{ Remarque }},
	halign title=flush center,
	coltitle=black,
	breakable,
	drop shadow=black!50!white,
	attach boxed title to top left={xshift=1cm,yshift=-\tcboxedtitleheight/2,yshifttext=-\tcboxedtitleheight/2},
	minipage boxed title=2.6cm,
	boxed title style={%
			colback=white,
			size=fbox,
			boxrule=1pt,
			boxsep=2pt,
			underlay={%
					\coordinate (dotA) at ($(interior.west) + (-0.5pt,0)$);
					\coordinate (dotB) at ($(interior.east) + (0.5pt,0)$);
					\begin{scope}
						\clip (interior.north west) rectangle ([xshift=3ex]interior.east);
						\filldraw [white, blur shadow={shadow opacity=60, shadow yshift=-.75ex}, rounded corners=2pt] (interior.north west) rectangle (interior.south east);
					\end{scope}
					\begin{scope}[gray!80!black]
						\fill (dotA) circle (2pt);
						\fill (dotB) circle (2pt);
					\end{scope}
				},
		},
	#1,
}

%================================
% STRATÉGIE BOX
%================================

\usetikzlibrary{arrows,calc,shadows.blur}
\tcbuselibrary{skins}
\newtcolorbox{strategy}[1][]{%
	enhanced jigsaw,
	colback=myb!20!white,%
	colframe=gray!80!black,
	size=small,
	boxrule=1pt,
	title=\colorbox{white!100}{\textbf{ Stratégie }},
	halign title=flush center,
	coltitle=black,
	breakable,
	drop shadow=black!50!white,
	attach boxed title to top left={xshift=1cm,yshift=-\tcboxedtitleheight/2,yshifttext=-\tcboxedtitleheight/2},
	minipage boxed title=2.5cm,
	boxed title style={%
			colback=white,
			size=fbox,
			boxrule=1pt,
			boxsep=2pt,
			underlay={%
					\coordinate (dotA) at ($(interior.west) + (-0.5pt,0)$);
					\coordinate (dotB) at ($(interior.east) + (0.5pt,0)$);
					\begin{scope}
						\clip (interior.north west) rectangle ([xshift=3ex]interior.east);
						\filldraw [white, blur shadow={shadow opacity=60, shadow yshift=-.75ex}, rounded corners=2pt] (interior.north west) rectangle (interior.south east);
					\end{scope}
					\begin{scope}[gray!80!black]
						\fill (dotA) circle (2pt);
						\fill (dotB) circle (2pt);
					\end{scope}
				},
		},
	#1,
}

%================================
% MÉTHODE BOX
%================================

\usetikzlibrary{arrows,calc,shadows.blur}
\tcbuselibrary{skins}
\newtcolorbox{methode}[1][]{%
	enhanced jigsaw,
	colback=white,%
	colframe=gray!80!black,
	size=small,
	boxrule=1pt,
	title=\textbf{Méthode},
	halign title=flush center,
	coltitle=black,
	breakable,
	drop shadow=black!50!white,
	attach boxed title to top left={xshift=1cm,yshift=-\tcboxedtitleheight/2,yshifttext=-\tcboxedtitleheight/2},
	minipage boxed title=2.5cm,
	boxed title style={%
			colback=white,
			size=fbox,
			boxrule=1pt,
			boxsep=2pt,
			underlay={%
					\coordinate (dotA) at ($(interior.west) + (-0.5pt,0)$);
					\coordinate (dotB) at ($(interior.east) + (0.5pt,0)$);
					\begin{scope}
						\clip (interior.north west) rectangle ([xshift=3ex]interior.east);
						\filldraw [white, blur shadow={shadow opacity=60, shadow yshift=-.75ex}, rounded corners=2pt] (interior.north west) rectangle (interior.south east);
					\end{scope}
					\begin{scope}[gray!80!black]
						\fill (dotA) circle (2pt);
						\fill (dotB) circle (2pt);
					\end{scope}
				},
		},
	#1,
}

%%%%%%%%%%%%%%%%%%%%%%%%%%%%%%%%%%%%%%%%%%%
% TABLE OF CONTENTS
%%%%%%%%%%%%%%%%%%%%%%%%%%%%%%%%%%%%%%%%%%%

\usepackage{tikz}

\definecolor{doc}{RGB}{0,60,110}
\usepackage{titletoc}
\contentsmargin{0cm}
\titlecontents{chapter}[3.7pc]
{\addvspace{30pt}%
	\begin{tikzpicture}[remember picture, overlay]%
		\draw[fill=doc!60,draw=doc!60] (-7,-.1) rectangle (-0.2,.6);%
		\pgftext[left,x=-3.5cm,y=0.2cm]{\color{white}\Large\sc\bfseries Chapitre\ \thecontentslabel};%
	\end{tikzpicture}\color{doc!60}\large\sc\bfseries}%
{}
{}
{\;\titlerule\;\large\sc\bfseries Page \thecontentspage
	\begin{tikzpicture}[remember picture, overlay]
		\draw[fill=doc!60,draw=doc!60] (2pt,0) rectangle (4,0.1pt);
	\end{tikzpicture}}%
\titlecontents{section}[3.7pc]
{\addvspace{2pt}}
{\contentslabel[\thecontentslabel]{2pc}}
{}
{\hfill\small \thecontentspage}
[]
\titlecontents*{subsection}[3.7pc]
{\addvspace{-1pt}\small}
{}
{}
{\ --- \small\thecontentspage}
[ \textbullet\ ][]

\makeatletter
\renewcommand{\tableofcontents}{%
	\chapter*{%
	  \vspace*{-20\p@}%
	  \begin{tikzpicture}[remember picture, overlay]%
		  \pgftext[right,x=15cm,y=0.2cm]{\color{doc!60}\Huge\sc\bfseries \contentsname};%
		  \draw[fill=doc!60,draw=doc!60] (13,-.75) rectangle (20,1);%
		  \clip (13,-.75) rectangle (20,1);
		  \pgftext[right,x=15cm,y=0.2cm]{\color{white}\Huge\sc\bfseries \contentsname};%
	  \end{tikzpicture}}%
	\@starttoc{toc}}
\makeatother


%%%%%%%%%%%%%%%%%%%%%%%%%%%%%%%%%%%%%%%%%%%
% MINTED FOR PYTHON ALGORITHMS
%%%%%%%%%%%%%%%%%%%%%%%%%%%%%%%%%%%%%%%%%%%

\usepackage{tcolorbox}
\tcbuselibrary{minted,breakable,xparse,skins}
\definecolor{bg}{gray}{0.95}
\DeclareTCBListing{mintedbox}{O{}m!O{}}{%
  breakable=true,
  listing engine=minted,
  listing only,
  minted language=#2,
  minted style=default,
  minted options={%
    linenos,
    gobble=0,
    breaklines=true,
    breakafter=,,
    fontsize=\small,
    numbersep=8pt,
    #1},
  boxsep=0pt,
  left skip=0pt,
  right skip=0pt,
  left=25pt,
  right=0pt,
  top=3pt,
  bottom=3pt,
  arc=5pt,
  leftrule=0pt,
  rightrule=0pt,
  bottomrule=2pt,
  toprule=2pt,
  colback=bg,
  colframe=orange!70,
  enhanced,
  overlay={%
    \begin{tcbclipinterior}
    \fill[orange!20!white] (frame.south west) rectangle ([xshift=20pt]frame.north west);
    \end{tcbclipinterior}},
  #3}
  
  
 % for braces
\usetikzlibrary{decorations.pathreplacing}


\SetDate[06/01/2026]

\begin{document}
\pagestyle{fancy}
\fancyhead[L]{Seconde}
\fancyhead[C]{\textbf{Fonctions carré et racine carrée}}
\fancyhead[R]{\today}

%%% SQ

\exe{}{
	En supposant que $f$ et $g$ soient paires, continuer leur graphe sur tout le domaine.
	\begin{center}
	\begin{tikzpicture}[>=stealth, scale=1]
		\begin{axis}[xmin = -4.1, xmax=4.1, ymin=-3.1, ymax=6.1, axis x line=middle, axis y line=middle, axis line style=->, grid=both, x=1.5cm]
			\addplot[no marks,BLUE_E, -, very thick] expression[domain=-4:-.001, samples=300]{sin(180*x)/x} node[pos=.92, left]{$\C_f$};
			\addplot[no marks,GREEN_E, -, very thick] expression[domain=.001:4, samples=300]{1+x*cos(180*x)} node[pos=.92, left]{$\C_g$};
		\end{axis}
	\end{tikzpicture}
	\end{center}
}{exe:draw-paire}{
	\begin{center}
	\begin{tikzpicture}[>=stealth, scale=1]
		\begin{axis}[xmin = -4.1, xmax=4.1, ymin=-3.1, ymax=6.1, axis x line=middle, axis y line=middle, axis line style=->, grid=both, x=1.5cm]
			\addplot[no marks,BLUE_E, -, very thick] expression[domain=-4:4, samples=300]{sin(180*x)/x} node[pos=.46, left]{$\C_f$};
			\addplot[no marks,GREEN_E, -, very thick] expression[domain=.001:4, samples=300]{1+x*cos(180*x)} node[pos=.92, left]{$\C_g$};
			\addplot[no marks,GREEN_E, -, very thick] expression[domain=-4:.001, samples=300]{1-x*cos(180*x)};
		\end{axis}
	\end{tikzpicture}
	\end{center}
}

\exe{, difficulty=1}{
	Montrer que les fonctions suivantes sont paires.
	\begin{multicols}{2}
	\begin{enumerate}
		\item $f(x) = 2x^2$
		\item $g(x) = 4x^2 + 7$
		\item $h(x) = \dfrac{1}{4x^2 + 7}$
		\item $k(x) = x^4$
	\end{enumerate}
	\end{multicols}
}{exe:paires}{
	Pour tout $x\in\R$ on a bien 
	%\begin{multicols}{2}
	\begin{enumerate}
		\item $f(-x) = 2(-x)^2 = 2x^2 = f(x)$.
		\item $g(-x) = 4(-x)^2 + 7 = 4x^2 + 7 = g(x)$
		\item $h(-x) = \dfrac{1}{4(-x)^2 + 7} = \dfrac{1}{4x^2 + 7} = h(x)$
		\item $k(-x) = (-x)^4 = \bigl( (-x)^2 \bigr)^2 = \bigl( x^2 \bigr)^2 = x^4 = k(x)$
	\end{enumerate}
	%\end{multicols}
}


\exe{, difficulty=0}{
	Montrer que la fonction $f(x) = x^2 - x^3$ n'est pas paire.
}{exe:non-paire}{
	Supposons que $f$ soit paire.
	Par définition, $f(-x) = f(x)$ pour tout $x\in\R$.
	
	Comme $(-x)^3 = (-x)(-x)^2 = -x^3$, on a $f(-x) = x^2 + x^3$.
	Par conséquent, $f(x) = f(-x) \iff x^2 - x^3 = x^2 + x^3 \iff x^3 = 0$.
	Or ceci n'est pas vrai pour tout $x\in\R$ : prendre n'importe quel $x$ non nul fournit un contre-exemple.
	Ainsi pour $x=1$, $f(1) = 0$ et $f(-1) = 2$ et $f$ n'est en effet pas paire.
}


\exe{}{
	Développer les expressions algébriques suivantes.
		\begin{multicols}{3}
		\begin{enumerate}[label=$\bullet$]
			\item $f(x) = (1+x)^2$
			\item $g(x) = (x-3)^2$
			\item $h(x) = (3x+2)(2-3x)$
			\item $F(x) = (3 + 2x)^2$
			\item $G(x) = (3x - 7)^2$
			\item $H(x) = (-7x - 2)^2$
		\end{enumerate}
		\end{multicols}
}{exe:developpement}{
		\begin{enumerate}[label=$\bullet$]
			\item $f(x) = (1+x)^2 = 1 + x^2 + 2x$
			\item $g(x) = (x-3)^2 = x^2 + 9 - 6x$
			\item $h(x) = (3x+2)(2-3x) = (2+3x)(2-3x) = 4 - (3x)^2 = 4-9x^2$
			\item $F(x) = (3 + 2x)^2 = 9 + 4x^2 + 12x$
			\item $G(x) = (3x - 7)^2 = 9x^2 + 49 - 42x$
			\item $H(x) = (-7x - 2)^2 = (7x+2)^2 = 49x^2 + 4 + 28x$
		\end{enumerate}
}



\exe{, difficulty=1}{
	Soit $f$ la fonction définie sur $\R$ par
		\[ f(x) = -10 + 3(3x-1)^2. \]
	\begin{enumerate}
		\item Développer réduire l'expression algébrique de $f$.
		\item Montrer que $f$ atteint son minimum en $x^\star=\frac13$ et donner sa valeur.
	\end{enumerate}
}{exe:extrema4-99}{
	Développons et réduisons calmement :
		\begin{align*}
			f(x) &= -10 + 3(3x-1)^2 \\
				&= -10 + 3 \bigl[ (3x)^2 + 1^2 - 2\times3x \bigr] \\
				&= -10 + 3 \bigl[ 9x^2 + 1 - 6x \bigr] \\
				&= -10 + 27x^2 + 3 - 18x \\
				&= 27x^2 - 18x - 7
		\end{align*}

	Lorsqu'on a affaire à une forme avec carré, on part systématiquement du fait qu'un carré est toujours positif pour construire $f(x)$.
	Pour tout $x\in\R$ réel, on a donc
		\begin{align*}
			(3x-1)^2 &\geq 0 \\
			3(3x-1)^2 &\geq 0 \\
			-10 + 3(3x-1)^2 &\geq -10 \\
			f(x) &\geq -10
		\end{align*}
	On en déduit que $f(x)$ est borné inférieurement par $-10$ pour tous les $x\in\R$ réels.
	
	Pour montrer que c'est un minimum atteint en $\frac13$, on calcule $f(\frac13) = -10$.
	En conclusion,
		\[ f(x) \geq f\left(\dfrac13\right)=-10, \]
	et ce pour tous les $x\in\R$. Par définition, $-10$ est le minimum de $f$, atteint en $\frac13$.
}

\exe{, difficulty=1}{
	Soit $f$ la fonction définie sur $\R$ par
		\[ f(x) = -3 - (x+1)^2. \]
	\begin{enumerate}
		\item Développer réduire l'expression algébrique de $f$.
		\item Donner le maximum de $f$ ainsi que l'antécédent $x^\star$ qui le réalise.
	\end{enumerate}
}{exe:extrema3-99}{
	Développons et réduisons en faisant attention au signe moins qui multiplie le carré tout entier :
		\begin{align*}
			f(x) &= -3 - (x+1)^2 \\
				&=-3 - \bigl[ x^2 + 2x + 1 \bigr] \\
				&= -3 -x^2 - 2x - 1 \\
				&= -x^2 - 2x - 4
		\end{align*}
		
	Lorsqu'on a affaire à une forme avec carré, on part systématiquement du fait qu'un carré est toujours positif pour enfin construire $f(x)$.
	Pour tout $x\in\R$ réel, on a donc
		\begin{align*}
			(x+1)^2 &\geq 0 \\
			-(x+1)^2 &\leq 0 \\
			-3 - (x+1)^2 &\leq -3 \\
			f(x) &\leq -3
		\end{align*}
	On en déduit que $f(x)$ est borné supérieurement par $-3$ pour tous les $x\in\R$ réels.
	
	Pour montrer que c'est un maximum atteint en $-1$, on calcule $f(-1) = -3 - (-1+1)^2 = -3 + 0^2 = -3$.
	En conclusion,
		\[ f(x) \leq f(-1)=-3, \]
	et ce pour tous les $x\in\R$. Par définition, $-3$ est le maximum de $f$, atteint en $-1$.
}


\exe{, difficulty=2}{
	Considérons la courbe représentative de la fonction $f(x) = x + 1$ et le point $A(2;1)$.
	Le but de l'exercice est de trouver le point de $\C_f$ le plus proche de $A$.
	\begin{enumerate}
		\item
		Grapher $\C_f$ sur le domaine $[0 ; 4]$ et placer le point $A$ dans un repère orthonormé.
		Le point $A$ appartient-il à $\C_f$ ? Justifier.
		\item
		Montrer qu'un point de $\C_f$ est de la forme $B(x ; x+1)$, avec $x\in\R$.
		\item
		Montrer que la distance au carré de $B$ à $A$ est égale à
			\[ AB^2 =  2x^2 - 4x + 4 = 2(x-1)^2 + 2. \]
		\emph{Formule de la distance au carré : $AB^2 = (x_A - x_B)^2 + (y_A - y_B)^2$.}
		\item
		Conclure que le point de $\C_f$ le plus proche de $A$ est $B(1 ; 2)$.
		Placer $B$ dans le repère déjà construit.
	\end{enumerate}
}{exe:proj-xp1}{
	\begin{center}
	\begin{tikzpicture}[>=stealth]
		\begin{axis}[xmin = 0, xmax=4, ymin=0, ymax=5, axis x line=middle, axis y line=middle, axis line style=->, grid=both, x=50pt, y=50pt, extra x ticks ={0}, extra y ticks = {0}, ytick distance=1,]
			\addplot[no marks, BLUE_E, very thick, -] expression[domain=0:4, samples=2]{x+1} 
			node[above left, pos=.8] {$\C_f$};
			\addplot[RED_E, mark=*, mark size = 2] (2,1) node[right] {$A$};
			\addplot[GREEN_E, mark=*, mark size = 2] (1,2) node[left] {$B$};
		\end{axis}
	\end{tikzpicture}
	\end{center}
	\begin{enumerate}
		\item
		D'après la propriété fondamentale, $A \not\in \C_f$ car $f(x_A) = f(2) = 2+1 = 3 \neq y_A$.
		\item
		D'après la propriété fondamentale, un point $(x ; y)$ appartient $\C_f$ si et seulement si $y = f(x) = x+1$.
		Tous les points de $\C_f$ sont donc de la forme $(x ; x+1)$.
		\item
		D'après la formule de la distance au carré, celle-ci est donnée par
			\begin{align*}
				(x - x_A)^2 + (x+1 - y_A)^2 &= (x-2)^2 + (x+1-1)^2,
											\\ &= x^2 - 4x + 4 + x^2,
											  \\ &= 2x^2 - 4x + 4.
			\end{align*}
		D'autre part, en développant $2(x-1)^2 + 2$, on obtient
			\begin{align*}
				2(x-1)^2 + 2 &= 2(x-1)(x-1) + 2,
							\\ &= 2[x^2 - x - x + 1] + 2,
							\\ &= 2x^2 - 4x + 2 + 2,
							\\ &= 2x^2 - 4x + 4,
			\end{align*}
		ce qui montre l'identité recherchée.
		\item
		Remarquons qu'un carré est toujours positif : l'expression $2(x-1)^2 + 2$ est donc toujours supérieure ou égale à 2, avec égalité quand $x-1$ est nul, c'est-à-dire quand $x=1$.
		En $x=1$, le point $(x ; x+1)$ correspond bien au $B(1 ; 2)$ proposé.
	\end{enumerate}
}

\newpage

%%% SQRT

\exe{}{
	Compléter les expressions.
	\begin{multicols}{3}
	\begin{enumerate}
		\item $\sqrt{25} = $
		\item $\sqrt{81} = $
		\item $\sqrt{121} = $
		\item $\sqrt{\phantom{xxx}} = 25$
		\item $\sqrt{\phantom{xxx}} = 12$
		\item $\sqrt{\phantom{xxx}} = 10^3$
	\end{enumerate}
	\end{multicols}
}{exe:sqrt1}{
	\begin{multicols}{3}
	\begin{enumerate}
		\item $\sqrt{25} = 5$
		\item $\sqrt{81} = 9$
		\item $\sqrt{121} = 11$
		\item $\sqrt{625} = 25$
		\item $\sqrt{144} = 12$
		\item $\sqrt{10^6} = 10^3$
	\end{enumerate}
	\end{multicols}
}

\exe{}{
	Calculer les carrés et racines carrées suivantes.
	\begin{multicols}{3}
	\begin{enumerate}
		\item $\sqrt{7^2}$
		\item $\sqrt{17}^2$
		\item $\sqrt{(-9)^2}$
		\item $\sqrt{10^4}$
		\item $\left(-\sqrt{4}\right)^2$
		\item $-\sqrt{15^2}$
	\end{enumerate}
	\end{multicols}
}{exe:sqrt2}{
	\begin{multicols}{3}
	\begin{enumerate}
		\item $\sqrt{7^2} = 7$
		\item $\sqrt{17}^2 = 17$
		\item $\sqrt{(-9)^2} = 9$
		\item $\sqrt{10^4} = 10^2 = 100$
		\item $\left(-\sqrt{4}\right)^2 = 4$
		\item $-\sqrt{15^2} = -15$
	\end{enumerate}
	\end{multicols}
}


\exe{}{
	Montrer que $\sqrt{7^{4052}} =7^{2026}$.
}{exe:sqrt8}{
	Le carré de $7^{2026}$ est égal à
		\[ \bigl(7^{2026}\bigr)^2 = 7^{2026 \times 2} = 7^{4052}. \]
	En outre, par définition, $\sqrt{7^{4052}}$ est l'unique nombre positif donc le carré est $7^{4052}$.
	C'est bien le cas de $7^{2026}$ : il est positif et son carré vaut $7^{4052}$.
	Il suit donc que $\sqrt{7^{4052}} =7^{2026}$.
}

\exe{}{
	Est-ce que l'égalité suivante est vraie ? $\sqrt{(-2)^{26}} \stackrel{?}{=} (-2)^{13}$
}{exe:sqrt9}{
	Par définition, $\sqrt{(-2)^{26}}$ est l'unique nombre positif donc le carré est $(-2)^{26}$.
	Or, bien que le carré de $(-2)^{13}$ valent $(-2)^{26}$, ce nombre n'est pas positif !
	En effet, 
		\[ (-2)^{13} = (-2) \times (-2)^{2\times6} = -2 \times 4^6 < 0. \]
	Plus généralement, multiplier un nombre impair de négatifs donne un négatif.
	
	L'égalité est donc fausse car le membre de gauche est positif alors que celui de droite est négatif.
}


% je vois pas trop l'intérêt de cet exo.
% Valérie en avait des mieux dans son plan de travail je crois.

%\exe{}{
%	Calculer les produits suivants.
%	\begin{multicols}{3}
%	\begin{enumerate}
%		\item $\sqrt{9} \cdot \sqrt{100}$
%		\item 
%		\item $\sqrt{169} \cdot \sqrt{81}$
%		\item $\sqrt{169 \cdot 81}$
%		\item $\sqrt{0,16} \cdot \sqrt{900}$
%		\item $\sqrt{0,16 \cdot 900}$
%	\end{enumerate}
%	\end{multicols}
%}{exe:sqrt3}{
%	TODO
%}

\exe{}{
	Écrire les nombres suivants sous la forme $a\sqrt{b}$ où $a\in\N$ est entier et $b\in\N$ est le \underline{plus petit} \underline{entier possible}.

	\begin{multicols}{3}
	\begin{enumerate}
		\item $\sqrt{12}$
		\item $\sqrt{150}$
		\item $5\sqrt{96}$
		\item $2\sqrt{300}$
		\item $\dfrac{12}{\sqrt{3}}$
		\item $\dfrac{18}{\sqrt{6}}$
		\item $\sqrt{125}$
	\end{enumerate}
	\end{multicols}
}{exe:sqrt4}{
	\begin{multicols}{2}
	\begin{enumerate}
		\item $\sqrt{12} = \sqrt{4\times3} = 2\sqrt3$
		\item $\sqrt{150} = \sqrt{25\times6} = 5\sqrt6$
		\item $5\sqrt{96} = 5\sqrt{16\times6} = 20\sqrt6$
		\item $2\sqrt{300} = 2\sqrt{3\times100} = 2\times10\sqrt3 = 20\sqrt3$
		\item $\dfrac{12}{\sqrt{3}} = \dfrac{12\sqrt3}{\sqrt3^2} = \dfrac{12\sqrt3}{3} = 4\sqrt3$
		\item $\dfrac{18}{\sqrt{6}} = \dfrac{18\sqrt6}{6} = 3\sqrt6$
		\item $\sqrt{125} = \sqrt{25\times5} = 5\sqrt5$
	\end{enumerate}
	\end{multicols}
}


\exe{}{
	Écrire les nombres suivants sous la forme $\sqrt{a}$ où $a\in\N$ est entier.
	\begin{multicols}{3}
	\begin{enumerate}
		\item $3\sqrt{2}$
		\item $50\sqrt{0,5}$
		\item $\dfrac12 \sqrt{48}$
	\end{enumerate}
	\end{multicols}
}{exe:sqrt5}{
	\begin{enumerate}
		\item $3\sqrt{2} = \sqrt{9\times2} = \sqrt{18}$
		\item $50\sqrt{0,5} = \sqrt{2500\times0,5} = \sqrt{1250}$
		\item $\dfrac12 \sqrt{48} = \sqrt{\left(\dfrac12\right)^2 \times 48}= \sqrt{\dfrac14 \times 48} = \sqrt{12}$
	\end{enumerate}
}

\exe{}{
	Développer les expressions suivantes.
		\begin{multicols}{3}
		\begin{enumerate}%[label=\arabic*)]
			\item $\left(1+\sqrt2\right)\left(1-\sqrt2\right)$
			\item $\left(1+\sqrt2\right)^2$
			\item $\left(1-\sqrt2\right)^2$
			\item $\left(\sqrt{3} - 2\sqrt2\right)^2$
			\item $\left(\sqrt5x + 2 \right)^2$
			\item $\left(\sqrt5x - \sqrt7\right)^2$
		\end{enumerate}
		\end{multicols}
}{exe:développement-sqrt}{
	\begin{enumerate}
		\item $\left(1+\sqrt2\right)\left(1-\sqrt2\right) = 1-\sqrt2^2 = 1-2 = -1$
		\item $\left(1+\sqrt2\right)^2 = 1 + \sqrt2^2 + 2\sqrt2 = 3 + 2\sqrt2$
		\item $\left(1-\sqrt2\right)^2 = 1 + 2 - 2\sqrt2 = 3 - 2\sqrt2$
		\item $\left(\sqrt{3} - 2\sqrt2\right)^2 = 3 + 4\times2 - 2\times\sqrt3\times2\times\sqrt2 = 11 - 4\sqrt6$
		\item $\left(\sqrt5x + 2 \right)^2 = 5x^2 + 4 + 4\sqrt5x$
		\item $\left(\sqrt5x - \sqrt7\right)^2 = 5x^2 + 7 - 2\sqrt{35}x$
	\end{enumerate}
}

\exe{}{
	Dans un même repère de domaine $[0 ; 2]$, grapher les courbes $y=x^2, y=x,$ et $y=\sqrt{x}$.
}{exe:graph-sqrt}{
	\begin{center}
	\begin{tikzpicture}[>=stealth, scale=1]
		\begin{axis}[xmin = -0.1, xmax=2.1, ymin=-.1, ymax=2.1, axis x line=middle, axis y line=middle, axis line style=->, grid=both, x=4cm, y=4cm]
			\addplot[no marks,BLUE_E, -, very thick] expression[domain=0:sqrt(2), samples=300]{x^2} node[above right]{$y=x^2$};
			\addplot[no marks,GREEN_E, -, very thick] expression[domain=0:2, samples=300]{x} node[right]{$y=x$};
			\addplot[no marks,RED_E, -, very thick] expression[domain=0:2, samples=300]{sqrt(x)} node[right]{$y=\sqrt{x}$};
		\end{axis}
	\end{tikzpicture}
	\end{center}
}

% ça sert pas à grand chose non plus damned
%\exe{}{
%	Calculer les fractions suivantes.
%	\begin{multicols}{4}
%	\begin{enumerate}
%		\item $\dfrac{\sqrt{64}}{\sqrt{4}}$
%		\item $\sqrt{\dfrac{64}4}$
%		\item $\dfrac{\sqrt{0,81}}{\sqrt{0,09}}$
%		\item $\sqrt{\dfrac{0,81}{0,09}}$
%	\end{enumerate}
%	\end{multicols}
%}{exe:sqrt6}{
%	TODO
%}

\subsection*{Exercices supplémentaires}

\exe{, difficulty=1}{
	Montrer que si $k$ est une fonction quelconque sur $\R$, alors la fonction $l(x) = \frac{k(x) + k(-x)}2$ est paire.
	Donner $l(x)$ à partir de la fonction $k(x) = 4x^4 - 2x^3 + 2x^2 -x + 2$.
}{exe:symétrisation}{
	Quitte à doubler $l(x)$, ce qui ne change pas la parité, on étudie $k(x) + k(-x)$.
		\[ l(-x) = k(-x) + k\bigl(-(-x)\bigr) = k(-x) + k(x) = k(x) + k(-x) = l(x) \]
	pour tout $x\in\R$, donc $l$ est paire.
	
	En prenant $k(x) = 4x^4 - 2x^3 + 2x^2 -x + 2$, on trouve $k(-x) = 4x^4 + 2x^3 + 2x^2 + x + 2$.
	Remarquons que les puissances paires ignorent le signe alors que les puissances impaires le recrachent !
	
	Par suite, 
		\begin{align*}
			l(x) = \dfrac{k(x) + k(-x)}2 &= \dfrac{4x^4 - 2x^3 + 2x^2 -x + 2 + 4x^4 + 2x^3 + 2x^2 + x + 2}2 \\
								&= \dfrac{8x^4 + 4x^2 + 4}2 \\
								&= 4x^4 + 2x^2 + 2
		\end{align*}
	Remarquons que $l$ a pris les puissances paires de $k$ et a jeté les puissances impaires.
	Il s'avère que la fonction $\frac{k(x) - k(-x)}2$, elle, fait l'inverse.
	
	Plus généralement, toute fonction est somme d'une fonction paire et d'une fonction impaire ($f$ est impaire si $f(-x) = -f(x)$).
	À démontrer !
}


\exe{, difficulty=2}{
	Supposons que le polynôme du second degré $p(x) = ax^2 + bx + c$ s'écrive de la forme $p(x) = a'(x-\alpha)^2 - \beta$.
	Montrer, par identification des coefficients en $x^2, x, 1$ que $a'=a$, puis que $\alpha = \frac{-b}{2a}$, et enfin que $\beta = \frac{b^2 - 4ac}{4a}$.
	
	Écrire $p(x) = 3x^2 - 3x - 1$ sous la forme $a(x-\alpha)^2 - \beta$ puis déduire le minimum de $p$ ainsi que l'antécédent le réalisant.
	
	\emph{On appelle la quantité $b^2 - 4ac$ le \emph{déterminant} du polynôme et on le note $\Delta$ (lu « delta »).}
}{exe:déterminant-2nddeg}{
	À rendre pour possibilité de points bonus à la prochaine évaluation.
	Ni correction ni points ne seront attribués à un travail suspect ou ne démontrant pas une bonne compréhension de l'exercice.
	
	Le lemme suivant est (sans doute) nécessaire. Il peut être admis ou démontré.
	
	\begin{lemma}
		Soient $a, b, c, a', b', c'$ six réels. Si l'identité
			\[ ax^2 + bx + c = a'x^2 + b'x + c' \]
		tient pour tout $x\in\R$, alors $a=a', b=b',$ et $c=c'$.
	\end{lemma}
	
	Pour démontrer le lemme, vous pouvez d'abord démontrer le résultat suivant (qui est en fait équivalent).
	
	\begin{lemma}
		Soient $a, b, c$ trois réels. Si l'identité
			\[ ax^2 + bx + c = 0 \]
		tient pour tout $x\in\R$, alors $a=b=c=0$.
		
		On dit alors que $1, x,$ et $x^2$ sont \emph{linéairement indépendants}.
	\end{lemma}
	
	
%	Forçons l'égalité $p(x) = a'(x-\alpha)^2 - \beta$ et voyons ce que chaque coefficient nous impose.
%		\begin{align*}
%			ax^2 + bx + c &= a'(x-\alpha)^2 - \beta \\
%							&= a'(x^2 + \alpha^2 - 2\alpha x) - \beta \\
%							&= a'x^2 + (-2a'\alpha)x + (a'\alpha^2 - \beta)
%		\end{align*}
%	 Les coefficients en $x^2$ doivent être égaux, donc $a' = a$ est forcé.
%	 L'équation devient donc
%	 	\[ ax^2 + bx + c = ax^2 + (-2a\alpha)x + (a\alpha^2-\beta). \]
%	Les coefficients en $x$ doivent être égaux, donc $b = -2a\alpha$ est forcé.
%	Il suit donc que
%		\[ \alpha = \dfrac{-b}{2a}. \]
%	L'équation devient donc
%		\[ ax^2 + bx + c = ax^2 + bx + \left(\dfrac{b^2}{4a} - \beta\right). \]
%	Le coefficients constants (en 1) doivent être égaux, donc $c = \dfrac{b^2}{4a} - \beta$.
%	Il suit donc que
%		\[ \beta = \dfrac{b^2}{4a} - c = \dfrac{b^2 - 4ac}{4a}, \]
%	comme requis.
%	
%	Dans le cas de $p(x) = 3x^2 - 3x - 1$, on spécialise les formules pour $\alpha$ et $\beta$ lorsque $a=3, b=-3,$ et $c=-1$ pour trouver
%		\begin{align*}
%			\alpha = \dfrac{-(-3)}{2\times3} = \dfrac12 && \et && \beta = \dfrac{(-3)^2 - 4\times3\times(-1)}{4\times3} = \dfrac{21}{12} = \dfrac74.
%		\end{align*}
%	Par conséquent,
%		\[ 3x^2  - 3x -1 = 3\left(x - \dfrac12\right)^2 - \dfrac74, \]
%	égalité qu'on peut vérifier en redéveloppant le membre de droite pour se rassurer :
%		\begin{align*}
%			3\left(x - \dfrac12\right)^2 - \dfrac74 &= 3\left[ x^2 + \dfrac14 - x\right] - \dfrac74, \\
%												&= 3x^2 + \dfrac34 - 3x - \dfrac74, \\
%												&= 3x^2 - 3x - 1 = p(x).
%		\end{align*}
%	En conclusion, $p$ atteint son minimum $-\frac74$ en $x^\star = \frac12$.
}


\exe{}{
	Montrer que la notation $x = \sqrt{-1}$ n'a pas de sens pour $x\in\R$ réel.
}{exe:sqrt-undef}{
	À supposer qu'il existe un tel $x\in\R$, alors il vérifierait $x^2 = -1$.
	Ce n'est pas possible car la fonction carré est toujours positive. \Large\Lightning
}
% clear
%\exe{}{
%	Montrer que l'égalité $x^2 = y^2$ n'implique pas $x=y$ pour $x, y\in\R$ généraux.
%}{exe:sqrt-non-bij}{
%	TODO
%}

\exe{,difficulty=1}{
	Donner la condition la plus faible possible sur les nombres $x, y \in \R$ pour que l'égalité $x^2 = y^2$ implique l'égalité $x=y$.
}{exe:sqrt-bij-under-condition}{
	Comme $x^2 = y^2 \iff x^2 - y^2 = 0 \iff (x-y)(x+y) = 0$ est équivalente à ce que $x=y$ ou $x=-y$, $x$ et $y$ diffèrent au plus par leur signe.

	Montrons donc que la condition « $x$ et $y$ ont le même signe » est équivalente à l'implication $x^2 = y^2 \implies x=y$.
	Mathématiquement, il faut montrer que
		\[ \bigl(\text{$x$ et $y$ ont le même signe}\bigr) \iff \Bigl( x^2 = y^2 \implies x=y \Bigr). \]
	Il s'agit donc de montrer que
		\begin{enumerate}[label=\roman*)]
			\item si $x$ et $y$ ont le même signe, alors $x^2 = y^2 \implies x=y$ ; et
			\item si $x$ et $y$ sont de signes différents, alors $x^2 = y^2 \centernot\implies x=y$.
		\end{enumerate}
	Le premier point découle du fait que $x=y$ ou $x=-y$ : seule la première égalité peut tenir si $x$ et $y$ ont le même signe.
	
	Pour le deuxième point, une non implication se montre le plus simplement par un contre-exemple.
	Ainsi $x = 2$ et $y=-2$ sont bien de signes différents, et $x^2 = y^2$ tient, mais $x=y$ ne tient pas.
	Il suit alors que $x^2 = y^2 \centernot\implies x=y$.
}

%%%%%%%%%%%

\newpage
\fancyhead[C]{\textbf{Solutions}}
\shipoutAnswer

\end{document}

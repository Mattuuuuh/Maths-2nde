				% ENABLE or DISABLE font change
				% use XeLaTeX if true
\newif\ifdys
				\dystrue
				\dysfalse

\newif\ifsolutions
				\solutionstrue
				\solutionsfalse

% DYSLEXIA SWITCH
\newif\ifdys
		
				% ENABLE or DISABLE font change
				% use XeLaTeX if true
				\dystrue
				\dysfalse


\ifdys

\documentclass[a4paper, 14pt]{extarticle}
\usepackage{amsmath,amsfonts,amsthm,amssymb,mathtools}

\tracinglostchars=3 % Report an error if a font does not have a symbol.
\usepackage{fontspec}
\usepackage{unicode-math}
\defaultfontfeatures{ Ligatures=TeX,
                      Scale=MatchUppercase }

\setmainfont{OpenDyslexic}[Scale=1.0]
\setmathfont{Fira Math} % Or maybe try KPMath-Sans?
\setmathfont{OpenDyslexic Italic}[range=it/{Latin,latin}]
\setmathfont{OpenDyslexic}[range=up/{Latin,latin,num}]

\else

\documentclass[a4paper, 12pt]{extarticle}

\usepackage[utf8x]{inputenc}
%fonts
\usepackage{amsmath,amsfonts,amsthm,amssymb,mathtools}
% comment below to default to computer modern
\usepackage{libertinus,libertinust1math}

\fi


\usepackage[french]{babel}
\usepackage[
a4paper,
margin=2cm,
nomarginpar,% We don't want any margin paragraphs
]{geometry}
\usepackage{icomma}

\usepackage{fancyhdr}
\usepackage{array}
\usepackage{hyperref}

\usepackage{multicol, enumerate}
\newcolumntype{P}[1]{>{\centering\arraybackslash}p{#1}}


\usepackage{stackengine}
\newcommand\xrowht[2][0]{\addstackgap[.5\dimexpr#2\relax]{\vphantom{#1}}}

% theorems

\theoremstyle{plain}
\newtheorem{theorem}{Th\'eor\`eme}
\newtheorem*{sol}{Solution}
\theoremstyle{definition}
\newtheorem{ex}{Exercice}
\newtheorem*{rpl}{Rappel}
\newtheorem{enigme}{Énigme}

% corps
\usepackage{calrsfs}
\newcommand{\C}{\mathcal{C}}
\newcommand{\R}{\mathbb{R}}
\newcommand{\Rnn}{\mathbb{R}^{2n}}
\newcommand{\Z}{\mathbb{Z}}
\newcommand{\N}{\mathbb{N}}
\newcommand{\Q}{\mathbb{Q}}

% variance
\newcommand{\Var}[1]{\text{Var}(#1)}

% domain
\newcommand{\D}{\mathcal{D}}


% date
\usepackage{advdate}
\AdvanceDate[0]


% plots
\usepackage{pgfplots}

% table line break
\usepackage{makecell}
%tablestuff
\def\arraystretch{2}
\setlength\tabcolsep{15pt}

%subfigures
\usepackage{subcaption}

\definecolor{myg}{RGB}{56, 140, 70}
\definecolor{myb}{RGB}{45, 111, 177}
\definecolor{myr}{RGB}{199, 68, 64}

% fake sections with no title to move around the merged pdf
\newcommand{\fakesection}[1]{%
  \par\refstepcounter{section}% Increase section counter
  \sectionmark{#1}% Add section mark (header)
  \addcontentsline{toc}{section}{\protect\numberline{\thesection}#1}% Add section to ToC
  % Add more content here, if needed.
}


% SOLUTION SWITCH
\newif\ifsolutions
				\solutionstrue
				%\solutionsfalse

\ifsolutions
	\newcommand{\exe}[2]{
		\begin{ex} #1  \end{ex}
		\begin{sol} #2 \end{sol}
	}
\else
	\newcommand{\exe}[2]{
		\begin{ex} #1  \end{ex}
	}
	
\fi


% tableaux var, signe
\usepackage{tkz-tab}


%pinfty minfty
\newcommand{\pinfty}{{+}\infty}
\newcommand{\minfty}{{-}\infty}

\begin{document}


\AdvanceDate[1]

\begin{document}
\pagestyle{fancy}
\fancyhead[L]{Seconde 13}
\fancyhead[C]{\textbf{Algorithmique 1 : affectation, condition, boucle \ifsolutions \, -- Solutions  \fi}}
\fancyhead[R]{\today}


\begin{figure}[h]
\begin{subfigure}{.5\textwidth}
\vspace{15pt}
\begin{mintedbox}{python}
a = 5
b = 15
m = (a+b)/2
print(a, b, m)

a = b
b = a
print(a, b, m)
\end{mintedbox}
\end{subfigure}
\hfill
\begin{subfigure}{.45\textwidth}
	\begin{tabular}{|c|c|c|c|}\hline
		Ligne & \texttt{a} & \texttt{b} & \texttt{m} \\ \hline
		$1$ &&& \\ \hline
		$2$ &&& \\ \hline
		$3$ &&& \\ \hline
		$4$ &&& \\ \hline
		$5$ &&& \\ \hline
		$6$ &&& \\ \hline
		$7$ &&& \\ \hline
		$8$ &&& \\ \hline
	\end{tabular}
\end{subfigure}

\begin{subfigure}{.5\textwidth}
\vspace{15pt}
\begin{mintedbox}{python}
a = -4
b = 12
print(a, b)

temp = a
a = b
b = temp
print(a, b, temp)
\end{mintedbox}
\end{subfigure}
\hfill
\begin{subfigure}{.45\textwidth}
	\begin{tabular}{|c|c|c|c|}\hline
		Ligne & \texttt{a} & \texttt{b} & \texttt{temp} \\ \hline
		$1$ &&& \\ \hline
		$2$ &&& \\ \hline
		$3$ &&& \\ \hline
		$4$ &&& \\ \hline
		$5$ &&& \\ \hline
		$6$ &&& \\ \hline
		$7$ &&& \\ \hline
		$8$ &&& \\ \hline
	\end{tabular}
\end{subfigure}

\begin{subfigure}{.5\textwidth}
\vspace{15pt}
\begin{mintedbox}{python}
a = 13
b = a+3
print(a, b)

a = a - 4
b = 1.5*b - 10
b = b+a
print(a, b)
\end{mintedbox}
\end{subfigure}
\hfill
\begin{subfigure}{.45\textwidth}
	\begin{tabular}{|c|c|c|}\hline
		Ligne & \texttt{a} & \texttt{b} \\ \hline
		$1$ && \\ \hline
		$2$ && \\ \hline
		$3$ && \\ \hline
		$4$ && \\ \hline
		$5$ && \\ \hline
		$6$ && \\ \hline
		$7$ && \\ \hline
		$8$ && \\ \hline
	\end{tabular}
\end{subfigure}

\caption{Trois programmes d'affectation.}
\label{fig:1}
\end{figure}

\exe{
	On considère les programmes de la figure \ref{fig:1}.
	\begin{enumerate}
		\item Remplir les tableaux avec la valeur de chaque variable après execution de chaque ligne.
		\item Qu'impriment les trois programmes ?
	\end{enumerate}
}{}

\begin{figure}[h]
\begin{subfigure}{.5\textwidth}
\vspace{15pt}
\begin{mintedbox}{python}
a = 13
b = 15
print(a < b)
C = (a != b)
print(C, not C)

print(C and True)
print(not (C and C))
print((not C) and C)

print(a < b or a > b)
print((not C) or C)
print(not (C or C))
\end{mintedbox}
\end{subfigure}
\hfill
\begin{subfigure}{.45\textwidth}
\begin{mintedbox}{python}
a = -5
b = -2
D = (a < b)
E = (a**2 < b**2)
print(D, E)

if D and not E:
	print('Décroissant')
elif D and E:
	print('Croissant')
\end{mintedbox}
\end{subfigure}

\caption{Deux programmes avec conditions.}
\label{fig:3}
\end{figure}

\exe{
	On considère les deux programmes de la figure \ref{fig:3} au verso.
	Qu'impriment-ils ?
}{}

%\newpage


\begin{figure}[h]
\begin{subfigure}{.5\textwidth}
\vspace{15pt}
\begin{mintedbox}{python}
a = 4.5
b = -2
if a < b:
	print(a)
else:
	print(b)
a=a-4
print(a+b)
\end{mintedbox}
\end{subfigure}
\hfill
\begin{subfigure}{.4\textwidth}
	\begin{tabular}{|c|c|c|c|}\hline
		Ligne & \texttt{a} & \texttt{b} \\ \hline
		$1$ && \\ \hline
		$2$ && \\ \hline
		$3$ && \\ \hline
		$4$ && \\ \hline
		$5$ && \\ \hline
		$6$ && \\ \hline
		$7$ && \\ \hline
		$8$ && \\ \hline
	\end{tabular}
\end{subfigure}

\begin{subfigure}{.5\textwidth}
\vspace{15pt}
\begin{mintedbox}{python}
k=0
a = 2.5
b = -1
for k in range(1, 6):
	b = b + 1
	if b > a:
		print(b)
\end{mintedbox}
\end{subfigure}
\hfill
\begin{subfigure}{.45\textwidth}
	\begin{tabular}{|c|c|c|c|}\hline
		Boucle indexée par \texttt{k} & \texttt{a} & \texttt{b} \\ \hline
		$0$ && \\ \hline
		$1$ && \\ \hline
		$2$ && \\ \hline
		$3$ && \\ \hline
		$4$ && \\ \hline
		$5$ && \\ \hline
	\end{tabular}
\end{subfigure}

\begin{subfigure}{.5\textwidth}
\vspace{15pt}
\begin{mintedbox}{python}
a = 2.5
b = -1
while (b < a or b < 5):
	b = b + 1
	if b > a:
		print(b)
\end{mintedbox}
\end{subfigure}
\hfill
\begin{subfigure}{.45\textwidth}
	\begin{tabular}{|c|c|c|c|}\hline
		Itération de la boucle & \texttt{a} & \texttt{b} \\ \hline
		$0$ && \\ \hline
		$1$ && \\ \hline
		$2$ && \\ \hline
		$3$ && \\ \hline
		$4$ && \\ \hline
		$5$ && \\ \hline
	\end{tabular}
\end{subfigure}

\caption{Trois programmes avec conditions et boucles.}
\label{fig:2}
\end{figure}


\exe{
	On considère les programmes de la figure \ref{fig:2} au verso.
	\begin{enumerate}
		%\item Remplir les tableaux avec la valeur de chaque variable après execution de chaque ligne.
		\item Remplir les tableaux avec la valeur de chaque variable.
		\item Qu'impriment les trois programmes ?
	\end{enumerate}
}{}

\end{document}

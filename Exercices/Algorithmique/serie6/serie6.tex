				% ENABLE or DISABLE font change
				% use XeLaTeX if true
\newif\ifdys
				\dystrue
				\dysfalse

\newif\ifsolutions
				\solutionstrue
				\solutionsfalse

%!TEX encoding = UTF8
%!TEX root =notes.tex


%%%%%%%%%%%%%%%%%%%%%%%%%%%%%%%%%
% PACKAGE IMPORTS
%%%%%%%%%%%%%%%%%%%%%%%%%%%%%%%%%


\usepackage[french]{babel}

\usepackage[tmargin=2cm,rmargin=1in,lmargin=1in,margin=0.85in,bmargin=2cm,footskip=.2in]{geometry}
\usepackage{amsmath,amsfonts,amsthm,amssymb,mathtools}
\usepackage[varbb]{newpxmath}
\usepackage{xfrac}
\usepackage[makeroom]{cancel}
\usepackage{mathtools}
\usepackage{bookmark}
\usepackage{enumitem}
\usepackage{hyperref,theoremref}
\hypersetup{
	pdftitle={Assignment},
	colorlinks=true, linkcolor=doc!90,
	bookmarksnumbered=true,
	bookmarksopen=true
}
\usepackage[most,many,breakable]{tcolorbox}
\usepackage{xcolor}
\usepackage{varwidth}
\usepackage{varwidth}
\usepackage{etoolbox}
%\usepackage{authblk}
\usepackage{nameref}
\usepackage{multicol,array}
\usepackage{tikz-cd}
\usepackage[ruled,vlined,linesnumbered]{algorithm2e}
\usepackage{comment} % enables the use of multi-line comments (\ifx \fi) 
\usepackage{import}
\usepackage{xifthen}
\usepackage{pdfpages}
\usepackage{transparent}


\newcommand\mycommfont[1]{\footnotesize\ttfamily\textcolor{blue}{#1}}
\SetCommentSty{mycommfont}
\newcommand{\incfig}[1]{%
    \def\svgwidth{\columnwidth}
    \import{./figures/}{#1.pdf_tex}
}

\usepackage{tikzsymbols}
%\renewcommand\qedsymbol{$\Laughey$}


%\usepackage{import}
%\usepackage{xifthen}
%\usepackage{pdfpages}
%\usepackage{transparent}


%%%%%%%%%%%%%%%%%%%%%%%%%%%%%%
% SELF MADE COLORS
%%%%%%%%%%%%%%%%%%%%%%%%%%%%%%



\definecolor{myg}{RGB}{56, 140, 70}
\definecolor{myb}{RGB}{45, 111, 177}
\definecolor{myr}{RGB}{199, 68, 64}
\definecolor{mytheorembg}{HTML}{F2F2F9}
\definecolor{mytheoremfr}{HTML}{00007B}
\definecolor{mylenmabg}{HTML}{FFFAF8}
\definecolor{mylenmafr}{HTML}{983b0f}
\definecolor{mypropbg}{HTML}{f2fbfc}
\definecolor{mypropfr}{HTML}{191971}
\definecolor{myexamplebg}{HTML}{F2FBF8}
\definecolor{myexamplefr}{HTML}{88D6D1}
\definecolor{myexampleti}{HTML}{2A7F7F}
\definecolor{mydefinitbg}{HTML}{E5E5FF}
\definecolor{mydefinitfr}{HTML}{3F3FA3}
\definecolor{notesgreen}{RGB}{0,162,0}
\definecolor{myp}{RGB}{197, 92, 212}
\definecolor{mygr}{HTML}{2C3338}
\definecolor{myred}{RGB}{127,0,0}
\definecolor{myyellow}{RGB}{169,121,69}
\definecolor{myexercisebg}{HTML}{F2FBF8}
\definecolor{myexercisefg}{HTML}{88D6D1}


%%%%%%%%%%%%%%%%%%%%%%%%%%%%
% TCOLORBOX SETUPS
%%%%%%%%%%%%%%%%%%%%%%%%%%%%

\setlength{\parindent}{1cm}
%================================
% THEOREM BOX
%================================

\tcbuselibrary{theorems,skins,hooks}
\newtcbtheorem[number within=chapter]{Theorem}{Théorème}
{%
	enhanced,
	breakable,
	colback = mytheorembg,
	frame hidden,
	boxrule = 0sp,
	borderline west = {2pt}{0pt}{mytheoremfr},
	sharp corners,
	detach title,
	before upper = \tcbtitle\par\smallskip,
	coltitle = mytheoremfr,
	fonttitle = \bfseries\sffamily,
	description font = \mdseries,
	separator sign none,
	segmentation style={solid, mytheoremfr},
}
{th}


\tcbuselibrary{theorems,skins,hooks}
\newtcolorbox{Theoremcon}
{%
	enhanced
	,breakable
	,colback = mytheorembg
	,frame hidden
	,boxrule = 0sp
	,borderline west = {2pt}{0pt}{mytheoremfr}
	,sharp corners
	,description font = \mdseries
	,separator sign none
}

%================================
% Corollery
%================================
\tcbuselibrary{theorems,skins,hooks}
\newtcbtheorem[use counter=tcb@cnt@Theorem]{Corollary}{Corollaire}
{%
	enhanced
	,breakable
	,colback = myp!10
	,frame hidden
	,boxrule = 0sp
	,borderline west = {2pt}{0pt}{myp!85!black}
	,sharp corners
	,detach title
	,before upper = \tcbtitle\par\smallskip
	,coltitle = myp!85!black
	,fonttitle = \bfseries\sffamily
	,description font = \mdseries
	,separator sign none
	,segmentation style={solid, myp!85!black}
}
{th}

%================================
% LENMA
%================================

\tcbuselibrary{theorems,skins,hooks}
\newtcbtheorem[use counter=tcb@cnt@Theorem]{Lemma}{Lemme}
{%
	enhanced,
	breakable,
	colback = mylenmabg,
	frame hidden,
	boxrule = 0sp,
	borderline west = {2pt}{0pt}{mylenmafr},
	sharp corners,
	detach title,
	before upper = \tcbtitle\par\smallskip,
	coltitle = mylenmafr,
	fonttitle = \bfseries\sffamily,
	description font = \mdseries,
	separator sign none,
	segmentation style={solid, mylenmafr},
}
{th}


%================================
% PROPOSITION
%================================

\tcbuselibrary{theorems,skins,hooks}
\newtcbtheorem[use counter=tcb@cnt@Theorem]{Prop}{Proposition}
{%
	enhanced,
	breakable,
	colback = mypropbg,
	frame hidden,
	boxrule = 0sp,
	borderline west = {2pt}{0pt}{mypropfr},
	sharp corners,
	detach title,
	before upper = \tcbtitle\par\smallskip,
	coltitle = mypropfr,
	fonttitle = \bfseries\sffamily,
	description font = \mdseries,
	separator sign none,
	segmentation style={solid, mypropfr},
}
{th}


%================================
% CLAIM
%================================

\tcbuselibrary{theorems,skins,hooks}
\newtcbtheorem[use counter=tcb@cnt@Theorem]{claim}{Claim}
{%
	enhanced
	,breakable
	,colback = myg!10
	,frame hidden
	,boxrule = 0sp
	,borderline west = {2pt}{0pt}{myg}
	,sharp corners
	,detach title
	,before upper = \tcbtitle\par\smallskip
	,coltitle = myg!85!black
	,fonttitle = \bfseries\sffamily
	,description font = \mdseries
	,separator sign none
	,segmentation style={solid, myg!85!black}
}
{th}



%================================
% Exercise
%================================

\tcbuselibrary{theorems,skins,hooks}
\newtcbtheorem[use counter=tcb@cnt@Theorem]{Exercise}{Exercice}
{%
	enhanced,
	breakable,
	colback = myexercisebg,
	frame hidden,
	boxrule = 0sp,
	borderline west = {2pt}{0pt}{myexercisefg},
	sharp corners,
	detach title,
	before upper = \tcbtitle\par\smallskip,
	coltitle = myexercisefg,
	fonttitle = \bfseries\sffamily,
	description font = \mdseries,
	separator sign none,
	segmentation style={solid, myexercisefg},
}
{th}

%================================
% EXAMPLE BOX
%================================

\newtcbtheorem[use counter=tcb@cnt@Theorem]{Example}{Exemple}
{%
	colback = myexamplebg
	,breakable
	,colframe = myexamplefr
	,coltitle = myexampleti
	,boxrule = 1pt
	,sharp corners
	,detach title
	,before upper=\tcbtitle\par\smallskip
	,fonttitle = \bfseries
	,description font = \mdseries
	,separator sign none
	,description delimiters parenthesis
}
{ex}

%================================
% DEFINITION BOX
%================================

\newtcbtheorem[use counter=tcb@cnt@Theorem]{Definition}{Définition}{enhanced,
	before skip=2mm,after skip=2mm, colback=red!5,colframe=red!80!black,boxrule=0.5mm,
	attach boxed title to top left={xshift=1cm,yshift*=1mm-\tcboxedtitleheight}, varwidth boxed title*=-3cm,
	boxed title style={frame code={
					\path[fill=tcbcolback]
					([yshift=-1mm,xshift=-1mm]frame.north west)
					arc[start angle=0,end angle=180,radius=1mm]
					([yshift=-1mm,xshift=1mm]frame.north east)
					arc[start angle=180,end angle=0,radius=1mm];
					\path[left color=tcbcolback!60!black,right color=tcbcolback!60!black,
						middle color=tcbcolback!80!black]
					([xshift=-2mm]frame.north west) -- ([xshift=2mm]frame.north east)
					[rounded corners=1mm]-- ([xshift=1mm,yshift=-1mm]frame.north east)
					-- (frame.south east) -- (frame.south west)
					-- ([xshift=-1mm,yshift=-1mm]frame.north west)
					[sharp corners]-- cycle;
				},interior engine=empty,
		},
	fonttitle=\bfseries,
	title={#2},#1}{def}

%================================
% Solution BOX
%================================

\makeatletter
\newtcbtheorem[use counter=tcb@cnt@Theorem]{question}{Question}{enhanced,
	breakable,
	colback=white,
	colframe=myb!80!black,
	attach boxed title to top left={yshift*=-\tcboxedtitleheight},
	fonttitle=\bfseries,
	title={#2},
	boxed title size=title,
	boxed title style={%
			sharp corners,
			rounded corners=northwest,
			colback=tcbcolframe,
			boxrule=0pt,
		},
	underlay boxed title={%
			\path[fill=tcbcolframe] (title.south west)--(title.south east)
			to[out=0, in=180] ([xshift=5mm]title.east)--
			(title.center-|frame.east)
			[rounded corners=\kvtcb@arc] |-
			(frame.north) -| cycle;
		},
	#1
}{def}
\makeatother

%================================
% SOLUTION BOX
%================================

\makeatletter
\newtcolorbox{solution}{enhanced,
	breakable,
	colback=white,
	colframe=myg!80!black,
	attach boxed title to top left={yshift*=-\tcboxedtitleheight},
	title=Solution,
	boxed title size=title,
	boxed title style={%
			sharp corners,
			rounded corners=northwest,
			colback=tcbcolframe,
			boxrule=0pt,
		},
	underlay boxed title={%
			\path[fill=tcbcolframe] (title.south west)--(title.south east)
			to[out=0, in=180] ([xshift=5mm]title.east)--
			(title.center-|frame.east)
			[rounded corners=\kvtcb@arc] |-
			(frame.north) -| cycle;
		},
}
\makeatother

%================================
% Question BOX
%================================

\makeatletter
\newtcbtheorem[use counter=tcb@cnt@Theorem]{qstion}{Question}{enhanced,
	breakable,
	colback=white,
	colframe=mygr,
	attach boxed title to top left={yshift*=-\tcboxedtitleheight},
	fonttitle=\bfseries,
	title={#2},
	boxed title size=title,
	boxed title style={%
			sharp corners,
			rounded corners=northwest,
			colback=tcbcolframe,
			boxrule=0pt,
		},
	underlay boxed title={%
			\path[fill=tcbcolframe] (title.south west)--(title.south east)
			to[out=0, in=180] ([xshift=5mm]title.east)--
			(title.center-|frame.east)
			[rounded corners=\kvtcb@arc] |-
			(frame.north) -| cycle;
		},
	#1
}{def}
\makeatother

\newtcbtheorem[number within=chapter]{wconc}{Wrong Concept}{
	breakable,
	enhanced,
	colback=white,
	colframe=myr,
	arc=0pt,
	outer arc=0pt,
	fonttitle=\bfseries\sffamily\large,
	colbacktitle=myr,
	attach boxed title to top left={},
	boxed title style={
			enhanced,
			skin=enhancedfirst jigsaw,
			arc=3pt,
			bottom=0pt,
			interior style={fill=myr}
		},
	#1
}{def}



%================================
% NOTE BOX
%================================

\usetikzlibrary{arrows,calc,shadows.blur}
\tcbuselibrary{skins}
\newtcolorbox{note}[1][]{%
	enhanced jigsaw,
	colback=gray!20!white,%
	colframe=gray!80!black,
	size=small,
	boxrule=1pt,
	title=\colorbox{white!100}{\textbf{ Remarque }},
	halign title=flush center,
	coltitle=black,
	breakable,
	drop shadow=black!50!white,
	attach boxed title to top left={xshift=1cm,yshift=-\tcboxedtitleheight/2,yshifttext=-\tcboxedtitleheight/2},
	minipage boxed title=2.6cm,
	boxed title style={%
			colback=white,
			size=fbox,
			boxrule=1pt,
			boxsep=2pt,
			underlay={%
					\coordinate (dotA) at ($(interior.west) + (-0.5pt,0)$);
					\coordinate (dotB) at ($(interior.east) + (0.5pt,0)$);
					\begin{scope}
						\clip (interior.north west) rectangle ([xshift=3ex]interior.east);
						\filldraw [white, blur shadow={shadow opacity=60, shadow yshift=-.75ex}, rounded corners=2pt] (interior.north west) rectangle (interior.south east);
					\end{scope}
					\begin{scope}[gray!80!black]
						\fill (dotA) circle (2pt);
						\fill (dotB) circle (2pt);
					\end{scope}
				},
		},
	#1,
}

%================================
% STRATÉGIE BOX
%================================

\usetikzlibrary{arrows,calc,shadows.blur}
\tcbuselibrary{skins}
\newtcolorbox{strategy}[1][]{%
	enhanced jigsaw,
	colback=myb!20!white,%
	colframe=gray!80!black,
	size=small,
	boxrule=1pt,
	title=\colorbox{white!100}{\textbf{ Stratégie }},
	halign title=flush center,
	coltitle=black,
	breakable,
	drop shadow=black!50!white,
	attach boxed title to top left={xshift=1cm,yshift=-\tcboxedtitleheight/2,yshifttext=-\tcboxedtitleheight/2},
	minipage boxed title=2.5cm,
	boxed title style={%
			colback=white,
			size=fbox,
			boxrule=1pt,
			boxsep=2pt,
			underlay={%
					\coordinate (dotA) at ($(interior.west) + (-0.5pt,0)$);
					\coordinate (dotB) at ($(interior.east) + (0.5pt,0)$);
					\begin{scope}
						\clip (interior.north west) rectangle ([xshift=3ex]interior.east);
						\filldraw [white, blur shadow={shadow opacity=60, shadow yshift=-.75ex}, rounded corners=2pt] (interior.north west) rectangle (interior.south east);
					\end{scope}
					\begin{scope}[gray!80!black]
						\fill (dotA) circle (2pt);
						\fill (dotB) circle (2pt);
					\end{scope}
				},
		},
	#1,
}

%================================
% MÉTHODE BOX
%================================

\usetikzlibrary{arrows,calc,shadows.blur}
\tcbuselibrary{skins}
\newtcolorbox{methode}[1][]{%
	enhanced jigsaw,
	colback=white,%
	colframe=gray!80!black,
	size=small,
	boxrule=1pt,
	title=\textbf{Méthode},
	halign title=flush center,
	coltitle=black,
	breakable,
	drop shadow=black!50!white,
	attach boxed title to top left={xshift=1cm,yshift=-\tcboxedtitleheight/2,yshifttext=-\tcboxedtitleheight/2},
	minipage boxed title=2.5cm,
	boxed title style={%
			colback=white,
			size=fbox,
			boxrule=1pt,
			boxsep=2pt,
			underlay={%
					\coordinate (dotA) at ($(interior.west) + (-0.5pt,0)$);
					\coordinate (dotB) at ($(interior.east) + (0.5pt,0)$);
					\begin{scope}
						\clip (interior.north west) rectangle ([xshift=3ex]interior.east);
						\filldraw [white, blur shadow={shadow opacity=60, shadow yshift=-.75ex}, rounded corners=2pt] (interior.north west) rectangle (interior.south east);
					\end{scope}
					\begin{scope}[gray!80!black]
						\fill (dotA) circle (2pt);
						\fill (dotB) circle (2pt);
					\end{scope}
				},
		},
	#1,
}

%%%%%%%%%%%%%%%%%%%%%%%%%%%%%%%%%%%%%%%%%%%
% TABLE OF CONTENTS
%%%%%%%%%%%%%%%%%%%%%%%%%%%%%%%%%%%%%%%%%%%

\usepackage{tikz}

\definecolor{doc}{RGB}{0,60,110}
\usepackage{titletoc}
\contentsmargin{0cm}
\titlecontents{chapter}[3.7pc]
{\addvspace{30pt}%
	\begin{tikzpicture}[remember picture, overlay]%
		\draw[fill=doc!60,draw=doc!60] (-7,-.1) rectangle (-0.2,.6);%
		\pgftext[left,x=-3.5cm,y=0.2cm]{\color{white}\Large\sc\bfseries Chapitre\ \thecontentslabel};%
	\end{tikzpicture}\color{doc!60}\large\sc\bfseries}%
{}
{}
{\;\titlerule\;\large\sc\bfseries Page \thecontentspage
	\begin{tikzpicture}[remember picture, overlay]
		\draw[fill=doc!60,draw=doc!60] (2pt,0) rectangle (4,0.1pt);
	\end{tikzpicture}}%
\titlecontents{section}[3.7pc]
{\addvspace{2pt}}
{\contentslabel[\thecontentslabel]{2pc}}
{}
{\hfill\small \thecontentspage}
[]
\titlecontents*{subsection}[3.7pc]
{\addvspace{-1pt}\small}
{}
{}
{\ --- \small\thecontentspage}
[ \textbullet\ ][]

\makeatletter
\renewcommand{\tableofcontents}{%
	\chapter*{%
	  \vspace*{-20\p@}%
	  \begin{tikzpicture}[remember picture, overlay]%
		  \pgftext[right,x=15cm,y=0.2cm]{\color{doc!60}\Huge\sc\bfseries \contentsname};%
		  \draw[fill=doc!60,draw=doc!60] (13,-.75) rectangle (20,1);%
		  \clip (13,-.75) rectangle (20,1);
		  \pgftext[right,x=15cm,y=0.2cm]{\color{white}\Huge\sc\bfseries \contentsname};%
	  \end{tikzpicture}}%
	\@starttoc{toc}}
\makeatother


%%%%%%%%%%%%%%%%%%%%%%%%%%%%%%%%%%%%%%%%%%%
% MINTED FOR PYTHON ALGORITHMS
%%%%%%%%%%%%%%%%%%%%%%%%%%%%%%%%%%%%%%%%%%%

\usepackage{tcolorbox}
\tcbuselibrary{minted,breakable,xparse,skins}
\definecolor{bg}{gray}{0.95}
\DeclareTCBListing{mintedbox}{O{}m!O{}}{%
  breakable=true,
  listing engine=minted,
  listing only,
  minted language=#2,
  minted style=default,
  minted options={%
    linenos,
    gobble=0,
    breaklines=true,
    breakafter=,,
    fontsize=\small,
    numbersep=8pt,
    #1},
  boxsep=0pt,
  left skip=0pt,
  right skip=0pt,
  left=25pt,
  right=0pt,
  top=3pt,
  bottom=3pt,
  arc=5pt,
  leftrule=0pt,
  rightrule=0pt,
  bottomrule=2pt,
  toprule=2pt,
  colback=bg,
  colframe=orange!70,
  enhanced,
  overlay={%
    \begin{tcbclipinterior}
    \fill[orange!20!white] (frame.south west) rectangle ([xshift=20pt]frame.north west);
    \end{tcbclipinterior}},
  #3}
  
  
 % for braces
\usetikzlibrary{decorations.pathreplacing}


\AdvanceDate[1]

\begin{document}
\pagestyle{fancy}
\fancyhead[L]{Seconde 13}
\fancyhead[C]{\textbf{Algorithmique 6 : encadrements et notation scientifique \ifsolutions \, -- Solutions  \fi}}
\fancyhead[R]{\today}

\begin{proprietes*}{puissances de 10}
	Soit $n\in\N$ un entier naturel. Comme
		\begin{enumerate}[label=$\bullet$]
			\item multiplier par $10$ décale la virgule d'une position vers la droite ; et
			\item diviser par $10$ décale la virgule d'une position vers la gauche ;
		\end{enumerate}
	on a nécessairement
		\begin{flalign*}
			&&
			10^n = 1\underbrace{00\dots00}_{\ifsolutions \text{$n$ fois} \fi}
			&&
			\text{et}
			&&
			10^{-n} = \dfrac1{10^n} = \underbrace{0,00\dots00}_{\ifsolutions \text{$n$ fois} \fi}1.
			&&
		\end{flalign*}
\end{proprietes*}

\exe{\label{ex:1}
	Sans calculatrice, donner la valeur numérique des fractions suivantes.
	
	\begin{multicols}{3}
	\begin{enumerate}[label=\roman*)]
		\item $\dfrac1{10}$
		\item $\dfrac1{5}$
		\item $\dfrac3{10^{5}}$
		\item $\dfrac7{20}$
		\item $\dfrac{395}{50}$
		\item $\dfrac{11}{200}$
	\end{enumerate}
	\end{multicols}

}{}

\begin{definition*}{notation scientifique}
	Soit $x\in\R$ un nombre réel \emph{décimal}, c'est-à-dire que $x$ s'écrit avec un nombre fini de chiffres après la virgule.
	Alors $x$ s'écrit
		\[ x = a \times 10^n, \]
	pour un certain nombre décimal $a \in [1 ; 10[$, et $n\in\Z$ entier relatif.
\end{definition*}

\exe{
	Sans calculatrice,  écrire les nombres de l'exercice \ref{ex:1} en notation scientifique.
}{}

\exe{
	Sans calculatrice, écrire les nombres suivants en notation scientifique.
		
	\begin{multicols}{3}
	\begin{enumerate}[label=\alph*)]
		\item $201$
		\item $10$
		\item $123 400 000$
		\item $0,8$
		\item $0,000327$
		\item $0,0090001$
	\end{enumerate}
	\end{multicols}
}{}


\begin{definition*}{encadrement}
	On dit que $a$ et $b$ \emph{encadrent} le nombre réel $x\in\R$ si $a < x < b$.
	\begin{enumerate}[label=$\bullet$]
		\item $b-a$ est l'\emph{amplitude} de l'encadrement.
		\item L'encadrement est à $10^{n}$ près si son amplitude est égale à $10^{n}$ (pour $n\in\Z$ entier relatif).
	\end{enumerate}
\end{definition*}

\exe{[Vrai ou faux]
	L'encadrement 
	
	\def\arraystretch{2}
	\setlength\tabcolsep{15pt}
	\begin{tabular}{c c c}
		\hspace{10cm} & Vrai & Faux \\
		$2,6 < \sqrt{7} < 2,8$ est à $10^{-1}$ près & $\square$ & $\square$  \\
		$3,14 < \pi < 3,15$ est à $10^{-2}$ près & $\square$ & $\square$  \\
		$-4,473 < -2\sqrt5 < -4,472$ est à $10^{-3}$ près & $\square$ & $\square$  \\
		$3,3 \times 10^{-4} < 3,3931 \times 10^{-4} < 3,4 \times 10^{-4}$ est à $10^{-5}$ près & $\square$ & $\square$  \\
	\end{tabular}
}{}

\newpage

\exe{
	Encadrer les nombres suivants à l'amplitude demandée.
	\begin{multicols}{2}
	\begin{enumerate}[label=\alph*)]
		\item $3,605 \times 10^{-2}$ à $10^{-4}$ près.
		\item $9 854,698 \times 10^3$ à $10^4$ près.
		\item $-31,45$ à $10^{-1}$ près.
		\item $-0,0125$ à $10^{-4}$ près.
	\end{enumerate}
	\end{multicols}
}{}

\exe{
	On donne l'encadrement du nombre d'or $\phi =  \dfrac{1+\sqrt5}2$ suivant.
		\[ 1,61803 < \dfrac{1+\sqrt5}2 < 1,618035 \]
	\begin{enumerate}
		\item Donner l'amplitude de l'encadrement en notation scientifique.
		\item Trouver un encadrement de $\sqrt{5}$ et donner son amplitude.
	\end{enumerate}
}{}

\hrule

\exe{
	Sans calculatrice, écrire les nombres suivants en notation scientifique.
		
	\begin{multicols}{3}
	\begin{enumerate}[label=\alph*)]
		\item $3 106 000$
		\item $1,8$
		\item $0,000080021$
		\item $134,1$
		\item $60$
		\item $0,09$
	\end{enumerate}
	\end{multicols}
}{}

\exe{
	Encadrer les nombres suivants à l'amplitude demandée.
	\begin{multicols}{2}
	\begin{enumerate}[label=\alph*)]
		\item $7 345 \times 10^6$ à $10^8$ près.
		\item $36,05 \times 10^{-4}$ à $10^{-1}$ près.
		\item $-2 048,1632$ à $10^{-2}$ près.
		\item $-4,1 \times 10^{-8}$ à $10^{-8}$ près.
	\end{enumerate}
	\end{multicols}
}{}

\exe{\label{ex:7}
	Le but de l'exercice est de montrer que $\dfrac13$ n'est pas décimal (il ne s'écrit pas avec un nombre fini de chiffres après la virgule) et qu'on ne peut donc pas l'écrire en notation scientifique.
	\begin{enumerate}
		\item Rappeler la définition de $\dfrac13$ et montrer que $\dfrac13 \neq 0,3333$.
		\item Supposons $\dfrac13$ décimal. Montrer qu'il existe un $n\in\N$ et un $a\in\Z$ tel que
			\[ 10^n \cdot \dfrac13 = a. \]
		\item Montrer que $10^n$ n'est pas un multiple de $3$.
		\item Conclure par contradiction.
	\end{enumerate}
}{}

\exe{[$\star$]
	Montrer, en reprenant le raisonnement de l'exercice \ref{ex:7}, que $\dfrac1q$ n'est pas décimal si $q\in\N^*$ est un entier qui ne divise aucune puissance de $10$.
	
	En raisonnant avec l'unicité de la décomposition en facteurs premiers, montrer que c'est le cas dès qu'un nombre premier différent de 2 et de 5 divise $q$. Par exemple, $\dfrac1{14}$ n'est pas décimal.
}{}

\exe{[$\star$]
	Le but de l'exercice est de montrer que $\sqrt7$ n'est pas rationnel (et donc pas décimal non plus).	
	À cette fin, pour $a, b\in\N^*$ deux entiers naturel non nuls, on suppose par contradiction que $\sqrt7 = \dfrac{a}{b}$, fraction irréductible. En particulier, $a$ et $b$ ne peuvent pas tous les deux être multiples de 7 (la fraction serait réductible).
	
	\begin{enumerate}
		\item Montrer que $a^2 = 7b^2$ et donc que $a^2$ est multiple de $7$.
		\item En raisonnant avec l'unicité de la décomposition en facteurs premiers, montrer si $7$ ne divise pas $a$, alors $7$ ne divise pas non plus $a^2$.
		\item En déduire que $a$ est également multiple de 7. On note $a=7c$.
		\item En mettant au carré, déduire que $7c^2 = b^2$ et donc que $b^2$ est aussi multiple de 7.
		\item Conclure par contradiction.
	\end{enumerate}
}{}

% un peu compliqué pour eux
%\exe{
%	Le but de l'exercice est de montrer que $\sqrt7$ est mal approximable par les rationnels.	
%	À cette fin, pour $a, b\in\N^*$ deux entiers naturel non nuls, on étudie l'expression
%		\[ \left| \dfrac{a}{b} - \sqrt7 \right|, \]
%	distance entre $\sqrt7$ et la fraction $\dfrac{a}{b}$.
%	\begin{enumerate}
%		\item Montrer que
%			\[ \left| \dfrac{a}{b} - \sqrt7 \right| = \dfrac1{b}\left|a - b \sqrt7 \right|. \]
%		\item Montrer que
%			\[ \left|a - b \sqrt7 \right| \cdot \left|a + b \sqrt7 \right| = \left| a^2 - 7b^2 \right|. \]
%		On appelle $a + b \sqrt7$ l'\emph{expression conjuguée} de $a - b \sqrt7$.
%		\item En raisonnant avec la décomposition en facteurs premiers, montrer que $a^2 - 7b^2$ est un entier non nul, et donc que $|a^2 - 7b^2| \geq 1$.
%		\item En déduire que
%			\[  \left| \dfrac{a}{b} - \sqrt7 \right| \geq \dfrac{1}{b \left|a + b \sqrt7 \right|}. \]
%	
%	\end{enumerate}
%}{}
%
% pour échantillonnage je pense
% \hrule
%
%\exe{
%	Donner les ensembles suivants sous forme d'intervalle.
%		\begin{multicols}{2}
%		\begin{enumerate}
%			\item $ B_{0;1} = \bigl\{ x \in \R \text{ tq. } |x| \leq 1 \bigr\}$
%			\item $ B_{0;2} = \bigl\{ x \in \R \text{ tq. } |x| \leq 2 \bigr\} $
%			\item $ B_{0;10^{-2}} = \bigl\{ x \in \R \text{ tq. } |x| \leq 10^{-2} \bigr\} $
%			\item $ B_{5;1} = \bigl\{ x \in \R \text{ tq. } |x-5| \leq 1 \bigr\} $
%			\item $ B_{-3; 2} = \bigl\{ x \in \R \text{ tq. } |x+3| \leq 1 \bigr\}$
%			\item $ B_{-10; 10^{-1}} = \bigl\{ x \in \R \text{ tq. } |x+10| \leq 10^{-1} \bigr\}$
%		\end{enumerate}
%		\end{multicols}
%}{}
%
%\begin{theoreme*}{intervalle symétrique}
%	Soit $c \in \R$ un réel, et $r \in \R_+$ un réel positif ou nul. 
%	Alors l'ensemble
%		\[ B_{c, r} = \bigl\{ x \in \R \text{ tq. } |x - c| \leq r \bigr\} \]
%	est égal à l'intervalle contenant tous les nombres réels à distance inférieure à $r$ de $c$.
%		\[ B_{c,r} = [c -r ; c+r] \]
%\end{theoreme*}
%
%\exe{
%	
%}{}

\end{document}

				% ENABLE or DISABLE font change
				% use XeLaTeX if true
\newif\ifdys
				\dystrue
				\dysfalse

\newif\ifsolutions
				\solutionstrue
				\solutionsfalse

% DYSLEXIA SWITCH
\newif\ifdys
		
				% ENABLE or DISABLE font change
				% use XeLaTeX if true
				\dystrue
				\dysfalse


\ifdys

\documentclass[a4paper, 14pt]{extarticle}
\usepackage{amsmath,amsfonts,amsthm,amssymb,mathtools}

\tracinglostchars=3 % Report an error if a font does not have a symbol.
\usepackage{fontspec}
\usepackage{unicode-math}
\defaultfontfeatures{ Ligatures=TeX,
                      Scale=MatchUppercase }

\setmainfont{OpenDyslexic}[Scale=1.0]
\setmathfont{Fira Math} % Or maybe try KPMath-Sans?
\setmathfont{OpenDyslexic Italic}[range=it/{Latin,latin}]
\setmathfont{OpenDyslexic}[range=up/{Latin,latin,num}]

\else

\documentclass[a4paper, 12pt]{extarticle}

\usepackage[utf8x]{inputenc}
%fonts
\usepackage{amsmath,amsfonts,amsthm,amssymb,mathtools}
% comment below to default to computer modern
\usepackage{libertinus,libertinust1math}

\fi


\usepackage[french]{babel}
\usepackage[
a4paper,
margin=2cm,
nomarginpar,% We don't want any margin paragraphs
]{geometry}
\usepackage{icomma}

\usepackage{fancyhdr}
\usepackage{array}
\usepackage{hyperref}

\usepackage{multicol, enumerate}
\newcolumntype{P}[1]{>{\centering\arraybackslash}p{#1}}


\usepackage{stackengine}
\newcommand\xrowht[2][0]{\addstackgap[.5\dimexpr#2\relax]{\vphantom{#1}}}

% theorems

\theoremstyle{plain}
\newtheorem{theorem}{Th\'eor\`eme}
\newtheorem*{sol}{Solution}
\theoremstyle{definition}
\newtheorem{ex}{Exercice}
\newtheorem*{rpl}{Rappel}
\newtheorem{enigme}{Énigme}

% corps
\usepackage{calrsfs}
\newcommand{\C}{\mathcal{C}}
\newcommand{\R}{\mathbb{R}}
\newcommand{\Rnn}{\mathbb{R}^{2n}}
\newcommand{\Z}{\mathbb{Z}}
\newcommand{\N}{\mathbb{N}}
\newcommand{\Q}{\mathbb{Q}}

% variance
\newcommand{\Var}[1]{\text{Var}(#1)}

% domain
\newcommand{\D}{\mathcal{D}}


% date
\usepackage{advdate}
\AdvanceDate[0]


% plots
\usepackage{pgfplots}

% table line break
\usepackage{makecell}
%tablestuff
\def\arraystretch{2}
\setlength\tabcolsep{15pt}

%subfigures
\usepackage{subcaption}

\definecolor{myg}{RGB}{56, 140, 70}
\definecolor{myb}{RGB}{45, 111, 177}
\definecolor{myr}{RGB}{199, 68, 64}

% fake sections with no title to move around the merged pdf
\newcommand{\fakesection}[1]{%
  \par\refstepcounter{section}% Increase section counter
  \sectionmark{#1}% Add section mark (header)
  \addcontentsline{toc}{section}{\protect\numberline{\thesection}#1}% Add section to ToC
  % Add more content here, if needed.
}


% SOLUTION SWITCH
\newif\ifsolutions
				\solutionstrue
				%\solutionsfalse

\ifsolutions
	\newcommand{\exe}[2]{
		\begin{ex} #1  \end{ex}
		\begin{sol} #2 \end{sol}
	}
\else
	\newcommand{\exe}[2]{
		\begin{ex} #1  \end{ex}
	}
	
\fi


% tableaux var, signe
\usepackage{tkz-tab}


%pinfty minfty
\newcommand{\pinfty}{{+}\infty}
\newcommand{\minfty}{{-}\infty}

\begin{document}


\AdvanceDate[0]

\begin{document}
\pagestyle{fancy}
\fancyhead[L]{Seconde 13}
\fancyhead[C]{\textbf{Algorithmique : applications \ifsolutions \, -- Solutions  \fi}}
\fancyhead[R]{\today}


\exe{[Statistiques]
}{}

\begin{mintedbox}{python}
import math

def MOYENNE(X):
	N = len(X)
	somme = sum(X)
%	somme = 0
%	for x in X:
%		somme  = somme + x
	return somme/N

def VARIANCE(X):
	moyenne = MOYENNE(X)
	Y = [(x - moyenne)**2 for x in X]
	return MOYENNE(Y)

def ECARTTYPE(X):
	return math.sqrt(VARIANCE(X))

X = (1, 2, 3, 4, 5, 6,7, 8, 9, 10)
mu = MOYENNE(X)
sigma = ECARTTYPE(X)
print(mu, sigma)
\end{mintedbox}


\exe{[Interpolation affine]

}{}

\begin{mintedbox}{python}
def interp(A,B):
	xA, yA = A
	xB, yB = B
	assert xA != xB, 'Les points sont alignés verticalement.'
	
	a = (yB-yA)/(xB-xA)
	b = yA - a*xA
	
	return a, b
	
def verif(a, b, A, B):
	# check if A, B belong to y=ax+b
	
	return 1	
\end{mintedbox}

\exe{[Intersection de droites]
	
	Calculer l'intersection $P$ des droites $y=x+1$ et $y=x-4$. Que remarque-t-on ?
	
	Calculer l'intersection $P$ des droites $y=3x+1$ et $y=2x-4$, et vérifier que $P$ appartient bien aux droites avec la fonction \texttt{verif}.
	
	Calculer l'intersection $P$ des droites $y=7x+1$ et $y=x-4$, et vérifier que $P$ appartient bien aux droites avec la fonction \texttt{verif}.
	Expliquer le résultat obtenu.
}{}

\begin{mintedbox}{python}
def intersection(a1, b1, a2, b2):
	assert a1 != a2, 'Les droites sont parallèles.'
	
	xP =  (b2-b1)/(a1-a2)
	yP = a1*xP+b1
	
	verif(a1,b1,a2,b2,xP,yP)
	
	return xP, yP
	
def verif(a1,b1,a2,b2,xP,yP):
	assert yP == a1*xP+b1 and yP == a2*xP+b2
	return 0
\end{mintedbox}

\exe{[Arithmétique exacte]
	On cherche à faire de l'arithmétique exacte en programmation pour éviter les problèmes de virgule flottante de l'exercice précédent.
	
	Pour cela, on encode toutes les valeurs en un couple numérateur-denominateur d'entiers relatifs.
	On déclare $x=\dfrac{p}q$ ($q\neq0$) en écrivant \texttt{x=(p,q)}.
	Pour un couple $x$, on extrait le numérateur et le dénominateur en écrivant \texttt{p, q = x}.
	Compléter les fonctions de somme, produit, et d'égalité entre rationnels.
}{}


\begin{mintedbox}{python}
x=1/6
y=7*x

VF = (y == 1+x)
print(VF)

def somme(x,y):
	a, b=x
	c, d=y
	return (a*d+b*c, b*d)

def produit(x,y):
	a, b=x
	c, d=y
	return (a*c, b*d)
	
def egalite(x,y):
	a, b=x
	c, d=y
	return a*d==b*c
	
x=(1,6)
sept=(7,1)
y=produit(sept,x)
un=(1,1)

VF = egalite(y, somme(x,un))

print(VF)
\end{mintedbox}


\exe{[$\star$, intersection en arithmétique exacte]
	Créer un programme calculant le point d'intersection de droites affines non parallèles de façon exacte.
	Pour cela, on supposera que tous les paramètres sont rationnels, donnés sous forme de couple numérateur-dénominateur comme à l'exercice précédent.
}{}


\end{document}

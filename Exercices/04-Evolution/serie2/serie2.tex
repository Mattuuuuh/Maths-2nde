%!TEX encoding = UTF8
%!TEX root =notes.tex


%%%%%%%%%%%%%%%%%%%%%%%%%%%%%%%%%
% PACKAGE IMPORTS
%%%%%%%%%%%%%%%%%%%%%%%%%%%%%%%%%


\usepackage[french]{babel}

\usepackage[tmargin=2cm,rmargin=1in,lmargin=1in,margin=0.85in,bmargin=2cm,footskip=.2in]{geometry}
\usepackage{amsmath,amsfonts,amsthm,amssymb,mathtools}
\usepackage[varbb]{newpxmath}
\usepackage{xfrac}
\usepackage[makeroom]{cancel}
\usepackage{mathtools}
\usepackage{bookmark}
\usepackage{enumitem}
\usepackage{hyperref,theoremref}
\hypersetup{
	pdftitle={Assignment},
	colorlinks=true, linkcolor=doc!90,
	bookmarksnumbered=true,
	bookmarksopen=true
}
\usepackage[most,many,breakable]{tcolorbox}
\usepackage{xcolor}
\usepackage{varwidth}
\usepackage{varwidth}
\usepackage{etoolbox}
%\usepackage{authblk}
\usepackage{nameref}
\usepackage{multicol,array}
\usepackage{tikz-cd}
\usepackage[ruled,vlined,linesnumbered]{algorithm2e}
\usepackage{comment} % enables the use of multi-line comments (\ifx \fi) 
\usepackage{import}
\usepackage{xifthen}
\usepackage{pdfpages}
\usepackage{transparent}


\newcommand\mycommfont[1]{\footnotesize\ttfamily\textcolor{blue}{#1}}
\SetCommentSty{mycommfont}
\newcommand{\incfig}[1]{%
    \def\svgwidth{\columnwidth}
    \import{./figures/}{#1.pdf_tex}
}

\usepackage{tikzsymbols}
%\renewcommand\qedsymbol{$\Laughey$}


%\usepackage{import}
%\usepackage{xifthen}
%\usepackage{pdfpages}
%\usepackage{transparent}


%%%%%%%%%%%%%%%%%%%%%%%%%%%%%%
% SELF MADE COLORS
%%%%%%%%%%%%%%%%%%%%%%%%%%%%%%



\definecolor{myg}{RGB}{56, 140, 70}
\definecolor{myb}{RGB}{45, 111, 177}
\definecolor{myr}{RGB}{199, 68, 64}
\definecolor{mytheorembg}{HTML}{F2F2F9}
\definecolor{mytheoremfr}{HTML}{00007B}
\definecolor{mylenmabg}{HTML}{FFFAF8}
\definecolor{mylenmafr}{HTML}{983b0f}
\definecolor{mypropbg}{HTML}{f2fbfc}
\definecolor{mypropfr}{HTML}{191971}
\definecolor{myexamplebg}{HTML}{F2FBF8}
\definecolor{myexamplefr}{HTML}{88D6D1}
\definecolor{myexampleti}{HTML}{2A7F7F}
\definecolor{mydefinitbg}{HTML}{E5E5FF}
\definecolor{mydefinitfr}{HTML}{3F3FA3}
\definecolor{notesgreen}{RGB}{0,162,0}
\definecolor{myp}{RGB}{197, 92, 212}
\definecolor{mygr}{HTML}{2C3338}
\definecolor{myred}{RGB}{127,0,0}
\definecolor{myyellow}{RGB}{169,121,69}
\definecolor{myexercisebg}{HTML}{F2FBF8}
\definecolor{myexercisefg}{HTML}{88D6D1}


%%%%%%%%%%%%%%%%%%%%%%%%%%%%
% TCOLORBOX SETUPS
%%%%%%%%%%%%%%%%%%%%%%%%%%%%

\setlength{\parindent}{1cm}
%================================
% THEOREM BOX
%================================

\tcbuselibrary{theorems,skins,hooks}
\newtcbtheorem[number within=chapter]{Theorem}{Théorème}
{%
	enhanced,
	breakable,
	colback = mytheorembg,
	frame hidden,
	boxrule = 0sp,
	borderline west = {2pt}{0pt}{mytheoremfr},
	sharp corners,
	detach title,
	before upper = \tcbtitle\par\smallskip,
	coltitle = mytheoremfr,
	fonttitle = \bfseries\sffamily,
	description font = \mdseries,
	separator sign none,
	segmentation style={solid, mytheoremfr},
}
{th}


\tcbuselibrary{theorems,skins,hooks}
\newtcolorbox{Theoremcon}
{%
	enhanced
	,breakable
	,colback = mytheorembg
	,frame hidden
	,boxrule = 0sp
	,borderline west = {2pt}{0pt}{mytheoremfr}
	,sharp corners
	,description font = \mdseries
	,separator sign none
}

%================================
% Corollery
%================================
\tcbuselibrary{theorems,skins,hooks}
\newtcbtheorem[use counter=tcb@cnt@Theorem]{Corollary}{Corollaire}
{%
	enhanced
	,breakable
	,colback = myp!10
	,frame hidden
	,boxrule = 0sp
	,borderline west = {2pt}{0pt}{myp!85!black}
	,sharp corners
	,detach title
	,before upper = \tcbtitle\par\smallskip
	,coltitle = myp!85!black
	,fonttitle = \bfseries\sffamily
	,description font = \mdseries
	,separator sign none
	,segmentation style={solid, myp!85!black}
}
{th}

%================================
% LENMA
%================================

\tcbuselibrary{theorems,skins,hooks}
\newtcbtheorem[use counter=tcb@cnt@Theorem]{Lemma}{Lemme}
{%
	enhanced,
	breakable,
	colback = mylenmabg,
	frame hidden,
	boxrule = 0sp,
	borderline west = {2pt}{0pt}{mylenmafr},
	sharp corners,
	detach title,
	before upper = \tcbtitle\par\smallskip,
	coltitle = mylenmafr,
	fonttitle = \bfseries\sffamily,
	description font = \mdseries,
	separator sign none,
	segmentation style={solid, mylenmafr},
}
{th}


%================================
% PROPOSITION
%================================

\tcbuselibrary{theorems,skins,hooks}
\newtcbtheorem[use counter=tcb@cnt@Theorem]{Prop}{Proposition}
{%
	enhanced,
	breakable,
	colback = mypropbg,
	frame hidden,
	boxrule = 0sp,
	borderline west = {2pt}{0pt}{mypropfr},
	sharp corners,
	detach title,
	before upper = \tcbtitle\par\smallskip,
	coltitle = mypropfr,
	fonttitle = \bfseries\sffamily,
	description font = \mdseries,
	separator sign none,
	segmentation style={solid, mypropfr},
}
{th}


%================================
% CLAIM
%================================

\tcbuselibrary{theorems,skins,hooks}
\newtcbtheorem[use counter=tcb@cnt@Theorem]{claim}{Claim}
{%
	enhanced
	,breakable
	,colback = myg!10
	,frame hidden
	,boxrule = 0sp
	,borderline west = {2pt}{0pt}{myg}
	,sharp corners
	,detach title
	,before upper = \tcbtitle\par\smallskip
	,coltitle = myg!85!black
	,fonttitle = \bfseries\sffamily
	,description font = \mdseries
	,separator sign none
	,segmentation style={solid, myg!85!black}
}
{th}



%================================
% Exercise
%================================

\tcbuselibrary{theorems,skins,hooks}
\newtcbtheorem[use counter=tcb@cnt@Theorem]{Exercise}{Exercice}
{%
	enhanced,
	breakable,
	colback = myexercisebg,
	frame hidden,
	boxrule = 0sp,
	borderline west = {2pt}{0pt}{myexercisefg},
	sharp corners,
	detach title,
	before upper = \tcbtitle\par\smallskip,
	coltitle = myexercisefg,
	fonttitle = \bfseries\sffamily,
	description font = \mdseries,
	separator sign none,
	segmentation style={solid, myexercisefg},
}
{th}

%================================
% EXAMPLE BOX
%================================

\newtcbtheorem[use counter=tcb@cnt@Theorem]{Example}{Exemple}
{%
	colback = myexamplebg
	,breakable
	,colframe = myexamplefr
	,coltitle = myexampleti
	,boxrule = 1pt
	,sharp corners
	,detach title
	,before upper=\tcbtitle\par\smallskip
	,fonttitle = \bfseries
	,description font = \mdseries
	,separator sign none
	,description delimiters parenthesis
}
{ex}

%================================
% DEFINITION BOX
%================================

\newtcbtheorem[use counter=tcb@cnt@Theorem]{Definition}{Définition}{enhanced,
	before skip=2mm,after skip=2mm, colback=red!5,colframe=red!80!black,boxrule=0.5mm,
	attach boxed title to top left={xshift=1cm,yshift*=1mm-\tcboxedtitleheight}, varwidth boxed title*=-3cm,
	boxed title style={frame code={
					\path[fill=tcbcolback]
					([yshift=-1mm,xshift=-1mm]frame.north west)
					arc[start angle=0,end angle=180,radius=1mm]
					([yshift=-1mm,xshift=1mm]frame.north east)
					arc[start angle=180,end angle=0,radius=1mm];
					\path[left color=tcbcolback!60!black,right color=tcbcolback!60!black,
						middle color=tcbcolback!80!black]
					([xshift=-2mm]frame.north west) -- ([xshift=2mm]frame.north east)
					[rounded corners=1mm]-- ([xshift=1mm,yshift=-1mm]frame.north east)
					-- (frame.south east) -- (frame.south west)
					-- ([xshift=-1mm,yshift=-1mm]frame.north west)
					[sharp corners]-- cycle;
				},interior engine=empty,
		},
	fonttitle=\bfseries,
	title={#2},#1}{def}

%================================
% Solution BOX
%================================

\makeatletter
\newtcbtheorem[use counter=tcb@cnt@Theorem]{question}{Question}{enhanced,
	breakable,
	colback=white,
	colframe=myb!80!black,
	attach boxed title to top left={yshift*=-\tcboxedtitleheight},
	fonttitle=\bfseries,
	title={#2},
	boxed title size=title,
	boxed title style={%
			sharp corners,
			rounded corners=northwest,
			colback=tcbcolframe,
			boxrule=0pt,
		},
	underlay boxed title={%
			\path[fill=tcbcolframe] (title.south west)--(title.south east)
			to[out=0, in=180] ([xshift=5mm]title.east)--
			(title.center-|frame.east)
			[rounded corners=\kvtcb@arc] |-
			(frame.north) -| cycle;
		},
	#1
}{def}
\makeatother

%================================
% SOLUTION BOX
%================================

\makeatletter
\newtcolorbox{solution}{enhanced,
	breakable,
	colback=white,
	colframe=myg!80!black,
	attach boxed title to top left={yshift*=-\tcboxedtitleheight},
	title=Solution,
	boxed title size=title,
	boxed title style={%
			sharp corners,
			rounded corners=northwest,
			colback=tcbcolframe,
			boxrule=0pt,
		},
	underlay boxed title={%
			\path[fill=tcbcolframe] (title.south west)--(title.south east)
			to[out=0, in=180] ([xshift=5mm]title.east)--
			(title.center-|frame.east)
			[rounded corners=\kvtcb@arc] |-
			(frame.north) -| cycle;
		},
}
\makeatother

%================================
% Question BOX
%================================

\makeatletter
\newtcbtheorem[use counter=tcb@cnt@Theorem]{qstion}{Question}{enhanced,
	breakable,
	colback=white,
	colframe=mygr,
	attach boxed title to top left={yshift*=-\tcboxedtitleheight},
	fonttitle=\bfseries,
	title={#2},
	boxed title size=title,
	boxed title style={%
			sharp corners,
			rounded corners=northwest,
			colback=tcbcolframe,
			boxrule=0pt,
		},
	underlay boxed title={%
			\path[fill=tcbcolframe] (title.south west)--(title.south east)
			to[out=0, in=180] ([xshift=5mm]title.east)--
			(title.center-|frame.east)
			[rounded corners=\kvtcb@arc] |-
			(frame.north) -| cycle;
		},
	#1
}{def}
\makeatother

\newtcbtheorem[number within=chapter]{wconc}{Wrong Concept}{
	breakable,
	enhanced,
	colback=white,
	colframe=myr,
	arc=0pt,
	outer arc=0pt,
	fonttitle=\bfseries\sffamily\large,
	colbacktitle=myr,
	attach boxed title to top left={},
	boxed title style={
			enhanced,
			skin=enhancedfirst jigsaw,
			arc=3pt,
			bottom=0pt,
			interior style={fill=myr}
		},
	#1
}{def}



%================================
% NOTE BOX
%================================

\usetikzlibrary{arrows,calc,shadows.blur}
\tcbuselibrary{skins}
\newtcolorbox{note}[1][]{%
	enhanced jigsaw,
	colback=gray!20!white,%
	colframe=gray!80!black,
	size=small,
	boxrule=1pt,
	title=\colorbox{white!100}{\textbf{ Remarque }},
	halign title=flush center,
	coltitle=black,
	breakable,
	drop shadow=black!50!white,
	attach boxed title to top left={xshift=1cm,yshift=-\tcboxedtitleheight/2,yshifttext=-\tcboxedtitleheight/2},
	minipage boxed title=2.6cm,
	boxed title style={%
			colback=white,
			size=fbox,
			boxrule=1pt,
			boxsep=2pt,
			underlay={%
					\coordinate (dotA) at ($(interior.west) + (-0.5pt,0)$);
					\coordinate (dotB) at ($(interior.east) + (0.5pt,0)$);
					\begin{scope}
						\clip (interior.north west) rectangle ([xshift=3ex]interior.east);
						\filldraw [white, blur shadow={shadow opacity=60, shadow yshift=-.75ex}, rounded corners=2pt] (interior.north west) rectangle (interior.south east);
					\end{scope}
					\begin{scope}[gray!80!black]
						\fill (dotA) circle (2pt);
						\fill (dotB) circle (2pt);
					\end{scope}
				},
		},
	#1,
}

%================================
% STRATÉGIE BOX
%================================

\usetikzlibrary{arrows,calc,shadows.blur}
\tcbuselibrary{skins}
\newtcolorbox{strategy}[1][]{%
	enhanced jigsaw,
	colback=myb!20!white,%
	colframe=gray!80!black,
	size=small,
	boxrule=1pt,
	title=\colorbox{white!100}{\textbf{ Stratégie }},
	halign title=flush center,
	coltitle=black,
	breakable,
	drop shadow=black!50!white,
	attach boxed title to top left={xshift=1cm,yshift=-\tcboxedtitleheight/2,yshifttext=-\tcboxedtitleheight/2},
	minipage boxed title=2.5cm,
	boxed title style={%
			colback=white,
			size=fbox,
			boxrule=1pt,
			boxsep=2pt,
			underlay={%
					\coordinate (dotA) at ($(interior.west) + (-0.5pt,0)$);
					\coordinate (dotB) at ($(interior.east) + (0.5pt,0)$);
					\begin{scope}
						\clip (interior.north west) rectangle ([xshift=3ex]interior.east);
						\filldraw [white, blur shadow={shadow opacity=60, shadow yshift=-.75ex}, rounded corners=2pt] (interior.north west) rectangle (interior.south east);
					\end{scope}
					\begin{scope}[gray!80!black]
						\fill (dotA) circle (2pt);
						\fill (dotB) circle (2pt);
					\end{scope}
				},
		},
	#1,
}

%================================
% MÉTHODE BOX
%================================

\usetikzlibrary{arrows,calc,shadows.blur}
\tcbuselibrary{skins}
\newtcolorbox{methode}[1][]{%
	enhanced jigsaw,
	colback=white,%
	colframe=gray!80!black,
	size=small,
	boxrule=1pt,
	title=\textbf{Méthode},
	halign title=flush center,
	coltitle=black,
	breakable,
	drop shadow=black!50!white,
	attach boxed title to top left={xshift=1cm,yshift=-\tcboxedtitleheight/2,yshifttext=-\tcboxedtitleheight/2},
	minipage boxed title=2.5cm,
	boxed title style={%
			colback=white,
			size=fbox,
			boxrule=1pt,
			boxsep=2pt,
			underlay={%
					\coordinate (dotA) at ($(interior.west) + (-0.5pt,0)$);
					\coordinate (dotB) at ($(interior.east) + (0.5pt,0)$);
					\begin{scope}
						\clip (interior.north west) rectangle ([xshift=3ex]interior.east);
						\filldraw [white, blur shadow={shadow opacity=60, shadow yshift=-.75ex}, rounded corners=2pt] (interior.north west) rectangle (interior.south east);
					\end{scope}
					\begin{scope}[gray!80!black]
						\fill (dotA) circle (2pt);
						\fill (dotB) circle (2pt);
					\end{scope}
				},
		},
	#1,
}

%%%%%%%%%%%%%%%%%%%%%%%%%%%%%%%%%%%%%%%%%%%
% TABLE OF CONTENTS
%%%%%%%%%%%%%%%%%%%%%%%%%%%%%%%%%%%%%%%%%%%

\usepackage{tikz}

\definecolor{doc}{RGB}{0,60,110}
\usepackage{titletoc}
\contentsmargin{0cm}
\titlecontents{chapter}[3.7pc]
{\addvspace{30pt}%
	\begin{tikzpicture}[remember picture, overlay]%
		\draw[fill=doc!60,draw=doc!60] (-7,-.1) rectangle (-0.2,.6);%
		\pgftext[left,x=-3.5cm,y=0.2cm]{\color{white}\Large\sc\bfseries Chapitre\ \thecontentslabel};%
	\end{tikzpicture}\color{doc!60}\large\sc\bfseries}%
{}
{}
{\;\titlerule\;\large\sc\bfseries Page \thecontentspage
	\begin{tikzpicture}[remember picture, overlay]
		\draw[fill=doc!60,draw=doc!60] (2pt,0) rectangle (4,0.1pt);
	\end{tikzpicture}}%
\titlecontents{section}[3.7pc]
{\addvspace{2pt}}
{\contentslabel[\thecontentslabel]{2pc}}
{}
{\hfill\small \thecontentspage}
[]
\titlecontents*{subsection}[3.7pc]
{\addvspace{-1pt}\small}
{}
{}
{\ --- \small\thecontentspage}
[ \textbullet\ ][]

\makeatletter
\renewcommand{\tableofcontents}{%
	\chapter*{%
	  \vspace*{-20\p@}%
	  \begin{tikzpicture}[remember picture, overlay]%
		  \pgftext[right,x=15cm,y=0.2cm]{\color{doc!60}\Huge\sc\bfseries \contentsname};%
		  \draw[fill=doc!60,draw=doc!60] (13,-.75) rectangle (20,1);%
		  \clip (13,-.75) rectangle (20,1);
		  \pgftext[right,x=15cm,y=0.2cm]{\color{white}\Huge\sc\bfseries \contentsname};%
	  \end{tikzpicture}}%
	\@starttoc{toc}}
\makeatother


%%%%%%%%%%%%%%%%%%%%%%%%%%%%%%%%%%%%%%%%%%%
% MINTED FOR PYTHON ALGORITHMS
%%%%%%%%%%%%%%%%%%%%%%%%%%%%%%%%%%%%%%%%%%%

\usepackage{tcolorbox}
\tcbuselibrary{minted,breakable,xparse,skins}
\definecolor{bg}{gray}{0.95}
\DeclareTCBListing{mintedbox}{O{}m!O{}}{%
  breakable=true,
  listing engine=minted,
  listing only,
  minted language=#2,
  minted style=default,
  minted options={%
    linenos,
    gobble=0,
    breaklines=true,
    breakafter=,,
    fontsize=\small,
    numbersep=8pt,
    #1},
  boxsep=0pt,
  left skip=0pt,
  right skip=0pt,
  left=25pt,
  right=0pt,
  top=3pt,
  bottom=3pt,
  arc=5pt,
  leftrule=0pt,
  rightrule=0pt,
  bottomrule=2pt,
  toprule=2pt,
  colback=bg,
  colframe=orange!70,
  enhanced,
  overlay={%
    \begin{tcbclipinterior}
    \fill[orange!20!white] (frame.south west) rectangle ([xshift=20pt]frame.north west);
    \end{tcbclipinterior}},
  #3}
  
  
 % for braces
\usetikzlibrary{decorations.pathreplacing}


\SetDate[04/11/2025]

\begin{document}
\pagestyle{fancy}
\fancyhead[L]{Seconde}
\fancyhead[C]{\textbf{Puissances et ordres de grandeur}}
\fancyhead[R]{\today}


\exe{}{
	Sans calculatrice, exprimer les nombres suivants sous la forme $q^n$, où $q \in \N$ et $n\in\Z$ sont des entiers.
	\begin{multicols}{3}
	\begin{enumerate}[label=\roman*)]
		\item $10^3 \times 10^5$
		\item $\left(4^5\right)^2$
		\item $\dfrac{5^3}{5^3}$
		\item $1$
		\item $\dfrac{2^4}{2^7}$
		\item $\left(2^{-1}\right)^3$
		\item $\left(2^{3}\right)^{-1}$
		\item $\left(\dfrac{1}{7^2}\right)^6$
		\item $\dfrac{10^{12}}{10^{-12}}$
		\item $\dfrac{10^{-5}}{10^{6}}$
	\end{enumerate}
	\end{multicols}
}{exe:puissances}{
	\begin{multicols}{3}
	\begin{enumerate}[label=\roman*)]
		\item $10^8$
		\item $4^{10}$
		\item $5^0 = 1^1 = 4^0 = q^0$ pour n'importe quel $q\neq0$
		\item $1^1 = 2^0 = 100^0 = q^0$ pour n'importe quel $q\neq0$
		\item $2^{-3}$
		\item $2^{-3}$
		\item $2^{-3}$
		\item $7^{-12}$
		\item $10^{24}$
		\item $10^{-11}$
	\end{enumerate}
	\end{multicols}
}

\exe{}{
	On estime que, dans l'univers, il y a au moins
		\begin{itemize}
			\item $10^{11}$ galaxies ; que chacune contient
			\item $10^{11}$ étoiles ; dont la masse moyenne est de
			\item $10^{32}$ \underline{kilogrammes} ; et que chaque \underline{gramme} de matière contient
			\item $10^{24}$ atomes.
		\end{itemize}
	Estimer le nombre d'atomes dans l'univers observable à partir de ces données.
}{exe:atomes-univers}{
	Le nombre d'étoiles est de donné par $10^{11} \times 10^{11} = 10^{22}$.
	Par conséquent, la masse totale est de $10^{22} \times10^{32} = 10^{54}$ kilogrammes, ce qui est égal à $10^{54} \times 10^3 = 10^{57}$ grammes.
	Finalement, le nombre d'atomes peut être estimé à $10^{57} \times10^{24} = 10^{81}$.
}


\exe{}{
	Montrer qu'on a environ $2^{10} \approx 10^3$. 	
	\begin{enumerate}
		\item En déduire approximativement l'ordre de grandeur de $2^{20}$ et le nombre de chiffres nécessaires pour l'écrire.
		\item En déduire approximativement l'ordre de grandeur de $2^{35}$ et le nombre de chiffres nécessaires pour l'écrire.
	\end{enumerate}
}{exe:grandeur-binaire}{
	D'abord, $2^{10} = 1024 \approx 10^3$, comme requis.
	\begin{enumerate}
		\item
		Il suit que $2^{20} = \left( 2^{10} \right)^2 \approx \left(10^3\right)^2 = 10^6$.
		Il faut donc 7 chiffres pour écrire $2^{20}$ en base 10 (et il en faut 21 pour l'écrire en base 2).
		\item 
		Similairement, $2^{35} = 2^5 \times \left( 2^{10} \right)^3 \approx 32 \times \left(10^3\right)^3 = 32 \times 10^9 = 3,2 \times 10^{10}$.
		Il faut donc 11 chiffres pour écrire $2^{35}$ en base 10.
	\end{enumerate}
}


\exe{}{
	Montrer que $10^{50} - 10^{30} = 10^{50} \bigl( 1 - 10^{-20} \bigr)$.
	Décrire le développement décimal de $1-10^{-20}$. De quel entier est-il très proche ?
	
	En déduire l'ordre de grandeur de $10^{50} - 10^{30}$.
}{exe:grandeur-soustraction}{
	Par distributivité,
		\[ 10^{50} \bigl( 1 - 10^{-20} \bigr) = 10^{50} \times 1 + 10^{50} \times 10^{-20} = 10^{50} + 10^{30}. \]
	Comme $10^{-20} = \underbrace{0,00\cdots00}_{\text{20 zéros}}1$, on en déduit que
		\[ 1-10^{-20} = 0,\underbrace{99\cdots99}_{\text{20 neufs}}, \]
	nombre très proche de 1.
	L'ordre de grandeur de $10^{50} - 10^{30}$ est donc $10^{50}$.
	Plus généralement, seuls les ordres dominants sont pris en compte lors de l'estimation de la grandeur d'un nombre.
}

\exe{}{
	Exprimer en écriture scientifique la valeur obtenue après une diminution de 60\% de $10^{60}$.
	Quel est son ordre de grandeur ?
}{exe:grandeur-évolution}{
	Après diminution de 60\%, il reste 40\% de la valeur.
	La valeur finale est donc $0,4 \times 10^{60} = 4 \times 10^{59}$.
	Son ordre de grandeur est de $10^{59}$.
	
	Les ordres de grandeurs ne changent pas drastiquement même lorsque sujets à une grande évolution.
	Par exemple, prendre 1\% d'une valeur ne la diminuera que de deux ordres de grandeur (car $1\% = 10^{-2}$).
}

\exe{}{
	En 2022, 776 millions de tonnes de riz ont été produites, chaque grain riz pesant environ 0,04g.
	Combien de grains de riz ont été produits en 2022 ?
	
	\emph{Indications : 1 million = $10^6$, et 1 tonne = $10^6$ grammes. }
}{exe:grain-riz}{
	D'après l'indication, $776 \times 10^6 \times 10^6 = 776 \times 10^{12}$ grammes de riz sont produits.
	Par proportionnalité, on divise par $0,04$ pour obtenir le nombre de grains de riz, soit
		\[ \dfrac{776 \times 10^{12}}{0,04} = 19400 \times 10^{12} = 1,94 \times 10^{16}. \]
}

\subsection*{Exercices supplémentaires}

\exe{, difficulty=2}{
    Considérons $p\geq 0$ une proportion et $100p$ le pourcentage associé.
    \begin{enumerate}
    \item À quelle évolution, en fonction de $p$, correspond une augmentation de $(100p)\%$ suivie d'une diminution de $(100p)\%$ ?
    \item Quel $p$ choisir pour trouver une diminution finale de $16\%$ ?
    \end{enumerate}
}{exe:sqrt-evol}{
    \begin{enumerate}
    \item Soit $N\geq0$ une quantité. Après une augmentation de $(100p)\%$ puis une diminution de $(100p)\%$,
    		la quantité est donnée par $(1-p) \cdot (1+p) \cdot N = (1-p^2) \cdot N$.
    		Ceci correspond à une diminution de $100\left(p^2\right) \%$.
    		
    		On comparera avec l'exercice 9, où $p=0,2$ et $p^2 = 0,04 = 4\%$.
    \item On pose l'égalité suivante
    		\[ 100p^2 = 16. \]
    	Remarquons que $16$ et $100$ sont tous les deux des carrés parfaits :
    		\[ p^2 = \dfrac{16}{100} = \left( \dfrac{4}{10} \right)^2, \]
	et donc $p = \dfrac{4}{10} = 40\%$, car $p\geq 0$.
	
	Vérification : au vu de la question 1, on calcule $0,4^2 = 0,16 = 16\%$.
    \end{enumerate}

}

\exe{, difficulty=1}{
	Soient $q\in\R$ et $n\in\Z$.
	Montrer que $q^n$ est positif si $n$ est pair.
	Montrer que $q^n$ est du signe de $q$ si $n$ est impair.
}{exe:signe-qn}{
	Si $n$ est pair, c'est un multiple de 2 : $n = 2k$ pour un entier $k \in \Z$.
	Dans ce cas, $q^n = \left(q^k\right)^2$, qui est un carré et est donc positif.
	
	Si $n$ est impair, alors l'entier d'avant est pair : $n-1 =2k$ pour un entier $k\in\Z$.
	Dans ce cas, $n=2k+1$, et $q^n = \left( q^k \right)^2 \times q$, qui est un nombre positif fois $q$, du signe de $q$.
}

\exe{, difficulty=2}{
	Combien d'opérations arithmétiques (additions ou multiplications) sont nécessaires \emph{a priori} pour calculer $17^{32}$ ?

	Montrer qu'on peut calculer $17^{32}$ en seulement 5 multiplications.
	
	\emph{Indication : montrer que le calcul $17^4 = \left(17^2\right)^2$ n'utilise que 2 multiplications.}
}{exe:17pow32}{
	Naïvement, $17^{32} = \underbrace{17 \times \cdots \times 17}_{\text{32 fois}}$, ce qui coûte 31 multiplications.

	Pour mieux optimiser le calcul, montrons d'abord l'indication.
	Déjà, $17^2 = 17 \times 17$, qui coûte une multiplication.
	Ensuite, en connaissant le résultat, calculer $\left(17^2\right)^2 =  17^2 \times 17^2$ ne coûte qu'une multiplication, et permet de calculer $17^4$ en 2 multiplications au total.
	
	Procédons de la même manière par mise au carré successives.
	Une fois $17^4$ connu, $17^4 \times 17^4 = 17^8$ ne coûte qu'une multiplication.
	
	De nouveau, une fois $17^8$ connu, $17^8 \times 17^8 = 17^{16}$ ne coûte qu'une multiplication.
	
	Finalement, une fois $17^{16}$ connu, $17^{16} \times 17^{16} = 17^{32}$ ne coûte qu'une multiplication.
	
	Au total : 5 multiplications pour le calcul (compter les signes $\times$ dans la description ci-dessus !).
	
	Généralement, il ne faut que $n$ multiplications pour calculer $q^{2^n}$ : c'est l'\emph{exponentiation rapide}.
	Si on souhaite calculer une puissance qui n'est pas une puissance de 2, alors il suffit de décomposer cette puissance en base 2.
	Par exemple, pour calculer $q^{137}$, on décompose
		\[ 137 = 128 + 8 + 1, \]
	ce qui implique que $q^{137} = q^{128} \times q^8 \times q$.
	Il suffit alors de 7 multiplications pour calculer $q^2, q^4, q^8, \dots, q^{128}$, et de deux de plus pour en déduire $q^{137}$.
	Bien sûr, n'effectuer que 9 multiplications est bien plus rapide que les 136 du calcul naïf, et c'est pour ça qu'une calculatrice peut calculer de grandes puissances très rapidement.
}



%%%%%%%%%%%

\newpage
\fancyhead[C]{\textbf{Solutions}}
\shipoutAnswer

\end{document}

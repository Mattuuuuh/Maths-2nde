%!TEX encoding = UTF8
%!TEX root =notes.tex


%%%%%%%%%%%%%%%%%%%%%%%%%%%%%%%%%
% PACKAGE IMPORTS
%%%%%%%%%%%%%%%%%%%%%%%%%%%%%%%%%


\usepackage[french]{babel}

\usepackage[tmargin=2cm,rmargin=1in,lmargin=1in,margin=0.85in,bmargin=2cm,footskip=.2in]{geometry}
\usepackage{amsmath,amsfonts,amsthm,amssymb,mathtools}
\usepackage[varbb]{newpxmath}
\usepackage{xfrac}
\usepackage[makeroom]{cancel}
\usepackage{mathtools}
\usepackage{bookmark}
\usepackage{enumitem}
\usepackage{hyperref,theoremref}
\hypersetup{
	pdftitle={Assignment},
	colorlinks=true, linkcolor=doc!90,
	bookmarksnumbered=true,
	bookmarksopen=true
}
\usepackage[most,many,breakable]{tcolorbox}
\usepackage{xcolor}
\usepackage{varwidth}
\usepackage{varwidth}
\usepackage{etoolbox}
%\usepackage{authblk}
\usepackage{nameref}
\usepackage{multicol,array}
\usepackage{tikz-cd}
\usepackage[ruled,vlined,linesnumbered]{algorithm2e}
\usepackage{comment} % enables the use of multi-line comments (\ifx \fi) 
\usepackage{import}
\usepackage{xifthen}
\usepackage{pdfpages}
\usepackage{transparent}


\newcommand\mycommfont[1]{\footnotesize\ttfamily\textcolor{blue}{#1}}
\SetCommentSty{mycommfont}
\newcommand{\incfig}[1]{%
    \def\svgwidth{\columnwidth}
    \import{./figures/}{#1.pdf_tex}
}

\usepackage{tikzsymbols}
%\renewcommand\qedsymbol{$\Laughey$}


%\usepackage{import}
%\usepackage{xifthen}
%\usepackage{pdfpages}
%\usepackage{transparent}


%%%%%%%%%%%%%%%%%%%%%%%%%%%%%%
% SELF MADE COLORS
%%%%%%%%%%%%%%%%%%%%%%%%%%%%%%



\definecolor{myg}{RGB}{56, 140, 70}
\definecolor{myb}{RGB}{45, 111, 177}
\definecolor{myr}{RGB}{199, 68, 64}
\definecolor{mytheorembg}{HTML}{F2F2F9}
\definecolor{mytheoremfr}{HTML}{00007B}
\definecolor{mylenmabg}{HTML}{FFFAF8}
\definecolor{mylenmafr}{HTML}{983b0f}
\definecolor{mypropbg}{HTML}{f2fbfc}
\definecolor{mypropfr}{HTML}{191971}
\definecolor{myexamplebg}{HTML}{F2FBF8}
\definecolor{myexamplefr}{HTML}{88D6D1}
\definecolor{myexampleti}{HTML}{2A7F7F}
\definecolor{mydefinitbg}{HTML}{E5E5FF}
\definecolor{mydefinitfr}{HTML}{3F3FA3}
\definecolor{notesgreen}{RGB}{0,162,0}
\definecolor{myp}{RGB}{197, 92, 212}
\definecolor{mygr}{HTML}{2C3338}
\definecolor{myred}{RGB}{127,0,0}
\definecolor{myyellow}{RGB}{169,121,69}
\definecolor{myexercisebg}{HTML}{F2FBF8}
\definecolor{myexercisefg}{HTML}{88D6D1}


%%%%%%%%%%%%%%%%%%%%%%%%%%%%
% TCOLORBOX SETUPS
%%%%%%%%%%%%%%%%%%%%%%%%%%%%

\setlength{\parindent}{1cm}
%================================
% THEOREM BOX
%================================

\tcbuselibrary{theorems,skins,hooks}
\newtcbtheorem[number within=chapter]{Theorem}{Théorème}
{%
	enhanced,
	breakable,
	colback = mytheorembg,
	frame hidden,
	boxrule = 0sp,
	borderline west = {2pt}{0pt}{mytheoremfr},
	sharp corners,
	detach title,
	before upper = \tcbtitle\par\smallskip,
	coltitle = mytheoremfr,
	fonttitle = \bfseries\sffamily,
	description font = \mdseries,
	separator sign none,
	segmentation style={solid, mytheoremfr},
}
{th}


\tcbuselibrary{theorems,skins,hooks}
\newtcolorbox{Theoremcon}
{%
	enhanced
	,breakable
	,colback = mytheorembg
	,frame hidden
	,boxrule = 0sp
	,borderline west = {2pt}{0pt}{mytheoremfr}
	,sharp corners
	,description font = \mdseries
	,separator sign none
}

%================================
% Corollery
%================================
\tcbuselibrary{theorems,skins,hooks}
\newtcbtheorem[use counter=tcb@cnt@Theorem]{Corollary}{Corollaire}
{%
	enhanced
	,breakable
	,colback = myp!10
	,frame hidden
	,boxrule = 0sp
	,borderline west = {2pt}{0pt}{myp!85!black}
	,sharp corners
	,detach title
	,before upper = \tcbtitle\par\smallskip
	,coltitle = myp!85!black
	,fonttitle = \bfseries\sffamily
	,description font = \mdseries
	,separator sign none
	,segmentation style={solid, myp!85!black}
}
{th}

%================================
% LENMA
%================================

\tcbuselibrary{theorems,skins,hooks}
\newtcbtheorem[use counter=tcb@cnt@Theorem]{Lemma}{Lemme}
{%
	enhanced,
	breakable,
	colback = mylenmabg,
	frame hidden,
	boxrule = 0sp,
	borderline west = {2pt}{0pt}{mylenmafr},
	sharp corners,
	detach title,
	before upper = \tcbtitle\par\smallskip,
	coltitle = mylenmafr,
	fonttitle = \bfseries\sffamily,
	description font = \mdseries,
	separator sign none,
	segmentation style={solid, mylenmafr},
}
{th}


%================================
% PROPOSITION
%================================

\tcbuselibrary{theorems,skins,hooks}
\newtcbtheorem[use counter=tcb@cnt@Theorem]{Prop}{Proposition}
{%
	enhanced,
	breakable,
	colback = mypropbg,
	frame hidden,
	boxrule = 0sp,
	borderline west = {2pt}{0pt}{mypropfr},
	sharp corners,
	detach title,
	before upper = \tcbtitle\par\smallskip,
	coltitle = mypropfr,
	fonttitle = \bfseries\sffamily,
	description font = \mdseries,
	separator sign none,
	segmentation style={solid, mypropfr},
}
{th}


%================================
% CLAIM
%================================

\tcbuselibrary{theorems,skins,hooks}
\newtcbtheorem[use counter=tcb@cnt@Theorem]{claim}{Claim}
{%
	enhanced
	,breakable
	,colback = myg!10
	,frame hidden
	,boxrule = 0sp
	,borderline west = {2pt}{0pt}{myg}
	,sharp corners
	,detach title
	,before upper = \tcbtitle\par\smallskip
	,coltitle = myg!85!black
	,fonttitle = \bfseries\sffamily
	,description font = \mdseries
	,separator sign none
	,segmentation style={solid, myg!85!black}
}
{th}



%================================
% Exercise
%================================

\tcbuselibrary{theorems,skins,hooks}
\newtcbtheorem[use counter=tcb@cnt@Theorem]{Exercise}{Exercice}
{%
	enhanced,
	breakable,
	colback = myexercisebg,
	frame hidden,
	boxrule = 0sp,
	borderline west = {2pt}{0pt}{myexercisefg},
	sharp corners,
	detach title,
	before upper = \tcbtitle\par\smallskip,
	coltitle = myexercisefg,
	fonttitle = \bfseries\sffamily,
	description font = \mdseries,
	separator sign none,
	segmentation style={solid, myexercisefg},
}
{th}

%================================
% EXAMPLE BOX
%================================

\newtcbtheorem[use counter=tcb@cnt@Theorem]{Example}{Exemple}
{%
	colback = myexamplebg
	,breakable
	,colframe = myexamplefr
	,coltitle = myexampleti
	,boxrule = 1pt
	,sharp corners
	,detach title
	,before upper=\tcbtitle\par\smallskip
	,fonttitle = \bfseries
	,description font = \mdseries
	,separator sign none
	,description delimiters parenthesis
}
{ex}

%================================
% DEFINITION BOX
%================================

\newtcbtheorem[use counter=tcb@cnt@Theorem]{Definition}{Définition}{enhanced,
	before skip=2mm,after skip=2mm, colback=red!5,colframe=red!80!black,boxrule=0.5mm,
	attach boxed title to top left={xshift=1cm,yshift*=1mm-\tcboxedtitleheight}, varwidth boxed title*=-3cm,
	boxed title style={frame code={
					\path[fill=tcbcolback]
					([yshift=-1mm,xshift=-1mm]frame.north west)
					arc[start angle=0,end angle=180,radius=1mm]
					([yshift=-1mm,xshift=1mm]frame.north east)
					arc[start angle=180,end angle=0,radius=1mm];
					\path[left color=tcbcolback!60!black,right color=tcbcolback!60!black,
						middle color=tcbcolback!80!black]
					([xshift=-2mm]frame.north west) -- ([xshift=2mm]frame.north east)
					[rounded corners=1mm]-- ([xshift=1mm,yshift=-1mm]frame.north east)
					-- (frame.south east) -- (frame.south west)
					-- ([xshift=-1mm,yshift=-1mm]frame.north west)
					[sharp corners]-- cycle;
				},interior engine=empty,
		},
	fonttitle=\bfseries,
	title={#2},#1}{def}

%================================
% Solution BOX
%================================

\makeatletter
\newtcbtheorem[use counter=tcb@cnt@Theorem]{question}{Question}{enhanced,
	breakable,
	colback=white,
	colframe=myb!80!black,
	attach boxed title to top left={yshift*=-\tcboxedtitleheight},
	fonttitle=\bfseries,
	title={#2},
	boxed title size=title,
	boxed title style={%
			sharp corners,
			rounded corners=northwest,
			colback=tcbcolframe,
			boxrule=0pt,
		},
	underlay boxed title={%
			\path[fill=tcbcolframe] (title.south west)--(title.south east)
			to[out=0, in=180] ([xshift=5mm]title.east)--
			(title.center-|frame.east)
			[rounded corners=\kvtcb@arc] |-
			(frame.north) -| cycle;
		},
	#1
}{def}
\makeatother

%================================
% SOLUTION BOX
%================================

\makeatletter
\newtcolorbox{solution}{enhanced,
	breakable,
	colback=white,
	colframe=myg!80!black,
	attach boxed title to top left={yshift*=-\tcboxedtitleheight},
	title=Solution,
	boxed title size=title,
	boxed title style={%
			sharp corners,
			rounded corners=northwest,
			colback=tcbcolframe,
			boxrule=0pt,
		},
	underlay boxed title={%
			\path[fill=tcbcolframe] (title.south west)--(title.south east)
			to[out=0, in=180] ([xshift=5mm]title.east)--
			(title.center-|frame.east)
			[rounded corners=\kvtcb@arc] |-
			(frame.north) -| cycle;
		},
}
\makeatother

%================================
% Question BOX
%================================

\makeatletter
\newtcbtheorem[use counter=tcb@cnt@Theorem]{qstion}{Question}{enhanced,
	breakable,
	colback=white,
	colframe=mygr,
	attach boxed title to top left={yshift*=-\tcboxedtitleheight},
	fonttitle=\bfseries,
	title={#2},
	boxed title size=title,
	boxed title style={%
			sharp corners,
			rounded corners=northwest,
			colback=tcbcolframe,
			boxrule=0pt,
		},
	underlay boxed title={%
			\path[fill=tcbcolframe] (title.south west)--(title.south east)
			to[out=0, in=180] ([xshift=5mm]title.east)--
			(title.center-|frame.east)
			[rounded corners=\kvtcb@arc] |-
			(frame.north) -| cycle;
		},
	#1
}{def}
\makeatother

\newtcbtheorem[number within=chapter]{wconc}{Wrong Concept}{
	breakable,
	enhanced,
	colback=white,
	colframe=myr,
	arc=0pt,
	outer arc=0pt,
	fonttitle=\bfseries\sffamily\large,
	colbacktitle=myr,
	attach boxed title to top left={},
	boxed title style={
			enhanced,
			skin=enhancedfirst jigsaw,
			arc=3pt,
			bottom=0pt,
			interior style={fill=myr}
		},
	#1
}{def}



%================================
% NOTE BOX
%================================

\usetikzlibrary{arrows,calc,shadows.blur}
\tcbuselibrary{skins}
\newtcolorbox{note}[1][]{%
	enhanced jigsaw,
	colback=gray!20!white,%
	colframe=gray!80!black,
	size=small,
	boxrule=1pt,
	title=\colorbox{white!100}{\textbf{ Remarque }},
	halign title=flush center,
	coltitle=black,
	breakable,
	drop shadow=black!50!white,
	attach boxed title to top left={xshift=1cm,yshift=-\tcboxedtitleheight/2,yshifttext=-\tcboxedtitleheight/2},
	minipage boxed title=2.6cm,
	boxed title style={%
			colback=white,
			size=fbox,
			boxrule=1pt,
			boxsep=2pt,
			underlay={%
					\coordinate (dotA) at ($(interior.west) + (-0.5pt,0)$);
					\coordinate (dotB) at ($(interior.east) + (0.5pt,0)$);
					\begin{scope}
						\clip (interior.north west) rectangle ([xshift=3ex]interior.east);
						\filldraw [white, blur shadow={shadow opacity=60, shadow yshift=-.75ex}, rounded corners=2pt] (interior.north west) rectangle (interior.south east);
					\end{scope}
					\begin{scope}[gray!80!black]
						\fill (dotA) circle (2pt);
						\fill (dotB) circle (2pt);
					\end{scope}
				},
		},
	#1,
}

%================================
% STRATÉGIE BOX
%================================

\usetikzlibrary{arrows,calc,shadows.blur}
\tcbuselibrary{skins}
\newtcolorbox{strategy}[1][]{%
	enhanced jigsaw,
	colback=myb!20!white,%
	colframe=gray!80!black,
	size=small,
	boxrule=1pt,
	title=\colorbox{white!100}{\textbf{ Stratégie }},
	halign title=flush center,
	coltitle=black,
	breakable,
	drop shadow=black!50!white,
	attach boxed title to top left={xshift=1cm,yshift=-\tcboxedtitleheight/2,yshifttext=-\tcboxedtitleheight/2},
	minipage boxed title=2.5cm,
	boxed title style={%
			colback=white,
			size=fbox,
			boxrule=1pt,
			boxsep=2pt,
			underlay={%
					\coordinate (dotA) at ($(interior.west) + (-0.5pt,0)$);
					\coordinate (dotB) at ($(interior.east) + (0.5pt,0)$);
					\begin{scope}
						\clip (interior.north west) rectangle ([xshift=3ex]interior.east);
						\filldraw [white, blur shadow={shadow opacity=60, shadow yshift=-.75ex}, rounded corners=2pt] (interior.north west) rectangle (interior.south east);
					\end{scope}
					\begin{scope}[gray!80!black]
						\fill (dotA) circle (2pt);
						\fill (dotB) circle (2pt);
					\end{scope}
				},
		},
	#1,
}

%================================
% MÉTHODE BOX
%================================

\usetikzlibrary{arrows,calc,shadows.blur}
\tcbuselibrary{skins}
\newtcolorbox{methode}[1][]{%
	enhanced jigsaw,
	colback=white,%
	colframe=gray!80!black,
	size=small,
	boxrule=1pt,
	title=\textbf{Méthode},
	halign title=flush center,
	coltitle=black,
	breakable,
	drop shadow=black!50!white,
	attach boxed title to top left={xshift=1cm,yshift=-\tcboxedtitleheight/2,yshifttext=-\tcboxedtitleheight/2},
	minipage boxed title=2.5cm,
	boxed title style={%
			colback=white,
			size=fbox,
			boxrule=1pt,
			boxsep=2pt,
			underlay={%
					\coordinate (dotA) at ($(interior.west) + (-0.5pt,0)$);
					\coordinate (dotB) at ($(interior.east) + (0.5pt,0)$);
					\begin{scope}
						\clip (interior.north west) rectangle ([xshift=3ex]interior.east);
						\filldraw [white, blur shadow={shadow opacity=60, shadow yshift=-.75ex}, rounded corners=2pt] (interior.north west) rectangle (interior.south east);
					\end{scope}
					\begin{scope}[gray!80!black]
						\fill (dotA) circle (2pt);
						\fill (dotB) circle (2pt);
					\end{scope}
				},
		},
	#1,
}

%%%%%%%%%%%%%%%%%%%%%%%%%%%%%%%%%%%%%%%%%%%
% TABLE OF CONTENTS
%%%%%%%%%%%%%%%%%%%%%%%%%%%%%%%%%%%%%%%%%%%

\usepackage{tikz}

\definecolor{doc}{RGB}{0,60,110}
\usepackage{titletoc}
\contentsmargin{0cm}
\titlecontents{chapter}[3.7pc]
{\addvspace{30pt}%
	\begin{tikzpicture}[remember picture, overlay]%
		\draw[fill=doc!60,draw=doc!60] (-7,-.1) rectangle (-0.2,.6);%
		\pgftext[left,x=-3.5cm,y=0.2cm]{\color{white}\Large\sc\bfseries Chapitre\ \thecontentslabel};%
	\end{tikzpicture}\color{doc!60}\large\sc\bfseries}%
{}
{}
{\;\titlerule\;\large\sc\bfseries Page \thecontentspage
	\begin{tikzpicture}[remember picture, overlay]
		\draw[fill=doc!60,draw=doc!60] (2pt,0) rectangle (4,0.1pt);
	\end{tikzpicture}}%
\titlecontents{section}[3.7pc]
{\addvspace{2pt}}
{\contentslabel[\thecontentslabel]{2pc}}
{}
{\hfill\small \thecontentspage}
[]
\titlecontents*{subsection}[3.7pc]
{\addvspace{-1pt}\small}
{}
{}
{\ --- \small\thecontentspage}
[ \textbullet\ ][]

\makeatletter
\renewcommand{\tableofcontents}{%
	\chapter*{%
	  \vspace*{-20\p@}%
	  \begin{tikzpicture}[remember picture, overlay]%
		  \pgftext[right,x=15cm,y=0.2cm]{\color{doc!60}\Huge\sc\bfseries \contentsname};%
		  \draw[fill=doc!60,draw=doc!60] (13,-.75) rectangle (20,1);%
		  \clip (13,-.75) rectangle (20,1);
		  \pgftext[right,x=15cm,y=0.2cm]{\color{white}\Huge\sc\bfseries \contentsname};%
	  \end{tikzpicture}}%
	\@starttoc{toc}}
\makeatother


%%%%%%%%%%%%%%%%%%%%%%%%%%%%%%%%%%%%%%%%%%%
% MINTED FOR PYTHON ALGORITHMS
%%%%%%%%%%%%%%%%%%%%%%%%%%%%%%%%%%%%%%%%%%%

\usepackage{tcolorbox}
\tcbuselibrary{minted,breakable,xparse,skins}
\definecolor{bg}{gray}{0.95}
\DeclareTCBListing{mintedbox}{O{}m!O{}}{%
  breakable=true,
  listing engine=minted,
  listing only,
  minted language=#2,
  minted style=default,
  minted options={%
    linenos,
    gobble=0,
    breaklines=true,
    breakafter=,,
    fontsize=\small,
    numbersep=8pt,
    #1},
  boxsep=0pt,
  left skip=0pt,
  right skip=0pt,
  left=25pt,
  right=0pt,
  top=3pt,
  bottom=3pt,
  arc=5pt,
  leftrule=0pt,
  rightrule=0pt,
  bottomrule=2pt,
  toprule=2pt,
  colback=bg,
  colframe=orange!70,
  enhanced,
  overlay={%
    \begin{tcbclipinterior}
    \fill[orange!20!white] (frame.south west) rectangle ([xshift=20pt]frame.north west);
    \end{tcbclipinterior}},
  #3}
  
  
 % for braces
\usetikzlibrary{decorations.pathreplacing}


\SetDate[04/11/2025]

\begin{document}
\pagestyle{fancy}
\fancyhead[L]{Seconde}
\fancyhead[C]{\textbf{Évolutions}}
\fancyhead[R]{\today}


\exe{}{
        Une classe de Seconde comprend $25$ filles pour $9$ garçons.
        Calculer le pourcentage de filles et de garçons dans la classe.
}{exe:sous-pop1}{
	Le nombre total d'élèves est de $25+9 = 34$.
	
	On calcule donc, pour les filles, la proportion $\dfrac{25}{34} \approx 0,73 = 73\%$.
	
	Idem pour les garçons, $\dfrac{9}{34} \approx 0,27 = 27\%$. 
	Remarquons que la somme des pourcentages est de $100\%$ car $\dfrac{25}{34} + \dfrac{9}{34} = \dfrac{34}{34} = 1 = 100\%$.
	On aurait donc pu déduire le pourcentage de garçons en calculant $100 - 73 = 27$.
}

\exe{}{
   En sachant que les $16 600$ espèces de fourmis constituent environ $1{,}3\%$ du total des espèces d'insectes répertoriées sur Terre, estimer le nombre total d'espèces d'insectes. 
}{exe:sous-pop2}{
	On a $\dfrac{16 600}{\text{nombre d'espèces d'insectes}} = 1,3\% = 0,013$.
	
	Par conséquent, 
		\[ \text{nombre d'espèces d'insectes} = \dfrac{16600}{0,013} \approx 1,3 \times 10^{6}, \]
	soit environ $1,3$ millions.
	
	Remarquons que la fraction $\dfrac{16600}{0,013}$ ne donne pas un nombre entier, car le pourcentage a été approximé (\og \emph{environ} $1,3$\% \fg).
}


\exe{}{
  En 2023 en France, $13\%$ des espèces (faune et flore) sont considérées comme menacées à l'échelle mondiale (catégories ``danger critique'' à ``vulnérable'' de l'UICN).
  Parmis celles-ci, $23\%$ sont en danger critique.
  
  Calculer le pourcentage d'espèces en danger critique par rapport au nombre total d'espèces.
}{exe:sous-pop3}{
	Notons $E$ l'ensemble des espèces indigènes à la France, $M$ la sous-population des espèces menacées, et $D$ la sous-population des espèces en danger critique.
	On a donc la suite d'inclusions
		\[ D \subset M \subset E. \]
	
	Le texte donne les informations
		\begin{align*}
			|M| = 0,13 \cdot |E| && \text{et} && |D| = 0,23 \cdot |M|
		\end{align*}
	Par conséquent, $|D| = 0,23 \times 0,13 \cdot |E| \approx 0,03 \cdot |E|$.
	Donc les espèces en danger critique constituent $3\%$ des espèces.

	Les proportions sont ainsi multiplicatives. Attention à ne pas naïvement multiplier les pourcentages, car $13\times 23 \approx 300$.
}


\exe{}{
	Calculer sans calculatrice les valeurs suivantes.
	\begin{multicols}{3}
	\begin{enumerate}
		\item $75\%$ de $60$
		\item $60\%$ de $75$
		\item $72\%$ de $25$
		\item $68\%$ de $20$
		\item $125\%$ de $40$
		\item $40\%$ de $125$
	\end{enumerate}
	\end{multicols}
}{exe:1}{
	\begin{multicols}{2}
	\begin{enumerate}
		\item $\dfrac34 \cdot 60 = 3 \cdot \dfrac{60}4 = 3 \cdot 15 = 45$
		\item $45$
		\item $\dfrac14 \cdot 72 = 18$
		\item $\dfrac15 \cdot 68 = \dfrac{136}{10} = 13,6$
		\item $40 + \dfrac14 \cdot 40 = 50$
		\item $50$
	\end{enumerate}
	\end{multicols}
}

\exe{}{
	Approximer sans calculatrice les valeurs suivantes.
	\begin{multicols}{3}
	\begin{enumerate}
		\item $33\%$ de $150$
		\item $166\%$ de $180$
		\item $11\%$ de $90$
		\item $89\%$ de $81$
		\item $16,6\%$ de $18$
		\item $83,4\%$ de $36$
	\end{enumerate}
	\end{multicols}
}{exe:2}{
	\begin{multicols}{2}
	\begin{enumerate}
		\item $\approx \dfrac13 \cdot 150 = 50$
		\item $\approx 180 + \dfrac23 \cdot 180 = 180 + 120 = 300$
		\item $\approx \dfrac19 \cdot 90 = 10$
		\item $\approx 81 - \dfrac19 \cdot 81 = 81 - 9 = 72$
		\item $\approx \dfrac16 \cdot 18 = 3$
		\item $\approx 36 - \dfrac16 \cdot 36 = 36 - 6 = 30$
	\end{enumerate}
	\end{multicols}
}

\exe{}{
	On estime la biomasse totale des fourmis sur Terre à $12$ millions de tonnes.
	Ceci serait égal à $20\%$ de la biomasse humaine.
	
	Estimer la biomasse totale des humains sur Terre en tonnes.
}{exe:3}{
	On a la relation
		\[ \text{biomasse des fourmis} = 0,2 \times \text{biomasse humaine}. \]
	D'où
		\[ \text{biomasse humaine} = \dfrac{12 \times 10^6}{0,2} \text{T} = 60 \times 10^6 \text{T}.\] 

}

\exe{}{
	À quelle évolution correspond une augmentation de $20\%$ suivie d'une diminution de $20\%$ ?
}{exe:5}{
	Augmenter une quantité $N$ de $20\%$ correspond à la multiplier par $1,2$.
	Une diminution, elle, multiplie par $0,8$.
	
	La quantité finale est donné par 
		\[ 0,8 \cdot (1,2 \cdot N) = (0,8 \cdot 1,2) \cdot N = 0,96 \cdot N, \]
	qui correspond à une diminution de $4\%$.
}

\exe{, difficulty=1}{
	Si on augmente le prix d'un objet de $150\%$, quel rabais faut-il appliquer pour retrouver le prix initial de l'objet ?
}{exe:6}{
	Notons $P$ le prix initial de l'objet.
	Le prix augmenté vaut donc $1,5 \cdot P$.
	Pour retrouver $P$, il faut multiplier le prix augmenté par l'inverse de $1,5$, soit $1,5^{-1} = \dfrac23 \approx 0,666 = 66,6\%$.
	Ceci correspond à une diminution de $33,4\%$.
}

\exe{}{
	Considérons deux écharpes, l'une à $250$€ et l'autre à $360$€.
	\begin{enumerate}
		\item Quelle augmentation de prix faut-il appliquer à la première écharpe pour qu'elle ait le prix de la deuxième ?
		\item Quel rabais faut-il appliquer à la deuxième écharpe pour qu'elle ait le prix de la première ?
	\end{enumerate}
}{exe:tailleur}{
	\begin{enumerate}
		\item 
		On calcule l'évolution $\dfrac{360}{250}= 1,44 = 144\%$. Ainsi, la deuxième écharpe vaut $144\%$ du prix de la première : une augmentation de $44\%$ est nécessaire.
		\item 
		On calcule l'évolution $\dfrac{250}{360}\approx 0,7 = 70\%$. La première écharpe vaut environ $70\%$ du prix de la deuxième : une diminution de $30\%$ est nécessaire.
	\end{enumerate}
}

\exe{}{
        À quelle évolution correspond une augmentation de $20\%$ suivie d'une diminution de $20\%$ ?
}{exe:augm-dim-20}{
	Augmenter une quantité $N$ de $20\%$ correspond à la multiplier par $1,2$.
	Une diminution, elle, multiplie par $0,8$.
	
	La quantité finale est donné par 
		\[ 0,8 \cdot (1,2 \cdot N) = (0,8 \cdot 1,2) \cdot N = 0,96 \cdot N, \]
	qui correspond à une diminution de $4\%$.
}

\exe{}{
	Si on augmente le prix d'un objet de $150\%$, quel rabais faut-il appliquer pour retrouver le prix initial de l'objet ?
}{exe:reciproque}{
	Notons $P$ le prix initial de l'objet.
	Le prix augmenté vaut donc $1,5 \cdot P$.
	Pour retrouver $P$, il faut multiplier le prix augmenté par l'inverse de $1,5$, soit $1,5^{-1} = \dfrac23 \approx 0,666 = 66,6\%$.
	Ceci correspond à une diminution de $33,4\%$.
}


\exe{, difficulty=1}{
	Lors d'un payement par carte bancaire, une commission valant 1\% du montant de la transaction est versée comme frais bancaire.
	Ainsi, lorsqu'un débiteur paye 100€, le créditeur reçoit 99€, et 1€ est versé à la banque.
	\begin{enumerate}
		\item Pour un payement de 200€, quelle quantité est versée en frais bancaire ? quelle quantité le créditeur reçoit-il ?
		\item Répondre à la même question pour un payement de 202€.
		\item Obtient-on la quantité initiale après une augmentation de 1\% suivie d'une diminution de 1\% ?
		\item Quelle quantité le débiteur doit-il verser pour que le créditeur reçoive exactement 200€ ? Est-ce un nombre réel ? rationnel ? décimal ? entier ?
	\end{enumerate}
}{exe:caution-matthieu-payup}{
	TODO
}

\exe{, difficulty=1}{
	Un magasin propose une remise de 15\% sur tous les articles à partir de deux articles achetés.
	Un client ne souhaite acheter qu'un seul article à 80€.
	
	\begin{enumerate}
		\item
		Le client peut-il, en ajoutant un deuxième article, obtenir un panier à 75€ ?
		Si oui, donner le prix du deuxième article au centime près.
		
		\item
		Quel est le prix maximal du deuxième article qu'il peut se permettre d'ajouter à son panier pour que le prix final reste sous les 80€ ?
		Arrondir au centime près.
	\end{enumerate}
}{exe:linvosges-matthieu}{
	
	\begin{enumerate}
		\item
		Soit $P$ le prix du deuxième article.
		On souhaite que $P$ vérifie
			\[ 0,85 (80+P) = 75. \]
		En résolvant, on obtient $P \approx 8,23$€.
		
		\item
		L'équation pour $P$ est, cette fois-ci, $0,85 (80+P) = 80$, qui donne $P \approx 14,12$€.
	\end{enumerate}
}


\exe{, difficulty=1}{
	Sans calculatrice, exprimer les nombres suivants sous la forme $q^n$, où $q \in \N$ et $n\in\Z$ sont des entiers.
	\begin{multicols}{3}
	\begin{enumerate}[label=\roman*)]
		\item $10^3 \times 10^5$
		\item $\left(4^5\right)^2$
		\item $\dfrac{5^3}{5^3}$
		\item $1$
		\item $\dfrac{2^4}{2^7}$
		\item $\left(2^{-1}\right)^3$
		\item $\left(2^{3}\right)^{-1}$
		\item $\left(\dfrac{1}{7^2}\right)^6$
		\item $\dfrac{10^{12}}{10^{-12}}$
		\item $\dfrac{10^{-5}}{10^{6}}$
	\end{enumerate}
	\end{multicols}
}{exe:puissances}{
	TODO
}

\exe{}{
	On estime que, dans l'univers, il y a au moins
		\begin{itemize}
			\item $10^{11}$ galaxies ; que chacune contient
			\item $10^{11}$ étoiles ; dont la masse moyenne est de
			\item $10^{32}$ kilogrammes ; et que chaque gramme de matière contient
			\item $10^{24}$ atomes.
		\end{itemize}
	Estimer le nombre d'atomes dans l'univers observable à partir de ces données.
}{exe:atomes-univers}{
	TODO
}


\exe{, difficulty=1}{
	Montrer qu'on a environ $2^{10} \approx 10^3$. 	
	\begin{enumerate}
		\item En déduire approximativement l'ordre de grandeur de $2^{20}$ et le nombre de chiffres nécessaires pour l'écrire.
		\item En déduire approximativement l'ordre de grandeur de $2^{35}$ et le nombre de chiffres nécessaires pour l'écrire.
	\end{enumerate}
}{exe:grandeur-binaire}{
	TODO
}

\exe{}{
	Montrer que $10^{50} - 10^{30} = 10^{50} \bigl( 1 - 10^{-20} \bigr)$.
	Décrire le développement décimal de $1-10^{-20}$. De quel entier est-il très proche ?
	
	En déduire l'ordre de grandeur de $10^{50} - 10^{30}$.
}{exe:grandeur-soustraction}{
	TODO
}

\exe{}{
	Exprimer en écriture scientifique la valeur obtenue après une diminution de 60\% de $10^{60}$.
	Quel est son ordre de grandeur ?
}{exe:grandeur-évolution}{
	TODO
}


\exe{, difficulty=2}{
	Une jeune femme dépose $10$€ à la banque. Celle-ci lui promet un taux d'intérêt à l'année de $3\%$.
	Ainsi, après la première année, il y aura $1,03 \times 10 = 10,3$€ sur son compte.
	La deuxième année, il y aura $1,03 \times 10,35 = 10,609$€, etc...
	
	\begin{enumerate}
		\item Combien d'argent aura-t-elle après $5$ ans ?
		\item Combien d'argent aura-t-elle après $50$ ans ?
		\item Combien d'argent y aura-t-il sur son compte après $1000$ ans ?
	\end{enumerate}
	
	%\hfill
}{exe:evol4}{
	\begin{enumerate}
		\item On multiplie $5$ fois par $1,03$, ce qui donne $1,03^5 \times 10\approx 11,59$€.
		\item On multiplie $50$ fois par $1,03$, ce qui donne $1,03^{50} \times 10\approx 43,84$€.
		\item On multiplie $1000$ fois par $1,03$, ce qui donne $1,03^{1000} \times 10\approx 6,87 \times 10^{13}$€, c'est-à-dire environ $68$ billions d'euros ($1$ billion = $1000$ milliards).
	\end{enumerate}
}

\exe{, difficulty=2}{
    Considérons $p\geq 0$ une proportion et $100p$ le pourcentage associé.
    \begin{enumerate}
    \item À quelle évolution, en fonction de $p$, correspond une augmentation de $(100p)\%$ suivie d'une diminution de $(100p)\%$ ?
    \item Quel $p$ choisir pour trouver une diminution finale de $16\%$ ?
    \end{enumerate}
}{exe:sqrt-evol}{
    \begin{enumerate}
    \item Soit $N\geq0$ une quantité. Après une augmentation de $(100p)\%$ puis une diminution de $(100p)\%$,
    		la quantité est donnée par $(1-p) \cdot (1+p) \cdot N = (1-p^2) \cdot N$.
    		Ceci correspond à une diminution de $100\left(p^2\right) \%$.
    		
    		On comparera avec l'exercice 9, où $p=0,2$ et $p^2 = 0,04 = 4\%$.
    \item On pose l'égalité suivante
    		\[ 100p^2 = 16. \]
    	Remarquons que $16$ et $100$ sont tous les deux des carrés parfaits :
    		\[ p^2 = \dfrac{16}{100} = \left( \dfrac{4}{10} \right)^2, \]
	et donc $p = \dfrac{4}{10} = 40\%$, car $p\geq 0$.
	
	Vérification : au vu de la question 1, on calcule $0,4^2 = 0,16 = 16\%$.
    \end{enumerate}

}



%%%%%%%%%%%

\newpage
\fancyhead[C]{\textbf{Solutions}}
\shipoutAnswer

\end{document}

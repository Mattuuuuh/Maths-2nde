% DYSLEXIA SWITCH
\newif\ifdys
		
				% ENABLE or DISABLE font change
				% use XeLaTeX if true
				\dystrue
				\dysfalse


\ifdys

\documentclass[a4paper, 14pt]{extarticle}
\usepackage{amsmath,amsfonts,amsthm,amssymb,mathtools}

\tracinglostchars=3 % Report an error if a font does not have a symbol.
\usepackage{fontspec}
\usepackage{unicode-math}
\defaultfontfeatures{ Ligatures=TeX,
                      Scale=MatchUppercase }

\setmainfont{OpenDyslexic}[Scale=1.0]
\setmathfont{Fira Math} % Or maybe try KPMath-Sans?
\setmathfont{OpenDyslexic Italic}[range=it/{Latin,latin}]
\setmathfont{OpenDyslexic}[range=up/{Latin,latin,num}]

\else

\documentclass[a4paper, 12pt]{extarticle}

\usepackage[utf8x]{inputenc}
%fonts
\usepackage{amsmath,amsfonts,amsthm,amssymb,mathtools}
% comment below to default to computer modern
\usepackage{libertinus,libertinust1math}

\fi


\usepackage[french]{babel}
\usepackage[
a4paper,
margin=2cm,
nomarginpar,% We don't want any margin paragraphs
]{geometry}
\usepackage{icomma}

\usepackage{fancyhdr}
\usepackage{array}
\usepackage{hyperref}

\usepackage{multicol, enumerate}
\newcolumntype{P}[1]{>{\centering\arraybackslash}p{#1}}


\usepackage{stackengine}
\newcommand\xrowht[2][0]{\addstackgap[.5\dimexpr#2\relax]{\vphantom{#1}}}

% theorems

\theoremstyle{plain}
\newtheorem{theorem}{Th\'eor\`eme}
\newtheorem*{sol}{Solution}
\theoremstyle{definition}
\newtheorem{ex}{Exercice}
\newtheorem*{rpl}{Rappel}
\newtheorem{enigme}{Énigme}

% corps
\usepackage{calrsfs}
\newcommand{\C}{\mathcal{C}}
\newcommand{\R}{\mathbb{R}}
\newcommand{\Rnn}{\mathbb{R}^{2n}}
\newcommand{\Z}{\mathbb{Z}}
\newcommand{\N}{\mathbb{N}}
\newcommand{\Q}{\mathbb{Q}}

% variance
\newcommand{\Var}[1]{\text{Var}(#1)}

% domain
\newcommand{\D}{\mathcal{D}}


% date
\usepackage{advdate}
\AdvanceDate[0]


% plots
\usepackage{pgfplots}

% table line break
\usepackage{makecell}
%tablestuff
\def\arraystretch{2}
\setlength\tabcolsep{15pt}

%subfigures
\usepackage{subcaption}

\definecolor{myg}{RGB}{56, 140, 70}
\definecolor{myb}{RGB}{45, 111, 177}
\definecolor{myr}{RGB}{199, 68, 64}

% fake sections with no title to move around the merged pdf
\newcommand{\fakesection}[1]{%
  \par\refstepcounter{section}% Increase section counter
  \sectionmark{#1}% Add section mark (header)
  \addcontentsline{toc}{section}{\protect\numberline{\thesection}#1}% Add section to ToC
  % Add more content here, if needed.
}


% SOLUTION SWITCH
\newif\ifsolutions
				\solutionstrue
				%\solutionsfalse

\ifsolutions
	\newcommand{\exe}[2]{
		\begin{ex} #1  \end{ex}
		\begin{sol} #2 \end{sol}
	}
\else
	\newcommand{\exe}[2]{
		\begin{ex} #1  \end{ex}
	}
	
\fi


% tableaux var, signe
\usepackage{tkz-tab}


%pinfty minfty
\newcommand{\pinfty}{{+}\infty}
\newcommand{\minfty}{{-}\infty}

\begin{document}


\SetDate[04/11/2025]

\begin{document}
\pagestyle{fancy}
\fancyhead[L]{Seconde}
\fancyhead[C]{\textbf{Évolutions}}
\fancyhead[R]{\today}


\exe{}{
        Une classe de Seconde comprend $25$ filles pour $9$ garçons.
        Calculer le pourcentage de filles et de garçons dans la classe.
}{exe:sous-pop1}{
	Le nombre total d'élèves est de $25+9 = 34$.
	
	On calcule donc, pour les filles, la proportion $\dfrac{25}{34} \approx 0,73 = 73\%$.
	
	Idem pour les garçons, $\dfrac{9}{34} \approx 0,27 = 27\%$. 
	Remarquons que la somme des pourcentages est de $100\%$ car $\dfrac{25}{34} + \dfrac{9}{34} = \dfrac{34}{34} = 1 = 100\%$.
	On aurait donc pu déduire le pourcentage de garçons en calculant $100 - 73 = 27$.
}

\exe{}{
   En sachant que les $16 600$ espèces de fourmis constituent environ $1{,}3\%$ du total des espèces d'insectes répertoriées sur Terre, estimer le nombre total d'espèces d'insectes. 
}{exe:sous-pop2}{
	On a $\dfrac{16 600}{\text{nombre d'espèces d'insectes}} = 1,3\% = 0,013$.
	
	Par conséquent, 
		\[ \text{nombre d'espèces d'insectes} = \dfrac{16600}{0,013} \approx 1,3 \times 10^{6}, \]
	soit environ $1,3$ millions.
	
	Remarquons que la fraction $\dfrac{16600}{0,013}$ ne donne pas un nombre entier, car le pourcentage a été approximé (\og \emph{environ} $1,3$\% \fg).
}


\exe{}{
  En 2023 en France, $13\%$ des espèces (faune et flore) sont considérées comme menacées à l'échelle mondiale (catégories ``danger critique'' à ``vulnérable'' de l'UICN).
  Parmis celles-ci, $23\%$ sont en danger critique.
  
  Calculer le pourcentage d'espèces en danger critique par rapport au nombre total d'espèces.
}{exe:sous-pop3}{
	Notons $E$ l'ensemble des espèces indigènes à la France, $M$ la sous-population des espèces menacées, et $D$ la sous-population des espèces en danger critique.
	On a donc la suite d'inclusions
		\[ D \subset M \subset E. \]
	
	Le texte donne les informations
		\begin{align*}
			|M| = 0,13 \cdot |E| && \text{et} && |D| = 0,23 \cdot |M|
		\end{align*}
	Par conséquent, $|D| = 0,23 \times 0,13 \cdot |E| \approx 0,03 \cdot |E|$.
	Donc les espèces en danger critique constituent $3\%$ des espèces.

	Les proportions sont ainsi multiplicatives. Attention à ne pas naïvement multiplier les pourcentages, car $13\times 23 \approx 300$.
}


\exe{}{
	Calculer sans calculatrice les valeurs suivantes.
	\begin{multicols}{3}
	\begin{enumerate}
		\item $75\%$ de $60$
		\item $60\%$ de $75$
		\item $72\%$ de $25$
		\item $68\%$ de $20$
		\item $125\%$ de $40$
		\item $40\%$ de $125$
	\end{enumerate}
	\end{multicols}
}{exe:1}{
	\begin{multicols}{2}
	\begin{enumerate}
		\item $\dfrac34 \cdot 60 = 3 \cdot \dfrac{60}4 = 3 \cdot 15 = 45$
		\item $45$
		\item $\dfrac14 \cdot 72 = 18$
		\item $\dfrac15 \cdot 68 = \dfrac{136}{10} = 13,6$
		\item $40 + \dfrac14 \cdot 40 = 50$
		\item $50$
	\end{enumerate}
	\end{multicols}
}

\exe{}{
	Approximer sans calculatrice les valeurs suivantes.
	\begin{multicols}{3}
	\begin{enumerate}
		\item $33\%$ de $150$
		\item $166\%$ de $180$
		\item $11\%$ de $90$
		\item $89\%$ de $81$
		\item $16,6\%$ de $18$
		\item $83,4\%$ de $36$
	\end{enumerate}
	\end{multicols}
}{exe:2}{
	\begin{multicols}{2}
	\begin{enumerate}
		\item $\approx \dfrac13 \cdot 150 = 50$
		\item $\approx 180 + \dfrac23 \cdot 180 = 180 + 120 = 300$
		\item $\approx \dfrac19 \cdot 90 = 10$
		\item $\approx 81 - \dfrac19 \cdot 81 = 81 - 9 = 72$
		\item $\approx \dfrac16 \cdot 18 = 3$
		\item $\approx 36 - \dfrac16 \cdot 36 = 36 - 6 = 30$
	\end{enumerate}
	\end{multicols}
}

\exe{}{
	On estime la biomasse totale des fourmis sur Terre à $12$ millions de tonnes.
	Ceci serait égal à $20\%$ de la biomasse humaine.
	
	Estimer la biomasse totale des humains sur Terre en tonnes.
}{exe:3}{
	On a la relation
		\[ \text{biomasse des fourmis} = 0,2 \times \text{biomasse humaine}. \]
	D'où
		\[ \text{biomasse humaine} = \dfrac{12 \times 10^6}{0,2} \text{T} = 60 \times 10^6 \text{T}.\] 

}

\exe{}{
	À quelle évolution correspond une augmentation de $20\%$ suivie d'une diminution de $20\%$ ?
}{exe:5}{
	Augmenter une quantité $N$ de $20\%$ correspond à la multiplier par $1,2$.
	Une diminution, elle, multiplie par $0,8$.
	
	La quantité finale est donné par 
		\[ 0,8 \cdot (1,2 \cdot N) = (0,8 \cdot 1,2) \cdot N = 0,96 \cdot N, \]
	qui correspond à une diminution de $4\%$.
}

\exe{, difficulty=1}{
	Si on augmente le prix d'un objet de $150\%$, quel rabais faut-il appliquer pour retrouver le prix initial de l'objet ?
}{exe:6}{
	Notons $P$ le prix initial de l'objet.
	Le prix augmenté vaut donc $1,5 \cdot P$.
	Pour retrouver $P$, il faut multiplier le prix augmenté par l'inverse de $1,5$, soit $1,5^{-1} = \dfrac23 \approx 0,666 = 66,6\%$.
	Ceci correspond à une diminution de $33,4\%$.
}

\exe{}{
	Considérons deux écharpes, l'une à $250$€ et l'autre à $360$€.
	\begin{enumerate}
		\item Quelle augmentation de prix faut-il appliquer à la première écharpe pour qu'elle ait le prix de la deuxième ?
		\item Quel rabais faut-il appliquer à la deuxième écharpe pour qu'elle ait le prix de la première ?
	\end{enumerate}
}{exe:tailleur}{
	\begin{enumerate}
		\item 
		On calcule l'évolution $\dfrac{360}{250}= 1,44 = 144\%$. Ainsi, la deuxième écharpe vaut $144\%$ du prix de la première : une augmentation de $44\%$ est nécessaire.
		\item 
		On calcule l'évolution $\dfrac{250}{360}\approx 0,7 = 70\%$. La première écharpe vaut environ $70\%$ du prix de la deuxième : une diminution de $30\%$ est nécessaire.
	\end{enumerate}
}

\exe{}{
        À quelle évolution correspond une augmentation de $20\%$ suivie d'une diminution de $20\%$ ?
}{exe:augm-dim-20}{
	Augmenter une quantité $N$ de $20\%$ correspond à la multiplier par $1,2$.
	Une diminution, elle, multiplie par $0,8$.
	
	La quantité finale est donné par 
		\[ 0,8 \cdot (1,2 \cdot N) = (0,8 \cdot 1,2) \cdot N = 0,96 \cdot N, \]
	qui correspond à une diminution de $4\%$.
}

\exe{}{
	Si on augmente le prix d'un objet de $150\%$, quel rabais faut-il appliquer pour retrouver le prix initial de l'objet ?
}{exe:reciproque}{
	Notons $P$ le prix initial de l'objet.
	Le prix augmenté vaut donc $1,5 \cdot P$.
	Pour retrouver $P$, il faut multiplier le prix augmenté par l'inverse de $1,5$, soit $1,5^{-1} = \dfrac23 \approx 0,666 = 66,6\%$.
	Ceci correspond à une diminution de $33,4\%$.
}


\exe{, difficulty=1}{
	Lors d'un payement par carte bancaire, une commission valant 1\% du montant de la transaction est versée comme frais bancaire.
	Ainsi, lorsqu'un débiteur paye 100€, le créditeur reçoit 99€, et 1€ est versé à la banque.
	\begin{enumerate}
		\item Pour un payement de 200€, quelle quantité est versée en frais bancaire ? quelle quantité le créditeur reçoit-il ?
		\item Répondre à la même question pour un payement de 202€.
		\item Obtient-on la quantité initiale après une augmentation de 1\% suivie d'une diminution de 1\% ?
		\item Quelle quantité le débiteur doit-il verser pour que le créditeur reçoive exactement 200€ ? Est-ce un nombre réel ? rationnel ? décimal ? entier ?
	\end{enumerate}
}{exe:caution-matthieu-payup}{
	TODO
}

\exe{, difficulty=1}{
	Un magasin propose une remise de 15\% sur tous les articles à partir de deux articles achetés.
	Un client ne souhaite acheter qu'un seul article à 80€.
	
	\begin{enumerate}
		\item
		Le client peut-il, en ajoutant un deuxième article, obtenir un panier à 75€ ?
		Si oui, donner le prix du deuxième article au centime près.
		
		\item
		Quel est le prix maximal du deuxième article qu'il peut se permettre d'ajouter à son panier pour que le prix final reste sous les 80€ ?
		Arrondir au centime près.
	\end{enumerate}
}{exe:linvosges-matthieu}{
	
	\begin{enumerate}
		\item
		Soit $P$ le prix du deuxième article.
		On souhaite que $P$ vérifie
			\[ 0,85 (80+P) = 75. \]
		En résolvant, on obtient $P \approx 8,23$€.
		
		\item
		L'équation pour $P$ est, cette fois-ci, $0,85 (80+P) = 80$, qui donne $P \approx 14,12$€.
	\end{enumerate}
}


\exe{, difficulty=1}{
	Sans calculatrice, exprimer les nombres suivants sous la forme $q^n$, où $q \in \N$ et $n\in\Z$ sont des entiers.
	\begin{multicols}{3}
	\begin{enumerate}[label=\roman*)]
		\item $10^3 \times 10^5$
		\item $\left(4^5\right)^2$
		\item $\dfrac{5^3}{5^3}$
		\item $1$
		\item $\dfrac{2^4}{2^7}$
		\item $\left(2^{-1}\right)^3$
		\item $\left(2^{3}\right)^{-1}$
		\item $\left(\dfrac{1}{7^2}\right)^6$
		\item $\dfrac{10^{12}}{10^{-12}}$
		\item $\dfrac{10^{-5}}{10^{6}}$
	\end{enumerate}
	\end{multicols}
}{exe:puissances}{
	TODO
}

\exe{}{
	On estime que, dans l'univers, il y a au moins
		\begin{itemize}
			\item $10^{11}$ galaxies ; que chacune contient
			\item $10^{11}$ étoiles ; dont la masse moyenne est de
			\item $10^{32}$ kilogrammes ; et que chaque gramme de matière contient
			\item $10^{24}$ atomes.
		\end{itemize}
	Estimer le nombre d'atomes dans l'univers observable à partir de ces données.
}{exe:atomes-univers}{
	TODO
}


\exe{, difficulty=1}{
	Montrer qu'on a environ $2^{10} \approx 10^3$. 	
	\begin{enumerate}
		\item En déduire approximativement l'ordre de grandeur de $2^{20}$ et le nombre de chiffres nécessaires pour l'écrire.
		\item En déduire approximativement l'ordre de grandeur de $2^{35}$ et le nombre de chiffres nécessaires pour l'écrire.
	\end{enumerate}
}{exe:grandeur-binaire}{
	TODO
}

\exe{}{
	Montrer que $10^{50} - 10^{30} = 10^{50} \bigl( 1 - 10^{-20} \bigr)$.
	Décrire le développement décimal de $1-10^{-20}$. De quel entier est-il très proche ?
	
	En déduire l'ordre de grandeur de $10^{50} - 10^{30}$.
}{exe:grandeur-soustraction}{
	TODO
}

\exe{}{
	Exprimer en écriture scientifique la valeur obtenue après une diminution de 60\% de $10^{60}$.
	Quel est son ordre de grandeur ?
}{exe:grandeur-évolution}{
	TODO
}


\exe{, difficulty=2}{
	Une jeune femme dépose $10$€ à la banque. Celle-ci lui promet un taux d'intérêt à l'année de $3\%$.
	Ainsi, après la première année, il y aura $1,03 \times 10 = 10,3$€ sur son compte.
	La deuxième année, il y aura $1,03 \times 10,35 = 10,609$€, etc...
	
	\begin{enumerate}
		\item Combien d'argent aura-t-elle après $5$ ans ?
		\item Combien d'argent aura-t-elle après $50$ ans ?
		\item Combien d'argent y aura-t-il sur son compte après $1000$ ans ?
	\end{enumerate}
	
	%\hfill
}{exe:evol4}{
	\begin{enumerate}
		\item On multiplie $5$ fois par $1,03$, ce qui donne $1,03^5 \times 10\approx 11,59$€.
		\item On multiplie $50$ fois par $1,03$, ce qui donne $1,03^{50} \times 10\approx 43,84$€.
		\item On multiplie $1000$ fois par $1,03$, ce qui donne $1,03^{1000} \times 10\approx 6,87 \times 10^{13}$€, c'est-à-dire environ $68$ billions d'euros ($1$ billion = $1000$ milliards).
	\end{enumerate}
}

\exe{, difficulty=2}{
    Considérons $p\geq 0$ une proportion et $100p$ le pourcentage associé.
    \begin{enumerate}
    \item À quelle évolution, en fonction de $p$, correspond une augmentation de $(100p)\%$ suivie d'une diminution de $(100p)\%$ ?
    \item Quel $p$ choisir pour trouver une diminution finale de $16\%$ ?
    \end{enumerate}
}{exe:sqrt-evol}{
    \begin{enumerate}
    \item Soit $N\geq0$ une quantité. Après une augmentation de $(100p)\%$ puis une diminution de $(100p)\%$,
    		la quantité est donnée par $(1-p) \cdot (1+p) \cdot N = (1-p^2) \cdot N$.
    		Ceci correspond à une diminution de $100\left(p^2\right) \%$.
    		
    		On comparera avec l'exercice 9, où $p=0,2$ et $p^2 = 0,04 = 4\%$.
    \item On pose l'égalité suivante
    		\[ 100p^2 = 16. \]
    	Remarquons que $16$ et $100$ sont tous les deux des carrés parfaits :
    		\[ p^2 = \dfrac{16}{100} = \left( \dfrac{4}{10} \right)^2, \]
	et donc $p = \dfrac{4}{10} = 40\%$, car $p\geq 0$.
	
	Vérification : au vu de la question 1, on calcule $0,4^2 = 0,16 = 16\%$.
    \end{enumerate}

}



%%%%%%%%%%%

\newpage
\fancyhead[C]{\textbf{Solutions}}
\shipoutAnswer

\end{document}

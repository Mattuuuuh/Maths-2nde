% DYSLEXIA SWITCH
\newif\ifdys
		
				% ENABLE or DISABLE font change
				% use XeLaTeX if true
				\dystrue
				\dysfalse


\ifdys

\documentclass[a4paper, 14pt]{extarticle}
\usepackage{amsmath,amsfonts,amsthm,amssymb,mathtools}

\tracinglostchars=3 % Report an error if a font does not have a symbol.
\usepackage{fontspec}
\usepackage{unicode-math}
\defaultfontfeatures{ Ligatures=TeX,
                      Scale=MatchUppercase }

\setmainfont{OpenDyslexic}[Scale=1.0]
\setmathfont{Fira Math} % Or maybe try KPMath-Sans?
\setmathfont{OpenDyslexic Italic}[range=it/{Latin,latin}]
\setmathfont{OpenDyslexic}[range=up/{Latin,latin,num}]

\else

\documentclass[a4paper, 12pt]{extarticle}

\usepackage[utf8x]{inputenc}
%fonts
\usepackage{amsmath,amsfonts,amsthm,amssymb,mathtools}
% comment below to default to computer modern
\usepackage{libertinus,libertinust1math}

\fi


\usepackage[french]{babel}
\usepackage[
a4paper,
margin=2cm,
nomarginpar,% We don't want any margin paragraphs
]{geometry}
\usepackage{icomma}

\usepackage{fancyhdr}
\usepackage{array}
\usepackage{hyperref}

\usepackage{multicol, enumerate}
\newcolumntype{P}[1]{>{\centering\arraybackslash}p{#1}}


\usepackage{stackengine}
\newcommand\xrowht[2][0]{\addstackgap[.5\dimexpr#2\relax]{\vphantom{#1}}}

% theorems

\theoremstyle{plain}
\newtheorem{theorem}{Th\'eor\`eme}
\newtheorem*{sol}{Solution}
\theoremstyle{definition}
\newtheorem{ex}{Exercice}
\newtheorem*{rpl}{Rappel}
\newtheorem{enigme}{Énigme}

% corps
\usepackage{calrsfs}
\newcommand{\C}{\mathcal{C}}
\newcommand{\R}{\mathbb{R}}
\newcommand{\Rnn}{\mathbb{R}^{2n}}
\newcommand{\Z}{\mathbb{Z}}
\newcommand{\N}{\mathbb{N}}
\newcommand{\Q}{\mathbb{Q}}

% variance
\newcommand{\Var}[1]{\text{Var}(#1)}

% domain
\newcommand{\D}{\mathcal{D}}


% date
\usepackage{advdate}
\AdvanceDate[0]


% plots
\usepackage{pgfplots}

% table line break
\usepackage{makecell}
%tablestuff
\def\arraystretch{2}
\setlength\tabcolsep{15pt}

%subfigures
\usepackage{subcaption}

\definecolor{myg}{RGB}{56, 140, 70}
\definecolor{myb}{RGB}{45, 111, 177}
\definecolor{myr}{RGB}{199, 68, 64}

% fake sections with no title to move around the merged pdf
\newcommand{\fakesection}[1]{%
  \par\refstepcounter{section}% Increase section counter
  \sectionmark{#1}% Add section mark (header)
  \addcontentsline{toc}{section}{\protect\numberline{\thesection}#1}% Add section to ToC
  % Add more content here, if needed.
}


% SOLUTION SWITCH
\newif\ifsolutions
				\solutionstrue
				%\solutionsfalse

\ifsolutions
	\newcommand{\exe}[2]{
		\begin{ex} #1  \end{ex}
		\begin{sol} #2 \end{sol}
	}
\else
	\newcommand{\exe}[2]{
		\begin{ex} #1  \end{ex}
	}
	
\fi


% tableaux var, signe
\usepackage{tkz-tab}


%pinfty minfty
\newcommand{\pinfty}{{+}\infty}
\newcommand{\minfty}{{-}\infty}

\begin{document}


\AdvanceDate[0]

\begin{document}
\pagestyle{fancy}
\fancyhead[L]{Seconde}
\fancyhead[C]{\textbf{Ensembles de nombres}}
\fancyhead[R]{\today}


\exe{}{
	Donner l'ensemble $A$ des entiers supérieurs ou égaux à 0 et inférieurs ou égaux à 7.
	
	Donner l'ensemble $B$ des éléments de $A$ qui sont pairs.
	
	A-t-on $A \subseteq B$ ? $B \subseteq A$ ?
}{exe:set1}{
	\begin{multicols}{2}
		$A = \bigset{ 0 ; 1 ; 2 ; 3 ; 4 ; 5 ; 6 ; 7}$
		
		$B = \bigset{ 0 ; 2 ; 4 ; 6 }$
	\end{multicols}
	On a $B \subseteq A$ car tous les éléments de $B$ sont des éléments de $A$, et $A \not\subseteq B$ car l'inverse est faux (7 appartient à $A$ mais pas à $B$, par exemple).
}

\exe{}{
	Décrire l'ensemble $A$ de l'exercice \ref{exe:set1} sous les formes $\bigset{ n \in \N \tq \dots }$ et $\bigset{ n \in \Z \tq \dots }$.
	
	Décrire l'ensemble $B$ de l'exercice \ref{exe:set1} sous la forme $\bigset{ n \in A \tq \dots }$.
}{exe:set2}{
	\begin{multicols}{2}
		$A = \bigset{ n \in \N \tq n \leq 7}$
		
		$A = \bigset{ n \in \Z \tq 0 \leq n \leq 7}$
		
		$B = \bigset{ n \in A \tq \text{$n$ est pair}}$
	\end{multicols}
}

\exe{}{
	Sans calculatrice, donner le développement décimal des fractions suivantes.
	\begin{multicols}{3}
	\begin{enumerate}[label=\roman*), leftmargin=60pt]
		\item $\dfrac1{10}$
		\item $\dfrac1{5}$
		\item $\dfrac3{10^{5}}$
		\item $\dfrac7{20}$
		\item $\dfrac{395}{50}$
		\item $\dfrac{11}{200}$
	\end{enumerate}
	\end{multicols}
}{exe:dev-decimaux}{
	\begin{multicols}{3}
	\begin{enumerate}[label=\roman*)]
		\item $\dfrac1{10} = 0,1.$
		\item $\dfrac1{5} = \dfrac2{10} = 0,2.$
		\item $\dfrac3{10^{5}} = 0,000~03.$
		\item $\dfrac7{20} = \dfrac{3,5}{10} = 0,35.$
		\item $\dfrac{395}{50} = \dfrac{790}{100} = 7,9.$
		\item $\dfrac{11}{200} = \dfrac{5,5}{100} = 0,055.$
	\end{enumerate}
	\end{multicols}
}

\exe{}{
	Sans calculatrice, écrire les nombres suivants en notation scientifique.
		
	\begin{multicols}{3}
	\begin{enumerate}[label=\alph*)]
		\item 201
		\item 10
		\item 123 400 000
		\item 0,8
		\item 0,000 327
		\item 0,009 000 1
	\end{enumerate}
	\end{multicols}
}{exe:notation-scientifique}{
	\begin{multicols}{3}
	\begin{enumerate}[label=\alph*)]
		\item 201 = $2,01 \times10^{2}$
		\item 10 = $1\times10^{1}$ (ou simplement 10...)
		\item 123 400 000 = $1,234 \times 10^{8}$
		\item 0,8 = $8\times10^{-1}$
		\item 0,000 327 = $3,27 \times 10^{-4}$
		\item 0,009 000 1 = $9,0001 \times 10^{-3}$
	\end{enumerate}
	\end{multicols}
}

\exe{, difficulty=1}{
	Montrer que $\frac19$ n'est pas décimal.
}{exe:nondecimal0}{
	Si $\dfrac19 = \frac{a}{10^n}$, alors $10^n = 9a$ et est multiple de $9$.
	Or l'entier d'avant est multiple de $9$ car
		\[ 10^n - 1 = \underbrace{99{\dots}99}_{\text{n fois}} = 9 \times \underbrace{11{\dots}11}_{\text{n fois}} . \]
	$10^n$ ne peut donc pas être multiple de $9$, une contradiction ! \Large\Lightning
}

\exe{, difficulty=2}{
	Montrer que si une certaine puissance de 10 est multiple entier de $k$, alors $\frac1k$ est décimal.
}{exe:nondecimal2}{
	Si $10^n$ est multiple entier de $k$, alors il existe un $a \in \Z$ tel que
		\[ ak = 10^n. \]
	Ainsi $\frac1k = \frac{a}{10^n} \in \DD$.
}

\exe{}{
	Montrer que les nombres suivants sont rationnels en les exprimant sous forme de fraction d'entiers.
	\[
	\begin{aligned}
		A &= 0,666{\dots} \\
		B &= 1,666{\dots} \\
	\end{aligned}
	\hspace{5cm}
	\begin{aligned}
		C &= 0,121212{\dots} \\
		D &= 0,34777{\dots} \\
	\end{aligned}
	\]
	\[
	E = 0,123456789123456789123{\dots} \text{ (nombre d'Amandine) }
	\]
}{exe:dev-to-fraction}{
	\begin{enumerate}[label=\Alph*.]
		\item 
		La relation $10A = 6 + A$ donne $A = \frac23$.
		\item
		La relation $10B = 15 + B$ donne $B = \frac{15}{9} = \frac53$.
		On aurait aussi pû utiliser que $B = 1+A$.
		\item
		La relation $100C = 12 + C$ donne $C = \frac{12}{99} = \frac{4}{33}$.
		\item
		En posant $\tilde{E} = 100E = 34,777{\dots}$, on obtient $10\tilde{E} = 313 + \tilde{E}$, d'où $\tilde{E} = \frac{313}{9}$, et donc $E = \frac{1}{100}E' = \frac{313}{900}$.
		\item 
		La relation $10^9 E = 123~456~789 + E$ donne $E = \frac{123~456~789}{999~999~999}$.
	\end{enumerate}
}

\exe{, difficulty=1}{
	Déterminer le développement décimal de $\frac1{11}$ sans calculatrice.
}{exe:111-dec}{
	Posons 
		\[ \dfrac{1}{11} = 0,n_1 n_2 n_3 n_4 \dots. \]
	Par multplication par 10 successives et étude des parties décimales, on obtient $n_1 = 0$, $n_2=9$, et $\frac1{11} = 0,n_3n_4n_5\dots$.
	D'où
		\[ \dfrac1{11} = 0,09~09~09~09~\cdots. \]
}

\exe{, difficulty=2}{
	Donner un nombre irrationnel.
}{exe:nombre-R}{
	Voir la démonstration du théorème 1.20 des notes de cours.
}

\exe{}{
	Vrai ou faux ? L'encadrement 
	
	\begin{tabular}{c c c}
		\hspace{10cm} & Vrai & Faux \\
		$2,6 < 2,6457 < 2,8$ est à $10^{-1}$ près & $\square$ & $\square$  \\
		$3,14 < 3,1415 < 3,15$ est à $10^{-2}$ près & $\square$ & $\square$  \\
		$-4,474 < -4,4735 < -4,473$ est à $10^{-3}$ près & $\square$ & $\square$  \\
		$3,3 \times 10^{-4} < 3,3931 \times 10^{-4} < 3,4 \times 10^{-4}$ est à $10^{-5}$ près & $\square$ & $\square$  \\
	\end{tabular}
}{exe:v-f-encadrement}{
	L'amplitude d'un encadrement $a < x < b$ est donnée par $b-a$.
	La première proposition est fausse car l'amplitude est de $0,2 = 2 \times 10^{-1}$.
	Pour la quatrième proposition, on utilise que $0,1 \times 10^{-4} = 10^{-1} \times 10^{-4} = 10^{-5}$.

	\begin{tabular}{c c c}
		\hspace{10cm} & Vrai & Faux \\
		$2,6 < 2,6457 < 2,8$ est à $10^{-1}$ près & $\times$ & \checkmark  \\
		$3,14 < 3,1415 < 3,15$ est à $10^{-2}$ près & \checkmark & $\times$  \\
		$-4,474 < -4,4735 < -4,473$ est à $10^{-3}$ près & \checkmark & $\times$  \\
		$3,3 \times 10^{-4} < 3,3931 \times 10^{-4} < 3,4 \times 10^{-4}$ est à $10^{-5}$ près & \checkmark & $\times$  \\
	\end{tabular}
}

\exe{}{
	Encadrer les nombres suivants à l'amplitude demandée.
	\begin{multicols}{2}
	\begin{enumerate}[label=\alph*)]
		\item $3,605 \times 10^{-2}$ à $10^{-4}$ près.
		\item $9~854,698 \times 10^3$ à $10^4$ près.
		\item $-31,45$ à $10^{-1}$ près.
		\item $-0,0125$ à $10^{-4}$ près.
	\end{enumerate}
	\end{multicols}
}{exe:encadrement}{
	\begin{multicols}{2}
	\begin{enumerate}[label=\alph*)]
		\item $3,600 \times 10^{-2} < 3,605 \times 10^{-2} < 3,61 \times 10^{-2}$
		\item $9~850 \times 10^3 < 9~854,698 \times 10^3 < 9~860 \times 10^3$
		\item $-31,5 < -31,45 < -31,4$
		\item $-0,01255 < -0,0125 < -0,01245$
	\end{enumerate}
	\end{multicols}
}

\newpage
\fancyhead[C]{\textbf{Exercices d'entraînement}}


\exe{}{
	Donner l'ensemble $A$ des entiers relatifs supérieurs ou égaux à -3 et inférieurs ou égaux à 2.
	
	Donner l'ensemble $B$ des éléments de $A$ qui sont positifs ou nuls.
	
	A-t-on $A \subseteq B$ ? $B \subseteq A$ ?
}{exe:set1E}{
	\begin{multicols}{2}
		$A = \bigset{ -3 ; -2 ; -1 ; 0 ; 1 ; 2}$
		
		$B = \bigset{ 0 ; 1 ; 2 }$
	\end{multicols}
	On a $B \subseteq A$ car tous les éléments de $B$ sont des éléments de $A$, et $A \not\subseteq B$ car l'inverse est faux (7 appartient à $A$ mais pas à $B$, par exemple).
}

\exe{}{
	Décrire l'ensemble $A$ de l'exercice \ref{exe:set1} sous les formes $\bigset{ n \in \Z \tq \dots }$.
	
	Décrire l'ensemble $B$ de l'exercice \ref{exe:set1} sous la forme $\bigset{ n \in A \tq \dots }$.
}{exe:set2E}{
	\begin{multicols}{2}
		$A = \bigset{ n \in \Z \tq -3 \leq n \leq 2}$
		
		$B = \bigset{ n \in A \tq n\geq0 }$
	\end{multicols}
}

\exe{}{
	Sans calculatrice, donner le développement décimal des fractions suivantes.
	\begin{multicols}{3}
	\begin{enumerate}[label=\roman*), leftmargin=60pt]
		\item $\dfrac1{100}$
		\item $\dfrac3{5}$
		\item $\dfrac2{10^{1}}$
		\item $\dfrac7{2}$
		\item $\dfrac{7}{20}$
		\item $\dfrac{31}{200}$
	\end{enumerate}
	\end{multicols}
}{exe:dev-decimauxE}{
	\begin{multicols}{3}
	\begin{enumerate}[label=\roman*)]
		\item $\dfrac1{100} = 0,01$
		\item $\dfrac3{5} = 0,6$
		\item $\dfrac2{10^{1}} = 0,2$
		\item $\dfrac7{2} = 3,5$
		\item $\dfrac{7}{20} = 0,35$
		\item $\dfrac{31}{200} = 0,062$
	\end{enumerate}
	\end{multicols}
}

\exe{}{
	Sans calculatrice, écrire les nombres suivants en notation scientifique.
		
	\begin{multicols}{3}
	\begin{enumerate}[label=\alph*)]
		\item 300
		\item 14
		\item 5 034 000
		\item 0,801
		\item 0,031 415
		\item 70,000 03
	\end{enumerate}
	\end{multicols}
}{exe:notation-scientifiqueE}{
	\begin{multicols}{3}
	\begin{enumerate}[label=\alph*)]
		\item $300 = 3 \times 10^2$
		\item $14 = 1,4\times10^1$ 
		\item $5~034~000 = 5,034 \times 10^6$
		\item $0,801 = 8,01 \times 10^{-1}$
		\item $0,031~415 = 3,1415\times10^{-2}$
		\item $70,000~03 = 7,000~003\times10^1$
	\end{enumerate}
	\end{multicols}
}

\exe{}{
	Montrer, à l'aide du cours, que $\frac1{30}$ n'est pas décimal.
}{exe:nondecimal0E}{
	Comme $\frac1{30} = \frac1{10} \times \frac13$, son développement décimal est celui de $\frac13$ décalé d'une décimale vers la droite.
	D'après le cours, son développement décimal est infini.
}

\exe{, difficulty=2}{
	Montrer que $\frac1{11}$ n'est pas décimal.
}{exe:nondecimal2E}{
	D'après l'exercice \ref{exe:nondecimal2}, il s'agit de montrer qu'aucune puissance de 10 n'est multiple entier de 11.
	
	Or, d'une part, 99, 9999, 999999, ... sont multiples de 11, donc l'entier avant les puissances paires de 10 sont multiples de 11.
	Les puissances paires de 10 ne sont donc pas multiples de 11.
	
	Et d'autre part, 11, 1001, 100001, ... sont multiples de 11 (car 0, 990, 99990 le sont).
	L'entier après chaque puissance impaire de 10 est donc multiple de 11, et celle-ci ne peut donc pas être multiple de 11.
}

\exe{}{
	Montrer que les nombres suivants sont rationnels en les exprimant sous forme de fraction d'entiers.
	\begin{align*}
		A = 0,123123123{\dots} &&
		B = 0,9999{\dots}
	\end{align*}
}{exe:dev-to-fractionE}{
	\begin{enumerate}[label=\Alph*.]
		\item
		La relation $1000A = 123 + A$ donne $A = \frac{123}{999} = \frac{41}{333}$.
		\item 
		La relation $10B = 9 + B$ donne $9B = 9$ et $B=1$.
	\end{enumerate}
}

\exe{, difficulty=1}{
	Déterminer le développement décimal de $\frac{12}{22}$ sans calculatrice.
}{exe:113-dec}{
	Posons 
		\[ \dfrac{12}{22} = 0,n_1 n_2 n_3 n_4 \dots. \]
	Par multplication par 10 successives et étude des parties décimales, on obtient $n_1 = 5$, $n_2=4$, et $\frac{12}{22} = 0,n_3 n_4 n_5\dots$.
	D'où
		\[ \dfrac{12}{22} = 0,54~54~54~54~\cdots. \]
}

%%%%%%%%%%%%

\newpage
\fancyhead[C]{\textbf{Solutions}}
\shipoutAnswer

\end{document}

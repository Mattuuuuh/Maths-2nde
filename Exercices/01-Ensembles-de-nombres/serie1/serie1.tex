% DYSLEXIA SWITCH
\newif\ifdys
		
				% ENABLE or DISABLE font change
				% use XeLaTeX if true
				\dystrue
				\dysfalse


\ifdys

\documentclass[a4paper, 14pt]{extarticle}
\usepackage{amsmath,amsfonts,amsthm,amssymb,mathtools}

\tracinglostchars=3 % Report an error if a font does not have a symbol.
\usepackage{fontspec}
\usepackage{unicode-math}
\defaultfontfeatures{ Ligatures=TeX,
                      Scale=MatchUppercase }

\setmainfont{OpenDyslexic}[Scale=1.0]
\setmathfont{Fira Math} % Or maybe try KPMath-Sans?
\setmathfont{OpenDyslexic Italic}[range=it/{Latin,latin}]
\setmathfont{OpenDyslexic}[range=up/{Latin,latin,num}]

\else

\documentclass[a4paper, 12pt]{extarticle}

\usepackage[utf8x]{inputenc}
%fonts
\usepackage{amsmath,amsfonts,amsthm,amssymb,mathtools}
% comment below to default to computer modern
\usepackage{libertinus,libertinust1math}

\fi


\usepackage[french]{babel}
\usepackage[
a4paper,
margin=2cm,
nomarginpar,% We don't want any margin paragraphs
]{geometry}
\usepackage{icomma}

\usepackage{fancyhdr}
\usepackage{array}
\usepackage{hyperref}

\usepackage{multicol, enumerate}
\newcolumntype{P}[1]{>{\centering\arraybackslash}p{#1}}


\usepackage{stackengine}
\newcommand\xrowht[2][0]{\addstackgap[.5\dimexpr#2\relax]{\vphantom{#1}}}

% theorems

\theoremstyle{plain}
\newtheorem{theorem}{Th\'eor\`eme}
\newtheorem*{sol}{Solution}
\theoremstyle{definition}
\newtheorem{ex}{Exercice}
\newtheorem*{rpl}{Rappel}
\newtheorem{enigme}{Énigme}

% corps
\usepackage{calrsfs}
\newcommand{\C}{\mathcal{C}}
\newcommand{\R}{\mathbb{R}}
\newcommand{\Rnn}{\mathbb{R}^{2n}}
\newcommand{\Z}{\mathbb{Z}}
\newcommand{\N}{\mathbb{N}}
\newcommand{\Q}{\mathbb{Q}}

% variance
\newcommand{\Var}[1]{\text{Var}(#1)}

% domain
\newcommand{\D}{\mathcal{D}}


% date
\usepackage{advdate}
\AdvanceDate[0]


% plots
\usepackage{pgfplots}

% table line break
\usepackage{makecell}
%tablestuff
\def\arraystretch{2}
\setlength\tabcolsep{15pt}

%subfigures
\usepackage{subcaption}

\definecolor{myg}{RGB}{56, 140, 70}
\definecolor{myb}{RGB}{45, 111, 177}
\definecolor{myr}{RGB}{199, 68, 64}

% fake sections with no title to move around the merged pdf
\newcommand{\fakesection}[1]{%
  \par\refstepcounter{section}% Increase section counter
  \sectionmark{#1}% Add section mark (header)
  \addcontentsline{toc}{section}{\protect\numberline{\thesection}#1}% Add section to ToC
  % Add more content here, if needed.
}


% SOLUTION SWITCH
\newif\ifsolutions
				\solutionstrue
				%\solutionsfalse

\ifsolutions
	\newcommand{\exe}[2]{
		\begin{ex} #1  \end{ex}
		\begin{sol} #2 \end{sol}
	}
\else
	\newcommand{\exe}[2]{
		\begin{ex} #1  \end{ex}
	}
	
\fi


% tableaux var, signe
\usepackage{tkz-tab}


%pinfty minfty
\newcommand{\pinfty}{{+}\infty}
\newcommand{\minfty}{{-}\infty}

\begin{document}


\AdvanceDate[0]

\begin{document}
\pagestyle{fancy}
\fancyhead[L]{Seconde}
\fancyhead[C]{\textbf{Ensembles de nombres}}
\fancyhead[R]{\today}


\exe{}{
	Donner l'ensemble $A$ des entiers supérieurs ou égaux à 0 et inférieurs ou égaux à 7.
	
	Donner l'ensemble $B$ des éléments de $A$ qui sont pairs.
}{exe:set1}{
	\begin{multicols}{2}
		$A = \bigset{ 0 ; 1 ; 2 ; 3 ; 4 ; 5 ; 6 ; 7}$
		
		$B = \bigset{ 0 ; 2 ; 4 ; 6 }$
	\end{multicols}
}

\exe{}{
	Décrire l'ensemble $A$ de l'exercice \ref{exe:set1} sous les formes $\bigset{ n \in \N \tq \dots }$ et $\bigset{ n \in \Z \tq \dots }$.
	
	Décrire l'ensemble $B$ de l'exercice \ref{exe:set1} sous la forme $\bigset{ n \in A \tq \dots }$.
}{exe:set2}{
	\begin{multicols}{2}
		$A = \bigset{ n \in \N \tq n \leq 7}$
		
		$A = \bigset{ n \in \Z \tq 0 \leq n \leq 7}$
		
		$B = \bigset{ n \in A \tq \text{$n$ est pair}}$
	\end{multicols}
}

\exe{}{
	Sans calculatrice, donner le développement décimal des fractions suivantes.
	\begin{multicols}{3}
	\begin{enumerate}[label=\roman*), leftmargin=60pt]
		\item $\dfrac1{10}$
		\item $\dfrac1{5}$
		\item $\dfrac3{10^{5}}$
		\item $\dfrac7{20}$
		\item $\dfrac{395}{50}$
		\item $\dfrac{11}{200}$
	\end{enumerate}
	\end{multicols}
}{exe:dev-decimaux}{
	\begin{multicols}{3}
	\begin{enumerate}[label=\roman*)]
		\item $\dfrac1{10} = 0,1.$
		\item $\dfrac1{5} = \dfrac2{10} = 0,2.$
		\item $\dfrac3{10^{5}} = 0,00003.$
		\item $\dfrac7{20} = \dfrac{3,5}{10} = 0,35.$
		\item $\dfrac{395}{50} = \dfrac{790}{100} = 7,9.$
		\item $\dfrac{11}{200} = \dfrac{5,5}{100} = 0,055.$
	\end{enumerate}
	\end{multicols}
}

\exe{}{
	Sans calculatrice, écrire les nombres suivants en notation scientifique.
		
	\begin{multicols}{3}
	\begin{enumerate}[label=\alph*)]
		\item 201
		\item 10
		\item 123 400 000
		\item 0,8
		\item 0,000 327
		\item 0,009 000 1
	\end{enumerate}
	\end{multicols}
}{exe:notation-scientifique}{
	\begin{multicols}{3}
	\begin{enumerate}[label=\alph*)]
		\item 201 = $2,01 \times10^{2}$
		\item 10 = $1\times10^{1}$ (ou simplement 10...)
		\item 123 400 000 = $1,234 \times 10^{8}$
		\item 0,8 = $8\times10^{-1}$
		\item 0,000 327 = $3,27 \times 10^{-4}$
		\item 0,009 000 1 = $9,0001 \times 10^{-3}$
	\end{enumerate}
	\end{multicols}
}

\exe{, difficulty=1}{
	Montrer que $\frac19$ n'est pas décimal.
}{exe:nondecimal0}{
	Si $\dfrac19 = \frac{a}{10^n}$, alors $10^n = 9a$ et est multiple de $9$.
	Or l'entier d'avant est multiple de $9$ car
		\[ 10^n - 1 = \underbrace{99{\dots}99}_{\text{n fois}} = 9 \times \underbrace{11{\dots}11}_{\text{n fois}} . \]
	$10^n$ ne peut donc pas être multiple de $9$, une contradiction ! \Large\Lightning
}

\exe{, difficulty=2}{
	Montrer que si une certaine puissance de 10 est multiple entier de $k$, alors $\frac1k$ est décimal.
}{exe:nondecimal2}{
	Si $10^n$ est multiple entier de $k$, alors il existe un $a \in \Z$ tel que
		\[ ak = 10^n. \]
	Ainsi $\frac1k = \frac{a}{10^n} \in \DD$.
}

\exe{}{
	Montrer que les nombres suivants sont rationnels en les exprimant sous forme de fraction d'entiers.
	\[
	\begin{aligned}
		A &= 0,666{\dots} \\
		B &= 9,999{\dots} \\
		C &= 0,121212{\dots} \\
	\end{aligned}
	\hspace{5cm}
	\begin{aligned}
		D &= 1,666{\dots} \\
		E &= 0,34777{\dots} \\
		F &= 0,123123123{\dots}
	\end{aligned}
	\]
	\[
	G = 0,123456789123456789123{\dots} \text{ (nombre d'Amandine) }
	\]
}{exe:dev-to-fraction}{
	\begin{enumerate}[label=\Alph*.]
		\item 
		La relation $10A = 6 + A$ donne $A = \frac23$.
		\item
		La relation $10B = 90 + B$ donne $A = 10$. Il s'avère donc que $10$ admet deux écritures décimales différentes.
		Pour se convaincre encore que $B=10$, on peut remarquer la chose suivante : il n'existe pas de nombre entre $B$ et $10$ strictement différent des deux. Or il existe toujours un nombre situé entre deux nombres distincts (leur moyenne, par exemple).
		\item
		La relation $100C = 12 + C$ donne $C = \frac{12}{99} = \frac{4}{33}$.
		\item
		La relation $10D = 15 + D$ donne $D = \frac{15}{9} = \frac53$.
		\item
		En posant $E' = 100E = 34,777{\dots}$, on obtient $10E' = 313 + E'$, d'où $E' = \frac{313}{9}$, et donc $E = \frac{1}{100}E' = \frac{313}{900}$.
		\item
		La relation $1000F = 123 + F$ donne $F = \frac{123}{999} = \frac{41}{333}$.
		\item 
		La relation $10^9 G = 123456789 + G$ donne $G = \frac{123456789}{999999999}$.
	\end{enumerate}
}

\exe{, difficulty=1}{
	Déterminer le développement décimal de $\frac1{11}$ sans calculatrice.
}{exe:111-dec}{
	Par multplication par 10 successives et étude des parties décimales, on obtient $n_1 = 0$, $\dfrac{100}{11} = 9 + \dfrac1{11}$ et $n_2=9$.
	D'où
		\[ \dfrac1{11} = 0,09~09~09~09~\cdots. \]
}

\exe{, difficulty=2}{
	Donner un nombre irrationnel.
}{exe:nombre-R}{
	Voir la démonstration du théorème 1.20 des notes de cours.
}

\exe{}{
	Vrai ou faux ? L'encadrement 
	
	\begin{tabular}{c c c}
		\hspace{10cm} & Vrai & Faux \\
		$2,6 < 2,6457 < 2,8$ est à $10^{-1}$ près & $\square$ & $\square$  \\
		$3,14 < 3,1415 < 3,15$ est à $10^{-2}$ près & $\square$ & $\square$  \\
		$-4,474 < -4,4735 < -4,473$ est à $10^{-3}$ près & $\square$ & $\square$  \\
		$3,3 \times 10^{-4} < 3,3931 \times 10^{-4} < 3,4 \times 10^{-4}$ est à $10^{-5}$ près & $\square$ & $\square$  \\
	\end{tabular}
}{exe:v-f-encadrement}{
	L'amplitude d'un encadrement $a < x < b$ est donnée par $b-a$.
	La première proposition est fausse car l'amplitude est de $0,2 = 2 \times 10^{-1}$.
	Pour la quatrième proposition, on utilise que $0,1 \times 10^{-4} = 10^{-1} \times 10^{-4} = 10^{-5}$.

	\begin{tabular}{c c c}
		\hspace{10cm} & Vrai & Faux \\
		$2,6 < 2,6457 < 2,8$ est à $10^{-1}$ près & $\times$ & \checkmark  \\
		$3,14 < 3,1415 < 3,15$ est à $10^{-2}$ près & \checkmark & $\times$  \\
		$-4,474 < -4,4735 < -4,473$ est à $10^{-3}$ près & \checkmark & $\times$  \\
		$3,3 \times 10^{-4} < 3,3931 \times 10^{-4} < 3,4 \times 10^{-4}$ est à $10^{-5}$ près & \checkmark & $\times$  \\
	\end{tabular}
}

\exe{}{
	Encadrer les nombres suivants à l'amplitude demandée.
	\begin{multicols}{2}
	\begin{enumerate}[label=\alph*)]
		\item $3,605 \times 10^{-2}$ à $10^{-4}$ près.
		\item $9~854,698 \times 10^3$ à $10^4$ près.
		\item $-31,45$ à $10^{-1}$ près.
		\item $-0,0125$ à $10^{-4}$ près.
	\end{enumerate}
	\end{multicols}
}{exe:encadrement}{
	\begin{multicols}{2}
	\begin{enumerate}[label=\alph*)]
		\item $3,600 \times 10^{-2} < 3,605 \times 10^{-2} < 3,61 \times 10^{-2}$
		\item $9~850 \times 10^3 < 9~854,698 \times 10^3 < 9~860 \times 10^3$
		\item $-31,5 < -31,45 < -31,4$
		\item $-0,01255 < -0,0125 < -0,01245$
	\end{enumerate}
	\end{multicols}
}


%%%%%%%%%%%%

\newpage
\fancyhead[C]{\textbf{Solutions}}
\shipoutAnswer

\end{document}

% DYSLEXIA SWITCH
\newif\ifdys
		
				% ENABLE or DISABLE font change
				% use XeLaTeX if true
				\dystrue
				\dysfalse


\ifdys

\documentclass[a4paper, 14pt]{extarticle}
\usepackage{amsmath,amsfonts,amsthm,amssymb,mathtools}

\tracinglostchars=3 % Report an error if a font does not have a symbol.
\usepackage{fontspec}
\usepackage{unicode-math}
\defaultfontfeatures{ Ligatures=TeX,
                      Scale=MatchUppercase }

\setmainfont{OpenDyslexic}[Scale=1.0]
\setmathfont{Fira Math} % Or maybe try KPMath-Sans?
\setmathfont{OpenDyslexic Italic}[range=it/{Latin,latin}]
\setmathfont{OpenDyslexic}[range=up/{Latin,latin,num}]

\else

\documentclass[a4paper, 12pt]{extarticle}

\usepackage[utf8x]{inputenc}
%fonts
\usepackage{amsmath,amsfonts,amsthm,amssymb,mathtools}
% comment below to default to computer modern
\usepackage{libertinus,libertinust1math}

\fi


\usepackage[french]{babel}
\usepackage[
a4paper,
margin=2cm,
nomarginpar,% We don't want any margin paragraphs
]{geometry}
\usepackage{icomma}

\usepackage{fancyhdr}
\usepackage{array}
\usepackage{hyperref}

\usepackage{multicol, enumerate}
\newcolumntype{P}[1]{>{\centering\arraybackslash}p{#1}}


\usepackage{stackengine}
\newcommand\xrowht[2][0]{\addstackgap[.5\dimexpr#2\relax]{\vphantom{#1}}}

% theorems

\theoremstyle{plain}
\newtheorem{theorem}{Th\'eor\`eme}
\newtheorem*{sol}{Solution}
\theoremstyle{definition}
\newtheorem{ex}{Exercice}
\newtheorem*{rpl}{Rappel}
\newtheorem{enigme}{Énigme}

% corps
\usepackage{calrsfs}
\newcommand{\C}{\mathcal{C}}
\newcommand{\R}{\mathbb{R}}
\newcommand{\Rnn}{\mathbb{R}^{2n}}
\newcommand{\Z}{\mathbb{Z}}
\newcommand{\N}{\mathbb{N}}
\newcommand{\Q}{\mathbb{Q}}

% variance
\newcommand{\Var}[1]{\text{Var}(#1)}

% domain
\newcommand{\D}{\mathcal{D}}


% date
\usepackage{advdate}
\AdvanceDate[0]


% plots
\usepackage{pgfplots}

% table line break
\usepackage{makecell}
%tablestuff
\def\arraystretch{2}
\setlength\tabcolsep{15pt}

%subfigures
\usepackage{subcaption}

\definecolor{myg}{RGB}{56, 140, 70}
\definecolor{myb}{RGB}{45, 111, 177}
\definecolor{myr}{RGB}{199, 68, 64}

% fake sections with no title to move around the merged pdf
\newcommand{\fakesection}[1]{%
  \par\refstepcounter{section}% Increase section counter
  \sectionmark{#1}% Add section mark (header)
  \addcontentsline{toc}{section}{\protect\numberline{\thesection}#1}% Add section to ToC
  % Add more content here, if needed.
}


% SOLUTION SWITCH
\newif\ifsolutions
				\solutionstrue
				%\solutionsfalse

\ifsolutions
	\newcommand{\exe}[2]{
		\begin{ex} #1  \end{ex}
		\begin{sol} #2 \end{sol}
	}
\else
	\newcommand{\exe}[2]{
		\begin{ex} #1  \end{ex}
	}
	
\fi


% tableaux var, signe
\usepackage{tkz-tab}


%pinfty minfty
\newcommand{\pinfty}{{+}\infty}
\newcommand{\minfty}{{-}\infty}

\begin{document}


\begin{document}
\pagestyle{fancy}
\fancyhead[L]{Seconde}
\fancyhead[C]{\textbf{Ensembles de nombres}}
\fancyhead[R]{\AdvanceDate[0]\today}

\section*{Appartenance à des ensembles}


\def\arraystretch{2}
\setlength\tabcolsep{5pt}

\begin{center}
\begin{tabular}{ | P{.15\linewidth} | P{.15\linewidth} | P{.15\linewidth} | P{.1\linewidth} | P{.1\linewidth} | P{.1\linewidth} | P{.1\linewidth} |  } 
  \hline\xrowht{10pt}
  $\in$ & $\{ 1 ; 3 ; -1\}$ & $\bigl\{ \frac13 ; \pi ; \pi + 2 \bigr\}$  & $\N$ & $\Z$ & $\DD$ & $\Q$ \\ \hline \xrowht{20pt}
  1 & & & & & & \\ \hline\xrowht{20pt}
  -1 & & & & & & \\ \hline\xrowht{20pt}
  6,3 & & & & & & \\ \hline\xrowht{20pt}
  $-1,4 \times 10 ^{80}$ & & & & & & \\ \hline\xrowht{20pt}
  $6,02 \times 10 ^{-23}$ & & & & & & \\ \hline\xrowht{20pt} 
  $\frac13$ & & & & & & \\  \hline\xrowht{20pt}
  $\frac{12}4$ & & & & & & \\ \hline \xrowht{20pt}
  $\sqrt{9}$ & & & & & & \\ \hline \xrowht{20pt}
  %$\sqrt{2}+1$ & & & & \\ \hline\xrowht{20pt}
  $\pi$ & & & & & & \\ \hline
\end{tabular}
\end{center}

\section*{Inclusions d'ensembles}

Un ensemble $A$ est inclus dans un ensemble $B$, noté $A \subseteq B$, dès que \phantom{tous les éléments de $A$ appartiennent à $B$.}

\vfill

\begin{center}
\begin{tabular}{ | P{.2\linewidth} | P{.15\linewidth} | P{.1\linewidth} | P{.1\linewidth} | P{.1\linewidth} | P{.1\linewidth} |  } 
  \hline\xrowht{10pt}
  $\subseteq $ & $\{ 1 ; \pi ; 3 ; 2\}$ & $\N$ & $\Z$ & $\DD$ & $\Q$  \\ \hline \xrowht{20pt}
  $\{ 1 ; -4 \}$ & & & & & \\ \hline\xrowht{20pt}
  $\{ 1 \}$ & & & & & \\ \hline\xrowht{20pt}
  $\bigl\{ -\frac12 ; 3 \bigr\}$ & & & & & \\ \hline\xrowht{20pt}
  $\bigl\{ 1, 2 ; \frac62 ; -\frac13 \bigr\}$ & & & & & \\ \hline\xrowht{20pt}
  $ \Z$ & & & & & \\ \hline\xrowht{20pt}
  $\{ \pi \} $ & & & & & \\ \hline
  
\end{tabular}
\end{center}

\vfill


\newpage
\fancyhead[C]{\textbf{Solutions}}


\def\arraystretch{2}
\setlength\tabcolsep{5pt}

\begin{center}
\begin{tabular}{ | P{.15\linewidth} | P{.15\linewidth} | P{.15\linewidth} | P{.1\linewidth} | P{.1\linewidth} | P{.1\linewidth} | P{.1\linewidth} |  } 
  \hline\xrowht{10pt}
  $\in$ & $\{ 1 ; 3 ; -1\}$ & $\bigl\{ \frac13 ; \pi ; \pi + 2 \bigr\}$  & $\N$ & $\Z$ & $\DD$ & $\Q$ \\ \hline \xrowht{20pt}
  1 & \checkmark & & \checkmark &  \checkmark & \checkmark & \checkmark \\ \hline\xrowht{20pt}
  -1 & \checkmark & & & \checkmark & \checkmark & \checkmark \\ \hline\xrowht{20pt}
  6,3 & & & & & \checkmark & \checkmark \\ \hline\xrowht{20pt}
  $-1,4 \times 10 ^{80}$ & & & & \checkmark & \checkmark & \checkmark \\ \hline\xrowht{20pt}
  $6,02 \times 10 ^{-23}$ & & & & & \checkmark & \checkmark \\ \hline\xrowht{20pt} 
  $\frac13$ & & \checkmark & & & & \checkmark \\  \hline\xrowht{20pt}
  $\frac{12}4$ & \checkmark & & \checkmark & \checkmark & \checkmark & \checkmark \\ \hline \xrowht{20pt}
  $\sqrt{9}$ & \checkmark & & \checkmark & \checkmark & \checkmark & \checkmark \\ \hline \xrowht{20pt}
  %$\sqrt{2}+1$ & & & & \\ \hline\xrowht{20pt}
  $\pi$ & & \checkmark & & & & \\ \hline
\end{tabular}
\end{center}

\section*{Inclusions d'ensembles}

Un ensemble $A$ est inclus dans un ensemble $B$, noté $A \subseteq B$, dès que tous les éléments de $A$ appartiennent à $B$.

\vfill

\begin{center}
\begin{tabular}{ | P{.2\linewidth} | P{.15\linewidth} | P{.1\linewidth} | P{.1\linewidth} | P{.1\linewidth} | P{.1\linewidth} |  } 
  \hline\xrowht{10pt}
  $\subseteq $ & $\{ 1 ; \pi ; 3 ; 2\}$ & $\N$ & $\Z$ & $\DD$ & $\Q$  \\ \hline \xrowht{20pt}
  $\{ 1 ; -4 \}$ & & & \checkmark & \checkmark & \checkmark \\ \hline\xrowht{20pt}
  $\{ 1 \}$ & \checkmark & \checkmark & \checkmark & \checkmark & \checkmark \\ \hline\xrowht{20pt}
  $\bigl\{ -\frac12 ; 3 \bigr\}$ & & & & \checkmark & \checkmark \\ \hline\xrowht{20pt}
  $\bigl\{ 1, 2 ; \frac62 ; -\frac13 \bigr\}$ & & & & & \checkmark \\ \hline\xrowht{20pt}
  $ \Z$ & & \checkmark & \checkmark & \checkmark & \checkmark \\ \hline\xrowht{20pt}
  $\{ \pi \} $ & \checkmark & & & & \\ \hline
  
\end{tabular}
\end{center}

\vfill

\end{document}
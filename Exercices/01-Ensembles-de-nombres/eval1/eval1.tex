% DYSLEXIA SWITCH
\newif\ifdys
		
				% ENABLE or DISABLE font change
				% use XeLaTeX if true
				\dystrue
				\dysfalse


\ifdys

\documentclass[a4paper, 14pt]{extarticle}
\usepackage{amsmath,amsfonts,amsthm,amssymb,mathtools}

\tracinglostchars=3 % Report an error if a font does not have a symbol.
\usepackage{fontspec}
\usepackage{unicode-math}
\defaultfontfeatures{ Ligatures=TeX,
                      Scale=MatchUppercase }

\setmainfont{OpenDyslexic}[Scale=1.0]
\setmathfont{Fira Math} % Or maybe try KPMath-Sans?
\setmathfont{OpenDyslexic Italic}[range=it/{Latin,latin}]
\setmathfont{OpenDyslexic}[range=up/{Latin,latin,num}]

\else

\documentclass[a4paper, 12pt]{extarticle}

\usepackage[utf8x]{inputenc}
%fonts
\usepackage{amsmath,amsfonts,amsthm,amssymb,mathtools}
% comment below to default to computer modern
\usepackage{libertinus,libertinust1math}

\fi


\usepackage[french]{babel}
\usepackage[
a4paper,
margin=2cm,
nomarginpar,% We don't want any margin paragraphs
]{geometry}
\usepackage{icomma}

\usepackage{fancyhdr}
\usepackage{array}
\usepackage{hyperref}

\usepackage{multicol, enumerate}
\newcolumntype{P}[1]{>{\centering\arraybackslash}p{#1}}


\usepackage{stackengine}
\newcommand\xrowht[2][0]{\addstackgap[.5\dimexpr#2\relax]{\vphantom{#1}}}

% theorems

\theoremstyle{plain}
\newtheorem{theorem}{Th\'eor\`eme}
\newtheorem*{sol}{Solution}
\theoremstyle{definition}
\newtheorem{ex}{Exercice}
\newtheorem*{rpl}{Rappel}
\newtheorem{enigme}{Énigme}

% corps
\usepackage{calrsfs}
\newcommand{\C}{\mathcal{C}}
\newcommand{\R}{\mathbb{R}}
\newcommand{\Rnn}{\mathbb{R}^{2n}}
\newcommand{\Z}{\mathbb{Z}}
\newcommand{\N}{\mathbb{N}}
\newcommand{\Q}{\mathbb{Q}}

% variance
\newcommand{\Var}[1]{\text{Var}(#1)}

% domain
\newcommand{\D}{\mathcal{D}}


% date
\usepackage{advdate}
\AdvanceDate[0]


% plots
\usepackage{pgfplots}

% table line break
\usepackage{makecell}
%tablestuff
\def\arraystretch{2}
\setlength\tabcolsep{15pt}

%subfigures
\usepackage{subcaption}

\definecolor{myg}{RGB}{56, 140, 70}
\definecolor{myb}{RGB}{45, 111, 177}
\definecolor{myr}{RGB}{199, 68, 64}

% fake sections with no title to move around the merged pdf
\newcommand{\fakesection}[1]{%
  \par\refstepcounter{section}% Increase section counter
  \sectionmark{#1}% Add section mark (header)
  \addcontentsline{toc}{section}{\protect\numberline{\thesection}#1}% Add section to ToC
  % Add more content here, if needed.
}


% SOLUTION SWITCH
\newif\ifsolutions
				\solutionstrue
				%\solutionsfalse

\ifsolutions
	\newcommand{\exe}[2]{
		\begin{ex} #1  \end{ex}
		\begin{sol} #2 \end{sol}
	}
\else
	\newcommand{\exe}[2]{
		\begin{ex} #1  \end{ex}
	}
	
\fi


% tableaux var, signe
\usepackage{tkz-tab}


%pinfty minfty
\newcommand{\pinfty}{{+}\infty}
\newcommand{\minfty}{{-}\infty}

\begin{document}


\AdvanceDate[0]
\reversemarginpar

\begin{document}
\pagestyle{fancy}
\fancyhead[L]{Seconde 5}
\fancyhead[C]{\textbf{Évaluation — Ensembles de nombres}}
\fancyhead[R]{\today}

\null\vspace{-30pt}
Consignes particulières : 
\begin{itemize}[label=$\bullet$]
	\item 
	La calculatrice est {interdite}.
	\item 
	Les exercices \ref{exe:1} et \ref{exe:2} peuvent être faits entièrement sur la feuille d'évaluation. Écrire son nom avant de rendre le sujet pour qu'il soit corrigé.
	%\item 
	%Toute trace de recherche est prise en compte.
\end{itemize}

\hrule

\exemulticols{2}{
	Compléter les pointillés en ajoutant un signe d'inclusion ou de non inclusion ($\subseteq$ ou $\not\subseteq$) entre les ensembles de nombres ci-contre.
}{
	\begin{center}
	\def\arraystretch{1.5}
	\setlength\tabcolsep{10pt}
	\begin{tabular}{ccc||ccc}
		$\N$ & \dots & $\Z$ & $\Q$ & \dots & $\Z$ \\
		$\Z$ & \dots & $\R$ & $\DD$ & \dots & $\Q$
	\end{tabular}
	\end{center}
}{exe:1}{
	On utilise le drapeau $\N \subseteq \Z \subseteq \DD \subseteq \Q \subseteq \R$ d'inclusions strictes vu en cours.
	\begin{center}
	\def\arraystretch{1.5}
	\setlength\tabcolsep{10pt}
	\begin{tabular}{ccc||ccc}
		$\N$ & $\subseteq$ & $\Z$ & $\Q$ & $\not\subseteq$ & $\Z$ \\
		$\Z$ & $\subseteq$ & $\R$ & $\DD$ & $\subseteq$ & $\Q$
	\end{tabular}
	\end{center}
}

\exe{2, difficulty=1}{
	Donner le développement décimal des fractions suivantes.
	\begin{multicols}{4}
	\begin{enumerate}[label=\roman*), leftmargin=60pt]
		\item $\dfrac{12}{100}$
		\item $\dfrac6{10^{3}}$
		\item $\dfrac{101}{50}$
		\item $\dfrac{7}{20}$
	\end{enumerate}
	\end{multicols}
}{exe:dev-decimaux}{
	\begin{multicols}{2}
	\begin{enumerate}[label=\roman*)]
		\item $\dfrac{12}{100} = 0,12$
		\item $\dfrac6{10^{3}} = \dfrac6{1000} = 0,006$
		\item $\dfrac{101}{50} = \dfrac{202}{100} = 2,02$
		\item $\dfrac{7}{20} = \dfrac{35}{100} = 0,35$
	\end{enumerate}
	\end{multicols}
}

\exe{{2,5}, difficulty=1}{
	Vrai ou faux ? Cocher la case correspondante.
	\vspace{-30pt}
	\begin{center}
	\begin{tabular}{c c c}
		\hspace{10cm} & Vrai & Faux \\
		$\dfrac12 = 0,5$ & $\square$ & $\square$  \\
		$\dfrac13 = 0,33$ & $\square$ & $\square$  \\
		$\dfrac13 = 0,33333$ & $\square$ & $\square$  \\ \vspace{-5pt}
		$1$ est un nombre rationnel & $\square$ & $\square$  \\
		Tous les nombres réels sont rationnels & $\square$ & $\square$ 
	\end{tabular}
	\end{center}
	\vspace{-10pt}
}{exe:2}{
	\begin{center}
	\begin{tabular}{c c c}
		\hspace{10cm} & Vrai & Faux \\
		$\dfrac12 = 0,5$ & $\times$ &  \\
		$\dfrac13 = 0,33$ & & $\times$  \\
		$\dfrac13 = 0,33333$ & & $\times$  \\ \vspace{-5pt}
		$1$ est un nombre rationnel & $\times$ &  \\
		Tous les nombres réels sont rationnels & & $\times$  
	\end{tabular}
	\end{center}
}

\exe{2, difficulty=0}{
	Donner deux ensembles, $A$ et $B$, vérifiant $A \not\subseteq B$ et $B \not\subseteq A$ : $A$ n'est pas inclus dans $B$, et $B$ n'est pas inclus dans $A$.
}{exe:1.1}{
	Pour que $A \not\subseteq B$, il faut que $A$ contienne un élément qui n'appartient pas à $B$.
	Inversement, $B$ doit contenir un élément qui n'appartient pas à $A$.
	On peut donc prendre, par exemple, $A = \bigset{1 ; 2}$, et $B = \bigset{1 ; 3}$.
	
	Un exemple minimal serait $A = \bigset{1}$, et $B = \bigset{2}$.
}

%\exe{, difficulty=2}{
%	On considère comme admis que le nombre $\pi$, périmètre d'un cercle de diamiètre 1, n'est pas rationnel.
%	Montrer que 
%		\begin{enumerate}
%			\item le nombre $\pi + 2$ n'est pas rationnel non plus ; et
%			\item le nombre $2\pi$ n'est pas rationnel non plus.
%		\end{enumerate}
%}{exe:irr-stable-pi}{
%	\begin{enumerate}
%		\item Si $\pi+2 = \frac{a}b$ est rationnel, alors $\pi = \frac{a-2b}{b}$ doit l'être aussi.
%		Ceci contredit l'énoncé.
%	
%		Alternativement, le développement décimal de $\pi=3,1415...$ est non périodique, d'après le cours et par hypothèse.
%		Ajouter 2 ne le change pas : $\pi+2=5,1415...$. Le développement décimal de $\pi+2$ est donc également apériodique, et c'est un nombre irrationnel.
%		
%		\item
%		Si $2\pi = \frac{a}b$ est rationnel, alors $\pi = \frac{a}{2b}$ doit l'être aussi.
%		Ceci contredit l'énoncé.
%		
%		Un argument avec le développement décimal fonctionne bien pour $\pi+2$, ou $100\pi$ (car le développement ne change pas, ou est simplement décalé), mais plus difficilement pour $2\pi$.
%		Si un raisonnement élégant l'utilisant vous apparaît, veuillez me le signaler.
%		
%	\end{enumerate}
%	Remarque : il s'avère que développement de $\pi$ est apériodique dans toutes les bases. 
%	En particulier en base 10, qu'on utilise couramment, mais aussi en base 2.
%	Dans cette base (dite \emph{binaire}), multiplier par 2 a pour effet de décaler le développement du nombre.
%}

\exe{1, difficulty=1}{
	On considère comme admis que le nombre $\pi$, périmètre d'un cercle de diamètre 1, n'est pas rationnel.
	Montrer que  le nombre $\pi + 2$ n'est pas rationnel non plus.
}{exe:irr-stable-pi}{
	Si $\pi+2 = \frac{a}b$ est rationnel, alors $\pi = \frac{a-2b}{b}$ doit l'être aussi.
	Ceci contredit l'énoncé.

	Alternativement, le développement décimal de $\pi=3,1415...$ est non périodique, d'après le cours, et par hypothèse.
	Ajouter 2 ne le change pas : $\pi+2=5,1415...$. Le développement décimal de $\pi+2$ est donc également apériodique, et c'est un nombre irrationnel.
}


\exe{2}{
	Donner les éléments de chaque ensemble suivant.
	\begin{multicols}{2}
	\begin{enumerate}[label=]
		%\item $A=\bigset{n \in \N \text{ tel que } 3 \leq n \leq 8 }$
		%\item $B = \bigset{a \in A \tq \text{$a$ est pair}}$
		\item $A = \bigset{n \in \Z \tq -1 \leq n \leq 1}$,
		\item $B = \bigset{a \in A \tq \text{$a$ est non nul}}$.
	\end{enumerate} 
	\end{multicols}
}{exe:3}{
	$ A= \bigset{ -1 ; 0 ; 1}$, et $B = \bigset{ -1 ; 1}$.
}

\exe{2, difficulty=1}{
	Exprimer les nombres suivants sous forme de fraction d'entiers.
	\begin{multicols}{2}
	\begin{enumerate}[label=]
		%\item $A = 0,666...$ ($6$ se répète à l'infini)
		\item $x = 0,555...$ ($5$ se répète à l'infini)
		%\item $C = 0,020202...$ ($02$ se répète à l'infini)
		\item $y = 1,545454...$ ($54$ se répète à l'infini)
	\end{enumerate} 
	\end{multicols}
}{exe:4}{
	\begin{enumerate}
		\item L'équation $10x = 5 + x$ implique que $x=\frac59$.
		\item L'équation 
			\[ 100y = 154,545454... = 153 + 1,545454... = 153 + y \]
		implique que $y = \frac{153}{99}$.
	\end{enumerate}
}

\exe{2}{
	Écrire les nombres suivants en notation scientifique.
	\begin{multicols}{2}
	\begin{enumerate}[label=\alph*)]
		%\item 304
		%\item 100
		\item 937 000 500
		%\item 0,4
		%\item 0,000 438
		\item 0,004 000 2
	\end{enumerate}
	\end{multicols}
}{exe:5}{
	\begin{enumerate}[label=\alph*)]
		%\item $304 = 3,04 \times 10^2$.
		%\item 100
		\item $937~000~500 = 9,370005 \times 10^{8}$.
		%\item 0,4
		%\item 0,000 438
		\item $0,004~000~2 = 4,0002 \times 10^{-3}$.
	\end{enumerate}
}

\exe{2, difficulty=1}{
	Donner le développement décimal de $\frac16$.
}{exe:dev-16}{
	$\frac16 < 1$, donc $\frac16 = 0,...$.
	Pour connaître la première décimale, on multiplie par 10, et on étudie l'unité.
	
	$\frac{10}6 = 1 + \frac46$, donc la première décimale est 1.
	On répère sur le reste : $\frac{40}6 = 6 + \frac46$, et la deuxième décimale est 6.
	On se convainc facilement que toutes les décimales après sont aussi 6, car le reste est toujours $\frac46$.
	
	En conclusion, $\frac16 = 0,166666...$, les 6 se répétant à l'infini.
}

\exe{2, difficulty=2}{
	Soit $k\in\N$ non nul. 
	On suppose qu'aucune puissance de 10 n'est dans la table de multiplication de $k$.
	Montrer que $\frac1k$ n'est pas décimal.
%	\\\\
%	\emph{Toute trace de recherche sera prise en compte.}
}{exe:6}{
	On démontre ce fait par l'absurde, en reprenant le début de la preuve du cours.
	
	Supposons, par contradiction, que $\frac1k$ est décimal : $\frac1k \in \DD$.
	Alors, par définition de $\DD$, son développement décimal est fini, de longueur $n\in\N$.
	
	Il suit que $10^n \times \frac1k$ est un nombre entier. Notons-le $a \in \N$ : 
		\[ 10^n \times \frac1k = a. \]
	En multipliant par $k$, on obtient la relation d'entiers équivalente : 
		\[ 10^n = k \times a. \]
	$a$ étant entier, ceci implique que $10^n$ est un multiple entier de $k$.
	Autrement dit, $10^n$ appartient à la table de multiplication de $k$.
	
	Ceci contredit l'hypothèse de l'énoncé. \Large\Lightning
}

\enlargethispage{20pt}
%\pagenumbering{gobble}

\exe{2, difficulty=3}{
	Soit $n\in\N$ un entier naturel. Montrer que
		\[ 1 + 10 + 10^2 + 10^3 + \cdots + 10^n = \dfrac{10^{n+1} - 1}9. \]
	La somme de gauche contient $n+1$ termes.
}{exe:10}{
	Choisissons $n=0 ; 1 ; 2 ; ...$ et calculons les membres de gauche et de droite de l'égalité pour mettre en évidence une structure permettant de généraliser pour n'importe quel $n\in\N$.
	
	En $n=0$, la somme de gauche est 1, et l'expression à droite est $\frac{10 - 1}9 = \frac99 = 1$.
	
	En $n=1$, la somme de gauche est $1+10 = 11$, et l'expression à droite est $\frac{100-1}9 = \frac{99}9 = 11$.
	
	En $n=2$, la somme de gauche est $1+10+100 = 111$, et l'expression à droite est $\frac{1000-1}9 = \frac{999}9 = 111$.
	
	Remarquons que, plus généralement,
		\[ 1 + 10 + 10^2 + 10^3 + \cdots + 10^n = \underbrace{111\cdots111}_{\text{$n+1$ fois}}. \]
	Multiplier ce nombre par 9 puis ajouter 1 donne
		\[ 9 \bigl(1 + 10 + 10^2 + 10^3 + \cdots + 10^n\bigr) + 1 = \underbrace{999\cdots999}_{\text{$n+1$ fois}} + 1 = 1\underbrace{000\cdots000}_{\text{$n+1$ fois}} = 10^{n+1}. \]
	Soustraire 1 puis diviser par 9 conclut donc.
}

%%%%%%%%%%%%

\newpage
\fancyhead[C]{\textbf{Solutions}}
\shipoutAnswer

\end{document}

%!TEX encoding = UTF8
%!TEX root =notes.tex


%%%%%%%%%%%%%%%%%%%%%%%%%%%%%%%%%
% PACKAGE IMPORTS
%%%%%%%%%%%%%%%%%%%%%%%%%%%%%%%%%


\usepackage[french]{babel}

\usepackage[tmargin=2cm,rmargin=1in,lmargin=1in,margin=0.85in,bmargin=2cm,footskip=.2in]{geometry}
\usepackage{amsmath,amsfonts,amsthm,amssymb,mathtools}
\usepackage[varbb]{newpxmath}
\usepackage{xfrac}
\usepackage[makeroom]{cancel}
\usepackage{mathtools}
\usepackage{bookmark}
\usepackage{enumitem}
\usepackage{hyperref,theoremref}
\hypersetup{
	pdftitle={Assignment},
	colorlinks=true, linkcolor=doc!90,
	bookmarksnumbered=true,
	bookmarksopen=true
}
\usepackage[most,many,breakable]{tcolorbox}
\usepackage{xcolor}
\usepackage{varwidth}
\usepackage{varwidth}
\usepackage{etoolbox}
%\usepackage{authblk}
\usepackage{nameref}
\usepackage{multicol,array}
\usepackage{tikz-cd}
\usepackage[ruled,vlined,linesnumbered]{algorithm2e}
\usepackage{comment} % enables the use of multi-line comments (\ifx \fi) 
\usepackage{import}
\usepackage{xifthen}
\usepackage{pdfpages}
\usepackage{transparent}


\newcommand\mycommfont[1]{\footnotesize\ttfamily\textcolor{blue}{#1}}
\SetCommentSty{mycommfont}
\newcommand{\incfig}[1]{%
    \def\svgwidth{\columnwidth}
    \import{./figures/}{#1.pdf_tex}
}

\usepackage{tikzsymbols}
%\renewcommand\qedsymbol{$\Laughey$}


%\usepackage{import}
%\usepackage{xifthen}
%\usepackage{pdfpages}
%\usepackage{transparent}


%%%%%%%%%%%%%%%%%%%%%%%%%%%%%%
% SELF MADE COLORS
%%%%%%%%%%%%%%%%%%%%%%%%%%%%%%



\definecolor{myg}{RGB}{56, 140, 70}
\definecolor{myb}{RGB}{45, 111, 177}
\definecolor{myr}{RGB}{199, 68, 64}
\definecolor{mytheorembg}{HTML}{F2F2F9}
\definecolor{mytheoremfr}{HTML}{00007B}
\definecolor{mylenmabg}{HTML}{FFFAF8}
\definecolor{mylenmafr}{HTML}{983b0f}
\definecolor{mypropbg}{HTML}{f2fbfc}
\definecolor{mypropfr}{HTML}{191971}
\definecolor{myexamplebg}{HTML}{F2FBF8}
\definecolor{myexamplefr}{HTML}{88D6D1}
\definecolor{myexampleti}{HTML}{2A7F7F}
\definecolor{mydefinitbg}{HTML}{E5E5FF}
\definecolor{mydefinitfr}{HTML}{3F3FA3}
\definecolor{notesgreen}{RGB}{0,162,0}
\definecolor{myp}{RGB}{197, 92, 212}
\definecolor{mygr}{HTML}{2C3338}
\definecolor{myred}{RGB}{127,0,0}
\definecolor{myyellow}{RGB}{169,121,69}
\definecolor{myexercisebg}{HTML}{F2FBF8}
\definecolor{myexercisefg}{HTML}{88D6D1}


%%%%%%%%%%%%%%%%%%%%%%%%%%%%
% TCOLORBOX SETUPS
%%%%%%%%%%%%%%%%%%%%%%%%%%%%

\setlength{\parindent}{1cm}
%================================
% THEOREM BOX
%================================

\tcbuselibrary{theorems,skins,hooks}
\newtcbtheorem[number within=chapter]{Theorem}{Théorème}
{%
	enhanced,
	breakable,
	colback = mytheorembg,
	frame hidden,
	boxrule = 0sp,
	borderline west = {2pt}{0pt}{mytheoremfr},
	sharp corners,
	detach title,
	before upper = \tcbtitle\par\smallskip,
	coltitle = mytheoremfr,
	fonttitle = \bfseries\sffamily,
	description font = \mdseries,
	separator sign none,
	segmentation style={solid, mytheoremfr},
}
{th}


\tcbuselibrary{theorems,skins,hooks}
\newtcolorbox{Theoremcon}
{%
	enhanced
	,breakable
	,colback = mytheorembg
	,frame hidden
	,boxrule = 0sp
	,borderline west = {2pt}{0pt}{mytheoremfr}
	,sharp corners
	,description font = \mdseries
	,separator sign none
}

%================================
% Corollery
%================================
\tcbuselibrary{theorems,skins,hooks}
\newtcbtheorem[use counter=tcb@cnt@Theorem]{Corollary}{Corollaire}
{%
	enhanced
	,breakable
	,colback = myp!10
	,frame hidden
	,boxrule = 0sp
	,borderline west = {2pt}{0pt}{myp!85!black}
	,sharp corners
	,detach title
	,before upper = \tcbtitle\par\smallskip
	,coltitle = myp!85!black
	,fonttitle = \bfseries\sffamily
	,description font = \mdseries
	,separator sign none
	,segmentation style={solid, myp!85!black}
}
{th}

%================================
% LENMA
%================================

\tcbuselibrary{theorems,skins,hooks}
\newtcbtheorem[use counter=tcb@cnt@Theorem]{Lemma}{Lemme}
{%
	enhanced,
	breakable,
	colback = mylenmabg,
	frame hidden,
	boxrule = 0sp,
	borderline west = {2pt}{0pt}{mylenmafr},
	sharp corners,
	detach title,
	before upper = \tcbtitle\par\smallskip,
	coltitle = mylenmafr,
	fonttitle = \bfseries\sffamily,
	description font = \mdseries,
	separator sign none,
	segmentation style={solid, mylenmafr},
}
{th}


%================================
% PROPOSITION
%================================

\tcbuselibrary{theorems,skins,hooks}
\newtcbtheorem[use counter=tcb@cnt@Theorem]{Prop}{Proposition}
{%
	enhanced,
	breakable,
	colback = mypropbg,
	frame hidden,
	boxrule = 0sp,
	borderline west = {2pt}{0pt}{mypropfr},
	sharp corners,
	detach title,
	before upper = \tcbtitle\par\smallskip,
	coltitle = mypropfr,
	fonttitle = \bfseries\sffamily,
	description font = \mdseries,
	separator sign none,
	segmentation style={solid, mypropfr},
}
{th}


%================================
% CLAIM
%================================

\tcbuselibrary{theorems,skins,hooks}
\newtcbtheorem[use counter=tcb@cnt@Theorem]{claim}{Claim}
{%
	enhanced
	,breakable
	,colback = myg!10
	,frame hidden
	,boxrule = 0sp
	,borderline west = {2pt}{0pt}{myg}
	,sharp corners
	,detach title
	,before upper = \tcbtitle\par\smallskip
	,coltitle = myg!85!black
	,fonttitle = \bfseries\sffamily
	,description font = \mdseries
	,separator sign none
	,segmentation style={solid, myg!85!black}
}
{th}



%================================
% Exercise
%================================

\tcbuselibrary{theorems,skins,hooks}
\newtcbtheorem[use counter=tcb@cnt@Theorem]{Exercise}{Exercice}
{%
	enhanced,
	breakable,
	colback = myexercisebg,
	frame hidden,
	boxrule = 0sp,
	borderline west = {2pt}{0pt}{myexercisefg},
	sharp corners,
	detach title,
	before upper = \tcbtitle\par\smallskip,
	coltitle = myexercisefg,
	fonttitle = \bfseries\sffamily,
	description font = \mdseries,
	separator sign none,
	segmentation style={solid, myexercisefg},
}
{th}

%================================
% EXAMPLE BOX
%================================

\newtcbtheorem[use counter=tcb@cnt@Theorem]{Example}{Exemple}
{%
	colback = myexamplebg
	,breakable
	,colframe = myexamplefr
	,coltitle = myexampleti
	,boxrule = 1pt
	,sharp corners
	,detach title
	,before upper=\tcbtitle\par\smallskip
	,fonttitle = \bfseries
	,description font = \mdseries
	,separator sign none
	,description delimiters parenthesis
}
{ex}

%================================
% DEFINITION BOX
%================================

\newtcbtheorem[use counter=tcb@cnt@Theorem]{Definition}{Définition}{enhanced,
	before skip=2mm,after skip=2mm, colback=red!5,colframe=red!80!black,boxrule=0.5mm,
	attach boxed title to top left={xshift=1cm,yshift*=1mm-\tcboxedtitleheight}, varwidth boxed title*=-3cm,
	boxed title style={frame code={
					\path[fill=tcbcolback]
					([yshift=-1mm,xshift=-1mm]frame.north west)
					arc[start angle=0,end angle=180,radius=1mm]
					([yshift=-1mm,xshift=1mm]frame.north east)
					arc[start angle=180,end angle=0,radius=1mm];
					\path[left color=tcbcolback!60!black,right color=tcbcolback!60!black,
						middle color=tcbcolback!80!black]
					([xshift=-2mm]frame.north west) -- ([xshift=2mm]frame.north east)
					[rounded corners=1mm]-- ([xshift=1mm,yshift=-1mm]frame.north east)
					-- (frame.south east) -- (frame.south west)
					-- ([xshift=-1mm,yshift=-1mm]frame.north west)
					[sharp corners]-- cycle;
				},interior engine=empty,
		},
	fonttitle=\bfseries,
	title={#2},#1}{def}

%================================
% Solution BOX
%================================

\makeatletter
\newtcbtheorem[use counter=tcb@cnt@Theorem]{question}{Question}{enhanced,
	breakable,
	colback=white,
	colframe=myb!80!black,
	attach boxed title to top left={yshift*=-\tcboxedtitleheight},
	fonttitle=\bfseries,
	title={#2},
	boxed title size=title,
	boxed title style={%
			sharp corners,
			rounded corners=northwest,
			colback=tcbcolframe,
			boxrule=0pt,
		},
	underlay boxed title={%
			\path[fill=tcbcolframe] (title.south west)--(title.south east)
			to[out=0, in=180] ([xshift=5mm]title.east)--
			(title.center-|frame.east)
			[rounded corners=\kvtcb@arc] |-
			(frame.north) -| cycle;
		},
	#1
}{def}
\makeatother

%================================
% SOLUTION BOX
%================================

\makeatletter
\newtcolorbox{solution}{enhanced,
	breakable,
	colback=white,
	colframe=myg!80!black,
	attach boxed title to top left={yshift*=-\tcboxedtitleheight},
	title=Solution,
	boxed title size=title,
	boxed title style={%
			sharp corners,
			rounded corners=northwest,
			colback=tcbcolframe,
			boxrule=0pt,
		},
	underlay boxed title={%
			\path[fill=tcbcolframe] (title.south west)--(title.south east)
			to[out=0, in=180] ([xshift=5mm]title.east)--
			(title.center-|frame.east)
			[rounded corners=\kvtcb@arc] |-
			(frame.north) -| cycle;
		},
}
\makeatother

%================================
% Question BOX
%================================

\makeatletter
\newtcbtheorem[use counter=tcb@cnt@Theorem]{qstion}{Question}{enhanced,
	breakable,
	colback=white,
	colframe=mygr,
	attach boxed title to top left={yshift*=-\tcboxedtitleheight},
	fonttitle=\bfseries,
	title={#2},
	boxed title size=title,
	boxed title style={%
			sharp corners,
			rounded corners=northwest,
			colback=tcbcolframe,
			boxrule=0pt,
		},
	underlay boxed title={%
			\path[fill=tcbcolframe] (title.south west)--(title.south east)
			to[out=0, in=180] ([xshift=5mm]title.east)--
			(title.center-|frame.east)
			[rounded corners=\kvtcb@arc] |-
			(frame.north) -| cycle;
		},
	#1
}{def}
\makeatother

\newtcbtheorem[number within=chapter]{wconc}{Wrong Concept}{
	breakable,
	enhanced,
	colback=white,
	colframe=myr,
	arc=0pt,
	outer arc=0pt,
	fonttitle=\bfseries\sffamily\large,
	colbacktitle=myr,
	attach boxed title to top left={},
	boxed title style={
			enhanced,
			skin=enhancedfirst jigsaw,
			arc=3pt,
			bottom=0pt,
			interior style={fill=myr}
		},
	#1
}{def}



%================================
% NOTE BOX
%================================

\usetikzlibrary{arrows,calc,shadows.blur}
\tcbuselibrary{skins}
\newtcolorbox{note}[1][]{%
	enhanced jigsaw,
	colback=gray!20!white,%
	colframe=gray!80!black,
	size=small,
	boxrule=1pt,
	title=\colorbox{white!100}{\textbf{ Remarque }},
	halign title=flush center,
	coltitle=black,
	breakable,
	drop shadow=black!50!white,
	attach boxed title to top left={xshift=1cm,yshift=-\tcboxedtitleheight/2,yshifttext=-\tcboxedtitleheight/2},
	minipage boxed title=2.6cm,
	boxed title style={%
			colback=white,
			size=fbox,
			boxrule=1pt,
			boxsep=2pt,
			underlay={%
					\coordinate (dotA) at ($(interior.west) + (-0.5pt,0)$);
					\coordinate (dotB) at ($(interior.east) + (0.5pt,0)$);
					\begin{scope}
						\clip (interior.north west) rectangle ([xshift=3ex]interior.east);
						\filldraw [white, blur shadow={shadow opacity=60, shadow yshift=-.75ex}, rounded corners=2pt] (interior.north west) rectangle (interior.south east);
					\end{scope}
					\begin{scope}[gray!80!black]
						\fill (dotA) circle (2pt);
						\fill (dotB) circle (2pt);
					\end{scope}
				},
		},
	#1,
}

%================================
% STRATÉGIE BOX
%================================

\usetikzlibrary{arrows,calc,shadows.blur}
\tcbuselibrary{skins}
\newtcolorbox{strategy}[1][]{%
	enhanced jigsaw,
	colback=myb!20!white,%
	colframe=gray!80!black,
	size=small,
	boxrule=1pt,
	title=\colorbox{white!100}{\textbf{ Stratégie }},
	halign title=flush center,
	coltitle=black,
	breakable,
	drop shadow=black!50!white,
	attach boxed title to top left={xshift=1cm,yshift=-\tcboxedtitleheight/2,yshifttext=-\tcboxedtitleheight/2},
	minipage boxed title=2.5cm,
	boxed title style={%
			colback=white,
			size=fbox,
			boxrule=1pt,
			boxsep=2pt,
			underlay={%
					\coordinate (dotA) at ($(interior.west) + (-0.5pt,0)$);
					\coordinate (dotB) at ($(interior.east) + (0.5pt,0)$);
					\begin{scope}
						\clip (interior.north west) rectangle ([xshift=3ex]interior.east);
						\filldraw [white, blur shadow={shadow opacity=60, shadow yshift=-.75ex}, rounded corners=2pt] (interior.north west) rectangle (interior.south east);
					\end{scope}
					\begin{scope}[gray!80!black]
						\fill (dotA) circle (2pt);
						\fill (dotB) circle (2pt);
					\end{scope}
				},
		},
	#1,
}

%================================
% MÉTHODE BOX
%================================

\usetikzlibrary{arrows,calc,shadows.blur}
\tcbuselibrary{skins}
\newtcolorbox{methode}[1][]{%
	enhanced jigsaw,
	colback=white,%
	colframe=gray!80!black,
	size=small,
	boxrule=1pt,
	title=\textbf{Méthode},
	halign title=flush center,
	coltitle=black,
	breakable,
	drop shadow=black!50!white,
	attach boxed title to top left={xshift=1cm,yshift=-\tcboxedtitleheight/2,yshifttext=-\tcboxedtitleheight/2},
	minipage boxed title=2.5cm,
	boxed title style={%
			colback=white,
			size=fbox,
			boxrule=1pt,
			boxsep=2pt,
			underlay={%
					\coordinate (dotA) at ($(interior.west) + (-0.5pt,0)$);
					\coordinate (dotB) at ($(interior.east) + (0.5pt,0)$);
					\begin{scope}
						\clip (interior.north west) rectangle ([xshift=3ex]interior.east);
						\filldraw [white, blur shadow={shadow opacity=60, shadow yshift=-.75ex}, rounded corners=2pt] (interior.north west) rectangle (interior.south east);
					\end{scope}
					\begin{scope}[gray!80!black]
						\fill (dotA) circle (2pt);
						\fill (dotB) circle (2pt);
					\end{scope}
				},
		},
	#1,
}

%%%%%%%%%%%%%%%%%%%%%%%%%%%%%%%%%%%%%%%%%%%
% TABLE OF CONTENTS
%%%%%%%%%%%%%%%%%%%%%%%%%%%%%%%%%%%%%%%%%%%

\usepackage{tikz}

\definecolor{doc}{RGB}{0,60,110}
\usepackage{titletoc}
\contentsmargin{0cm}
\titlecontents{chapter}[3.7pc]
{\addvspace{30pt}%
	\begin{tikzpicture}[remember picture, overlay]%
		\draw[fill=doc!60,draw=doc!60] (-7,-.1) rectangle (-0.2,.6);%
		\pgftext[left,x=-3.5cm,y=0.2cm]{\color{white}\Large\sc\bfseries Chapitre\ \thecontentslabel};%
	\end{tikzpicture}\color{doc!60}\large\sc\bfseries}%
{}
{}
{\;\titlerule\;\large\sc\bfseries Page \thecontentspage
	\begin{tikzpicture}[remember picture, overlay]
		\draw[fill=doc!60,draw=doc!60] (2pt,0) rectangle (4,0.1pt);
	\end{tikzpicture}}%
\titlecontents{section}[3.7pc]
{\addvspace{2pt}}
{\contentslabel[\thecontentslabel]{2pc}}
{}
{\hfill\small \thecontentspage}
[]
\titlecontents*{subsection}[3.7pc]
{\addvspace{-1pt}\small}
{}
{}
{\ --- \small\thecontentspage}
[ \textbullet\ ][]

\makeatletter
\renewcommand{\tableofcontents}{%
	\chapter*{%
	  \vspace*{-20\p@}%
	  \begin{tikzpicture}[remember picture, overlay]%
		  \pgftext[right,x=15cm,y=0.2cm]{\color{doc!60}\Huge\sc\bfseries \contentsname};%
		  \draw[fill=doc!60,draw=doc!60] (13,-.75) rectangle (20,1);%
		  \clip (13,-.75) rectangle (20,1);
		  \pgftext[right,x=15cm,y=0.2cm]{\color{white}\Huge\sc\bfseries \contentsname};%
	  \end{tikzpicture}}%
	\@starttoc{toc}}
\makeatother


%%%%%%%%%%%%%%%%%%%%%%%%%%%%%%%%%%%%%%%%%%%
% MINTED FOR PYTHON ALGORITHMS
%%%%%%%%%%%%%%%%%%%%%%%%%%%%%%%%%%%%%%%%%%%

\usepackage{tcolorbox}
\tcbuselibrary{minted,breakable,xparse,skins}
\definecolor{bg}{gray}{0.95}
\DeclareTCBListing{mintedbox}{O{}m!O{}}{%
  breakable=true,
  listing engine=minted,
  listing only,
  minted language=#2,
  minted style=default,
  minted options={%
    linenos,
    gobble=0,
    breaklines=true,
    breakafter=,,
    fontsize=\small,
    numbersep=8pt,
    #1},
  boxsep=0pt,
  left skip=0pt,
  right skip=0pt,
  left=25pt,
  right=0pt,
  top=3pt,
  bottom=3pt,
  arc=5pt,
  leftrule=0pt,
  rightrule=0pt,
  bottomrule=2pt,
  toprule=2pt,
  colback=bg,
  colframe=orange!70,
  enhanced,
  overlay={%
    \begin{tcbclipinterior}
    \fill[orange!20!white] (frame.south west) rectangle ([xshift=20pt]frame.north west);
    \end{tcbclipinterior}},
  #3}
  
  
 % for braces
\usetikzlibrary{decorations.pathreplacing}


\AdvanceDate[0]
\reversemarginpar

\begin{document}
\pagestyle{fancy}
\fancyhead[L]{Seconde 5}
\fancyhead[C]{\textbf{Évaluation — Ensembles de nombres}}
\fancyhead[R]{\today}

\null\vspace{-30pt}
Consignes particulières : 
\begin{itemize}[label=$\bullet$]
	\item 
	La calculatrice est {interdite}.
	\item 
	Les exercices \ref{exe:1} et \ref{exe:2} peuvent être faits entièrement sur la feuille d'évaluation. Écrire son nom avant de rendre le sujet pour qu'il soit corrigé.
	%\item 
	%Toute trace de recherche est prise en compte.
\end{itemize}

\hrule

\exemulticols{2}{
	Compléter les pointillés en ajoutant un signe d'inclusion ou de non inclusion ($\subseteq$ ou $\not\subseteq$) entre les ensembles de nombres ci-contre.
}{
	\begin{center}
	\def\arraystretch{1.5}
	\setlength\tabcolsep{10pt}
	\begin{tabular}{ccc||ccc}
		$\N$ & \dots & $\Z$ & $\Q$ & \dots & $\Z$ \\
		$\Z$ & \dots & $\R$ & $\DD$ & \dots & $\Q$
	\end{tabular}
	\end{center}
}{exe:1}{
	On utilise le drapeau $\N \subseteq \Z \subseteq \DD \subseteq \Q \subseteq \R$ d'inclusions strictes vues en cours.
	\begin{center}
	\def\arraystretch{1.5}
	\setlength\tabcolsep{10pt}
	\begin{tabular}{ccc||ccc}
		$\N$ & $\subseteq$ & $\Z$ & $\Q$ & $\not\subseteq$ & $\Z$ \\
		$\Z$ & $\subseteq$ & $\R$ & $\DD$ & $\subseteq$ & $\Q$
	\end{tabular}
	\end{center}
}

\exe{2, difficulty=1}{
	Donner le développement décimal des fractions suivantes.
	\begin{multicols}{4}
	\begin{enumerate}[label=\roman*), leftmargin=60pt]
		\item $\dfrac{12}{100}$
		\item $\dfrac6{10^{3}}$
		\item $\dfrac{101}{50}$
		\item $\dfrac{7}{20}$
	\end{enumerate}
	\end{multicols}
}{exe:dev-decimaux}{
	\begin{multicols}{2}
	\begin{enumerate}[label=\roman*)]
		\item $\dfrac{12}{100} = 0,12$
		\item $\dfrac6{10^{3}} = \dfrac6{1000} = 0,006$
		\item $\dfrac{101}{50} = \dfrac{202}{100} = 2,02$
		\item $\dfrac{7}{20} = \dfrac{35}{100} = 0,35$
	\end{enumerate}
	\end{multicols}
}

\exe{{2,5}, difficulty=1}{
	Vrai ou faux ? Cocher la case correspondante.
	\vspace{-30pt}
	\begin{center}
	\begin{tabular}{c c c}
		\hspace{10cm} & Vrai & Faux \\
		$\dfrac12 = 0,5$ & $\square$ & $\square$  \\
		$\dfrac13 = 0,33$ & $\square$ & $\square$  \\
		$\dfrac13 = 0,33333$ & $\square$ & $\square$  \\ \vspace{-5pt}
		$1$ est un nombre rationnel & $\square$ & $\square$  \\
		Tous les nombres réels sont rationnels & $\square$ & $\square$ 
	\end{tabular}
	\end{center}
	\vspace{-10pt}
}{exe:2}{
	\begin{center}
	\begin{tabular}{c c c}
		\hspace{10cm} & Vrai & Faux \\
		$\dfrac12 = 0,5$ & $\times$ &  \\
		$\dfrac13 = 0,33$ & & $\times$  \\
		$\dfrac13 = 0,33333$ & & $\times$  \\ \vspace{-5pt}
		$1$ est un nombre rationnel & $\times$ &  \\
		Tous les nombres réels sont rationnels & & $\times$  
	\end{tabular}
	\end{center}
}

\exe{2, difficulty=0}{
	Donner deux ensembles, $A$ et $B$, vérifiant $A \not\subseteq B$ et $B \not\subseteq A$ : $A$ n'est pas inclus dans $B$, et $B$ n'est pas inclus dans $A$.
}{exe:1.1}{
	Pour que $A \not\subseteq B$, il faut que $A$ contienne un élément qui n'appartient pas à $B$.
	Inversement, $B$ doit contenir un élément qui n'appartient pas à $A$.
	On peut donc prendre, par exemple, $A = \bigset{1 ; 2}$, et $B = \bigset{1 ; 3}$.
	
	Un exemple minimal serait $A = \bigset{1}$, et $B = \bigset{2}$.
}

%\exe{, difficulty=2}{
%	On considère comme admis que le nombre $\pi$, périmètre d'un cercle de diamiètre 1, n'est pas rationnel.
%	Montrer que 
%		\begin{enumerate}
%			\item le nombre $\pi + 2$ n'est pas rationnel non plus ; et
%			\item le nombre $2\pi$ n'est pas rationnel non plus.
%		\end{enumerate}
%}{exe:irr-stable-pi}{
%	\begin{enumerate}
%		\item Si $\pi+2 = \frac{a}b$ est rationnel, alors $\pi = \frac{a-2b}{b}$ doit l'être aussi.
%		Ceci contredit l'énoncé.
%	
%		Alternativement, le développement décimal de $\pi=3,1415...$ est non périodique, d'après le cours et par hypothèse.
%		Ajouter 2 ne le change pas : $\pi+2=5,1415...$. Le développement décimal de $\pi+2$ est donc également apériodique, et c'est un nombre irrationnel.
%		
%		\item
%		Si $2\pi = \frac{a}b$ est rationnel, alors $\pi = \frac{a}{2b}$ doit l'être aussi.
%		Ceci contredit l'énoncé.
%		
%		Un argument avec le développement décimal fonctionne bien pour $\pi+2$, ou $100\pi$ (car le développement ne change pas, ou est simplement décalé), mais plus difficilement pour $2\pi$.
%		Si un raisonnement élégant l'utilisant vous apparaît, veuillez me le signaler.
%		
%	\end{enumerate}
%	Remarque : il s'avère que développement de $\pi$ est apériodique dans toutes les bases. 
%	En particulier en base 10, qu'on utilise couramment, mais aussi en base 2.
%	Dans cette base (dite \emph{binaire}), multiplier par 2 a pour effet de décaler le développement du nombre.
%}

\exe{1, difficulty=1}{
	On considère comme admis que le nombre $\pi$, périmètre d'un cercle de diamètre 1, n'est pas rationnel.
	Montrer que  le nombre $\pi + 2$ n'est pas rationnel non plus.
}{exe:irr-stable-pi}{
	Si $\pi+2 = \frac{a}b$ est rationnel, alors $\pi = \frac{a-2b}{b}$ doit l'être aussi.
	Ceci contredit l'énoncé.

	Alternativement, le développement décimal de $\pi=3,1415...$ est non périodique, d'après le cours, et par hypothèse.
	Ajouter 2 ne le change pas : $\pi+2=5,1415...$. Le développement décimal de $\pi+2$ est donc également apériodique, et c'est un nombre irrationnel.
}


\exe{2}{
	Donner les éléments de chaque ensemble suivant.
	\begin{multicols}{2}
	\begin{enumerate}[label=]
		%\item $A=\bigset{n \in \N \text{ tel que } 3 \leq n \leq 8 }$
		%\item $B = \bigset{a \in A \tq \text{$a$ est pair}}$
		\item $A = \bigset{n \in \Z \tq -1 \leq n \leq 1}$,
		\item $B = \bigset{a \in A \tq \text{$a$ est non nul}}$.
	\end{enumerate} 
	\end{multicols}
}{exe:3}{
	$ A= \bigset{ -1 ; 0 ; 1}$, et $B = \bigset{ -1 ; 1}$.
}

\exe{2, difficulty=1}{
	Exprimer les nombres suivants sous forme de fraction d'entiers.
	\begin{multicols}{2}
	\begin{enumerate}[label=]
		%\item $A = 0,666...$ ($6$ se répète à l'infini)
		\item $x = 0,555...$ ($5$ se répète à l'infini)
		%\item $C = 0,020202...$ ($02$ se répète à l'infini)
		\item $y = 1,545454...$ ($54$ se répète à l'infini)
	\end{enumerate} 
	\end{multicols}
}{exe:4}{
	\begin{enumerate}
		\item L'équation $10x = 5 + x$ implique que $x=\frac59$.
		\item L'équation 
			\[ 100y = 154,545454... = 153 + 1,545454... = 153 + y \]
		implique que $y = \frac{153}{99}$.
	\end{enumerate}
}

\exe{2}{
	Écrire les nombres suivants en notation scientifique.
	\begin{multicols}{2}
	\begin{enumerate}[label=\alph*)]
		%\item 304
		%\item 100
		\item 937 000 500
		%\item 0,4
		%\item 0,000 438
		\item 0,004 000 2
	\end{enumerate}
	\end{multicols}
}{exe:5}{
	\begin{enumerate}[label=\alph*)]
		%\item $304 = 3,04 \times 10^2$.
		%\item 100
		\item $937~000~500 = 9,370005 \times 10^{8}$.
		%\item 0,4
		%\item 0,000 438
		\item $0,004~000~2 = 4,0002 \times 10^{-3}$.
	\end{enumerate}
}

\exe{3, difficulty=1}{
	Donner le développement décimal de $\frac16$.
	En déduire un encadrement de $\frac16$ à $10^{-2}$ près.
}{exe:dev-16}{
	$\frac16 < 1$, donc $\frac16 = 0,...$.
	Pour connaître la première décimale, on multiplie par 10, et on étudie l'unité.
	
	$\frac{10}6 = 1 + \frac46$, donc la première décimale est 1.
	On répère sur le reste : $\frac{40}6 = 6 + \frac46$, et la deuxième décimale est 6.
	On se convainc facilement que toutes les décimales après sont aussi 6, car le reste est toujours $\frac46$.
	
	En conclusion, $\frac16 = 0,166666...$, les 6 se répétant à l'infini.
}

\exe{2, difficulty=2}{
	Soit $k\in\N$ non nul. 
	On suppose qu'aucune puissance de 10 n'est dans la table de multiplication de $k$.
	Montrer que $\frac1k$ n'est pas décimal.
%	\\\\
%	\emph{Toute trace de recherche sera prise en compte.}
}{exe:6}{
	On démontre ce fait par l'absurde, en reprenant le début de la preuve du cours.
	
	Supposons, par contradiction, que $\frac1k$ est décimal : $\frac1k \in \DD$.
	Alors, par définition de $\DD$, son développement décimal est fini, de longueur $n\in\N$.
	
	Il suit que $10^n \times \frac1k$ est un nombre entier. Notons-le $a \in \N$ : 
		\[ 10^n \times \frac1k = a. \]
	En multipliant par $k$, on obtient la relation d'entiers équivalente : 
		\[ 10^n = k \times a. \]
	$a$ étant entier, ceci implique que $10^n$ est un multiple entier de $k$.
	Autrement dit, $10^n$ appartient à la table de multiplication de $k$.
	
	Ceci contredit l'hypothèse de l'énoncé. \Large\Lightning
}

\enlargethispage{20pt}
%\pagenumbering{gobble}

\exe{2, difficulty=3}{
	Soit $n\in\N$ un entier naturel. Montrer que
		\[ 1 + 10 + 10^2 + 10^3 + \cdots + 10^n = \dfrac{10^{n+1} - 1}9. \]
	La somme de gauche contient $n+1$ termes.
}{exe:10}{
	Choisissons $n=0 ; 1 ; 2 ; ...$ et calculons les membres de gauche et de droite de l'égalité pour mettre en évidence une structure permettant de généraliser pour n'importe quel $n\in\N$.
	
	En $n=0$, la somme de gauche est 1, et l'expression à droite est $\frac{10 - 1}9 = \frac99 = 1$.
	
	En $n=1$, la somme de gauche est $1+10 = 11$, et l'expression à droite est $\frac{100-1}9 = \frac{99}9 = 11$.
	
	En $n=2$, la somme de gauche est $1+10+100 = 111$, et l'expression à droite est $\frac{1000-1}9 = \frac{999}9 = 111$.
	
	Remarquons que, plus généralement,
		\[ 1 + 10 + 10^2 + 10^3 + \cdots + 10^n = \underbrace{111\cdots111}_{\text{$n+1$ fois}}. \]
	Multiplier ce nombre par 9 puis ajouter 1 donne
		\[ 9 \bigl(1 + 10 + 10^2 + 10^3 + \cdots + 10^n\bigr) + 1 = \underbrace{999\cdots999}_{\text{$n+1$ fois}} + 1 = 1\underbrace{000\cdots000}_{\text{$n+1$ fois}} = 10^{n+1}. \]
	Soustraire 1 puis diviser par 9 conclut donc.
}

%%%%%%%%%%%%

\newpage
\fancyhead[C]{\textbf{Solutions}}
\shipoutAnswer

\end{document}

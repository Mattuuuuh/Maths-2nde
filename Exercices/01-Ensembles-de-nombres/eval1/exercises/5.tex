%!TEX root = ../eval1.tex

\exe{1, difficulty=1}{
	On considère comme admis que le nombre $\pi$, périmètre d'un cercle de diamètre 1, n'est pas rationnel.
	Montrer que le nombre $\pi + 2$ n'est pas rationnel non plus.
}{exe:irr-stable-pi}{
	Si $\pi+2 = \frac{a}b$ est rationnel, alors $\pi = \frac{a-2b}{b}$ doit l'être aussi.
	Ceci contredit l'énoncé.

	Alternativement, le développement décimal de $\pi=3,1415...$ est non périodique, d'après le cours, et par hypothèse.
	Ajouter 2 ne le change pas : $\pi+2=5,1415...$. Le développement décimal de $\pi+2$ est donc également apériodique, et c'est un nombre irrationnel.
}
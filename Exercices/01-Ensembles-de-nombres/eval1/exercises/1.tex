%!TEX root = ../eval1.tex

\exemulticols{2}{
	Compléter les pointillés en ajoutant un signe d'inclusion ou de non inclusion ($\subseteq$ ou $\not\subseteq$) entre les ensembles de nombres ci-contre.
}{
	\begin{center}
	\def\arraystretch{1.5}
	\setlength\tabcolsep{10pt}
	\begin{tabular}{ccc||ccc}
		$\N$ & \dots & $\Z$ & $\Q$ & \dots & $\Z$ \\
		$\Z$ & \dots & $\R$ & $\DD$ & \dots & $\Q$
	\end{tabular}
	\end{center}
}{exe:1}{
	On utilise le drapeau $\N \subseteq \Z \subseteq \DD \subseteq \Q \subseteq \R$ d'inclusions strictes vues en cours.
	\begin{center}
	\def\arraystretch{1.5}
	\setlength\tabcolsep{10pt}
	\begin{tabular}{ccc||ccc}
		$\N$ & $\subseteq$ & $\Z$ & $\Q$ & $\not\subseteq$ & $\Z$ \\
		$\Z$ & $\subseteq$ & $\R$ & $\DD$ & $\subseteq$ & $\Q$
	\end{tabular}
	\end{center}
}
% DYSLEXIA SWITCH
\newif\ifdys
		
				% ENABLE or DISABLE font change
				% use XeLaTeX if true
				\dystrue
				\dysfalse


\ifdys

\documentclass[a4paper, 14pt]{extarticle}
\usepackage{amsmath,amsfonts,amsthm,amssymb,mathtools}

\tracinglostchars=3 % Report an error if a font does not have a symbol.
\usepackage{fontspec}
\usepackage{unicode-math}
\defaultfontfeatures{ Ligatures=TeX,
                      Scale=MatchUppercase }

\setmainfont{OpenDyslexic}[Scale=1.0]
\setmathfont{Fira Math} % Or maybe try KPMath-Sans?
\setmathfont{OpenDyslexic Italic}[range=it/{Latin,latin}]
\setmathfont{OpenDyslexic}[range=up/{Latin,latin,num}]

\else

\documentclass[a4paper, 12pt]{extarticle}

\usepackage[utf8x]{inputenc}
%fonts
\usepackage{amsmath,amsfonts,amsthm,amssymb,mathtools}
% comment below to default to computer modern
\usepackage{libertinus,libertinust1math}

\fi


\usepackage[french]{babel}
\usepackage[
a4paper,
margin=2cm,
nomarginpar,% We don't want any margin paragraphs
]{geometry}
\usepackage{icomma}

\usepackage{fancyhdr}
\usepackage{array}
\usepackage{hyperref}

\usepackage{multicol, enumerate}
\newcolumntype{P}[1]{>{\centering\arraybackslash}p{#1}}


\usepackage{stackengine}
\newcommand\xrowht[2][0]{\addstackgap[.5\dimexpr#2\relax]{\vphantom{#1}}}

% theorems

\theoremstyle{plain}
\newtheorem{theorem}{Th\'eor\`eme}
\newtheorem*{sol}{Solution}
\theoremstyle{definition}
\newtheorem{ex}{Exercice}
\newtheorem*{rpl}{Rappel}
\newtheorem{enigme}{Énigme}

% corps
\usepackage{calrsfs}
\newcommand{\C}{\mathcal{C}}
\newcommand{\R}{\mathbb{R}}
\newcommand{\Rnn}{\mathbb{R}^{2n}}
\newcommand{\Z}{\mathbb{Z}}
\newcommand{\N}{\mathbb{N}}
\newcommand{\Q}{\mathbb{Q}}

% variance
\newcommand{\Var}[1]{\text{Var}(#1)}

% domain
\newcommand{\D}{\mathcal{D}}


% date
\usepackage{advdate}
\AdvanceDate[0]


% plots
\usepackage{pgfplots}

% table line break
\usepackage{makecell}
%tablestuff
\def\arraystretch{2}
\setlength\tabcolsep{15pt}

%subfigures
\usepackage{subcaption}

\definecolor{myg}{RGB}{56, 140, 70}
\definecolor{myb}{RGB}{45, 111, 177}
\definecolor{myr}{RGB}{199, 68, 64}

% fake sections with no title to move around the merged pdf
\newcommand{\fakesection}[1]{%
  \par\refstepcounter{section}% Increase section counter
  \sectionmark{#1}% Add section mark (header)
  \addcontentsline{toc}{section}{\protect\numberline{\thesection}#1}% Add section to ToC
  % Add more content here, if needed.
}


% SOLUTION SWITCH
\newif\ifsolutions
				\solutionstrue
				%\solutionsfalse

\ifsolutions
	\newcommand{\exe}[2]{
		\begin{ex} #1  \end{ex}
		\begin{sol} #2 \end{sol}
	}
\else
	\newcommand{\exe}[2]{
		\begin{ex} #1  \end{ex}
	}
	
\fi


% tableaux var, signe
\usepackage{tkz-tab}


%pinfty minfty
\newcommand{\pinfty}{{+}\infty}
\newcommand{\minfty}{{-}\infty}

\begin{document}


\AdvanceDate[0]

\begin{document}
\pagestyle{fancy}
\fancyhead[L]{Seconde 13}
\fancyhead[C]{\textbf{5 petits problèmes (+1)}}
\fancyhead[R]{\today}

\pagenumbering{gobble}

\centering
\hfill
\parbox{.3\textwidth}{
	\begin{exercice}{}{}
		Louis achète 1 DVD et 2 CD pour un total de 26€.
		\\\\
		Son frère Paul achète 2 DVD et 2 CD pour un total de 38€.
		\\\\
		Quel est le prix d'un DVD ? et d'un CD ?
	\end{exercice}	
}
\hfill
\parbox{.5\textwidth}{
	\begin{exercice}{}{}
		En voyage scolaire à Paris, Marie achète une petite Tour Eiffel et 3 porte-clés Arc de Triomphe pour 23€.
		\\\\
		Son amie Anne achète quant à elle 5 petites Tour Eiffel et 3 porte-clés Arc de Triomphe pour 55€.
		\\\\
		Combien coûte une petite Tour Eiffel ?
		et un arc de Triomphe en porte-clés ?
		\\\\
		Combien coûteraient 10 Tours Eiffel et 6 porte-clés ?
	\end{exercice}	
}
	
\parbox{.7\textwidth}{
	\begin{exercice}{}{}
		En 2023, les cotisation du club de danse se sont élevée à 2 712€ pour 32 enfants et 54 adultes.
		\\\\
		En 2024, sans changement de tarif, les cotisations ont rapporté seulement 2 616€ pour 16 enfants et 62 adultes.
		\\\\
		Quel est le montant des cotisations enfant et adulte ?
	\end{exercice}	
}


\parbox{.8\textwidth}{
	\begin{exercice}{}{}
		À la terrasse d'un café, le serveur annonce : \og 2 thés et 3 sodas, cela fera 12,20€ s'il vous plaît ! \fg
		\\\\
		Les clients ne sont pas d'accord et protestent : \og Non ! nous avions commandé 3 thés et 2 sodas ! \fg
		\\\\
		Le serveur, confus, corrige sa commande et revient en annonçant : \og vous avez gagné 80 centimes ! \fg
		\\\\
		Mais, au fait, combien coûte un thé ? et un soda ?
	\end{exercice}	
}

\hfill
\parbox{.3\textwidth}{
	\begin{exercice*}{bonus}{}
		Dans un troupeau de dromadaires et de chameaux, on compte 28 têtes et 45 bosses.
		\\\\
		Combien y a-t-il de dromadaires et de chameaux ?
	\end{exercice*}	
}
\hfill
\parbox{.6\textwidth}{
	\begin{exercice}{}{}
		Émilien explique : \og j'ai ajouté le triple d'un premier nombre à un second et j'ai trouvé 2. \fg
		\\\\
		Anissa lui répond : \og j'ai ajouté le double de ton premier nombre au double du second et j'ai trouvé 28. \fg
		\\\\
		Devine quels sont les nombres choisis par Émilien et Anissa.
	\end{exercice}	
}



\end{document}

%!TEX encoding = UTF8
%!TEX root =notes.tex


%%%%%%%%%%%%%%%%%%%%%%%%%%%%%%%%%
% PACKAGE IMPORTS
%%%%%%%%%%%%%%%%%%%%%%%%%%%%%%%%%


\usepackage[french]{babel}

\usepackage[tmargin=2cm,rmargin=1in,lmargin=1in,margin=0.85in,bmargin=2cm,footskip=.2in]{geometry}
\usepackage{amsmath,amsfonts,amsthm,amssymb,mathtools}
\usepackage[varbb]{newpxmath}
\usepackage{xfrac}
\usepackage[makeroom]{cancel}
\usepackage{mathtools}
\usepackage{bookmark}
\usepackage{enumitem}
\usepackage{hyperref,theoremref}
\hypersetup{
	pdftitle={Assignment},
	colorlinks=true, linkcolor=doc!90,
	bookmarksnumbered=true,
	bookmarksopen=true
}
\usepackage[most,many,breakable]{tcolorbox}
\usepackage{xcolor}
\usepackage{varwidth}
\usepackage{varwidth}
\usepackage{etoolbox}
%\usepackage{authblk}
\usepackage{nameref}
\usepackage{multicol,array}
\usepackage{tikz-cd}
\usepackage[ruled,vlined,linesnumbered]{algorithm2e}
\usepackage{comment} % enables the use of multi-line comments (\ifx \fi) 
\usepackage{import}
\usepackage{xifthen}
\usepackage{pdfpages}
\usepackage{transparent}


\newcommand\mycommfont[1]{\footnotesize\ttfamily\textcolor{blue}{#1}}
\SetCommentSty{mycommfont}
\newcommand{\incfig}[1]{%
    \def\svgwidth{\columnwidth}
    \import{./figures/}{#1.pdf_tex}
}

\usepackage{tikzsymbols}
%\renewcommand\qedsymbol{$\Laughey$}


%\usepackage{import}
%\usepackage{xifthen}
%\usepackage{pdfpages}
%\usepackage{transparent}


%%%%%%%%%%%%%%%%%%%%%%%%%%%%%%
% SELF MADE COLORS
%%%%%%%%%%%%%%%%%%%%%%%%%%%%%%



\definecolor{myg}{RGB}{56, 140, 70}
\definecolor{myb}{RGB}{45, 111, 177}
\definecolor{myr}{RGB}{199, 68, 64}
\definecolor{mytheorembg}{HTML}{F2F2F9}
\definecolor{mytheoremfr}{HTML}{00007B}
\definecolor{mylenmabg}{HTML}{FFFAF8}
\definecolor{mylenmafr}{HTML}{983b0f}
\definecolor{mypropbg}{HTML}{f2fbfc}
\definecolor{mypropfr}{HTML}{191971}
\definecolor{myexamplebg}{HTML}{F2FBF8}
\definecolor{myexamplefr}{HTML}{88D6D1}
\definecolor{myexampleti}{HTML}{2A7F7F}
\definecolor{mydefinitbg}{HTML}{E5E5FF}
\definecolor{mydefinitfr}{HTML}{3F3FA3}
\definecolor{notesgreen}{RGB}{0,162,0}
\definecolor{myp}{RGB}{197, 92, 212}
\definecolor{mygr}{HTML}{2C3338}
\definecolor{myred}{RGB}{127,0,0}
\definecolor{myyellow}{RGB}{169,121,69}
\definecolor{myexercisebg}{HTML}{F2FBF8}
\definecolor{myexercisefg}{HTML}{88D6D1}


%%%%%%%%%%%%%%%%%%%%%%%%%%%%
% TCOLORBOX SETUPS
%%%%%%%%%%%%%%%%%%%%%%%%%%%%

\setlength{\parindent}{1cm}
%================================
% THEOREM BOX
%================================

\tcbuselibrary{theorems,skins,hooks}
\newtcbtheorem[number within=chapter]{Theorem}{Théorème}
{%
	enhanced,
	breakable,
	colback = mytheorembg,
	frame hidden,
	boxrule = 0sp,
	borderline west = {2pt}{0pt}{mytheoremfr},
	sharp corners,
	detach title,
	before upper = \tcbtitle\par\smallskip,
	coltitle = mytheoremfr,
	fonttitle = \bfseries\sffamily,
	description font = \mdseries,
	separator sign none,
	segmentation style={solid, mytheoremfr},
}
{th}


\tcbuselibrary{theorems,skins,hooks}
\newtcolorbox{Theoremcon}
{%
	enhanced
	,breakable
	,colback = mytheorembg
	,frame hidden
	,boxrule = 0sp
	,borderline west = {2pt}{0pt}{mytheoremfr}
	,sharp corners
	,description font = \mdseries
	,separator sign none
}

%================================
% Corollery
%================================
\tcbuselibrary{theorems,skins,hooks}
\newtcbtheorem[use counter=tcb@cnt@Theorem]{Corollary}{Corollaire}
{%
	enhanced
	,breakable
	,colback = myp!10
	,frame hidden
	,boxrule = 0sp
	,borderline west = {2pt}{0pt}{myp!85!black}
	,sharp corners
	,detach title
	,before upper = \tcbtitle\par\smallskip
	,coltitle = myp!85!black
	,fonttitle = \bfseries\sffamily
	,description font = \mdseries
	,separator sign none
	,segmentation style={solid, myp!85!black}
}
{th}

%================================
% LENMA
%================================

\tcbuselibrary{theorems,skins,hooks}
\newtcbtheorem[use counter=tcb@cnt@Theorem]{Lemma}{Lemme}
{%
	enhanced,
	breakable,
	colback = mylenmabg,
	frame hidden,
	boxrule = 0sp,
	borderline west = {2pt}{0pt}{mylenmafr},
	sharp corners,
	detach title,
	before upper = \tcbtitle\par\smallskip,
	coltitle = mylenmafr,
	fonttitle = \bfseries\sffamily,
	description font = \mdseries,
	separator sign none,
	segmentation style={solid, mylenmafr},
}
{th}


%================================
% PROPOSITION
%================================

\tcbuselibrary{theorems,skins,hooks}
\newtcbtheorem[use counter=tcb@cnt@Theorem]{Prop}{Proposition}
{%
	enhanced,
	breakable,
	colback = mypropbg,
	frame hidden,
	boxrule = 0sp,
	borderline west = {2pt}{0pt}{mypropfr},
	sharp corners,
	detach title,
	before upper = \tcbtitle\par\smallskip,
	coltitle = mypropfr,
	fonttitle = \bfseries\sffamily,
	description font = \mdseries,
	separator sign none,
	segmentation style={solid, mypropfr},
}
{th}


%================================
% CLAIM
%================================

\tcbuselibrary{theorems,skins,hooks}
\newtcbtheorem[use counter=tcb@cnt@Theorem]{claim}{Claim}
{%
	enhanced
	,breakable
	,colback = myg!10
	,frame hidden
	,boxrule = 0sp
	,borderline west = {2pt}{0pt}{myg}
	,sharp corners
	,detach title
	,before upper = \tcbtitle\par\smallskip
	,coltitle = myg!85!black
	,fonttitle = \bfseries\sffamily
	,description font = \mdseries
	,separator sign none
	,segmentation style={solid, myg!85!black}
}
{th}



%================================
% Exercise
%================================

\tcbuselibrary{theorems,skins,hooks}
\newtcbtheorem[use counter=tcb@cnt@Theorem]{Exercise}{Exercice}
{%
	enhanced,
	breakable,
	colback = myexercisebg,
	frame hidden,
	boxrule = 0sp,
	borderline west = {2pt}{0pt}{myexercisefg},
	sharp corners,
	detach title,
	before upper = \tcbtitle\par\smallskip,
	coltitle = myexercisefg,
	fonttitle = \bfseries\sffamily,
	description font = \mdseries,
	separator sign none,
	segmentation style={solid, myexercisefg},
}
{th}

%================================
% EXAMPLE BOX
%================================

\newtcbtheorem[use counter=tcb@cnt@Theorem]{Example}{Exemple}
{%
	colback = myexamplebg
	,breakable
	,colframe = myexamplefr
	,coltitle = myexampleti
	,boxrule = 1pt
	,sharp corners
	,detach title
	,before upper=\tcbtitle\par\smallskip
	,fonttitle = \bfseries
	,description font = \mdseries
	,separator sign none
	,description delimiters parenthesis
}
{ex}

%================================
% DEFINITION BOX
%================================

\newtcbtheorem[use counter=tcb@cnt@Theorem]{Definition}{Définition}{enhanced,
	before skip=2mm,after skip=2mm, colback=red!5,colframe=red!80!black,boxrule=0.5mm,
	attach boxed title to top left={xshift=1cm,yshift*=1mm-\tcboxedtitleheight}, varwidth boxed title*=-3cm,
	boxed title style={frame code={
					\path[fill=tcbcolback]
					([yshift=-1mm,xshift=-1mm]frame.north west)
					arc[start angle=0,end angle=180,radius=1mm]
					([yshift=-1mm,xshift=1mm]frame.north east)
					arc[start angle=180,end angle=0,radius=1mm];
					\path[left color=tcbcolback!60!black,right color=tcbcolback!60!black,
						middle color=tcbcolback!80!black]
					([xshift=-2mm]frame.north west) -- ([xshift=2mm]frame.north east)
					[rounded corners=1mm]-- ([xshift=1mm,yshift=-1mm]frame.north east)
					-- (frame.south east) -- (frame.south west)
					-- ([xshift=-1mm,yshift=-1mm]frame.north west)
					[sharp corners]-- cycle;
				},interior engine=empty,
		},
	fonttitle=\bfseries,
	title={#2},#1}{def}

%================================
% Solution BOX
%================================

\makeatletter
\newtcbtheorem[use counter=tcb@cnt@Theorem]{question}{Question}{enhanced,
	breakable,
	colback=white,
	colframe=myb!80!black,
	attach boxed title to top left={yshift*=-\tcboxedtitleheight},
	fonttitle=\bfseries,
	title={#2},
	boxed title size=title,
	boxed title style={%
			sharp corners,
			rounded corners=northwest,
			colback=tcbcolframe,
			boxrule=0pt,
		},
	underlay boxed title={%
			\path[fill=tcbcolframe] (title.south west)--(title.south east)
			to[out=0, in=180] ([xshift=5mm]title.east)--
			(title.center-|frame.east)
			[rounded corners=\kvtcb@arc] |-
			(frame.north) -| cycle;
		},
	#1
}{def}
\makeatother

%================================
% SOLUTION BOX
%================================

\makeatletter
\newtcolorbox{solution}{enhanced,
	breakable,
	colback=white,
	colframe=myg!80!black,
	attach boxed title to top left={yshift*=-\tcboxedtitleheight},
	title=Solution,
	boxed title size=title,
	boxed title style={%
			sharp corners,
			rounded corners=northwest,
			colback=tcbcolframe,
			boxrule=0pt,
		},
	underlay boxed title={%
			\path[fill=tcbcolframe] (title.south west)--(title.south east)
			to[out=0, in=180] ([xshift=5mm]title.east)--
			(title.center-|frame.east)
			[rounded corners=\kvtcb@arc] |-
			(frame.north) -| cycle;
		},
}
\makeatother

%================================
% Question BOX
%================================

\makeatletter
\newtcbtheorem[use counter=tcb@cnt@Theorem]{qstion}{Question}{enhanced,
	breakable,
	colback=white,
	colframe=mygr,
	attach boxed title to top left={yshift*=-\tcboxedtitleheight},
	fonttitle=\bfseries,
	title={#2},
	boxed title size=title,
	boxed title style={%
			sharp corners,
			rounded corners=northwest,
			colback=tcbcolframe,
			boxrule=0pt,
		},
	underlay boxed title={%
			\path[fill=tcbcolframe] (title.south west)--(title.south east)
			to[out=0, in=180] ([xshift=5mm]title.east)--
			(title.center-|frame.east)
			[rounded corners=\kvtcb@arc] |-
			(frame.north) -| cycle;
		},
	#1
}{def}
\makeatother

\newtcbtheorem[number within=chapter]{wconc}{Wrong Concept}{
	breakable,
	enhanced,
	colback=white,
	colframe=myr,
	arc=0pt,
	outer arc=0pt,
	fonttitle=\bfseries\sffamily\large,
	colbacktitle=myr,
	attach boxed title to top left={},
	boxed title style={
			enhanced,
			skin=enhancedfirst jigsaw,
			arc=3pt,
			bottom=0pt,
			interior style={fill=myr}
		},
	#1
}{def}



%================================
% NOTE BOX
%================================

\usetikzlibrary{arrows,calc,shadows.blur}
\tcbuselibrary{skins}
\newtcolorbox{note}[1][]{%
	enhanced jigsaw,
	colback=gray!20!white,%
	colframe=gray!80!black,
	size=small,
	boxrule=1pt,
	title=\colorbox{white!100}{\textbf{ Remarque }},
	halign title=flush center,
	coltitle=black,
	breakable,
	drop shadow=black!50!white,
	attach boxed title to top left={xshift=1cm,yshift=-\tcboxedtitleheight/2,yshifttext=-\tcboxedtitleheight/2},
	minipage boxed title=2.6cm,
	boxed title style={%
			colback=white,
			size=fbox,
			boxrule=1pt,
			boxsep=2pt,
			underlay={%
					\coordinate (dotA) at ($(interior.west) + (-0.5pt,0)$);
					\coordinate (dotB) at ($(interior.east) + (0.5pt,0)$);
					\begin{scope}
						\clip (interior.north west) rectangle ([xshift=3ex]interior.east);
						\filldraw [white, blur shadow={shadow opacity=60, shadow yshift=-.75ex}, rounded corners=2pt] (interior.north west) rectangle (interior.south east);
					\end{scope}
					\begin{scope}[gray!80!black]
						\fill (dotA) circle (2pt);
						\fill (dotB) circle (2pt);
					\end{scope}
				},
		},
	#1,
}

%================================
% STRATÉGIE BOX
%================================

\usetikzlibrary{arrows,calc,shadows.blur}
\tcbuselibrary{skins}
\newtcolorbox{strategy}[1][]{%
	enhanced jigsaw,
	colback=myb!20!white,%
	colframe=gray!80!black,
	size=small,
	boxrule=1pt,
	title=\colorbox{white!100}{\textbf{ Stratégie }},
	halign title=flush center,
	coltitle=black,
	breakable,
	drop shadow=black!50!white,
	attach boxed title to top left={xshift=1cm,yshift=-\tcboxedtitleheight/2,yshifttext=-\tcboxedtitleheight/2},
	minipage boxed title=2.5cm,
	boxed title style={%
			colback=white,
			size=fbox,
			boxrule=1pt,
			boxsep=2pt,
			underlay={%
					\coordinate (dotA) at ($(interior.west) + (-0.5pt,0)$);
					\coordinate (dotB) at ($(interior.east) + (0.5pt,0)$);
					\begin{scope}
						\clip (interior.north west) rectangle ([xshift=3ex]interior.east);
						\filldraw [white, blur shadow={shadow opacity=60, shadow yshift=-.75ex}, rounded corners=2pt] (interior.north west) rectangle (interior.south east);
					\end{scope}
					\begin{scope}[gray!80!black]
						\fill (dotA) circle (2pt);
						\fill (dotB) circle (2pt);
					\end{scope}
				},
		},
	#1,
}

%================================
% MÉTHODE BOX
%================================

\usetikzlibrary{arrows,calc,shadows.blur}
\tcbuselibrary{skins}
\newtcolorbox{methode}[1][]{%
	enhanced jigsaw,
	colback=white,%
	colframe=gray!80!black,
	size=small,
	boxrule=1pt,
	title=\textbf{Méthode},
	halign title=flush center,
	coltitle=black,
	breakable,
	drop shadow=black!50!white,
	attach boxed title to top left={xshift=1cm,yshift=-\tcboxedtitleheight/2,yshifttext=-\tcboxedtitleheight/2},
	minipage boxed title=2.5cm,
	boxed title style={%
			colback=white,
			size=fbox,
			boxrule=1pt,
			boxsep=2pt,
			underlay={%
					\coordinate (dotA) at ($(interior.west) + (-0.5pt,0)$);
					\coordinate (dotB) at ($(interior.east) + (0.5pt,0)$);
					\begin{scope}
						\clip (interior.north west) rectangle ([xshift=3ex]interior.east);
						\filldraw [white, blur shadow={shadow opacity=60, shadow yshift=-.75ex}, rounded corners=2pt] (interior.north west) rectangle (interior.south east);
					\end{scope}
					\begin{scope}[gray!80!black]
						\fill (dotA) circle (2pt);
						\fill (dotB) circle (2pt);
					\end{scope}
				},
		},
	#1,
}

%%%%%%%%%%%%%%%%%%%%%%%%%%%%%%%%%%%%%%%%%%%
% TABLE OF CONTENTS
%%%%%%%%%%%%%%%%%%%%%%%%%%%%%%%%%%%%%%%%%%%

\usepackage{tikz}

\definecolor{doc}{RGB}{0,60,110}
\usepackage{titletoc}
\contentsmargin{0cm}
\titlecontents{chapter}[3.7pc]
{\addvspace{30pt}%
	\begin{tikzpicture}[remember picture, overlay]%
		\draw[fill=doc!60,draw=doc!60] (-7,-.1) rectangle (-0.2,.6);%
		\pgftext[left,x=-3.5cm,y=0.2cm]{\color{white}\Large\sc\bfseries Chapitre\ \thecontentslabel};%
	\end{tikzpicture}\color{doc!60}\large\sc\bfseries}%
{}
{}
{\;\titlerule\;\large\sc\bfseries Page \thecontentspage
	\begin{tikzpicture}[remember picture, overlay]
		\draw[fill=doc!60,draw=doc!60] (2pt,0) rectangle (4,0.1pt);
	\end{tikzpicture}}%
\titlecontents{section}[3.7pc]
{\addvspace{2pt}}
{\contentslabel[\thecontentslabel]{2pc}}
{}
{\hfill\small \thecontentspage}
[]
\titlecontents*{subsection}[3.7pc]
{\addvspace{-1pt}\small}
{}
{}
{\ --- \small\thecontentspage}
[ \textbullet\ ][]

\makeatletter
\renewcommand{\tableofcontents}{%
	\chapter*{%
	  \vspace*{-20\p@}%
	  \begin{tikzpicture}[remember picture, overlay]%
		  \pgftext[right,x=15cm,y=0.2cm]{\color{doc!60}\Huge\sc\bfseries \contentsname};%
		  \draw[fill=doc!60,draw=doc!60] (13,-.75) rectangle (20,1);%
		  \clip (13,-.75) rectangle (20,1);
		  \pgftext[right,x=15cm,y=0.2cm]{\color{white}\Huge\sc\bfseries \contentsname};%
	  \end{tikzpicture}}%
	\@starttoc{toc}}
\makeatother


%%%%%%%%%%%%%%%%%%%%%%%%%%%%%%%%%%%%%%%%%%%
% MINTED FOR PYTHON ALGORITHMS
%%%%%%%%%%%%%%%%%%%%%%%%%%%%%%%%%%%%%%%%%%%

\usepackage{tcolorbox}
\tcbuselibrary{minted,breakable,xparse,skins}
\definecolor{bg}{gray}{0.95}
\DeclareTCBListing{mintedbox}{O{}m!O{}}{%
  breakable=true,
  listing engine=minted,
  listing only,
  minted language=#2,
  minted style=default,
  minted options={%
    linenos,
    gobble=0,
    breaklines=true,
    breakafter=,,
    fontsize=\small,
    numbersep=8pt,
    #1},
  boxsep=0pt,
  left skip=0pt,
  right skip=0pt,
  left=25pt,
  right=0pt,
  top=3pt,
  bottom=3pt,
  arc=5pt,
  leftrule=0pt,
  rightrule=0pt,
  bottomrule=2pt,
  toprule=2pt,
  colback=bg,
  colframe=orange!70,
  enhanced,
  overlay={%
    \begin{tcbclipinterior}
    \fill[orange!20!white] (frame.south west) rectangle ([xshift=20pt]frame.north west);
    \end{tcbclipinterior}},
  #3}
  
  
 % for braces
\usetikzlibrary{decorations.pathreplacing}


\SetDate[10/02/2026]

\begin{document}
\pagestyle{fancy}
\fancyhead[L]{Seconde}
\fancyhead[C]{\textbf{Vecteurs}}
\fancyhead[R]{\today}

\subsection*{Automatismes}

\exemulticols{}{
	Placer les points suivants dans le repère ci-contre.
		\begin{multicols}{2}
		\begin{enumerate}[label=,itemsep=10pt]
			\item $A(2;3)$
			\item $B(-2;2)$
			\item $C(2;-1)$
			\item $D(-2;-1)$
		\end{enumerate}
		\end{multicols}
		\vfill\,
}{
	\centering
	\begin{tikzpicture}[>=stealth, scale=.875]
		\begin{axis}[xmin = -3.5, xmax=3.5, ymin=-3.5, ymax=3.5, axis x line=middle, axis y line=middle, axis line style=<->, xlabel={}, ylabel={}, xtick distance=1, ytick distance=1, grid=both]
		\end{axis}
	\end{tikzpicture}
}{exe:aut1}{
	\begin{center}
	\begin{tikzpicture}[>=stealth, scale=.875]
		\begin{axis}[xmin = -3.5, xmax=3.5, ymin=-3.5, ymax=3.5, axis x line=middle, axis y line=middle, axis line style=<->, xlabel={}, ylabel={}, xtick distance=1, ytick distance=1, grid=both]
		\addplot[BLUE_E, mark=*, mark size = 1] (2,3) node[right] {$A(2;3)$};
		\addplot[GREEN_E, mark=*, mark size = 1] (-2,2) node[above] {$B(-2;2)$};
		\addplot[RED_E, mark=*, mark size = 1] (2,-1) node[right] {$C(2;-1)$};
		\addplot[PINK, mark=*, mark size = 1] (-2,-1) node[below] {$D(-2;-1)$};
		\end{axis}
	\end{tikzpicture}
	\end{center}
}
\vspace{-40pt}

\exe{}{
	Calculer la différence $x-y$ lorsque
		\begin{multicols}{3}
		\begin{enumerate}[itemsep=10pt]
			\item $x=10$ et $y=7$ 
			\item $x = 7$ et $y = 10$
			\item $x=-6$ et $y = 5$
			\item $x = -7$ et $y=-11$
			\item $x=0$ et $y =-4$
			\item $x =-1,4$ et $y=-0,7$
		\end{enumerate}
		\end{multicols}
}{exe:aut2}{
	\begin{enumerate}[itemsep=10pt]
			\item $10-7 = 3$
			\item $7-10 = -3$
			\item $-6-5 = -11$
			\item $-7-(-11) = -7+11 = 4$
			\item $0 - (-4)= 4$
			\item $-1,4 - (-0,7) = -1,4+0,7 = -0,7$
	\end{enumerate}
}

%^\vspace{-10pt}
\subsection*{Exercices}

\exe{}{
	Placer les points $A(-4 ;3), B(2 ; 2), C(0 ; -2)$ dans un repère et répondre au questions.
	
	\begin{enumerate}
		\item Quelle translation $\vec{AB}$ effectuer pour envoyer le point $A$ sur le point $B$ ?
		\item Quelle translation $\vec{BC}$ effectuer pour envoyer le point $B$ sur le point $C$ ?
		\item Quelle translation $\vec{AC}$ effectuer pour envoyer le point $A$ sur le point $C$ ?
		\item Comparer $\vec{AC}$ et $\vec{AB} + \vec{BC}$ et compléter la phrase suivante.
			\begin{center}
				%\og Envoyer $A$ sur $B$, puis $B$ sur $C$ correspond à envoyer $A$ sur $\dots$ \fg
				\og L'addition de la translation qui envoie $A$ sur $B$ à celle qui envoie $B$ sur $C$ \\[10pt] est égale à la translation qui envoie $A$ sur $\dots$. \fg
			\end{center}
		%\item Comparer $\vec{AC}$ et $\vec{AB} + \vec{BC}$.
		
		%\item Pour $A, B, C$ trois points quelconques, démontrer l'égalité
		%	\[ \vec{AC} = \vec{AB} + \vec{BC}. \]
	\end{enumerate}

}{exe:intro1}{
	
	\begin{enumerate}
		\item On calcule
			\[ \vec{AB} = B - A = \pvec{2 - (-4)}{2-3} = \pvec{6}{-1}. \]
		\item On calcule
			\[ \vec{BC} = C - B = \pvec{0 - 2}{-2 - 2} = \pvec{-2}{-4}. \]
		\item On calcule
			\[ \vec{AC} = C - A = \pvec{0 - (-4)}{-2 -3} = \pvec{4}{-5}. \]
		\item
			La somme $\vec{AB} + \vec{BC}$ est donnée par
				\[ \vec{AB} + \vec{BC} = \pvec{6}{-1} + \pvec{-2}{-4} = \pvec{6 - 2}{-1-4} = \pvec{4}{-5} = \vec{AC}. \]
			\begin{center}
				\og L'addition de la translation qui envoie $A$ sur $B$ à celle qui envoie $B$ sur $C$ \\[10pt] est égale à la translation qui envoie $A$ sur {\color{RED_E} $C$}. \fg
			\end{center}
		%\item
		%	On a, plus généralement, $\vec{AB} + \vec{BC} = B - A + C - B = C - A = \vec{AC}$.
	\end{enumerate}
}


%\begin{figure}[h!]
%	\begin{subfigure}{.5\textwidth}
%		\begin{tikzpicture}[>=stealth, scale=1.1]
%		\begin{axis}[xmin = -10.5, xmax=10.5, ymin=-10.5, ymax=10.5, axis x line=middle, axis y line=middle, axis line style=<->, xlabel={}, ylabel={}, xtick = {-10, -8, ..., 8, 10}, ytick = {-10, -8, ..., 8, 10}, grid=both]
%			
%		\end{axis}
%		\end{tikzpicture}
%	\caption{Repère pour l'exercice \ref{exe:intro1}.}
%	\end{subfigure}
%	\begin{subfigure}{.5\textwidth}
%		\begin{tikzpicture}[>=stealth, scale=1.1]
%		\begin{axis}[xmin = -10.5, xmax=10.5, ymin=-10.5, ymax=10.5, axis x line=middle, axis y line=middle, axis line style=<->, xlabel={}, ylabel={}, xtick = {-10, -8, ..., 8, 10}, ytick = {-10, -8, ..., 8, 10}, grid=both]
%			
%		\end{axis}
%		\end{tikzpicture}
%	\caption{Repère pour l'exercice \ref{exe:intro2}.}
%	\end{subfigure}
%\end{figure}


\exe{}{
	Calculer les coordonnées puis dessiner les vecteurs suivants dans un repère.
		\begin{align*}
			u = \pvec{-2}{4}, && v = 2u, && w = -\dfrac32u, && z = -\dfrac12u,
		\end{align*}	
	\begin{enumerate}
		\item Compléter les phrases suivantes.
			\begin{center}
				\og Lorsque le vecteur $u$ est multiplié par $\dots$ pour obtenir $v$, son sens $\dots\dots\dots\dots\dots\dots$ \fg \\[10pt]
				\og Lorsque le vecteur $u$ est multiplié par $\dots$ pour obtenir $w$, son sens$\dots\dots\dots\dots\dots\dots$ \fg \\[10pt]
				\og Lorsque le vecteur $u$ est multiplié par  $\dots$ pour obtenir $z$, son sens $\dots\dots\dots\dots\dots\dots$ \fg
			\end{center}
		\item Calculer les normes de $u, v, w,$ et $z$ et les écrire sous forme $a\sqrt{b}$ où $a, b\in\N$ et $b$ est le plus petit possible.
		\item Compléter les phrases suivantes.
			\begin{center}
				\og Lorsque le vecteur $u$ est multiplié par $\dots$ pour obtenir $v$, sa norme $\Vert u\Vert$ est multipliée par $\dots$. \fg \\[10pt]
				\og Lorsque le vecteur $u$ est multiplié par $\dots$  pour obtenir $w$, sa norme $\Vert u\Vert$ est multipliée par $\dots$. \fg \\[10pt]
				\og Lorsque le vecteur $u$ est multiplié par $\dots$ pour obtenir $z$, sa norme $\Vert u\Vert$ est multipliée par $\dots$. \fg
			\end{center}
	\end{enumerate}

}{exe:intro2}{
	\begin{enumerate}
		\item
			\begin{center}
				\og Lorsque le vecteur $u$ est multiplié par {\color{RED_E} $2$} pour obtenir $v$, son sens {\color{RED_E} ne change pas} \fg \\[10pt]
				\og Lorsque le vecteur $u$ est multiplié par  {\color{RED_E} $-\dfrac32$} pour obtenir $w$, son sens {\color{RED_E} change} \fg \\[10pt]
				\og Lorsque le vecteur $u$ est multiplié par  {\color{RED_E} $-\dfrac12$} pour obtenir $z$, son sens {\color{RED_E} change} \fg
			\end{center}
		\item
			\begin{align*}
				\norm{u} &= \sqrt{(-2)^2 + 4^2} = \sqrt{4 + 16} = \sqrt{20} = \sqrt{4 \times 5} = 2 \sqrt{5}. \\
				\norm{v} &= \sqrt{(-4)^2 + 8^2} = \sqrt{16 + 64} = \sqrt{80} = \sqrt{16 \times 5} = 4\sqrt{5}. \\
				\norm{w} &= \sqrt{3^2 + (-6)^2} = \sqrt{9 + 36} = \sqrt{45} = \sqrt{9 \times 5} = 3 \sqrt{5}. \\
				\norm{z} &= \sqrt{1^2 + (-2)^2} = \sqrt{1 + 4} = \sqrt{5}.
			\end{align*}
	
		On remarque qu'on a les relations suivantes.
			\begin{align*}
				\norm{v} = 2 \norm{u} && \norm{w} = \dfrac32 \norm{u} && \norm{z} = \dfrac12 \norm{u}
			\end{align*}
		\item
			\begin{center}
				\og Lorsque le vecteur $u$ est multiplié par {\color{RED_E} $2$} pour obtenir $v$, sa norme $\Vert u\Vert$ est multipliée par {\color{RED_E} $2$}. \fg \\[10pt]
				\og Lorsque le vecteur $u$ est multiplié par {\color{RED_E} $-\frac32$} pour obtenir $w$, sa norme $\Vert u\Vert$ est multipliée par {\color{RED_E} $\frac32$}. \fg \\[10pt]
				\og Lorsque le vecteur $u$ est multiplié par {\color{RED_E} $-\frac12$} pour obtenir $z$, sa norme $\Vert u\Vert$ est multipliée par {\color{RED_E} $\frac12$}. \fg
			\end{center}
	\end{enumerate}
}

\exe{}{
	À l'aide de la figure suivante et de la relation de Chasles, déterminer les sommes vectorielles suivantes en complétant les pointillés.

		\begin{center}
		\begin{tikzpicture}[>=stealth, scale=1]
		\newcommand\len{.8cm}
		\begin{axis}[xmin = 0, x=\len, y= \len, xmax=8, ymin=0, ymax=4, axis line style={transparent}, xlabel={}, ylabel={}, ticks = none, grid = both, enlargelimits={abs=0.8}, minor x tick num = 1, minor y tick num = 1]
			\draw[very thick, -, RED_E] (axis cs:0,0) -- (axis cs: 8,0);
			\draw[very thick, -, RED_E] (axis cs:0,4) -- (axis cs: 8,4);
			\draw[very thick, -, RED_E] (axis cs:0,0) -- (axis cs: 0,4);
			\draw[very thick, -, RED_E] (axis cs:8,0) -- (axis cs: 8,4);
			\draw[very thick, -, RED_E] (axis cs:4,0) -- (axis cs: 4,4);
			\draw[very thick, -, RED_E] (axis cs:0,0) -- (axis cs: 4,4);
			\draw[very thick, -, RED_E] (axis cs:0,4) -- (axis cs: 4,0);
			\draw[very thick, -, RED_E] (axis cs:4,0) -- (axis cs: 8,4);
			\draw[very thick, -, RED_E] (axis cs:4,4) -- (axis cs: 8,0);
			
			\addplot[black] (0,0) node[below] {$A$};
			\addplot[black] (4,0) node[below] {$B$};
			\addplot[black] (8,0) node[below] {$C$};
			\addplot[black] (8,4) node[above] {$D$};
			\addplot[black] (4,4) node[above] {$E$};
			\addplot[black] (0,4) node[above] {$F$};
			\addplot[black] (2,2) node[below=2pt] {$G$};
			\addplot[black] (6,2) node[below=2pt] {$H$};
			
		\end{axis}
		\end{tikzpicture}
		%\vline
	\end{center}
	
	\begin{multicols}{2}
	\begin{enumerate}
		\item $\vec{BC} = \vec{E \dots}$
		\item $\vec{BE} = \vec{\dots F}$
		\item $\vec{AB} + \vec{FA} = \vec{F \dots} + \vec{\dots B} = \vec{\vphantom{A} \dots\dots}$
		\item $\vec{FD} + \vec{EB} = \vec{FD} + \vec{D\dots} = \vec{\vphantom{A}\dots\dots}$
		\item $\vec{AE} + \vec{GB} + \vec{DH} = \vec{AE} + \vec{E\dots} + \vec{\vphantom{A}\dots\dots} = \vec{\vphantom{A}\dots\dots}$
	\end{enumerate}
	\end{multicols}

}{exe:Chasles}{
	\begin{multicols}{2}
	\begin{enumerate}
		\item $\vec{BC} = \vec{ED}$
		\item $\vec{BE} = \vec{AF}$
		\item $\vec{AB} + \vec{FA} = \vec{FE} + \vec{EB} = \vec{FB}$
		\item $\vec{FD} + \vec{EB} = \vec{FD} + \vec{DC} = \vec{FC}$
		\item $\vec{AE} + \vec{GB} + \vec{DH} = \vec{AE} + \vec{EH} + \vec{HB} = \vec{AB}$
	\end{enumerate}
	\end{multicols}
}

\exe{}{
	Construire, dans le plan vierge de droite, les sommes 
		\[u+v+w \qquad \text{ et } \qquad \dfrac12u - 2v - w,\]
	où les vecteurs $u, v, w$ sont donnés dans le plan de gauche ci-dessous.
	
	%\begin{center}
		\begin{tikzpicture}[>=stealth, scale=1]
		\begin{axis}[xmin = -10, xmax=10, ymin=-10, ymax=10, axis x line=none, axis y line=none, axis line style=<->, xlabel={}, ylabel={}, ticks = none]
			\draw[very thick, ->, GREEN_E] (axis cs:-3,-4) -- (axis cs: 0,0) node[above] {$u$};
			\draw[very thick, ->, RED_E] (axis cs:-2,0) -- (axis cs: -5,6) node[above] {$v$};
			\draw[very thick, ->, BLUE_E] (axis cs:0,3) -- (axis cs: 8,-2) node[above] {$w$};
		\end{axis}
		\end{tikzpicture}
		\vline
	%\end{center}
}{exe:rep-graph}{
	\begin{center}
		\begin{tikzpicture}[>=stealth, scale=1]
		\begin{axis}[xmin = -10, xmax=20, ymin=-15, ymax=10, axis x line=none, axis y line=none, axis line style=<->, xlabel={}, ylabel={}, ticks = none]
			\draw[very thick, ->, black, dotted] (axis cs:-3,-4) -- (axis cs: 0,0) node[pos=.5, above left] {$u$};
			\draw[very thick, ->, black, dotted] (axis cs:-2+2,0) -- (axis cs: -5+2,6) node[pos=.5, left] {$v$};
			\draw[very thick, ->, black, dotted] (axis cs:0-3,3+3) -- (axis cs: 8-3,-2+3) node[pos=.5, above] {$w$};
			
			\draw[very thick, ->, RED_E] (axis cs:-3,-4) -- (axis cs: 8-3,-2+3) node[pos=.5, below right] {$u+v+w$};
			
			
			
			\draw[very thick, ->, black, dotted] (axis cs:-3+15,-4) -- (axis cs:-1.5+15, -2) node[pos=.5, above left] {$\frac12u$};
			\draw[very thick, ->, black, dotted] (axis cs:-2+2-1.5+15, 0-2) -- (axis cs: 4+2-1.5+15, -12-2) node[pos=.5, above right] {$-2v$};
			\draw[very thick, ->, black, dotted] (axis cs:0+4+2-1.5+15,3-3-12-2) -- (axis cs: -8+4+2-1.5+15, 8-3-12-2) node[pos=.5, below left] {$-w$};
			
			\draw[very thick, ->, GREEN_E](axis cs:-3+15,-4) -- (axis cs: -8+4+2-1.5+15, 8-3-12-2) node[pos=.8, left] {$\frac12u-2v-w$};
		\end{axis}
		\end{tikzpicture}
	\end{center}
	
	On met les vecteurs bout à bout pour créer la somme.
	La multiplication par un scalaire ne change pas la direction mais multiplie la norme et peut changer le sens si celui-ci est négatif.
}

\exe{, difficulty=1}{
	Soient $A(3;-1), B(-1; -5),$ et $C(3 ; 4)$.
	\begin{multicols}{2}
	\begin{enumerate}
		\item Calculer $\vec{AB}, \vec{BA}$, et $\vec{CA}$.
		\item Calculer $\norm{\vec{AB}}, \norm{\vec{BA}}$, $\norm{\vec{AC}}$.
		\item Calculer $3 \vec{AB} + 3 \vec{CA}$ et $3 \vec{CB}$.
		\item Calculer $\norm{-4 \vec{AB}}$ et $\norm{-\dfrac1{13} \vec{CA}}$.
	\end{enumerate}
	\end{multicols}

}{exe:calcul}{

	\begin{enumerate}
		\item 
			\begin{align*}
				\vec{AB} = B - A = \pvec{-4}{-4} && \vec{BA} = - \vec{AB} = \pvec44 && \vec{CA} = A - C = \pvec0{-5}.
			\end{align*}
		\item
			\begin{align*}
				\norm{\vec{AB}} &= \sqrt{(-4)^2 + (-4)^2} = 4\sqrt{2} \\ \norm{\vec{BA}} &= \norm{- \vec{AB}} = \norm{\vec{AB}} = 4\sqrt{2} \\ \norm{\vec{CA}} &= \sqrt{(-5)^2} = |-5| = 5.
			\end{align*}
		\item
			On utilise la relation de Chasles $\vec{CB} = \vec{CA} + \vec{AB}$ pour obtenir immédiatement le résultat.
			\begin{align*}
				3 \vec{AB} + 3 \vec{CA} &= 3 \pvec{-4}{-4} + 3 \pvec0{-5} = \pvec{-12}{-27} \\
				3 \vec{CB} &= 3 \left( \vec{CA} + \vec{AB} \right) =  3 \vec{AB} + 3 \vec{CA} = \pvec{-12}{-27}
			\end{align*}
		\item
			\begin{gather*}
				\norm{-4 \vec{AB}} = |-4| \cdot \norm{\vec{AB}} = 4\norm{\vec{AB}} = 16\sqrt{2}  \\
				\norm{-\dfrac1{13} \vec{CA}} = \dfrac1{13} \norm{\vec{CA}} = \dfrac5{13}
			\end{gather*}
			
	\end{enumerate}

}

\exe{, difficulty=1}{
	Considérons le point $A(1;3)$ et les vecteurs $u = \pvec62, v = \pvec1{-3}$.
	Posons $B=A+u$ et $C = A+v$.
	Tracer le triangle $ABC$ dans un repère orthonormé et démontrer qu'il est rectangle en $A$. 
	
	\emph{Formule de la longueur : $AB^2 = (x_A - x_B)^2 + (y_A - y_B)^2$.}
}{exe:Pyth}{
	D'abord, $B = A +u = (1+6 ; 3+2) = (7;5)$ et $C = A+v = (1+1 ; 3 - 3) = (2 ; 0)$.
	
	\begin{center}
	\begin{tikzpicture}[>=stealth, scale=1]
		\begin{axis}[xmin = -0.9, xmax=7.9, ymin=-0.9, ymax=5.9, axis x line=middle, axis y line=middle, axis line style=<->, xlabel={}, ylabel={}, grid=both, grid style = {opacity=.5}, xtick distance=1, ytick distance = 1]
		
			\addplot[GREY, mark=*, mark size = 1] (0,0) node[below=8pt, left] {$O$};
			\draw[dashed, PINK, thick] (axis cs:1,3) -- (axis cs:7,5) -- (axis cs:2,0) -- (axis cs:1,3);
			
			\addplot[BLUE_E, mark=*, mark size = 1] (1,3) node[above] {$A(1 ; 3)$};
			\addplot[RED_E, mark=*, mark size = 1] (7,5) node[above left] {$B(7 ; 5)$};
			\addplot[GREEN_E, mark=*, mark size = 1] (2,0) node[above right] {$C(2; 0)$};
			
		\end{axis}
	\end{tikzpicture}
	\end{center}
	
	Ensuite, pour montrer que $ABC$ est rectangle en $A$, la réciproque du théorème de Pythagore énonce qu'il faut montrer l'égalité
		\[ BC^2 \stackrel{?}{=} AB^2 + AC^2. \]
	D'après la formule de la longueur, d'une part
		\[ BC^2 = (7-2)^2 + (0-5)^2 = 25+25=50, \]
	et d'autre part
		\[ AB^2 + AC^2 = (7-1)^2 + (5-3)^2 + (2-1)^2 + (0-3)^2 = 36+4+1+9 = 50, \]
	ce qui conclut.
}

\exe{,difficulty=1}{
	Soit $u = \pvec{4}{2}$ et $v = \pvec{1}{3}$ deux vecteurs.
	Représenter dans un repère les points
		\begin{align*}
			A(-3;-1), && B = A+u, && C = A+u+v, && \text{ et } && D=A+v,
		\end{align*}
	et montrer que quadrilatère $ABCD$ est un parallélogramme en montrant que les droites $(AB)$ et $(CD)$ sont parallèles puis que les droites $(AD)$ et $(BC)$ sont parallèles.
}{exe:parall}{
	\begin{enumerate}
		\item $ABCD$ semble être un parallélogramme car ses cotés opposés sont parallèles deux à deux.
		\item On calcule $\vec{AB} = B-A = u$ et $\vec{CD} = D-C = A+v - (A+u+v) = -u$ pour voir qu'ils sont colinéaires : les droites $(AB)$ et $(CD)$ sont donc parallèles.
		
		Idem avec $\vec{AD} = D - A = v$ et $\vec{BC} = C - B = A + u + v - (A + u) = v$, ce qui conclut similairement. 
	\end{enumerate}
	
}

\exe{, difficulty=1}{
	Considérons les points $A(3;2), B(-3 ; 7), C(-2; -3),$ et  $D(4;-6)$.
	
	Les droites $(AB)$ et $(CD)$ sont-elles parallèles ?
	
	Si non, donner un point $\tilde{B}$ tel que les droites $(A\tilde{B})$ et $(CD)$ soient parallèles.

}{exe:parall2}{
	Il s'agit de vérifier si $\vec{AB}$ et $\vec{CD}$ sont colinéaires.
	Calculons donc
		\[ \vec{AB} = \pvec{-6}{5}, \qquad \qquad \vec{CD} = \pvec{6}{-3}. \]
	Les vecteurs ne sont pas colinéaires : s'il existait un $k\in\R$ tel que $\vec{AB} = k \vec{CD}$, on aurait nécessairement $k=-1$ par la première coordonnée, et $k=-\dfrac53 \neq -1$ par le seconde.
	Un calcul de déterminant permet également de montrer que les vecteurs ne sont pas colinéaires, celui-ci étant non nul.
	Les droites ne sont donc pas parallèles.
	
	Pour que $(A\tilde{B})$ et $(CD)$ soient parallèles, elles doivent partager un vecteur directeur.
	Par exemple, le vecteur $\vec{CD}$ convient.
	
	On peut alors choisir $\tilde{B}$ parmis les $A+k \cdot \vec{CD}$ où $k\in\R$ est un scalaire quelconque.
	En prenant $k=-\frac13$ pour montrer qu'on aime les fractions, on peut définir $\tilde{B} = (1 ; 3)$.
	
	On vérifiera que $\vec{A\tilde{B}} = \pvec{-2}{1}$, qui est bien colinéaire à $\vec{CD} = \pvec{6}{-3}$.
}


%\newpage


\exe{}{
	Considérons les points
		\begin{align*}
			A(-1; -6), && B(-3; -12), && C(5; 12), && D(7 ; 17).
		\end{align*}
	
	\begin{enumerate}
		\item Les points $A, B, C$ sont-ils alignés ?
		\item Les point $D, C, B$ sont-ils alignés ?
		\item Trouver un point $E$ distinct de $A, B, C, D$ et appartenant à $(AB)$.
	\end{enumerate}
}{exe:alignement}{
	\begin{enumerate}
		\item
		Pour déterminer si $A, B, C$ sont alignés, calculons par exemple $\vec{AB}$ et $\vec{BC}$ et regardons s'ils sont colinéaires.
			\begin{align*}
				\vec{AB} = \pvec{-3-(-1)}{-12-(-6)} = \pvec{-2}{-6} && \vec{BC} = \pvec{5-(-3)}{12-(-12)} = \pvec{8}{24}
			\end{align*}
		Soit on remarque que $-4\vec{AB} = \vec{BC}$, soit on calcule $\det\left(\vec{AB}, \vec{BC}\right) = (-2)(24)-(-6)(8) = -48+48 = 0$ pour voir que les vecteurs sont bien colinéaires.
		
		Par conséquent, les droites $(AB)$ et $(BC)$ sont parallèles et partagent le point $B$.
		Elles sont donc confondues et les points $A, B, C$ sont alignés.
		\item 
		Nous avons déjà calculé $\vec{BC}$. Calculons désormais $\vec{DC}$ :
			\[ \vec{DC} = \pvec{5-7}{12-17} = \pvec{-2}{-5}. \]
		Ce vecteur n'est pas colinéaire à $\vec{BC}$ car $\det\left(\vec{BC}, \vec{DC}\right) = (8)(-5)-(24)(-2) = -40+48 \neq 0$.
		\item 
		La droite $(AB)$ peut être vue comme l'ensemble des translatés de $A$ (ou de $B$) par les multiples du vecteur $\vec{AB} = \pvec{-2}{-6}$.
		Ainsi, $A + \vec{AB} = B$ donne un point de $(AB)$, mais il nous en faut un nouveau.
		$A+2\vec{AB} = (-1;-6) + \pvec{-4}{-12} = (-5 ; -18)$ convient.
		D'autres points (une infinité !) fonctionnent, comme $A - \vec{AB} ; A+12\vec{AB} ; A - \frac{\pi}{\sqrt2} \vec{AB} ; \dots$. 
	\end{enumerate}
}



\exe{, difficulty=1}{
	Considérons quatre points $A(-4; 1), B(-3 ; -3), C(4; -4), D(3; 1)$.
	
	Le quadralitère $ABCD$ est-il un parallélogramme ?
	Si non, donner un point $\tilde{D}$ tel que $ABC\tilde{D}$ en soit un.
}{exe:quad}{
	Il s'agit de vérifier si $\vec{AB}$ et $\vec{CD}$ sont colinéaires et si $\vec{BC}$ et $\vec{AD}$ le sont aussi.
	Calculons
		\[ \vec{AB} = B-A = \pvec{1}{-4}, \qquad \qquad \vec{CD} = D - C = \pvec{-1}{5}. \]
	Ces deux vecteurs ne sont pas colinéaires car s'ils l'étaient, on aurait un $k \in \R$ tel que
		\[ \vec{AB} = k \vec{CD}, \]
	ce qui donnerait $k = -1$ en étudiant la première coordonnée, et $k=-\dfrac45$ en regardant la deuxième.
	Ça n'est bien sûr pas possible.
	Un calcul de déterminant permet également de montrer que les vecteurs ne sont pas colinéaires, celui-ci étant non nul.
	
	En faisant un dessin, on remarque que le point $\tilde{D}$ doit nécessairement vérifier
		\[ \tilde{D} + \vec{AB} = C \qquad \iff \qquad \tilde{D} = C - \vec{AB} = (4 - 1 ; -4 - (-4) ) = (3 ;0). \]
	Par construction, $\vec{AB} = \vec{\tilde{D}C}$, qui montre que deux côtés opposés sont parallèles.
	Le parallélisme des deux autres côtés se fait de la même manière.
}

% pas ouf
%\exe{}{
%	Soient $A, B, C$ trois points tels que $B$ soit le milieu du segment $[AC]$.
%	Faire un dessin puis montrer que $\vec{AB} = \vec{BC}$.
%}{exe:milieu}{
%	La formule du milieu du cours
%		\[ B = \dfrac{A+C}2 \]
%	est équivalente à
%		\[ 2B = A + C \qquad \iff \qquad B-A = C-B \qquad \iff \qquad \vec{AB} = \vec{BC}. \]
%}

\exe{, difficulty=1}{
	Considérons les vecteurs $u = \pvec{2t}{5+t}$ et $v = \pvec{t-1}{3+t}$ dépendant d'un paramètre $t\in\R$.
	\begin{enumerate}
		\item
		Montrer que $\det(u, v) = 4 + (t+1)^2$.
		\item
		Justifier que $\det(u, v) \geq 4$ pour tout $t\in\R$.
		\item
		Conclure que $u$ et $v$ ne sont jamais colinéaires, peu importe la valeur de $t\in\R$.
	\end{enumerate}
}{exe:det-canonique}{
	\begin{enumerate}
		\item
		D'après la formule du déterminant,
			\[ \det(u, v) = (2t)(3+t)-(5+t)(t-1) = 6t + 2t^2 - 5t + 5 -t^2 + t = t^2 + 2t + 5. \]
		Développons désormais la forme factorisée proposée pour montrer l'égalité recherchée :
			\begin{align*}
				4 + (t-1)^2 &= 4 + (t^2 + 1 + 2t), \\
							&= t^2 + 2t + 5, \\
							&= \det(u,v).
			\end{align*}
		\item
		Le carré étant toujours positif, il suit que $(t+1)^2 \geq 0$ pour tout $t\in\R$.
		En ajoutant 4, $\det(u, v) \geq 4$ pour tout $t\in\R$ comme requis.
		\item
		D'après le cours, $u$ et $v$ sont colinéaires si et seulement si $\det(u, v) = 0$.
		Or cette quantité est toujours supérieure à 4 et ne peut donc pas être nulle.
		$u$ et $v$ ne peuvent donc pas être colinéaires.
	\end{enumerate}
}

\exe{, difficulty=1}{
	Pour chacune des propositions suivantes, montrer qu'elle est toujours vraie ou trouver un contre-exemple.
	\begin{enumerate}
		%\item $\vec{AB} = \vec{CD} \iff \vec{AC} = \vec{BD}$.
		\item Soient $u, v$ deux vecteurs tels que $v = 5u$.
		Alors $\det(u, v) = 0$.
		\item Si $\vec{AB}$ et $\vec{CD}$ sont colinéaires, alors les points $A, B, C$, et $D$ sont alignés.
		\item Pour $u, v$ deux vecteurs, on a $\det(u, v) = -\det(v, u)$.
		\item Si $\norm{v} = \sqrt{5}$, alors $v = \pvec{2}{1}$.
	\end{enumerate}
}{exe:vf}{

	\begin{enumerate}
		%\item $\vec{AB} = \vec{CD} \iff \vec{AC} = \vec{BD}$.
		\item 
		C'est vrai : d'après le cours, si $u$ et $v$ sont colinéaires, alors $\det(u, v) = 0$.
		On peut aussi le démontrer par le calcul.
		
		\item 
		C'est faux en général : on sait que $(AB)$ et $(CD)$ sont parallèles, mais pas si les droites sont confondues.
		On pourra donc construire un contre-exemple comme $A(0;0), B(1;0)$ et $C(1;0), D(1;1)$.
		
		\item 
		C'est vrai par le calcul. Soit $u = \pvec{a}{b}$ et $v=\pvec{c}{d}$.
		Alors $\det(u, v) = ad - bc$ et $\det(v, u) = cb - ad$.
		
		\item 
		C'est faux car la norme d'un vecteur ne peut pas donner ses coordonnées (sauf si la norme est nulle).
		
		Par exemple, comme $\norm{v} = \norm{-v}$, on a que $v = \pvec{-2}{-1}$ est un contre-exemple.
		
		On pourrait aussi échanger les coordonnées pour avoir $v = \pvec12$ qui donne un autre contre-exemple.
	\end{enumerate}
}

% si y'a le temps ?
%\exe{, difficulty=1}{
%	On étudie le \emph{système homogène} suivant qu'on ne cherche pas à résoudre.
%		\[ \begin{cases*} 12x - 5y = 0, \\ -7x + 2y = 0.\end{cases*} \]
%	\begin{enumerate}
%		\item 
%		Montrer que le système est équivalent à l'égalité de vecteurs
%			\[ x \cdot u + y \cdot v = \pvec{0}{0}, \]
%		pour les deux vecteurs $u = \pvec{12}{-7}$ et $v = \pvec{-5}{2}$.
%	
%		\item
%		Montrer qu'une solution $(x ;y)$ non nulle existe si et seulement si $u$ et $v$ sont colinéaires, et donc si et seulement si $\det(u, v) = 0$.
%		
%		\item	
%		Déduire dans ce cas que l'unique solution du système est la solution nulle $(x ;y) = (0;0)$.
%	\end{enumerate}
%}{exe:systemes10}{
%	TODO
%}
%
%\exe{, difficulty=1}{
%	\begin{enumerate}
%		\item
%		Si $u$ et $v$ sont colinéaires, et $w$ un troisième vecteur quelconque, montrer que le système $x \cdot u + y \cdot v = w$ admet une solution si et seulement si $w$ est colinéaire à $u$ et $v$.
%
%		\item
%		À partir des vecteurs colinéaires $u = \pvec23$ et $v = \pvec{-6}{-9}$ et d'un vecteur $w$ à poser, construire un système linéaire $x \cdot u + y \cdot v = w$ n'ayant aucune solution.
%		\item
%		À partir des vecteurs colinéaires $u = \pvec1{-4}$ et $v = \pvec{-2}{8}$ et d'un vecteur $w$ à poser, construire un système linéaire $x \cdot u + y \cdot v = w$ ayant un nombre infini de solutions.
%	\end{enumerate}
%}{exe:systemes11}{
%	TODO
%}


\subsection*{Approfondissement}

\exe{, difficulty=3}{
	Cet exercice généralise l'exercice \ref{exe:Pyth} et vise à donner une condition simple d'orthogonalité (question \ref{eq:1}), ainsi qu'à faire le lien entre le déterminant et l'aire d'un parallélogramme appelé \emph{parallélogramme fondamental} (questions \ref{eq:2}-\ref{eq:3}).

	Dénotons $O(0;0)$ l'origine du repère.
	On dit de deux vecteurs $u$ et $v$ qu'ils sont \emph{orthogonaux} dès que les droites $(OU)$ et $(OV)$, où $U = O + u$ et $V = O + v$, sont perpendiculaires.
	
	Soient $u = \pvec{a}{b}, v = \pvec{x}{y}$ deux vecteurs orthogonaux. 
	\begin{enumerate}
		\item
		Tracer un repère générique dans lequel doivent figurer : l'origine $O$ ; les points $U$ et $V$ ; le point $W$ tel que $OVWU$ est un parallélogramme.
		\item\label{eq:1}
		Montrer à l'aide de la réciproque de Pythagore dans le triangle $OVU$ que
			\[
				ax+by=0.
			\]
		\item\label{eq:2}
		Montrer à l'aide de la relation ci-dessus que
			\[
				\norm{u}^2 \cdot \norm{v}^2 = \left( \det(u, v) \right)^2.
			\]
		\item\label{eq:3}
		En conclure que l'aire du rectangle $OVWU$ est donnée par $\abs{\det(u, v)}$.
	\end{enumerate}
}{exe:det-is-signedvolume}{
	C'est très dur. Je suis désolé. Bon courage.
	
	\begin{center}
	\begin{tikzpicture}[>=stealth, scale=1]
		\begin{axis}[xmin = -1.9, xmax=7.9, ymin=-0.9, ymax=5.9, axis x line=middle, axis y line=middle, axis line style=<->, xlabel={}, ylabel={}, grid=none, grid style = {opacity=.5}, ticks = none]
		
			
			\draw[GREY, dashed, thick] (axis cs:4,1) -- (axis cs:3,5) -- (axis cs:-1,4);
			\draw[GREY, pattern=north west lines] (axis cs:0,0) -- (axis cs:4,1) -- (axis cs:3,5) -- (axis cs:-1,4) -- (axis cs:0,0);
			
			\addplot[GREY, mark=*, mark size = 1] (0,0) node[below=8pt, left] {$O$};
			\addplot[PINK, mark=*, mark size = 1] (-1,4) node[left] {$U(a;b)$};
			\addplot[ORANGE, mark=*, mark size = 1] (4,1) node[right] {$V(x;y)$};
			\addplot[BLUE_E, mark=*, mark size = 1] (3,5) node[above right] {$W$};
			
			\draw[PINK, ->, thick] (axis cs:0,0) -- (axis cs:-1,4) node[midway, left]{$u$};
			\draw[ORANGE, ->, thick] (axis cs:0,0) -- (axis cs:4,1) node[midway, below]{$v$};
			
			\draw[<-] (axis cs:3,3) arc[start angle=-90, end angle=-30, radius=30pt] node[right] {$\abs{\det(u,v)}$};
		\end{axis}
	\end{tikzpicture}
	\end{center}
	
	Ce résultat se généralise encore davantage : la valeur absolue du déterminant est égale à l'aire du parallélogramme fondamental de sommets $0, u, v, u+v$ même si $u$ et $v$ ne sont pas orthogonaux.
	Ceci permet d'ailleurs de justifier géométriquement que si $\det(u,v) = 0$, alors $u$ et $v$ sont colinéaires : l'aire nulle n'advient que lorsque $u$ et $v$ ont la même direction (le parallélogramme est plat).
}


% à mettre dans le chapitre affine
%\exe{}{
%	Considérons $A(x_A ; y_A)$ et $B(x_B;y_B)$ deux points différents de $O(0;0)$, l'origine du plan.
%	Le théorème de Pythagore énonce que le triangle $OAB$ est rectangle en $O$ si et seulement si
%		\[ OA^2 + OB^2 = AB^2. \]
%	Démontrer que cette condition est équivalente à
%		\[ x_A \cdot x_B + y_A \cdot y_B = 0. \]
%}{exe:star1}{
%	todo
%}
%
%\exe{}{
%	\begin{enumerate}		
%		\item
%		Esquisser les courbes des fonctions $F(x) = 2x -3$ et $G(x) = 1 -\dfrac12 x$ sur $[0 ; 4]$.
%		Que dire de celles-ci ?
%		\item 
%		Soit $f(x) = ax+b$ une fonction affine sur $\R$.
%		Montrer que $\pvec{1}a$ dirige $
%%		\item
%%		Montrer que $\C_f$ et $\C_g$ sont perpendiculaires si et seulement si $1 + a a' =0$ 
%		\item
%		Conclure que, pour deux fonctions affines $F$ et $G$ quelconques (ne passant pas forcément par l'origine), les droites $\C_F$ et $\C_G$ sont perpendiculaires si et seulement si le produit des coefficients directeurs vaut $-1$ à l'aide de l'exercice \ref{exe:star1}.
%	\end{enumerate}
%}{exe:star2}{
%	todo
%}

%%%%%%%%%%%

\newpage
\fancyhead[C]{\textbf{Solutions}}
\shipoutAnswer

\end{document}

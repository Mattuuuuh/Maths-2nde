				% ENABLE or DISABLE font change
				% use XeLaTeX if true
\newif\ifdys
				\dystrue
				\dysfalse

\newif\ifsolutions
				\solutionstrue
				\solutionsfalse

% DYSLEXIA SWITCH
\newif\ifdys
		
				% ENABLE or DISABLE font change
				% use XeLaTeX if true
				\dystrue
				\dysfalse


\ifdys

\documentclass[a4paper, 14pt]{extarticle}
\usepackage{amsmath,amsfonts,amsthm,amssymb,mathtools}

\tracinglostchars=3 % Report an error if a font does not have a symbol.
\usepackage{fontspec}
\usepackage{unicode-math}
\defaultfontfeatures{ Ligatures=TeX,
                      Scale=MatchUppercase }

\setmainfont{OpenDyslexic}[Scale=1.0]
\setmathfont{Fira Math} % Or maybe try KPMath-Sans?
\setmathfont{OpenDyslexic Italic}[range=it/{Latin,latin}]
\setmathfont{OpenDyslexic}[range=up/{Latin,latin,num}]

\else

\documentclass[a4paper, 12pt]{extarticle}

\usepackage[utf8x]{inputenc}
%fonts
\usepackage{amsmath,amsfonts,amsthm,amssymb,mathtools}
% comment below to default to computer modern
\usepackage{libertinus,libertinust1math}

\fi


\usepackage[french]{babel}
\usepackage[
a4paper,
margin=2cm,
nomarginpar,% We don't want any margin paragraphs
]{geometry}
\usepackage{icomma}

\usepackage{fancyhdr}
\usepackage{array}
\usepackage{hyperref}

\usepackage{multicol, enumerate}
\newcolumntype{P}[1]{>{\centering\arraybackslash}p{#1}}


\usepackage{stackengine}
\newcommand\xrowht[2][0]{\addstackgap[.5\dimexpr#2\relax]{\vphantom{#1}}}

% theorems

\theoremstyle{plain}
\newtheorem{theorem}{Th\'eor\`eme}
\newtheorem*{sol}{Solution}
\theoremstyle{definition}
\newtheorem{ex}{Exercice}
\newtheorem*{rpl}{Rappel}
\newtheorem{enigme}{Énigme}

% corps
\usepackage{calrsfs}
\newcommand{\C}{\mathcal{C}}
\newcommand{\R}{\mathbb{R}}
\newcommand{\Rnn}{\mathbb{R}^{2n}}
\newcommand{\Z}{\mathbb{Z}}
\newcommand{\N}{\mathbb{N}}
\newcommand{\Q}{\mathbb{Q}}

% variance
\newcommand{\Var}[1]{\text{Var}(#1)}

% domain
\newcommand{\D}{\mathcal{D}}


% date
\usepackage{advdate}
\AdvanceDate[0]


% plots
\usepackage{pgfplots}

% table line break
\usepackage{makecell}
%tablestuff
\def\arraystretch{2}
\setlength\tabcolsep{15pt}

%subfigures
\usepackage{subcaption}

\definecolor{myg}{RGB}{56, 140, 70}
\definecolor{myb}{RGB}{45, 111, 177}
\definecolor{myr}{RGB}{199, 68, 64}

% fake sections with no title to move around the merged pdf
\newcommand{\fakesection}[1]{%
  \par\refstepcounter{section}% Increase section counter
  \sectionmark{#1}% Add section mark (header)
  \addcontentsline{toc}{section}{\protect\numberline{\thesection}#1}% Add section to ToC
  % Add more content here, if needed.
}


% SOLUTION SWITCH
\newif\ifsolutions
				\solutionstrue
				%\solutionsfalse

\ifsolutions
	\newcommand{\exe}[2]{
		\begin{ex} #1  \end{ex}
		\begin{sol} #2 \end{sol}
	}
\else
	\newcommand{\exe}[2]{
		\begin{ex} #1  \end{ex}
	}
	
\fi


% tableaux var, signe
\usepackage{tkz-tab}


%pinfty minfty
\newcommand{\pinfty}{{+}\infty}
\newcommand{\minfty}{{-}\infty}

\begin{document}


\AdvanceDate[0]

\begin{document}
\pagestyle{fancy}
\fancyhead[L]{Seconde 13}
\fancyhead[C]{\textbf{Arithmétique : nombres rationnels \ifsolutions -- Solutions  \fi}}
\fancyhead[R]{\today}

\exe{[Vrai ou faux]
	Justifier la réponse.
	
	\def\arraystretch{2}
	\setlength\tabcolsep{15pt}
	\begin{tabular}{c c c}
		\hspace{10cm} & Vrai & Faux \\
		$\dfrac13 = 0,3$ & $\square$ & $\square$  \\
		$\dfrac13 = 0,33$ & $\square$ & $\square$  \\
		$\dfrac13 = 0,333$ & $\square$ & $\square$  \\
		$\dfrac13 = 0,3333$ & $\square$ & $\square$  \\
	\end{tabular}
}{}

\exe{[Problème]
	Supposons que le développement décimal de $\dfrac13$ soit fini.
	
	Que dire alors du nombre $10^n \times \dfrac13$ pour un $n\in\N$ suffisamment grand (mais fini) ?
	
	En déduire que $10^n$ est divisible par 3.
}{}

\exe{[Problème]
	Montrer que $10^n - 1$ est toujours divisible par 3.
}{}

\hrule

\exe{[Problème]
	Montrer que 
		\[ 0,3333... = \dfrac13, \]
	où les points de suspension ... indiquent que le développement décimal est infini (il ne s'arrête jamais).
}{}

\exe{\label{ex:3}
	Écrire chaque nombre, chacun ayant un développement décimal infini, sous forme de fraction de deux entiers.
	\begin{enumerate}
		\item $A = 0,666...$
		\item $B = 0,999...$
		\item $C = 0,12121212...$
		\item $D = 1,666...$
		\item $E = 0,34777...$
		\item $F = 0,123123123...$
		\item[($\star$)] $x = 0,123456789123456789...$
	\end{enumerate}
}{}

\exe{[Problème]
	Donner quelques termes en plus de la somme infinie suivante. 
		\[ \dfrac1{10} + \dfrac1{100} + \dfrac1{1 000} + \dfrac1{10 000} + \dots \]
	Écrire le nombre en écriture décimale, puis l'exprimer sous forme de fraction de deux entiers comme à l'exercice \ref{ex:3}.
}{}

\exe{[Problème]
	Donner quelques termes en plus de la somme infinie suivante.
		\[ \dfrac12 + \dfrac14 + \dfrac18 + \dfrac1{16} + \dots \]
	À quoi est-elle égale ? Un résultat entier est attendu.
}{}

\exe{[Problème]
	On considère la somme finie suivante
		\[ S = 1 + 2 + 4 + 8 + 16 + \dots + 2^{n-1} +  2^n, \]
	où $n\in\N$ est un entier naturel.
	
	Montrer que $2S = S - 1 + 2^{n+1}$, et en déduire la valeur de $S$ sous forme close (sans points de suspension).
}{}


\end{document}

% DYSLEXIA SWITCH
\newif\ifdys
		
				% ENABLE or DISABLE font change
				% use XeLaTeX if true
				\dystrue
				\dysfalse


\ifdys

\documentclass[a4paper, 14pt]{extarticle}
\usepackage{amsmath,amsfonts,amsthm,amssymb,mathtools}

\tracinglostchars=3 % Report an error if a font does not have a symbol.
\usepackage{fontspec}
\usepackage{unicode-math}
\defaultfontfeatures{ Ligatures=TeX,
                      Scale=MatchUppercase }

\setmainfont{OpenDyslexic}[Scale=1.0]
\setmathfont{Fira Math} % Or maybe try KPMath-Sans?
\setmathfont{OpenDyslexic Italic}[range=it/{Latin,latin}]
\setmathfont{OpenDyslexic}[range=up/{Latin,latin,num}]

\else

\documentclass[a4paper, 12pt]{extarticle}
\usepackage{amsmath,amsfonts,amsthm,amssymb,mathtools}

\fi


\usepackage[french]{babel}
\usepackage[
a4paper,
margin=2cm,
nomarginpar,% We don't want any margin paragraphs
]{geometry}
\usepackage{fancyhdr}
\usepackage{array}

\usepackage{multicol, enumerate}
\newcolumntype{P}[1]{>{\centering\arraybackslash}p{#1}}


\usepackage{stackengine}
\newcommand\xrowht[2][0]{\addstackgap[.5\dimexpr#2\relax]{\vphantom{#1}}}

% theorems

\theoremstyle{plain}
\newtheorem{theorem}{Th\'eor\`eme}
\newtheorem*{sol}{Solution}
\theoremstyle{definition}
\newtheorem{ex}{Exercice}

% corps
\newcommand{\C}{\mathbb{C}}
\newcommand{\R}{\mathbb{R}}
\newcommand{\Rnn}{\mathbb{R}^{2n}}
\newcommand{\Z}{\mathbb{Z}}
\newcommand{\N}{\mathbb{N}}
\newcommand{\Q}{\mathbb{Q}}

% domain
\newcommand{\D}{\mathbb{D}}


% date
\usepackage{advdate}
\AdvanceDate[-2]


% SOLUTION SWITCH
\newif\ifsolutions
				\solutionstrue
				%\solutionsfalse

\ifsolutions
	\newcommand{\exe}[2]{
		\begin{ex} #1  \end{ex}
		\begin{sol} #2 \end{sol}
	}
\else
	\newcommand{\exe}[2]{
		\begin{ex} #1  \end{ex}
	}
	
\fi

\begin{document}
\pagestyle{fancy}
\fancyhead[L]{Seconde 13}
\fancyhead[C]{\textbf{ \ifsolutions Solutions 1 \else Exercices 1 \fi}}
\fancyhead[R]{\today}

%\subsection*{Parité}


\subsection*{Multiples et diviseurs}

\exe{
	\begin{enumerate}
		\item Écrire  l'\emph{ensemble} des diviseurs de $24$.
		\item Écrire  l'\emph{ensemble} des diviseurs de  $33$.
	\end{enumerate}
}{
	\begin{align*}
		\mathcal{D}_{24}& = \{ 1, 2, 3, 4,  6, 8, 12, 24 \}, \\
		\mathcal{D}_{33} &= \{ 1, 3, 11, 33 \}.
	\end{align*}
}


\exe{
	\begin{enumerate}
		\item Donner  l'\emph{ensemble} des multiples de $17$ inférieurs ou égaux à $100$.
		\item Donner  l'\emph{ensemble} des multiples de  $34$ inférieurs ou égaux à $100$.
	\end{enumerate}
}{
	\begin{align*}
		\mathcal{M}_{17}& = \{ 0, 17, 34, 51, 68, 85 \}, \\
		\mathcal{M}_{34} &= \{  0, 34, 68  \}.
	\end{align*}
}

\hrule

\exe{
	Soient $a, b \in \N$ deux entiers naturels tels que
		\[ a | b. \]
	Montrer que $a$ divise également tous les multiples de $b$.
}{
	La relation $a | b$ signifie qu'il existe un entier naturel $k\in\N$ tel que 
		\[ b = a \times k. \]
	Or les multiples de $b$ sont $\{ 0, b, 2b, 3b, \dots \}$. Il s'écrivent donc 
		\[  n \times b, \]
	pour $n \in \N$ un entier naturel.
	
	On a donc 
		\[ n \times b  = n \times (a \times k) = a \times  (k \times n). \]
	Le multiple de $b$ est donc divisible par $a$ car $k \times n$ est un entier naturel.

	Si on prend $a = 2$, on retrouve un lemme du cours : si $b$ est pair, alors tous les multiples de $b$ sont pairs.
}

\exe{
	Soient $a, b, c \in \N$ trois entiers naturels vérifiant
		\[ a | b \qquad  \text{ et } \qquad b | c. \]
	Montrer que $a$ divise $c$.
}{
	La relation $b | c$ est équivalente à dire que $c$ est un multiple de $b$. 
	En effet, on a la relation
		\[ c = b \times n, \]
	pour un certain $n\in\N$ entier naturel.
	
	L'exercice ci-dessus conclut donc, car il montre que $a$ divise tout multiple de $b$, et donc $c$.
}

\hrule

\subsection*{Nombres premiers}

\exe{
	Énumérer tous les nombres premiers inférieurs ou égaux à $15$.
}{
	En notant $\mathcal{S}$ cet ensemble, on a
	\[ \mathcal{S} = \{ 2, 3, 5, 7, 11, 13 \}. \]
}

\exe{[Livre n°32-35 p.50]
	Décomposer les nombres suivants en produit de facteurs premiers.
	\begin{multicols}{3}
	\begin{enumerate}[i)]
		\item  $64 $
		\item $25$
		\item $81$
		\item  $6 \times 121 \times 64$
		\item $45 \times 125 \times 9$
		\item $10^2 \times 15^3$
	\end{enumerate}
	\end{multicols}
}{

	\begin{multicols}{3}
	\begin{enumerate}[i)]
		\item  $64 = 2^6 $
		\item $25 = 5^2$
		\item $81 = 9^2$
		\item  $6 \times 121 \times 64 = 2^7 \times 3 \times 11^2$
		\item $45 \times 125 \times 9 = 3^4 \times 5^4$
		\item $10^2 \times 15^3 = 2^2 \times 3^3 \times 5^5$
	\end{enumerate}
	\end{multicols}


}


\exe{[Livre n°40 p.50]
	On considère le nombre $n=2^6 \times 3^2 \times 7^7$.
	Quelles sont les affirmations exactes ? Justifier.
	\begin{multicols}{2}
	\begin{enumerate}[a)]
		\item $2^3 \times 3^3$ divise $n$
		\item $2^3 \times 7^3$ divise $n$
		\item $2^3 \times 7$ divise $n$
		\item $3^2 \times 7^7$ divise $n$
	\end{enumerate}
	\end{multicols}
}{

	\textbf{Remarque importante} : si $a$ divise $n$, on peut écrire 
		\[ n = a \times k, \]
	pour $k \in \N$ un entier naturel.
	
	En décomposant $a$ et $k$ en produit de nombres premiers, on trouve la décomposition de $n$ (voir exercice 6iv) par exemple).
	Ainsi, si un nombre premier apparaît dans la décomposition de $a$, il doit apparaître au moins autant de fois dans la décomposition de $b$.
	

	\begin{enumerate}[a)]
		\item $2^3 \times 3^3$ ne divise pas $n$ car $3$ n'apparaît que $2$ fois dans la décomposition de $n$.
		\item $2^3 \times 7^3$ divise $n$ car $n = (2^3 \times 7^3) \times (2^3 \times 3^2 \times 7^4)$.
		\item $2^3 \times 7$ divise $n$ car $n = (2^3 \times 7) \times ( 2^3 \times 3^2 \times 7^6)$.
		\item $3^2 \times 7^7$ divise $n$ car $n=(3^2 \times 7^7) \times 2^6 $.
	\end{enumerate}
}

\hrule

\exe{
	Soit $n \in \N$ un entier naturel.
	\begin{enumerate}
		\item Écrire la décomposition en produit de facteurs premiers de $10^n$.
		\item Démontrer par l'absurde que le rationnel $\dfrac{5}{12} \in \Q$ n'est pas un nombre décimal.
	\end{enumerate}
}
{
	\begin{enumerate}
		\item $10^n = 2^n \times 5^n$.
		\item On reprend la preuve par l'absurde vue en cours jusqu'à trouver deux entiers $a\in\Z$, $n\in\N$ tels que
			\[ \dfrac{a}{10^n} = \dfrac{5}{12}. \]
		En manipulant l'égalité on trouve
			\[ 3 \times 4 a = 5 \times 10^n = 2^n 5^{n+1}. \]
		Ainsi le premier $3$ apparaît à gauche mais pas à droite, ce qui contredit l'unicité du théorème de décomposition vu en cours (théorème fondamental de l'arithmétique).
	\end{enumerate}
}

\hrule

\subsection*{Coprimalité}


\exe{
	\begin{enumerate}
		\item Donner les diviseurs \emph{communs} à $6$ et à $15$.
		\item Simplifier la fraction $\dfrac{6}{15}$ en \emph{fraction irréductible}.
	\end{enumerate}
}{

	\begin{enumerate}
		\item $\mathcal{D}_6 \cap \mathcal{D}_{15} = \{ 1, 3 \}$.
		\item $\dfrac{6}{15} = \dfrac{3 \times 2}{3 \times 5} = \dfrac25$, irréductible car $2$ et $5$ sont premiers entre eux. 
	\end{enumerate}
	
}


\exe{
	\begin{enumerate}
		\item Donner les multiples \emph{communs}  à $10$ et $35$ inférieurs ou égaux à $100$.
		\item Écrire la somme $\dfrac{3}{10} + \dfrac{6}{35}$ en \emph{fraction irréductible}.
	\end{enumerate}
}{
	\begin{enumerate}
		\item $\mathcal{M}_{10} \cap \mathcal{M}_{35} = \{ 0, 70 \}$.
		\item  $\dfrac{3}{10} + \dfrac{6}{35} = \dfrac{21}{70} + \dfrac{12}{70} = \dfrac{33}{70}$, irréductible car $33$ et $70$ sont premiers entre eux.
	\end{enumerate}
	
	\textbf{Remarque importante} : un diviseur de $33 = 3 \times 11$ admet uniquement $3$ et $11$ dans sa décomposition en produit de nombres premiers, d'après l'exercice $7$.
	Idem pour $70 = 2 \times 5 \times 7$, un diviseur admet uniquement $2, 5$, et $7$ dans sa décomposition.
	
	Donc le seul diviseur commun à $33$ et $70$ est $1$.
}

\hrule

\exe{
	Soit $n \in \N$ un entier naturel \emph{non nul}.
	
	Montrer que la fraction $\dfrac{n}{n+1}$ est irréductible.
}{

	Le but est de montrer que le seul diviseur commun à $n$ et $n+1$ est $1$.
	Appelons $d \in \N$ un diviseur commun à $n$ et $n+1$.
	On a alors
		\begin{align*}
			n = d \times a = da,    &&  n+1 = d \times b = db,
		\end{align*}
	pour deux entiers naturels $a, b \in \N$.
	
	Ainsi
		\[ 1 = (n+1) - n = d b - d a = d(b-a). \]
	Donc $d$ est un diviseur de $1$ car $b-a$ est un entier : il est forcément égal à $1$.
	
	Finalement, le seul diviseur commun à $n$ et $n+1$ doit être $1$.
}

\hrule


%\exe{
%	Soient $a, b, c \in \N$ trois entiers naturels tels que
%		\[ a | b \qquad  \text{ et } \qquad a | c. \]
%	\begin{enumerate}
%		\item Montrer que $a$ divise la somme $b + c$.
%		\item Est-ce que $a+b$ divise toujours $c$ ?
%	\end{enumerate}
%}



\end{document}
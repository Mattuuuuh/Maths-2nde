				% ENABLE or DISABLE font change
				% use XeLaTeX if true
\newif\ifdys
				\dystrue
				\dysfalse

\newif\ifsolutions
				\solutionstrue
				\solutionsfalse

% DYSLEXIA SWITCH
\newif\ifdys
		
				% ENABLE or DISABLE font change
				% use XeLaTeX if true
				\dystrue
				\dysfalse


\ifdys

\documentclass[a4paper, 14pt]{extarticle}
\usepackage{amsmath,amsfonts,amsthm,amssymb,mathtools}

\tracinglostchars=3 % Report an error if a font does not have a symbol.
\usepackage{fontspec}
\usepackage{unicode-math}
\defaultfontfeatures{ Ligatures=TeX,
                      Scale=MatchUppercase }

\setmainfont{OpenDyslexic}[Scale=1.0]
\setmathfont{Fira Math} % Or maybe try KPMath-Sans?
\setmathfont{OpenDyslexic Italic}[range=it/{Latin,latin}]
\setmathfont{OpenDyslexic}[range=up/{Latin,latin,num}]

\else

\documentclass[a4paper, 12pt]{extarticle}

\usepackage[utf8x]{inputenc}
%fonts
\usepackage{amsmath,amsfonts,amsthm,amssymb,mathtools}
% comment below to default to computer modern
\usepackage{libertinus,libertinust1math}

\fi


\usepackage[french]{babel}
\usepackage[
a4paper,
margin=2cm,
nomarginpar,% We don't want any margin paragraphs
]{geometry}
\usepackage{icomma}

\usepackage{fancyhdr}
\usepackage{array}
\usepackage{hyperref}

\usepackage{multicol, enumerate}
\newcolumntype{P}[1]{>{\centering\arraybackslash}p{#1}}


\usepackage{stackengine}
\newcommand\xrowht[2][0]{\addstackgap[.5\dimexpr#2\relax]{\vphantom{#1}}}

% theorems

\theoremstyle{plain}
\newtheorem{theorem}{Th\'eor\`eme}
\newtheorem*{sol}{Solution}
\theoremstyle{definition}
\newtheorem{ex}{Exercice}
\newtheorem*{rpl}{Rappel}
\newtheorem{enigme}{Énigme}

% corps
\usepackage{calrsfs}
\newcommand{\C}{\mathcal{C}}
\newcommand{\R}{\mathbb{R}}
\newcommand{\Rnn}{\mathbb{R}^{2n}}
\newcommand{\Z}{\mathbb{Z}}
\newcommand{\N}{\mathbb{N}}
\newcommand{\Q}{\mathbb{Q}}

% variance
\newcommand{\Var}[1]{\text{Var}(#1)}

% domain
\newcommand{\D}{\mathcal{D}}


% date
\usepackage{advdate}
\AdvanceDate[0]


% plots
\usepackage{pgfplots}

% table line break
\usepackage{makecell}
%tablestuff
\def\arraystretch{2}
\setlength\tabcolsep{15pt}

%subfigures
\usepackage{subcaption}

\definecolor{myg}{RGB}{56, 140, 70}
\definecolor{myb}{RGB}{45, 111, 177}
\definecolor{myr}{RGB}{199, 68, 64}

% fake sections with no title to move around the merged pdf
\newcommand{\fakesection}[1]{%
  \par\refstepcounter{section}% Increase section counter
  \sectionmark{#1}% Add section mark (header)
  \addcontentsline{toc}{section}{\protect\numberline{\thesection}#1}% Add section to ToC
  % Add more content here, if needed.
}


% SOLUTION SWITCH
\newif\ifsolutions
				\solutionstrue
				%\solutionsfalse

\ifsolutions
	\newcommand{\exe}[2]{
		\begin{ex} #1  \end{ex}
		\begin{sol} #2 \end{sol}
	}
\else
	\newcommand{\exe}[2]{
		\begin{ex} #1  \end{ex}
	}
	
\fi


% tableaux var, signe
\usepackage{tkz-tab}


%pinfty minfty
\newcommand{\pinfty}{{+}\infty}
\newcommand{\minfty}{{-}\infty}

\begin{document}


\AdvanceDate[2]

\begin{document}
\pagestyle{fancy}
\fancyhead[L]{Seconde 13}
\fancyhead[C]{\textbf{Évaluation blanche -- Vecteurs \ifsolutions -- Solutions  \fi}}
\fancyhead[R]{\today}

\exe{[Calcul]
	Soient $A(3;-1), B(-1; -5),$ et $C(3 ; 4)$.
	\begin{multicols}{2}
	\begin{enumerate}
		\item Calculer $\vec{AB}, \vec{BA}$, et $\vec{CA}$.
		\item Calculer $\norm{\vec{AB}}, \norm{\vec{BA}}$, $\norm{\vec{AC}}$.
		\item Calculer $3 \vec{AB} + 3 \vec{CA}$ et $3 \vec{CB}$.
		\item Calculer $\norm{-4 \vec{AB}}$ et $\norm{-\dfrac1{13} \vec{CA}}$.
	\end{enumerate}
	\end{multicols}

}{

	\begin{enumerate}
		\item 
			\begin{align*}
				\vec{AB} = B - A = \pvec{-4}{-4} && \vec{BA} = - \vec{AB} = \pvec44 && \vec{CA} = A - C = \pvec0{-5}.
			\end{align*}
		\item
			\begin{align*}
				\norm{\vec{AB}} &= \sqrt{(-4)^2 + (-4)^2} = 4\sqrt{2} \\ \norm{\vec{BA}} &= \norm{- \vec{AB}} = \norm{\vec{AB}} = 4\sqrt{2} \\ \norm{\vec{CA}} &= \sqrt{(-5)^2} = |-5| = 5.
			\end{align*}
		\item
			On utilise la relation de Chasle $\vec{CB} = \vec{CA} + \vec{AB}$ pour obtenir immédiatement le résultat.
			\begin{align*}
				3 \vec{AB} + 3 \vec{CA} &= 3 \pvec{-4}{-4} + 3 \pvec0{-5} = \pvec{-12}{-27} \\
				3 \vec{CB} &= 3 \left( \vec{CA} + \vec{AB} \right) =  3 \vec{AB} + 3 \vec{CA} = \pvec{-12}{-27}
			\end{align*}
		\item
			\begin{gather*}
				\norm{-4 \vec{AB}} = |-4| \cdot \norm{\vec{AB}} = 4\norm{\vec{AB}} = 16\sqrt{2}  \\
				\norm{-\dfrac1{13} \vec{CA}} = \dfrac1{13} \norm{\vec{CA}} = \dfrac5{13}
			\end{gather*}
			
	\end{enumerate}

}

\exe{[Représentation graphique]
	Construire, dans le plan vierge de droite, les sommes 
		\[u+v+w \qquad \text{ et } \qquad \dfrac12u - 2v - w,\]
	où les vecteurs $u, v, w$ sont donnés dans le plan de gauche ci-dessous.
	
	%\begin{center}
		\begin{tikzpicture}[>=stealth, scale=1]
		\begin{axis}[xmin = -10, xmax=10, ymin=-10, ymax=10, axis x line=none, axis y line=none, axis line style=<->, xlabel={}, ylabel={}, ticks = none]
			\draw[very thick, ->, myg] (axis cs:-3,-4) -- (axis cs: 0,0) node[above] {$u$};
			\draw[very thick, ->, myr] (axis cs:-2,0) -- (axis cs: -5,6) node[above] {$v$};
			\draw[very thick, ->, myb] (axis cs:0,3) -- (axis cs: 8,-2) node[above] {$w$};
		\end{axis}
		\end{tikzpicture}
		\vline
		\ifsolutions		
		\begin{tikzpicture}[>=stealth, scale=1]
		\begin{axis}[xmin = -10, xmax=20, ymin=-15, ymax=10, axis x line=none, axis y line=none, axis line style=<->, xlabel={}, ylabel={}, ticks = none]
			\draw[very thick, ->, black, dotted] (axis cs:-3,-4) -- (axis cs: 0,0) node[pos=.5, above left] {$u$};
			\draw[very thick, ->, black, dotted] (axis cs:-2+2,0) -- (axis cs: -5+2,6) node[pos=.5, left] {$v$};
			\draw[very thick, ->, black, dotted] (axis cs:0-3,3+3) -- (axis cs: 8-3,-2+3) node[pos=.5, above] {$w$};
			
			\draw[very thick, ->, red] (axis cs:-3,-4) -- (axis cs: 8-3,-2+3) node[pos=.5, below right] {$u+v+w$};
			
			
			
			\draw[very thick, ->, black, dotted] (axis cs:-3+15,-4) -- (axis cs:-1.5+15, -2) node[pos=.5, above left] {$\frac12u$};
			\draw[very thick, ->, black, dotted] (axis cs:-2+2-1.5+15, 0-2) -- (axis cs: 4+2-1.5+15, -12-2) node[pos=.5, above right] {$-2v$};
			\draw[very thick, ->, black, dotted] (axis cs:0+4+2-1.5+15,3-3-12-2) -- (axis cs: -8+4+2-1.5+15, 8-3-12-2) node[pos=.5, below left] {$-w$};
			
			\draw[very thick, ->, green](axis cs:-3+15,-4) -- (axis cs: -8+4+2-1.5+15, 8-3-12-2) node[pos=.8, left] {$\frac12u-2v-w$};
		\end{axis}
		\end{tikzpicture}
		\fi
	%\end{center}
}{
	On met les vecteurs bout à bout pour créer la somme.
	La multiplication par un scalaire ne change pas la direction mais multiplie la norme et peut changer le sens si celui-ci est négatif.
}


\exe{[Vrai ou faux]
	Pour chacune des propositions suivantes, montrer qu'elle est toujours vraie ou trouver un contre-exemple.
	\begin{enumerate}
		%\item $\vec{AB} = \vec{CD} \iff \vec{AC} = \vec{BD}$.
		\item Soient $u, v$ deux vecteurs tels que $v = 5u$.
		Alors $\det(u,v) = 0$.
		\item Si $\vec{AB}$ et $\vec{CD}$ sont colinéaires, alors les points $A, B, C$, et $D$ sont alignés.
		\item Pour $u, v$ deux vecteurs, on a $\det(u, v) = -\det(v, u)$.
		\item Si $\norm{v} = \sqrt{5}$, alors $v = \pvec{2}{1}$.
	\end{enumerate}
}{

	\begin{enumerate}
		%\item $\vec{AB} = \vec{CD} \iff \vec{AC} = \vec{BD}$.
		\item 
		C'est vrai : d'après le cours, si $u$ et $v$ sont colinéaires, alors $\det(u, v) = 0$.
		On peut aussi le démontrer par le calcul.
		
		\item 
		C'est faux en général : on sait que $(AB)$ et $(CD)$ sont parallèles, mais pas si les droites sont confondues.
		On pourra donc construire un contre-exemple comme $A(0;0), B(1;0)$ et $C(1;0), D(1;1)$.
		
		\item 
		C'est vrai par le calcul. Soit $u = \pvec{a}{b}$ et $v=\pvec{c}{d}$.
		Alors $\det(u, v) = ad - bc$ et $\det(v, u) = cb - ad$.
		
		\item 
		C'est faux car la norme d'un vecteur ne peut pas donner ses coordonnées (sauf si la norme est nulle).
		
		Par exemple, comme $\norm{v} = \norm{-v}$, on a que $v = \pvec{-2}{-1}$ est un contre-exemple.
		
		On pourrait aussi échanger les coordonnées pour avoir $v = \pvec12$ qui donne un autre contre-exemple.
	\end{enumerate}
}

\exe{[Parallélisme]

	Soit $u = \pvec{4}{2}$ et $v = \pvec{1}{3}$ deux vecteurs.
	Représenter dans un repère les points
		\begin{align*}
			A(-3;-1), && B = A+u, && C = A+u+v, && \text{ et } && D=A+v,
		\end{align*}
	et répondre aux questions suivantes.
	\begin{enumerate}
		\item Que dire du quadrilatère $ABCD$ visuellement ?
		\item Montrer que $(AB)$ et $(CD)$ sont parallèles puis que $(AD)$ et $(BC)$ sont parallèles.
		\item Démontrer de façon générale cette propriété pour $A, u, v$ un point et deux vecteurs quelconques.
	\end{enumerate}
}{
	\begin{enumerate}
		\item $ABCD$ semble être un parallélogramme car ses cotés opposés sont parallèles deux à deux.
		\item On calcule $\vec{AB} = B-A = u$ et $\vec{CD} = D-C = A+v - (A+u+v) = -u$ pour voir qu'ils sont coléinaires : les droites $(AB)$ et $(CD)$ sont donc parallèles.
		
		Idem avec $\vec{AD} = D - A = v$ et $\vec{BC} = C - B = A + u + v - (A + u) = v$, ce qui conclut similairement. 
		\item 
		La démonstration employée ci-dessus ne dépend aucunement des coordonnées de $A, u$, ou $v$.
		En mathématiques, il est souvent plus facile de calculer en gardant les lettres qu'en les remplaçant avec des nombres.
	\end{enumerate}
	
}

\ifsolutions
\else
\newpage
\fi

\exe{[Vecteur directeur]
	On considère un point $A(5 ; -12)$ et un vecteur $v = \pvec{2}{-1}$.
	Soit $(d)$ la droite passant par $A$ et dirigée par $v$.
	
	\begin{enumerate}
		\item
		Donner $4$ points distincts appartenant à $(d)$.
		\item
		Trouver la fonction affine $f$ telle que $(d) = \C_f$.
	\end{enumerate}
}{
	
	\begin{enumerate}
		\item
		D'après le cours, $A + k \cdot v$ appartient à la droite pour n'importe quel $k\in\R$.
		On prendra donc $k=0$ pour retrouver $A$, $k=1, 2, 3$ pour avoir $B(7;-13), C(9;-13)$, et $D(11;-14)$.
		
		\item
		On cherche $a$ et $b$ de l'expression algébrique de $f$ : $f(x) = ax+b$.
		
		$v$ est colinéaire au vecteur $\pvec{1}{-\frac12}$, ce qui nous donne immédiatement $a=-\dfrac12$ d'après le cours.
		On pourrait utiliser l'appartenance d'un point à $\C_f$ pour trouver $b$, mais on peut à la place créer un point d'abscisse nulle en utilisant le raisonnement de la question 1.
		
		En effet, $A + k \cdot v$ appartient toujours à la droite.
		Le $k\in\R$ donnant un point d'abscisse nulle vérifie donc $5 + 2k = 0 \iff k = -\dfrac52.$
		
		Il suit que $A - \dfrac52v = \left(0 ; -12 + \dfrac52 \right) = \left(0 ; \dfrac{-19}2 \right).$
		Par conséquent, $b = \dfrac{-19}2$, et 
			\[ f(x) = -\dfrac12 x - \dfrac{19}2. \]
	\end{enumerate}

}

\exe{[Quadrilatère]
	Considérons quatre points $A(-4; 1), B(-3 ; -3), C(4; -4), D(3; 1)$.
	
	Le quadralitère $ABCD$ est-il un parallélogramme ?
	Si non, donner un point $\tilde{D}$ tel que $ABC\tilde{D}$ en soit un.
}{
	Il s'agit de vérifier si $\vec{AB}$ et $\vec{CD}$ sont colinéaires et si $\vec{BC}$ et $\vec{AD}$ le sont aussi.
	On calcule
		\[ \vec{AB} = B-A = \pvec{1}{-4}, \qquad \qquad \vec{CD} = D - C = \pvec{-1}{5}. \]
	Ces deux vecteurs ne sont pas colinéaires car s'ils l'étaient, on aurait un $k \in \R$ tel que
		\[ \vec{AB} = k \vec{CD}, \]
	ce qui donnerait $k = -1$ en étudiant la première coordonnée, et $k=-\dfrac45$ en regardant la deuxième.
	Ça n'est bien sûr pas possible.
	
	En faisant un dessin, on remarque que le point $\tilde{D}$ doit nécessairement vérifier
		\[ \tilde{D} + \vec{AB} = C \qquad \iff \qquad \tilde{D} = C - \vec{AB} = (4 - 1 ; -4 - (-4) ) = (3 ;0). \]
	Par construction, $\vec{AB} = C- \tilde{D} = \vec{\tilde{D}C}$, qui montre que deux côtés opposés sont parallèles.
	Le parallélisme des deux autres côtés se fait de la même manière.
}

\exe{[Milieu]
	Soient $A, B, C$ trois points tels que $B$ soit le milieu du segment $[AC]$.
	Faire un dessin puis montrer que $\vec{AB} = \vec{BC}$.
}{
	La formule du milieu du cours
		\[ B = \dfrac{A+C}2 \]
	est équivalente à
		\[ 2B = A + C \qquad \iff \qquad B-A = C-B \qquad \iff \qquad \vec{AB} = \vec{BC}. \]
}

\exe{[Parallélisme]
	Soient les points $A(3;2), B(-3 ; 7), C(-2; -3), D(4;-6)$.
	
	Les droites $(AB)$ et $(CD)$ sont-elles parallèles ?
	
	Si non, donner un point $\tilde{B}$ tel que les droites $(A\tilde{B})$ et $(CD)$ soient parallèles.
}{
	Il s'agit de vérifier si $\vec{AB}$ et $\vec{CD}$ sont colinéaires.
	On calcule
		\[ \vec{AB} = B-A = \pvec{-6}{5}, \qquad \qquad \vec{CD} = D - C = \pvec{6}{-3}. \]
	Les vecteurs ne sont pas colinéaires : s'il existait un $k\in\R$ tel que $\vec{AB} = k \vec{CD}$, on aurait nécessairement $k=-1$ par la première coordonnée, et $k=-\dfrac53 \neq -1$ par le seconde.
	Les droites ne sont donc pas parallèles.
	
	Pour que $(A\tilde{B})$ et $(CD)$ soient parallèles, elles doivent partager un vecteur directeur.
	Par exemple, le vecteur $\vec{CD}$ convient.
	
	On peut alors choisir $\tilde{B}$ parmis les $A+k \cdot \vec{CD}$ où $k\in\R$ est un scalaire quelconque.
	En prenant $k=-\dfrac13$ pour montrer qu'on aime les fractions, on peut définir $\tilde{B} = (1 ; 3)$.
	
	On vérifiera que $\vec{A\tilde{B}} = \pvec{-2}{1}$, qui est bien colinéaire à $\vec{CD} = \pvec{6}{-3}$.
}


\end{document}
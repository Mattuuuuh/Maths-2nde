				% ENABLE or DISABLE font change
				% use XeLaTeX if true
\newif\ifdys
				\dystrue
				\dysfalse

\newif\ifsolutions
				\solutionstrue
				\solutionsfalse

% DYSLEXIA SWITCH
\newif\ifdys
		
				% ENABLE or DISABLE font change
				% use XeLaTeX if true
				\dystrue
				\dysfalse


\ifdys

\documentclass[a4paper, 14pt]{extarticle}
\usepackage{amsmath,amsfonts,amsthm,amssymb,mathtools}

\tracinglostchars=3 % Report an error if a font does not have a symbol.
\usepackage{fontspec}
\usepackage{unicode-math}
\defaultfontfeatures{ Ligatures=TeX,
                      Scale=MatchUppercase }

\setmainfont{OpenDyslexic}[Scale=1.0]
\setmathfont{Fira Math} % Or maybe try KPMath-Sans?
\setmathfont{OpenDyslexic Italic}[range=it/{Latin,latin}]
\setmathfont{OpenDyslexic}[range=up/{Latin,latin,num}]

\else

\documentclass[a4paper, 12pt]{extarticle}

\usepackage[utf8x]{inputenc}
%fonts
\usepackage{amsmath,amsfonts,amsthm,amssymb,mathtools}
% comment below to default to computer modern
\usepackage{libertinus,libertinust1math}

\fi


\usepackage[french]{babel}
\usepackage[
a4paper,
margin=2cm,
nomarginpar,% We don't want any margin paragraphs
]{geometry}
\usepackage{icomma}

\usepackage{fancyhdr}
\usepackage{array}
\usepackage{hyperref}

\usepackage{multicol, enumerate}
\newcolumntype{P}[1]{>{\centering\arraybackslash}p{#1}}


\usepackage{stackengine}
\newcommand\xrowht[2][0]{\addstackgap[.5\dimexpr#2\relax]{\vphantom{#1}}}

% theorems

\theoremstyle{plain}
\newtheorem{theorem}{Th\'eor\`eme}
\newtheorem*{sol}{Solution}
\theoremstyle{definition}
\newtheorem{ex}{Exercice}
\newtheorem*{rpl}{Rappel}
\newtheorem{enigme}{Énigme}

% corps
\usepackage{calrsfs}
\newcommand{\C}{\mathcal{C}}
\newcommand{\R}{\mathbb{R}}
\newcommand{\Rnn}{\mathbb{R}^{2n}}
\newcommand{\Z}{\mathbb{Z}}
\newcommand{\N}{\mathbb{N}}
\newcommand{\Q}{\mathbb{Q}}

% variance
\newcommand{\Var}[1]{\text{Var}(#1)}

% domain
\newcommand{\D}{\mathcal{D}}


% date
\usepackage{advdate}
\AdvanceDate[0]


% plots
\usepackage{pgfplots}

% table line break
\usepackage{makecell}
%tablestuff
\def\arraystretch{2}
\setlength\tabcolsep{15pt}

%subfigures
\usepackage{subcaption}

\definecolor{myg}{RGB}{56, 140, 70}
\definecolor{myb}{RGB}{45, 111, 177}
\definecolor{myr}{RGB}{199, 68, 64}

% fake sections with no title to move around the merged pdf
\newcommand{\fakesection}[1]{%
  \par\refstepcounter{section}% Increase section counter
  \sectionmark{#1}% Add section mark (header)
  \addcontentsline{toc}{section}{\protect\numberline{\thesection}#1}% Add section to ToC
  % Add more content here, if needed.
}


% SOLUTION SWITCH
\newif\ifsolutions
				\solutionstrue
				%\solutionsfalse

\ifsolutions
	\newcommand{\exe}[2]{
		\begin{ex} #1  \end{ex}
		\begin{sol} #2 \end{sol}
	}
\else
	\newcommand{\exe}[2]{
		\begin{ex} #1  \end{ex}
	}
	
\fi


% tableaux var, signe
\usepackage{tkz-tab}


%pinfty minfty
\newcommand{\pinfty}{{+}\infty}
\newcommand{\minfty}{{-}\infty}

\begin{document}


\AdvanceDate[0]

\begin{document}
\pagestyle{fancy}
\fancyhead[L]{Seconde 13}
\fancyhead[C]{\textbf{Vecteurs 2 \ifsolutions -- Solutions  \fi}}
\fancyhead[R]{\today}



\exe{
	Dessiner les points 
		\begin{align*}
			A = (-1 ; -8), && B = (-7 ; 9), && C = (4 ; 7), && D = (7 ; -1,5),
		\end{align*}
	tracer les droites $(AB)$ et $(CD)$ dans un repère, et répondre au questions.
	
	\begin{enumerate}
		\item Que dire des droites $(AB)$ et $(CD)$ ?
		\item Montrer que les droites sont parallèles en calculant le coefficient directeur des fonctions affines associées.
		\item Calculer les vecteurs $\vec{AB}, \vec{BA}, \vec{CD}$ et $\vec{DC}$ et compléter les phrases suivantes.
			\begin{center}
			\begin{multicols}{2}
				\og Si je multiplie $\vec{AB}$ par \ifsolutions {\color{myr} $-1$} \else $\dots$ \fi,j'obtiens $\vec{BA}$ \fg \\ \vspace{10pt}
				\og Si je multiplie $\vec{AB}$ par \ifsolutions {\color{myr} $\dfrac12$} \else $\dots$ \fi,j'obtiens $\vec{DC}$ \fg \\ \vspace{10pt}
				\og Si je multiplie $\vec{DC}$ par \ifsolutions {\color{myr} $2$} \else $\dots$ \fi,j'obtiens $\vec{AB}$ \fg \\ \vspace{10pt}
				\og Si je multiplie $\vec{BA}$ par \ifsolutions {\color{myr} $\dfrac12$} \else $\dots$ \fi,j'obtiens $\vec{CD}$ \fg
			\end{multicols}
			\end{center}
		
		\item Trouver les nombres $a, a'\in\R$ tels que $\vec{AB}$ soit colinéaire à $\pvec{1}{a}$ et $\vec{CD}$ soit colinéaire à $\pvec{1}{a'}$.
		Que dire des coefficients $a$ et $a'$ ? 
		
		\item Montrer que, pour des points $A, B, C, D$ quelconques, si $(AB)$ et $(CD)$ sont parallèles non verticales, alors $\vec{AB}$ et $\vec{CD}$ sont colinéaires.
		%On traitera à part le cas où les vecteurs sont verticaux (première composante nulle).
		% cas droite verticale à part ?
	\end{enumerate}

}{
	\begin{enumerate}
		\item Les droites semblent parallèles.
		
		\item On calcule d'une part
			\[ \dfrac{y_B - y_A}{x_B - x_A} = \dfrac{17}{-6} = \dfrac{-17}{6}, \]
		puis d'autre part
			\[ \dfrac{y_D - y_C}{x_D - x_C} = \dfrac{-8,5}{3} = \dfrac{-17}{6}. \]
		Les coefficients directeurs des fonctions affines sont égaux et les droites $(AB)$ et $(CD)$ sont donc bien parallèles d'après le théorème de parallélisme du cours.

		\item 
		Les vecteurs sont donnés par
			\begin{align*}
				\vec{AB} = B-A = \pvec{-6}{17} && \vec{BA} = A - B = \pvec{6}{-17} \\
				\vec{CD} = D - C = \pvec{3}{-8,5} && \vec{DC} = C - D = \pvec{-3}{8,5}
			\end{align*}
		On remarque donc qu'on a $\vec{AB} = - \vec{BA}, \vec{DC} = \dfrac12 \vec{AB}, \vec{AB} = 2 \vec{DC},$ et $\vec{CD} = \dfrac12 \vec{BA}$.
		
		\item Pour faire aparaître un vecteur colinéaire ayant $1$ en première coordonnée, il suffit de factoriser un vecteur par sa première coordonnée.
		On a donc
			\[ \vec{AB} = \pvec{-6}{17} = -6 \cdot \pvec{1}{\frac{-17}{6}}, \]
		et le vecteur $\vec{AB}$ est colinéaire à $\pvec{1}{\frac{-17}{6}}$.
		On en déduit $a = \dfrac{-17}{6}$, qui est exactement le coefficient directeur calculé à la question 2.
		
		Idem pour $\vec{CD}$, on trouve
			\[ \vec{CD} = \pvec{3}{-8,5} = 3 \cdot \pvec{1}{\frac{-8,5}{3}} = 3 \cdot \pvec{1}{\frac{-17}{6}}, \]
		et on trouve le même résultat pour $a'$.
		
		\item
		Si les droites sont parallèles non verticales, on peut calculer le coefficient directeur $a$ des fonctions affines associées.
		
		Pour $(AB)$, on a
			\[ \vec{AB} = B - A = \pvec{x_B - x_A}{y_B - y_A} = (x_B - x_A) \cdot \pvec{1}{a}, \]
		où $a = \dfrac{y_B - y_A}{x_B-x_A}$ est exactement le coefficient directeur en question.
		
		Le même raisonnement pour $(CD)$ implique que $\vec{AB}$ et $\vec{CD}$ sont tous les deux colinéaires à $\pvec{1}{a}$. Ils sont donc nécessairement colinéaires.
		
		On montre que la colinéarité est une condition suffisante au parallélisme des droites non verticales en utilisant que si $\pvec{1}{a}$ est colinéaire à $\pvec{1}{a'}$, alors $a=a'$.
	\end{enumerate}
}

	\ifsolutions
	\else
	\begin{center}
		\begin{tikzpicture}[>=stealth, scale=1]
		\begin{axis}[xmin = -10.5, xmax=10.5, ymin=-10.5, ymax=10.5, axis x line=middle, axis y line=middle, axis line style=<->, xlabel={}, ylabel={}, xtick = {-10, -8, ..., 8, 10}, ytick = {-10, -8, ..., 8, 10}, grid=both]
			
		\end{axis}
		\end{tikzpicture}
		\begin{tikzpicture}[>=stealth, scale=1]
		\begin{axis}[xmin = -10.5, xmax=10.5, ymin=-10.5, ymax=10.5, axis x line=middle, axis y line=middle, axis line style=<->, xlabel={}, ylabel={}, xtick = {-10, -8, ..., 8, 10}, ytick = {-10, -8, ..., 8, 10}, grid=both]
			
		\end{axis}
		\end{tikzpicture}
	\end{center}
	\fi

\exe{\label{ex:1}

	Dessiner les points 
		\begin{align*}
			A = (-6 ;6), && B = (-5 ; 4), && C = (0 ; -6), && D = (2; -10),
		\end{align*}
	dans un repère et répondre au questions.
	
	\begin{enumerate}
		\item Que dire des points visuellement ? 
		\item Trouver la fonction affine $f$ telle que $\C_f$ passe par $A$ et $B$, puis démontrer que $C$ et $D$ appartiennent également à $\C_f$.
		\item Calculer $\det\left(\vec{AB}, \vec{AD}\right)$ et en déduire que $D \in(AB)$.
		\item Calculer $\det\left(\vec{CB}, \vec{DB}\right)$ et en déduire que $C \in (DB)$.
				
	\end{enumerate}

}{

	\begin{enumerate}
		\item Les points semblent alignés.
		\item En se rappelant le chapitre Fonctions affines (si lointain), on calcule d'abord $a$ à l'aide du théorème du cours, puis $b$ à l'aide d'une appartenance d'un point.
		
			\[ a = \dfrac{y_B - y_A}{x_B - x_A} = \dfrac{4 - 6}{-5 - (-6)} = -2. \]
		
		En notant $f(x) = ax + b = -2x + b$, l'appartenance de $A$ à $\C_f$ donne
			\begin{align*}
				y_A &= f(x_A) \\
				6 &= f(-6) \\
				6 &= 12 + b \\
				b &= -6
			\end{align*}
		On trouve donc $f(x) = -2x - 6$ telle que $\C_f = (AB)$.
		
		Pour montrer que $C, D \in \C_f$, on vérifie que $f(x_C) = f(0) = -6 = y_C$, puis $f(x_D) = -10 = y_D$.
		
		\item 
		On calcule 
			\[ \det\left(\vec{AB}, \vec{AD}\right) = \det\left(\pvec{1}{-2}, \pvec{8}{-16}\right) = 1(-16) - (-2)(8) = 0. \]
		D'après le cours, ceci suffit à conclure que $A, B, D$ sont alignés et donc que $D \in (AB)$.
		
		
		\item 
		On calcule 
			\[ \det\left(\vec{CB}, \vec{DB}\right) = \det\left(\pvec{-5}{10}, \pvec{-7}{14}\right) = (-5)(14) - (10)(-7) = 0. \]
		D'après le cours, ceci suffit à conclure que $C, B, D$ sont alignés et donc que $C \in (BD) = (AB)$.
	\end{enumerate}
	
	La méthode vectorielle de démonstration que les points sont alignés est plus rapide que celle affine (et elle s'applique aussi aux points alignés verticalement).
	D'ailleurs, il est inutile de calculer explicitement le détérminant si les vecteurs sont clairement colinéaires car celui-ci sera forcément nul.
	C'est le cas des vecteur $\pvec{1}{-2}$ et $\pvec{8}{-16}$ de la question 3. par exemple.
}


\end{document}

%% INPUT PREAMBLE.TEX
%% THEN INPUT VARS_{i}.ADR
%% THEN RUN THIS
%% DYSLEXIA SWITCH
\newif\ifdys
		
				% ENABLE or DISABLE font change
				% use XeLaTeX if true
				\dystrue
				\dysfalse


\ifdys

\documentclass[a4paper, 14pt]{extarticle}
\usepackage{amsmath,amsfonts,amsthm,amssymb,mathtools}

\tracinglostchars=3 % Report an error if a font does not have a symbol.
\usepackage{fontspec}
\usepackage{unicode-math}
\defaultfontfeatures{ Ligatures=TeX,
                      Scale=MatchUppercase }

\setmainfont{OpenDyslexic}[Scale=1.0]
\setmathfont{Fira Math} % Or maybe try KPMath-Sans?
\setmathfont{OpenDyslexic Italic}[range=it/{Latin,latin}]
\setmathfont{OpenDyslexic}[range=up/{Latin,latin,num}]

\else

\documentclass[a4paper, 12pt]{extarticle}

\usepackage[utf8x]{inputenc}
%fonts
\usepackage{amsmath,amsfonts,amsthm,amssymb,mathtools}
% comment below to default to computer modern
\usepackage{libertinus,libertinust1math}

\fi


\usepackage[french]{babel}
\usepackage[
a4paper,
margin=2cm,
nomarginpar,% We don't want any margin paragraphs
]{geometry}
\usepackage{icomma}

\usepackage{fancyhdr}
\usepackage{array}
\usepackage{hyperref}

\usepackage{multicol, enumerate}
\newcolumntype{P}[1]{>{\centering\arraybackslash}p{#1}}


\usepackage{stackengine}
\newcommand\xrowht[2][0]{\addstackgap[.5\dimexpr#2\relax]{\vphantom{#1}}}

% theorems

\theoremstyle{plain}
\newtheorem{theorem}{Th\'eor\`eme}
\newtheorem*{sol}{Solution}
\theoremstyle{definition}
\newtheorem{ex}{Exercice}
\newtheorem*{rpl}{Rappel}
\newtheorem{enigme}{Énigme}

% corps
\usepackage{calrsfs}
\newcommand{\C}{\mathcal{C}}
\newcommand{\R}{\mathbb{R}}
\newcommand{\Rnn}{\mathbb{R}^{2n}}
\newcommand{\Z}{\mathbb{Z}}
\newcommand{\N}{\mathbb{N}}
\newcommand{\Q}{\mathbb{Q}}

% variance
\newcommand{\Var}[1]{\text{Var}(#1)}

% domain
\newcommand{\D}{\mathcal{D}}


% date
\usepackage{advdate}
\AdvanceDate[0]


% plots
\usepackage{pgfplots}

% table line break
\usepackage{makecell}
%tablestuff
\def\arraystretch{2}
\setlength\tabcolsep{15pt}

%subfigures
\usepackage{subcaption}

\definecolor{myg}{RGB}{56, 140, 70}
\definecolor{myb}{RGB}{45, 111, 177}
\definecolor{myr}{RGB}{199, 68, 64}

% fake sections with no title to move around the merged pdf
\newcommand{\fakesection}[1]{%
  \par\refstepcounter{section}% Increase section counter
  \sectionmark{#1}% Add section mark (header)
  \addcontentsline{toc}{section}{\protect\numberline{\thesection}#1}% Add section to ToC
  % Add more content here, if needed.
}


% SOLUTION SWITCH
\newif\ifsolutions
				\solutionstrue
				%\solutionsfalse

\ifsolutions
	\newcommand{\exe}[2]{
		\begin{ex} #1  \end{ex}
		\begin{sol} #2 \end{sol}
	}
\else
	\newcommand{\exe}[2]{
		\begin{ex} #1  \end{ex}
	}
	
\fi


% tableaux var, signe
\usepackage{tkz-tab}


%pinfty minfty
\newcommand{\pinfty}{{+}\infty}
\newcommand{\minfty}{{-}\infty}

\begin{document}
\input{adr/vars_12345.adr}

\pagestyle{fancy}
\fancyhead[L]{Seconde 13}
\fancyhead[C]{\textbf{Devoir Maison 2 -- \seed \ifsolutions \, -- Solutions  \fi}}
\fancyhead[R]{\today}

Dans un repère d'origine $O$, on considère les trois points suivants, dépendant d'un nombre réel $x \in [0;\xmax]$.
	\begin{align*}
		A  = x \cdot (\xA;\yA) && B = (\xmax-x)\cdot(\xB ; \yB) && P = \LAMBDA \cdot (\xmax-x)\cdot (\xA;\yA).
	\end{align*}

On admettra que les points $O, P$, et $A$ sont alignés (ils sont multiples d'un même point).
	
	
\exe{
	Donner les coordonnées des points $A, B,$ et $P$ lorsque $x=\xfirst$ et lorsque $x=\xsecond$.
	
	Tracer ces points dans deux repères qui contiennent l'origine $O$.
}{
	On trace le triangle $OBA$ et le projeté orthogonal $P$ du sommet $B$ sur $(OA)$ lorsque $x=\xfirst$ puis $x=\xsecond$.

	\begin{figure}[h!]
		\begin{center}
		\begin{tikzpicture}[>=stealth, scale=1]
			\begin{axis}[xmin = \xlow-5, xmax=\xhigh+5, ymin=\ylow-5, ymax=\yhigh+5, axis x line=middle, axis y line=middle, axis line style=-]
				\addplot[black, mark=*, mark size = 1, thick] (\xAfirst, \yAfirst) node[above] {$A$};
				\addplot[black, mark=*, mark size = 1, thick] (0,0) node[above right] {$O$};
				\addplot[black, mark=*, mark size = 1, thick] (\xBfirst, \yBfirst) node[above] {$B$};
				\addplot[black, mark=*, mark size = 1, thick] (\xPfirst, \yPfirst) node[above] {$P$};
				
				% triangle OBA
				\draw (axis cs:0,0) -- (axis cs:\xBfirst,\yBfirst);
				\draw (axis cs:\xAfirst, \yAfirst) -- (axis cs:\xBfirst,\yBfirst);
				\draw (axis cs:0,0) -- (axis cs:\xAfirst, \yAfirst);
				
				% height P projected onto (OA)
				\draw[dotted] (axis cs:0,0) -- (axis cs:\xPfirst, \yPfirst);
				\draw[dotted] (axis cs:\xBfirst, \yBfirst) -- (axis cs:\xPfirst, \yPfirst);
			\end{axis}
		\end{tikzpicture}
		\end{center}
		\caption{Lorsque $x=\xfirst$. $A = (\xAfirst; \yAfirst), B = (\xBfirst ; \yBfirst), P = (\xPfirst; \yPfirst)$.}
	\end{figure}
	\begin{figure}[h!]
	\begin{center}
	\begin{tikzpicture}[>=stealth, scale=1]
		\begin{axis}[xmin = \xlow-5, xmax=\xhigh+5, ymin=\ylow-5, ymax=\yhigh+5, axis x line=middle, axis y line=middle, axis line style=-]
			\addplot[black, mark=*, mark size = 1, thick] (\xAsecond, \yAsecond) node[above] {$A$};
			\addplot[black, mark=*, mark size = 1, thick] (0,0) node[above right] {$O$};
			\addplot[black, mark=*, mark size = 1, thick] (\xBsecond, \yBsecond) node[above] {$B$};
			\addplot[black, mark=*, mark size = 1, thick] (\xPsecond, \yPsecond) node[above] {$P$};
			
			% triangle OBA
			\draw (axis cs:0,0) -- (axis cs:\xBsecond,\yBsecond);
			\draw (axis cs:\xAsecond, \yAsecond) -- (axis cs:\xBsecond,\yBsecond);
			\draw (axis cs:0,0) -- (axis cs:\xAsecond, \yAsecond);
			
			% height P projected onto (OA)
			\draw[dotted] (axis cs:0,0) -- (axis cs:\xPsecond, \yPsecond);
			\draw[dotted] (axis cs:\xBsecond, \yBsecond) -- (axis cs:\xPsecond, \yPsecond);
		\end{axis}
	\end{tikzpicture}
	\end{center}
	\caption{Lorsque $x=\xsecond$. $A = (\xAsecond; \yAsecond), B = (\xBsecond ; \yBsecond), P = (\xPsecond; \yPsecond)$}
	\end{figure}
	
}

\exe{
	 À l'aide de la formule de la longueur de segment vue en cours, montrer que
	 	\begin{align*}
	 		OB^2 = \Bnormsq\cdot |\xmax-x|^2, && \text{ et } && OP^2 = \Pnormsq\cdot |\xmax-x|^2.
		\end{align*}
}{
	En général, pour connaître la longueur $AB^2$, le cours nous donne la formule
		\[ AB^2 = |x_A - x_B|^2 + |y_A - y_B|^2. \]
	Remarquons que les valeurs absolues peuvent disparaître lorsque mises au carré : $|x|^2 = x^2$ pour tout $x\in\R$ réel.
	De plus en général, on a $(ab)^2 = a^2 b^2$, ce qui permet de distribuer le carré.
	
	En particulier lorsqu'un des point est nul, on somme simplement le carré des coordonées du deuxième point.
	Ainsi,
		\begin{align*}
			OB^2 &= x_B^2 + y_B^2 \\
					&= \left[ (\xmax-x)\cdot(\xB) \right]^2 + \left[ (\xmax-x)\cdot(\yB) \right]^2 \\
					&= (\xmax-x)^2 \cdot \left[ (\xB)^2 + (\yB)^2 \right] \\
					&= \Bnormsq\cdot |\xmax-x|^2,
		\end{align*}
	comme souhaité.
	
	Idem pour le calcul de $OP$.
		\begin{align*}
			OP^2 &= x_P^2 + y_P^2 \\
					&= \left[ \LAMBDA \cdot (\xmax-x)\cdot(\xA) \right]^2 + \left[ \LAMBDA \cdot (\xmax-x)\cdot(\yA) \right]^2 \\
					&= \left( \LAMBDA \right)^2 \cdot (\xmax-x)^2 \cdot \left[(\xA)^2 + (\yA)^2\right] \\
					&= \Pnormsq\cdot |\xmax-x|^2
		\end{align*}

}
\exe{
	Similairement, montrer que
	 	\[ BP^2 = \BPnormsq \cdot |\xmax-x|^2.\] 
}{
	On réutilise la formule de la distance en remarquant que $|\xmax-x|^2$ se factorise à nouveau.
	
	\begin{align*}
	%B = (\xmax-x)\cdot(\xB ; \yB) && P = \LAMBDA \cdot (\xmax-x)\cdot (\xA;\yA).
		BP^2 &= (x_B - x_P)^2 + (y_B - y_P)^2 \\
				&= \left[  (\xmax-x)\cdot(\xB) - \LAMBDA \cdot (\xmax-x)\cdot (\xA) \right]^2 + 
				\left[  (\xmax-x)\cdot(\yB) - \LAMBDA \cdot (\xmax-x)\cdot (\yA) \right]^2 \\
				&= (\xmax-x)^2 \cdot \left(\xB - \LAMBDA \cdot (\xA)\right)^2 
				+ (\xmax-x)^2 \cdot \left(\yB - \LAMBDA \cdot (\yA)\right)^2 \\
				&= (\xmax-x)^2 \cdot \BPnormsq
	\end{align*}
}
\exe{
	 Démontrer que le triangle $OBP$ est rectangle en $P$ à l'aide de la réciproque du théorème de Pythagore.
}{
	Il s'agit ici de vérifier que l'égalité
		\[ OB^2 = OP^2 + BP^2 \]
	tient bien.
	Le membre de gauche a été calculé comme étant
		\[ OB^2 = \Bnormsq\cdot |\xmax-x|^2, \]
	et le membre de droite comme étant
		\begin{align*}
			OP^2 + BP^2 &=  \Pnormsq\cdot |\xmax-x|^2 + \BPnormsq \cdot |\xmax-x|^2 \\
							&= \left( \Pnormsq + \BPnormsq \right) \cdot |\xmax-x|^2 \\
							&= \Bnormsq\cdot |\xmax-x|^2
		\end{align*}
	L'égalité est donc bien vérifiée quelque soit $x \in [0; \xmax]$ et le triangle $OBP$ est rectangle en $P$.
}

\exe{
	 En déduire que $P$ est le projeté orthogonal de $B$ sur $(OA)$ et donc que $BP$ est la hauteur du triangle $OAB$ de base $OA$.
}{
	D'après l'énoncé, les points $O, A,$ et $P$ sont alignés.
	Il suit donc que les droites $(OA)$ et $(OP)$ sont confondues et donc que $P$ appartient à la droite $(OA)$.
	
	De plus, le triangle $OBP$ est rectangle en $P$, donc les droites $(BP)$ et $(OP) =(OA)$ sont perpendiculaires.
	Par définition, $P$ est donc le projeté orthogonal de $B$ sur $(OA)$ car c'est le point de $(OA)$ tel que $(OA)$ et $(BP)$ sont perpendiculaires.
	
	On conclut que le triangle $OAB$ admet une base $OA$ et une hauteur $BP$ car $P$ est le projeté orthogonal de $B$ sur $(OA)$.
}

\exe{
	 Démontrer que l'aire $\mathcal{A}(x)$ du triangle $OAB$ est donnée par, pour $x\in[0;\xmax]$,
		\[ \mathcal{A}(x) = \prodovertwo \cdot |\xmax-x| \cdot |x| =  \prodovertwo(\xmax-x)x. \]
}{
	L'aire du triangle est donné par la formule
		\begin{align*}
			\text{Aire} &= \dfrac{\text{Base $\cdot$ Hauteur}}2 \\
						&= \dfrac{OA \cdot BP}2.
		\end{align*}
	Nous avons déjà calculés $BP^2$ et donc $BP$ se déduit en prenant sa racine carrée (en n'oubliant pas la propriété de la racine $\sqrt{a\cdot b} = \sqrt{a} \cdot \sqrt{b}$).
	On pourrait d'ailleurs aussi calculer le carré de l'aire puis prendre une unique racine carrée à la fin.
	
	Comme la longueur $BP$ est positive, on trouve
	\begin{align*}
		BP = |BP| = \sqrt{BP^2} = \sqrt{\BPnormsq \cdot |\xmax-x|^2} = \sqrt{\BPnormsq} \cdot |\xmax-x|.
	\end{align*}
	
	D'autre part,
		\[ OA^2 = x_A^2 + y_A^2 = x^2 \left[ (\xA)^2 + (\yA)^2 \right] =  \Anormsq x^2. \]
	D'où 
		\[ OA = \sqrt{\Anormsq} \cdot |x|, \]
	et donc on conclut que	
	\begin{align*}
		\mathcal{A}(x) &=  \dfrac{OA \cdot BP}2 \\
						&= \dfrac12 \cdot OA \cdot BP \\
						&= \dfrac12 \sqrt{\Anormsq} \cdot |x|  \cdot \sqrt{\BPnormsq} \cdot |\xmax-x| \\
						&= \dfrac12 \sqrt{\Anormsq \cdot \BPnormsq} \cdot |x| \cdot |\xmax-x| \\
						&= \prodovertwo \cdot |\xmax-x| \cdot |x|
	\end{align*}
	Finalement, comme la variable $x$ appartient à l'intervalle $[0; \xmax]$, les expressions à l'intérieur des valeurs absolues sont toujours positives et les barres de valeur absolue peuvent disparaître.
}


\exe{
	 Esquisser la courbe $\C_\mathcal{A}$ de l'aire $\mathcal{A}$, fonction réelle sur $[0;\xmax]$.
}{
	\begin{figure}[h!]
	\begin{center}
	\begin{tikzpicture}[>=stealth, scale=1]
		\begin{axis}[xmin = 0, xmax=\xmax, ymin=0, ymax=\BETAval+10, axis x line=middle, axis y line=middle, axis line style=-]
			\addplot[myb, thick, domain =0:\xmax, samples=50] {\prodovertwoval * (\xmax-x)*x}  node[pos=.5, right=15pt] {$\mathcal{C}_\mathcal{A}$};
		\end{axis}
	\end{tikzpicture}
	\end{center}
	\caption{Courbe représentative de $\mathcal{A}$ sur le domaine $[0; \xmax]$.}
	\end{figure}


}

\exe{
	 À l'aide de $\C_\mathcal{A}$, estimer la valeur du $x^\star \in [0;\xmax]$ telle que $\mathcal{A}(x^\star)$ soit maximale.
}{
	On estime la valeur maximale de $\mathcal{A}$ au pic de la courbe représentative de $\mathcal{A}$, c'est-à-dire lorsque $x^\star \approx \ALPHA$.
}


\exe{
	 Montrer que, pour tout $x\in[0;\xmax]$, on a l'identité
		\[ \mathcal{A}(x) =  \BETA - \prodovertwo\left(x-\ALPHA\right)^2. \]
}{
	On part de l'expression de droite qu'on développe pour arriver à l'expression de $\mathcal{A}(x)$ trouvée à la question 6.
	À l'aide de l'identité remarquable $(a-b)^2 = a^2 + b^2 - 2ab$, on obtient le développement suivant.
	
	\begin{align*}
		&~\quad \BETA - \prodovertwo\left(x-\ALPHA\right)^2 \\
		&= \BETA - \prodovertwo \left[ x^2 - 2 \cdot x \cdot \ALPHA + \left( \ALPHA \right)^2 \right]  \\
		&= \BETA - \prodovertwo \cdot x^2 + \prodovertwo \cdot 2 \cdot \ALPHA \cdot x - \prodovertwo \cdot \left( \ALPHA \right)^2 \\
		&= - \prodovertwo \cdot x^2 + \prodovertwo \cdot \xmax \cdot x,
	\end{align*}
	où on utilise que les constantes s'annulent car $\BETA = \prodovertwo \cdot \left( \ALPHA \right)^2$.

	On factorise le résultat par $\prodovertwo$ puis $x$ pour bien obtenir
	\begin{align*}
		\BETA - \prodovertwo\left(x-\ALPHA\right)^2 &= - \prodovertwo \cdot x^2 + \prodovertwo \cdot \xmax \cdot x \\
														&= \prodovertwo \cdot x \cdot (\xmax - x) \\
														&= \mathcal{A}(x),
	\end{align*}
	comme souhaité.
}


\exe{
	 En déduire que, pour tout $x\in[0;\xmax]$,
		\[ \mathcal{A}(x) \leq  \BETA , \]
	et donc que $\mathcal{A}$ atteint son maximum en $x^\star=\ALPHA$.
}{
	Un carré est toujours positif, donc son opposé toujours négatif.
	L'expression de $\mathcal{A}(x)$ de la question $9$ permet donc de conclure car on a, pour tout $x$ du domaine,
		\begin{align*}
			- \prodovertwo\left(x-\ALPHA\right)^2 &\leq 0 \\
			\BETA - \prodovertwo\left(x-\ALPHA\right)^2 &\leq \BETA \\
			\mathcal{A}(x) \leq \BETA.
		\end{align*}
		
	En outre, l'égalité n'est vérifiée que lorsque le carré est nul, c'est-à-dire quand l'expression à l'intérieur du carré est nulle.
	On a donc
		\begin{align*}
			\mathcal{A}(x^\star) = \BETA && \iff && \left(x^\star-\ALPHA\right)^2 = 0 && \iff && x^\star - \ALPHA = 0 && \iff x^\star = \ALPHA.
		\end{align*}
}


\end{document}

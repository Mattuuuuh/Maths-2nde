				% ENABLE or DISABLE font change
				% use XeLaTeX if true
\newif\ifdys
				\dystrue
				\dysfalse

\newif\ifsolutions
				\solutionstrue
				\solutionsfalse

%!TEX encoding = UTF8
%!TEX root =notes.tex


%%%%%%%%%%%%%%%%%%%%%%%%%%%%%%%%%
% PACKAGE IMPORTS
%%%%%%%%%%%%%%%%%%%%%%%%%%%%%%%%%


\usepackage[french]{babel}

\usepackage[tmargin=2cm,rmargin=1in,lmargin=1in,margin=0.85in,bmargin=2cm,footskip=.2in]{geometry}
\usepackage{amsmath,amsfonts,amsthm,amssymb,mathtools}
\usepackage[varbb]{newpxmath}
\usepackage{xfrac}
\usepackage[makeroom]{cancel}
\usepackage{mathtools}
\usepackage{bookmark}
\usepackage{enumitem}
\usepackage{hyperref,theoremref}
\hypersetup{
	pdftitle={Assignment},
	colorlinks=true, linkcolor=doc!90,
	bookmarksnumbered=true,
	bookmarksopen=true
}
\usepackage[most,many,breakable]{tcolorbox}
\usepackage{xcolor}
\usepackage{varwidth}
\usepackage{varwidth}
\usepackage{etoolbox}
%\usepackage{authblk}
\usepackage{nameref}
\usepackage{multicol,array}
\usepackage{tikz-cd}
\usepackage[ruled,vlined,linesnumbered]{algorithm2e}
\usepackage{comment} % enables the use of multi-line comments (\ifx \fi) 
\usepackage{import}
\usepackage{xifthen}
\usepackage{pdfpages}
\usepackage{transparent}


\newcommand\mycommfont[1]{\footnotesize\ttfamily\textcolor{blue}{#1}}
\SetCommentSty{mycommfont}
\newcommand{\incfig}[1]{%
    \def\svgwidth{\columnwidth}
    \import{./figures/}{#1.pdf_tex}
}

\usepackage{tikzsymbols}
%\renewcommand\qedsymbol{$\Laughey$}


%\usepackage{import}
%\usepackage{xifthen}
%\usepackage{pdfpages}
%\usepackage{transparent}


%%%%%%%%%%%%%%%%%%%%%%%%%%%%%%
% SELF MADE COLORS
%%%%%%%%%%%%%%%%%%%%%%%%%%%%%%



\definecolor{myg}{RGB}{56, 140, 70}
\definecolor{myb}{RGB}{45, 111, 177}
\definecolor{myr}{RGB}{199, 68, 64}
\definecolor{mytheorembg}{HTML}{F2F2F9}
\definecolor{mytheoremfr}{HTML}{00007B}
\definecolor{mylenmabg}{HTML}{FFFAF8}
\definecolor{mylenmafr}{HTML}{983b0f}
\definecolor{mypropbg}{HTML}{f2fbfc}
\definecolor{mypropfr}{HTML}{191971}
\definecolor{myexamplebg}{HTML}{F2FBF8}
\definecolor{myexamplefr}{HTML}{88D6D1}
\definecolor{myexampleti}{HTML}{2A7F7F}
\definecolor{mydefinitbg}{HTML}{E5E5FF}
\definecolor{mydefinitfr}{HTML}{3F3FA3}
\definecolor{notesgreen}{RGB}{0,162,0}
\definecolor{myp}{RGB}{197, 92, 212}
\definecolor{mygr}{HTML}{2C3338}
\definecolor{myred}{RGB}{127,0,0}
\definecolor{myyellow}{RGB}{169,121,69}
\definecolor{myexercisebg}{HTML}{F2FBF8}
\definecolor{myexercisefg}{HTML}{88D6D1}


%%%%%%%%%%%%%%%%%%%%%%%%%%%%
% TCOLORBOX SETUPS
%%%%%%%%%%%%%%%%%%%%%%%%%%%%

\setlength{\parindent}{1cm}
%================================
% THEOREM BOX
%================================

\tcbuselibrary{theorems,skins,hooks}
\newtcbtheorem[number within=chapter]{Theorem}{Théorème}
{%
	enhanced,
	breakable,
	colback = mytheorembg,
	frame hidden,
	boxrule = 0sp,
	borderline west = {2pt}{0pt}{mytheoremfr},
	sharp corners,
	detach title,
	before upper = \tcbtitle\par\smallskip,
	coltitle = mytheoremfr,
	fonttitle = \bfseries\sffamily,
	description font = \mdseries,
	separator sign none,
	segmentation style={solid, mytheoremfr},
}
{th}


\tcbuselibrary{theorems,skins,hooks}
\newtcolorbox{Theoremcon}
{%
	enhanced
	,breakable
	,colback = mytheorembg
	,frame hidden
	,boxrule = 0sp
	,borderline west = {2pt}{0pt}{mytheoremfr}
	,sharp corners
	,description font = \mdseries
	,separator sign none
}

%================================
% Corollery
%================================
\tcbuselibrary{theorems,skins,hooks}
\newtcbtheorem[use counter=tcb@cnt@Theorem]{Corollary}{Corollaire}
{%
	enhanced
	,breakable
	,colback = myp!10
	,frame hidden
	,boxrule = 0sp
	,borderline west = {2pt}{0pt}{myp!85!black}
	,sharp corners
	,detach title
	,before upper = \tcbtitle\par\smallskip
	,coltitle = myp!85!black
	,fonttitle = \bfseries\sffamily
	,description font = \mdseries
	,separator sign none
	,segmentation style={solid, myp!85!black}
}
{th}

%================================
% LENMA
%================================

\tcbuselibrary{theorems,skins,hooks}
\newtcbtheorem[use counter=tcb@cnt@Theorem]{Lemma}{Lemme}
{%
	enhanced,
	breakable,
	colback = mylenmabg,
	frame hidden,
	boxrule = 0sp,
	borderline west = {2pt}{0pt}{mylenmafr},
	sharp corners,
	detach title,
	before upper = \tcbtitle\par\smallskip,
	coltitle = mylenmafr,
	fonttitle = \bfseries\sffamily,
	description font = \mdseries,
	separator sign none,
	segmentation style={solid, mylenmafr},
}
{th}


%================================
% PROPOSITION
%================================

\tcbuselibrary{theorems,skins,hooks}
\newtcbtheorem[use counter=tcb@cnt@Theorem]{Prop}{Proposition}
{%
	enhanced,
	breakable,
	colback = mypropbg,
	frame hidden,
	boxrule = 0sp,
	borderline west = {2pt}{0pt}{mypropfr},
	sharp corners,
	detach title,
	before upper = \tcbtitle\par\smallskip,
	coltitle = mypropfr,
	fonttitle = \bfseries\sffamily,
	description font = \mdseries,
	separator sign none,
	segmentation style={solid, mypropfr},
}
{th}


%================================
% CLAIM
%================================

\tcbuselibrary{theorems,skins,hooks}
\newtcbtheorem[use counter=tcb@cnt@Theorem]{claim}{Claim}
{%
	enhanced
	,breakable
	,colback = myg!10
	,frame hidden
	,boxrule = 0sp
	,borderline west = {2pt}{0pt}{myg}
	,sharp corners
	,detach title
	,before upper = \tcbtitle\par\smallskip
	,coltitle = myg!85!black
	,fonttitle = \bfseries\sffamily
	,description font = \mdseries
	,separator sign none
	,segmentation style={solid, myg!85!black}
}
{th}



%================================
% Exercise
%================================

\tcbuselibrary{theorems,skins,hooks}
\newtcbtheorem[use counter=tcb@cnt@Theorem]{Exercise}{Exercice}
{%
	enhanced,
	breakable,
	colback = myexercisebg,
	frame hidden,
	boxrule = 0sp,
	borderline west = {2pt}{0pt}{myexercisefg},
	sharp corners,
	detach title,
	before upper = \tcbtitle\par\smallskip,
	coltitle = myexercisefg,
	fonttitle = \bfseries\sffamily,
	description font = \mdseries,
	separator sign none,
	segmentation style={solid, myexercisefg},
}
{th}

%================================
% EXAMPLE BOX
%================================

\newtcbtheorem[use counter=tcb@cnt@Theorem]{Example}{Exemple}
{%
	colback = myexamplebg
	,breakable
	,colframe = myexamplefr
	,coltitle = myexampleti
	,boxrule = 1pt
	,sharp corners
	,detach title
	,before upper=\tcbtitle\par\smallskip
	,fonttitle = \bfseries
	,description font = \mdseries
	,separator sign none
	,description delimiters parenthesis
}
{ex}

%================================
% DEFINITION BOX
%================================

\newtcbtheorem[use counter=tcb@cnt@Theorem]{Definition}{Définition}{enhanced,
	before skip=2mm,after skip=2mm, colback=red!5,colframe=red!80!black,boxrule=0.5mm,
	attach boxed title to top left={xshift=1cm,yshift*=1mm-\tcboxedtitleheight}, varwidth boxed title*=-3cm,
	boxed title style={frame code={
					\path[fill=tcbcolback]
					([yshift=-1mm,xshift=-1mm]frame.north west)
					arc[start angle=0,end angle=180,radius=1mm]
					([yshift=-1mm,xshift=1mm]frame.north east)
					arc[start angle=180,end angle=0,radius=1mm];
					\path[left color=tcbcolback!60!black,right color=tcbcolback!60!black,
						middle color=tcbcolback!80!black]
					([xshift=-2mm]frame.north west) -- ([xshift=2mm]frame.north east)
					[rounded corners=1mm]-- ([xshift=1mm,yshift=-1mm]frame.north east)
					-- (frame.south east) -- (frame.south west)
					-- ([xshift=-1mm,yshift=-1mm]frame.north west)
					[sharp corners]-- cycle;
				},interior engine=empty,
		},
	fonttitle=\bfseries,
	title={#2},#1}{def}

%================================
% Solution BOX
%================================

\makeatletter
\newtcbtheorem[use counter=tcb@cnt@Theorem]{question}{Question}{enhanced,
	breakable,
	colback=white,
	colframe=myb!80!black,
	attach boxed title to top left={yshift*=-\tcboxedtitleheight},
	fonttitle=\bfseries,
	title={#2},
	boxed title size=title,
	boxed title style={%
			sharp corners,
			rounded corners=northwest,
			colback=tcbcolframe,
			boxrule=0pt,
		},
	underlay boxed title={%
			\path[fill=tcbcolframe] (title.south west)--(title.south east)
			to[out=0, in=180] ([xshift=5mm]title.east)--
			(title.center-|frame.east)
			[rounded corners=\kvtcb@arc] |-
			(frame.north) -| cycle;
		},
	#1
}{def}
\makeatother

%================================
% SOLUTION BOX
%================================

\makeatletter
\newtcolorbox{solution}{enhanced,
	breakable,
	colback=white,
	colframe=myg!80!black,
	attach boxed title to top left={yshift*=-\tcboxedtitleheight},
	title=Solution,
	boxed title size=title,
	boxed title style={%
			sharp corners,
			rounded corners=northwest,
			colback=tcbcolframe,
			boxrule=0pt,
		},
	underlay boxed title={%
			\path[fill=tcbcolframe] (title.south west)--(title.south east)
			to[out=0, in=180] ([xshift=5mm]title.east)--
			(title.center-|frame.east)
			[rounded corners=\kvtcb@arc] |-
			(frame.north) -| cycle;
		},
}
\makeatother

%================================
% Question BOX
%================================

\makeatletter
\newtcbtheorem[use counter=tcb@cnt@Theorem]{qstion}{Question}{enhanced,
	breakable,
	colback=white,
	colframe=mygr,
	attach boxed title to top left={yshift*=-\tcboxedtitleheight},
	fonttitle=\bfseries,
	title={#2},
	boxed title size=title,
	boxed title style={%
			sharp corners,
			rounded corners=northwest,
			colback=tcbcolframe,
			boxrule=0pt,
		},
	underlay boxed title={%
			\path[fill=tcbcolframe] (title.south west)--(title.south east)
			to[out=0, in=180] ([xshift=5mm]title.east)--
			(title.center-|frame.east)
			[rounded corners=\kvtcb@arc] |-
			(frame.north) -| cycle;
		},
	#1
}{def}
\makeatother

\newtcbtheorem[number within=chapter]{wconc}{Wrong Concept}{
	breakable,
	enhanced,
	colback=white,
	colframe=myr,
	arc=0pt,
	outer arc=0pt,
	fonttitle=\bfseries\sffamily\large,
	colbacktitle=myr,
	attach boxed title to top left={},
	boxed title style={
			enhanced,
			skin=enhancedfirst jigsaw,
			arc=3pt,
			bottom=0pt,
			interior style={fill=myr}
		},
	#1
}{def}



%================================
% NOTE BOX
%================================

\usetikzlibrary{arrows,calc,shadows.blur}
\tcbuselibrary{skins}
\newtcolorbox{note}[1][]{%
	enhanced jigsaw,
	colback=gray!20!white,%
	colframe=gray!80!black,
	size=small,
	boxrule=1pt,
	title=\colorbox{white!100}{\textbf{ Remarque }},
	halign title=flush center,
	coltitle=black,
	breakable,
	drop shadow=black!50!white,
	attach boxed title to top left={xshift=1cm,yshift=-\tcboxedtitleheight/2,yshifttext=-\tcboxedtitleheight/2},
	minipage boxed title=2.6cm,
	boxed title style={%
			colback=white,
			size=fbox,
			boxrule=1pt,
			boxsep=2pt,
			underlay={%
					\coordinate (dotA) at ($(interior.west) + (-0.5pt,0)$);
					\coordinate (dotB) at ($(interior.east) + (0.5pt,0)$);
					\begin{scope}
						\clip (interior.north west) rectangle ([xshift=3ex]interior.east);
						\filldraw [white, blur shadow={shadow opacity=60, shadow yshift=-.75ex}, rounded corners=2pt] (interior.north west) rectangle (interior.south east);
					\end{scope}
					\begin{scope}[gray!80!black]
						\fill (dotA) circle (2pt);
						\fill (dotB) circle (2pt);
					\end{scope}
				},
		},
	#1,
}

%================================
% STRATÉGIE BOX
%================================

\usetikzlibrary{arrows,calc,shadows.blur}
\tcbuselibrary{skins}
\newtcolorbox{strategy}[1][]{%
	enhanced jigsaw,
	colback=myb!20!white,%
	colframe=gray!80!black,
	size=small,
	boxrule=1pt,
	title=\colorbox{white!100}{\textbf{ Stratégie }},
	halign title=flush center,
	coltitle=black,
	breakable,
	drop shadow=black!50!white,
	attach boxed title to top left={xshift=1cm,yshift=-\tcboxedtitleheight/2,yshifttext=-\tcboxedtitleheight/2},
	minipage boxed title=2.5cm,
	boxed title style={%
			colback=white,
			size=fbox,
			boxrule=1pt,
			boxsep=2pt,
			underlay={%
					\coordinate (dotA) at ($(interior.west) + (-0.5pt,0)$);
					\coordinate (dotB) at ($(interior.east) + (0.5pt,0)$);
					\begin{scope}
						\clip (interior.north west) rectangle ([xshift=3ex]interior.east);
						\filldraw [white, blur shadow={shadow opacity=60, shadow yshift=-.75ex}, rounded corners=2pt] (interior.north west) rectangle (interior.south east);
					\end{scope}
					\begin{scope}[gray!80!black]
						\fill (dotA) circle (2pt);
						\fill (dotB) circle (2pt);
					\end{scope}
				},
		},
	#1,
}

%================================
% MÉTHODE BOX
%================================

\usetikzlibrary{arrows,calc,shadows.blur}
\tcbuselibrary{skins}
\newtcolorbox{methode}[1][]{%
	enhanced jigsaw,
	colback=white,%
	colframe=gray!80!black,
	size=small,
	boxrule=1pt,
	title=\textbf{Méthode},
	halign title=flush center,
	coltitle=black,
	breakable,
	drop shadow=black!50!white,
	attach boxed title to top left={xshift=1cm,yshift=-\tcboxedtitleheight/2,yshifttext=-\tcboxedtitleheight/2},
	minipage boxed title=2.5cm,
	boxed title style={%
			colback=white,
			size=fbox,
			boxrule=1pt,
			boxsep=2pt,
			underlay={%
					\coordinate (dotA) at ($(interior.west) + (-0.5pt,0)$);
					\coordinate (dotB) at ($(interior.east) + (0.5pt,0)$);
					\begin{scope}
						\clip (interior.north west) rectangle ([xshift=3ex]interior.east);
						\filldraw [white, blur shadow={shadow opacity=60, shadow yshift=-.75ex}, rounded corners=2pt] (interior.north west) rectangle (interior.south east);
					\end{scope}
					\begin{scope}[gray!80!black]
						\fill (dotA) circle (2pt);
						\fill (dotB) circle (2pt);
					\end{scope}
				},
		},
	#1,
}

%%%%%%%%%%%%%%%%%%%%%%%%%%%%%%%%%%%%%%%%%%%
% TABLE OF CONTENTS
%%%%%%%%%%%%%%%%%%%%%%%%%%%%%%%%%%%%%%%%%%%

\usepackage{tikz}

\definecolor{doc}{RGB}{0,60,110}
\usepackage{titletoc}
\contentsmargin{0cm}
\titlecontents{chapter}[3.7pc]
{\addvspace{30pt}%
	\begin{tikzpicture}[remember picture, overlay]%
		\draw[fill=doc!60,draw=doc!60] (-7,-.1) rectangle (-0.2,.6);%
		\pgftext[left,x=-3.5cm,y=0.2cm]{\color{white}\Large\sc\bfseries Chapitre\ \thecontentslabel};%
	\end{tikzpicture}\color{doc!60}\large\sc\bfseries}%
{}
{}
{\;\titlerule\;\large\sc\bfseries Page \thecontentspage
	\begin{tikzpicture}[remember picture, overlay]
		\draw[fill=doc!60,draw=doc!60] (2pt,0) rectangle (4,0.1pt);
	\end{tikzpicture}}%
\titlecontents{section}[3.7pc]
{\addvspace{2pt}}
{\contentslabel[\thecontentslabel]{2pc}}
{}
{\hfill\small \thecontentspage}
[]
\titlecontents*{subsection}[3.7pc]
{\addvspace{-1pt}\small}
{}
{}
{\ --- \small\thecontentspage}
[ \textbullet\ ][]

\makeatletter
\renewcommand{\tableofcontents}{%
	\chapter*{%
	  \vspace*{-20\p@}%
	  \begin{tikzpicture}[remember picture, overlay]%
		  \pgftext[right,x=15cm,y=0.2cm]{\color{doc!60}\Huge\sc\bfseries \contentsname};%
		  \draw[fill=doc!60,draw=doc!60] (13,-.75) rectangle (20,1);%
		  \clip (13,-.75) rectangle (20,1);
		  \pgftext[right,x=15cm,y=0.2cm]{\color{white}\Huge\sc\bfseries \contentsname};%
	  \end{tikzpicture}}%
	\@starttoc{toc}}
\makeatother


%%%%%%%%%%%%%%%%%%%%%%%%%%%%%%%%%%%%%%%%%%%
% MINTED FOR PYTHON ALGORITHMS
%%%%%%%%%%%%%%%%%%%%%%%%%%%%%%%%%%%%%%%%%%%

\usepackage{tcolorbox}
\tcbuselibrary{minted,breakable,xparse,skins}
\definecolor{bg}{gray}{0.95}
\DeclareTCBListing{mintedbox}{O{}m!O{}}{%
  breakable=true,
  listing engine=minted,
  listing only,
  minted language=#2,
  minted style=default,
  minted options={%
    linenos,
    gobble=0,
    breaklines=true,
    breakafter=,,
    fontsize=\small,
    numbersep=8pt,
    #1},
  boxsep=0pt,
  left skip=0pt,
  right skip=0pt,
  left=25pt,
  right=0pt,
  top=3pt,
  bottom=3pt,
  arc=5pt,
  leftrule=0pt,
  rightrule=0pt,
  bottomrule=2pt,
  toprule=2pt,
  colback=bg,
  colframe=orange!70,
  enhanced,
  overlay={%
    \begin{tcbclipinterior}
    \fill[orange!20!white] (frame.south west) rectangle ([xshift=20pt]frame.north west);
    \end{tcbclipinterior}},
  #3}
  
  
 % for braces
\usetikzlibrary{decorations.pathreplacing}


\AdvanceDate[0]

\begin{document}
\pagestyle{fancy}
\fancyhead[L]{Seconde 13}
\fancyhead[C]{\textbf{Fonctions affines 1\ifsolutions -- Solutions  \fi}}
\fancyhead[R]{\today}


\exe{
	Donner $4$ points appartenant à la courbe représentative de chaque fonction affine et l'esquisser dans un repère de domaine $\D = [-3 ; 3]$.
	\begin{multicols}{2}
	\begin{enumerate}
		\item $f(x) = 1 + 2x$
		\item $g(x) = 2 + 2x$
		\item $h(x) = - x$
		\item $F(x) = 1 - x$
		\item $G(x) = -\frac12$
		\item $H(x) = 2 - \frac13x$
	\end{enumerate}
	\end{multicols}
}{
	Un point de de la courbe représentative de $f$ prend la forme $(x ; f(x))$, où $x$ fait partie du domaine d'étude.
	
	Pour chaque fonction, les points de sa courbe représentative sont alignés.
	La courbe d'une fonction affine est en fait une droite.
	
	Les courbes représentatives sont données ci-dessous.
	\begin{center}
	\begin{tikzpicture}[>=stealth, scale=1]
		\begin{axis}[xmin = -3, xmax=3, ymin=-6, ymax=8, axis x line=middle, axis y line=middle, axis line style=<->, xlabel={}, ylabel={}, xtick = {-3, ..., 3}, ytick = {-10, -8, ..., 8, 10}, grid=both]
		
			\addplot[myg, thick, domain =-3:3, samples=2] {1+2*x}  node[pos=.2, below] {$(\mathcal{C}_f)$};
			\addplot[myr, thick, domain =-3:3, samples=2] {2+2*x}  node[pos=.7, above] {$(\mathcal{C}_g)$};
			\addplot[myb, thick, domain =-3:3, samples=2] {-x}  node[pos=.2, above] {$(\mathcal{C}_h)$};
		
			
		\end{axis}
	\end{tikzpicture}
	\begin{tikzpicture}[>=stealth, scale=1]
		\begin{axis}[xmin = -3, xmax=3, ymin=-6, ymax=8, axis x line=middle, axis y line=middle, axis line style=<->, xlabel={}, ylabel={}, xtick = {-3, ..., 3}, ytick = {-10, -8, ..., 8, 10}, grid=both]
		
			\addplot[myg, thick, domain =-3:3, samples=2] {1-x}  node[pos=.2, above] {$(\mathcal{C}_F)$};
			\addplot[myr, thick, domain =-3:3, samples=2] {-.5}  node[pos=.2, below] {$(\mathcal{C}_G)$};
			\addplot[myb, thick, domain =-3:3, samples=2] {2-x/3}  node[pos=.7, above] {$(\mathcal{C}_H)$};
		
			
		\end{axis}
	\end{tikzpicture}
	\end{center}

}

\exe{
	Pour chaque fonction affine sur $\R$ suivante, déterminer son coefficient directeur $a$ et son ordonnée à l'origine $b$.
	\begin{multicols}{2}
	\begin{enumerate}
		\item $f(x) = 2x + 1$
		\item $f(x) = 1 + 2x$
		\item $f(x) = - x$
		\item $f(x) = -42$
		\item $f(x) = \sqrt{3} \cdot x + 2$
		\item $f(x) = 2 - \sqrt{5} \cdot x$
		\item $f(x) = 1 - x$
		\item $f(x) = 0$
	\end{enumerate}
	\end{multicols}
}{
	On a en général
		\[ f(x) = a \cdot x + b. \]
	On lit donc le coefficient directeur $a$ comme le nombre réel qui multiplie $x$.
		\begin{itemize}
			\item Si $x$ n'apparaît pas, $f$ est constante, et $a$ est nul.
			\item Si aucun nombre n'apparait devant $x$, on peut le créer en écrivant $x = (1) \cdot x$.
			\item Idem pour $-x = (-1) \cdot x$ : on change d'écriture pour mettre en exergue le coefficient qui multiplie $x$.
		\end{itemize}
	
	On lit l'ordonnée à l'origine $b$ comme la partie constante de la fonction.
	Si rien n'apparait, on peut la créer en ajoutant $0$. Par exemple, $2x = 2x+0$.
	La relation 
		\[ f(0) = a \cdot 0 + b = b \]
	permet aussi de calculer $b$ en évaluant $f$ en $0$.

	\begin{enumerate}
		\item 
			\begin{align*}
				a = 2	&& b = 1
			\end{align*}
		\item
			\begin{align*}
				a = 2	&& b = 1
			\end{align*}
		\item 
			\begin{align*}
				a = -1	&& b = 0
			\end{align*}
		\item 
			\begin{align*}
				a = 0	&& b = -42
			\end{align*}
		\item 
			\begin{align*}
				a = \sqrt{3}	&& b = 2
			\end{align*}
		\item 
			\begin{align*}
				a = -\sqrt{5}	&& b = 2
			\end{align*}
		\item 
			\begin{align*}
				a = -1	&& b = 1
			\end{align*}
		\item
			\begin{align*}
				a = 0	&& b = 0
			\end{align*}
	\end{enumerate}


}


\exe{\label{ex:3} \, \\
		\begin{center}
		\ifdys
		\newcommand{\scale}{1.04}
		\else
		\newcommand{\scale}{1.095}
		\fi
		\begin{tikzpicture}[>=stealth, scale=\scale]
		\begin{axis}[xmin = -10, xmax=10, ymin=-10, ymax=10, axis x line=middle, axis y line=middle, axis line style=<->, xlabel={}, ylabel={}, xtick = {-10, -8, ..., 8, 10}, ytick = {-10, -8, ..., 8, 10}, grid=both]
		
			\addplot[myr, thick, domain =-9:9, samples=2] {-x}  node[pos=.9, above=6pt] {$(\mathcal{C}_f)$};
			\addplot[myr, thick, dotted, domain =-10:-9, samples=2] {-x} ;
			\addplot[myr, thick, dotted, domain =9:10, samples=2] {-x};
		
		
			\addplot[myg, thick, domain =-9:9, samples=2] {x/2+1}  node[pos=.9, below=6pt] {$(\mathcal{C}_g)$};
			\addplot[myg, thick, dotted, domain =-10:-9, samples=2] {x/2+1} ;
			\addplot[myg, thick, dotted, domain =9:10, samples=2] {x/2+1};
		
		
			\addplot[black, thick, domain =-9:9, samples=2] {7}  node[pos=.7, above] {$(\mathcal{C}_h)$};
			\addplot[black, thick, dotted, domain =-10:-9, samples=2] {7} ;
			\addplot[black, thick, dotted, domain =9:10, samples=2] {7};
		
			
		\end{axis}
		\end{tikzpicture}
		\begin{tikzpicture}[>=stealth, scale=\scale]
		\begin{axis}[xmin = -10, xmax=10, ymin=-10, ymax=10, axis x line=middle, axis y line=middle, axis line style=<->, xlabel={}, ylabel={}, xtick = {-10, -8, ..., 8, 10}, ytick = {-10, -8, ..., 8, 10}, grid=both]
		
			\addplot[myr, thick, domain =-9:9, samples=2] {-x/3 + 1}  node[pos=.9, below=5pt] {$(\mathcal{C}_F)$};
			\addplot[myr, thick, dotted, domain =-10:-9, samples=2] {-x/3 + 1} ;
			\addplot[myr, thick, dotted, domain =9:10, samples=2] {-x/3 + 1};
		
		
			\addplot[myg, thick, domain =-5:7, samples=2] {3*x/2-2}  node[pos=.2, left] {$(\mathcal{C}_G)$};
			\addplot[myg, thick, dotted, domain =-6:-5, samples=2] {3*x/2-2} ;
			\addplot[myg, thick, dotted, domain =7:8, samples=2] {3*x/2-2};
		
		
			\addplot[black, thick, domain =-9:9, samples=2] {-x/6 + 5}  node[pos=.2, above] {$(\mathcal{C}_H)$};
			\addplot[black, thick, dotted, domain =-10:-9, samples=2] {-x/6 + 5} ;
			\addplot[black, thick, dotted, domain =9:10, samples=2] {-x/6 + 5};
		
			
		\end{axis}
		\end{tikzpicture}
	\end{center}
	
	\begin{enumerate}
		\item Pour chacune des droites ci-dessus, donner $2$ points lui appartenant, dont un d'abscisse nulle.
		\item En déduire l'ordonnée à l'origine $b$ de chacune des fonctions affines $f, g, h, F, G,$ et $H$.
		%\item Donner le signe (strictement positif, strictement négatif, ou nul) du coefficient directeur $a$ de chacune des fonctions affines $f, g, h, F, G,$ et $H$.
	\end{enumerate}
}{
	\begin{enumerate}
		\item[$f$:] 
		\begin{enumerate}[label=\arabic*.]
			\item $(0;0)$ et $(2;-2)$ appartiennent à $\C_f$.
			\item On en déduit que $f(x) = ax + 0 = ax$ pour un certain paramètre $a$ encore indéterminé.
		\end{enumerate}
		\item[$g$:] 
		\begin{enumerate}[label=\arabic*.]
			\item $(0;1)$ et $(-2; 0)$ appartiennent à $\C_f$.
			\item On en déduit que $f(x) = ax + 1$ pour un certain paramètre $a$ encore indéterminé.
		\end{enumerate}
		\item[$h$:] 
		\begin{enumerate}[label=\arabic*.]
			\item $(0;7)$ et $(6;7)$ appartiennent à $\C_f$.
			\item On en déduit que $f(x) = ax + 7$ pour un certain paramètre $a$ encore indéterminé. Comme $h$ est constante, on peut deviner que $a=0$...
		\end{enumerate}
		\item[$F$:] 
		\begin{enumerate}[label=\arabic*.]
			\item $(0;1)$ et $(3;0)$ appartiennent à $\C_f$.
			\item On en déduit que $f(x) = ax + 1$ pour un certain paramètre $a$ encore indéterminé.
		\end{enumerate}
		\item[$G$:] 
		\begin{enumerate}[label=\arabic*.]
			\item $(0;-2)$ et $(-4;-8)$ appartiennent à $\C_f$.
			\item On en déduit que $f(x) = ax -2$ pour un certain paramètre $a$ encore indéterminé.
		\end{enumerate}
		\item[$H$:] 
		\begin{enumerate}[label=\arabic*.]
			\item $(0;5)$ et $(6;4)$ appartiennent à $\C_f$.
			\item On en déduit que $f(x) = ax +5$ pour un certain paramètre $a$ encore indéterminé.
		\end{enumerate}
	\end{enumerate}
}

\ifsolutions
\else
\newpage
\fi

\exe{
	\begin{multicols}{2}
	Soit $f(x) = ax+b$ la fonction affine donnée graphiquement pour $x\in[-0,6 ; 1]$ ci-contre.
	
	Déterminer les paramètres $a$ et $b$.
	\vfill
	
	\begin{center}
	\begin{tikzpicture}[>=stealth, scale=1]
	\begin{axis}[xmin = -1.2, xmax=1, ymin=-4, ymax=4, axis x line=middle, axis y line=middle, axis line style=<->, xlabel={}, ylabel={}, xtick = {-4, -3, ..., 4}, ytick = {-4, -3, ..., 4}, grid=both]
		
		% (d)
		\addplot[myb, thick, domain =-0.55:.9, samples=2] {-1+5*x}  node[pos = .8, above=2pt] {$\C_f$};
		\addplot[myb, thick, dotted, domain =-.6:-.55, samples=2] {-1+5*x} ;
		\addplot[myb, thick, dotted, domain =.9:1, samples=2] {-1+5*x};
		
		% 2 points
		\addplot[black, mark=*, mark size = 1] (.5,1.5) node[below] {$(0,5 ; 1,5)$};
		\addplot[black, mark=*, mark size = 1] (-.5,-3.5);
		\addplot[black] (-.7,-2.7) node{$(-0,5 ; -3,5)$};
	\end{axis}
	\end{tikzpicture}
	\end{center}
	\end{multicols}
}{
	Notons $f(x) = ax + b$ la fonction affine.
	On utilise le lemme du cours qui dit que $(0 ; b) \in \C_f$ pour déduire que $b=-1$ car $(0; -1) \in \C_f$.
	
	On utilise enfin le théorème du cours qui implique que, en choisissant $A(-0,5 ; -3,5)$ et $B(0,5;1,5)$,
		\[ a = \dfrac{y_B - y_A}{x_B - x_A} = \dfrac{1,5 - (-3,5)}{0,5 - (-0,5)} = \dfrac41 = 4. \]
	Remarquons que, quand $x$ augmente de $1$, $f(x)$ augmente de $4$, le coefficient directeur.
	De plus, l'ordre des points n'a en fait pas d'importance : on aurait pû choisir $B(-0,5 ; -3,5)$ et $A(0,5;1,5)$ pour trouver le même coefficient directeur.
	
	On conclut que 
		\[ f(x) = 4x - 1. \]
}

\exe{
	\begin{multicols}{2}
	Soit $g(x) = ax+b$ la fonction affine donnée graphiquement pour $x\in[-1 ; 3]$ ci-contre.
	
	Déterminer les paramètres $a$ et $b$.
	\vfill
	
		\begin{center}
		\begin{tikzpicture}[>=stealth, scale=1]
		\begin{axis}[xmin = -1, xmax=3, ymin=-4, ymax=4, axis x line=middle, axis y line=middle, axis line style=<->, xlabel={}, ylabel={}, xtick = {-4, -3, ..., 4}, ytick = {-4, -3, ..., 4}, grid=both]
			
			% (d)
			\addplot[myb, thick, domain =-0.8:2.6, samples=2] {2-2*x}  node[above=3pt] {$\C_g$};
			\addplot[myb, thick, dotted, domain =-1:-.8, samples=2] {2-2*x} ;
			\addplot[myb, thick, dotted, domain =2.6:3, samples=2] {2-2*x};
			
			% (0,b)
			\addplot[black, mark=*, mark size = 1] (0,2) node[above right] {$(0;2)$};
			
			% (1, b+a)
			\addplot[black, mark=*, mark size = 1] (1,0) node[above right] {$(1;0)$};
			
			% (2, b+2a)
			\addplot[black, mark=*, mark size = 1] (2,-2) node[above right] {$(2;-2)$};
		\end{axis}
		\end{tikzpicture}
		\end{center}
	\end{multicols}
}{
	Notons $g(x) = ax + b$ la fonction affine.
	On utilise le lemme du cours qui dit que $(0 ; b) \in \C_f$ pour déduire que $b=2$ car $(0; 2) \in \C_f$.
	
	On utilise enfin le théorème du cours qui implique que
		\[ a = \dfrac{y_B - y_A}{x_B - x_A}. \]
	Plusieurs choix s'offrent à nous pour les points $A$ et $B$.
	Si on prend $A(1;0)$ et $B(2;-2)$, on trouve
		\[ a = \dfrac{-2 - 0}{2 - 1} = -2. \]
	Si on prend $A(2 ; -2)$ et $B(0;2)$, on trouve
		\[ a = \dfrac{2 - (-2)}{0 - 2} = \dfrac{-4}{2} = -2. \]
	Il y a en fait $6$ choix possibles ($3$ pour $A$ puis $2$ pour $B$). Chacun d'eux donne (heureusement) $a=-2$.
	
	Donc
		\[ f(x) = -2x + 2. \]
}

\exe{
		Calculer le coefficient directeur $a$ de chacune des fonctions affines $f, g, h, F, G,$ et $H$ de l'exercice \ref{ex:3}.
}{
	On calcule le coefficient directeur à l'aide des paires de points qu'on a donné à l'exercice \ref{ex:3}.

	\begin{enumerate}
		\item[$f$:] 
			Les points $(0;0)$ et $(2;-2)$ appartiennent à $\C_f$.
			Donc 
				\[ a = \dfrac{-2 - 0}{2-0} = -1, \]
			et $f(x) = -x$.
		\item[$g$:] 
			Les points $(0;1)$ et $(-2;0)$ appartiennent à $\C_g$.
			Donc 
				\[ a = \dfrac{0 - 1}{-2-0} = \dfrac12, \]
			et $g(x) = \dfrac12 x + 1$.
		\item[$h$:] 
			Les points $(0;7)$ et $(6;7)$ appartiennent à $\C_h$.
			Donc 
				\[ a = \dfrac{7-7}{6-0} = 0, \]
			et $h(x) = 7$, une fonction constante.
		\item[$F$:] 
			Les points $(0;1)$ et $(3;0)$ appartiennent à $\C_F$.
			Donc 
				\[ a = \dfrac{0-1}{3-0} = -\dfrac13, \]
			et $F(x) = -\dfrac13 x + 1$, une fonction constante.
		\item[$G$:] 
			Les points $(0;-2)$ et $(-4;-8)$ appartiennent à $\C_G$.
			Donc 
				\[ a = \dfrac{-8-(-2)}{-4-0} = \dfrac{-6}{-4} = \dfrac32, \]
			et $G(x) = \dfrac32 x - 2$, une fonction constante.
		\item[$H$:] 
			Les points $(0;5)$ et $(6;4)$ appartiennent à $\C_H$.
			Donc 
				\[ a = \dfrac{4-5}{6-0} = \dfrac{-1}{6}, \]
			et $H(x) = -\dfrac16 x + 5$, une fonction constante.
	\end{enumerate}
}


\exe{[Interpolation]
	Pour chacune des paires de points $A, B$ suivantes, calculer les paramètres ($a$ et $b$) de l'unique fonction affine $f$ telle que $A, B \in \C_f$.
	
	\begin{multicols}{2}
	\begin{enumerate}
		\item $A(1;2), B(4;-4)$.
		\item $A(2;8), B(4;7)$.
		\item $A(4;7), B(2;8)$.
		\item $A(-3; -3), B(-2; -1)$.
		\item $A(2;5), B(-10; 5)$.
		\item $A(-3;4), B(12;-11)$.
	\end{enumerate}
	\end{multicols}

}{
	On utilise la convention $a, b \in \R$ pour désigner respectivement le coefficient directeur et l'ordonnée à l'origine de la fonction affine $f$.
	$f(x) = ax+b$ pour tout $x\in\R$.

	\begin{enumerate}
		\item 
			D'après le cours,
			\begin{align*}
				a = \dfrac{y_A - y_B}{x_A - x_B}, && b = \dfrac{x_A y_B - x_B y_A}{x_A - x_B}.
			\end{align*}
			Par suite,
			\begin{align*}
				a = \dfrac{2 - (-4)}{1 - 4} = -2, && b = \dfrac{1 \cdot (-4) - 4 \cdot 2}{1 - 4} = 4.
			\end{align*}
				
		\item 
			\begin{align*}
				a = \dfrac{8-7}{2-4} = -\dfrac12, && b = \dfrac{2\cdot7 - 4\cdot8}{2-4} = 9.
			\end{align*}
			
			Visualisation : le point $A$ est à gauche du point $B$ car son abscisse est plus petite, mais est plus haut que le point $B$, car son ordonnée est plus grande. 
			La droite $(AB)$ doit donc être décroissante, ce qu'on a bien trouvé car $a<0$.
		\item
			\begin{align*}
				a = \dfrac{7-8}{4-2} = -\dfrac12, && b = \dfrac{4\cdot8 - 2\cdot7}{4-2} = 9.
			\end{align*}
			
		Les formules ne dépendent bien sûr pas de l'ordre de $A$ et $B$ car la droite $(AB)$ est égale à la droite $(BA)$.
					
		\item
			\begin{align*}
				a = \dfrac{-3-(-1)}{-3-(-2)} = 2, && b = \dfrac{(-3)\cdot(-1) - (-3)\cdot(-2)}{-3 - (-2)} = 3.
			\end{align*}
			
		\item
			\begin{align*}
				a = \dfrac{5-5}{2-(-10)} = 0, && b = \dfrac{2\cdot5 - 5 \cdot(-10)}{2-(-10)} = 5.
			\end{align*}
			
			Visualisation : les deux points sont à la même hauteur car leurs abscisses sont égales.
			La droite $(AB)$ est donc constante (ou horizontale), d'où $a=0$, et sa hauteur est celle des points, d'où $b=5$.
			
		\item $A(-3;4), B(12;-11)$.
			\begin{align*}
				a = \dfrac{4-(-11)}{-3-12} = -1, && b = \dfrac{(-3)\cdot(-11) - 4\cdot12}{-3-12} = 1.
			\end{align*}
			
			Vérification : on vérifie que $A$ et $B$ vérifient bien l'équation $y=f(x) = -x +1$ en remplaçant les valeurs par les coordonnées.
			Ainsi pour $A$ on a bien $y=4$ à gauche, et $f(-3) = -(-3) + 1 = 4$ à droite.
			Pour $B$, on a $y=-11$ à gauche, et $f(12) = -12 + 1 = -11$ à droite.
	\end{enumerate}


}


\end{document}

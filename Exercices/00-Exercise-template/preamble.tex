\documentclass[a4paper, 12pt]{extarticle}
\usepackage[french]{babel}
\usepackage[
a4paper,
margin=2cm,
]{geometry}

\usepackage[utf8x]{inputenc}
%fonts
\usepackage{libertinus,libertinust1math}
\usepackage{amsmath,amsthm,amssymb,mathtools}

%virgules
\usepackage{icomma}

% HEADER, ARRAY, ENUM, MULTIOCL
\usepackage{fancyhdr}
\usepackage{array}
\usepackage{multicol, enumitem}
\newcolumntype{P}[1]{>{\centering\arraybackslash}p{#1}}
\usepackage{stackengine}
\newcommand\xrowht[2][0]{\addstackgap[.5\dimexpr#2\relax]{\vphantom{#1}}}

% theorems
\theoremstyle{theorem}
\newtheorem{thm}{Théorème}
\theoremstyle{plain}
\newtheorem*{sol}{Solution}
\theoremstyle{definition}
\newtheorem{ex}{Exercice}
\newtheorem{dfn}{Définition}
\newtheorem*{dfn*}{Définition}

% exercices
\usepackage[answerdelayed, lastexercise]{exercise}
\renewcommand{\ExerciseHeader}{
	\textbf{
	\theExercise.
	\theExerciseDifficulty
	}
}
\renewcommand{\DifficultyMarker}{$\star$}
\renewcommand{\AnswerHeader}{
	\textbf{
	\theExercise.
	}
}
\newcommand{\exe}[4]{
	\begin{Exercise}[title=#1, label=#3]
		#2
	\end{Exercise}
	\begin{Answer}[ref=#3]
		#4
	\end{Answer}
}

\usepackage[framemethod=tikz]{mdframed}
% def
\surroundwithmdframed[
	hidealllines=true,
	%leftline=true,
	%bottomline=true,
	innerleftmargin=10pt,
	innerrightmargin=10pt,
	innertopmargin=-4pt,
	nobreak=true,
]{Exercise}

%couleurs
\usepackage{tcolorbox}
\definecolor{myg}{RGB}{56, 140, 70}
\definecolor{myb}{RGB}{45, 111, 177}
\definecolor{myr}{RGB}{199, 68, 64}
\definecolor{mygr}{HTML}{2C3338}

%thm
\tcbuselibrary{theorems,skins,hooks}
\newcounter{commonbox}
\makeatletter
\newtcbtheorem[use counter=commonbox]{exercice}{Exercice}%
{
	enhanced,
	colback=white,
	colframe=mygr,
	separator sign none,
	attach boxed title to top left={yshift*=-\tcboxedtitleheight},
	fonttitle=\bfseries,
	title={#2},
	boxed title size=title,
	boxed title style={%
			sharp corners,
			rounded corners=northwest,
			colback=tcbcolframe,
			boxrule=0pt,
		},
	underlay boxed title={%
			\path[fill=tcbcolframe] (title.south west)--(title.south east)
			to[out=0, in=180] ([xshift=5mm]title.east)--
			(title.center-|frame.east)
			[rounded corners=\kvtcb@arc] |-
			(frame.north) -| cycle;
		},
	#1
}{th}
\makeatother

% corps
\newcommand{\R}{\mathbb{R}}
\newcommand{\Rnn}{\mathbb{R}^{2n}}
\newcommand{\Z}{\mathbb{Z}}
\newcommand{\N}{\mathbb{N}}
\newcommand{\Q}{\mathbb{Q}}

% domain
\newcommand{\D}{\mathcal{D}}
% for calligraphic C
\usepackage{calrsfs}
\newcommand{\C}{\mathcal{C}}

% date
\usepackage{advdate}

% ensembles tq. 
\newcommand{\xRtq}[1]{
	$\left\{ x \in \R \text{ tq. } #1 \right\}$
}

% vabs
\newcommand{\vabs}[1]{
	\left| #1 \right|
}

%pinfty minfty
\newcommand{\pinfty}{{+}\infty}
\newcommand{\minfty}{{-}\infty}

% plots
\usepackage{pgfplots}

%subfigures
\usepackage{subcaption}

%hyperlink footnote
\usepackage{hyperref}

%wider tabulars
%\def\arraystretch{2}
\setlength\tabcolsep{15pt}

% vecteur
\renewcommand{\vec}[1]{\overrightarrow{#1}}

% vecteur pmatrix
\newcommand{\pvec}[2]{\begin{pmatrix} #1 \\ #2 \end{pmatrix}}

% vecteur norme
\newcommand{\norm}[1]{\left\Vert #1 \right\Vert}

% tableaux var, signe
\usepackage{tkz-tab}

%wider tabulars
\def\arraystretch{2}
\setlength\tabcolsep{15pt}
\usepackage{makecell} %pour \thead dans tabular ex3 (aligner verticalement le coeff de proportionnalité)


% vrai/faux tick
\newcommand{\ifsol}[1]{\ifsolutions #1 \fi}

% package systeme pour les systèmes d'équations bien alignés
\usepackage{systeme}
\sysalign{r,r}
\syseqspace{3pt}
\syssignspace{3pt}

\newcommand{\sys}[2]{\systeme{#1{,}, #2.}}

% for striked out implies sign (\centernot\implies)
\usepackage{centernot}

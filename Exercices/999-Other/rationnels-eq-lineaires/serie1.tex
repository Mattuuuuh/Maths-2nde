%!TEX encoding = UTF8
%!TEX root =notes.tex


%%%%%%%%%%%%%%%%%%%%%%%%%%%%%%%%%
% PACKAGE IMPORTS
%%%%%%%%%%%%%%%%%%%%%%%%%%%%%%%%%


\usepackage[french]{babel}

\usepackage[tmargin=2cm,rmargin=1in,lmargin=1in,margin=0.85in,bmargin=2cm,footskip=.2in]{geometry}
\usepackage{amsmath,amsfonts,amsthm,amssymb,mathtools}
\usepackage[varbb]{newpxmath}
\usepackage{xfrac}
\usepackage[makeroom]{cancel}
\usepackage{mathtools}
\usepackage{bookmark}
\usepackage{enumitem}
\usepackage{hyperref,theoremref}
\hypersetup{
	pdftitle={Assignment},
	colorlinks=true, linkcolor=doc!90,
	bookmarksnumbered=true,
	bookmarksopen=true
}
\usepackage[most,many,breakable]{tcolorbox}
\usepackage{xcolor}
\usepackage{varwidth}
\usepackage{varwidth}
\usepackage{etoolbox}
%\usepackage{authblk}
\usepackage{nameref}
\usepackage{multicol,array}
\usepackage{tikz-cd}
\usepackage[ruled,vlined,linesnumbered]{algorithm2e}
\usepackage{comment} % enables the use of multi-line comments (\ifx \fi) 
\usepackage{import}
\usepackage{xifthen}
\usepackage{pdfpages}
\usepackage{transparent}


\newcommand\mycommfont[1]{\footnotesize\ttfamily\textcolor{blue}{#1}}
\SetCommentSty{mycommfont}
\newcommand{\incfig}[1]{%
    \def\svgwidth{\columnwidth}
    \import{./figures/}{#1.pdf_tex}
}

\usepackage{tikzsymbols}
%\renewcommand\qedsymbol{$\Laughey$}


%\usepackage{import}
%\usepackage{xifthen}
%\usepackage{pdfpages}
%\usepackage{transparent}


%%%%%%%%%%%%%%%%%%%%%%%%%%%%%%
% SELF MADE COLORS
%%%%%%%%%%%%%%%%%%%%%%%%%%%%%%



\definecolor{myg}{RGB}{56, 140, 70}
\definecolor{myb}{RGB}{45, 111, 177}
\definecolor{myr}{RGB}{199, 68, 64}
\definecolor{mytheorembg}{HTML}{F2F2F9}
\definecolor{mytheoremfr}{HTML}{00007B}
\definecolor{mylenmabg}{HTML}{FFFAF8}
\definecolor{mylenmafr}{HTML}{983b0f}
\definecolor{mypropbg}{HTML}{f2fbfc}
\definecolor{mypropfr}{HTML}{191971}
\definecolor{myexamplebg}{HTML}{F2FBF8}
\definecolor{myexamplefr}{HTML}{88D6D1}
\definecolor{myexampleti}{HTML}{2A7F7F}
\definecolor{mydefinitbg}{HTML}{E5E5FF}
\definecolor{mydefinitfr}{HTML}{3F3FA3}
\definecolor{notesgreen}{RGB}{0,162,0}
\definecolor{myp}{RGB}{197, 92, 212}
\definecolor{mygr}{HTML}{2C3338}
\definecolor{myred}{RGB}{127,0,0}
\definecolor{myyellow}{RGB}{169,121,69}
\definecolor{myexercisebg}{HTML}{F2FBF8}
\definecolor{myexercisefg}{HTML}{88D6D1}


%%%%%%%%%%%%%%%%%%%%%%%%%%%%
% TCOLORBOX SETUPS
%%%%%%%%%%%%%%%%%%%%%%%%%%%%

\setlength{\parindent}{1cm}
%================================
% THEOREM BOX
%================================

\tcbuselibrary{theorems,skins,hooks}
\newtcbtheorem[number within=chapter]{Theorem}{Théorème}
{%
	enhanced,
	breakable,
	colback = mytheorembg,
	frame hidden,
	boxrule = 0sp,
	borderline west = {2pt}{0pt}{mytheoremfr},
	sharp corners,
	detach title,
	before upper = \tcbtitle\par\smallskip,
	coltitle = mytheoremfr,
	fonttitle = \bfseries\sffamily,
	description font = \mdseries,
	separator sign none,
	segmentation style={solid, mytheoremfr},
}
{th}


\tcbuselibrary{theorems,skins,hooks}
\newtcolorbox{Theoremcon}
{%
	enhanced
	,breakable
	,colback = mytheorembg
	,frame hidden
	,boxrule = 0sp
	,borderline west = {2pt}{0pt}{mytheoremfr}
	,sharp corners
	,description font = \mdseries
	,separator sign none
}

%================================
% Corollery
%================================
\tcbuselibrary{theorems,skins,hooks}
\newtcbtheorem[use counter=tcb@cnt@Theorem]{Corollary}{Corollaire}
{%
	enhanced
	,breakable
	,colback = myp!10
	,frame hidden
	,boxrule = 0sp
	,borderline west = {2pt}{0pt}{myp!85!black}
	,sharp corners
	,detach title
	,before upper = \tcbtitle\par\smallskip
	,coltitle = myp!85!black
	,fonttitle = \bfseries\sffamily
	,description font = \mdseries
	,separator sign none
	,segmentation style={solid, myp!85!black}
}
{th}

%================================
% LENMA
%================================

\tcbuselibrary{theorems,skins,hooks}
\newtcbtheorem[use counter=tcb@cnt@Theorem]{Lemma}{Lemme}
{%
	enhanced,
	breakable,
	colback = mylenmabg,
	frame hidden,
	boxrule = 0sp,
	borderline west = {2pt}{0pt}{mylenmafr},
	sharp corners,
	detach title,
	before upper = \tcbtitle\par\smallskip,
	coltitle = mylenmafr,
	fonttitle = \bfseries\sffamily,
	description font = \mdseries,
	separator sign none,
	segmentation style={solid, mylenmafr},
}
{th}


%================================
% PROPOSITION
%================================

\tcbuselibrary{theorems,skins,hooks}
\newtcbtheorem[use counter=tcb@cnt@Theorem]{Prop}{Proposition}
{%
	enhanced,
	breakable,
	colback = mypropbg,
	frame hidden,
	boxrule = 0sp,
	borderline west = {2pt}{0pt}{mypropfr},
	sharp corners,
	detach title,
	before upper = \tcbtitle\par\smallskip,
	coltitle = mypropfr,
	fonttitle = \bfseries\sffamily,
	description font = \mdseries,
	separator sign none,
	segmentation style={solid, mypropfr},
}
{th}


%================================
% CLAIM
%================================

\tcbuselibrary{theorems,skins,hooks}
\newtcbtheorem[use counter=tcb@cnt@Theorem]{claim}{Claim}
{%
	enhanced
	,breakable
	,colback = myg!10
	,frame hidden
	,boxrule = 0sp
	,borderline west = {2pt}{0pt}{myg}
	,sharp corners
	,detach title
	,before upper = \tcbtitle\par\smallskip
	,coltitle = myg!85!black
	,fonttitle = \bfseries\sffamily
	,description font = \mdseries
	,separator sign none
	,segmentation style={solid, myg!85!black}
}
{th}



%================================
% Exercise
%================================

\tcbuselibrary{theorems,skins,hooks}
\newtcbtheorem[use counter=tcb@cnt@Theorem]{Exercise}{Exercice}
{%
	enhanced,
	breakable,
	colback = myexercisebg,
	frame hidden,
	boxrule = 0sp,
	borderline west = {2pt}{0pt}{myexercisefg},
	sharp corners,
	detach title,
	before upper = \tcbtitle\par\smallskip,
	coltitle = myexercisefg,
	fonttitle = \bfseries\sffamily,
	description font = \mdseries,
	separator sign none,
	segmentation style={solid, myexercisefg},
}
{th}

%================================
% EXAMPLE BOX
%================================

\newtcbtheorem[use counter=tcb@cnt@Theorem]{Example}{Exemple}
{%
	colback = myexamplebg
	,breakable
	,colframe = myexamplefr
	,coltitle = myexampleti
	,boxrule = 1pt
	,sharp corners
	,detach title
	,before upper=\tcbtitle\par\smallskip
	,fonttitle = \bfseries
	,description font = \mdseries
	,separator sign none
	,description delimiters parenthesis
}
{ex}

%================================
% DEFINITION BOX
%================================

\newtcbtheorem[use counter=tcb@cnt@Theorem]{Definition}{Définition}{enhanced,
	before skip=2mm,after skip=2mm, colback=red!5,colframe=red!80!black,boxrule=0.5mm,
	attach boxed title to top left={xshift=1cm,yshift*=1mm-\tcboxedtitleheight}, varwidth boxed title*=-3cm,
	boxed title style={frame code={
					\path[fill=tcbcolback]
					([yshift=-1mm,xshift=-1mm]frame.north west)
					arc[start angle=0,end angle=180,radius=1mm]
					([yshift=-1mm,xshift=1mm]frame.north east)
					arc[start angle=180,end angle=0,radius=1mm];
					\path[left color=tcbcolback!60!black,right color=tcbcolback!60!black,
						middle color=tcbcolback!80!black]
					([xshift=-2mm]frame.north west) -- ([xshift=2mm]frame.north east)
					[rounded corners=1mm]-- ([xshift=1mm,yshift=-1mm]frame.north east)
					-- (frame.south east) -- (frame.south west)
					-- ([xshift=-1mm,yshift=-1mm]frame.north west)
					[sharp corners]-- cycle;
				},interior engine=empty,
		},
	fonttitle=\bfseries,
	title={#2},#1}{def}

%================================
% Solution BOX
%================================

\makeatletter
\newtcbtheorem[use counter=tcb@cnt@Theorem]{question}{Question}{enhanced,
	breakable,
	colback=white,
	colframe=myb!80!black,
	attach boxed title to top left={yshift*=-\tcboxedtitleheight},
	fonttitle=\bfseries,
	title={#2},
	boxed title size=title,
	boxed title style={%
			sharp corners,
			rounded corners=northwest,
			colback=tcbcolframe,
			boxrule=0pt,
		},
	underlay boxed title={%
			\path[fill=tcbcolframe] (title.south west)--(title.south east)
			to[out=0, in=180] ([xshift=5mm]title.east)--
			(title.center-|frame.east)
			[rounded corners=\kvtcb@arc] |-
			(frame.north) -| cycle;
		},
	#1
}{def}
\makeatother

%================================
% SOLUTION BOX
%================================

\makeatletter
\newtcolorbox{solution}{enhanced,
	breakable,
	colback=white,
	colframe=myg!80!black,
	attach boxed title to top left={yshift*=-\tcboxedtitleheight},
	title=Solution,
	boxed title size=title,
	boxed title style={%
			sharp corners,
			rounded corners=northwest,
			colback=tcbcolframe,
			boxrule=0pt,
		},
	underlay boxed title={%
			\path[fill=tcbcolframe] (title.south west)--(title.south east)
			to[out=0, in=180] ([xshift=5mm]title.east)--
			(title.center-|frame.east)
			[rounded corners=\kvtcb@arc] |-
			(frame.north) -| cycle;
		},
}
\makeatother

%================================
% Question BOX
%================================

\makeatletter
\newtcbtheorem[use counter=tcb@cnt@Theorem]{qstion}{Question}{enhanced,
	breakable,
	colback=white,
	colframe=mygr,
	attach boxed title to top left={yshift*=-\tcboxedtitleheight},
	fonttitle=\bfseries,
	title={#2},
	boxed title size=title,
	boxed title style={%
			sharp corners,
			rounded corners=northwest,
			colback=tcbcolframe,
			boxrule=0pt,
		},
	underlay boxed title={%
			\path[fill=tcbcolframe] (title.south west)--(title.south east)
			to[out=0, in=180] ([xshift=5mm]title.east)--
			(title.center-|frame.east)
			[rounded corners=\kvtcb@arc] |-
			(frame.north) -| cycle;
		},
	#1
}{def}
\makeatother

\newtcbtheorem[number within=chapter]{wconc}{Wrong Concept}{
	breakable,
	enhanced,
	colback=white,
	colframe=myr,
	arc=0pt,
	outer arc=0pt,
	fonttitle=\bfseries\sffamily\large,
	colbacktitle=myr,
	attach boxed title to top left={},
	boxed title style={
			enhanced,
			skin=enhancedfirst jigsaw,
			arc=3pt,
			bottom=0pt,
			interior style={fill=myr}
		},
	#1
}{def}



%================================
% NOTE BOX
%================================

\usetikzlibrary{arrows,calc,shadows.blur}
\tcbuselibrary{skins}
\newtcolorbox{note}[1][]{%
	enhanced jigsaw,
	colback=gray!20!white,%
	colframe=gray!80!black,
	size=small,
	boxrule=1pt,
	title=\colorbox{white!100}{\textbf{ Remarque }},
	halign title=flush center,
	coltitle=black,
	breakable,
	drop shadow=black!50!white,
	attach boxed title to top left={xshift=1cm,yshift=-\tcboxedtitleheight/2,yshifttext=-\tcboxedtitleheight/2},
	minipage boxed title=2.6cm,
	boxed title style={%
			colback=white,
			size=fbox,
			boxrule=1pt,
			boxsep=2pt,
			underlay={%
					\coordinate (dotA) at ($(interior.west) + (-0.5pt,0)$);
					\coordinate (dotB) at ($(interior.east) + (0.5pt,0)$);
					\begin{scope}
						\clip (interior.north west) rectangle ([xshift=3ex]interior.east);
						\filldraw [white, blur shadow={shadow opacity=60, shadow yshift=-.75ex}, rounded corners=2pt] (interior.north west) rectangle (interior.south east);
					\end{scope}
					\begin{scope}[gray!80!black]
						\fill (dotA) circle (2pt);
						\fill (dotB) circle (2pt);
					\end{scope}
				},
		},
	#1,
}

%================================
% STRATÉGIE BOX
%================================

\usetikzlibrary{arrows,calc,shadows.blur}
\tcbuselibrary{skins}
\newtcolorbox{strategy}[1][]{%
	enhanced jigsaw,
	colback=myb!20!white,%
	colframe=gray!80!black,
	size=small,
	boxrule=1pt,
	title=\colorbox{white!100}{\textbf{ Stratégie }},
	halign title=flush center,
	coltitle=black,
	breakable,
	drop shadow=black!50!white,
	attach boxed title to top left={xshift=1cm,yshift=-\tcboxedtitleheight/2,yshifttext=-\tcboxedtitleheight/2},
	minipage boxed title=2.5cm,
	boxed title style={%
			colback=white,
			size=fbox,
			boxrule=1pt,
			boxsep=2pt,
			underlay={%
					\coordinate (dotA) at ($(interior.west) + (-0.5pt,0)$);
					\coordinate (dotB) at ($(interior.east) + (0.5pt,0)$);
					\begin{scope}
						\clip (interior.north west) rectangle ([xshift=3ex]interior.east);
						\filldraw [white, blur shadow={shadow opacity=60, shadow yshift=-.75ex}, rounded corners=2pt] (interior.north west) rectangle (interior.south east);
					\end{scope}
					\begin{scope}[gray!80!black]
						\fill (dotA) circle (2pt);
						\fill (dotB) circle (2pt);
					\end{scope}
				},
		},
	#1,
}

%================================
% MÉTHODE BOX
%================================

\usetikzlibrary{arrows,calc,shadows.blur}
\tcbuselibrary{skins}
\newtcolorbox{methode}[1][]{%
	enhanced jigsaw,
	colback=white,%
	colframe=gray!80!black,
	size=small,
	boxrule=1pt,
	title=\textbf{Méthode},
	halign title=flush center,
	coltitle=black,
	breakable,
	drop shadow=black!50!white,
	attach boxed title to top left={xshift=1cm,yshift=-\tcboxedtitleheight/2,yshifttext=-\tcboxedtitleheight/2},
	minipage boxed title=2.5cm,
	boxed title style={%
			colback=white,
			size=fbox,
			boxrule=1pt,
			boxsep=2pt,
			underlay={%
					\coordinate (dotA) at ($(interior.west) + (-0.5pt,0)$);
					\coordinate (dotB) at ($(interior.east) + (0.5pt,0)$);
					\begin{scope}
						\clip (interior.north west) rectangle ([xshift=3ex]interior.east);
						\filldraw [white, blur shadow={shadow opacity=60, shadow yshift=-.75ex}, rounded corners=2pt] (interior.north west) rectangle (interior.south east);
					\end{scope}
					\begin{scope}[gray!80!black]
						\fill (dotA) circle (2pt);
						\fill (dotB) circle (2pt);
					\end{scope}
				},
		},
	#1,
}

%%%%%%%%%%%%%%%%%%%%%%%%%%%%%%%%%%%%%%%%%%%
% TABLE OF CONTENTS
%%%%%%%%%%%%%%%%%%%%%%%%%%%%%%%%%%%%%%%%%%%

\usepackage{tikz}

\definecolor{doc}{RGB}{0,60,110}
\usepackage{titletoc}
\contentsmargin{0cm}
\titlecontents{chapter}[3.7pc]
{\addvspace{30pt}%
	\begin{tikzpicture}[remember picture, overlay]%
		\draw[fill=doc!60,draw=doc!60] (-7,-.1) rectangle (-0.2,.6);%
		\pgftext[left,x=-3.5cm,y=0.2cm]{\color{white}\Large\sc\bfseries Chapitre\ \thecontentslabel};%
	\end{tikzpicture}\color{doc!60}\large\sc\bfseries}%
{}
{}
{\;\titlerule\;\large\sc\bfseries Page \thecontentspage
	\begin{tikzpicture}[remember picture, overlay]
		\draw[fill=doc!60,draw=doc!60] (2pt,0) rectangle (4,0.1pt);
	\end{tikzpicture}}%
\titlecontents{section}[3.7pc]
{\addvspace{2pt}}
{\contentslabel[\thecontentslabel]{2pc}}
{}
{\hfill\small \thecontentspage}
[]
\titlecontents*{subsection}[3.7pc]
{\addvspace{-1pt}\small}
{}
{}
{\ --- \small\thecontentspage}
[ \textbullet\ ][]

\makeatletter
\renewcommand{\tableofcontents}{%
	\chapter*{%
	  \vspace*{-20\p@}%
	  \begin{tikzpicture}[remember picture, overlay]%
		  \pgftext[right,x=15cm,y=0.2cm]{\color{doc!60}\Huge\sc\bfseries \contentsname};%
		  \draw[fill=doc!60,draw=doc!60] (13,-.75) rectangle (20,1);%
		  \clip (13,-.75) rectangle (20,1);
		  \pgftext[right,x=15cm,y=0.2cm]{\color{white}\Huge\sc\bfseries \contentsname};%
	  \end{tikzpicture}}%
	\@starttoc{toc}}
\makeatother


%%%%%%%%%%%%%%%%%%%%%%%%%%%%%%%%%%%%%%%%%%%
% MINTED FOR PYTHON ALGORITHMS
%%%%%%%%%%%%%%%%%%%%%%%%%%%%%%%%%%%%%%%%%%%

\usepackage{tcolorbox}
\tcbuselibrary{minted,breakable,xparse,skins}
\definecolor{bg}{gray}{0.95}
\DeclareTCBListing{mintedbox}{O{}m!O{}}{%
  breakable=true,
  listing engine=minted,
  listing only,
  minted language=#2,
  minted style=default,
  minted options={%
    linenos,
    gobble=0,
    breaklines=true,
    breakafter=,,
    fontsize=\small,
    numbersep=8pt,
    #1},
  boxsep=0pt,
  left skip=0pt,
  right skip=0pt,
  left=25pt,
  right=0pt,
  top=3pt,
  bottom=3pt,
  arc=5pt,
  leftrule=0pt,
  rightrule=0pt,
  bottomrule=2pt,
  toprule=2pt,
  colback=bg,
  colframe=orange!70,
  enhanced,
  overlay={%
    \begin{tcbclipinterior}
    \fill[orange!20!white] (frame.south west) rectangle ([xshift=20pt]frame.north west);
    \end{tcbclipinterior}},
  #3}
  
  
 % for braces
\usetikzlibrary{decorations.pathreplacing}


\SetDate[12/02/2026]

\begin{document}
\pagestyle{fancy}
\fancyhead[L]{Seconde}
\fancyhead[C]{\textbf{Rationnels et équations linéaires}}
\fancyhead[R]{\today}

\subsection*{Manipulations de rationnels}

\exe{}{
	Écrire les sommes et différences de rationnels sous forme d'une fraction irréductible.
	\begin{multicols}{3}
	\begin{enumerate}[itemsep=10pt]
		\item $1 - \dfrac12$
		\item $\dfrac78 + \dfrac{9}{40}$
		\item $\dfrac{3}{4} + \dfrac{2}{5}$
		\item $\dfrac{7}{6} - \dfrac{5}{8}$
		\item $\dfrac{9}{10} + \dfrac{2}{3} - \dfrac{1}{4}$
		\item $\dfrac{-5}{7} + \dfrac{8}{9}$
%		\item $\dfrac{4}{5} - \dfrac{3}{10} + \dfrac{-5}{6}$
%		\item $\dfrac{-1}{2} + \dfrac{7}{8}$
%		\item $\dfrac{11}{12} - \dfrac{5}{9} + \dfrac{7}{10}$
%		\item $\dfrac{2}{3} - \dfrac{5}{4}$
%		\item $\dfrac{-1}{3} + \dfrac{7}{9} - \dfrac{5}{6}$
%		\item $\dfrac{3}{8} - \dfrac{1}{2}$
	\end{enumerate}
	\end{multicols}
}{exe:rationnals}
{
	\begin{multicols}{2}
	\begin{enumerate}
		\item $1 - \dfrac{1}{2} = \dfrac{1}{2}$
		\item $\dfrac{7}{8} + \dfrac{9}{40} = \dfrac{11}{10}$
		\item $\dfrac{3}{4} + \dfrac{2}{5} = \dfrac{23}{20}$
		\item $\dfrac{7}{6} - \dfrac{5}{8} = \dfrac{13}{24}$
		\item $\dfrac{9}{10} + \dfrac{2}{3} - \dfrac{1}{4} = \dfrac{79}{60}$
		\item $\dfrac{-5}{7} + \dfrac{8}{9} = \dfrac{11}{63}$
%		\item $\dfrac{4}{5} - \dfrac{3}{10} + \dfrac{-5}{6} = \dfrac{-1}{3}$
%		\item $\dfrac{-1}{2} + \dfrac{7}{8} = \dfrac{3}{8}$
%		\item $\dfrac{11}{12} - \dfrac{5}{9} + \dfrac{7}{10} = \dfrac{191}{180}$
%		\item $\dfrac{2}{3} - \dfrac{5}{4} = \dfrac{-7}{12}$
%		\item $\dfrac{-1}{3} + \dfrac{7}{9} - \dfrac{5}{6} = \dfrac{-7}{18}$
%		\item $\dfrac{3}{8} - \dfrac{1}{2} = \dfrac{-1}{8}$
	\end{enumerate}
	\end{multicols}
}

\begin{dfn}
	L'\emph{inverse} d'un nombre $a \in \R$ non nul désigne la quantité qui, lorsque multipliée par $a$, donne $1$.
	Elle est notée $\frac{1}{a}$ ou $a^{-1}$.
	
	Diviser par $a$ est équivalent à multipler par son inverse $\frac1a$.
\end{dfn}

\begin{remarque}
	L'inverse de $2$ est donc $0,5$ car $2\times0,5 = 1$. 
	D'où $\frac12 = 0,5$.
	
	La notation $\frac12$ est préférée à $0{,}5$ car généralement plus précise : on a vu que $\frac13$ n'est pas décimal et donc qu'on ne peut pas exprimer tous les rationnels de façon exacte en écriture décimale.
\end{remarque}
	
\begin{remarque}
	L'inverse de $\frac47$ est $\frac74$ car $\frac47 \times \frac74 = 1$.
	Autrement dit, $\frac{\ \ 1 \ \ }{\frac47} = \left( \frac47 \right)^{-1} = \frac74$.
\end{remarque}

\exe{}{
	Écrire les produits et divisions de rationnels sous forme de fraction irréductible.
	
	\begin{multicols}{3}
	\begin{enumerate}[itemsep=10pt]
		\item $\dfrac{\ \ -3 \ \ }{\dfrac54}$
		\item $\dfrac{63}{20} \times \dfrac{15}{14}$
		\item $\dfrac{3}{4} \times \dfrac{2}{5}$
		\item $\dfrac{\ \ \dfrac{7}{6}\ \ }{\dfrac{5}{8}}$
		\item $\dfrac{9}{10} \times \dfrac{2}{3} \times \dfrac{4}{1}$
		\item $\dfrac{-5}{7} \times \dfrac{8}{9}$
%		\item $\left(\dfrac{4}{5}\right)^{-1} \times \dfrac{3}{10}$
%		\item $\dfrac{-1}{2} \times \left(\dfrac{7}{8}\right)^{-1}$
%		\item $\dfrac{11}{12} \times \dfrac{5}{9} \times \left(\dfrac{7}{10}\right)^{-1}$
%		\item $\dfrac{2}{3} \times \dfrac{5}{4}$
%		\item $\dfrac{-1}{3} \times \left(\dfrac{7}{9}\right)^{-1}$
%		\item $\dfrac{3}{8} \times \dfrac{1}{2}$
	\end{enumerate}
	\end{multicols}
}{exe:rationnals2}
{
	\begin{multicols}{2}
	\begin{enumerate}[itemsep=10pt]
		\item $\dfrac{-3}{\dfrac{5}{4}} = \dfrac{-12}{5}$
		\item $\dfrac{63}{20} \times \dfrac{15}{14} = \dfrac{27}{8}$
		\item $\dfrac{3}{4} \times \dfrac{2}{5} = \dfrac{3}{10}$
		\item $\dfrac{\ \ \dfrac{7}{6}\ \ }{\dfrac{5}{8}} = \dfrac{28}{15}$
		\item $\dfrac{9}{10} \times \dfrac{2}{3} \times \dfrac{4}{1} = \dfrac{12}{5}$
		\item $\dfrac{-5}{7} \times \dfrac{8}{9} = \dfrac{-40}{63}$
%		\item $\left(\dfrac{4}{5}\right)^{-1} \times \dfrac{3}{10} = \dfrac{3}{8}$
%		\item $\dfrac{-1}{2} \times \left(\dfrac{7}{8}\right)^{-1} = \dfrac{-4}{7}$
%		\item $\dfrac{11}{12} \times \dfrac{5}{9} \times \left(\dfrac{7}{10}\right)^{-1} = \dfrac{275}{378}$
%		\item $\dfrac{2}{3} \times \dfrac{5}{4} = \dfrac{5}{6}$
%		\item $\dfrac{-1}{3} \times \left(\dfrac{7}{9}\right)^{-1} = \dfrac{-3}{7}$
%		\item $\dfrac{3}{8} \times \dfrac{1}{2} = \dfrac{3}{16}$
	\end{enumerate}
	\end{multicols}
}


\subsection*{Équations linéaires}

\begin{dfn}
	Une équation linéaire est une identité du type
		\[ a  x + b = c  x + d, \]
	où $a, b, c, d \in \R$ sont des constantes réelles, et $x \in \R$ est un nombre réel à trouver à l'aide de l'équation.
\end{dfn}

\begin{remarque}
	Pour résoudre une équation linéaire, on regroupe les multiples de $x$ d'un côté et les constantes de l'autre : 
		\[ (a-c)  x = d - b. \]
	Si $a-c$ est non nul, on multiplie les deux côté par son inverse pour déduire 
		\[ x = (d-b)\times \dfrac{1}{a-c} =  \dfrac{d-b}{a-c}. \]
	Une réponse sous forme de fraction est généralement attendue.
\end{remarque}

\newpage

\exe{}{
	Trouver le $x\in\Q$ rationnel vérifiant chaque équation suivante, et l'exprimer sous forme de fraction irréductible.

	\begin{multicols}{2}
	\begin{enumerate}[itemsep=10pt]
		\item $3x + 5 = 2x + 7$
		\item $4x - 6 = 3x + 8$
		\item $5x + 10 = 2x + 4$
		\item $-2x + 3 = 4x - 1$
		\item $7x - 5 = 5x + 9$
		\item $6x + 4 = 3x + 12$
		\item $-3x + 2 = -5x - 4$
		\item $8x - 9 = 6x + 3$
		\item $2x + 11 = -3x + 5$
		\item $-4x + 7 = -x + 10$
	\end{enumerate}
	\end{multicols}
}{exe:inter-affine}
{
	\begin{multicols}{2}
	\begin{enumerate}[itemsep=10pt]
		\item $3x + 5 = 2x + 7 \iff x = 2$
		\item $4x - 6 = 3x + 8 \iff x = 14$
		\item $5x + 10 = 2x + 4 \iff x = -2$
		\item $-2x + 3 = 4x - 1 \iff x = \dfrac{2}{3}$
		\item $7x - 5 = 5x + 9 \iff x = 7$
		\item $6x + 4 = 3x + 12 \iff x = \dfrac{8}{3}$
		\item $-3x + 2 = -5x - 4 \iff x = -3$
		\item $8x - 9 = 6x + 3 \iff x = 6$
		\item $2x + 11 = -3x + 5 \iff x = \dfrac{-6}{5}$
		\item $-4x + 7 = -x + 10 \iff x = -1$
	\end{enumerate}
	\end{multicols}
}

\exe{}{
	Trouver le $x \in \Q$ rationnel non nul vérifiant chaque équation suivante, et l'exprimer sous forme de fraction irréductible.
	
	\begin{multicols}{2}
	\begin{enumerate}[itemsep=10pt]
		\item $9 = \dfrac3x$
		\item $-4 = \dfrac2x$
		\item $0,2422 = \dfrac1x$
		\item $0,7 = \dfrac{0,1}x$
		\item $3 - \dfrac4x = -9 + \dfrac1x$
		\item $-2 + \dfrac{-2}x = \dfrac3x - 10$
	\end{enumerate}
	\end{multicols}

}{exe:eq-lin}
{
	\begin{multicols}{2}
	\begin{enumerate}[itemsep=10pt]
		\item $9 = \dfrac3x \iff x = \dfrac13$
		\item $-4 = \dfrac2x \iff x=-\dfrac12$
		\item $0,2422 = \dfrac1x \iff x = \dfrac{10000}{2422} = \dfrac{5000}{1211}$
		\item $0,7 = \dfrac{0,1}x \iff x = \dfrac17$
		\item $3 - \dfrac4x = -9 + \dfrac1x \iff x = \dfrac5{12}$
		\item $-2 + \dfrac{-2}x = \dfrac3x - 10 \iff x = \dfrac58$
	\end{enumerate}
	\end{multicols}
}


\exe{, difficulty=1}{
	Multipliez l'équation 
		\[ \dfrac{a}b + \dfrac{c}d = x \]
	par $b$ puis $d$ à gauche et à droite et en déduire la règle d'addition des fractions :
		\[ x = \dfrac{ad + cb}{bd}. \]
}{exe:common-denom}{
	En multipliant par $bd$, on trouve
		\[ (bd) x =  bd \left( \dfrac{a}b + \dfrac{c}d \right) = ad + bc, \]
	d'où
		\[ x = \dfrac{ad+bc}{bd}. \]
}


%%%%%%%%%%%

\newpage
\fancyhead[C]{\textbf{Solutions}}
\fancyfoot[C]{\thepage}
\shipoutAnswer

\end{document}
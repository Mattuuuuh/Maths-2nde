%% INPUT PREAMBLE.TEX
%% THEN INPUT VARS_{i}.ADR
%% THEN RUN THIS
%% DYSLEXIA SWITCH
\newif\ifdys
		
				% ENABLE or DISABLE font change
				% use XeLaTeX if true
				\dystrue
				\dysfalse


\ifdys

\documentclass[a4paper, 14pt]{extarticle}
\usepackage{amsmath,amsfonts,amsthm,amssymb,mathtools}

\tracinglostchars=3 % Report an error if a font does not have a symbol.
\usepackage{fontspec}
\usepackage{unicode-math}
\defaultfontfeatures{ Ligatures=TeX,
                      Scale=MatchUppercase }

\setmainfont{OpenDyslexic}[Scale=1.0]
\setmathfont{Fira Math} % Or maybe try KPMath-Sans?
\setmathfont{OpenDyslexic Italic}[range=it/{Latin,latin}]
\setmathfont{OpenDyslexic}[range=up/{Latin,latin,num}]

\else

\documentclass[a4paper, 12pt]{extarticle}

\usepackage[utf8x]{inputenc}
%fonts
\usepackage{amsmath,amsfonts,amsthm,amssymb,mathtools}
% comment below to default to computer modern
\usepackage{libertinus,libertinust1math}

\fi


\usepackage[french]{babel}
\usepackage[
a4paper,
margin=2cm,
nomarginpar,% We don't want any margin paragraphs
]{geometry}
\usepackage{icomma}

\usepackage{fancyhdr}
\usepackage{array}
\usepackage{hyperref}

\usepackage{multicol, enumerate}
\newcolumntype{P}[1]{>{\centering\arraybackslash}p{#1}}


\usepackage{stackengine}
\newcommand\xrowht[2][0]{\addstackgap[.5\dimexpr#2\relax]{\vphantom{#1}}}

% theorems

\theoremstyle{plain}
\newtheorem{theorem}{Th\'eor\`eme}
\newtheorem*{sol}{Solution}
\theoremstyle{definition}
\newtheorem{ex}{Exercice}
\newtheorem*{rpl}{Rappel}
\newtheorem{enigme}{Énigme}

% corps
\usepackage{calrsfs}
\newcommand{\C}{\mathcal{C}}
\newcommand{\R}{\mathbb{R}}
\newcommand{\Rnn}{\mathbb{R}^{2n}}
\newcommand{\Z}{\mathbb{Z}}
\newcommand{\N}{\mathbb{N}}
\newcommand{\Q}{\mathbb{Q}}

% variance
\newcommand{\Var}[1]{\text{Var}(#1)}

% domain
\newcommand{\D}{\mathcal{D}}


% date
\usepackage{advdate}
\AdvanceDate[0]


% plots
\usepackage{pgfplots}

% table line break
\usepackage{makecell}
%tablestuff
\def\arraystretch{2}
\setlength\tabcolsep{15pt}

%subfigures
\usepackage{subcaption}

\definecolor{myg}{RGB}{56, 140, 70}
\definecolor{myb}{RGB}{45, 111, 177}
\definecolor{myr}{RGB}{199, 68, 64}

% fake sections with no title to move around the merged pdf
\newcommand{\fakesection}[1]{%
  \par\refstepcounter{section}% Increase section counter
  \sectionmark{#1}% Add section mark (header)
  \addcontentsline{toc}{section}{\protect\numberline{\thesection}#1}% Add section to ToC
  % Add more content here, if needed.
}


% SOLUTION SWITCH
\newif\ifsolutions
				\solutionstrue
				%\solutionsfalse

\ifsolutions
	\newcommand{\exe}[2]{
		\begin{ex} #1  \end{ex}
		\begin{sol} #2 \end{sol}
	}
\else
	\newcommand{\exe}[2]{
		\begin{ex} #1  \end{ex}
	}
	
\fi


% tableaux var, signe
\usepackage{tkz-tab}


%pinfty minfty
\newcommand{\pinfty}{{+}\infty}
\newcommand{\minfty}{{-}\infty}

\begin{document}
\input{adr/vars_24545.adr}

\pagestyle{fancy}
\fancyhead[L]{Seconde 13}
\fancyhead[C]{\textbf{Devoir Maison 1 -- \seed \ifsolutions \, -- Solutions  \fi}}
\fancyhead[R]{\today}


\exe{
	On considère la série statistique $X$ suivante, dépendente de deux entiers naturels $a,b\in\N$.
	
		\begin{center}
		\begin{tabular}{|c|c|c|c|}\hline
		Valeur   & \kmp & \kpq & \nval \\ \hline
		Effectif & $a$ & $b$ & \n \\ \hline
		\end{tabular}
		\end{center}
		
	\begin{enumerate}
		\item
		Monter que, pour le couple $(a;b) = (\sola;\solb)$, la moyenne de la série est $\overline{X} = \k$.
		\item
		Montrer que la condition $\overline{X} = \k$ est équivalente à la relation
			\[ - \psd \cdot a + \qsd \cdot b = 1. \]
		\item
		En déduire que $\psd$ divise $\qsd \cdot b - 1$ et trouver le plus petit entier naturel $b_0\in\N$ vérifiant
			\[ \psd | \left( \qsd\cdot b_0 - 1 \right). \]
		\item
		Posons 
			\[ a_0 = \dfrac{\qsd \cdot b_0 - 1}\psd. \]
		Montrer que pour le couple d'entiers $(a;b) = (a_0 ; b_0)$, la série $X$ est de moyenne $\k$.
		\item
		Montrer que pour tout $n\in\N$, la série $X$ associée au couple
			\[ (a;b) = (a_0 ; b_0) + n \cdot (\qsd ; \psd) \]
		est de moyenne $\k$.
		\item
		Représenter ces points $(a;b)$ dans un repère pour $n \in \{0 ; 1 ; 2; 3\}$. 
		Que dire sur les points ?
	\end{enumerate}
}
{
	\begin{enumerate}
		\item
		On remplace $a$ et $b$ par $\sola$ et $\solb$ respectivement et on trouve bien une moyenne égale à
			\[ \dfrac{\kmp\cdot\sola + \kpq\cdot\solb+\nval\cdot\n}{\sola+\solb+\n} = \k. \]
		
		\item 
		On écrit la suite d'équations équivalentes à $\overline{X} = \k$ suivante.
			\begin{align*}
				\dfrac{\kmp\cdot a + \kpq\cdot b+\nval\cdot\n}{a + b+\n} &= \k \\
				\kmp\cdot a + \kpq\cdot b+\nval\cdot\n &= \k \cdot (a+b+\n) \\
				 - \psd \cdot a + \qsd \cdot b &= 1.
			\end{align*}
		
		\item
		On réarrange les termes pour trouver que 
			\[ \qsd \cdot b - 1 = \psd \cdot a, \]
		c'est-à-dire que $\qsd\cdot b - 1$ est un multiple de $\psd$, car $a$ est un entier naturel.
		
		Comme $b$ est lui-même un entier naturel (c'est un effectif !), on peut essayer des valeurs $b=0, 1, 2, \dots$ jusqu'a ce que $\qsd b -1$ soit divisble par $\psd$.
		La valeur
			\[ b_0 = \bzero \]
		est la première qui fonctionne.
		
		\item
		On a alors
			\[ a_0 = \dfrac{9 \cdot \bzero - 1}{\psd} = \azero. \]
		Presque par construction on trouve bien que 
			\[ -\psd \cdot \azero  + \qsd \cdot \bzero = 1, \]
		condition équivalente à ce que $\overline{X} = \k$ d'après la deuxième question (on peut aussi revérifier en remplaçant dans le tableau comme à la question 1).
		
		\item
		Chaque couple de la forme
			\begin{align*}
				(a;b) &= (\azero; \bzero) + n \cdot (\qsd ; \psd) \\
					&= (\azero+ \qsd n ; \bzero + \psd n )
			\end{align*}
		vérifie bien
			\begin{align*}
				 - \psd \cdot a + \qsd \cdot b &= - \psd (\azero + \qsd n ) + \qsd (\bzero + \psd n ) \\
				 							&= -\psd \cdot \azero +  \qsd \cdot \bzero + n (- \psd \cdot \qsd + \qsd \cdot \psd) \\
				 							&= 1 + n \cdot 0 \\
				 							&= 1.
			\end{align*}
		En fait, le couple $(\qsd; \psd)$ est une solution du problème dit \emph{homogène} car
			\[ -\psd \cdot \qsd + \qsd \cdot \psd = 0, \]
		et la somme d'une solution du problème initial avec une solution du problème homogène donne une autre solution du problème initial.
		
		\item
		On voit les couples comme les coordonnées de points et on place les quatre premiers ainsi.
			\begin{center}
			\begin{tikzpicture}[>=stealth, scale=1]
			\begin{axis}[xmin = 0, xmax=\athree+5, ymin=0, ymax=\bthree+5, axis x line=middle, axis y line=middle, axis line style=->, grid=both]
				\addplot[black, mark=*, mark size = 1, thick] (\azero,\bzero) node[below] {$(\azero;\bzero)$};
				\addplot[black, mark=*, mark size = 1, thick] (\sola,\solb) node[below] {$(\sola;\solb)$};
				\addplot[black, mark=*, mark size = 1, thick] (\atwo,\btwo) node[below] {$(\atwo;\btwo)$};
				\addplot[black, mark=*, mark size = 1, thick] (\athree,\bthree) node[below] {$(\athree;\bthree)$};
			\end{axis}
			\end{tikzpicture}
			\end{center}
		On remarque que tous les points sont alignés. 
		Les points trouvés sont en fait tous les points à coordonnées entières d'une certaine droite.
		On étudiera les équations de droites dans un chapitre dédié plus tard. En attendant, on pourra écrire \texttt{$ -\psd x + \qsd y = 1$} dans Geogebra pour voir la droite.
	\end{enumerate}
}

\exe{
	On considère la série statistique $X$ suivante, dépendente de deux entiers naturels $a,b\in\N$.
	
		\begin{center}
		\begin{tabular}{|c|c|c|c|}\hline
		Valeur   & \kmpB & \kpqB & \nvalB \\ \hline
		Effectif & $a$ & $b$ & \nB \\ \hline
		\end{tabular}
		\end{center}
		
	\begin{enumerate}
		\item
		Montrer que la condition $\overline{X} = \kB$ est équivalente à la relation
			\[ - \pB \cdot a + \qB \cdot b = \dmnB. \]
		\item
		Trouver le plus grand diviseur commun à $\pB$ et $\qB$ et montrer qu'il divise nécessairement 
			\[ - \pB \cdot a + \qB \cdot b.\]
		\item
		Conclure par contradiction qu'il n'existe pas d'entiers naturels $a,b\in\N$ tels que la série $X$ ait une moyenne égale à $\kB$.
		\item 
		Trouver un moyenne possible et deux couples $(a,b)$ distincts qui la réalisent.
		Y a-t-il une infinité de tels couples ?
	\end{enumerate}
}
{
	\begin{enumerate}
		\item
			Ceci se montre de façon équivalente à la première question de l'exercice 1.
		\item
			Le plus grand diviseur commun à $\pB$ et $\qB$ est donné par $\divisor$ :
				\begin{align*}
					\pB &= \divisor \cdot \psdB, \\
					\qB &= \divisor \cdot \qsdB.
				\end{align*}
			Ainsi, $\divisor$ divise la combinaison entière $ - \pB \cdot a + \qB \cdot b$ pour n'importe quels entiers $a$ et $b$ car on a bien
				\[ - \pB \cdot a + \qB \cdot b = \divisor \cdot ( - \psdB \cdot a + \qsdB \cdot b ), \]
			multiple de $\divisor$.
		\item
		Supposons qu'il existe un tel couple $(a;b)$ d'entiers naturels tels que la moyenne de la série soit $\kB$.
		Dans ce cas, la relation 
			\[ - \pB \cdot a + \qB \cdot b = \dmnB. \]
		doit être vérifiée.
		Or on a montré que le membre de gauche est une multiple de $\divisor$.
		Cependant, comme le membre de droite, $\dmnB$, n'est évidemment pas un multiple de $\divisor$, on arrive à une contradiction.
		\item 
		On peut choisir un couple arbitrairement et calculer la moyenne correspondante.
		Choisir $(a;b) = (0;0)$ par exemple donne une moyenne de $\overline{X} = \nvalB$.
		
		Au regard de l'exercice 1, il existe nécessairement une infinité de couples vérifiant $\overline{X} = \nvalB$, car on peut en créer en ajoutant autant de fois qu'on le souhaite la solution $(\qB ; \pB)$ du probème homogène à $(0;0)$.
	\end{enumerate}

}

\ifsolutions
\else
\newpage 
\fi

\exe{
	Écrire un tableau Valeur/Effectif pour chaque histogramme de la figure \ref{fig:hist}.
	On assignera la valeur moyenne à chaque élément d'une classe.
	Par exemple, de l'histogramme \ref{fig:a} on lit $\EFFCONCa$ notes de valeur $\CONCa,5$.
	
	Ensuite, pour chaque série obtenue, calculer
		\begin{multicols}{2}
		\begin{enumerate}[i)]
			\item La moyenne ;
			\item L'écart type ;
			\item La médiane ;
			\item Le premier quartile ;
			\item Le troisième quartile ; et
			\item L'écart interquartile.
		\end{enumerate}
		\end{multicols}
		
	Comparer les séries statistiques en s'appuyant sur les valeurs calculées.
}{
	Pour l'histogramme \ref{fig:a}, on lit le tableau Valeur/Effectif suivant.
			\begin{center}
			\begin{tabular}{|c|c|c|c|c|}\hline
			Note  & \vala & \valb & \valc & \vald \\ \hline
			Effectif & \EFFCONCa & \EFFCONCb & \EFFCONCc & \EFFCONCd \\ \hline
			\end{tabular}
			\end{center}
	
	On en déduit les valeurs suivantes, arrondies au centième.
	\begin{enumerate}[i)]
		\item La moyenne est de $\overline{X} = \Concaverage$ ;
		\item L'écart type est $\sigma(X) = \Concstd$ ;
		\item La médiane est $\Concmedian$ ;
		\item Le premier quartile est $Q_1 = \Concfirstquartile$ ;
		\item Le troisième quartile est $Q_3 = \Concthirdquartile$ ; et
		\item L'écart interquartile est $Q_3 - Q_1 = \Concecart$.
	\end{enumerate}

	Pour l'histogramme \ref{fig:b}, on lit le tableau Valeur/Effectif suivant.
			\begin{center}
			\begin{tabular}{|c|c|c|c|c|c|}\hline
			Note  & \valA & \valB & \valC & \valD & \valE \\ \hline
			Effectif & \EFFDISPa & \EFFDISPb & \EFFDISPc & \EFFDISPd  & \EFFDISPe \\ \hline
			\end{tabular}
			\end{center}
	
	On en déduit les valeurs suivantes, arrondies au centième.
	\begin{enumerate}[i)]
		\item La moyenne est de $\overline{X} = \Dispaverage$ ;
		\item L'écart type est $\sigma(X) = \Dispstd$ ;
		\item La médiane est $\Dispmedian$ ;
		\item Le premier quartile est $Q_1 = \Dispfirstquartile$ ;
		\item Le troisième quartile est $Q_3 = \Dispthirdquartile$ ; et
		\item L'écart interquartile est $Q_3 - Q_1 = \Dispecart$.
	\end{enumerate}


	En conclusion, la première série statistique est bien plus concentrée autour de sa moyenne que la deuxième.
	On le voit qualitativement en comparant les écarts types ou en comparant les écarts interquartiles.
	L'écart $|Q_1 - Q_3|$ donne l'étendue que prend la moitié des valeurs centrales. 
	Plus cet écart est grand, plus il existe des valeurs centrales espacées les unes des autres.
	
	Ce résultat n'est pas surprenant car on distingue un groupe uni de valeurs sur l'histogramme \ref{fig:a}, et deux groupes disjoints sur l'histogramme \ref{fig:b}.
	Dans le contexte des notes, un écart type élevé indique une haute hétérogénéité de résultats dans une classe. 
	L'histrogramme \ref{fig:b} permet dans ce cas de créer des groupes de niveau.
}

\ifsolutions
\newpage
\fi

\begin{figure}[t]
  \centering
  \begin{subfigure}[b]{.45\textwidth}
    \centering
  \begin{tikzpicture}[scale=1]
    \begin{axis}[
    	ymin=0,
        %ymin=0, ymax=8,
        %minor y tick num = 3,
        xmin = 0, xmax=20,
        xtick = {2, 4, ..., 18},
        area style,
        xlabel = {Note sur $20$},
        ylabel = {Effectif}
      ]
      \addplot+[ybar interval,mark=no, draw=myg, fill=myg!50] plot coordinates {
        (\CONCa, \EFFCONCa) (\CONCb, \EFFCONCb) (\CONCc, \EFFCONCc) (\CONCd, \EFFCONCd) (\CONCd+1, 0)
      };
    \end{axis} 
  \end{tikzpicture}
  \caption{Première série.}
  \label{fig:a}
  \end{subfigure}
  \hfill
  % NOT LINE BREAK!!
  \begin{subfigure}[b]{.45\textwidth}
    \centering
  \begin{tikzpicture}[scale=1]
    \begin{axis}[
    	ymin=0,
        %ymin=0, ymax=8,
        %minor y tick num = 3,
        xmin = 0, xmax=20,
        xtick = {2, 4, ..., 18},
        area style,
        xlabel = {Note sur $20$},
        ylabel = {Effectif}
      ]
      \addplot+[ybar interval,mark=no, draw=myb, fill=myb!50] plot coordinates {
        (\DISPa, \EFFDISPa) (\DISPb, \EFFDISPb) (\DISPc, \EFFDISPc) (\DISPc+1, 0) (\DISPd, \EFFDISPd) (\DISPe, \EFFDISPe) (\DISPe+1, 0)
      };
    \end{axis} 
  \end{tikzpicture}
  \caption{Deuxième série.}
  \label{fig:b}
  \end{subfigure}
  
  
  \caption{Histogrammes de notes (min $0$, max $20$). Les classes sont de la forme $[k;k+1[$ où $k\in\N$ est un entier naturel. 
  La colonne entre $\CONCa$ et $\CONCb$ compte donc toutes les notes appartenant à $[\CONCa;\CONCb[$.}
  \label{fig:hist}
\end{figure}


\end{document}

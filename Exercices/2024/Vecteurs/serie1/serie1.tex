				% ENABLE or DISABLE font change
				% use XeLaTeX if true
\newif\ifdys
				\dystrue
				\dysfalse

\newif\ifsolutions
				\solutionstrue
				\solutionsfalse

% DYSLEXIA SWITCH
\newif\ifdys
		
				% ENABLE or DISABLE font change
				% use XeLaTeX if true
				\dystrue
				\dysfalse


\ifdys

\documentclass[a4paper, 14pt]{extarticle}
\usepackage{amsmath,amsfonts,amsthm,amssymb,mathtools}

\tracinglostchars=3 % Report an error if a font does not have a symbol.
\usepackage{fontspec}
\usepackage{unicode-math}
\defaultfontfeatures{ Ligatures=TeX,
                      Scale=MatchUppercase }

\setmainfont{OpenDyslexic}[Scale=1.0]
\setmathfont{Fira Math} % Or maybe try KPMath-Sans?
\setmathfont{OpenDyslexic Italic}[range=it/{Latin,latin}]
\setmathfont{OpenDyslexic}[range=up/{Latin,latin,num}]

\else

\documentclass[a4paper, 12pt]{extarticle}

\usepackage[utf8x]{inputenc}
%fonts
\usepackage{amsmath,amsfonts,amsthm,amssymb,mathtools}
% comment below to default to computer modern
\usepackage{libertinus,libertinust1math}

\fi


\usepackage[french]{babel}
\usepackage[
a4paper,
margin=2cm,
nomarginpar,% We don't want any margin paragraphs
]{geometry}
\usepackage{icomma}

\usepackage{fancyhdr}
\usepackage{array}
\usepackage{hyperref}

\usepackage{multicol, enumerate}
\newcolumntype{P}[1]{>{\centering\arraybackslash}p{#1}}


\usepackage{stackengine}
\newcommand\xrowht[2][0]{\addstackgap[.5\dimexpr#2\relax]{\vphantom{#1}}}

% theorems

\theoremstyle{plain}
\newtheorem{theorem}{Th\'eor\`eme}
\newtheorem*{sol}{Solution}
\theoremstyle{definition}
\newtheorem{ex}{Exercice}
\newtheorem*{rpl}{Rappel}
\newtheorem{enigme}{Énigme}

% corps
\usepackage{calrsfs}
\newcommand{\C}{\mathcal{C}}
\newcommand{\R}{\mathbb{R}}
\newcommand{\Rnn}{\mathbb{R}^{2n}}
\newcommand{\Z}{\mathbb{Z}}
\newcommand{\N}{\mathbb{N}}
\newcommand{\Q}{\mathbb{Q}}

% variance
\newcommand{\Var}[1]{\text{Var}(#1)}

% domain
\newcommand{\D}{\mathcal{D}}


% date
\usepackage{advdate}
\AdvanceDate[0]


% plots
\usepackage{pgfplots}

% table line break
\usepackage{makecell}
%tablestuff
\def\arraystretch{2}
\setlength\tabcolsep{15pt}

%subfigures
\usepackage{subcaption}

\definecolor{myg}{RGB}{56, 140, 70}
\definecolor{myb}{RGB}{45, 111, 177}
\definecolor{myr}{RGB}{199, 68, 64}

% fake sections with no title to move around the merged pdf
\newcommand{\fakesection}[1]{%
  \par\refstepcounter{section}% Increase section counter
  \sectionmark{#1}% Add section mark (header)
  \addcontentsline{toc}{section}{\protect\numberline{\thesection}#1}% Add section to ToC
  % Add more content here, if needed.
}


% SOLUTION SWITCH
\newif\ifsolutions
				\solutionstrue
				%\solutionsfalse

\ifsolutions
	\newcommand{\exe}[2]{
		\begin{ex} #1  \end{ex}
		\begin{sol} #2 \end{sol}
	}
\else
	\newcommand{\exe}[2]{
		\begin{ex} #1  \end{ex}
	}
	
\fi


% tableaux var, signe
\usepackage{tkz-tab}


%pinfty minfty
\newcommand{\pinfty}{{+}\infty}
\newcommand{\minfty}{{-}\infty}

\begin{document}


\AdvanceDate[0]

\begin{document}
\pagestyle{fancy}
\fancyhead[L]{Seconde 13}
\fancyhead[C]{\textbf{Vecteurs 3 \ifsolutions -- Solutions  \fi}}
\fancyhead[R]{\today}

\exe{[Calcul]
	Soient $A(3;-1), B(-1; -5),$ et $C(3 ; 4)$.
	\begin{multicols}{2}
	\begin{enumerate}
		\item Calculer $\vec{AB}, \vec{BA}$, et $\vec{CA}$.
		\item Calculer $\norm{\vec{AB}}, \norm{\vec{BA}}$, $\norm{\vec{AC}}$.
		\item Calculer $3 \vec{AB} + 3 \vec{CA}$ et $3 \vec{CB}$.
		\item Calculer $\norm{-4 \vec{AB}}$ et $\norm{-\dfrac1{13} \vec{CA}}$.
	\end{enumerate}
	\end{multicols}

}{}

\exe{[Représentation graphique]
	Construire géométriquement les sommes 
		\[u+v+w \qquad \text{ et } \qquad \dfrac12u - 2v - w,\]
	où les vecteurs $u, v, w$ sont donnés dans le plan ci-dessous.
	
	\begin{center}
		\begin{tikzpicture}[>=stealth, scale=1]
		\begin{axis}[xmin = -10, xmax=10, ymin=-10, ymax=10, axis x line=none, axis y line=none, axis line style=<->, xlabel={}, ylabel={}, ticks = none]
			\draw[very thick, ->, myg] (axis cs:-3,-4) -- (axis cs: 0,0) node[above] {$u$};
			\draw[very thick, ->, myr] (axis cs:-2,0) -- (axis cs: -7,6) node[above] {$v$};
			\draw[very thick, ->, myb] (axis cs:0,3) -- (axis cs: 8,-2) node[above] {$w$};
		\end{axis}
		\end{tikzpicture}
	\end{center}
}{}


\exe{[Vrai ou faux]
	Pour chacune des propositions suivantes, montrer qu'elle est toujours vraie ou trouver un contre-exemple.
	\begin{enumerate}
		\item $\vec{AB} = \vec{CD} \iff \vec{AC} = \vec{BD}$.
		\item Soient $u, v$ deux vecteurs tels que $v = 5u$.
		Alors $\det(u,v) = 0$.
		\item Si $\vec{AB}$ et $\vec{CD}$ sont colinéaires, alors les points $A, B, C$, et $D$ sont alignés.
		\item Soient $u, v$ deux vecteurs.
		Alors $\det(u,v) = -\det(v, u)$.
		\item Si $\norm{v} = \sqrt{5}$, alors $u = \pvec{2}{1}$.
	\end{enumerate}
}{}

\exe{[Parallélisme]

	Soit $u = \pvec{4}{2}, v = \pvec{1}{3}$ deux vecteurs et $A(-3; -1)$ un point du plan.

	Représenter dans un repère les points
		\begin{align*}
			A, && B = A+u, && C = A+u+v, && \text{ et } && D=A+v,
		\end{align*}
	et répondre aux questions suivantes.
	\begin{enumerate}
		\item Que dire du quadrilatère $ABCD$ visuellement ?
		\item Montrer que $(AB)$ et $(CD)$ sont parallèles puis que $(AD)$ et $(BC)$ sont parallèles.
		\item Démontrer de façon générale cette propriété pour $A, u, v$ un point et deux vecteurs quelconques.
	\end{enumerate}
}{}

\newpage

\exe{[Vecteur directeur]
	On considère un point $A(5 ; -12)$ et un vecteur $v = \pvec{2}{-1}$.
	Soit $(d)$ la droite passant par $A$ et dirigée par $v$.
	
	\begin{enumerate}
		\item
		Donner $4$ points distincts appartenant à $(d)$.
		\item
		Trouver la fonction affine $f$ telle que $(d) = \C_f$.
	\end{enumerate}
}{}

\exe{[Quadrilatère]
	Considérons quatre points $A(-4; 1), B(-3 ; -3), C(4; -4), D(3; 1)$.
	
	Le quadralitère $ABCD$ est-il un parallélogramme ?
	Si non, donner un point $\tilde{D}$ tel que $ABC\tilde{D}$ en soit un.
}{}

\exe{[Milieu]
	Soient $A, B, C$ trois points tels que $B$ soit le milieu du segment $[AC]$.
	Faire un dessin puis montrer que $\vec{AB} = \vec{BC}$.
}{}

\exe{[Parallélisme]
	Soient les points $A(3;2), B(-3 ; 7), C(-2; -3), D(3;1)$.
	
	Les droites $(AB)$ et $(CD)$ sont-elles parallèles ?
	
	Si non, donner un point $\tilde{B}$ tel que les droites $(A\tilde{B})$ et $(CD)$ soient parallèles.
}{}


%	\begin{center}
%		\begin{tikzpicture}[>=stealth, scale=1]
%		\begin{axis}[xmin = -6.5, xmax=6.5, ymin=-6.5, ymax=6.5, axis x line=middle, axis y line=middle, axis line style=<->, xlabel={}, ylabel={}, xtick = {-6, -5, ..., 5, 6}, ytick = {-6, -5, ..., 5, 6}, grid=both]
%			
%		\end{axis}
%		\end{tikzpicture}
%		\begin{tikzpicture}[>=stealth, scale=1]
%		\begin{axis}[xmin = -6.5, xmax=6.5, ymin=-6.5, ymax=6.5, axis x line=middle, axis y line=middle, axis line style=<->, xlabel={}, ylabel={}, xtick = {-6, -5, ..., 5, 6}, ytick = {-6, -5, ..., 5, 6}, grid=both]
%			
%		\end{axis}
%		\end{tikzpicture}
%	\end{center}


\end{document}

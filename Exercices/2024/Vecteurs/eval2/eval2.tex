				% ENABLE or DISABLE font change
				% use XeLaTeX if true
\newif\ifdys
				\dystrue
				\dysfalse

\newif\ifsolutions
				\solutionstrue
				\solutionsfalse

% DYSLEXIA SWITCH
\newif\ifdys
		
				% ENABLE or DISABLE font change
				% use XeLaTeX if true
				\dystrue
				\dysfalse


\ifdys

\documentclass[a4paper, 14pt]{extarticle}
\usepackage{amsmath,amsfonts,amsthm,amssymb,mathtools}

\tracinglostchars=3 % Report an error if a font does not have a symbol.
\usepackage{fontspec}
\usepackage{unicode-math}
\defaultfontfeatures{ Ligatures=TeX,
                      Scale=MatchUppercase }

\setmainfont{OpenDyslexic}[Scale=1.0]
\setmathfont{Fira Math} % Or maybe try KPMath-Sans?
\setmathfont{OpenDyslexic Italic}[range=it/{Latin,latin}]
\setmathfont{OpenDyslexic}[range=up/{Latin,latin,num}]

\else

\documentclass[a4paper, 12pt]{extarticle}

\usepackage[utf8x]{inputenc}
%fonts
\usepackage{amsmath,amsfonts,amsthm,amssymb,mathtools}
% comment below to default to computer modern
\usepackage{libertinus,libertinust1math}

\fi


\usepackage[french]{babel}
\usepackage[
a4paper,
margin=2cm,
nomarginpar,% We don't want any margin paragraphs
]{geometry}
\usepackage{icomma}

\usepackage{fancyhdr}
\usepackage{array}
\usepackage{hyperref}

\usepackage{multicol, enumerate}
\newcolumntype{P}[1]{>{\centering\arraybackslash}p{#1}}


\usepackage{stackengine}
\newcommand\xrowht[2][0]{\addstackgap[.5\dimexpr#2\relax]{\vphantom{#1}}}

% theorems

\theoremstyle{plain}
\newtheorem{theorem}{Th\'eor\`eme}
\newtheorem*{sol}{Solution}
\theoremstyle{definition}
\newtheorem{ex}{Exercice}
\newtheorem*{rpl}{Rappel}
\newtheorem{enigme}{Énigme}

% corps
\usepackage{calrsfs}
\newcommand{\C}{\mathcal{C}}
\newcommand{\R}{\mathbb{R}}
\newcommand{\Rnn}{\mathbb{R}^{2n}}
\newcommand{\Z}{\mathbb{Z}}
\newcommand{\N}{\mathbb{N}}
\newcommand{\Q}{\mathbb{Q}}

% variance
\newcommand{\Var}[1]{\text{Var}(#1)}

% domain
\newcommand{\D}{\mathcal{D}}


% date
\usepackage{advdate}
\AdvanceDate[0]


% plots
\usepackage{pgfplots}

% table line break
\usepackage{makecell}
%tablestuff
\def\arraystretch{2}
\setlength\tabcolsep{15pt}

%subfigures
\usepackage{subcaption}

\definecolor{myg}{RGB}{56, 140, 70}
\definecolor{myb}{RGB}{45, 111, 177}
\definecolor{myr}{RGB}{199, 68, 64}

% fake sections with no title to move around the merged pdf
\newcommand{\fakesection}[1]{%
  \par\refstepcounter{section}% Increase section counter
  \sectionmark{#1}% Add section mark (header)
  \addcontentsline{toc}{section}{\protect\numberline{\thesection}#1}% Add section to ToC
  % Add more content here, if needed.
}


% SOLUTION SWITCH
\newif\ifsolutions
				\solutionstrue
				%\solutionsfalse

\ifsolutions
	\newcommand{\exe}[2]{
		\begin{ex} #1  \end{ex}
		\begin{sol} #2 \end{sol}
	}
\else
	\newcommand{\exe}[2]{
		\begin{ex} #1  \end{ex}
	}
	
\fi


% tableaux var, signe
\usepackage{tkz-tab}


%pinfty minfty
\newcommand{\pinfty}{{+}\infty}
\newcommand{\minfty}{{-}\infty}

\begin{document}


\AdvanceDate[3]

\begin{document}
\pagestyle{fancy}
\fancyhead[L]{Seconde 13}
\fancyhead[C]{\textbf{Évaluation -- Vecteurs \ifsolutions -- Solutions  \fi}}
\fancyhead[R]{\today}

\exe{[5pts]\label{ex:1}
	On considère le point $A(-4; -2)$ et les deux vecteurs $u = \pvec{6}{2}$ et $v = \pvec{1}{-3}$.
	On pose de plus les trois points
		\begin{align*}
			B = A + u, && C = B + v, && D = C - u.
		\end{align*}
	Les questions $1$, $2$, et $3$ peuvent être faites séparément en admettant les résultats des questions précédentes.
	
	\begin{enumerate}
		\item 
		Montrer que $B(2;0), C(3;-3)$, et $D(-3;-5)$ puis dessiner le quadrilatère $ABCD$ dans un repère.
		
		\item
		\begin{enumerate}[label=(\alph*)]
			\item Calculer les vecteurs $\vec{AB}$ et $\vec{DC}$, et montrer qu'ils sont colinéaires.
			\item Calculer les vecteurs $\vec{AD}$ et $\vec{CB}$, et montrer qu'ils sont colinéaires.
			\item Montrer qu'on a $\norm{\vec{AB}}= 2\sqrt{10}$, $\norm{\vec{AD}} = \sqrt{10}$, et $\norm{\vec{BD}} = 5\sqrt{2}$.
			\item Vérifier qu'on ait bien $\det\left(\vec{AB}, \vec{AD}\right) = -20$.
		\end{enumerate}
		
		\item
		\begin{enumerate}[label=(\alph*)]
			\item Montrer que le quadrilatère $ABCD$ est un parallélogramme à l'aide des questions 2(a-b).
			\item Montrer que le triangle $ABD$ est rectangle en $A$ à l'aide de la réciproque du théorème de Pythagore et de la question 2(c).
			\item En déduire que $ABCD$ est un rectangle et calculer son aire à l'aide de la question 2(c).
			\item Comparer l'aire obtenue à $\left| \det\left(\vec{AB}, \vec{AD}\right) \right|$, la valeur absolue du déterminant calculé à la question 2(d).
		\end{enumerate}
		
	\end{enumerate}


}{}

%\exe{[Représentation graphique, 3pts]
\exe{[2pts]
	Construire en bas de page les vecteurs 
		\begin{align*}
			a = u+v+w, && b = -2u, && \text{ et } && c = 3u - \dfrac12v - \dfrac13w,
		\end{align*}
	où les vecteurs $u, v, w$ sont donnés dans le plan ci-dessous.
	
	\begin{center}
		\begin{tikzpicture}[>=stealth, scale=1]
		\begin{axis}[xmin = -10, xmax=10, ymin=-10, ymax=10, axis x line=none, axis y line=none, axis line style=<->, xlabel={}, ylabel={}, ticks = none]
			\draw[very thick, ->, myg] (axis cs:-1,-1) -- (axis cs: 1,-4) node[right] {$u$};
			\draw[very thick, ->, myr] (axis cs: -2,8) -- (axis cs: -5,-1) node[left] {$v$};
			\draw[very thick, ->, myb] (axis cs:0,1) -- (axis cs: 8,5) node[above] {$w$};
		\end{axis}
		\end{tikzpicture}
		%\vline
	\end{center}
	\vspace{-1cm}
	\underline{\textbf{Constructions :}}
}{}

\newpage

\exe{[3pts]
	À l'aide de la figure suivante et de la relation de Chasles, déterminer les sommes vectorielles suivantes en complétant les pointillés.

		\begin{center}
		\begin{tikzpicture}[>=stealth, scale=1]
		\newcommand\len{.8cm}
		\begin{axis}[xmin = 0, x=\len, y= \len, xmax=8, ymin=0, ymax=4, axis line style={ draw opacity=0 }, xlabel={}, ylabel={}, ticks = none, grid = both, enlargelimits={abs=0.8}, minor x tick num = 1, minor y tick num = 1]
			\draw[very thick, -, myr] (axis cs:0,0) -- (axis cs: 8,0);
			\draw[very thick, -, myr] (axis cs:0,4) -- (axis cs: 8,4);
			\draw[very thick, -, myr] (axis cs:0,0) -- (axis cs: 0,4);
			\draw[very thick, -, myr] (axis cs:8,0) -- (axis cs: 8,4);
			\draw[very thick, -, myr] (axis cs:4,0) -- (axis cs: 4,4);
			\draw[very thick, -, myr] (axis cs:0,0) -- (axis cs: 4,4);
			\draw[very thick, -, myr] (axis cs:0,4) -- (axis cs: 4,0);
			\draw[very thick, -, myr] (axis cs:4,0) -- (axis cs: 8,4);
			\draw[very thick, -, myr] (axis cs:4,4) -- (axis cs: 8,0);
			
			\addplot[black] (0,0) node[below] {$A$};
			\addplot[black] (4,0) node[below] {$B$};
			\addplot[black] (8,0) node[below] {$C$};
			\addplot[black] (8,4) node[above] {$D$};
			\addplot[black] (4,4) node[above] {$E$};
			\addplot[black] (0,4) node[above] {$F$};
			\addplot[black] (2,2) node[below=2pt] {$G$};
			\addplot[black] (6,2) node[below=2pt] {$H$};
			
		\end{axis}
		\end{tikzpicture}
		%\vline
	\end{center}
	
	\begin{multicols}{2}
	\begin{enumerate}
		\item $\vec{BC} = \vec{E \dots}$
		\item $\vec{BE} = \vec{\dots F}$
		\item $\vec{AB} + \vec{FA} = \vec{F \dots} + \vec{\dots B} = \vec{\vphantom{A} \dots\dots}$
		\item $\vec{FD} + \vec{EB} = \vec{FD} + \vec{D\dots} = \vec{\vphantom{A}\dots\dots}$
		\item $\vec{AE} + \vec{GB} + \vec{DH} = \vec{AE} + \vec{E\dots} + \vec{\vphantom{A}\dots\dots} = \vec{\vphantom{A}\dots\dots}$
	\end{enumerate}
	\end{multicols}

}{}


%\exe{[Vrai ou faux, 4pts]
\exe{[4pts]
	Pour chacune des propositions suivantes, \underline{montrer} qu'elle est toujours vraie ou \underline{donner} un contre-exemple.
	
	Soient $u, v, w$ trois vecteurs.
	\begin{enumerate}
		\item Si $v = -u$, alors $\det(u, v) = 0$.
		\item Si $u$ est colinéaire à $v$, et $v$ est colinéaire à $w$, alors $u$ est colinéaire à $w$.
		\item  Si $\norm{u} = \norm{v}$, alors $u$ et $v$ sont colinéaires.
		\item Si $u$ et $v$ sont colinéaires, alors $\norm{u} = \norm{v}$.
	\end{enumerate}
}{}

%\exe{[Vecteur directeur, 3pts]
\exe{[3pts]
	On considère un point $A(-4;5)$ et un vecteur $v = \pvec{3}{-2}$.
	Soit $(d)$ la droite passant par $A$ et dirigée par $v$.
	
	\begin{enumerate}
		\item
		Donner $4$ points distincts appartenant à $(d)$.
		\item
		Trouver la fonction affine $f$ telle que $(d) = \C_f$.
	\end{enumerate}
}{}

%\exe{[Alignement, 3pts]
\exe{[3pts]
	Considérons les points
		\begin{align*}
			A(-1; -6), && B(-3; -12), && C(5; 12), && D(7 ; 17).
		\end{align*}
	
	\begin{enumerate}
		\item Le point $C$ appartient-il à la droite $(AB)$ ?
		\item Le point $D$ appartient-il à la droite $(CB)$ ?
		\item Trouver un point $E$ distinct de $A, B, C, D$ et appartenant à $(AB)$.
	\end{enumerate}
}{}

%\section*{Bonus (2pts)}
\section*{Bonus : généralisation de l'exercice \ref{ex:1}}

%\exe{[Généralisation de l'exercice \ref{ex:1}]
\exe{[4pts]
	Soient $u = \pvec{a}{b}, 	v = \pvec{c}{d}$	deux vecteurs tels que le triangle de sommets $O(0;0), O+u, O+v$ est rectangle en $O$. 
	La question 2 peut être faite en admettant le résultat de la question 1.
	Des points partiels peuvent être attribués même si la tentative est infructueuse.
	\begin{enumerate}
		\item Montrer à l'aide de la réciproque de Pythagore que 
			\[ a\cdot c + b\cdot d = 0. \]
		\item
		Montrer à l'aide de la relation ci-dessus que
			\[ \norm{u}^2 \cdot \norm{v}^2 = \left( \det(u, v) \right)^2. \]
	\end{enumerate}
}{}


\end{document}
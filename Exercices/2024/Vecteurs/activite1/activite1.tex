				% ENABLE or DISABLE font change
				% use XeLaTeX if true
\newif\ifdys
				\dystrue
				\dysfalse

\newif\ifsolutions
				\solutionstrue
				\solutionsfalse

% DYSLEXIA SWITCH
\newif\ifdys
		
				% ENABLE or DISABLE font change
				% use XeLaTeX if true
				\dystrue
				\dysfalse


\ifdys

\documentclass[a4paper, 14pt]{extarticle}
\usepackage{amsmath,amsfonts,amsthm,amssymb,mathtools}

\tracinglostchars=3 % Report an error if a font does not have a symbol.
\usepackage{fontspec}
\usepackage{unicode-math}
\defaultfontfeatures{ Ligatures=TeX,
                      Scale=MatchUppercase }

\setmainfont{OpenDyslexic}[Scale=1.0]
\setmathfont{Fira Math} % Or maybe try KPMath-Sans?
\setmathfont{OpenDyslexic Italic}[range=it/{Latin,latin}]
\setmathfont{OpenDyslexic}[range=up/{Latin,latin,num}]

\else

\documentclass[a4paper, 12pt]{extarticle}

\usepackage[utf8x]{inputenc}
%fonts
\usepackage{amsmath,amsfonts,amsthm,amssymb,mathtools}
% comment below to default to computer modern
\usepackage{libertinus,libertinust1math}

\fi


\usepackage[french]{babel}
\usepackage[
a4paper,
margin=2cm,
nomarginpar,% We don't want any margin paragraphs
]{geometry}
\usepackage{icomma}

\usepackage{fancyhdr}
\usepackage{array}
\usepackage{hyperref}

\usepackage{multicol, enumerate}
\newcolumntype{P}[1]{>{\centering\arraybackslash}p{#1}}


\usepackage{stackengine}
\newcommand\xrowht[2][0]{\addstackgap[.5\dimexpr#2\relax]{\vphantom{#1}}}

% theorems

\theoremstyle{plain}
\newtheorem{theorem}{Th\'eor\`eme}
\newtheorem*{sol}{Solution}
\theoremstyle{definition}
\newtheorem{ex}{Exercice}
\newtheorem*{rpl}{Rappel}
\newtheorem{enigme}{Énigme}

% corps
\usepackage{calrsfs}
\newcommand{\C}{\mathcal{C}}
\newcommand{\R}{\mathbb{R}}
\newcommand{\Rnn}{\mathbb{R}^{2n}}
\newcommand{\Z}{\mathbb{Z}}
\newcommand{\N}{\mathbb{N}}
\newcommand{\Q}{\mathbb{Q}}

% variance
\newcommand{\Var}[1]{\text{Var}(#1)}

% domain
\newcommand{\D}{\mathcal{D}}


% date
\usepackage{advdate}
\AdvanceDate[0]


% plots
\usepackage{pgfplots}

% table line break
\usepackage{makecell}
%tablestuff
\def\arraystretch{2}
\setlength\tabcolsep{15pt}

%subfigures
\usepackage{subcaption}

\definecolor{myg}{RGB}{56, 140, 70}
\definecolor{myb}{RGB}{45, 111, 177}
\definecolor{myr}{RGB}{199, 68, 64}

% fake sections with no title to move around the merged pdf
\newcommand{\fakesection}[1]{%
  \par\refstepcounter{section}% Increase section counter
  \sectionmark{#1}% Add section mark (header)
  \addcontentsline{toc}{section}{\protect\numberline{\thesection}#1}% Add section to ToC
  % Add more content here, if needed.
}


% SOLUTION SWITCH
\newif\ifsolutions
				\solutionstrue
				%\solutionsfalse

\ifsolutions
	\newcommand{\exe}[2]{
		\begin{ex} #1  \end{ex}
		\begin{sol} #2 \end{sol}
	}
\else
	\newcommand{\exe}[2]{
		\begin{ex} #1  \end{ex}
	}
	
\fi


% tableaux var, signe
\usepackage{tkz-tab}


%pinfty minfty
\newcommand{\pinfty}{{+}\infty}
\newcommand{\minfty}{{-}\infty}

\begin{document}


\AdvanceDate[0]

\begin{document}
\pagestyle{fancy}
\fancyhead[L]{Seconde 13}
\fancyhead[C]{\textbf{Vecteurs 1 \ifsolutions -- Solutions  \fi}}
\fancyhead[R]{\today}

\exe{

	Dessiner les points 
		\begin{align*}
			A = (-6 ;3), && B = (5 ; 6), && C = (0 ; -7),
		\end{align*}
	dans un repère et répondre au questions.
	
	\begin{enumerate}
		\item Quelle translation $\vec{AB}$ effectuer pour envoyer le point $A$ sur le point $B$ ?
		\item Quelle translation $\vec{BC}$ effectuer pour envoyer le point $B$ sur le point $C$ ?
		\item Quelle translation $\vec{AC}$ effectuer pour envoyer le point $A$ sur le point $C$ ?
		\item Comparer $\vec{AC}$ et $\vec{AB} + \vec{BC}$ et compléter la phrase suivante.
			\begin{center}
				\og Lorsque $A$ est envoyé sur $B$, puis $B$ est envoyé sur $C$, alors $A$ est envoyé sur $\dots$ \fg
			\end{center}
		%\item Comparer $\vec{AC}$ et $\vec{AB} + \vec{BC}$.
		
		\item Pour $A, B, C$ trois points quelconques, démontrer l'égalité
			\[ \vec{AC} = \vec{AB} + \vec{BC}. \]
	\end{enumerate}

}{
	
	\begin{enumerate}
		\item On calcule
			\[ \vec{AB} = B - A = \pvec{5 - (-6)}{6 - 3} = \pvec{11}{3}. \]
		\item On calcule
			\[ \vec{BC} = C - B = \pvec{0 - 5}{-7 - 6} = \pvec{-5}{-13}. \]
		\item On calcule
			\[ \vec{AC} = C - A = \pvec{0 - (-6)}{-7 -3} = \pvec{6}{-10}. \]
		\item
			La somme $\vec{AB} + \vec{BC}$ est donnée par
				\[ \vec{AB} + \vec{BC} = \pvec{11}3 + \pvec{-5}{-13} = \pvec{11 - 5}{3-13} = \pvec{6}{-10} = \vec{AC}. \]
		\item
			On a, plus généralement, $\vec{AB} + \vec{BC} = B - A + C - B = C - A = \vec{AC}$.
	\end{enumerate}
}

	\ifsolutions
	\else
	\begin{center}
		\begin{tikzpicture}[>=stealth, scale=1]
		\begin{axis}[xmin = -10.5, xmax=10.5, ymin=-10.5, ymax=10.5, axis x line=middle, axis y line=middle, axis line style=<->, xlabel={}, ylabel={}, xtick = {-10, -8, ..., 8, 10}, ytick = {-10, -8, ..., 8, 10}, grid=both]
			
		\end{axis}
		\end{tikzpicture}
		\begin{tikzpicture}[>=stealth, scale=1]
		\begin{axis}[xmin = -10.5, xmax=10.5, ymin=-10.5, ymax=10.5, axis x line=middle, axis y line=middle, axis line style=<->, xlabel={}, ylabel={}, xtick = {-10, -8, ..., 8, 10}, ytick = {-10, -8, ..., 8, 10}, grid=both]
			
		\end{axis}
		\end{tikzpicture}
	\end{center}
	\fi

\exe{
	Dessiner les vecteurs
		\begin{align*}
			u = \pvec{-2}{4}, && v = 2u, && w = -\dfrac32u, && z = -\dfrac12u,
		\end{align*}
	dans un repère, et répondre aux questions suivantes.
	
	\begin{enumerate}
		\item Compléter les phrases suivantes.
			\begin{center}
				\og Lorsque le vecteur $u$ est multiplié par \ifsolutions {\color{myr} $2$} \else $\dots$ \fi pour obtenir $v$, son sens \ifsolutions {\color{myr} ne change pas} \else $\dots\dots\dots\dots\dots\dots$ \fi \fg \\ \vspace{10pt}
				\og Lorsque le vecteur $u$ est multiplié par  \ifsolutions {\color{myr} $-\dfrac32$} \else $\dots$ \fi pour obtenir $w$, son sens \ifsolutions {\color{myr} change} \else $\dots\dots\dots\dots\dots\dots$ \fi \fg\\ \vspace{10pt}
				\og Lorsque le vecteur $u$ est multiplié par  \ifsolutions {\color{myr} $-\dfrac12$} \else $\dots$ \fi pour obtenir $z$, son sens \ifsolutions {\color{myr} change} \else $\dots\dots\dots\dots\dots\dots$ \fi \fg
			\end{center}
		\item Calculer les normes de $u, v, w,$ et $z$ et les écrire sous forme $a\sqrt{b}$ où $a, b\in\N$ et $b$ est le plus petit possible.
		\item Compléter les phrases suivantes.
			\begin{center}
				\og Lorsque le vecteur $u$ est multiplié par \ifsolutions {\color{myr} $2$} \else $\dots$ \fi pour obtenir $v$, sa norme $\Vert u\Vert$ est multipliée par \ifsolutions {\color{myr} $2$} \else $\dots$ \fi  \fg \\ \vspace{10pt}
				\og Lorsque le vecteur $u$ est multiplié par \ifsolutions {\color{myr} $-\dfrac32$} \else $\dots$ \fi pour obtenir $w$, sa norme $\Vert u\Vert$ est multipliée par \ifsolutions {\color{myr} $\dfrac32$} \else $\dots$ \fi\fg \\ \vspace{10pt}
				\og Lorsque le vecteur $u$ est multiplié par \ifsolutions {\color{myr} $-\dfrac12$} \else $\dots$ \fi pour obtenir $z$, sa norme $\Vert u\Vert$ est multipliée par \ifsolutions {\color{myr} $\dfrac12$} \else $\dots$ \fi \fg
			\end{center}
		\item Pour $u = \pvec{x}{y}$ et $\kappa\in\R$ quelconques, démontrer l'égalité 
			\[\Vert \kappa \cdot u \Vert = | \kappa | \cdot  \Vert u \Vert. \]
	\end{enumerate}

}{
	\begin{enumerate}
		\item[2.]
			\begin{align*}
				\norm{u} &= \sqrt{(-2)^2 + 4^2} = \sqrt{4 + 16} = \sqrt{20} = \sqrt{4 \times 5} = 2 \sqrt{5}. \\
				\norm{v} &= \sqrt{(-4)^2 + 8^2} = \sqrt{16 + 64} = \sqrt{80} = \sqrt{16 \times 5} = 4\sqrt{5}. \\
				\norm{w} &= \sqrt{3^2 + (-6)^2} = \sqrt{9 + 36} = \sqrt{45} = \sqrt{9 \times 5} = 3 \sqrt{5}. \\
				\norm{z} &= \sqrt{1^2 + (-2)^2} = \sqrt{1 + 4} = \sqrt{5}.
			\end{align*}
	
		On remarque qu'on a les relations suivantes.
			\begin{align*}
				\norm{v} = 2 \norm{u} && \norm{w} = \dfrac32 \norm{u} && \norm{z} = \dfrac12 \norm{u}
			\end{align*}
	
		\item[4.]
			On vérifie l'égalité calmement en utilisant que $\sqrt{x^2} = |x|$.
				\begin{align*}
					\norm{\kappa \cdot u} &= \norm{ \pvec{\kappa \cdot x}{\kappa \cdot y}} \\
										&= \sqrt{ (\kappa \cdot x)^2 + (\kappa \cdot y)^2} \\
										&= \sqrt{ \kappa^2 \cdot x^2 + \kappa^2 \cdot y^2} \\
										&= \sqrt{ \kappa^2  (x^2 + y^2) } \\
										&= \sqrt{\kappa^2} \sqrt{x^2 + y^2} \\
										&= |\kappa| \cdot \norm{u}.
				\end{align*}
	\end{enumerate}
}

\end{document}

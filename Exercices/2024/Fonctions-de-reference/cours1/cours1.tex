				% ENABLE or DISABLE font change
				% use XeLaTeX if true
\newif\ifdys
				\dystrue
				\dysfalse

\newif\ifsolutions
				\solutionstrue
				\solutionsfalse

% DYSLEXIA SWITCH
\newif\ifdys
		
				% ENABLE or DISABLE font change
				% use XeLaTeX if true
				\dystrue
				\dysfalse


\ifdys

\documentclass[a4paper, 14pt]{extarticle}
\usepackage{amsmath,amsfonts,amsthm,amssymb,mathtools}

\tracinglostchars=3 % Report an error if a font does not have a symbol.
\usepackage{fontspec}
\usepackage{unicode-math}
\defaultfontfeatures{ Ligatures=TeX,
                      Scale=MatchUppercase }

\setmainfont{OpenDyslexic}[Scale=1.0]
\setmathfont{Fira Math} % Or maybe try KPMath-Sans?
\setmathfont{OpenDyslexic Italic}[range=it/{Latin,latin}]
\setmathfont{OpenDyslexic}[range=up/{Latin,latin,num}]

\else

\documentclass[a4paper, 12pt]{extarticle}

\usepackage[utf8x]{inputenc}
%fonts
\usepackage{amsmath,amsfonts,amsthm,amssymb,mathtools}
% comment below to default to computer modern
\usepackage{libertinus,libertinust1math}

\fi


\usepackage[french]{babel}
\usepackage[
a4paper,
margin=2cm,
nomarginpar,% We don't want any margin paragraphs
]{geometry}
\usepackage{icomma}

\usepackage{fancyhdr}
\usepackage{array}
\usepackage{hyperref}

\usepackage{multicol, enumerate}
\newcolumntype{P}[1]{>{\centering\arraybackslash}p{#1}}


\usepackage{stackengine}
\newcommand\xrowht[2][0]{\addstackgap[.5\dimexpr#2\relax]{\vphantom{#1}}}

% theorems

\theoremstyle{plain}
\newtheorem{theorem}{Th\'eor\`eme}
\newtheorem*{sol}{Solution}
\theoremstyle{definition}
\newtheorem{ex}{Exercice}
\newtheorem*{rpl}{Rappel}
\newtheorem{enigme}{Énigme}

% corps
\usepackage{calrsfs}
\newcommand{\C}{\mathcal{C}}
\newcommand{\R}{\mathbb{R}}
\newcommand{\Rnn}{\mathbb{R}^{2n}}
\newcommand{\Z}{\mathbb{Z}}
\newcommand{\N}{\mathbb{N}}
\newcommand{\Q}{\mathbb{Q}}

% variance
\newcommand{\Var}[1]{\text{Var}(#1)}

% domain
\newcommand{\D}{\mathcal{D}}


% date
\usepackage{advdate}
\AdvanceDate[0]


% plots
\usepackage{pgfplots}

% table line break
\usepackage{makecell}
%tablestuff
\def\arraystretch{2}
\setlength\tabcolsep{15pt}

%subfigures
\usepackage{subcaption}

\definecolor{myg}{RGB}{56, 140, 70}
\definecolor{myb}{RGB}{45, 111, 177}
\definecolor{myr}{RGB}{199, 68, 64}

% fake sections with no title to move around the merged pdf
\newcommand{\fakesection}[1]{%
  \par\refstepcounter{section}% Increase section counter
  \sectionmark{#1}% Add section mark (header)
  \addcontentsline{toc}{section}{\protect\numberline{\thesection}#1}% Add section to ToC
  % Add more content here, if needed.
}


% SOLUTION SWITCH
\newif\ifsolutions
				\solutionstrue
				%\solutionsfalse

\ifsolutions
	\newcommand{\exe}[2]{
		\begin{ex} #1  \end{ex}
		\begin{sol} #2 \end{sol}
	}
\else
	\newcommand{\exe}[2]{
		\begin{ex} #1  \end{ex}
	}
	
\fi


% tableaux var, signe
\usepackage{tkz-tab}


%pinfty minfty
\newcommand{\pinfty}{{+}\infty}
\newcommand{\minfty}{{-}\infty}

\begin{document}


\AdvanceDate[1]

\begin{document}
\pagestyle{fancy}
\fancyhead[L]{Seconde 13}
\fancyhead[C]{\textbf{Fonctions de référence \ifsolutions \\ Solutions  \fi}}
\fancyhead[R]{\today}
%\pagenumbering{gobble}

\ex{
	On donne la courbe $\C_f$ graphiquement ci-dessous.
	
	\begin{enumerate}
		\item Esquisser, dans le même repère, la courbe de $g(x) = f(x+1)$.
		\item Esquisser, dans le même repère, la courbe de $h(x) = f(x-2)$.
		\item Remplir le tableau de variations ci-dessous.
		\item Peut-on déduire le signe de $g$ en connaissant celui de $f$ en général ?
	\end{enumerate}
	
	\begin{multicols}{2}
	\begin{tikzpicture}[scale=1.1]
		\begin{axis}[xmin = -10, xmax=7, ymin=-4.25, ymax=3.25, axis x line=middle, axis y line=middle, axis line style=->, grid=both,
		ytick={-4,-3,...,4},
		%ytick={-4,-3,...,2,3}, xtick={-11, -10,...,-4,-3},
	    	%every y tick label/.style={
	        %anchor=near yticklabel opposite,
	        %xshift=0.2em,
	    	%}
	    	]
		% g cos
		\addplot[no marks, myb, -, very thick] expression[domain=-10:7, samples=50]{.01*(x+6)*(x+3)*(x-5)}
		node[pos=.35, above]{$\mathcal{C}_f$};
		\ifsolutions
		\addplot[no marks, myr, -, very thick] expression[domain=-10:7, samples=50]{.01*(x+6)*(x+3)*(x-5)+1}
		node[pos=.35, above]{$\mathcal{C}_g$};
		\addplot[no marks, myg, -, very thick] expression[domain=-10:7, samples=50]{.01*(x+6)*(x+3)*(x-5)-2}
		node[pos=.35,above]{$\mathcal{C}_h$};
		\fi
		\end{axis}
	\end{tikzpicture}
	
	\begin{tikzpicture}
		\tkzTabInit
		 [espcl=2]
	       		{$x$ / 1 , Variation de $f(x)$ / 2, Variation de $g(x)$ / 2, Variation de $h(x)$ / 2}
	       		{$-10$,\ifsolutions {-4,6} \fi,\ifsolutions {2} \fi,$7$}
	       	
	       	\ifsolutions	
		\tkzTabVar
			{-/{$-4,1$}, +/{$0,2$}, -/{$-1,2$}, +/{$2,5$}}
		\tkzTabVar
			{-/{$-3,1$}, +/{$1,2$}, -/{$-0,2$}, +/{$3,5$}}
		\tkzTabVar
			{-/{$-6,1$}, +/{$-1,2$}, -/{$-2,2$}, +/{$1,5$}}
		\fi
	\end{tikzpicture}
	\end{multicols}
}

\begin{propriete}[label=prop:1]{ajout d'une constante}{}
	Soit $c\in\R$ un nombre réel et $g(x) = f(x)+c$.
	\begin{enumerate}
		\item Pour obtenir $\C_g$, on translate $\C_f$ \ifsolutions verticalement de $c$ unités. \fi
		\item Les variations de $g$ et de $f$ sont \ifsolutions identiques. \fi
		\item Les signes de $g$ et de $f$ sont \ifsolutions différents en général. \fi
	\end{enumerate}
\end{propriete}

\begin{demonstration*}{de la propriété \ref{prop:1} dans le cas $f$ croissante}{}
	\begin{enumerate}
		\item
		\begin{enumerate}[label=$\bullet$]
			\item Le point $A$ de $\C_f$ d'abscisse $x$ a pour coordonnées \ifsolutions $A(x ; f(x))$. \fi
			\item Le vecteur qui translate verticalement de $c$ unités a pour coordonnées $v = $ \ifsolutions $ \pvec{0}{c}$. \fi
			\item Donc le translaté $A+v$ a pour coordonnées \ifsolutions $(x ; f(x)+c) = (x ; g(x)) \in \C_g$. \fi
%			\item Le point $A$ de $\C_f$ d'abscisse $x$ est \ifsolutions $A(x ; f(x))$. \fi
%			\item Le point $B$ de $\C_g$ d'abscisse $x$ est \ifsolutions $B(x ; g(x))$. \fi
%			\item Or $g(x) = f(x)+c$, donc $\vec{AB} =$ \ifsolutions $\pvec{x-x}{g(x) - f(x)} = \pvec{0}{c}$. \fi
		\end{enumerate}
		
		\item 
		\begin{enumerate}[label=$\bullet$]
			\item 
			Par définition, $f$ est croissante sur un intervalle $I$, dès que \ifsolutions pour tous les $x < y$ de $I$, on a \fi
				 \[ \ifsolutions f(x) < f(y). \fi \] 
			\item 
			Soient $x < y$ deux éléments de $I$. Comme $f$ est croissante, on sait que
				 \[ \ifsolutions f(x) < f(y), \fi \]
			et donc que 
				 \begin{align*} \ifsolutions f(x)+c < f(y)+c  && \iff && g(x) < g(y). \fi \end{align*}
			\item 
			Donc $g$ est croissante sur $I$ : \ifsolutions pour tous les $x < y$ de $I$, on a \fi
				 \[ \ifsolutions g(x) < g(y). \fi \] 
		\end{enumerate}
	\end{enumerate}
\end{demonstration*}


\newpage

\ex{
	On donne la courbe $\C_f$ graphiquement ci-dessous.
	
	\begin{enumerate}
		\item Esquisser, dans le même repère, la courbe de $g(x) = f(2x)$.
		\item Esquisser, dans le même repère, la courbe de $h(x) = f(-x)$.
		\item Remplir le tableau de variations ci-dessous.
		\item Peut-on déduire le signe de $g$ en connaissant celui de $f$ en général ?
	\end{enumerate}
	
	\begin{multicols}{2}
	\begin{tikzpicture}[scale=1.1]
		\begin{axis}[xmin = -10, xmax=7, ymin=-4.25, ymax=3.25, axis x line=middle, axis y line=middle, axis line style=->, grid=both,
		ytick={-4,-3,...,4},
		%ytick={-4,-3,...,2,3}, xtick={-11, -10,...,-4,-3},
	    	%every y tick label/.style={
	        %anchor=near yticklabel opposite,
	        %xshift=0.2em,
	    	%}
	    	]
		\addplot[no marks, myb, -, very thick] expression[domain=-10:7, samples=100]{.01*(x+6)*(x+3)*(x-5)}
		node[pos=.65, above]{$\mathcal{C}_f$};
		\ifsolutions
		\addplot[no marks, myr, -, very thick] expression[domain=-10:7, samples=100]{.02*(x+6)*(x+3)*(x-5)}
		node[pos=.65, below]{$\mathcal{C}_g$};
		\addplot[no marks, myg, -, very thick] expression[domain=-10:7, samples=100]{-.01*(x+6)*(x+3)*(x-5)}
		node[pos=.65, above]{$\mathcal{C}_h$};
		\fi
		\end{axis}
	\end{tikzpicture}
	
	\begin{tikzpicture}
		\tkzTabInit
		 [espcl=2]
	       		{$x$ / 1 , Variation de $f(x)$ / 2, Variation de $g(x)$ / 2, Variation de $h(x)$ / 2}
	       		{$-10$,\ifsolutions {-4,6} \fi,\ifsolutions {2} \fi,$7$}
	       	
	       	\ifsolutions	
		\tkzTabVar
			{-/{$-4,1$}, +/{$0,2$}, -/{$-1,2$}, +/{$2,5$}}
		\tkzTabVar
			{-/{$8,2$}, +/{$0,4$}, -/{$-2,4$}, +/{$5$}}
		\tkzTabVar
			{+/{$4,1$}, -/{$-0,2$}, +/{$1,2$}, -/{$-2,5$}}
		\fi
	       		
		%\tkzTabVar
		%	{+/, R/, -/}
	\end{tikzpicture}
	\end{multicols}
%	\begin{center}
%	\begin{tikzpicture}
%		\tkzTabInit
%		 [espcl=4]
%	       		{$x$ / 1 , Signe de $f(x)$ / 2, Signe de $g(x)$ / 2, Signe de $h(x)$ / 2}
%	       		{$-10$,,,$7$}
%	       		
%		%\tkzTabVar
%		%	{+/, R/, -/}
%	\end{tikzpicture}
%	\end{center}
}{}

\begin{propriete}[label=prop:2]{multiplication par une constante}{}
	Soit $c\in\R$ un nombre réel et $g(x) = c\cdot f(x)$.
	\begin{enumerate}
		\item Si $c > 0$, alors
		\begin{itemize}
			\item les variations de $g$ et de $f$ sont \ifsolutions identiques. \fi
			\item les signes de $g$ et de $f$ sont \ifsolutions identiques. \fi
	\end{itemize}
		\item Si $c < 0$, alors
		\begin{itemize}
			\item les variations de $g$ et de $f$ sont \ifsolutions opposées. \fi % (croissante devient décroissante ; décroissante devient croissante ; et constante reste constant). \fi
			\item les signes de $g$ et de $f$ sont \ifsolutions opposés. \fi % (positif devient négatif ; négatif devient positif ; nul reste nul). \fi
	\end{itemize}
	\end{enumerate}
\end{propriete}

\setlength{\columnsep}{1.5cm}
\setlength{\columnseprule}{1pt}

\begin{demonstration*}{de la propriété \ref{prop:2} dans le cas $f$ décroissante}{}
	\begin{enumerate}[label=$\bullet$]
		\item 
			Par définition, $f$ est décroissante sur un intervalle $I$, dès que \ifsolutions pour tous les $x < y$ de $I$, on a \fi
				 \[ \ifsolutions f(x) > f(y). \fi \] 
		\item 
		Soient $x < y$ deux éléments de $I$. Comme $f$ est décroissante, on sait que
			 \[ \ifsolutions f(x) > f(y), \fi \]
		et donc que 
			\begin{multicols}{2}
			\underline{si $c > 0$},
			\begin{align*} \ifsolutions  c \cdot f(x) &> c \cdot f(y) \\  g(x) &< g(y). \fi \end{align*}
			et $g$ est \ifsolutions décroissante sur $I$. \fi
			
			\underline{si $c < 0$},
			 \begin{align*} \ifsolutions  c \cdot f(x) &< c \cdot f(y) \\  g(x) &> g(y). \fi \end{align*}
			et $g$ est \ifsolutions croissante sur $I$. \fi
			\end{multicols}
	\end{enumerate}
\end{demonstration*}




\end{document}

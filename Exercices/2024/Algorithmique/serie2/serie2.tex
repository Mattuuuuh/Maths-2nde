				% ENABLE or DISABLE font change
				% use XeLaTeX if true
\newif\ifdys
				\dystrue
				\dysfalse

\newif\ifsolutions
				\solutionstrue
				\solutionsfalse

% DYSLEXIA SWITCH
\newif\ifdys
		
				% ENABLE or DISABLE font change
				% use XeLaTeX if true
				\dystrue
				\dysfalse


\ifdys

\documentclass[a4paper, 14pt]{extarticle}
\usepackage{amsmath,amsfonts,amsthm,amssymb,mathtools}

\tracinglostchars=3 % Report an error if a font does not have a symbol.
\usepackage{fontspec}
\usepackage{unicode-math}
\defaultfontfeatures{ Ligatures=TeX,
                      Scale=MatchUppercase }

\setmainfont{OpenDyslexic}[Scale=1.0]
\setmathfont{Fira Math} % Or maybe try KPMath-Sans?
\setmathfont{OpenDyslexic Italic}[range=it/{Latin,latin}]
\setmathfont{OpenDyslexic}[range=up/{Latin,latin,num}]

\else

\documentclass[a4paper, 12pt]{extarticle}

\usepackage[utf8x]{inputenc}
%fonts
\usepackage{amsmath,amsfonts,amsthm,amssymb,mathtools}
% comment below to default to computer modern
\usepackage{libertinus,libertinust1math}

\fi


\usepackage[french]{babel}
\usepackage[
a4paper,
margin=2cm,
nomarginpar,% We don't want any margin paragraphs
]{geometry}
\usepackage{icomma}

\usepackage{fancyhdr}
\usepackage{array}
\usepackage{hyperref}

\usepackage{multicol, enumerate}
\newcolumntype{P}[1]{>{\centering\arraybackslash}p{#1}}


\usepackage{stackengine}
\newcommand\xrowht[2][0]{\addstackgap[.5\dimexpr#2\relax]{\vphantom{#1}}}

% theorems

\theoremstyle{plain}
\newtheorem{theorem}{Th\'eor\`eme}
\newtheorem*{sol}{Solution}
\theoremstyle{definition}
\newtheorem{ex}{Exercice}
\newtheorem*{rpl}{Rappel}
\newtheorem{enigme}{Énigme}

% corps
\usepackage{calrsfs}
\newcommand{\C}{\mathcal{C}}
\newcommand{\R}{\mathbb{R}}
\newcommand{\Rnn}{\mathbb{R}^{2n}}
\newcommand{\Z}{\mathbb{Z}}
\newcommand{\N}{\mathbb{N}}
\newcommand{\Q}{\mathbb{Q}}

% variance
\newcommand{\Var}[1]{\text{Var}(#1)}

% domain
\newcommand{\D}{\mathcal{D}}


% date
\usepackage{advdate}
\AdvanceDate[0]


% plots
\usepackage{pgfplots}

% table line break
\usepackage{makecell}
%tablestuff
\def\arraystretch{2}
\setlength\tabcolsep{15pt}

%subfigures
\usepackage{subcaption}

\definecolor{myg}{RGB}{56, 140, 70}
\definecolor{myb}{RGB}{45, 111, 177}
\definecolor{myr}{RGB}{199, 68, 64}

% fake sections with no title to move around the merged pdf
\newcommand{\fakesection}[1]{%
  \par\refstepcounter{section}% Increase section counter
  \sectionmark{#1}% Add section mark (header)
  \addcontentsline{toc}{section}{\protect\numberline{\thesection}#1}% Add section to ToC
  % Add more content here, if needed.
}


% SOLUTION SWITCH
\newif\ifsolutions
				\solutionstrue
				%\solutionsfalse

\ifsolutions
	\newcommand{\exe}[2]{
		\begin{ex} #1  \end{ex}
		\begin{sol} #2 \end{sol}
	}
\else
	\newcommand{\exe}[2]{
		\begin{ex} #1  \end{ex}
	}
	
\fi


% tableaux var, signe
\usepackage{tkz-tab}


%pinfty minfty
\newcommand{\pinfty}{{+}\infty}
\newcommand{\minfty}{{-}\infty}

\begin{document}


\AdvanceDate[1]

\begin{document}
\pagestyle{fancy}
\fancyhead[L]{Seconde 13}
\fancyhead[C]{\textbf{Algorithmique 2 : fonctions \ifsolutions \, -- Solutions  \fi}}
\fancyhead[R]{\today}

\exe{
	On considère les deux programmes de la figure \ref{fig:1}. Qu'impriment-t-ils ?
}{}

\begin{figure}[h!]
\begin{subfigure}{.5\textwidth}
\begin{mintedbox}{python}
def plusun(a):
	return a+1
	
print(plusun(4))
print(plusun(-1))
print(plusun(0))
\end{mintedbox}
\end{subfigure}
\begin{subfigure}{.5\textwidth}
\begin{mintedbox}{python}
def somme(a,b):
	return a+b

print(somme(4,3))
print(somme(-1,2))
print(somme(-3, -10))
\end{mintedbox}
\end{subfigure}

\caption{Fonctions à un et deux arguments.}
\label{fig:1}
\end{figure}


\exe{
	On considère les deux programmes de la figure \ref{fig:2}.
	\begin{enumerate}
		\item Qu'impriment les deux programmes ?
		\item Décrire ce que renvoie \texttt{seconde(a,b)} pour deux nombres réels $a$ et $b$.
	\end{enumerate}
}{}

\begin{figure}[h!]
\begin{subfigure}{.5\textwidth}
\begin{mintedbox}{python}
def premiere(a, b):
	if a <= b:
		return b
	else:
		return a
		
print(premiere(2,10))
print(premiere(10,2))
print(premiere(-5,3))
print(premiere(3,-5))
\end{mintedbox}
\end{subfigure}
\begin{subfigure}{.5\textwidth}
\begin{mintedbox}{python}
def seconde(a,b):
	if a <= b:
		return b
	return a


print(seconde(3,11))
print(seconde(11,3))
print(seconde(-6,0))
print(seconde(0,-6))
\end{mintedbox}
\end{subfigure}

\caption{Deux fonctions à deux arguments.}
\label{fig:2}
\end{figure}




\begin{figure}[h!]
\begin{subfigure}{.5\textwidth}
\begin{mintedbox}{python}
def triangulaire(n):
	if n==1:
		return 1
	return n + triangulaire(n-1)
	
print(triangulaire(1))
print(triangulaire(2))
print(triangulaire(3))
print(triangulaire(5))
\end{mintedbox}
\caption{Nombres triangulaires}
\label{fig:3a}
\end{subfigure}
\begin{subfigure}{.5\textwidth}
\begin{mintedbox}{python}
def factorielle(n):
	if n==1:
		return 1
	return n*factorielle(n-1)
	
print(factorielle(1))
print(factorielle(2))
print(factorielle(3))
print(factorielle(5))
\end{mintedbox}
\caption{Factorielle}
\label{fig:3b}
\end{subfigure}

\caption{Fonctions récursives.}
\label{fig:3}
\end{figure}


\exe{
	On considère les programmes de la figure \ref{fig:3}.
	\begin{enumerate}
		\item Qu'impriment les deux programmes ?
		\item Décrire ce que \texttt{triangulaire(n)} renvoie pour un entier $n\in\N$.
		\item Que se passe-t-il si on appelle \texttt{triangulaire(0)} ? Comment régler ce problème ?
		\item Décrire ce que \texttt{factorielle(n)} renvoie pour un entier $n\in\N$.
		\item Que se passe-t-il si on appelle \texttt{factorielle(-10)} ? Comment régler ce problème ?
	\end{enumerate}
}{}

\end{document}

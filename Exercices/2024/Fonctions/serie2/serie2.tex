				% ENABLE or DISABLE font change
				% use XeLaTeX if true
\newif\ifdys
				\dystrue
				\dysfalse

\newif\ifsolutions
				\solutionstrue
				%\solutionsfalse

% DYSLEXIA SWITCH
\newif\ifdys
		
				% ENABLE or DISABLE font change
				% use XeLaTeX if true
				\dystrue
				\dysfalse


\ifdys

\documentclass[a4paper, 14pt]{extarticle}
\usepackage{amsmath,amsfonts,amsthm,amssymb,mathtools}

\tracinglostchars=3 % Report an error if a font does not have a symbol.
\usepackage{fontspec}
\usepackage{unicode-math}
\defaultfontfeatures{ Ligatures=TeX,
                      Scale=MatchUppercase }

\setmainfont{OpenDyslexic}[Scale=1.0]
\setmathfont{Fira Math} % Or maybe try KPMath-Sans?
\setmathfont{OpenDyslexic Italic}[range=it/{Latin,latin}]
\setmathfont{OpenDyslexic}[range=up/{Latin,latin,num}]

\else

\documentclass[a4paper, 12pt]{extarticle}

\usepackage[utf8x]{inputenc}
%fonts
\usepackage{amsmath,amsfonts,amsthm,amssymb,mathtools}
% comment below to default to computer modern
\usepackage{libertinus,libertinust1math}

\fi


\usepackage[french]{babel}
\usepackage[
a4paper,
margin=2cm,
nomarginpar,% We don't want any margin paragraphs
]{geometry}
\usepackage{icomma}

\usepackage{fancyhdr}
\usepackage{array}
\usepackage{hyperref}

\usepackage{multicol, enumerate}
\newcolumntype{P}[1]{>{\centering\arraybackslash}p{#1}}


\usepackage{stackengine}
\newcommand\xrowht[2][0]{\addstackgap[.5\dimexpr#2\relax]{\vphantom{#1}}}

% theorems

\theoremstyle{plain}
\newtheorem{theorem}{Th\'eor\`eme}
\newtheorem*{sol}{Solution}
\theoremstyle{definition}
\newtheorem{ex}{Exercice}
\newtheorem*{rpl}{Rappel}
\newtheorem{enigme}{Énigme}

% corps
\usepackage{calrsfs}
\newcommand{\C}{\mathcal{C}}
\newcommand{\R}{\mathbb{R}}
\newcommand{\Rnn}{\mathbb{R}^{2n}}
\newcommand{\Z}{\mathbb{Z}}
\newcommand{\N}{\mathbb{N}}
\newcommand{\Q}{\mathbb{Q}}

% variance
\newcommand{\Var}[1]{\text{Var}(#1)}

% domain
\newcommand{\D}{\mathcal{D}}


% date
\usepackage{advdate}
\AdvanceDate[0]


% plots
\usepackage{pgfplots}

% table line break
\usepackage{makecell}
%tablestuff
\def\arraystretch{2}
\setlength\tabcolsep{15pt}

%subfigures
\usepackage{subcaption}

\definecolor{myg}{RGB}{56, 140, 70}
\definecolor{myb}{RGB}{45, 111, 177}
\definecolor{myr}{RGB}{199, 68, 64}

% fake sections with no title to move around the merged pdf
\newcommand{\fakesection}[1]{%
  \par\refstepcounter{section}% Increase section counter
  \sectionmark{#1}% Add section mark (header)
  \addcontentsline{toc}{section}{\protect\numberline{\thesection}#1}% Add section to ToC
  % Add more content here, if needed.
}


% SOLUTION SWITCH
\newif\ifsolutions
				\solutionstrue
				%\solutionsfalse

\ifsolutions
	\newcommand{\exe}[2]{
		\begin{ex} #1  \end{ex}
		\begin{sol} #2 \end{sol}
	}
\else
	\newcommand{\exe}[2]{
		\begin{ex} #1  \end{ex}
	}
	
\fi


% tableaux var, signe
\usepackage{tkz-tab}


%pinfty minfty
\newcommand{\pinfty}{{+}\infty}
\newcommand{\minfty}{{-}\infty}

\begin{document}


\AdvanceDate[0]

\begin{document}
\pagestyle{fancy}
\fancyhead[L]{Seconde 13}
\fancyhead[C]{\textbf{Fonctions 2 \ifsolutions -- Solutions  \fi}}
\fancyhead[R]{\today}


	\exe{
		Considérons la fonction $f: \left]{-}\dfrac72 ; \dfrac{11}2 \right[ \rightarrow\R$ donnée algébriquement par
			\[ f(x) = \dfrac17-x. \]
		Pour chaque point suivant, déterminer s'il appartient à $\C_f$ ou non.
		
		\begin{multicols}{2}
		\begin{enumerate}[i)]
			\item $\left(0; \dfrac17\right)$
			\item $\left(\dfrac17 ; 0\right)$
			\item $\left(\dfrac27 ; \dfrac37\right)$
			\item $\left(-\dfrac{13}7 ; 2\right)$
			\item $\left(\dfrac67 ; 1\right)$
			\item $\left(\dfrac27 ; -\dfrac17\right)$
		\end{enumerate}
		\end{multicols}
	
	}{
	On rappelle la propriété fondamentale
		\begin{align*}
			(x;y) \in \C_f && \iff && y = f(x)
		\end{align*}
	Pour savoir si un point $(x;y)$ appartient à $\C_f$, il s'agit de vérifier si l'égalité $y=f(x)$ tient.
	
		\begin{enumerate}[i)]
			\item On applique la propriété pour $x = 0, y= \dfrac17$.
			D'une part, $f(x) = f(0) = \dfrac17 - 0 = \dfrac17$, et d'autre part $y=\dfrac17$.
			On a donc bien $y=f(x)$ pour ce couple, et il appartient à $\C_f$.
				\[ \left(0; \dfrac17\right) \in \C_f. \]
			
			\item On choisit $(x;y) = \left(\dfrac17 ; 0\right)$, et on compare $f(x) = f\left(\dfrac17\right) = 0$ à $y=0$. 
			On a donc bien
				\[ \left(\dfrac17 ; 0\right) \in \C_f. \]
				
			\item On choisit $(x;y) = \left(\dfrac27 ; \dfrac37\right)$, et on compare $f(x) = f\left(\dfrac27\right) = -\dfrac17$ à $y=\dfrac37$. 
			D'où
				\[ \left(\dfrac27 ; \dfrac37\right) \not\in \C_f. \]
				
			\item On choisit $(x;y) = \left(-\dfrac{13}7 ; 2\right)$, et on compare $f(x) = f\left(-\dfrac{13}7\right) = 2$ à $y=2$. 
			On a donc bien
				\[ \left(-\dfrac{13}7 ; 2\right) \in \C_f. \]
			\item On choisit $(x;y) = \left(\dfrac67 ; 1\right)$, et on compare $f(x) = f\left(\dfrac67\right) = -\dfrac57$ à $y=1$. 
			Par suite,
				\[ \left(\dfrac67 ; 1\right) \not\in \C_f. \]
			\item On choisit $(x;y) = \left(\dfrac27 ; -\dfrac17\right)$, et on compare $f(x) = f\left(\dfrac27\right) = \dfrac{-1}7$ à $y=-\dfrac17$. 
			D'où
				\[ \left(\dfrac27 ; -\dfrac17\right) \not\in \C_f. \]
		\end{enumerate}
	
	}

	
	\exe{
		Considérons deux fonctions $f, g: ]{-}3 ; 3[ \rightarrow\R$ données algébriquement par
			\begin{align*}
				f(x) = x^2 - 2\cdot x && g(x) = (x-1)^2
			\end{align*}
		
		\begin{enumerate}
			\item Esquisser les représentations graphiques de $f$ et de $g$ dans un même repère.
			\item Démontrer que $g(x) - 1^2 = f(x)$ pour tout $x$ du domaine.
			\item En déduire que $(x-1)^2 = x^2 - 2\cdot x + 1$ pour tout $x$ du domaine.
		\end{enumerate}
	}{
	
		\begin{enumerate}
			\item On choisit plusieurs valeurs de $x\in]{-}3 ; 3[$ et on représente les points
				\[ \left(x ; f(x) \right), \]
			qu'on relie pour esquisser $\C_f$.
			On fait idem pour $g$, ce qui donne le graphique ci-dessous.
			
			\begin{center}
			\begin{tikzpicture}[>=stealth]
				\begin{axis}[xmin = -3.1, xmax=3.1, ymin=-1.1, ymax=15.1, axis x line=middle, axis y line=middle, axis line style=->, grid=both]
					\addplot[no marks, violet, -] expression[domain=-3:3, samples=100]{x^2 -2*x}
					node[pos=.45, below=10pt]{$\mathcal{C}_f$};
					\addplot[no marks, blue, -] expression[domain=-3:3, samples=100]{(x-1)^2}
					node[pos=.4, right]{$\mathcal{C}_g$};
				\end{axis}
			
			\end{tikzpicture}
			\end{center}
			
			
			\item 
			L'identité remarquable $a^2 - b^2 = (a+b)(a-b)$ donne en l'occurrence
				\begin{align*}
					(x-1)^2 - 1^2 &= (x-1+1) \cdot (x-1-1) \\
									&= x \cdot (x-2) \\
									&= x^2 - 2\cdot x = f(x).
				\end{align*}
			
			\item 
			On déduit alors que 
				\[ (x-1)^2 = g(x) = f(x) + 1 = x^2 - 2\cdot x + 1 \]
			pour tout $x\in]{-}3;3[$.
		\end{enumerate}
	
	}
	
	\exe{
		Considérons la représentation graphique suivante d'une fonction $f$ définie sur $\D = ]{-}3,4 ; 2,3[$.
		
			\begin{center}
			\begin{tikzpicture}[>=stealth]
				\begin{axis}[xmin = -3.4, xmax=2.3, ymin=-5.1, ymax=5.1, axis x line=middle, axis y line=middle, axis line style=->, grid=both]
					\addplot[no marks, blue, -] expression[domain=-5:3, samples=50]{x^3 /3 - 2*x +3}
					node[pos=.3, right]{$\mathcal{C}_f$};
				\end{axis}
			\end{tikzpicture}
			\end{center}
		\begin{enumerate}
			\item Donner approximativement les images de $-1,5$ et de $-\dfrac{20}7$ par $f$.
			\item Énumérer approximativement les antécédents de $-2$ et de $2$ par $f$.
			\item Donner approximativement un réel qui admet exactement deux antécédents par $f$.
			\item Si $f$ était définie sur $\R$ tout entier, serait-il toujours possible de connaître l'image de $-2$ ? Et tous les antécédents de $-2$ ?
		\end{enumerate}
		Supposons désormais que $f(x) = 3-2\cdot x +\dfrac13 \cdot x^3$ pour tout $x\in\D$ du domaine.
		\begin{enumerate}
			\item[5.] Vérifier à la calculatrice les réponses aux deux premières questions.
			\item[6.] Montrer sans calculatrice que l'image par $f$ de $-3$ est $0$ et que l'image par $f$ de $0$ est $3$.
		\end{enumerate}
	}{
	
			\begin{enumerate}
			\item On détermine approximativement
				\begin{align*}
				f(-1,5) \approx 4,5 && f\left(-\dfrac{20}7\right) \approx f(-2,86) \approx 2
				 \end{align*}
			\item On cherche d'abord les antécédents de $-2$, c'est-à-dire les nombres $x$ du domaine vérifiants
				\[ f(x) = -2. \]
			Pour ça, on se pose \og à hauteur -2 \fg : on tracer une droite horizontale d'ordonnée $-2$ et on regarde les points d'intersection.
				\begin{center}
				\begin{tikzpicture}[>=stealth]
					\begin{axis}[xmin = -3.4, xmax=2.3, ymin=-5.1, ymax=5.1, axis x line=middle, axis y line=middle, axis line style=->, grid=both]
						\addplot[no marks, blue, -] expression[domain=-5:3, samples=50]{x^3 /3 - 2*x +3};
						
						\addplot[no marks, red, -] expression[domain=-5:3, samples=2]{-2};
					\end{axis}
				\end{tikzpicture}
				\end{center}
			
			
			En l'occurrence, seul $x\approx -3,2$ fonctionne :
				\[ f(-3,2) \approx -2. \]
			Pour les antécédents de $2$, on fait idem en regardant les points de la courbe d'ordonnée $2$.
			On trouve trois valeurs approximatives : $-2,8 ; 0,5 ; $ et $2,1$.
			
				\begin{center}
				\begin{tikzpicture}[>=stealth]
					\begin{axis}[xmin = -3.4, xmax=2.3, ymin=-5.1, ymax=5.1, axis x line=middle, axis y line=middle, axis line style=->, grid=both]
						\addplot[no marks, blue, -] expression[domain=-5:3, samples=50]{x^3 /3 - 2*x +3};
						
						\addplot[no marks, red, -] expression[domain=-5:3, samples=2]{2};
					\end{axis}
				\end{tikzpicture}
				\end{center}
			
			\item 
			On a plusieurs choix ici. 
			Il faut placer une droite horizontale telle qu'elle s'intersecte exactement deux fois avec la courbe de $f$.
			Un choix clair est $4$ (en violet ci-dessous).
			Un choix moins clair est $1,1$, en faisant en sorte que la courbe frôle la droite horizontale qu'on place en ordonnée $1,1$ (en rouge ci-dessous).
			On dit alors que la droite est \emph{tangente} à la courbe.
			
				\begin{center}
				\begin{tikzpicture}[>=stealth]
					\begin{axis}[xmin = -3.4, xmax=2.3, ymin=-5.1, ymax=5.1, axis x line=middle, axis y line=middle, axis line style=->, grid=both]
						\addplot[no marks, blue, -] expression[domain=-5:3, samples=50]{x^3 /3 - 2*x +3};
						
						\addplot[no marks, red, -] expression[domain=-5:3, samples=2]{1.1};
						\addplot[no marks, violet, -] expression[domain=-5:3, samples=2]{4};
					\end{axis}
				\end{tikzpicture}
				\end{center}
			
			\item L'image est unique et ne dépend pas du domaine (tant que celui-ci contient $-2$ !).
			On peut donc toujours connaître $f(-2)$, même sachant qu'on connaisse pas $f$ sur $\R$ tout entier mais que sur un domaine restreint.
			
			Les antécédents, eux, dépendent du domaine choisi.
			Comme on ne sait pas du tout à quoi ressemble $\C_f$ en dehors du domaine choisi, il est impossible de déterminer tous les réels $x\in\R$ antécédents de $-2$ par $f$.
			
			\item On calcule grâce à la calculatrice (ou sans...)
				\begin{align*}
					f(-1,5) = 4,875 && f\left(-\dfrac{20}7 \right) \approx 0,94.
				\end{align*}
			\item On calcule à la main que
				\begin{align*}
					f(-3) &= 3 - 2 \cdot (-3) + \dfrac13 \cdot (-3)^3, \\
						&= 3 + 6 - 9, \\
						&= 0.
				\end{align*}
			Pour l'image de $0$, remarquons que seul le terme ne dépendant pas de $x$ subsiste. 
			On l'appelle le terme \emph{constant}, et on trouve $f(0) = 3$, comme requis.
		\end{enumerate}
	
	
	}
	
	\exe{
		Esquisser la courbe de la fonction $f:[-2; 4]\rightarrow\R$ donnée algébriquement par
			\[ f(x) = 3. \]
		Que dire de $f$ et de $\C_f$ ?
	}{
	Pour tracer $\C_f$, on choisit des valeurs de $x$ du domaine $[{-2};4]$, on calcule $f(x)$, et on place les points $(x ; f(x))$ obtenus.
	
		\begin{center}
		\begin{tikzpicture}[>=stealth]
			\begin{axis}[xmin = -2.1, xmax=4.1, ymin=2, ymax=4, axis x line=middle, axis y line=middle, axis line style=->, grid=both]
				\addplot[no marks, blue, -] expression[domain=-2:4, samples=2]{3} 
				node[above, pos=.5] {$\C_f$};
			\end{axis}
		\end{tikzpicture}
		\end{center}
	
	On dit ici que $f$ est \emph{constante} car elle ne dépend pas de $x$.
	$\C_f$ est donc une droite horizontale.
	}
	
	\exe{
		Esquisser la courbe de la fonction $f:[-5;3]\rightarrow\R$ donnée algébriquement par
			\[ f(x) = 1-x. \]
		Que dire $\C_f$ ?
	}{
	$\C_f$ est une droite et on dit que $f$ est \emph{affine}.
		\begin{center}
		\begin{tikzpicture}[>=stealth]
			\begin{axis}[xmin = -2.1, xmax=4.1, ymin=-3.1, ymax=4.1, axis x line=middle, axis y line=middle, axis line style=->, grid=both]
				\addplot[no marks, blue, -] expression[domain=-2:4, samples=2]{1-x} 
				node[above, pos=.4] {$\C_f$};
			\end{axis}
		\end{tikzpicture}
		\end{center}
	}
	
	\exe{
		Esquisser la courbe de la fonction $f:[3;10]\rightarrow\R$ donnée algébriquement par
			\[ f(x) = \dfrac3x + 1. \]
	}{
	Il faut choisir suffisamment de $x \in [3;10]$ du domaine afin de pouvoir bien tracer l'allure de la courbe.
	Attention, celle-ci n'est pas une droite !
	
		\begin{center}
		\begin{tikzpicture}[>=stealth]
			\begin{axis}[xmin = 3, xmax=10, ymin=1, ymax=2.1, axis x line=middle, axis y line=middle, axis line style=->, grid=both]
				\addplot[no marks, blue, -] expression[domain=3:10, samples=50]{3/x + 1} 
				node[above, pos=.4] {$\C_f$};
			\end{axis}
		\end{tikzpicture}
		\end{center}
	
	}
	
	
	
	\exe{
		Un fonction $f$ admet une représentation graphique suivante.
			\begin{center}
			\begin{tikzpicture}[>=stealth]
				\begin{axis}[xmin = -3.1, xmax=1.1, ymin=-3.1, ymax=5.1, axis x line=middle, axis y line=middle, axis line style=->, grid=both]
					\addplot[no marks, blue, -] expression[domain=-3:2, samples=2]{-2/3 - 2*x}
					node[pos=.3, right]{$\mathcal{C}_f$};
				\end{axis}
			
			\end{tikzpicture}
			\end{center}
		Parmis les expressions algébriques suivantes, trouver celle qui correspond à $f(x)$.
			\begin{multicols}{2}
			\begin{enumerate}[i)]
				\item $1-x$
				\item $\dfrac{-1-x}3$
				\item $\left(x+\dfrac13\right)^2$
				\item $-2\cdot x - \dfrac23$
			\end{enumerate}
			\end{multicols}
	}{
	Clairement $f(0) < 0$ est strictement négatif.
	L'expression de $f$ ne peut donc pas être la première ni la troisième.
	
	Ensuite, si l'expression de $f$ était la deuxième, on aurait $f(-1) = 0$, ce qui n'est clairement pas le cas.
	Ainsi
		\[ f(x) = -2\cdot x - \frac23. \]	
	}
	
	
	
	\subsection*{Exercices supplémentaires}
	
	\exe{
		Pour chaque point suivant, donner l'expression algébrique d'une fonction réelle $f$ sur $\R$ telle qu'il existe un réel $y\in\R$ admettant
			\begin{enumerate}
				\item exactement un antécédent
				\item exactement deux antécédents
				\item exactement trois antécédents
				\item une infinité d'antécédents
			\end{enumerate}	
	}{
		\begin{enumerate}
			\item La fonction affine $f(x) = x$ donne une droite dont chaque image admet un unique antécédent.
			\item La fonction affine $f(x) = x^2$ donne une parabole. En résolvant $f(x) =1$, on peut démontrer que seuls $1$ et $-1$ sont les antécédents de $1$.
			\item Considérons $f(x) = (x-1)\cdot(x-2)\cdot(x-3)$. Résoudre $f(x)=0$ pour montrer que seuls $1,2,$ et $3$ sont antécédents de $0$.
			\item Une fonction constante fonctionne bien. Par exemple $f(x) = 0$.
		\end{enumerate}	
	}
	
	\exe{
		Donner graphiquement une fonction sur $\R$ non constante telle que toutes les images par $f$ admettent un nombre infini d'antécédents.
	}{
	On peut définir par exemple
		\[ f(x) = \begin{cases} +1 \text{ si $x\geq 0$}, \\
								-1 \text{ sinon. }
				\end{cases}. \]
	Pour une fonction plus intéressante, on pourrait prendre une fonction en vague.
	La fonction sinus fonctionne bien : entrer par exemple \texttt{y=sin(x)} sur Geogebra.
	}
	
	
	\exe{
		Comparer les représentations graphiques des fonctions suivantes données algébriquement.
			\begin{align*}
				f(x) = x^2 && g(x) = x^2 - 3 && h(x) = (x+4)^2.
			\end{align*}
	}{
	On graphe les fonctions ci-dessous. 
	On remarque que $\C_g$ est juste en dessous de $\C_f$. C'est logique car
		\[ g(x) = f(x) - 3, \]
	donc quand on place les points pour $\C_f$ et $\C_g$, ceux de $\C_g$ sont $3$ en dessous de ceux de $\C_f$.
	
	On remarque aussi que $\C_h$ est la courbe de $\C_f$ décalée vers la gauche.
	C'est également cohérent car
		\[ h(x) = f(x+4). \]
	Pour connaître l'image de $x$ par $h$, on prend l'image de $x+4$ par $f$.
	On a donc par exemple $h(-4) = f(0), h(3,9) = f(0,1),$ etc... 
	On comprend bien pourquoi $\C_h$ est la $\C_f$ décalée de $4$ vers la gauche.
	
		\begin{center}
		\begin{tikzpicture}[>=stealth]
			\begin{axis}[xmin = -8, xmax=5, ymin=-3.1, ymax=20.1, axis x line=middle, axis y line=middle, axis line style=->, grid=both]
				\addplot[no marks, blue, -] expression[domain=-8:5, samples=50]{x^2}
				node[pos=.6, right]{$\mathcal{C}_f$};
				\addplot[no marks, red, -] expression[domain=-8:5, samples=50]{x^2-3}
				node[pos=.8, right]{$\mathcal{C}_g$};
				\addplot[no marks, violet, -] expression[domain=-8:5, samples=50]{(x+4)^2}
				node[pos=.1, right]{$\mathcal{C}_h$};
			\end{axis}
			
		
		\end{tikzpicture}
		\end{center}
	}

\end{document}

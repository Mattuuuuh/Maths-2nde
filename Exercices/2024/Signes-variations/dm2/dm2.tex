%% INPUT PREAMBLE.TEX
%% THEN INPUT VARS_{i}.ADR
%% THEN RUN THIS
%%!TEX encoding = UTF8
%!TEX root =notes.tex


%%%%%%%%%%%%%%%%%%%%%%%%%%%%%%%%%
% PACKAGE IMPORTS
%%%%%%%%%%%%%%%%%%%%%%%%%%%%%%%%%


\usepackage[french]{babel}

\usepackage[tmargin=2cm,rmargin=1in,lmargin=1in,margin=0.85in,bmargin=2cm,footskip=.2in]{geometry}
\usepackage{amsmath,amsfonts,amsthm,amssymb,mathtools}
\usepackage[varbb]{newpxmath}
\usepackage{xfrac}
\usepackage[makeroom]{cancel}
\usepackage{mathtools}
\usepackage{bookmark}
\usepackage{enumitem}
\usepackage{hyperref,theoremref}
\hypersetup{
	pdftitle={Assignment},
	colorlinks=true, linkcolor=doc!90,
	bookmarksnumbered=true,
	bookmarksopen=true
}
\usepackage[most,many,breakable]{tcolorbox}
\usepackage{xcolor}
\usepackage{varwidth}
\usepackage{varwidth}
\usepackage{etoolbox}
%\usepackage{authblk}
\usepackage{nameref}
\usepackage{multicol,array}
\usepackage{tikz-cd}
\usepackage[ruled,vlined,linesnumbered]{algorithm2e}
\usepackage{comment} % enables the use of multi-line comments (\ifx \fi) 
\usepackage{import}
\usepackage{xifthen}
\usepackage{pdfpages}
\usepackage{transparent}


\newcommand\mycommfont[1]{\footnotesize\ttfamily\textcolor{blue}{#1}}
\SetCommentSty{mycommfont}
\newcommand{\incfig}[1]{%
    \def\svgwidth{\columnwidth}
    \import{./figures/}{#1.pdf_tex}
}

\usepackage{tikzsymbols}
%\renewcommand\qedsymbol{$\Laughey$}


%\usepackage{import}
%\usepackage{xifthen}
%\usepackage{pdfpages}
%\usepackage{transparent}


%%%%%%%%%%%%%%%%%%%%%%%%%%%%%%
% SELF MADE COLORS
%%%%%%%%%%%%%%%%%%%%%%%%%%%%%%



\definecolor{myg}{RGB}{56, 140, 70}
\definecolor{myb}{RGB}{45, 111, 177}
\definecolor{myr}{RGB}{199, 68, 64}
\definecolor{mytheorembg}{HTML}{F2F2F9}
\definecolor{mytheoremfr}{HTML}{00007B}
\definecolor{mylenmabg}{HTML}{FFFAF8}
\definecolor{mylenmafr}{HTML}{983b0f}
\definecolor{mypropbg}{HTML}{f2fbfc}
\definecolor{mypropfr}{HTML}{191971}
\definecolor{myexamplebg}{HTML}{F2FBF8}
\definecolor{myexamplefr}{HTML}{88D6D1}
\definecolor{myexampleti}{HTML}{2A7F7F}
\definecolor{mydefinitbg}{HTML}{E5E5FF}
\definecolor{mydefinitfr}{HTML}{3F3FA3}
\definecolor{notesgreen}{RGB}{0,162,0}
\definecolor{myp}{RGB}{197, 92, 212}
\definecolor{mygr}{HTML}{2C3338}
\definecolor{myred}{RGB}{127,0,0}
\definecolor{myyellow}{RGB}{169,121,69}
\definecolor{myexercisebg}{HTML}{F2FBF8}
\definecolor{myexercisefg}{HTML}{88D6D1}


%%%%%%%%%%%%%%%%%%%%%%%%%%%%
% TCOLORBOX SETUPS
%%%%%%%%%%%%%%%%%%%%%%%%%%%%

\setlength{\parindent}{1cm}
%================================
% THEOREM BOX
%================================

\tcbuselibrary{theorems,skins,hooks}
\newtcbtheorem[number within=chapter]{Theorem}{Théorème}
{%
	enhanced,
	breakable,
	colback = mytheorembg,
	frame hidden,
	boxrule = 0sp,
	borderline west = {2pt}{0pt}{mytheoremfr},
	sharp corners,
	detach title,
	before upper = \tcbtitle\par\smallskip,
	coltitle = mytheoremfr,
	fonttitle = \bfseries\sffamily,
	description font = \mdseries,
	separator sign none,
	segmentation style={solid, mytheoremfr},
}
{th}


\tcbuselibrary{theorems,skins,hooks}
\newtcolorbox{Theoremcon}
{%
	enhanced
	,breakable
	,colback = mytheorembg
	,frame hidden
	,boxrule = 0sp
	,borderline west = {2pt}{0pt}{mytheoremfr}
	,sharp corners
	,description font = \mdseries
	,separator sign none
}

%================================
% Corollery
%================================
\tcbuselibrary{theorems,skins,hooks}
\newtcbtheorem[use counter=tcb@cnt@Theorem]{Corollary}{Corollaire}
{%
	enhanced
	,breakable
	,colback = myp!10
	,frame hidden
	,boxrule = 0sp
	,borderline west = {2pt}{0pt}{myp!85!black}
	,sharp corners
	,detach title
	,before upper = \tcbtitle\par\smallskip
	,coltitle = myp!85!black
	,fonttitle = \bfseries\sffamily
	,description font = \mdseries
	,separator sign none
	,segmentation style={solid, myp!85!black}
}
{th}

%================================
% LENMA
%================================

\tcbuselibrary{theorems,skins,hooks}
\newtcbtheorem[use counter=tcb@cnt@Theorem]{Lemma}{Lemme}
{%
	enhanced,
	breakable,
	colback = mylenmabg,
	frame hidden,
	boxrule = 0sp,
	borderline west = {2pt}{0pt}{mylenmafr},
	sharp corners,
	detach title,
	before upper = \tcbtitle\par\smallskip,
	coltitle = mylenmafr,
	fonttitle = \bfseries\sffamily,
	description font = \mdseries,
	separator sign none,
	segmentation style={solid, mylenmafr},
}
{th}


%================================
% PROPOSITION
%================================

\tcbuselibrary{theorems,skins,hooks}
\newtcbtheorem[use counter=tcb@cnt@Theorem]{Prop}{Proposition}
{%
	enhanced,
	breakable,
	colback = mypropbg,
	frame hidden,
	boxrule = 0sp,
	borderline west = {2pt}{0pt}{mypropfr},
	sharp corners,
	detach title,
	before upper = \tcbtitle\par\smallskip,
	coltitle = mypropfr,
	fonttitle = \bfseries\sffamily,
	description font = \mdseries,
	separator sign none,
	segmentation style={solid, mypropfr},
}
{th}


%================================
% CLAIM
%================================

\tcbuselibrary{theorems,skins,hooks}
\newtcbtheorem[use counter=tcb@cnt@Theorem]{claim}{Claim}
{%
	enhanced
	,breakable
	,colback = myg!10
	,frame hidden
	,boxrule = 0sp
	,borderline west = {2pt}{0pt}{myg}
	,sharp corners
	,detach title
	,before upper = \tcbtitle\par\smallskip
	,coltitle = myg!85!black
	,fonttitle = \bfseries\sffamily
	,description font = \mdseries
	,separator sign none
	,segmentation style={solid, myg!85!black}
}
{th}



%================================
% Exercise
%================================

\tcbuselibrary{theorems,skins,hooks}
\newtcbtheorem[use counter=tcb@cnt@Theorem]{Exercise}{Exercice}
{%
	enhanced,
	breakable,
	colback = myexercisebg,
	frame hidden,
	boxrule = 0sp,
	borderline west = {2pt}{0pt}{myexercisefg},
	sharp corners,
	detach title,
	before upper = \tcbtitle\par\smallskip,
	coltitle = myexercisefg,
	fonttitle = \bfseries\sffamily,
	description font = \mdseries,
	separator sign none,
	segmentation style={solid, myexercisefg},
}
{th}

%================================
% EXAMPLE BOX
%================================

\newtcbtheorem[use counter=tcb@cnt@Theorem]{Example}{Exemple}
{%
	colback = myexamplebg
	,breakable
	,colframe = myexamplefr
	,coltitle = myexampleti
	,boxrule = 1pt
	,sharp corners
	,detach title
	,before upper=\tcbtitle\par\smallskip
	,fonttitle = \bfseries
	,description font = \mdseries
	,separator sign none
	,description delimiters parenthesis
}
{ex}

%================================
% DEFINITION BOX
%================================

\newtcbtheorem[use counter=tcb@cnt@Theorem]{Definition}{Définition}{enhanced,
	before skip=2mm,after skip=2mm, colback=red!5,colframe=red!80!black,boxrule=0.5mm,
	attach boxed title to top left={xshift=1cm,yshift*=1mm-\tcboxedtitleheight}, varwidth boxed title*=-3cm,
	boxed title style={frame code={
					\path[fill=tcbcolback]
					([yshift=-1mm,xshift=-1mm]frame.north west)
					arc[start angle=0,end angle=180,radius=1mm]
					([yshift=-1mm,xshift=1mm]frame.north east)
					arc[start angle=180,end angle=0,radius=1mm];
					\path[left color=tcbcolback!60!black,right color=tcbcolback!60!black,
						middle color=tcbcolback!80!black]
					([xshift=-2mm]frame.north west) -- ([xshift=2mm]frame.north east)
					[rounded corners=1mm]-- ([xshift=1mm,yshift=-1mm]frame.north east)
					-- (frame.south east) -- (frame.south west)
					-- ([xshift=-1mm,yshift=-1mm]frame.north west)
					[sharp corners]-- cycle;
				},interior engine=empty,
		},
	fonttitle=\bfseries,
	title={#2},#1}{def}

%================================
% Solution BOX
%================================

\makeatletter
\newtcbtheorem[use counter=tcb@cnt@Theorem]{question}{Question}{enhanced,
	breakable,
	colback=white,
	colframe=myb!80!black,
	attach boxed title to top left={yshift*=-\tcboxedtitleheight},
	fonttitle=\bfseries,
	title={#2},
	boxed title size=title,
	boxed title style={%
			sharp corners,
			rounded corners=northwest,
			colback=tcbcolframe,
			boxrule=0pt,
		},
	underlay boxed title={%
			\path[fill=tcbcolframe] (title.south west)--(title.south east)
			to[out=0, in=180] ([xshift=5mm]title.east)--
			(title.center-|frame.east)
			[rounded corners=\kvtcb@arc] |-
			(frame.north) -| cycle;
		},
	#1
}{def}
\makeatother

%================================
% SOLUTION BOX
%================================

\makeatletter
\newtcolorbox{solution}{enhanced,
	breakable,
	colback=white,
	colframe=myg!80!black,
	attach boxed title to top left={yshift*=-\tcboxedtitleheight},
	title=Solution,
	boxed title size=title,
	boxed title style={%
			sharp corners,
			rounded corners=northwest,
			colback=tcbcolframe,
			boxrule=0pt,
		},
	underlay boxed title={%
			\path[fill=tcbcolframe] (title.south west)--(title.south east)
			to[out=0, in=180] ([xshift=5mm]title.east)--
			(title.center-|frame.east)
			[rounded corners=\kvtcb@arc] |-
			(frame.north) -| cycle;
		},
}
\makeatother

%================================
% Question BOX
%================================

\makeatletter
\newtcbtheorem[use counter=tcb@cnt@Theorem]{qstion}{Question}{enhanced,
	breakable,
	colback=white,
	colframe=mygr,
	attach boxed title to top left={yshift*=-\tcboxedtitleheight},
	fonttitle=\bfseries,
	title={#2},
	boxed title size=title,
	boxed title style={%
			sharp corners,
			rounded corners=northwest,
			colback=tcbcolframe,
			boxrule=0pt,
		},
	underlay boxed title={%
			\path[fill=tcbcolframe] (title.south west)--(title.south east)
			to[out=0, in=180] ([xshift=5mm]title.east)--
			(title.center-|frame.east)
			[rounded corners=\kvtcb@arc] |-
			(frame.north) -| cycle;
		},
	#1
}{def}
\makeatother

\newtcbtheorem[number within=chapter]{wconc}{Wrong Concept}{
	breakable,
	enhanced,
	colback=white,
	colframe=myr,
	arc=0pt,
	outer arc=0pt,
	fonttitle=\bfseries\sffamily\large,
	colbacktitle=myr,
	attach boxed title to top left={},
	boxed title style={
			enhanced,
			skin=enhancedfirst jigsaw,
			arc=3pt,
			bottom=0pt,
			interior style={fill=myr}
		},
	#1
}{def}



%================================
% NOTE BOX
%================================

\usetikzlibrary{arrows,calc,shadows.blur}
\tcbuselibrary{skins}
\newtcolorbox{note}[1][]{%
	enhanced jigsaw,
	colback=gray!20!white,%
	colframe=gray!80!black,
	size=small,
	boxrule=1pt,
	title=\colorbox{white!100}{\textbf{ Remarque }},
	halign title=flush center,
	coltitle=black,
	breakable,
	drop shadow=black!50!white,
	attach boxed title to top left={xshift=1cm,yshift=-\tcboxedtitleheight/2,yshifttext=-\tcboxedtitleheight/2},
	minipage boxed title=2.6cm,
	boxed title style={%
			colback=white,
			size=fbox,
			boxrule=1pt,
			boxsep=2pt,
			underlay={%
					\coordinate (dotA) at ($(interior.west) + (-0.5pt,0)$);
					\coordinate (dotB) at ($(interior.east) + (0.5pt,0)$);
					\begin{scope}
						\clip (interior.north west) rectangle ([xshift=3ex]interior.east);
						\filldraw [white, blur shadow={shadow opacity=60, shadow yshift=-.75ex}, rounded corners=2pt] (interior.north west) rectangle (interior.south east);
					\end{scope}
					\begin{scope}[gray!80!black]
						\fill (dotA) circle (2pt);
						\fill (dotB) circle (2pt);
					\end{scope}
				},
		},
	#1,
}

%================================
% STRATÉGIE BOX
%================================

\usetikzlibrary{arrows,calc,shadows.blur}
\tcbuselibrary{skins}
\newtcolorbox{strategy}[1][]{%
	enhanced jigsaw,
	colback=myb!20!white,%
	colframe=gray!80!black,
	size=small,
	boxrule=1pt,
	title=\colorbox{white!100}{\textbf{ Stratégie }},
	halign title=flush center,
	coltitle=black,
	breakable,
	drop shadow=black!50!white,
	attach boxed title to top left={xshift=1cm,yshift=-\tcboxedtitleheight/2,yshifttext=-\tcboxedtitleheight/2},
	minipage boxed title=2.5cm,
	boxed title style={%
			colback=white,
			size=fbox,
			boxrule=1pt,
			boxsep=2pt,
			underlay={%
					\coordinate (dotA) at ($(interior.west) + (-0.5pt,0)$);
					\coordinate (dotB) at ($(interior.east) + (0.5pt,0)$);
					\begin{scope}
						\clip (interior.north west) rectangle ([xshift=3ex]interior.east);
						\filldraw [white, blur shadow={shadow opacity=60, shadow yshift=-.75ex}, rounded corners=2pt] (interior.north west) rectangle (interior.south east);
					\end{scope}
					\begin{scope}[gray!80!black]
						\fill (dotA) circle (2pt);
						\fill (dotB) circle (2pt);
					\end{scope}
				},
		},
	#1,
}

%================================
% MÉTHODE BOX
%================================

\usetikzlibrary{arrows,calc,shadows.blur}
\tcbuselibrary{skins}
\newtcolorbox{methode}[1][]{%
	enhanced jigsaw,
	colback=white,%
	colframe=gray!80!black,
	size=small,
	boxrule=1pt,
	title=\textbf{Méthode},
	halign title=flush center,
	coltitle=black,
	breakable,
	drop shadow=black!50!white,
	attach boxed title to top left={xshift=1cm,yshift=-\tcboxedtitleheight/2,yshifttext=-\tcboxedtitleheight/2},
	minipage boxed title=2.5cm,
	boxed title style={%
			colback=white,
			size=fbox,
			boxrule=1pt,
			boxsep=2pt,
			underlay={%
					\coordinate (dotA) at ($(interior.west) + (-0.5pt,0)$);
					\coordinate (dotB) at ($(interior.east) + (0.5pt,0)$);
					\begin{scope}
						\clip (interior.north west) rectangle ([xshift=3ex]interior.east);
						\filldraw [white, blur shadow={shadow opacity=60, shadow yshift=-.75ex}, rounded corners=2pt] (interior.north west) rectangle (interior.south east);
					\end{scope}
					\begin{scope}[gray!80!black]
						\fill (dotA) circle (2pt);
						\fill (dotB) circle (2pt);
					\end{scope}
				},
		},
	#1,
}

%%%%%%%%%%%%%%%%%%%%%%%%%%%%%%%%%%%%%%%%%%%
% TABLE OF CONTENTS
%%%%%%%%%%%%%%%%%%%%%%%%%%%%%%%%%%%%%%%%%%%

\usepackage{tikz}

\definecolor{doc}{RGB}{0,60,110}
\usepackage{titletoc}
\contentsmargin{0cm}
\titlecontents{chapter}[3.7pc]
{\addvspace{30pt}%
	\begin{tikzpicture}[remember picture, overlay]%
		\draw[fill=doc!60,draw=doc!60] (-7,-.1) rectangle (-0.2,.6);%
		\pgftext[left,x=-3.5cm,y=0.2cm]{\color{white}\Large\sc\bfseries Chapitre\ \thecontentslabel};%
	\end{tikzpicture}\color{doc!60}\large\sc\bfseries}%
{}
{}
{\;\titlerule\;\large\sc\bfseries Page \thecontentspage
	\begin{tikzpicture}[remember picture, overlay]
		\draw[fill=doc!60,draw=doc!60] (2pt,0) rectangle (4,0.1pt);
	\end{tikzpicture}}%
\titlecontents{section}[3.7pc]
{\addvspace{2pt}}
{\contentslabel[\thecontentslabel]{2pc}}
{}
{\hfill\small \thecontentspage}
[]
\titlecontents*{subsection}[3.7pc]
{\addvspace{-1pt}\small}
{}
{}
{\ --- \small\thecontentspage}
[ \textbullet\ ][]

\makeatletter
\renewcommand{\tableofcontents}{%
	\chapter*{%
	  \vspace*{-20\p@}%
	  \begin{tikzpicture}[remember picture, overlay]%
		  \pgftext[right,x=15cm,y=0.2cm]{\color{doc!60}\Huge\sc\bfseries \contentsname};%
		  \draw[fill=doc!60,draw=doc!60] (13,-.75) rectangle (20,1);%
		  \clip (13,-.75) rectangle (20,1);
		  \pgftext[right,x=15cm,y=0.2cm]{\color{white}\Huge\sc\bfseries \contentsname};%
	  \end{tikzpicture}}%
	\@starttoc{toc}}
\makeatother


%%%%%%%%%%%%%%%%%%%%%%%%%%%%%%%%%%%%%%%%%%%
% MINTED FOR PYTHON ALGORITHMS
%%%%%%%%%%%%%%%%%%%%%%%%%%%%%%%%%%%%%%%%%%%

\usepackage{tcolorbox}
\tcbuselibrary{minted,breakable,xparse,skins}
\definecolor{bg}{gray}{0.95}
\DeclareTCBListing{mintedbox}{O{}m!O{}}{%
  breakable=true,
  listing engine=minted,
  listing only,
  minted language=#2,
  minted style=default,
  minted options={%
    linenos,
    gobble=0,
    breaklines=true,
    breakafter=,,
    fontsize=\small,
    numbersep=8pt,
    #1},
  boxsep=0pt,
  left skip=0pt,
  right skip=0pt,
  left=25pt,
  right=0pt,
  top=3pt,
  bottom=3pt,
  arc=5pt,
  leftrule=0pt,
  rightrule=0pt,
  bottomrule=2pt,
  toprule=2pt,
  colback=bg,
  colframe=orange!70,
  enhanced,
  overlay={%
    \begin{tcbclipinterior}
    \fill[orange!20!white] (frame.south west) rectangle ([xshift=20pt]frame.north west);
    \end{tcbclipinterior}},
  #3}
  
  
 % for braces
\usetikzlibrary{decorations.pathreplacing}

%\input{adr/vars_44284.adr}
%\newcommand{\seed}{TEST}

\pagestyle{fancy}
\fancyhead[L]{Seconde 13}
\fancyhead[C]{\textbf{Devoir Maison 4 --- \seed \ifsolutions \, --- Solutions  \fi}}
\fancyhead[R]{\today}

\fakesection{Devoir \seed}

\exe{
	Considérons trois fonctions du deuxième degré définies sur $\R$.
	\begin{align*}
		f(x) = \fa x^2  \fb x \fc, && g(x) = \left(\gaI x \gbI\right)\left(\gaII x \gbII\right), && h(x) = -\hbeta + (\ha x \hb)^2.
	\end{align*}
	
	\begin{enumerate}
		\item Montrer que $g = f$ en développant $g(x)$.
		\item Remplir le tableau de signes de $f$ à l'aide de l'expression de $g$.
		\item Montrer que $h = f$ en développant $h(x)$.
		%\item Remplir le tableau de variations de $f$ à l'aide de l'expression de $h$.
		\item Donner le minimum de $f$ sur $\R$ et l'antécédent $x^\star \in \R$ qui le réalise.
	\end{enumerate}

	\begin{center}
	\begin{tikzpicture}
		\tkzTabInit
		 [lgt=3,espcl=4]
	       		{$x$ / 1, \ifsolutions Signe de $\gaI x \gbI$ \fi /1, \ifsolutions Signe de $\gaII x \gbII$ \fi/1,  Signe de $f(x)$ / 1}
			{$\minfty$,\ifsolutions $\xzero$ \fi,\ifsolutions $\xone$ \fi,$\pinfty$}
		\ifsolutions
                \tkzTabLine
                        {,-,z,+,,+}
                \tkzTabLine
                        {,-,,-,z,+}
                \tkzTabLine
                        {,+,z,-,z,+,}
                \fi
	\end{tikzpicture}
	\end{center}

}{
	L'axiome de distributivité
		\[ a (b+c) = ab + ac \]
	implique la propriété de double distributivité :
		\[ (a+b)(c+d) = a(c+d) + b(c+d) = ac + ad + bc + bd. \]

	\begin{enumerate}
		\item
		On peut soit utiliser l'identité remarquable $(a+b)(a-b) = a^2 - b^2$ pour faire disparaître immédiatement la racine, ou alors on distribue comme suit.
			\[ g(x) = (\gaI x) \cdot (\gaII x) + (\gaI x) \cdot (\gbII) +  (\gbI) \cdot (\gaII x) + (\gbI) \cdot (\gbII) \]
		Les puissances de $x$ ne se mélangent pas, on les regroupe donc :
			\[ g(x) = (\gaI\cdot\gaII)x^2 + \left[(\gaI)\cdot(\gbII)+(\gbI)\cdot(\gaII)\right]x + (\gbI) \cdot (\gbII). \]
		On retrouve bien 
			\[ g(x) = \fa x^2 \fb x \fc = f(x) \]
		en développant le coefficient constant et car les racines carrées s'annulent agréablement.

		\item
		$f(x)$ est produit de deux fonctions affines : on étudie le signe de chacune pour en déduire le signe du produit.
		En outre, les deux fonctions affines ont pour coefficient directeur $\gaI > 0$ et sont donc croissantes.
		Un dessin rapide permet de se rappeler des règles de signes ; une fonction croissante est négative, puis nulle en sa racine, puis positive.
		On calcule où chaque fonction s'annule en posant 
			\begin{align*}
				\gaI x \gbI = 0, && \text{ et } && \gaII x \gbII = 0. \\
				x = \xzero, && && x=\xone.
			\end{align*}
		On complète le tableau en faisant bien attention à mettre les racines dans l'ordre croissant.

		\item
		L'identité remarquable $(a+b)^2 = a^2 + b^2 + 2ab$ découle bien sûr de l'axiome de distributivité :
			\[ (a+b)^2 = (a+b)(a+b) = a(a+b) + b(a+b) = a^2 + ab + ba + b^2 = a^2 + b^2 + 2ab. \]
		On l'utilise ici pour développer $(\ha x \hb)^2$ puis ajouter $-\hbeta$.
			\begin{align*}
				-\hbeta + (\ha x \hb)^2 &= -\hbeta + (\ha x)^2 + (\hb)^2 + 2(\ha x)(\hb) \\
							&= \fa x^2 \fb x + (\hb)^2 - \hbeta \\
							&= f(x)
			\end{align*}

		\item
		On répète ce qui a été vu en cours : on part systématiquement du fait qu'un carré est toujours positif, et on crée $f(x)$ :
			\begin{align*}
				(\ha x \hb)^2 &\geq 0, \\
				-\hbeta + (\ha x \hb)^2 &\geq \hbeta, \\
				f(x) &\geq -\hbeta.
			\end{align*}
		On a donc $f(x) \geq -\hbeta$ pour tout $x\in\R$. 
		Avec du recul, c'est assez logique : calculer $h(x)$ revient à prendre $-\hbeta$ et à ajouter quelque chose de positif.
		De plus, $f(x^\star) = -\hbeta$ si et seulement si le carré ajouté est nul, donc si et seulement si
			\[ \ha x^\star \hb = 0. \]
		On résoud pour trouver $x^\star = \xstar$ qui réalise bien le minimum car $f(x) \geq f(x^\star)$ pour tout $x\in\R$.

		Remarquons qu'on a utilisé $E^2 = 0 \iff E = 0$ ici. En effet, si le produit $E^2 = E \times E$ est nul, forcément $E$ ou $E$ est nul...
		Soyons vigilant sur le fait qu'en général $E^2 = a$ n'implique pas $E = \sqrt{a}$ mais $|E| = \sqrt{a}$ (et donc $E = \sqrt{a}$ ou $-\sqrt{a}$).
}

\exe{
	Construire, en posant des nombres réels $a, b, c\in\R$, une fonction de la forme
			\[ f(x) = ax^2 + bx + c \]
	telle que $-3$ soit le maximum de $f$ sur $\R$, atteint en $x^\star = -1$.
}{
	On souhaite une fonction $f$ dont le maximum est $-3$.
	On considère donc une fonction $f(x)$ qui soustrait toujours quelque chose de positif à $-3$, par exemple
		\[ F(x) = -3 - x^2. \]
	Le problème est que le maximum est n'est pas atteint en $-1$ mais en $0$ ici, il faut donc mettre au carré une expression qui s'annule en $-1$.
	Comme vu en cours, $x+1$ est une telle fonction, et 
		\[ f(x) = -3 - (x+1)^2 = 3 - (x^2 + 2x + 1) = 3 - x^2 - 2x - 1 = -x^2 - 2x + 2 \]
	fonctionne très bien !
	Notons qu'on aurait pû aussi choisir
		\begin{align*}
			g(x) = -3 -2(x+1)^2,&& \text{ ou }&& h(x) = -3 - 160(x+1)^2.
		\end{align*}
	La plus simple étant $f(x) = -x^2 - 2x + 2$, on pose $a=-1, b=-2,$ et $c=2$.
}

\exe{
	On souhaite connaître les solutions de l'équation quartique d'inconnue $x\in\R$ :
		\begin{align}
			\eqIa x^4 + \eqIb x^2 \eqIc = 0. \label{eq:1}
		\end{align}
	\begin{enumerate}
		\item Montrer que $\eqIa x^4 + \eqIb x^2 \eqIc = (\aV x^2 - \bV)(\cV - \dV x^2)$.
		\item En déduire l'ensemble des solutions de l'équation \eqref{eq:1}.
		Exprimer ces solutions sous la forme $q \sqrt{n}$ où $q\in\Q$ est rationnel et $n\in\N$ est un entier naturel.
		
		\item Substituer les nombres réels obtenus dans l'équation \eqref{eq:1} pour vérifier qu'ils sont bien solutions.
	\end{enumerate}
}{
	\begin{enumerate}
		\item
		Lorsqu'une identité comme celle-ci est à démontrer, on part systématiquement de la forme factorisée, ici à droite.
		On développe donc calmement en se souvenant que $x^2 x^2 = xxxx = x^4$.
			\begin{align*}
				(\aV x^2 - \bV)(\cV - \dV x^2) &= (\aV x^2) \cdot (\cV) + (\aV x^2)\cdot(-\dV x^2) + (-\bV)\cdot(\cV) + (-\bV)\cdot(-\dV x^2) \\
								&= (\aV \cdot \cV)x^2 - (\aV \cdot \dV)x^4 - \bV \cdot \cV + (\bV\cdot\dV)x^2 \\
								&= \eqIa x^4 + (\aV\cdot\cV+\bV\cdot\dV)x^2 \eqIc \\
								&= \eqIa x^4 + \eqIb x^2 \eqIc
			\end{align*}

		\item
		On applique la propriété du produit nul vue en classe :
			\begin{align*}
				(\aV x^2 - \bV)(\cV - \dV x^2) = 0 && \iff (\aV x^2 - \bV) = 0 \qquad\text{ ou }\qquad (\cV - \dV x^2) = 0.
			\end{align*}
		On a donc
			\begin{align*}
				x^2 = \frac{\bV}{\aV} && \text{ ou } && x^2 = \dfrac{\cV}{\dV}.
			\end{align*}
		On fait bien attention au fait que $\sqrt{x^2} = |x| \neq x$, et donc au fait qu'il y ait toujours deux solutions à $x^2 = a$ : $\sqrt{a}$ et $-\sqrt{a}$.
		On aura donc ici quatre solutions : 
			\[ x \in \left\{  \sqrt{\frac{\bV}{\aV}} ; -\sqrt{\frac{\bV}{\aV}} ; \sqrt{\frac{\cV}{\dV}} ; -\sqrt{\frac{\cV}{\dV}} \right\}. \]
		Pour exprimer les solutions sous la forme $q\sqrt{n}$ avec $n\in\N$ entier, on reprend les propriétés des racines vues en cours.
		Par exemple, 
			\[ \sqrt{\frac{\bV}{\aV}} = \sqrt{\frac{\bV \cdot \aV}{\aV^2}} = \dfrac{1}{\aV} \sqrt{\bV\cdot\aV}. \]
		On peut ensuite, si voulu, réduire la racine carrée au maximum en extrayant les carrés parfaits comme vu en cours.

		\item
		Substituer signifie remplacer $x$ par chacune des valeurs solutions trouvées pour vérifier qu'elles vérifient bien l'équation \eqref{eq:1}.
		C'est calculatoire mais cela permet de vérifier qu'on ait bien trouvé des solutions (et qu'on ait bien compris ce que $x^4$ signifie).
	\end{enumerate}
}

\exe{
	Construire, en posant des nombres réels $a, b, c\in\R$, une équation de la forme
		\begin{align}
			ax^2 + bx + c = 0 \label{eq:2}
		\end{align}
	telle que l'ensemble des $x\in\R$ solutions de l'équation \eqref{eq:2} soit $\left\{ \qI ; \qII \right\}$.
	
	Une fois $a, b$, et $c$ posés, substituer $\qI$ et $\qII$ dans l'équation obtenue pour vérifier qu'ils en sont bien solutions.
}{
	Comme vu en cours, la fonction 
		\[ f(x) = \left(x \mqI\right)\left(x \mqII\right) \]
	s'annule en $\qI$ est en $\qII$.
	On développe l'expression pour obtenir
		\[ f(x) = x^2 \sumIV x \prodIV. \]
	On pose donc $a = 1, b=\bIV, c=\cIV$
}

\end{document}

% DYSLEXIA SWITCH
\newif\ifdys
		
				% ENABLE or DISABLE font change
				% use XeLaTeX if true
				\dystrue
				\dysfalse


\ifdys

\documentclass[a4paper, 14pt]{extarticle}
\usepackage{amsmath,amsfonts,amsthm,amssymb,mathtools}

\tracinglostchars=3 % Report an error if a font does not have a symbol.
\usepackage{fontspec}
\usepackage{unicode-math}
\defaultfontfeatures{ Ligatures=TeX,
                      Scale=MatchUppercase }

\setmainfont{OpenDyslexic}[Scale=1.0]
\setmathfont{Fira Math} % Or maybe try KPMath-Sans?
\setmathfont{OpenDyslexic Italic}[range=it/{Latin,latin}]
\setmathfont{OpenDyslexic}[range=up/{Latin,latin,num}]

\else

\documentclass[a4paper, 12pt]{extarticle}
\usepackage{amsmath,amsfonts,amsthm,amssymb,mathtools}

\fi


\usepackage[french]{babel}
\usepackage[
a4paper,
margin=2cm,
nomarginpar,% We don't want any margin paragraphs
]{geometry}
\usepackage{fancyhdr}
\usepackage{array}
\usepackage{amsmath,amsfonts,amsthm,amssymb,mathtools,}
\newcolumntype{P}[1]{>{\centering\arraybackslash}p{#1}}

\usepackage{enumitem, multicol}

\usepackage{stackengine}
\newcommand\xrowht[2][0]{\addstackgap[.5\dimexpr#2\relax]{\vphantom{#1}}}

% theorems

\theoremstyle{plain}
\newtheorem{theorem}{Th\'eor\`eme}
\newtheorem*{sol}{Solution}
\theoremstyle{definition}
\newtheorem{ex}{Exercice}
\newtheorem{definition}{Définition}


% corps
\newcommand{\C}{\mathbb{C}}
\newcommand{\R}{\mathbb{R}}
\newcommand{\Rnn}{\mathbb{R}^{2n}}
\newcommand{\Z}{\mathbb{Z}}
\newcommand{\N}{\mathbb{N}}
\newcommand{\Q}{\mathbb{Q}}

% domain
\newcommand{\D}{\mathbb{D}}



% plots
\usepackage{pgfplots}

% for calligraphic C
\usepackage{calrsfs}

% euro
\usepackage{lmodern,textcomp}

% ensembles tq. 
\newcommand{\xRtq}[1]{
	$\left\{ x \in \R \text{ tq. } #1 \right\}$
}


% ensembles tq. 
\newcommand{\vabs}[1]{
	\left| #1 \right|
}


%pinfty minfty
\newcommand{\pinfty}{{+}\infty}
\newcommand{\minfty}{{-}\infty}

% SOLUTION SWITCH
\newif\ifsolutions
				\solutionstrue
				\solutionsfalse
				% LES SOLUTIONS SONT FAUSSES (DE L'AUTRE ÉVAL)

\ifsolutions
	\newcommand{\exe}[2]{
		\begin{ex} #1  \end{ex}
		\begin{sol} #2 \end{sol}
	}
\else
	\newcommand{\exe}[2]{
		\begin{ex} #1  \end{ex}
	}
	
\fi



% date
\usepackage{advdate}
\AdvanceDate[1]


\begin{document}
\pagestyle{fancy}
\fancyhead[L]{Seconde 13}
\fancyhead[C]{\textbf{Évaluation -- Droite réelle \ifsolutions -- Solutions \fi}}
\fancyhead[R]{\today}

\subsection*{Intervalles et inégalités (8pts)}

\exe{
	Écrire les ensembles suivants sous forme d'intervalle.
	
	\begin{multicols}{2}
	\begin{enumerate}
		\item \xRtq{x\leq-3}
		\item \xRtq{{-}124< x}
		\item \xRtq{{-}\dfrac1{12} \leq x < \dfrac{21}{13}}
		\item \xRtq{18 \geq x > -11}
	\end{enumerate}
	\end{multicols}
}
{

	\begin{multicols}{2}
	\begin{enumerate}
		\item $\left]-\infty; 3\right]$
		\item $\left]-\infty ; -7\right[$
		\item $\left] -\infty ; 31\right]$
		\item $\left]-12; +\infty\right[$
		\item $\left[-\dfrac13 ; \dfrac{22}7 \right[$
		\item $\left]-13 ; 9 \right]$
	\end{enumerate}
	\end{multicols}

}


\exe{
	Écrire les ensembles suivants sous forme d'intervalle.
	
	\begin{multicols}{2}
	\begin{enumerate}
		\item \xRtq{x + 7 \leq 17}
		\item \xRtq{8x  - 14 \leq 10}
		\item \xRtq{5 \geq -3x + 8}
		\item \xRtq{-4 < 5x + 6 < 16}
	\end{enumerate}
	\end{multicols}
}
{


	\begin{multicols}{2}
	\begin{enumerate}
		\item $\left]-\infty ; 5\right]$
		\item $\left]6 ; +\infty\right[$
		\item $\left]-\infty; 3 \right]$
		\item $\left[1 ; +\infty \right[$
		\item $\left] -\dfrac{10}{13} ; \dfrac{7}{13} \right[$
		\item $\left[ -5 ; 3 \right]$\
	\end{enumerate}
	\end{multicols}
}

\subsection*{Union et intersection (6pts)}

\exe{
	Donner les ensembles ou intervalles correspondants aux unions suivantes.
	\begin{multicols}{2}
	\begin{enumerate}
		\item $\{ 3 ; 5 ; -0 ; -2 \} \cup \{  1; 4 ; -2 ; 3\}$
		\item $[0 ; \pinfty[ \cup ]4 ; \pinfty[$
	\end{enumerate}
	\end{multicols}
}
{

	\begin{multicols}{2}
	\begin{enumerate}
		\item $\{3;5;1;-2;1;3\}$
		\item $[{-}2 ; \pinfty[$
		\item $ ]\minfty ; 12[$
		\item $]{-3};7[ \cup ]8; 13[$
	\end{enumerate}
	\end{multicols}

}


\exe{
	Donner les ensembles ou intervalles correspondants aux intersections suivantes.
	\begin{multicols}{2}
	\begin{enumerate}
		\item $\{ 3 ; 5 ; 0 ; -2 \} \cap \{  2; 4 ; -2 ; 3\}$
		\item $ ]\minfty ; 4[ \cap ]\minfty ; 3{,}99]$
		\item $]{-6};18[ \bigcap \left]\dfrac32; 10^6\right[ \bigcap ]{-}11; 10[$
	\end{enumerate}
	\end{multicols}
}
{

	\begin{multicols}{2}
	\begin{enumerate}
		\item $\{ 1 ; 2 ; -4\}$
		\item $]-2; 2[$
		\item $]\minfty ; 11{,}99]$
		\item $\left]\dfrac72 ; 10\right[$
	\end{enumerate}
	\end{multicols}

}


\subsection*{Valeurs absolues (6pts)}


\exe{
	À  quoi ces valeurs absolues sont-elles égales ?
	\begin{multicols}{2}
	\begin{enumerate}
		\item $\vabs{7}$
		\item $\vabs{14 - 5}$
		\item $\vabs{5-14}$
		\item $\vabs{\dfrac14 - \dfrac13}$
	\end{enumerate}
	\end{multicols}
}
{

	\begin{multicols}{2}
	\begin{enumerate}
		\item $3$
		\item $11$
		\item $5$
		\item $5$
		\item $\dfrac16$
		\item $\dfrac{11}{14}$
	\end{enumerate}
	\end{multicols}

}

\exe{
	Écrire les ensembles suivants sous forme d'ensemble fini, d'intervalle, ou d'union d'intervalles.
	\begin{multicols}{2}
	\begin{enumerate}
		\item \xRtq{\vabs{x} = 9}
		\item \xRtq{\vabs{1-3x} = 9}
		\item \xRtq{\vabs{2x + 3} \leq 9}
	\end{enumerate}
	\end{multicols}
}
{

	\begin{multicols}{2}
	\begin{enumerate}
		\item $\{ -4 ; 4\}$
		\item $ \{ -\dfrac43 ; 2 \}$
		\item $\left[ -\dfrac72 ; \dfrac12\right]$
		\item $\left]-\infty ; -2\right] \cup \left[ \dfrac{14}3 ; +\infty \right[$
	\end{enumerate}
	\end{multicols}

}

\subsection*{Bonus (2pts)}

\exe{
	Donner l'ensemble des $x\in\R$ réels vérifiant
		\[ |7 - x| = |3x + 3|. \]
}
{
	Comme la valeur absolue d'une expression vaut soit l'expression, soit son opposé, on a les possibilités suivantes : $10-x = 2x+4$ ou $10-x = -2x-4$ (il y a en fait deux autres possibilités, mais redondantes !).
	La résolution des équations et leur vérification donne les valeurs suivantes :
	$\{ 2 ; -14 \}$.
}



\end{document}
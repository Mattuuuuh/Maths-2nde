% DYSLEXIA SWITCH
\newif\ifdys
		
				% ENABLE or DISABLE font change
				% use XeLaTeX if true
				\dystrue
				\dysfalse


\ifdys

\documentclass[a4paper, 14pt]{extarticle}
\usepackage{amsmath,amsfonts,amsthm,amssymb,mathtools}

\tracinglostchars=3 % Report an error if a font does not have a symbol.
\usepackage{fontspec}
\usepackage{unicode-math}
\defaultfontfeatures{ Ligatures=TeX,
                      Scale=MatchUppercase }

\setmainfont{OpenDyslexic}[Scale=1.0]
\setmathfont{Fira Math} % Or maybe try KPMath-Sans?
\setmathfont{OpenDyslexic Italic}[range=it/{Latin,latin}]
\setmathfont{OpenDyslexic}[range=up/{Latin,latin,num}]

\else

\documentclass[a4paper, 12pt]{extarticle}
\usepackage{amsmath,amsfonts,amsthm,amssymb,mathtools}

\fi


\usepackage[french]{babel}
\usepackage[
a4paper,
margin=2cm,
nomarginpar,% We don't want any margin paragraphs
]{geometry}
\usepackage{fancyhdr}
\usepackage{array}
\usepackage{amsmath,amsfonts,amsthm,amssymb,mathtools,}
\newcolumntype{P}[1]{>{\centering\arraybackslash}p{#1}}

\usepackage{enumitem, multicol}

\usepackage{stackengine}
\newcommand\xrowht[2][0]{\addstackgap[.5\dimexpr#2\relax]{\vphantom{#1}}}

% theorems

\theoremstyle{plain}
\newtheorem{theorem}{Th\'eor\`eme}
\newtheorem*{sol}{Solution}
\theoremstyle{definition}
\newtheorem{ex}{Exercice}
\newtheorem{definition}{Définition}


% corps
\newcommand{\C}{\mathbb{C}}
\newcommand{\R}{\mathbb{R}}
\newcommand{\Rnn}{\mathbb{R}^{2n}}
\newcommand{\Z}{\mathbb{Z}}
\newcommand{\N}{\mathbb{N}}
\newcommand{\Q}{\mathbb{Q}}

% domain
\newcommand{\D}{\mathbb{D}}



% plots
\usepackage{pgfplots}

% for calligraphic C
\usepackage{calrsfs}

% euro
\usepackage{lmodern,textcomp}

% x dans R tq. 
\newcommand{\xRtq}[1]{
	$\left\{ x \in \R \text{ tq. } #1 \right\}$
}


% vabs
\newcommand{\vabs}[1]{
	\left| #1 \right|
}

% intervalles fermés
\newcommand{\inter}[2]{
	$\left[ #1 ; #2 \right]$
}

% point plan
\newcommand{\point}[3]{
	#1\left(#2 ; #3 \right)
}


%pinfty minfty
\newcommand{\pinfty}{{+}\infty}
\newcommand{\minfty}{{-}\infty}

% SOLUTION SWITCH
\newif\ifsolutions
				\solutionstrue
				%\solutionsfalse

\ifsolutions
	\newcommand{\exe}[3]{
		\begin{ex}[#3] #1  \end{ex}
		\begin{sol} #2 \end{sol}
	}
\else
	\newcommand{\exe}[3]{
		\begin{ex}[#3] #1  \end{ex}
	}
	
\fi

% date
\usepackage{advdate}
\AdvanceDate[1]


\begin{document}
\pagestyle{fancy}
\fancyhead[L]{Seconde 13}
\fancyhead[C]{\textbf{Évaluation — Droite et plan \ifsolutions -- Solutions \fi}}
\fancyhead[R]{\today}

\section*{Droite réelle (8pts)}

\exe{
	Donner le milieu de chaque intervalle suivant.
	
	\begin{multicols}{2}
	\begin{enumerate}
		\item \inter{-1}{1}
		\item \inter{\dfrac{99}{200}}{\dfrac{101}{200}}
		\item \inter{-10^8 - 2}{10^8}
		%\item \inter{-\dfrac43}{\dfrac83}
	\end{enumerate}
	\end{multicols}
}{

	\begin{multicols}{2}
	\begin{enumerate}
		\item $0$
		\item $\dfrac12$
		\item $-1$
		\item $\dfrac23$
	\end{enumerate}
	\end{multicols}

}{}

\exe{
	Donner la longueur de chaque intervalle suivant.
	
	\begin{multicols}{2}
	\begin{enumerate}
		\item \inter{-1}{1}
		\item \inter{\dfrac{99}{200}}{\dfrac{101}{200}}
		\item \inter{10^{40}-1}{10^{40}+1}
		%\item \inter{-\dfrac43}{\dfrac83}
	\end{enumerate}
	\end{multicols}
}{

	\begin{multicols}{2}
	\begin{enumerate}
		\item $2$
		\item $\frac1{100}$
		\item $2$
		\item $4$
	\end{enumerate}
	\end{multicols}


}{}

\section*{Plan cartésien (12pts)}

\exe{
	Tracer un repère et y placer les points suivants. Il est recommandé de calculer les coordonnées des points à placer avant de tracer le repère.
	\begin{multicols}{2}
	\begin{enumerate}
		\item $\point{A}{2}{3}$
		\item $\point{B}{-1}{3}$
		\item $C = A+B$
		\item $D=A-B$
		\item $E=\frac12A$
		\item $F=-3B$
	\end{enumerate}
	\end{multicols}
}{ \, \\
		\begin{tikzpicture}[>=stealth, scale=1.5]
		\begin{axis}[xmin = -1.9, xmax=4.9, ymin=-9.9, ymax=7.9, ytick = {-9, ..., 7}, axis x line=middle, axis y line=middle, axis line style=<->, xlabel={}, ylabel={}, grid=both]
						
			\addplot[red, mark=*, mark size = 1] (2,3) node[right] {$A$};
			\addplot[red, mark=*, mark size = 1] (-1,3) node[left] {$B$};
			\addplot[red, mark=*, mark size = 1] (1,6) node[below] {$C$};
			\addplot[red, mark=*, mark size = 1] (3,0) node[above] {$D$};
			\addplot[red, mark=*, mark size = 1] (1,1.5) node[left] {$E$};
			\addplot[red, mark=*, mark size = 1] (3,-9) node[right] {$F$};
			
		\end{axis}
		\end{tikzpicture}
}{}

\exe{
	Considérons les points
		\begin{align*}
			\point{A}{1}{1}, && \point{B}{3}{1}, && \point{C}{2}{\sqrt{3}+1}.
		\end{align*}
	Démontrer que le triangle $ABC$ est équilatéral en calculant la longueur de chaque côté.
	
	\noindent
	\emph{Rappel : un triangle équilatéral est un triangle dont les trois côtés ont la même longueur}.
}{
	On calcule 
		\begin{align*}
			AB = \sqrt{2^2 + 0^2} = 2, && AC = \sqrt{1^2 + \sqrt{3}^2} = \sqrt{1+3} = 2, && BC = \sqrt{4} = 2.
		\end{align*}

}{}

\exe{
	Considérons les points
		\begin{align*}
			\point{A}{0}{1}, && \point{B}{-3}{0}, && \point{C}{1}{-2}, && \point{D}{-2}{-3}.
		\end{align*}
	Démontrer que le quadrilatère $BACD$ est un parallélogramme en comparant le milieu de ses deux diagonales.
	
	\noindent
	\emph{Rappel : un parallélogramme est un quadrilatère dont les diagonales se coupent en leur milieu}.
}{

		Le milieu du segment $[BC]$ est donné par
			\[ \dfrac12 (B+C) = (-1;-1),\]
		et le milieu du segment $[AD]$ est donné par
			\[ \dfrac12 (A+D) = (-1;-1).\]
		Le quadrilatère est donc bien un parallélogramme.

		\begin{tikzpicture}[>=stealth, scale=1.5]
		\begin{axis}[xmin = -3.9, xmax=1.9, ymin=-3.9, ymax=1.9, axis x line=middle, axis y line=middle, axis line style=<->, xlabel={}, ylabel={}, grid=both]
						
			\addplot[red, mark=*, mark size = 1] (0,1) node[right] {$A$};
			\addplot[red, mark=*, mark size = 1] (-3,0) node[left] {$B$};
			\addplot[red, mark=*, mark size = 1] (1,-2) node[below] {$C$};
			\addplot[red, mark=*, mark size = 1] (-2,-3) node[below] {$D$};
			
		\end{axis}
		\end{tikzpicture}
		
}{}

%\exe{
%	Soient $\point{A}{-\dfrac32}{\dfrac53}$ et $\point{M}{-\dfrac53}{-1}$ deux points du plan.
%	
%	Quel point $C$ faut-il choisir pour que $M$ soit le milieu du segment $[AC]$ ? Donner ses coordonnées.
%}{
%	La contrainte que $M$ soit le milieu de $[AC]$ s'écrit
%		\begin{align*}
%			M = \dfrac12(A+C) && \iff && 2M = A+C && \iff && C = 2M-A
%		\end{align*}
%	Par suite,
%		\[ C= 2M-A = 2\cdot\left(-\dfrac53;-1\right) - \left(-\dfrac32;\dfrac53\right) = \left(-\dfrac{11}6; -\dfrac{11}3\right). \]
%	
%}{3pts}

%\exe{
%	Soient $\point{G}{-4}{-1}$ et $\point{D}{-1}{3}$ et $x\in\R$ un paramètre réel.
%	
%	Pour quel(s) $x\in\R$ est-ce que la longueur du segment entre les points $xG$ et $xD$ est-elle égale à $5$ ?
%}{
%
%	La longueur du segment est donnée par
%		\[ 5 = \sqrt{|-4x + x|^2 + |-x - 3x|^2} = \sqrt{9|x|^2 + 16|x|^2} = \sqrt{25|x|^2}. \]
%	En mettant l'équation au carré, on trouve
%		\begin{align*}
%			25 = 25|x|^2 && \iff &&  |x|^2 = 1 && \iff && |x| = 1
%		\end{align*}
%	D'où on trouve les solutions $x=1$ ou $x=-1$.
%}

\section*{Bonus (2pts)}

\exe{
	Soient $A(1;1), B(3;1)$ deux points du plan, et $C(2;x)$ un point dépendant d'un paramètre réel $x\in\R$.
	
	Pour quel(s) $x\in\R$ le triangle $ABC$ est-il équilatéral ?
}{}{}

\end{document}
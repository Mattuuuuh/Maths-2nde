				% ENABLE or DISABLE font change
				% use XeLaTeX if true
\newif\ifdys
				\dystrue
				\dysfalse

\newif\ifsolutions
				\solutionstrue
				\solutionsfalse

% DYSLEXIA SWITCH
\newif\ifdys
		
				% ENABLE or DISABLE font change
				% use XeLaTeX if true
				\dystrue
				\dysfalse


\ifdys

\documentclass[a4paper, 14pt]{extarticle}
\usepackage{amsmath,amsfonts,amsthm,amssymb,mathtools}

\tracinglostchars=3 % Report an error if a font does not have a symbol.
\usepackage{fontspec}
\usepackage{unicode-math}
\defaultfontfeatures{ Ligatures=TeX,
                      Scale=MatchUppercase }

\setmainfont{OpenDyslexic}[Scale=1.0]
\setmathfont{Fira Math} % Or maybe try KPMath-Sans?
\setmathfont{OpenDyslexic Italic}[range=it/{Latin,latin}]
\setmathfont{OpenDyslexic}[range=up/{Latin,latin,num}]

\else

\documentclass[a4paper, 12pt]{extarticle}

\usepackage[utf8x]{inputenc}
%fonts
\usepackage{amsmath,amsfonts,amsthm,amssymb,mathtools}
% comment below to default to computer modern
\usepackage{libertinus,libertinust1math}

\fi


\usepackage[french]{babel}
\usepackage[
a4paper,
margin=2cm,
nomarginpar,% We don't want any margin paragraphs
]{geometry}
\usepackage{icomma}

\usepackage{fancyhdr}
\usepackage{array}
\usepackage{hyperref}

\usepackage{multicol, enumerate}
\newcolumntype{P}[1]{>{\centering\arraybackslash}p{#1}}


\usepackage{stackengine}
\newcommand\xrowht[2][0]{\addstackgap[.5\dimexpr#2\relax]{\vphantom{#1}}}

% theorems

\theoremstyle{plain}
\newtheorem{theorem}{Th\'eor\`eme}
\newtheorem*{sol}{Solution}
\theoremstyle{definition}
\newtheorem{ex}{Exercice}
\newtheorem*{rpl}{Rappel}
\newtheorem{enigme}{Énigme}

% corps
\usepackage{calrsfs}
\newcommand{\C}{\mathcal{C}}
\newcommand{\R}{\mathbb{R}}
\newcommand{\Rnn}{\mathbb{R}^{2n}}
\newcommand{\Z}{\mathbb{Z}}
\newcommand{\N}{\mathbb{N}}
\newcommand{\Q}{\mathbb{Q}}

% variance
\newcommand{\Var}[1]{\text{Var}(#1)}

% domain
\newcommand{\D}{\mathcal{D}}


% date
\usepackage{advdate}
\AdvanceDate[0]


% plots
\usepackage{pgfplots}

% table line break
\usepackage{makecell}
%tablestuff
\def\arraystretch{2}
\setlength\tabcolsep{15pt}

%subfigures
\usepackage{subcaption}

\definecolor{myg}{RGB}{56, 140, 70}
\definecolor{myb}{RGB}{45, 111, 177}
\definecolor{myr}{RGB}{199, 68, 64}

% fake sections with no title to move around the merged pdf
\newcommand{\fakesection}[1]{%
  \par\refstepcounter{section}% Increase section counter
  \sectionmark{#1}% Add section mark (header)
  \addcontentsline{toc}{section}{\protect\numberline{\thesection}#1}% Add section to ToC
  % Add more content here, if needed.
}


% SOLUTION SWITCH
\newif\ifsolutions
				\solutionstrue
				%\solutionsfalse

\ifsolutions
	\newcommand{\exe}[2]{
		\begin{ex} #1  \end{ex}
		\begin{sol} #2 \end{sol}
	}
\else
	\newcommand{\exe}[2]{
		\begin{ex} #1  \end{ex}
	}
	
\fi


% tableaux var, signe
\usepackage{tkz-tab}


%pinfty minfty
\newcommand{\pinfty}{{+}\infty}
\newcommand{\minfty}{{-}\infty}

\begin{document}


\AdvanceDate[2]

\begin{document}
\pagestyle{fancy}
\fancyhead[L]{Seconde 13}
\fancyhead[C]{\textbf{Évaluation -- Fonctions affines \ifsolutions -- Solutions  \fi}}
\fancyhead[R]{\today}

\begin{theorem}[label=thm:1]{}{}
	Soit $f(x) = ax + b$ une fonction affine sur $\R$.
	Considérons $A(x_A ; y_A)$ et $B(x_B ; y_B)$ deux points distincts appartenant à $\C_f$.
	
	Alors le coefficient directeur $a$ est égal à
	\vspace{5pt}
		\[ a = .......................................... \]
	\,
\end{theorem}

\exe{[2pts]
	Compléter le théorème \ref{thm:1} vu en classe.
}{}

\exe{[4pts]
	Pour chaque fonction affine  sur $\R$ suivante, déterminer son coefficient directeur $a$ et son ordonnée à l'origine $b$.
	\begin{multicols}{2}
	\begin{enumerate}
		\item $f(x) = 2x+1$
		\item $f(x) = 1 + 2x$
		\item $f(x) = 2- \dfrac23 x $
		\item $f(x) = x$
	\end{enumerate}
	\end{multicols}
}{}

\exe{[3pts]
	Considérons une fonction affine $f$ telle que les points $(2;1)$ et $(-1 ; -3)$ appartiennent à $\C_f$.
	\begin{enumerate}
		\item Esquisser $\C_f$ sur $[-3; 3]$.
		\item Estimer \emph{graphiquement} à l'aide de $\C_f$ la valeur de l'ordonnée à l'origine de $f$.
	\end{enumerate}
}{}

\exe{[3pts]
	Considérons une fonction affine $f$ telle que les points $A(2;1)$ et $B(-1 ; -3)$ appartiennent à $\C_f$.
	
	Trouver $f$ exactement par interpolation affine.
}{}

\exe{[4pts]
	Pour chacune des paires de fonctions affines $f, g$ sur $\R$, calculer $\C_f \cap \C_g$.
	
	\begin{multicols}{2}
	\begin{enumerate}
		\item $f(x) = 5x -8, g(x) = -x+3$.
		\item $f(x) = \dfrac23 x + 1, g(x) = 5$.
		\item $f(x) = x+2, g(x) = x-5$.
		\item $f(x) = -4x+12, g(x) = 12-4x$.
	\end{enumerate}
	\end{multicols}
}{}

\exe{[4pts]
	Soit $f$ la fonction affine sur $\R$ donnée par
		\[ f(x) =3-x. \]
	\begin{enumerate}
		\item
		Déterminer l'expression algébrique de la fonction affine $g$ telle que $\C_g$ soit parallèle à $\C_f$ et passe par $(0;41)$.
		\item
		Déterminer l'expression algébrique de la fonction affine $h$ telle que $\C_h$ soit parallèle à $\C_g$ et passe par $(-3;12)$.
	\end{enumerate}
}{}

\subsection*{Bonus (2pts)}

\exe{
	Démontrer le théorème \ref{thm:1} après avoir justifié que $x_A \neq x_B$.
}{}

\end{document}
				% ENABLE or DISABLE font change
				% use XeLaTeX if true
\newif\ifdys
				\dystrue
				\dysfalse

\newif\ifsolutions
				\solutionstrue
				\solutionsfalse

% DYSLEXIA SWITCH
\newif\ifdys
		
				% ENABLE or DISABLE font change
				% use XeLaTeX if true
				\dystrue
				\dysfalse


\ifdys

\documentclass[a4paper, 14pt]{extarticle}
\usepackage{amsmath,amsfonts,amsthm,amssymb,mathtools}

\tracinglostchars=3 % Report an error if a font does not have a symbol.
\usepackage{fontspec}
\usepackage{unicode-math}
\defaultfontfeatures{ Ligatures=TeX,
                      Scale=MatchUppercase }

\setmainfont{OpenDyslexic}[Scale=1.0]
\setmathfont{Fira Math} % Or maybe try KPMath-Sans?
\setmathfont{OpenDyslexic Italic}[range=it/{Latin,latin}]
\setmathfont{OpenDyslexic}[range=up/{Latin,latin,num}]

\else

\documentclass[a4paper, 12pt]{extarticle}

\usepackage[utf8x]{inputenc}
%fonts
\usepackage{amsmath,amsfonts,amsthm,amssymb,mathtools}
% comment below to default to computer modern
\usepackage{libertinus,libertinust1math}

\fi


\usepackage[french]{babel}
\usepackage[
a4paper,
margin=2cm,
nomarginpar,% We don't want any margin paragraphs
]{geometry}
\usepackage{icomma}

\usepackage{fancyhdr}
\usepackage{array}
\usepackage{hyperref}

\usepackage{multicol, enumerate}
\newcolumntype{P}[1]{>{\centering\arraybackslash}p{#1}}


\usepackage{stackengine}
\newcommand\xrowht[2][0]{\addstackgap[.5\dimexpr#2\relax]{\vphantom{#1}}}

% theorems

\theoremstyle{plain}
\newtheorem{theorem}{Th\'eor\`eme}
\newtheorem*{sol}{Solution}
\theoremstyle{definition}
\newtheorem{ex}{Exercice}
\newtheorem*{rpl}{Rappel}
\newtheorem{enigme}{Énigme}

% corps
\usepackage{calrsfs}
\newcommand{\C}{\mathcal{C}}
\newcommand{\R}{\mathbb{R}}
\newcommand{\Rnn}{\mathbb{R}^{2n}}
\newcommand{\Z}{\mathbb{Z}}
\newcommand{\N}{\mathbb{N}}
\newcommand{\Q}{\mathbb{Q}}

% variance
\newcommand{\Var}[1]{\text{Var}(#1)}

% domain
\newcommand{\D}{\mathcal{D}}


% date
\usepackage{advdate}
\AdvanceDate[0]


% plots
\usepackage{pgfplots}

% table line break
\usepackage{makecell}
%tablestuff
\def\arraystretch{2}
\setlength\tabcolsep{15pt}

%subfigures
\usepackage{subcaption}

\definecolor{myg}{RGB}{56, 140, 70}
\definecolor{myb}{RGB}{45, 111, 177}
\definecolor{myr}{RGB}{199, 68, 64}

% fake sections with no title to move around the merged pdf
\newcommand{\fakesection}[1]{%
  \par\refstepcounter{section}% Increase section counter
  \sectionmark{#1}% Add section mark (header)
  \addcontentsline{toc}{section}{\protect\numberline{\thesection}#1}% Add section to ToC
  % Add more content here, if needed.
}


% SOLUTION SWITCH
\newif\ifsolutions
				\solutionstrue
				%\solutionsfalse

\ifsolutions
	\newcommand{\exe}[2]{
		\begin{ex} #1  \end{ex}
		\begin{sol} #2 \end{sol}
	}
\else
	\newcommand{\exe}[2]{
		\begin{ex} #1  \end{ex}
	}
	
\fi


% tableaux var, signe
\usepackage{tkz-tab}


%pinfty minfty
\newcommand{\pinfty}{{+}\infty}
\newcommand{\minfty}{{-}\infty}

\begin{document}


\AdvanceDate[4]

\begin{document}
\pagestyle{fancy}
\fancyhead[L]{Seconde 13}
\fancyhead[C]{\textbf{Évaluation : Probabilités \ifsolutions -- Solutions  \fi}}
\fancyhead[R]{\today}


%\exe{
%	Démontrer, à l'aide d'un ou plusieurs diagrammes de Venn, la formule d'inclusion-exclusion
%		\[ P(A\cup B) = P(A) + P(B) - P(A\cap B). \]
%}{}


%\exe{
%	Démontrer, à l'aide d'un ou plusieurs diagrammes de Venn, la relation d'ensembles
%		\[ \overline{A} = \left(\overline{A} \cup \overline{B} \right) \cap \left(\overline{A} \cup B \right). \]
%}{
%}


\exe{	
	On lance un grand nombre de fois un D$6$, dé à $6$ faces.
	Les résultats sont décrits dans le tableau suivant.
	\begin{center}
	\begin{tabular}{|c|c|c|c|c|c|c|} \hline
		Face & 1 & 2 & 3 & 4 & 5 & 6 \\ \hline
		Nombre de tirages & 19699 & 27549 & 9097 & 22999 & 27989 & 6970  \\ \hline
		Fréquence & & & & & & \\ \hline
	\end{tabular}
	\end{center}
	
	Compléter le tableau, modéliser la réalité en définissant un univers et une loi de probabilité, et l'utiliser pour répondre aux questions suivantes.
	\begin{enumerate}
		\item Quelle est la probabilité d'obtenir un multiple de $3$ ?
		\item Quelle est la probabilité d'obtenir exactement trois $1$ en quatre lancers consécutifs ?
		\item Quelle est la probabilité d'obtenir au moins un $3$ après dix lancers consécutifs ?
	\end{enumerate}
}{}


\exe{
	On tire une boule dans une urne contenant $3$ boules rouges et $4$ boules vertes.
	\begin{enumerate}[label=$\bullet$]
		\item Si on tire une boule rouge, on jette un D$3$ bien équilibré, dé à $3$ faces numérotées de $1$ à $3$.
		\item Si on tire une boule verte, on jette un D$5$ bien équilibré, dé à $5$ faces numérotées de $1$ à $5$.
	\end{enumerate}

	Créer un arbre de probabilité pour cette situation et répondre aux questions suivantes.
	\begin{enumerate}
		\item Quelle est la probabilité \emph{exacte} d'obtenir $4$ ? 
		\item Quelle est la probabilité \emph{exacte} d'obtenir un nombre pair ?
	\end{enumerate}

}
{}

\exe{
	Un D$3$, dé à $3$ faces numérotées de $1$ à $3$, a été pipé de telle sorte que
		\begin{itemize}[label=$\bullet$]
			\item la probabilité d'obtenir $3$ est le triple de la probabilité d'obtenir $1$ ; et
			\item la probabilité d'obtenir $2$ est le double de la probabilité d'obtenir $1$.
		\end{itemize}
	On jette le dé et on note la face du dessus. Donner la probabilité \emph{exacte} de chacune des issues.
}

\exe{
	On considère le lancer d'un D10, dé à 10 faces numérotées de $1$ à $10$.
	On note le nombre de la face du dessus.
	
	Soient $A, B$ les deux événements suivants.
		\begin{align*}
			A : \text{\og le nombre est pair \fg}, && B : \text{\og le nombre est multiple de $5$ \fg}.
		\end{align*}
		
	\begin{enumerate}
		\item Donner l'univers $\Omega$ des issues possibles ainsi que son cardinal $|\Omega|$.
		\item Donner l'ensemble des issues associé à l'événement $B$.
		\item Décrire avec des mots l'événement $A \cap B$ puis donner l'ensemble des issues qui lui est associé.
		\item Décrire avec des mots l'événement $\overline{A}$ puis donner l'ensemble des issues qui lui est associé.
	\end{enumerate}
}{}



%\newpage

\subsection*{Bonus}

\exe{
	\begin{enumerate}
		\item
		Soit $\Omega = \{1 ; 2 ; 3 \}$.
		Combien de sous-ensembles $A \subseteq \Omega$ existe-t-il ?
		\item
		Soit $\Omega = \{1 ; 2 ; 3 ; \dots ; n-1 ; n \}$ dépendant d'un entier naturel $n\in\N$ non nul.
		Combien de sous-ensembles $A \subseteq \Omega$ existe-t-il ?
	\end{enumerate}´
}{}

\end{document}

				% ENABLE or DISABLE font change
				% use XeLaTeX if true
\newif\ifdys
				\dystrue
				\dysfalse

\newif\ifsolutions
				\solutionstrue
				\solutionsfalse

% DYSLEXIA SWITCH
\newif\ifdys
		
				% ENABLE or DISABLE font change
				% use XeLaTeX if true
				\dystrue
				\dysfalse


\ifdys

\documentclass[a4paper, 14pt]{extarticle}
\usepackage{amsmath,amsfonts,amsthm,amssymb,mathtools}

\tracinglostchars=3 % Report an error if a font does not have a symbol.
\usepackage{fontspec}
\usepackage{unicode-math}
\defaultfontfeatures{ Ligatures=TeX,
                      Scale=MatchUppercase }

\setmainfont{OpenDyslexic}[Scale=1.0]
\setmathfont{Fira Math} % Or maybe try KPMath-Sans?
\setmathfont{OpenDyslexic Italic}[range=it/{Latin,latin}]
\setmathfont{OpenDyslexic}[range=up/{Latin,latin,num}]

\else

\documentclass[a4paper, 12pt]{extarticle}

\usepackage[utf8x]{inputenc}
%fonts
\usepackage{amsmath,amsfonts,amsthm,amssymb,mathtools}
% comment below to default to computer modern
\usepackage{libertinus,libertinust1math}

\fi


\usepackage[french]{babel}
\usepackage[
a4paper,
margin=2cm,
nomarginpar,% We don't want any margin paragraphs
]{geometry}
\usepackage{icomma}

\usepackage{fancyhdr}
\usepackage{array}
\usepackage{hyperref}

\usepackage{multicol, enumerate}
\newcolumntype{P}[1]{>{\centering\arraybackslash}p{#1}}


\usepackage{stackengine}
\newcommand\xrowht[2][0]{\addstackgap[.5\dimexpr#2\relax]{\vphantom{#1}}}

% theorems

\theoremstyle{plain}
\newtheorem{theorem}{Th\'eor\`eme}
\newtheorem*{sol}{Solution}
\theoremstyle{definition}
\newtheorem{ex}{Exercice}
\newtheorem*{rpl}{Rappel}
\newtheorem{enigme}{Énigme}

% corps
\usepackage{calrsfs}
\newcommand{\C}{\mathcal{C}}
\newcommand{\R}{\mathbb{R}}
\newcommand{\Rnn}{\mathbb{R}^{2n}}
\newcommand{\Z}{\mathbb{Z}}
\newcommand{\N}{\mathbb{N}}
\newcommand{\Q}{\mathbb{Q}}

% variance
\newcommand{\Var}[1]{\text{Var}(#1)}

% domain
\newcommand{\D}{\mathcal{D}}


% date
\usepackage{advdate}
\AdvanceDate[0]


% plots
\usepackage{pgfplots}

% table line break
\usepackage{makecell}
%tablestuff
\def\arraystretch{2}
\setlength\tabcolsep{15pt}

%subfigures
\usepackage{subcaption}

\definecolor{myg}{RGB}{56, 140, 70}
\definecolor{myb}{RGB}{45, 111, 177}
\definecolor{myr}{RGB}{199, 68, 64}

% fake sections with no title to move around the merged pdf
\newcommand{\fakesection}[1]{%
  \par\refstepcounter{section}% Increase section counter
  \sectionmark{#1}% Add section mark (header)
  \addcontentsline{toc}{section}{\protect\numberline{\thesection}#1}% Add section to ToC
  % Add more content here, if needed.
}


% SOLUTION SWITCH
\newif\ifsolutions
				\solutionstrue
				%\solutionsfalse

\ifsolutions
	\newcommand{\exe}[2]{
		\begin{ex} #1  \end{ex}
		\begin{sol} #2 \end{sol}
	}
\else
	\newcommand{\exe}[2]{
		\begin{ex} #1  \end{ex}
	}
	
\fi


% tableaux var, signe
\usepackage{tkz-tab}


%pinfty minfty
\newcommand{\pinfty}{{+}\infty}
\newcommand{\minfty}{{-}\infty}

\begin{document}


\AdvanceDate[0]

\begin{document}
\pagestyle{fancy}
\fancyhead[L]{Seconde 13}
\fancyhead[C]{\textbf{Échantillonnage : histogramme et dénombrement \ifsolutions \, -- Solutions  \fi}}
\fancyhead[R]{\today}

%% NE PAS LEUR DONNER EN FAIT, C'EST MIEUX. %%
\begin{mintedbox}{python}
# module numpy pour jet de pièce
import numpy as np
# module pyplot pour l'histogramme
import matplotlib.pyplot as plt

# renvoie le nombre de pile après 12 jets de pièce
def jet12():
	return sum(np.random.rand(12)>.5)

# jette 12 dés 1 000 fois et  crée un liste du nombre de pile obtenus
N = 1000
X = [jet12() for i in range(N)]

# crée et affiche l'histogramme du nombre de pile
plt.hist(X, bins=range(14))
plt.show()

# renvoie le nombre de 6 dans une liste X
def num6(X):
	nombre_de_6 = 0
	# à compléter
	return nombre_de_6

f = num6(X) / N
k = 2**12 * f
print(k)
\end{mintedbox}

\exe{
	On jette une pièce équilibrée $12$ fois de suite.
	Le but de l'exercice est de calculer le nombre $k$ de façons d'obtenir exactement $6$ \og pile \fg en $12$ lancers.
	Pour cela, on considère la probabilité de l'événement correspondant,
		\[ p = P(\text{\og obtenir exactement $6$ fois pile en $12$ lancers \fg}). \]
	\begin{enumerate}
		\item À l'aide d'un croquis d'arbre de probabilité, montrer que
			\begin{align*}
				p = k \cdot \left(\dfrac12\right)^{12},
			\end{align*}
		et en déduire que
			\[ k = 2^{12} \cdot p. \]
		\item On appelle  \texttt{N} = 1 000 fois la fonction \texttt{jet12()} et on crée une liste \texttt{X} de résultats (ligne 12), puis on crée et on affiche l'histogramme des résultats (lignes 15-16).
		La hauteur de chaque colonne donne le nombre de « pile » obtenus, entre 0 et 12.
		Pourquoi l'histogramme est-il symétrique ?
		%Obtenir aucun « pile » est très rare, et idem pour aucun « face » (et donc 12 « pile »). En fait les probabilités sont nécessairement symétriques, et l'histogramme l'est aussi.
		\item Compléter la fonction \texttt{num6} qui prend une liste \texttt{X} afin qu'elle renvoie le nombre de 6 qu'elle contient.
		\item Interpréter les lignes 24 et 25. Qu'est-ce que \texttt{f} ?
		\item Augmenter le nombre d'appels à la fonction et noter les valeurs de $k$ obtenues. $k$ se stabilise-t-il autour d'un certain entier ?
		\item Comparer la valeur de $k$ avec 
			\[ 
			\dfrac{12\times11\times10\times9\times8\times7}{6\times5\times4\times3\times2\times1}.
			\]
	\end{enumerate}
}{}


\end{document}

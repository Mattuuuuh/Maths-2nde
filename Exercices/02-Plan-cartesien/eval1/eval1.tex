% DYSLEXIA SWITCH
\newif\ifdys
		
				% ENABLE or DISABLE font change
				% use XeLaTeX if true
				\dystrue
				\dysfalse


\ifdys

\documentclass[a4paper, 14pt]{extarticle}
\usepackage{amsmath,amsfonts,amsthm,amssymb,mathtools}

\tracinglostchars=3 % Report an error if a font does not have a symbol.
\usepackage{fontspec}
\usepackage{unicode-math}
\defaultfontfeatures{ Ligatures=TeX,
                      Scale=MatchUppercase }

\setmainfont{OpenDyslexic}[Scale=1.0]
\setmathfont{Fira Math} % Or maybe try KPMath-Sans?
\setmathfont{OpenDyslexic Italic}[range=it/{Latin,latin}]
\setmathfont{OpenDyslexic}[range=up/{Latin,latin,num}]

\else

\documentclass[a4paper, 12pt]{extarticle}

\usepackage[utf8x]{inputenc}
%fonts
\usepackage{amsmath,amsfonts,amsthm,amssymb,mathtools}
% comment below to default to computer modern
\usepackage{libertinus,libertinust1math}

\fi


\usepackage[french]{babel}
\usepackage[
a4paper,
margin=2cm,
nomarginpar,% We don't want any margin paragraphs
]{geometry}
\usepackage{icomma}

\usepackage{fancyhdr}
\usepackage{array}
\usepackage{hyperref}

\usepackage{multicol, enumerate}
\newcolumntype{P}[1]{>{\centering\arraybackslash}p{#1}}


\usepackage{stackengine}
\newcommand\xrowht[2][0]{\addstackgap[.5\dimexpr#2\relax]{\vphantom{#1}}}

% theorems

\theoremstyle{plain}
\newtheorem{theorem}{Th\'eor\`eme}
\newtheorem*{sol}{Solution}
\theoremstyle{definition}
\newtheorem{ex}{Exercice}
\newtheorem*{rpl}{Rappel}
\newtheorem{enigme}{Énigme}

% corps
\usepackage{calrsfs}
\newcommand{\C}{\mathcal{C}}
\newcommand{\R}{\mathbb{R}}
\newcommand{\Rnn}{\mathbb{R}^{2n}}
\newcommand{\Z}{\mathbb{Z}}
\newcommand{\N}{\mathbb{N}}
\newcommand{\Q}{\mathbb{Q}}

% variance
\newcommand{\Var}[1]{\text{Var}(#1)}

% domain
\newcommand{\D}{\mathcal{D}}


% date
\usepackage{advdate}
\AdvanceDate[0]


% plots
\usepackage{pgfplots}

% table line break
\usepackage{makecell}
%tablestuff
\def\arraystretch{2}
\setlength\tabcolsep{15pt}

%subfigures
\usepackage{subcaption}

\definecolor{myg}{RGB}{56, 140, 70}
\definecolor{myb}{RGB}{45, 111, 177}
\definecolor{myr}{RGB}{199, 68, 64}

% fake sections with no title to move around the merged pdf
\newcommand{\fakesection}[1]{%
  \par\refstepcounter{section}% Increase section counter
  \sectionmark{#1}% Add section mark (header)
  \addcontentsline{toc}{section}{\protect\numberline{\thesection}#1}% Add section to ToC
  % Add more content here, if needed.
}


% SOLUTION SWITCH
\newif\ifsolutions
				\solutionstrue
				%\solutionsfalse

\ifsolutions
	\newcommand{\exe}[2]{
		\begin{ex} #1  \end{ex}
		\begin{sol} #2 \end{sol}
	}
\else
	\newcommand{\exe}[2]{
		\begin{ex} #1  \end{ex}
	}
	
\fi


% tableaux var, signe
\usepackage{tkz-tab}


%pinfty minfty
\newcommand{\pinfty}{{+}\infty}
\newcommand{\minfty}{{-}\infty}

\begin{document}


\AdvanceDate[0]
\reversemarginpar

\begin{document}
\pagestyle{fancy}
\fancyhead[L]{Seconde 5}
\fancyhead[C]{\textbf{Évaluation — Plan cartésien}}
\fancyhead[R]{\today}

\null\vspace{-30pt}
Consignes particulières : 
\begin{itemize}[label=$\bullet$]
	\item 
	La calculatrice est {interdite}.
	\item 
	On supposera l'existence d'un nombre positif noté $\sqrt3$ dont le carré vaut 3 pour résoudre l'exercice \ref{exe:équilatéral}.
	En d'autres termes, $\sqrt3 > 0$, et $\bigl(\sqrt3\bigr)^2 = 3$.
\end{itemize}

\hrule


\exe{1}{
	Donner des points du plan par leur coordonnées tels que, lorsque reliés adéquatement, on puisse lire la première lettre de votre prénom.
}{exe:prénom}{
	Le prénom de l'auteur commençant par $M$, celui-ci propose les points $A(0;0), B(0;3)$, $C(1;2), D(2;3), E(2;0)$.
	\begin{center}
	\begin{tikzpicture}[>=stealth, scale=1]
		\begin{axis}[xmin = -.1, xmax=2.1, xtick={ -3, ..., 5}, ymin=-.1, ymax=3.1, ytick={-3, ..., 5}, axis x line=middle, axis y line=middle, axis line style=<->, xlabel={}, ylabel={}, grid=both, grid style = {opacity=.5}, clip=false, x=5cm]
			
			\addplot[BLUE_E, mark=*, mark size = 1] (0,0) node[above left] {$A(0;0)$};
			\addplot[RED_E, mark=*, mark size = 1] (0,3) node[above left] {$B(0;3)$};
			\addplot[GREEN_E, mark=*, mark size = 1] (1,2) node[below] {$C(1;2)$};
			\addplot[BLUE_E, mark=*, mark size = 1] (2,3) node[above right] {$D(2;3)$};
			\addplot[RED_E, mark=*, mark size = 1] (2,0) node[above right] {$E(2;0)$};
			
			\draw[-, thick,dashed] (axis cs:0,0) -- (axis cs:0,3);
			\draw[-, thick, dashed] (axis cs:0,3) -- (axis cs:1,2);
			\draw[-, thick,dashed] (axis cs:1,2) -- (axis cs:2,3);
			\draw[-, thick,dashed] (axis cs:2,3) -- (axis cs:2,0);
		\end{axis}
	\end{tikzpicture}
	\end{center}
}

\exemulticols{}{
	Donner approximativement les coordonnées de chaque point du repère ci-contre.
	\begin{align*}
		&A(\phantom{2} ; \phantom{3}) \\[10pt]
		&B \\[10pt]
		&C \\[10pt]
		&D \\[10pt]
		&E
	\end{align*}
	\vfill\null
}{
	\begin{center}
	\begin{tikzpicture}[>=stealth, scale=1.2]
		\begin{axis}[xmin = -2.9, xmax=4.9, xtick={ -3, ..., 5}, ymin=-2.9, ymax=4.9, ytick={-3, ..., 5}, axis x line=middle, axis y line=middle, axis line style=<->, xlabel={}, ylabel={}, grid=both, grid style = {opacity=.5}]			
			\addplot[BLUE_E, mark=*, mark size = 1] (3,0) node[above right] {$A$};
			\addplot[RED_E, mark=*, mark size = 1] (-1,3) node[above left] {$B$};
			\addplot[GREEN_E, mark=*, mark size = 1] (3,3) node[above right] {$C$};
			\addplot[PURPLE_E, mark=*, mark size = 1] (0,-2) node[right] {$D$};
			\addplot[GOLD, mark=*, mark size = 1] (-2.5,0) node[above] {$E$};
		\end{axis}
	\end{tikzpicture}
	\end{center}
}{exe:lecture-coord}{
	\begin{align*}
		A(3 ; 0) && B(-1 ; 3) && C(3 ; 3) &&
		D(0;-2) && E(-2,5 ; 0)
	\end{align*}
}


\exe{}{
	On considère le triangle de sommets $A(1;2), B(-3 ; 5), C(-5 ; -6)$.
	\begin{enumerate}
		\item Tracer le triangle $ABC$ dans un repère.
		\item Calculer
		\begin{enumerate}[label=\roman*)]
			\item $AB^2$
			\item $AC^2$
			\item $BC^2$
		\end{enumerate}
		\item Que dire du triangle ?
	\end{enumerate}
}{exe:Trex}{
	\begin{multicols}{2}
	\begin{center}
	\begin{tikzpicture}[>=stealth, scale=1]
		\begin{axis}[xmin = -5.1, xmax=1.1, ymin=-6.1, ymax=5.1,axis x line=middle, axis y line=middle, axis line style=<->, xlabel={}, ylabel={}, grid=both, grid style = {opacity=.5}, x=20pt, y=20pt]			
			\addplot[BLUE_E, mark=*, mark size = 1] (1,2) node[above right] {$A$};
			\addplot[RED_E, mark=*, mark size = 1] (-3,5) node[above left] {$B$};
			\addplot[GREEN_E, mark=*, mark size = 1] (-5,-6) node[below right] {$C$};
			
			\draw[-, thick] (axis cs:1,2) -- (axis cs:-3,5) -- (axis cs:-5,-6) -- (axis cs:1,2);
		\end{axis}
	\end{tikzpicture}
	\end{center}
	
	\begin{enumerate}
		\item[2.] 
			\begin{align*}
				AB^2 &= \norm{A-B}^2 = \norm{(4 ; -3)}^2 = 16 + 9 = 25, \\
				AC^2 &= \norm{A-C}^2 = \norm{(6 ; 8)}^2 = 36 + 64 = 100, \\
				BC^2 &= \norm{B-C}^2 = \norm{(2 ; 11)}^2 = 4 + 121 = 125.
			\end{align*}
		\item[3.] D'après la réciproque du théorème de Pythagore, le triangle est rectangle en $A$ car $BC^2 = AB^2 + AC^2$.
	\end{enumerate}
	\end{multicols}
}


\exe{}{
	Considérons les points
		\begin{align*}
			\point{A}{0}{1}, && \point{B}{-3}{0}, && \point{C}{1}{-2}, && \point{D}{-2}{-3}.
		\end{align*}
	Démontrer que le quadrilatère $BACD$ est un parallélogramme en comparant le milieu de ses deux diagonales.
	
	\emph{Un parallélogramme est un quadrilatère dont les diagonales se coupent en leur milieu}.
}{exe:parallélogramme}{
	Le milieu du segment $[BC]$ est donné par
		\[ M = \dfrac12 (B+C) = (-1;-1),\]
	et le milieu du segment $[AD]$ est donné par
		\[ M' = \dfrac12 (A+D) = (-1;-1) = M.\]
	Le quadrilatère est donc bien un parallélogramme.

	\centering
	\begin{tikzpicture}[>=stealth, scale=1]
	\begin{axis}[xmin = -3.9, xmax=1.9, ymin=-3.9, ymax=1.9, axis x line=middle, axis y line=middle, axis line style=<->, xlabel={}, ylabel={}, grid=both, x=2cm, xtick = {-3, -2, ..., 1}]
					
		\addplot[RED_E, mark=*, mark size = 1] (0,1) node[above right] {$A$};
		\addplot[RED_E, mark=*, mark size = 1] (-3,0) node[above left] {$B$};
		\addplot[RED_E, mark=*, mark size = 1] (1,-2) node[below right] {$C$};
		\addplot[RED_E, mark=*, mark size = 1] (-2,-3) node[below left] {$D$};
		
		\draw[dashed, thick] (axis cs:-3,0) -- (axis cs:1,-2);
		\draw[dashed, thick] (axis cs:0,1) -- (axis cs:-2,-3);
		
		\addplot[black, mark=*, mark size = 1] (-1,-1) node[below] {$M$};
		
		\draw[thick] (axis cs:0,1) -- (axis cs:-3,0) -- (axis cs:-2,-3) -- (axis cs:1,-2) -- (axis cs:0,1);
	\end{axis}
	\end{tikzpicture}
}


\exe{}{
	Considérons les points $\point{A}{1}{1}, \point{B}{3}{1}, C\bigl(2 ; \sqrt{3}+1\bigr)$.
	Démontrer que le triangle $ABC$ est équilatéral en calculant le carré de la longueur de chaque côté.
	
	\emph{Un triangle équilatéral est un triangle dont les trois côtés ont la même longueur}.
}{exe:équilatéral}{
	On calcule 
		\begin{align*}
			AB^2 = {2^2 + 0^2} = 4, && AC^2 = {1^2 + \sqrt{3}^2} = 4, && BC^2 = 4.
		\end{align*}
}



\exe{, difficulty=1}{
	Considérons $A(-2;-1), B(-1;1)$, et $C(-3 ; 2)$ trois points.
	Quel point $D$ poser de telle sorte que le quadrilatère $ABDC$ soit un parallélogramme ?
	
	\emph{Indication : calculer le milieu $M$ de $BC$. $M$ doit aussi être le milieu de $AD$.}
}{exe:parall-constr}{

	\begin{multicols}{2}
	\begin{center}
	\begin{tikzpicture}[>=stealth, scale=1]
		\begin{axis}[xmin = -4.1, xmax=1.1, ymin=-3.1, ymax=4.1,axis x line=middle, axis y line=middle, axis line style=<->, xlabel={}, ylabel={}, grid=both, grid style = {opacity=.5}, x=20pt, y=20pt]			
			\addplot[BLUE_E, mark=*, mark size = 1] (-2,-1) node[below] {$A$};
			\addplot[RED_E, mark=*, mark size = 1] (-1,1) node[right] {$B$};
			\addplot[GREEN_E, mark=*, mark size = 1] (-3,2) node[left] {$C$};
			\addplot[YELLOW_E, mark=*, mark size = 1] (-2,4) node[above] {$D$};
			
			\draw[-, thick] (axis cs:-2,-1) -- (axis cs:-1,1) -- (axis cs:-2,4) -- (axis cs:-3,2) -- (axis cs:-2,-1);
		\end{axis}
	\end{tikzpicture}
	\end{center}
	
	Calculons d'abord 
		\[ M = \dfrac{B+C}2 = \bigl(-2 ; \dfrac32 \bigr). \]
	
	Imposons ensuite $M  = \dfrac{A+D}2$, équivalent à $D = 2M - A$.
	Il suit que 
		\[ D = (-4 +2 ; 3 +1) = (-2 ; 4). \]
	\end{multicols}
}

\exe{, difficulty=3}{
	Considérons les points $A(2;3)$ et $B(3 ; 1)$.
	Quel(s) point(s) $C$ choisir pour que $ABC$ soit un triangle équilatéral ?
}{exe:7}{

}

%%%%%%%%%%%%

\newpage
\fancyhead[C]{\textbf{Solutions}}
\shipoutAnswer

\end{document}

% DYSLEXIA SWITCH
\newif\ifdys
		
				% ENABLE or DISABLE font change
				% use XeLaTeX if true
				\dystrue
				\dysfalse


\ifdys

\documentclass[a4paper, 14pt]{extarticle}
\usepackage{amsmath,amsfonts,amsthm,amssymb,mathtools}

\tracinglostchars=3 % Report an error if a font does not have a symbol.
\usepackage{fontspec}
\usepackage{unicode-math}
\defaultfontfeatures{ Ligatures=TeX,
                      Scale=MatchUppercase }

\setmainfont{OpenDyslexic}[Scale=1.0]
\setmathfont{Fira Math} % Or maybe try KPMath-Sans?
\setmathfont{OpenDyslexic Italic}[range=it/{Latin,latin}]
\setmathfont{OpenDyslexic}[range=up/{Latin,latin,num}]

\else

\documentclass[a4paper, 12pt]{extarticle}

\usepackage[utf8x]{inputenc}
%fonts
\usepackage{amsmath,amsfonts,amsthm,amssymb,mathtools}
% comment below to default to computer modern
\usepackage{libertinus,libertinust1math}

\fi


\usepackage[french]{babel}
\usepackage[
a4paper,
margin=2cm,
nomarginpar,% We don't want any margin paragraphs
]{geometry}
\usepackage{icomma}

\usepackage{fancyhdr}
\usepackage{array}
\usepackage{hyperref}

\usepackage{multicol, enumerate}
\newcolumntype{P}[1]{>{\centering\arraybackslash}p{#1}}


\usepackage{stackengine}
\newcommand\xrowht[2][0]{\addstackgap[.5\dimexpr#2\relax]{\vphantom{#1}}}

% theorems

\theoremstyle{plain}
\newtheorem{theorem}{Th\'eor\`eme}
\newtheorem*{sol}{Solution}
\theoremstyle{definition}
\newtheorem{ex}{Exercice}
\newtheorem*{rpl}{Rappel}
\newtheorem{enigme}{Énigme}

% corps
\usepackage{calrsfs}
\newcommand{\C}{\mathcal{C}}
\newcommand{\R}{\mathbb{R}}
\newcommand{\Rnn}{\mathbb{R}^{2n}}
\newcommand{\Z}{\mathbb{Z}}
\newcommand{\N}{\mathbb{N}}
\newcommand{\Q}{\mathbb{Q}}

% variance
\newcommand{\Var}[1]{\text{Var}(#1)}

% domain
\newcommand{\D}{\mathcal{D}}


% date
\usepackage{advdate}
\AdvanceDate[0]


% plots
\usepackage{pgfplots}

% table line break
\usepackage{makecell}
%tablestuff
\def\arraystretch{2}
\setlength\tabcolsep{15pt}

%subfigures
\usepackage{subcaption}

\definecolor{myg}{RGB}{56, 140, 70}
\definecolor{myb}{RGB}{45, 111, 177}
\definecolor{myr}{RGB}{199, 68, 64}

% fake sections with no title to move around the merged pdf
\newcommand{\fakesection}[1]{%
  \par\refstepcounter{section}% Increase section counter
  \sectionmark{#1}% Add section mark (header)
  \addcontentsline{toc}{section}{\protect\numberline{\thesection}#1}% Add section to ToC
  % Add more content here, if needed.
}


% SOLUTION SWITCH
\newif\ifsolutions
				\solutionstrue
				%\solutionsfalse

\ifsolutions
	\newcommand{\exe}[2]{
		\begin{ex} #1  \end{ex}
		\begin{sol} #2 \end{sol}
	}
\else
	\newcommand{\exe}[2]{
		\begin{ex} #1  \end{ex}
	}
	
\fi


% tableaux var, signe
\usepackage{tkz-tab}


%pinfty minfty
\newcommand{\pinfty}{{+}\infty}
\newcommand{\minfty}{{-}\infty}

\begin{document}


\AdvanceDate[0]
\reversemarginpar
\setlength{\marginparsep}{.5cm}

\begin{document}
\pagestyle{fancy}
\fancyhead[L]{Seconde 5}
\fancyhead[C]{\textbf{Évaluation — Plan cartésien}}
\fancyhead[R]{\today}

\null\vspace{-30pt}
Consignes particulières : 
\begin{itemize}[label=$\bullet$]
	\item 
	La calculatrice est {interdite}.
	\item
	L'évaluation fait 2 pages.
	\item
	L'exercice \ref{exe:lecture-coord} peut être fait entièrement sur la feuille d'évaluation. Écrire son nom avant de rendre le sujet pour qu'il soit corrigé.
	\item 
	On supposera l'existence d'un nombre positif noté $\sqrt{12}$ dont le carré vaut 12 pour résoudre l'exercice \ref{exe:équilatéral}.
	En d'autres termes, $\sqrt{12} > 0$, et $\bigl(\sqrt{12}\bigr)^2 = 3$.
\end{itemize}

\marginpar{[pts]}
\hrule


\exe{2}{
	Donner des points du plan par leur coordonnées tels que, lorsque reliés adéquatement, on puisse lire la première lettre de votre prénom.
}{exe:prénom}{
	Le prénom du correcteur commençant par $M$, celui-ci propose les points $A(0;0), B(0;3)$, $C(1;2), D(2;3), E(2;0)$.
	\begin{center}
	\begin{tikzpicture}[>=stealth, scale=1]
		\begin{axis}[xmin = -.1, xmax=2.1, xtick={ -3, ..., 5}, ymin=-.1, ymax=3.1, ytick={-3, ..., 5}, axis x line=middle, axis y line=middle, axis line style=<->, xlabel={}, ylabel={}, grid=both, grid style = {opacity=.5}, clip=false, x=5cm]
			
			\addplot[BLUE_E, mark=*, mark size = 1] (0,0) node[above left] {$A(0;0)$};
			\addplot[RED_E, mark=*, mark size = 1] (0,3) node[above left] {$B(0;3)$};
			\addplot[GREEN_E, mark=*, mark size = 1] (1,2) node[below] {$C(1;2)$};
			\addplot[BLUE_E, mark=*, mark size = 1] (2,3) node[above right] {$D(2;3)$};
			\addplot[RED_E, mark=*, mark size = 1] (2,0) node[above right] {$E(2;0)$};
			
			\draw[-, thick,dashed] (axis cs:0,0) -- (axis cs:0,3);
			\draw[-, thick, dashed] (axis cs:0,3) -- (axis cs:1,2);
			\draw[-, thick,dashed] (axis cs:1,2) -- (axis cs:2,3);
			\draw[-, thick,dashed] (axis cs:2,3) -- (axis cs:2,0);
		\end{axis}
	\end{tikzpicture}
	\end{center}
}

\exemulticols{3}{
	Donner approximativement les coordonnées de chaque point du repère ci-contre.
	\begin{align*}
		&A\hspace{3cm} \\[10pt]
		&B \\[10pt]
		&C \\[10pt]
		&D \\[10pt]
		&E \\[10pt]
		&O
	\end{align*}
	%\vfill\null
}{
	\begin{center}
	\begin{tikzpicture}[>=stealth, scale=1.2]
		\begin{axis}[xmin = -4.9, xmax=4.9, ymin=-4.9, ymax=4.9, axis x line=middle, axis y line=middle, axis line style=<->, xlabel={}, ylabel={}, grid=both, grid style = {opacity=.5}, xtick distance=1, ytick distance=1]			
			\addplot[BLUE_E, mark=*, mark size = 1] (2,0) node[above] {$A$};
			\addplot[RED_E, mark=*, mark size = 1] (-3,3) node[above left] {$B$};
			\addplot[GREEN_E, mark=*, mark size = 1] (3,1.5) node[above right] {$C$};
			\addplot[PURPLE_E, mark=*, mark size = 1] (0,-1.5) node[right] {$D$};
			\addplot[GOLD_E, mark=*, mark size = 1] (-2.5,-3) node[left] {$E$};
			\addplot[BLACK, mark=*, mark size = 1] (0,0) node[above left] {$O$};
		\end{axis}
	\end{tikzpicture}
	\end{center}
}{exe:lecture-coord}{
	\begin{align*}
		A(3 ; 0) && B(-1 ; 3) && C(3 ; 3) &&
		D(0;-2) && E(-2,5 ; 0)
	\end{align*}
}

% Lagrange four-square theorem
\exemulticols{5}{
	Considérons le triangle de sommets 
		\[ A(3;4), B(9 ; -5), \et C(-9 ; -4).\]
	
	Les questions 1, 2, et 3 peuvent être faites séparément.
	\begin{enumerate}
		\item Tracer le triangle $ABC$ dans un repère.
		\item Montrer par le calcul que
		\begin{enumerate}[label=\roman*)]
			\item $AB^2 = 117$
			\item $AC^2 = 208$
			\item $BC^2 = 325$
		\end{enumerate}
		\item Que dire du triangle $ABC$ ?
	\end{enumerate}
}{
	%\hfill
	%Aide aux calculs
	
	\def\arraystretch{1.1}
	\setlength\tabcolsep{15pt}
	\hfill
	\begin{tabular}{|c|c|}\hline
		$n$ & $n^2$ \\ \hline
		11 & 121 \\ \hline
		12 & 144 \\ \hline
		13 & 169 \\ \hline
		14 & 196 \\ \hline
		15 & 225 \\ \hline
		16 & 256 \\ \hline
		17 & 289 \\ \hline
		18 & 324 \\ \hline
		19 & 361 \\ \hline
		20 & 400 \\ \hline
	\end{tabular}
}{exe:Trex}{
	\begin{center}
	\begin{tikzpicture}[>=stealth, scale=1]
		\begin{axis}[xmin = -9.1, xmax=9.1, ymin=-5.1, ymax=5.1,axis x line=middle, axis y line=middle, axis line style=<->, xlabel={}, ylabel={}, grid=both, grid style = {opacity=.5}, x=20pt, y=20pt]			
			\addplot[BLUE_E, mark=*, mark size = 1] (3,4) node[above] {$A$};
			\addplot[RED_E, mark=*, mark size = 1] (9,-5) node[right] {$B$};
			\addplot[GREEN_E, mark=*, mark size = 1] (-9,-4) node[left] {$C$};
			
			\draw[-, thick] (axis cs:3,4) -- (axis cs:9,-5) -- (axis cs:-9,-4) -- (axis cs:3,4);
		\end{axis}
	\end{tikzpicture}
	\end{center}
	
	\begin{enumerate}
		\item[2.] 
			\begin{align*}
				AB^2 &= \norm{A-B}^2 = \norm{(-6 ; 9)}^2 = 36 + 81 = 117, \\
				AC^2 &= \norm{A-C}^2 = \norm{(12 ; 8)}^2 = 144 + 64 = 208, \\
				BC^2 &= \norm{B-C}^2 = \norm{(18 ; -1)}^2 = 324 + 1 = 325.
			\end{align*}
		\item[3.] D'après la réciproque du théorème de Pythagore, le triangle est rectangle en $A$ car $BC^2 = AB^2 + AC^2$.
	\end{enumerate}
}


\exe{2}{
	Considérons les points
		\begin{align*}
			\point{A}{3}{-1}, && \point{B}{4}{2}, && \point{C}{-2}{3}, && \point{D}{-1}{6}.
		\end{align*}
	Montrer que le quadrilatère $CDBA$ est un parallélogramme en comparant le milieu de ses deux diagonales.
	
	\emph{Un parallélogramme est un quadrilatère dont les diagonales se coupent en leur milieu}.
}{exe:parallélogramme}{
	Le milieu du segment $[BC]$ est donné par
		\[ M = \dfrac12 (B+C) = (1;2,5),\]
	et le milieu du segment $[AD]$ est donné par
		\[ M' = \dfrac12 (A+D) = (1;2,5).\]
	Comme $M' = M$, le quadrilatère est bien un parallélogramme.

	\centering
	\begin{tikzpicture}[>=stealth, scale=1]
	\begin{axis}[xmin = -2.1, xmax=4.1, ymin=-1.1, ymax=6.1, axis x line=middle, axis y line=middle, axis line style=<->, xlabel={}, ylabel={}, grid=both, clip=false, ytick distance=1]
					
		\addplot[RED_E, mark=*, mark size = 1] (3,-1) node[right] {$A$};
		\addplot[RED_E, mark=*, mark size = 1] (4,2) node[right] {$B$};
		\addplot[RED_E, mark=*, mark size = 1] (-2,3) node[left] {$C$};
		\addplot[RED_E, mark=*, mark size = 1] (-1,6) node[left] {$D$};
		
		\draw[dashed, thick] (axis cs:3,-1) -- (axis cs:-1,6);
		\draw[dashed, thick] (axis cs:4,2) -- (axis cs:-2,3);
		
		\addplot[black, mark=*, mark size = 1] (1,2.5) node[above right] {$M$};
		
		\draw[thick] (axis cs:3,-1) -- (axis cs:4,2) -- (axis cs:-1,6) -- (axis cs:-2,3) -- (axis cs:3,-1);
	\end{axis}
	\end{tikzpicture}
}


\exe{3, difficulty=1}{
	Considérons les points $\point{A}{2}{1}, \point{B}{2}{5}, C\bigl(\sqrt{12}+2 ; 3\bigr)$.
	Montrer que le triangle $ABC$ est équilatéral en calculant le carré de la longueur de chaque côté.
	
	\emph{Un triangle équilatéral est un triangle dont les trois côtés ont la même longueur}.
}{exe:équilatéral}{
	On calcule 
		\begin{align*}
			AB^2 = {0^2 + 4^2} = 16, && AC^2 = {\sqrt{12}^2 + 2^2} = 16, && BC^2 = 16.
		\end{align*}
}

\exe{3, difficulty=3}{
	Considérons les points $A(2;3)$ et $B(3 ; 1)$ et $(d)$ la médiatrice du segment $[AB]$ : ce sont tous les points $M(x;y)$ à équidistance de $A$ et $B$.
	
	Montrer que $x$ et $y$ vérifient 
		\[ y = \dfrac12 x + \dfrac34. \]
		
	\emph{Toute trace de recherche sera prise en compte.}
}{exe:médiatrice}{
	TODO
}

\exe{2, difficulty=1}{
	Le but de cet exercice est de généraliser l'exercice \ref{exe:parallélogramme}.

	Soient $A, B, C$ trois points du plan distincts quelconques.
	Montrer que le quadrilatère $ABDC$, où $D = B+C-A$, est un parallélogramme.
}{exe:parall-gen}{
	On calcule le milieu des segments $[AD]$ et $[BC]$ et on compare.
	
	Le milieu du segment $[AD]$ est donné par $\frac12(A+D) = \frac12 (B+C)$. 
	Celui du segment $[BC]$ est donné par $\frac12(B+C)$.
}



\exe{2, difficulty=1}{
	Soient $\point{A}12$ et $\point{M}3{-1}$ deux points du plan.
	Quel point $C$ faut-il choisir pour que $M$ soit le milieu du segment $[AC]$ ? Donner ses coordonnées.
}{exe:milieu2}{
	La contrainte que $M$ soit le milieu de $[AC]$ s'écrit
		\begin{align*}
			M = \dfrac12(A+C) && \iff && 2M = A+C && \iff && C = 2M-A
		\end{align*}
	Par suite,
		\[ C= 2M-A = 2\cdot\left(-\dfrac53;-1\right) - \left(-\dfrac32;\dfrac53\right) = \left(-\dfrac{11}6; -\dfrac{11}3\right). \]
	
}



\exe{2, difficulty=1}{
	Soit $A(x;x)$ un point dépendant d'un paramètre réel $x\in\R$.
	
	Où se situe le point $A$ ? Tracer ses positions possibles dans le repère ci-dessous.
	
	\begin{center}
	\begin{tikzpicture}[>=stealth, scale=1]
		\begin{axis}[xmin = -5.1, xmax=5.1, ymin=-5.1, ymax=5.1, axis x line=middle, axis y line=middle, axis line style=<->, xlabel={}, ylabel={}, grid=both, grid style = {opacity=.5}, clip=false, xtick distance = 1, ytick distance=1, x=1cm, y=1cm]
		\end{axis}
	\end{tikzpicture}
	\end{center}
}{exe:diagonale}{
	TODO
}{}


% to use
%\exe{2}{
%	Montrer que le milieu $M$ de $A$ et de $B$, tel que défini dans le cours, est à équidistance de $A$ et de $B$.
%}{exe:milieu-équid}{
%	TODO
%}

% idk how to prove that actually lol.
% was hoping for easy Pythagoras but angles are required.
%\exe{2, difficulty=2}{
%	Considérons $A$ et $B$ deux points distincts du plan, ainsi que $M$ le milieu du segment $[AB]$.
%	
%	Posons, en outre, $C$, un point à équidistance de $A$ et de $B$, et distinct de $M$.
%	
%	Montrer que $AMC$ est rectangle en $M$.
%}{exe:médiatrice-proof}{
%	TODO
%}

% trop dur
\exe{20, difficulty=3}{
	Le but de cet exercice est de généraliser l'exercice \ref{exe:Trex}.
	
	Montrer que le triangle de sommets $O(0;0), A(x_A ; y_A)$, et $B(x_B ; y_B)$ est rectangle en $O$ si et seulement si
		\[ x_A x_B + y_A y_B = 0. \]
}{exe:Trex-gen}{
	TODO
}

%%%%%%%%%%%%

\newpage
\fancyhead[C]{\textbf{Solutions}}
\shipoutAnswer

\end{document}

%!TEX root = ../eval1.tex

\newcommand{\xA}{\number\numexpr2-\exeIII\relax}
\newcommand{\yA}{\number\numexpr6+\exeIII\relax}
\newcommand{\xB}{\number\numexpr8-\exeIII\relax}
\newcommand{\yB}{\number\numexpr-3+\exeIII\relax}
\newcommand{\yC}{\number\numexpr-10-\exeIII\relax}
\newcommand{\xC}{\number\numexpr-2+\exeIII\relax}

\exemulticols{3}{
	Considérons le triangle de sommets 
		\[ A(\xA;\yA), \quad B(\xB ; \yB), \quad\et\quad C(\xC ; \yC).\]
	
	Les questions 1 et 2 peuvent être faites séparément.
	\begin{enumerate}
		%\item Tracer le triangle $ABC$ dans un repère.
		\item Montrer par le calcul que
		\begin{enumerate}[label=\roman*)]
			\item $AB^2 = 117$
			\item $AC^2 = 208$
			\item $BC^2 = 325$
		\end{enumerate}
		\item Que dire du triangle $ABC$ ?
	\end{enumerate}
}{
	%\hfill
	%Aide aux calculs
	\def\arraystretch{1.1}
	\setlength\tabcolsep{15pt}
	%\hfill
	\begin{center}
	\begin{tabular}{|c|c|}\hline
		$n$ & $n^2$ \\ \hline
		11 & 121 \\ \hline
		12 & 144 \\ \hline
		13 & 169 \\ \hline
		14 & 196 \\ \hline
		15 & 225 \\ \hline
		16 & 256 \\ \hline
		17 & 289 \\ \hline
		18 & 324 \\ \hline
		19 & 361 \\ \hline
		20 & 400 \\ \hline
	\end{tabular}
	\end{center}
}{exe:Trex}{
	\begin{center}
	\begin{tikzpicture}[>=stealth, scale=1]
		\begin{axis}[xmin = -9.1, xmax=9.1, ymin=-5.1, ymax=5.1,axis x line=middle, axis y line=middle, axis line style=<->, xlabel={}, ylabel={}, grid=both, grid style = {opacity=.5}, x=20pt, y=20pt]			
			\addplot[BLUE_E, mark=*, mark size = 1] (2,6) node[above] {$A$};
			\addplot[RED_E, mark=*, mark size = 1] (8,-3) node[right] {$B$};
			\addplot[GREEN_E, mark=*, mark size = 1] (-10,-2) node[left] {$C$};
			
			\draw[-, thick] (axis cs:2,6) -- (axis cs:8,-3) -- (axis cs:-10,-2) -- (axis cs:2,6);
		\end{axis}
	\end{tikzpicture}
	\end{center}
	
	\begin{enumerate}
		\item%[2.] 
			\begin{align*}
				AB^2 &= \norm{A-B}^2 = \norm{(-6 ; 9)}^2 = 36 + 81 = 117, \\
				AC^2 &= \norm{A-C}^2 = \norm{(12 ; 8)}^2 = 144 + 64 = 208, \\
				BC^2 &= \norm{B-C}^2 = \norm{(18 ; -1)}^2 = 324 + 1 = 325.
			\end{align*}
		\item%[3.] 
		D'après la réciproque du théorème de Pythagore, le triangle est rectangle en $A$ car $BC^2 = AB^2 + AC^2$.
	\end{enumerate}
}
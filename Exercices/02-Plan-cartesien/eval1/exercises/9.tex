%!TEX root = ../eval1.tex

\exe{2, difficulty=0}{
	Montrer que
		\[ \dfrac1{12} = \dfrac1{100} \times \left( 8 + \dfrac{1}3 \right). \]
	En déduire le développement décimal de $\frac1{12}$ en utilisant uniquement la relation ci-dessus ainsi que l'égalité $\frac13 = 0,333\dots$ (3 se répète à l'infini).
}{exe:16-nondec}{
	\begin{align*}
		\dfrac1{100} \times \left( 8 + \dfrac{1}3 \right) &= \dfrac1{100} \times \left( \dfrac{24}3 + \dfrac{1}3 \right) \\
		&= \dfrac1{100} \times \dfrac{25}3 \\
		&= \dfrac1{100} \times \dfrac{25}3 \\
		&= \dfrac{25}{100 \times 3} \\
		&= \dfrac{25}{25\times4\times3} \\
		&= \dfrac{1}{4\times3} \\
		&= \dfrac{1}{12}
	\end{align*}
}

% this is crazy lol
%\exe{20, difficulty=1}{
%	Le but de cet exercice est de généraliser l'exercice \ref{exe:Trex}.
%	
%	Montrer que le triangle de sommets $A(x_A ; y_A), B(x_B ; y_B),$ et $C(x_C ; y_C)$ est rectangle en $A$ si et seulement si
%		\[ (x_B - x_A) (x_C - x_A) + (y_B - y_A) (y_C - y_A) = 0. \]
%}{exe:Trex-gen}{
%	TODO
%}
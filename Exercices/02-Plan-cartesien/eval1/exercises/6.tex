%!TEX root = ../eval1.tex

\newcommand{\ordonnee}{\number\numexpr2+\exeVI\relax}

\exe{2, difficulty=2}{
	L'origine $O(0 ; 0)$ appartient-elle au segment $[AB]$ où $A(\number\numexpr3-\exeVI\relax;\ordonnee)$ et $B(\number\numexpr-4+\exeVI\relax;\ordonnee)$ ? Justifier.
}{exe:C-on-AB}{
	Les deux points ont la même ordonnée : ils sont donc alignés horizontalement.
	Tous les points de $[AB]$ sont d'ordonnée $\ordonnee$, et l'origine du repère ne peut donc pas appartenir au segment.
	En effet, $\lambda A + (1-\lambda)B$ est d'ordonnée $\ordonnee\lambda + (1-\lambda)\ordonnee = \ordonnee$, peu importe la valeur de $\lambda\in\R·$.
}
%!TEX root = ../eval1.tex

\exe{3, difficulty=1}{
	Soient $A(x_A; y_A), B(x_B ; y_B) , C(x_C; y_C)$ trois points du plan distincts.
	Les questions 1 et 2 peuvent être faites séparément.
	\begin{enumerate}
		\item
		Montrer que le quadrilatère $ABDC$, où $D(x_B + x_C - x_A ; y_B + y_C - y_A)$, est un parallélogramme. (autrement dit, $D = B+C-A$)
		\item
		Donner un exemple de parallélogramme à l'aide de la question précédente. Quatre points et leurs coordonnées sont attendus.
	\end{enumerate}
}{exe:parall-gen}{
	\begin{enumerate}
		\item
		On calcule le milieu des segments $[AD]$ et $[BC]$ et on compare.
		Les opérations sur les couples permettent de nous défaire des coordonnées encombrantes.
		
		Le milieu du segment $[AD]$ est donné par $\frac12(A+D) = \frac12 (B+C)$. 
		Celui du segment $[BC]$ est donné par $\frac12(B+C)$.
		\item
		On prend, par exemple, $A(0;0), B(1 ; 1), C(1 ; 0),$ et $D = B+C-A = (2 ; 1)$.
	\end{enumerate}
}
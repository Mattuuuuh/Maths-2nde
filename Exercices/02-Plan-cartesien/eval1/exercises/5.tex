%!TEX root = ../eval1.tex

%\renewcommand{\exeV}{0}
\newcommand{\xAV}{\number\numexpr2+\exeV-3\relax}
\newcommand{\yAV}{\number\numexpr1+\exeV-4\relax}
\newcommand{\xBV}{\number\numexpr2+\exeV-3\relax}
\newcommand{\yBV}{\number\numexpr5+\exeV-4\relax}
\newcommand{\xCV}{\number\numexpr2+\exeV-3\relax}
\newcommand{\yCV}{\number\numexpr3+\exeV-4\relax}

\exe{3, difficulty=1}{
	Considérons les points $\point{A}{\xAV}{\yAV}, \point{B}{\xBV}{\yBV}, C\bigl(\sqrt{12}+\xCV ; \yCV\bigr)$.
	Montrer, par le calcul, que le triangle $ABC$ est équilatéral.
}{exe:equilateral}{
	On calcule 
		\begin{align*}
			AB^2 = {0^2 + 4^2} = 16, && AC^2 = {\sqrt{12}^2 + 2^2} = 16, && BC^2 = 16.
		\end{align*}
}
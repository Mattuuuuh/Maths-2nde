%!TEX root = ../eval1.tex

\exe{2}{
	Considérons les points
		\begin{align*}
			\point{A}{\dfrac{10}3}{-\dfrac76}, && \point{B}{\dfrac{13}3}{\dfrac{11}6}, && \point{C}{\dfrac{-5}3}{\dfrac{17}6}, && \point{D}{-\dfrac23}{\dfrac{35}6}.
		\end{align*}
	Montrer que le quadrilatère $CDBA$ est un parallélogramme.
}{exe:parallélogramme}{
	Le milieu du segment $[BC]$ est donné par
		\[ M = \dfrac12 (B+C) = (1;2,5),\]
	et le milieu du segment $[AD]$ est donné par
		\[ M' = \dfrac12 (A+D) = (1;2,5).\]
	Comme $M' = M$, le quadrilatère est bien un parallélogramme.

	\centering
	\begin{tikzpicture}[>=stealth, scale=1]
	\begin{axis}[xmin = -2.1, xmax=4.1, ymin=-1.1, ymax=6.1, axis x line=middle, axis y line=middle, axis line style=<->, xlabel={}, ylabel={}, grid=both, clip=false, ytick distance=1]
					
		\addplot[RED_E, mark=*, mark size = 1] (3,-1) node[right] {$A$};
		\addplot[RED_E, mark=*, mark size = 1] (4,2) node[right] {$B$};
		\addplot[RED_E, mark=*, mark size = 1] (-2,3) node[left] {$C$};
		\addplot[RED_E, mark=*, mark size = 1] (-1,6) node[left] {$D$};
		
		\draw[dashed, thick] (axis cs:3,-1) -- (axis cs:-1,6);
		\draw[dashed, thick] (axis cs:4,2) -- (axis cs:-2,3);
		
		\addplot[black, mark=*, mark size = 1] (1,2.5) node[above right] {$M$};
		
		\draw[thick] (axis cs:3,-1) -- (axis cs:4,2) -- (axis cs:-1,6) -- (axis cs:-2,3) -- (axis cs:3,-1);
	\end{axis}
	\end{tikzpicture}
}
%!TEX root = ../eval1.tex

%\exemulticols{3}{
%	Donner approximativement les coordonnées de chaque point du repère ci-contre.
%	\begin{align*}
%		&A\hspace{3cm} \\[10pt]
%		&B \\[10pt]
%		&C \\[10pt]
%		&D \\[10pt]
%		&E \\[10pt]
%		&O
%	\end{align*}
%	%\vfill\null
%}{
%	\begin{center}
%	\begin{tikzpicture}[>=stealth, scale=1.2]
%		\begin{axis}[xmin = -4.9, xmax=4.9, ymin=-4.9, ymax=4.9, axis x line=middle, axis y line=middle, axis line style=<->, xlabel={}, ylabel={}, grid=both, grid style = {opacity=.5}, xtick distance=1, ytick distance=1]			
%			\addplot[BLUE_E, mark=*, mark size = 1] (2,0) node[above] {$A$};
%			\addplot[RED_E, mark=*, mark size = 1] (-3,3) node[above left] {$B$};
%			\addplot[GREEN_E, mark=*, mark size = 1] (3,1.5) node[above right] {$C$};
%			\addplot[PURPLE_E, mark=*, mark size = 1] (0,-1.5) node[right] {$D$};
%			\addplot[GOLD_E, mark=*, mark size = 1] (-2.5,-3) node[left] {$E$};
%			\addplot[BLACK, mark=*, mark size = 1] (0,0) node[above left] {$O$};
%		\end{axis}
%	\end{tikzpicture}
%	\end{center}
%}{exe:lecture-coord}{
%	\begin{align*}
%		A(3 ; 0) && B(-1 ; 3) && C(3 ; 3) &&
%		D(0;-2) && E(-2,5 ; 0)
%	\end{align*}
%}

\exe{1}{
	Quelle est la différence entre $(2;3)$ et $\{2 ; 3\}$ ?
}{exe:diff-par-brack}{
	$(2;3)$ est un couple ordonné (2 est le premier élément, 3 le deuxième).
	
	$\{2 ; 3\}$ est un ensemble non ordonné.
}

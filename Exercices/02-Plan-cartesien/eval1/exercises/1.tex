%!TEX root = ../eval1.tex

\exe{2}{
	Donner des points du plan par leur coordonnées tels que, lorsque reliés adéquatement, on puisse lire la première lettre de votre prénom.
}{exe:prénom}{
	Le prénom du correcteur commençant par $M$, celui-ci propose les points $A(0;0), B(0;3)$, $C(1;2), D(2;3), E(2;0)$.
	\begin{center}
	\begin{tikzpicture}[>=stealth, scale=1]
		\begin{axis}[xmin = -.1, xmax=2.1, xtick={ -3, ..., 5}, ymin=-.1, ymax=3.1, ytick={-3, ..., 5}, axis x line=middle, axis y line=middle, axis line style=<->, xlabel={}, ylabel={}, grid=both, grid style = {opacity=.5}, clip=false, x=5cm]
			
			\addplot[BLUE_E, mark=*, mark size = 1] (0,0) node[above left] {$A(0;0)$};
			\addplot[RED_E, mark=*, mark size = 1] (0,3) node[above left] {$B(0;3)$};
			\addplot[GREEN_E, mark=*, mark size = 1] (1,2) node[below] {$C(1;2)$};
			\addplot[BLUE_E, mark=*, mark size = 1] (2,3) node[above right] {$D(2;3)$};
			\addplot[RED_E, mark=*, mark size = 1] (2,0) node[above right] {$E(2;0)$};
			
			\draw[-, thick,dashed] (axis cs:0,0) -- (axis cs:0,3);
			\draw[-, thick, dashed] (axis cs:0,3) -- (axis cs:1,2);
			\draw[-, thick,dashed] (axis cs:1,2) -- (axis cs:2,3);
			\draw[-, thick,dashed] (axis cs:2,3) -- (axis cs:2,0);
		\end{axis}
	\end{tikzpicture}
	\end{center}
}
%!TEX encoding = UTF8
%!TEX root =notes.tex


%%%%%%%%%%%%%%%%%%%%%%%%%%%%%%%%%
% PACKAGE IMPORTS
%%%%%%%%%%%%%%%%%%%%%%%%%%%%%%%%%


\usepackage[french]{babel}

\usepackage[tmargin=2cm,rmargin=1in,lmargin=1in,margin=0.85in,bmargin=2cm,footskip=.2in]{geometry}
\usepackage{amsmath,amsfonts,amsthm,amssymb,mathtools}
\usepackage[varbb]{newpxmath}
\usepackage{xfrac}
\usepackage[makeroom]{cancel}
\usepackage{mathtools}
\usepackage{bookmark}
\usepackage{enumitem}
\usepackage{hyperref,theoremref}
\hypersetup{
	pdftitle={Assignment},
	colorlinks=true, linkcolor=doc!90,
	bookmarksnumbered=true,
	bookmarksopen=true
}
\usepackage[most,many,breakable]{tcolorbox}
\usepackage{xcolor}
\usepackage{varwidth}
\usepackage{varwidth}
\usepackage{etoolbox}
%\usepackage{authblk}
\usepackage{nameref}
\usepackage{multicol,array}
\usepackage{tikz-cd}
\usepackage[ruled,vlined,linesnumbered]{algorithm2e}
\usepackage{comment} % enables the use of multi-line comments (\ifx \fi) 
\usepackage{import}
\usepackage{xifthen}
\usepackage{pdfpages}
\usepackage{transparent}


\newcommand\mycommfont[1]{\footnotesize\ttfamily\textcolor{blue}{#1}}
\SetCommentSty{mycommfont}
\newcommand{\incfig}[1]{%
    \def\svgwidth{\columnwidth}
    \import{./figures/}{#1.pdf_tex}
}

\usepackage{tikzsymbols}
%\renewcommand\qedsymbol{$\Laughey$}


%\usepackage{import}
%\usepackage{xifthen}
%\usepackage{pdfpages}
%\usepackage{transparent}


%%%%%%%%%%%%%%%%%%%%%%%%%%%%%%
% SELF MADE COLORS
%%%%%%%%%%%%%%%%%%%%%%%%%%%%%%



\definecolor{myg}{RGB}{56, 140, 70}
\definecolor{myb}{RGB}{45, 111, 177}
\definecolor{myr}{RGB}{199, 68, 64}
\definecolor{mytheorembg}{HTML}{F2F2F9}
\definecolor{mytheoremfr}{HTML}{00007B}
\definecolor{mylenmabg}{HTML}{FFFAF8}
\definecolor{mylenmafr}{HTML}{983b0f}
\definecolor{mypropbg}{HTML}{f2fbfc}
\definecolor{mypropfr}{HTML}{191971}
\definecolor{myexamplebg}{HTML}{F2FBF8}
\definecolor{myexamplefr}{HTML}{88D6D1}
\definecolor{myexampleti}{HTML}{2A7F7F}
\definecolor{mydefinitbg}{HTML}{E5E5FF}
\definecolor{mydefinitfr}{HTML}{3F3FA3}
\definecolor{notesgreen}{RGB}{0,162,0}
\definecolor{myp}{RGB}{197, 92, 212}
\definecolor{mygr}{HTML}{2C3338}
\definecolor{myred}{RGB}{127,0,0}
\definecolor{myyellow}{RGB}{169,121,69}
\definecolor{myexercisebg}{HTML}{F2FBF8}
\definecolor{myexercisefg}{HTML}{88D6D1}


%%%%%%%%%%%%%%%%%%%%%%%%%%%%
% TCOLORBOX SETUPS
%%%%%%%%%%%%%%%%%%%%%%%%%%%%

\setlength{\parindent}{1cm}
%================================
% THEOREM BOX
%================================

\tcbuselibrary{theorems,skins,hooks}
\newtcbtheorem[number within=chapter]{Theorem}{Théorème}
{%
	enhanced,
	breakable,
	colback = mytheorembg,
	frame hidden,
	boxrule = 0sp,
	borderline west = {2pt}{0pt}{mytheoremfr},
	sharp corners,
	detach title,
	before upper = \tcbtitle\par\smallskip,
	coltitle = mytheoremfr,
	fonttitle = \bfseries\sffamily,
	description font = \mdseries,
	separator sign none,
	segmentation style={solid, mytheoremfr},
}
{th}


\tcbuselibrary{theorems,skins,hooks}
\newtcolorbox{Theoremcon}
{%
	enhanced
	,breakable
	,colback = mytheorembg
	,frame hidden
	,boxrule = 0sp
	,borderline west = {2pt}{0pt}{mytheoremfr}
	,sharp corners
	,description font = \mdseries
	,separator sign none
}

%================================
% Corollery
%================================
\tcbuselibrary{theorems,skins,hooks}
\newtcbtheorem[use counter=tcb@cnt@Theorem]{Corollary}{Corollaire}
{%
	enhanced
	,breakable
	,colback = myp!10
	,frame hidden
	,boxrule = 0sp
	,borderline west = {2pt}{0pt}{myp!85!black}
	,sharp corners
	,detach title
	,before upper = \tcbtitle\par\smallskip
	,coltitle = myp!85!black
	,fonttitle = \bfseries\sffamily
	,description font = \mdseries
	,separator sign none
	,segmentation style={solid, myp!85!black}
}
{th}

%================================
% LENMA
%================================

\tcbuselibrary{theorems,skins,hooks}
\newtcbtheorem[use counter=tcb@cnt@Theorem]{Lemma}{Lemme}
{%
	enhanced,
	breakable,
	colback = mylenmabg,
	frame hidden,
	boxrule = 0sp,
	borderline west = {2pt}{0pt}{mylenmafr},
	sharp corners,
	detach title,
	before upper = \tcbtitle\par\smallskip,
	coltitle = mylenmafr,
	fonttitle = \bfseries\sffamily,
	description font = \mdseries,
	separator sign none,
	segmentation style={solid, mylenmafr},
}
{th}


%================================
% PROPOSITION
%================================

\tcbuselibrary{theorems,skins,hooks}
\newtcbtheorem[use counter=tcb@cnt@Theorem]{Prop}{Proposition}
{%
	enhanced,
	breakable,
	colback = mypropbg,
	frame hidden,
	boxrule = 0sp,
	borderline west = {2pt}{0pt}{mypropfr},
	sharp corners,
	detach title,
	before upper = \tcbtitle\par\smallskip,
	coltitle = mypropfr,
	fonttitle = \bfseries\sffamily,
	description font = \mdseries,
	separator sign none,
	segmentation style={solid, mypropfr},
}
{th}


%================================
% CLAIM
%================================

\tcbuselibrary{theorems,skins,hooks}
\newtcbtheorem[use counter=tcb@cnt@Theorem]{claim}{Claim}
{%
	enhanced
	,breakable
	,colback = myg!10
	,frame hidden
	,boxrule = 0sp
	,borderline west = {2pt}{0pt}{myg}
	,sharp corners
	,detach title
	,before upper = \tcbtitle\par\smallskip
	,coltitle = myg!85!black
	,fonttitle = \bfseries\sffamily
	,description font = \mdseries
	,separator sign none
	,segmentation style={solid, myg!85!black}
}
{th}



%================================
% Exercise
%================================

\tcbuselibrary{theorems,skins,hooks}
\newtcbtheorem[use counter=tcb@cnt@Theorem]{Exercise}{Exercice}
{%
	enhanced,
	breakable,
	colback = myexercisebg,
	frame hidden,
	boxrule = 0sp,
	borderline west = {2pt}{0pt}{myexercisefg},
	sharp corners,
	detach title,
	before upper = \tcbtitle\par\smallskip,
	coltitle = myexercisefg,
	fonttitle = \bfseries\sffamily,
	description font = \mdseries,
	separator sign none,
	segmentation style={solid, myexercisefg},
}
{th}

%================================
% EXAMPLE BOX
%================================

\newtcbtheorem[use counter=tcb@cnt@Theorem]{Example}{Exemple}
{%
	colback = myexamplebg
	,breakable
	,colframe = myexamplefr
	,coltitle = myexampleti
	,boxrule = 1pt
	,sharp corners
	,detach title
	,before upper=\tcbtitle\par\smallskip
	,fonttitle = \bfseries
	,description font = \mdseries
	,separator sign none
	,description delimiters parenthesis
}
{ex}

%================================
% DEFINITION BOX
%================================

\newtcbtheorem[use counter=tcb@cnt@Theorem]{Definition}{Définition}{enhanced,
	before skip=2mm,after skip=2mm, colback=red!5,colframe=red!80!black,boxrule=0.5mm,
	attach boxed title to top left={xshift=1cm,yshift*=1mm-\tcboxedtitleheight}, varwidth boxed title*=-3cm,
	boxed title style={frame code={
					\path[fill=tcbcolback]
					([yshift=-1mm,xshift=-1mm]frame.north west)
					arc[start angle=0,end angle=180,radius=1mm]
					([yshift=-1mm,xshift=1mm]frame.north east)
					arc[start angle=180,end angle=0,radius=1mm];
					\path[left color=tcbcolback!60!black,right color=tcbcolback!60!black,
						middle color=tcbcolback!80!black]
					([xshift=-2mm]frame.north west) -- ([xshift=2mm]frame.north east)
					[rounded corners=1mm]-- ([xshift=1mm,yshift=-1mm]frame.north east)
					-- (frame.south east) -- (frame.south west)
					-- ([xshift=-1mm,yshift=-1mm]frame.north west)
					[sharp corners]-- cycle;
				},interior engine=empty,
		},
	fonttitle=\bfseries,
	title={#2},#1}{def}

%================================
% Solution BOX
%================================

\makeatletter
\newtcbtheorem[use counter=tcb@cnt@Theorem]{question}{Question}{enhanced,
	breakable,
	colback=white,
	colframe=myb!80!black,
	attach boxed title to top left={yshift*=-\tcboxedtitleheight},
	fonttitle=\bfseries,
	title={#2},
	boxed title size=title,
	boxed title style={%
			sharp corners,
			rounded corners=northwest,
			colback=tcbcolframe,
			boxrule=0pt,
		},
	underlay boxed title={%
			\path[fill=tcbcolframe] (title.south west)--(title.south east)
			to[out=0, in=180] ([xshift=5mm]title.east)--
			(title.center-|frame.east)
			[rounded corners=\kvtcb@arc] |-
			(frame.north) -| cycle;
		},
	#1
}{def}
\makeatother

%================================
% SOLUTION BOX
%================================

\makeatletter
\newtcolorbox{solution}{enhanced,
	breakable,
	colback=white,
	colframe=myg!80!black,
	attach boxed title to top left={yshift*=-\tcboxedtitleheight},
	title=Solution,
	boxed title size=title,
	boxed title style={%
			sharp corners,
			rounded corners=northwest,
			colback=tcbcolframe,
			boxrule=0pt,
		},
	underlay boxed title={%
			\path[fill=tcbcolframe] (title.south west)--(title.south east)
			to[out=0, in=180] ([xshift=5mm]title.east)--
			(title.center-|frame.east)
			[rounded corners=\kvtcb@arc] |-
			(frame.north) -| cycle;
		},
}
\makeatother

%================================
% Question BOX
%================================

\makeatletter
\newtcbtheorem[use counter=tcb@cnt@Theorem]{qstion}{Question}{enhanced,
	breakable,
	colback=white,
	colframe=mygr,
	attach boxed title to top left={yshift*=-\tcboxedtitleheight},
	fonttitle=\bfseries,
	title={#2},
	boxed title size=title,
	boxed title style={%
			sharp corners,
			rounded corners=northwest,
			colback=tcbcolframe,
			boxrule=0pt,
		},
	underlay boxed title={%
			\path[fill=tcbcolframe] (title.south west)--(title.south east)
			to[out=0, in=180] ([xshift=5mm]title.east)--
			(title.center-|frame.east)
			[rounded corners=\kvtcb@arc] |-
			(frame.north) -| cycle;
		},
	#1
}{def}
\makeatother

\newtcbtheorem[number within=chapter]{wconc}{Wrong Concept}{
	breakable,
	enhanced,
	colback=white,
	colframe=myr,
	arc=0pt,
	outer arc=0pt,
	fonttitle=\bfseries\sffamily\large,
	colbacktitle=myr,
	attach boxed title to top left={},
	boxed title style={
			enhanced,
			skin=enhancedfirst jigsaw,
			arc=3pt,
			bottom=0pt,
			interior style={fill=myr}
		},
	#1
}{def}



%================================
% NOTE BOX
%================================

\usetikzlibrary{arrows,calc,shadows.blur}
\tcbuselibrary{skins}
\newtcolorbox{note}[1][]{%
	enhanced jigsaw,
	colback=gray!20!white,%
	colframe=gray!80!black,
	size=small,
	boxrule=1pt,
	title=\colorbox{white!100}{\textbf{ Remarque }},
	halign title=flush center,
	coltitle=black,
	breakable,
	drop shadow=black!50!white,
	attach boxed title to top left={xshift=1cm,yshift=-\tcboxedtitleheight/2,yshifttext=-\tcboxedtitleheight/2},
	minipage boxed title=2.6cm,
	boxed title style={%
			colback=white,
			size=fbox,
			boxrule=1pt,
			boxsep=2pt,
			underlay={%
					\coordinate (dotA) at ($(interior.west) + (-0.5pt,0)$);
					\coordinate (dotB) at ($(interior.east) + (0.5pt,0)$);
					\begin{scope}
						\clip (interior.north west) rectangle ([xshift=3ex]interior.east);
						\filldraw [white, blur shadow={shadow opacity=60, shadow yshift=-.75ex}, rounded corners=2pt] (interior.north west) rectangle (interior.south east);
					\end{scope}
					\begin{scope}[gray!80!black]
						\fill (dotA) circle (2pt);
						\fill (dotB) circle (2pt);
					\end{scope}
				},
		},
	#1,
}

%================================
% STRATÉGIE BOX
%================================

\usetikzlibrary{arrows,calc,shadows.blur}
\tcbuselibrary{skins}
\newtcolorbox{strategy}[1][]{%
	enhanced jigsaw,
	colback=myb!20!white,%
	colframe=gray!80!black,
	size=small,
	boxrule=1pt,
	title=\colorbox{white!100}{\textbf{ Stratégie }},
	halign title=flush center,
	coltitle=black,
	breakable,
	drop shadow=black!50!white,
	attach boxed title to top left={xshift=1cm,yshift=-\tcboxedtitleheight/2,yshifttext=-\tcboxedtitleheight/2},
	minipage boxed title=2.5cm,
	boxed title style={%
			colback=white,
			size=fbox,
			boxrule=1pt,
			boxsep=2pt,
			underlay={%
					\coordinate (dotA) at ($(interior.west) + (-0.5pt,0)$);
					\coordinate (dotB) at ($(interior.east) + (0.5pt,0)$);
					\begin{scope}
						\clip (interior.north west) rectangle ([xshift=3ex]interior.east);
						\filldraw [white, blur shadow={shadow opacity=60, shadow yshift=-.75ex}, rounded corners=2pt] (interior.north west) rectangle (interior.south east);
					\end{scope}
					\begin{scope}[gray!80!black]
						\fill (dotA) circle (2pt);
						\fill (dotB) circle (2pt);
					\end{scope}
				},
		},
	#1,
}

%================================
% MÉTHODE BOX
%================================

\usetikzlibrary{arrows,calc,shadows.blur}
\tcbuselibrary{skins}
\newtcolorbox{methode}[1][]{%
	enhanced jigsaw,
	colback=white,%
	colframe=gray!80!black,
	size=small,
	boxrule=1pt,
	title=\textbf{Méthode},
	halign title=flush center,
	coltitle=black,
	breakable,
	drop shadow=black!50!white,
	attach boxed title to top left={xshift=1cm,yshift=-\tcboxedtitleheight/2,yshifttext=-\tcboxedtitleheight/2},
	minipage boxed title=2.5cm,
	boxed title style={%
			colback=white,
			size=fbox,
			boxrule=1pt,
			boxsep=2pt,
			underlay={%
					\coordinate (dotA) at ($(interior.west) + (-0.5pt,0)$);
					\coordinate (dotB) at ($(interior.east) + (0.5pt,0)$);
					\begin{scope}
						\clip (interior.north west) rectangle ([xshift=3ex]interior.east);
						\filldraw [white, blur shadow={shadow opacity=60, shadow yshift=-.75ex}, rounded corners=2pt] (interior.north west) rectangle (interior.south east);
					\end{scope}
					\begin{scope}[gray!80!black]
						\fill (dotA) circle (2pt);
						\fill (dotB) circle (2pt);
					\end{scope}
				},
		},
	#1,
}

%%%%%%%%%%%%%%%%%%%%%%%%%%%%%%%%%%%%%%%%%%%
% TABLE OF CONTENTS
%%%%%%%%%%%%%%%%%%%%%%%%%%%%%%%%%%%%%%%%%%%

\usepackage{tikz}

\definecolor{doc}{RGB}{0,60,110}
\usepackage{titletoc}
\contentsmargin{0cm}
\titlecontents{chapter}[3.7pc]
{\addvspace{30pt}%
	\begin{tikzpicture}[remember picture, overlay]%
		\draw[fill=doc!60,draw=doc!60] (-7,-.1) rectangle (-0.2,.6);%
		\pgftext[left,x=-3.5cm,y=0.2cm]{\color{white}\Large\sc\bfseries Chapitre\ \thecontentslabel};%
	\end{tikzpicture}\color{doc!60}\large\sc\bfseries}%
{}
{}
{\;\titlerule\;\large\sc\bfseries Page \thecontentspage
	\begin{tikzpicture}[remember picture, overlay]
		\draw[fill=doc!60,draw=doc!60] (2pt,0) rectangle (4,0.1pt);
	\end{tikzpicture}}%
\titlecontents{section}[3.7pc]
{\addvspace{2pt}}
{\contentslabel[\thecontentslabel]{2pc}}
{}
{\hfill\small \thecontentspage}
[]
\titlecontents*{subsection}[3.7pc]
{\addvspace{-1pt}\small}
{}
{}
{\ --- \small\thecontentspage}
[ \textbullet\ ][]

\makeatletter
\renewcommand{\tableofcontents}{%
	\chapter*{%
	  \vspace*{-20\p@}%
	  \begin{tikzpicture}[remember picture, overlay]%
		  \pgftext[right,x=15cm,y=0.2cm]{\color{doc!60}\Huge\sc\bfseries \contentsname};%
		  \draw[fill=doc!60,draw=doc!60] (13,-.75) rectangle (20,1);%
		  \clip (13,-.75) rectangle (20,1);
		  \pgftext[right,x=15cm,y=0.2cm]{\color{white}\Huge\sc\bfseries \contentsname};%
	  \end{tikzpicture}}%
	\@starttoc{toc}}
\makeatother


%%%%%%%%%%%%%%%%%%%%%%%%%%%%%%%%%%%%%%%%%%%
% MINTED FOR PYTHON ALGORITHMS
%%%%%%%%%%%%%%%%%%%%%%%%%%%%%%%%%%%%%%%%%%%

\usepackage{tcolorbox}
\tcbuselibrary{minted,breakable,xparse,skins}
\definecolor{bg}{gray}{0.95}
\DeclareTCBListing{mintedbox}{O{}m!O{}}{%
  breakable=true,
  listing engine=minted,
  listing only,
  minted language=#2,
  minted style=default,
  minted options={%
    linenos,
    gobble=0,
    breaklines=true,
    breakafter=,,
    fontsize=\small,
    numbersep=8pt,
    #1},
  boxsep=0pt,
  left skip=0pt,
  right skip=0pt,
  left=25pt,
  right=0pt,
  top=3pt,
  bottom=3pt,
  arc=5pt,
  leftrule=0pt,
  rightrule=0pt,
  bottomrule=2pt,
  toprule=2pt,
  colback=bg,
  colframe=orange!70,
  enhanced,
  overlay={%
    \begin{tcbclipinterior}
    \fill[orange!20!white] (frame.south west) rectangle ([xshift=20pt]frame.north west);
    \end{tcbclipinterior}},
  #3}
  
  
 % for braces
\usetikzlibrary{decorations.pathreplacing}


\SetDate[15/10/2025]
\reversemarginpar
\setlength{\marginparsep}{.5cm}

\begin{document}
\pagestyle{fancy}
\fancyhead[L]{Seconde}
\fancyhead[C]{\textbf{Évaluation — Plan cartésien}}
\fancyhead[R]{\today}

\null\vspace{-30pt}
Consignes particulières : 
\begin{itemize}[label=$\bullet$]
	\item 
	La calculatrice est {interdite}. Une aide aux calculs est disponible à l'exercice \ref{exe:Trex}.
	\item
	Les sujets d'évaluation sont individuels. Écrire son nom avant de rendre son sujet.
	\item
	Les exercices \ref{exe:prénom} et \ref{exe:diagonale} peuvent être faits entièrement sur la feuille d'évaluation. 
	\item 
	On supposera l'existence d'un nombre positif noté $\sqrt{12}$ dont le carré vaut 12 pour résoudre l'exercice \ref{exe:equilateral}.
	En d'autres termes, $\sqrt{12} > 0$, et $\bigl(\sqrt{12}\bigr)^2 = 12$.
	\item
	L'évaluation fait 2 pages. La somme des points est \total{points}.
\end{itemize}

\marginpar{[pts]}
\hrule

%!TEX root = ../eval1.tex

\needspace{4cm}
\exe{2}{
	\begin{enumerate}
		%\item Placer des points dans le repère ci-dessous tels que, lorsque reliés adéquatement, on puisse lire la première lettre de votre prénom.
		\item Placer des points dans le repère et les relier afin qu'on puisse lire la première lettre de votre prénom.
		\item Donner les coordonnées de chaque point placé.
	\end{enumerate}
}{exe:prénom}{
	Le prénom du correcteur commençant par $M$, celui-ci propose les points $A(-4;0), B(-4;4), C(-2,5;2,5), D(-1;4), E(-1;0)$.
	\begin{center}
	\begin{tikzpicture}[>=stealth, scale=.8]
		\begin{axis}[xmin = -5.1, xmax=1.1, ymin=-1.1, ymax=5.1, axis x line=middle, axis y line=middle, axis line style=<->, xlabel={}, ylabel={}, grid=both, grid style = {opacity=.5}, clip=false, xtick distance = 2, ytick distance=1, x=2cm, y=1cm]
			\addplot[BLUE_E, mark=*, mark size = 1] (-4,0) node[above left] {$A(-4;0)$};
			\addplot[RED_E, mark=*, mark size = 1] (-4,4) node[above left] {$B(-4;4)$};
			\addplot[GREEN_E, mark=*, mark size = 1] (-2.5,2.5) node[below] {$C(-2,5;2,5)$};
			\addplot[BLUE_E, mark=*, mark size = 1] (-1,4) node[above right] {$D(-1;4)$};
			\addplot[RED_E, mark=*, mark size = 1] (-1,0) node[above right] {$E(-1;0)$};
			
			\draw[-, thick,dashed] (axis cs:-4,0) -- (axis cs:-4,4);
			\draw[-, thick, dashed] (axis cs:-4,4) -- (axis cs:-2.5,2.5);
			\draw[-, thick,dashed] (axis cs:-2.5,2.5) -- (axis cs:-1,4);
			\draw[-, thick,dashed] (axis cs:-1,4) -- (axis cs:-1,0);
		\end{axis}
	\end{tikzpicture}
	\end{center}
}


	\begin{center}
	\begin{tikzpicture}[>=stealth, scale=.8]
		\begin{axis}[xmin = -5.1, xmax=1.1, ymin=-1.1, ymax=5.1, axis x line=middle, axis y line=middle, axis line style=<->, xlabel={}, ylabel={}, grid=both, grid style = {opacity=.5}, clip=false, xtick distance = 2, ytick distance=1, x=2cm, y=1cm]
		\end{axis}
	\end{tikzpicture}
	\end{center}
	

%!TEX root = ../eval1.tex

\exemulticols{3}{
	Donner approximativement les coordonnées de chaque point du repère ci-contre.
	\begin{align*}
		&A\hspace{3cm} \\[10pt]
		&B \\[10pt]
		&C \\[10pt]
		&D \\[10pt]
		&E \\[10pt]
		&O
	\end{align*}
	%\vfill\null
}{
	\begin{center}
	\begin{tikzpicture}[>=stealth, scale=1.2]
		\begin{axis}[xmin = -4.9, xmax=4.9, ymin=-4.9, ymax=4.9, axis x line=middle, axis y line=middle, axis line style=<->, xlabel={}, ylabel={}, grid=both, grid style = {opacity=.5}, xtick distance=1, ytick distance=1]			
			\addplot[BLUE_E, mark=*, mark size = 1] (2,0) node[above] {$A$};
			\addplot[RED_E, mark=*, mark size = 1] (-3,3) node[above left] {$B$};
			\addplot[GREEN_E, mark=*, mark size = 1] (3,1.5) node[above right] {$C$};
			\addplot[PURPLE_E, mark=*, mark size = 1] (0,-1.5) node[right] {$D$};
			\addplot[GOLD_E, mark=*, mark size = 1] (-2.5,-3) node[left] {$E$};
			\addplot[BLACK, mark=*, mark size = 1] (0,0) node[above left] {$O$};
		\end{axis}
	\end{tikzpicture}
	\end{center}
}{exe:lecture-coord}{
	\begin{align*}
		A(3 ; 0) && B(-1 ; 3) && C(3 ; 3) &&
		D(0;-2) && E(-2,5 ; 0)
	\end{align*}
}


%!TEX root = ../eval1.tex

\exe{{3}, difficulty=1}{
	Vrai ou faux ? Cocher la case correspondante.
	\vspace{-30pt}
	\begin{center}
	\def\arraystretch{1.5}
	\setlength\tabcolsep{15pt}
	\begin{tabular}{c c c}
		\hspace{10cm} & Vrai & Faux \\
		$\frac12 = 0,5$ & $\square$ & $\square$  \\
		$\frac13 = 0,33$ & $\square$ & $\square$  \\
		$\frac13 = 0,33333$ & $\square$ & $\square$  \\ %\vspace{-5pt}
		$1$ est un nombre rationnel & $\square$ & $\square$  \\
		Tous les nombres réels sont rationnels & $\square$ & $\square$ \\
		L'encadrement $-0,334 < -\frac13 < -0,333$ est à $10^{-3}$ près  & $\square$ & $\square$ 
	\end{tabular}
	\end{center}
	\vspace{-10pt}
}{exe:2}{
	\begin{center}
	\begin{tabular}{c c c}
		\hspace{10cm} & Vrai & Faux \\
		$\dfrac12 = 0,5$ & $\times$ &  \\
		$\dfrac13 = 0,33$ & & $\times$  \\
		$\dfrac13 = 0,33333$ & & $\times$  \\ \vspace{-5pt}
		$1$ est un nombre rationnel & $\times$ &  \\
		Tous les nombres réels sont rationnels & & $\times$  \\
		L'encadrement $-0,334 < -\dfrac13 < -0,333$ est à $10^{-3}$ près  & $\times$ &
	\end{tabular}
	\end{center}
}

%!TEX root = ../eval1.tex

\exe{2}{
	Considérons les points
		\begin{align*}
			\point{A}{\dfrac{10}3}{-\dfrac76}, && \point{B}{\dfrac{13}3}{\dfrac{11}6}, && \point{C}{\dfrac{-5}3}{\dfrac{17}6}, && \point{D}{-\dfrac23}{\dfrac{35}6}.
		\end{align*}
	Montrer que le quadrilatère $CDBA$ est un parallélogramme en comparant le milieu de ses deux diagonales.
	
	\emph{Un parallélogramme est un quadrilatère dont les diagonales se coupent en leur milieu}.
}{exe:parallélogramme}{
	Le milieu du segment $[BC]$ est donné par
		\[ M = \dfrac12 (B+C) = (1;2,5),\]
	et le milieu du segment $[AD]$ est donné par
		\[ M' = \dfrac12 (A+D) = (1;2,5).\]
	Comme $M' = M$, le quadrilatère est bien un parallélogramme.

	\centering
	\begin{tikzpicture}[>=stealth, scale=1]
	\begin{axis}[xmin = -2.1, xmax=4.1, ymin=-1.1, ymax=6.1, axis x line=middle, axis y line=middle, axis line style=<->, xlabel={}, ylabel={}, grid=both, clip=false, ytick distance=1]
					
		\addplot[RED_E, mark=*, mark size = 1] (3,-1) node[right] {$A$};
		\addplot[RED_E, mark=*, mark size = 1] (4,2) node[right] {$B$};
		\addplot[RED_E, mark=*, mark size = 1] (-2,3) node[left] {$C$};
		\addplot[RED_E, mark=*, mark size = 1] (-1,6) node[left] {$D$};
		
		\draw[dashed, thick] (axis cs:3,-1) -- (axis cs:-1,6);
		\draw[dashed, thick] (axis cs:4,2) -- (axis cs:-2,3);
		
		\addplot[black, mark=*, mark size = 1] (1,2.5) node[above right] {$M$};
		
		\draw[thick] (axis cs:3,-1) -- (axis cs:4,2) -- (axis cs:-1,6) -- (axis cs:-2,3) -- (axis cs:3,-1);
	\end{axis}
	\end{tikzpicture}
}

%!TEX root = ../eval1.tex

\exe{1, difficulty=1}{
	On considère comme admis que le nombre $\pi$, périmètre d'un cercle de diamètre 1, n'est pas rationnel.
	Montrer que le nombre $\pi + 2$ n'est pas rationnel non plus.
}{exe:irr-stable-pi}{
	Si $\pi+2 = \frac{a}b$ est rationnel, alors $\pi = \frac{a-2b}{b}$ doit l'être aussi.
	Ceci contredit l'énoncé.

	Alternativement, le développement décimal de $\pi=3,1415...$ est non périodique, d'après le cours, et par hypothèse.
	Ajouter 2 ne le change pas : $\pi+2=5,1415...$. Le développement décimal de $\pi+2$ est donc également apériodique, et c'est un nombre irrationnel.
}

\newpage

%!TEX root = ../eval1.tex

\newcommand{\ordonnee}{\number\numexpr2+\exeVI\relax}

\exe{1, difficulty=2}{
	L'origine $O(0 ; 0)$ appartient-elle au segment $[AB]$ où $A(\number\numexpr3-\exeVI\relax;\ordonnee)$ et $B(\number\numexpr-4+\exeVI\relax;\ordonnee)$ ? Justifier.
}{exe:C-on-AB}{
	Les deux points ont la même ordonnée : ils sont donc alignés horizontalement.
	Tous les points de $[AB]$ sont d'ordonnée $\ordonnee$, et l'origine du repère ne peut donc pas appartenir au segment.
	En effet, $\lambda A + (1-\lambda)B$ est d'ordonnée $\ordonnee\lambda + (1-\lambda)\ordonnee = \ordonnee$, peu importe la valeur de $\lambda\in\R·$.
}

%!TEX root = ../eval1.tex

\newcommand{\xAVII}{\number\numexpr3 + 2*\exeVII-8\relax}
\newcommand{\yAVII}{\number\numexpr2+2*\exeVII-8\relax}
\newcommand{\xMVII}{\number\numexpr-1+\exeVII-4\relax}
\newcommand{\yMVII}{\number\numexpr3+\exeVII-4\relax}

\exe{2, difficulty=1}{
	Soient $\point{A}{\xAVII}{\yAVII}$ et $\point{M}{\xMVII}{\yMVII}$ deux points du plan.
	Quel point $B$ faut-il choisir pour que $M$ soit le milieu du segment $[AB]$ ? Donner ses coordonnées.
}{exe:milieu4}{
	La contrainte que $M$ soit le milieu de $[AB]$ s'écrit
		\begin{align*}
			M = \dfrac12(A+B) && \iff && 2M = A+B && \iff && B = 2M-A
		\end{align*}
	Par suite,
		\[ B= 2M-A = \left(-5;4\right). \]
}

%!TEX root = ../eval1.tex

\renewcommand{\exeVIII}{3}

\newcommand{\xAVIII}{\number\numexpr3 + 2*\exeVIII-8\relax}
\newcommand{\yAVIII}{\number\numexpr2+2*\exeVIII-8\relax}
\newcommand{\xMVIII}{\number\numexpr-1+\exeVIII-4\relax}
\newcommand{\yMVIII}{\number\numexpr3+\exeVIII-4\relax}

\exe{2, difficulty=1}{
	Soient $\point{A}{\xAVIII}{\yAVIII}$ et $\point{M}{\xMVIII}{\yMVIII}$ deux points du plan.
	Quel point $B$ faut-il choisir pour que $M$ soit le milieu du segment $[AB]$ ? Donner ses coordonnées.
}{exe:milieu2}{
	La contrainte que $M$ soit le milieu de $[AB]$ s'écrit
		\begin{align*}
			M = \dfrac12(A+B) && \iff && 2M = A+B && \iff && B = 2M-A
		\end{align*}
	Par suite,
		\[ B= 2M-A = \left(-5;4\right). \]
}

%!TEX root = ../eval1.tex

\exe{3, difficulty=1}{
	Donner le développement décimal de $\frac16$.
	En déduire un encadrement de $\frac16$ à $10^{-2}$ près.
}{exe:dev-16}{
	$\frac16 < 1$, donc $\frac16 = 0,...$.
	Pour connaître la première décimale, on multiplie par 10, et on étudie l'unité.
	
	$\frac{10}6 = 1 + \frac46$, donc la première décimale est 1.
	On répère sur le reste : $\frac{40}6 = 6 + \frac46$, et la deuxième décimale est 6.
	On se convainc facilement que toutes les décimales après sont aussi 6, car le reste est toujours $\frac46$.
	
	En conclusion, $\frac16 = 0,166666...$, les 6 se répétant à l'infini.
}

%%%%%%%%%%%%

%\newpage
%\fancyhead[C]{\textbf{Solutions}}
%\shipoutAnswer

\end{document}

% DYSLEXIA SWITCH
\newif\ifdys
		
				% ENABLE or DISABLE font change
				% use XeLaTeX if true
				\dystrue
				\dysfalse


\ifdys

\documentclass[a4paper, 14pt]{extarticle}
\usepackage{amsmath,amsfonts,amsthm,amssymb,mathtools}

\tracinglostchars=3 % Report an error if a font does not have a symbol.
\usepackage{fontspec}
\usepackage{unicode-math}
\defaultfontfeatures{ Ligatures=TeX,
                      Scale=MatchUppercase }

\setmainfont{OpenDyslexic}[Scale=1.0]
\setmathfont{Fira Math} % Or maybe try KPMath-Sans?
\setmathfont{OpenDyslexic Italic}[range=it/{Latin,latin}]
\setmathfont{OpenDyslexic}[range=up/{Latin,latin,num}]

\else

\documentclass[a4paper, 12pt]{extarticle}

\usepackage[utf8x]{inputenc}
%fonts
\usepackage{amsmath,amsfonts,amsthm,amssymb,mathtools}
% comment below to default to computer modern
\usepackage{libertinus,libertinust1math}

\fi


\usepackage[french]{babel}
\usepackage[
a4paper,
margin=2cm,
nomarginpar,% We don't want any margin paragraphs
]{geometry}
\usepackage{icomma}

\usepackage{fancyhdr}
\usepackage{array}
\usepackage{hyperref}

\usepackage{multicol, enumerate}
\newcolumntype{P}[1]{>{\centering\arraybackslash}p{#1}}


\usepackage{stackengine}
\newcommand\xrowht[2][0]{\addstackgap[.5\dimexpr#2\relax]{\vphantom{#1}}}

% theorems

\theoremstyle{plain}
\newtheorem{theorem}{Th\'eor\`eme}
\newtheorem*{sol}{Solution}
\theoremstyle{definition}
\newtheorem{ex}{Exercice}
\newtheorem*{rpl}{Rappel}
\newtheorem{enigme}{Énigme}

% corps
\usepackage{calrsfs}
\newcommand{\C}{\mathcal{C}}
\newcommand{\R}{\mathbb{R}}
\newcommand{\Rnn}{\mathbb{R}^{2n}}
\newcommand{\Z}{\mathbb{Z}}
\newcommand{\N}{\mathbb{N}}
\newcommand{\Q}{\mathbb{Q}}

% variance
\newcommand{\Var}[1]{\text{Var}(#1)}

% domain
\newcommand{\D}{\mathcal{D}}


% date
\usepackage{advdate}
\AdvanceDate[0]


% plots
\usepackage{pgfplots}

% table line break
\usepackage{makecell}
%tablestuff
\def\arraystretch{2}
\setlength\tabcolsep{15pt}

%subfigures
\usepackage{subcaption}

\definecolor{myg}{RGB}{56, 140, 70}
\definecolor{myb}{RGB}{45, 111, 177}
\definecolor{myr}{RGB}{199, 68, 64}

% fake sections with no title to move around the merged pdf
\newcommand{\fakesection}[1]{%
  \par\refstepcounter{section}% Increase section counter
  \sectionmark{#1}% Add section mark (header)
  \addcontentsline{toc}{section}{\protect\numberline{\thesection}#1}% Add section to ToC
  % Add more content here, if needed.
}


% SOLUTION SWITCH
\newif\ifsolutions
				\solutionstrue
				%\solutionsfalse

\ifsolutions
	\newcommand{\exe}[2]{
		\begin{ex} #1  \end{ex}
		\begin{sol} #2 \end{sol}
	}
\else
	\newcommand{\exe}[2]{
		\begin{ex} #1  \end{ex}
	}
	
\fi


% tableaux var, signe
\usepackage{tkz-tab}


%pinfty minfty
\newcommand{\pinfty}{{+}\infty}
\newcommand{\minfty}{{-}\infty}

\begin{document}


\AdvanceDate[0]

\begin{document}
\pagestyle{fancy}
\fancyhead[L]{Seconde}
\fancyhead[C]{\textbf{Plan cartésien}}
\fancyhead[R]{\today}

\exemulticols{}{
	Donner approximativement les coordonnées de chaque point du repère ci-contre.
	\begin{align*}
		&A(\phantom{2} ; \phantom{3}) \\[10pt]
		&B \\[10pt]
		&C \\[10pt]
		&D \\[10pt]
		&E
	\end{align*}
	\vfill\null
}{
	\begin{center}
	\begin{tikzpicture}[>=stealth, scale=1.2]
		\begin{axis}[xmin = -2.9, xmax=4.9, xtick={ -3, ..., 5}, ymin=-2.9, ymax=4.9, ytick={-3, ..., 5}, axis x line=middle, axis y line=middle, axis line style=<->, xlabel={}, ylabel={}, grid=both, grid style = {opacity=.5}]			
			\addplot[BLUE_E, mark=*, mark size = 1] (2,3) node[above right] {$A$};
			\addplot[RED_E, mark=*, mark size = 1] (-1,2) node[above left] {$B$};
			\addplot[GREEN_E, mark=*, mark size = 1] (2.5,-1) node[below right] {$C$};
			\addplot[PURPLE_E, mark=*, mark size = 1] (0,-2) node[right] {$D$};
			\addplot[GOLD, mark=*, mark size = 1] (-2.5,0) node[above] {$E$};
		\end{axis}
	\end{tikzpicture}
	\end{center}
}{exe:lecture-coord}{
	\begin{align*}
		A(2 ; 3) && B(-1 ; 2) && C(2,5 ; -1) &&
		D(0;-2) && E(-2,5 ; 0)
	\end{align*}
}


\exe{}{
	Donner le carré de la norme des points $(0 ; 0)$, $(3 ; 4)$, $(-3 ; 4)$, $(-1; -1)$.
}{exe:norme}{
	\begin{align*}
		&\norm{(0 ; 0)}^2 = 0^2 + 0^2 = 0, && \norm{(3 ; 4)}^2 = 3^2 + 4^2 = 9 + 16 = 25, \\
		&\norm{(-3 ; -4)}^2 = (-3)^2 + (-4)^2 = 9 + 16 = 25, && \norm{(1 ; -1)}^2 = 1^2 + (-1)^2 = 2.
	\end{align*}
}

\exe{}{
	Tracer un repère et y placer les points suivants. Il est recommandé de calculer les coordonnées des points à placer avant de tracer le repère afin de décider d'un pas adéquat.
	\begin{multicols}{3}
	\begin{enumerate}[leftmargin=50pt]
		\item $\point{A}{2}{3}$
		\item $\point{B}{-1}{3}$
		\item $C = A+B$
		\item $D=A-B$
		\item $E=\frac12A$
		\item $F=-3B$
	\end{enumerate}
	\end{multicols}
}{exe:points-à-placer}{
	\begin{center}
	\begin{tikzpicture}[>=stealth, scale=1]
		\begin{axis}[xmin = -1.5, xmax=3.5, ymin=-9.5, ymax=6.5, ytick = {-8, -6, ..., 6}, xtick = {-1, 0, ..., 3}, axis x line=middle, axis y line=middle, axis line style=<->, xlabel={}, ylabel={}, grid=both, x=2cm]
						
			\addplot[RED_E, mark=*, mark size = 1] (2,3) node[right] {$A$};
			\addplot[RED_E, mark=*, mark size = 1] (-1,3) node[left] {$B$};
			\addplot[RED_E, mark=*, mark size = 1] (1,6) node[below] {$C$};
			\addplot[RED_E, mark=*, mark size = 1] (3,0) node[above] {$D$};
			\addplot[RED_E, mark=*, mark size = 1] (1,1.5) node[left] {$E$};
			\addplot[RED_E, mark=*, mark size = 1] (3,-9) node[right] {$F$};
			
		\end{axis}
	\end{tikzpicture}
	\end{center}
}

\exe{}{
	On considère le triangle de sommets $A(1;2), B(-3 ; 5), C(-5 ; -6)$.
	\begin{enumerate}
		\item Tracer le triangle $ABC$ dans un repère.
		\item Calculer le carré de chacun des côtés. 
		\item Que dire du triangle ?
	\end{enumerate}
}{exe:Trex}{
	\begin{multicols}{2}
	\begin{center}
	\begin{tikzpicture}[>=stealth, scale=1]
		\begin{axis}[xmin = -5.1, xmax=1.1, ymin=-6.1, ymax=5.1,axis x line=middle, axis y line=middle, axis line style=<->, xlabel={}, ylabel={}, grid=both, grid style = {opacity=.5}, x=20pt, y=20pt]			
			\addplot[BLUE_E, mark=*, mark size = 1] (1,2) node[above right] {$A$};
			\addplot[RED_E, mark=*, mark size = 1] (-3,5) node[above left] {$B$};
			\addplot[GREEN_E, mark=*, mark size = 1] (-5,-6) node[below right] {$C$};
			
			\draw[-, thick] (axis cs:1,2) -- (axis cs:-3,5) -- (axis cs:-5,-6) -- (axis cs:1,2);
		\end{axis}
	\end{tikzpicture}
	\end{center}
	
	\begin{enumerate}
		\item[2.] 
			\begin{align*}
				AB^2 &= \norm{A-B}^2 = \norm{(4 ; -3)}^2 = 16 + 9 = 25, \\
				AC^2 &= \norm{A-C}^2 = \norm{(6 ; 8)}^2 = 36 + 64 = 100, \\
				BC^2 &= \norm{B-C}^2 = \norm{(2 ; 11)}^2 = 4 + 121 = 125.
			\end{align*}
		\item[3.] D'après la réciproque du théorème de Pythagore, le triangle est rectangle en $A$ car $BC^2 = AB^2 + AC^2$.
	\end{enumerate}
	\end{multicols}
}

\exe{}{
	Tracer dans un repère l'ensemble des points à distance 3 de l'origine.
}{exe:C03}{
	Cet ensemble forme un cercle de rayon 3 centré en l'origine.
	\begin{center}
	\begin{tikzpicture}[>=stealth, scale=1]
		\begin{axis}[xmin = -3.1, xmax=3.1, ymin=-3.1, ymax=3.1,axis x line=middle, axis y line=middle, axis line style=<->, xlabel={}, ylabel={}, grid=both, grid style = {opacity=.5}, x=30pt, y=30pt]
			\draw[BLUE_E, thick,radius=90pt] (axis cs:0,0) circle node[pos=0, above right] {$\Vert x \Vert = 3$};
		\end{axis}
	\end{tikzpicture}
	\end{center}
}

\exe{, difficulty=2}{
	Combien de points à coordonnées entières du plan sont à distance $5$ de l'origine du repère ?
}{exe:norm5}{
	L'exercice \ref{exe:norme} nous en donne déjà 2 : $(3 ; 4)$ et $(-3 ; -4)$, dont la norme vaut $25 = 5^2$.
	On en déduit facilement que $(-3 ; 4)$ et $(3 ; -4)$ fonctionnent aussi, car le carré ignore le signe.
	
	En échangeant $x$ et $y$, on trouve aussi les points $(\pm4 ; \pm3)$, où $\pm$ signifie « plus ou moins ».
	
	Cherchons-en d'autres : $(x ; y)$ doit vérifier $x^2 + y^2 = 25$ avec $x, y$ entiers.
	En testant des valeurs de $x=0 ; 1 ; 2 ; 5$, on trouve les points $(0 ; 5), (0 ; -5)$, et $(5 ; 0), (-5 ; 0)$.
	
	Il existe donc finalement 12 tels points.
}


\exe{, difficulty=1}{
	Soient $\point{G}{-4}{-1}$ et $\point{D}{-1}{3}$ et $x\in\R$ un paramètre réel.
	Pour quel(s) $x\in\R$ est-ce que la longueur du segment entre les points $xG$ et $xD$ est-elle égale à $5$ ?
}{exe:milieu-x}{
	La longueur du segment est donnée par
		\[ 5 = \sqrt{(-4x + x)^2 + (-x - 3x)^2} = \sqrt{9x^2 + 16x^2} = \sqrt{25x^2}. \]
	En mettant l'équation au carré, on trouve
		\begin{align*}
			25 = 25x^2 && \iff &&  x^2 = 1.
		\end{align*}
	D'où on trouve les solutions $x=1$ ou $x=-1$.
}

\begin{remarque}
	On supposera l'existence d'un nombre positif noté $\sqrt3$ dont le carré vaut 3 pour résoudre les exercices suivants.
	En d'autres termes, $\sqrt3 > 0$, et $\bigl(\sqrt3\bigr)^2 = 3$.
\end{remarque}

\exe{}{
	Considérons les points $\point{A}{1}{1}, \point{B}{3}{1}, C\bigl(2 ; \sqrt{3}+1\bigr)$.
	Démontrer que le triangle $ABC$ est équilatéral en calculant le carré de la longueur de chaque côté.
	
	\emph{Un triangle équilatéral est un triangle dont les trois côtés ont la même longueur}.
}{exe:équilatéral}{
	On calcule 
		\begin{align*}
			AB^2 = {2^2 + 0^2} = 4, && AC^2 = {1^2 + \sqrt{3}^2} = 4, && BC^2 = 4.
		\end{align*}
}

\exe{}{
	Montrer que $(-2)^2 = 4$ et en déduire qu'il n'existe pas qu'une seule solution réelle à l'équation $x^2 = 4$.
}{exe:carre4}{
	Par définition, $(-2)^2 = (-2) \cdot (-2) = 4$.
	Comme $2^2 = 4$, il existe au moins deux solutions réelles à l'équation $x^2 = 4$ : 2 et $-2$.
}

\exe{, difficulty=2}{
	Montrer que $x^2 - 4 = (x-2)(x+2)$ pour tout $x\in\R$ et en déduire les deux seules solutions réelles de l'équation $x^2 = 4$.
}{exe:carre4-all}{
	Par distributivité, $(x-2)(x+2) = x^2 - 4$.
	Il suit que $x^2 = 4 \iff x^2 - 4 = 0 \iff (x-2)(x+2) = 0$.
	Or le produit de deux nombre n'est nul que si l'un des deux est nul.
	On a donc soit $x-2=0 \iff x=2$, soit $x+2=0 \iff x=-2$.
}

\exe{, difficulty=2}{
	Soient $A(1;1), B(3;1)$ deux points du plan, et $C(2;x)$ un point dépendant d'un paramètre réel $x\in\R$.
	Pour quel(s) $x\in\R$ le triangle $ABC$ est-il équilatéral ? 
}{exe:équilatéral2}{
	On souhaite que $AB = AC = BC$ soit vérifié, c'est-à-dire que
		\[ 2 = \sqrt{1 + (1-x)^2} = \sqrt{1 + (1-x)^2}. \]
	La deuxième égalité est redondante et, en mettant au carré, on trouve
		\begin{align*}
		4 = 1 + (1-x)^2 && \iff && (1-x)^2 = 3.
		\end{align*}
	En reprenant l'exercice \ref{exe:carre4-all}, on trouve deux solutions :
	\vspace{-20pt}
		\begin{multicols}{2}
		\begin{align*}
			1-x &= \sqrt{3}, \\
			x &= 1-\sqrt{3}.
		\end{align*}
			
		\begin{align*}
			1-x &= -\sqrt{3}, \\
			x &= 1 +\sqrt{3}.
		\end{align*}
		\end{multicols}
}{}


\exe{}{
	Représenter les points $A(1;1)$ et $B(3;-1)$ dans un repère orthonormé.
	Représenter le point
		\[ \lambda A + (1-\lambda)B, \]
	pour certaines valeurs de $\lambda$ (lu « lambda ») entre 0 et 1.
	
	Quel $\lambda$ choisir pour obtenir 
		\begin{multicols}{2}
		\begin{itemize}
			\item le point $A$ ?
			\item le point $B$ ?
			%\item le milieu du segment $[AB]$ ?
			\item le point $C\left(\dfrac32; \dfrac12\right)$ ?
		\end{itemize}
		\end{multicols}
}{exe:milieu-segment}{
	On part de $\lambda=0$ et on augment petit à petit vers 1.
	\begin{itemize}
		\item
		En $\lambda = 0$, on trouve $B$.
		\item
		En $\lambda = 0,25$, on trouve $0,25A + 0,75B = (0,25 ; 0,25) + (2,25 ; -0,75) = (2,5 ; -0,5)$
		\item
		En $\lambda = 0,5$, on trouve $0,5 A + 0,5 B = (0,5 + 1,5 ; 0,5 - 0,5) = (2 ;0)$
		\item
		En $\lambda = 0,75$, on trouve $0,75A + 0,25B = (0,75 ; 0,75) + (0,75 ; -0,25) = (1,5 ; 0,5) = C$
		\item
		En $\lambda=1$, on trouve $A$.
	\end{itemize}
	On trace en fait le segment $[AB]$ en choisissant $\lambda$ entre 0 et 1.
	
	\centering
	\begin{tikzpicture}[>=stealth, scale=1]
		\begin{axis}[xmin = -0.1, xmax=3.1, ymin=-1.1, ymax=1.1,axis x line=middle, axis y line=middle, axis line style=<->, xlabel={}, ylabel={}, grid=both, grid style = {opacity=.5}]
			\addplot[BLUE_E, mark=*, mark size = 1] (1,1) node[left] {$A$};
			\addplot[RED_E, mark=*, mark size = 1] (3,-1) node[right] {$B$};
			\draw[-, thick] (axis cs:1,1) -- (axis cs:3,-1);
		\end{axis}
	\end{tikzpicture}
}

\exe{}{
	Considérons les points
		\begin{align*}
			\point{A}{0}{1}, && \point{B}{-3}{0}, && \point{C}{1}{-2}, && \point{D}{-2}{-3}.
		\end{align*}
	Démontrer que le quadrilatère $BACD$ est un parallélogramme en comparant le milieu de ses deux diagonales.
	
	\emph{Un parallélogramme est un quadrilatère dont les diagonales se coupent en leur milieu}.
}{exe:parallélogramme}{
	Le milieu du segment $[BC]$ est donné par
		\[ M = \dfrac12 (B+C) = (-1;-1),\]
	et le milieu du segment $[AD]$ est donné par
		\[ M' = \dfrac12 (A+D) = (-1;-1) = M.\]
	Le quadrilatère est donc bien un parallélogramme.

	\centering
	\begin{tikzpicture}[>=stealth, scale=1]
		\begin{axis}[xmin = -.1, xmax=2.1, xtick={ -3, ..., 5}, ymin=-.1, ymax=3.1, ytick={-3, ..., 5}, axis x line=middle, axis y line=middle, axis line style=<->, xlabel={}, ylabel={}, grid=both, grid style = {opacity=.5}, clip=false, x=5cm]
			
			\addplot[BLUE_E, mark=*, mark size = 1] (0,0) node[above left] {$A(0;0)$};
			\addplot[RED_E, mark=*, mark size = 1] (0,3) node[above left] {$B(0;3)$};
			\addplot[GREEN_E, mark=*, mark size = 1] (1,2) node[below] {$C(1;2)$};
			\addplot[BLUE_E, mark=*, mark size = 1] (2,3) node[above right] {$D(2;3)$};
			\addplot[RED_E, mark=*, mark size = 1] (2,0) node[above right] {$E(2;0)$};
			
			\draw[-, thick,dashed] (axis cs:0,0) -- (axis cs:0,3);
			\draw[-, thick, dashed] (axis cs:0,3) -- (axis cs:1,2);
			\draw[-, thick,dashed] (axis cs:1,2) -- (axis cs:2,3);
			\draw[-, thick,dashed] (axis cs:2,3) -- (axis cs:2,0);
		\end{axis}
	\end{tikzpicture}
}

\exe{, difficulty=1}{
	Soient $\point{A}12$ et $\point{M}3{-1}$ deux points du plan.
	Quel point $C$ faut-il choisir pour que $M$ soit le milieu du segment $[AC]$ ? Donner ses coordonnées.
}{exe:milieu2}{
	La contrainte que $M$ soit le milieu de $[AC]$ s'écrit
		\begin{align*}
			M = \dfrac12(A+C) && \iff && 2M = A+C && \iff && C = 2M-A
		\end{align*}
	Par suite,
		\[ C= 2M-A = 2\cdot\left(-\dfrac53;-1\right) - \left(-\dfrac32;\dfrac53\right) = \left(-\dfrac{11}6; -\dfrac{11}3\right). \]
	
}

\exe{, difficulty=1}{
	Soient $\point{A}{-\frac32}{\frac53}$ et $\point{M}{-\frac53}{-1}$ deux points du plan.
	Quel point $C$ faut-il choisir pour que $M$ soit le milieu du segment $[AC]$ ? Donner ses coordonnées.
}{exe:milieu2}{
	La contrainte que $M$ soit le milieu de $[AC]$ s'écrit
		\begin{align*}
			M = \dfrac12(A+C) && \iff && 2M = A+C && \iff && C = 2M-A
		\end{align*}
	Par suite,
		\[ C= 2M-A = 2\cdot\left(-\dfrac53;-1\right) - \left(-\dfrac32;\dfrac53\right) = \left(-\dfrac{11}6; -\dfrac{11}3\right). \]
	
}

\exe{, difficulty=2}{
	L'origine $O(0 ; 0)$ appartient-elle au segment $[AB]$ où $A(3;1)$ et $B(-4;-1)$ ?
}{exe:C-on-AB}{
	On souhaite savoir si $O = \lambda A + (1-\lambda)B$ pour un certain $\lambda$ entre 0 et 1.
	C'est-à-dire, si 
		\[ (0 ; 0) = \lambda(3 ;1) + (1-\lambda)(-4;-1) = \bigl(3\lambda - 4(1-\lambda) ; \lambda - (1-\lambda) \bigr) = \bigl(7\lambda - 4 ; 2 \lambda - 1 \bigr). \]
	
	D'une part, en $x$, il faut que $7\lambda- 4 = 0 \iff \lambda = \frac47$.
	D'autre part, en $y$, il faut que $2\lambda - 1 = 0 \iff \lambda = \frac12$.
	
	Comme ces deux valeurs ne sont pas les mêmes, un seul $\lambda$ ne peut pas fonctionner, et $O$ n'appartient pas au segment $[AB]$.
}

%%%%%%%%%%%

\newpage
\fancyhead[C]{\textbf{Solutions}}
\shipoutAnswer

\end{document}

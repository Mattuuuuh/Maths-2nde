%% INPUT PREAMBLE.TEX
%% THEN INPUT VARS_{i}.ADR
%% THEN RUN THIS
%% DYSLEXIA SWITCH
\newif\ifdys
		
				% ENABLE or DISABLE font change
				% use XeLaTeX if true
				\dystrue
				\dysfalse


\ifdys

\documentclass[a4paper, 14pt]{extarticle}
\usepackage{amsmath,amsfonts,amsthm,amssymb,mathtools}

\tracinglostchars=3 % Report an error if a font does not have a symbol.
\usepackage{fontspec}
\usepackage{unicode-math}
\defaultfontfeatures{ Ligatures=TeX,
                      Scale=MatchUppercase }

\setmainfont{OpenDyslexic}[Scale=1.0]
\setmathfont{Fira Math} % Or maybe try KPMath-Sans?
\setmathfont{OpenDyslexic Italic}[range=it/{Latin,latin}]
\setmathfont{OpenDyslexic}[range=up/{Latin,latin,num}]

\else

\documentclass[a4paper, 12pt]{extarticle}

\usepackage[utf8x]{inputenc}
%fonts
\usepackage{amsmath,amsfonts,amsthm,amssymb,mathtools}
% comment below to default to computer modern
\usepackage{libertinus,libertinust1math}

\fi


\usepackage[french]{babel}
\usepackage[
a4paper,
margin=2cm,
nomarginpar,% We don't want any margin paragraphs
]{geometry}
\usepackage{icomma}

\usepackage{fancyhdr}
\usepackage{array}
\usepackage{hyperref}

\usepackage{multicol, enumerate}
\newcolumntype{P}[1]{>{\centering\arraybackslash}p{#1}}


\usepackage{stackengine}
\newcommand\xrowht[2][0]{\addstackgap[.5\dimexpr#2\relax]{\vphantom{#1}}}

% theorems

\theoremstyle{plain}
\newtheorem{theorem}{Th\'eor\`eme}
\newtheorem*{sol}{Solution}
\theoremstyle{definition}
\newtheorem{ex}{Exercice}
\newtheorem*{rpl}{Rappel}
\newtheorem{enigme}{Énigme}

% corps
\usepackage{calrsfs}
\newcommand{\C}{\mathcal{C}}
\newcommand{\R}{\mathbb{R}}
\newcommand{\Rnn}{\mathbb{R}^{2n}}
\newcommand{\Z}{\mathbb{Z}}
\newcommand{\N}{\mathbb{N}}
\newcommand{\Q}{\mathbb{Q}}

% variance
\newcommand{\Var}[1]{\text{Var}(#1)}

% domain
\newcommand{\D}{\mathcal{D}}


% date
\usepackage{advdate}
\AdvanceDate[0]


% plots
\usepackage{pgfplots}

% table line break
\usepackage{makecell}
%tablestuff
\def\arraystretch{2}
\setlength\tabcolsep{15pt}

%subfigures
\usepackage{subcaption}

\definecolor{myg}{RGB}{56, 140, 70}
\definecolor{myb}{RGB}{45, 111, 177}
\definecolor{myr}{RGB}{199, 68, 64}

% fake sections with no title to move around the merged pdf
\newcommand{\fakesection}[1]{%
  \par\refstepcounter{section}% Increase section counter
  \sectionmark{#1}% Add section mark (header)
  \addcontentsline{toc}{section}{\protect\numberline{\thesection}#1}% Add section to ToC
  % Add more content here, if needed.
}


% SOLUTION SWITCH
\newif\ifsolutions
				\solutionstrue
				%\solutionsfalse

\ifsolutions
	\newcommand{\exe}[2]{
		\begin{ex} #1  \end{ex}
		\begin{sol} #2 \end{sol}
	}
\else
	\newcommand{\exe}[2]{
		\begin{ex} #1  \end{ex}
	}
	
\fi


% tableaux var, signe
\usepackage{tkz-tab}


%pinfty minfty
\newcommand{\pinfty}{{+}\infty}
\newcommand{\minfty}{{-}\infty}

\begin{document}

%\input{adr/vars_44284.adr}
%\newcommand{\seed}{TEST}

\pagestyle{fancy}
\fancyhead[L]{Seconde 13}
\fancyhead[C]{\textbf{Devoir Maison 5 --- \seed \ifsolutions \\ Solutions  \fi}}
\fancyhead[R]{\today}

\fakesection{Devoir \seed}

\exe{
	On considère le système de deux équations à deux inconnues $x, y \in \R$ suivant.

	\[\systeme[xy]{
		\Aa x \Ab y = \ba{,},
		\Ac x \Ad y = \bb.
	}\]
	

	\begin{enumerate}
		\item Trouver la solution exacte $(x;y)$ du système en explicitant sa démarche.
		\item Vérifier que le couple $(x;y)$ trouvé est bien solution du système.
	\end{enumerate}
}{
	En faisant comme en cours et en exercices, on modifie le système pour annuler un coefficient en $x$ ou en $y$ pour obtenir $x$ ou $y$. On substitue alors la valeur obtenue dans une équation au choix pour trouver l'autre inconnue.
	Ceci nous donne alors la solution
		\[ (x ; y) = \left(\x ; \y\right), \]
	dont on vérifiera bien sûr la justesse en substituant les valeurs dans le système.
}

\exe{
	
	On considère le système de trois équations à trois inconnues $x, y, z \in \R$ suivant.

	\[\systeme{
		\AaII x \AbII y \AcII z = \baII{,}, 
		\AdII x \AeII y \AfII z = \bbII{,}, 
		\AgII x \AhII y \AiII z = \bcII.
	}\]
	
	
	\begin{enumerate}
		\item Ajouter aux équations 2 et 3 un multiple de la première équation pour annuler leur coefficient en $x$.
		\item Écrire le système équivalent obtenu et vérifier que les deux dernières équations forment le sous-système de deux équations à deux inconnues suivant.
			\[\systeme[yz]{
				\AaIII y \AbIII z = \baIII{,},
				\AcIII y \AdIII z = \bbIII.
			}\]
		\item Résoudre ce sous-système pour trouver $y$ et $z$.
		\item Trouver $x$.
		\item Vérifier que le triplet $(x;y;z)$ trouvé est bien solution du système initial.
	\end{enumerate}

}{
	Pour annuler le coefficient en $x$, on ajoute $\multI$ fois la première équation à la deuxième, et $\multII$ fois la première à la troisième.
	On obtient bien le sous-système requis, qu'on résoud sans trop de peine pour obtenir
		\[ (y; z) = \left(\yII ; \zII\right). \]
	On substitue ensuite dans une équation au choix du système initial pour trouver $x = \xII$.
	On conclut donc que l'unique solution est 
		\[ (x ; y; z) = \left(\xII ; \yII ; \zII\right), \]
	qu'on vérifiera de même en substituant dans le système originel.

	En conclusion, résoudre un système de 3 équations linéares à 3 inconnues n'est pas particulièrement plus difficile que résoudre un système de deux 2 équations linéaires à 2 inconnues.
	En faisant certaines opérations révérsibles, on peut systématiquement réduire le nombre d'inconnues de 1 et continuer. 
	Ce procédé s'appelle l'\emph{algorithme du pivot de Gauss}, ou juste l'\emph{élimination de Gauss}, portant le nom de Carl Friedrich Gauss, un des plus grands mathématiciens ayant existé.
}

\ifsolutions \newpage \fi

\exe{
	On considère le système d'inconnues $x, y\in\R$ suivant qu'on ne cherche pas à résoudre.
		\[ \systeme{
			\AaIV x \AbIV y = \baIV {,},
			\AcIV x \AdIV y = \bbIV.
		}\]
	\begin{enumerate}
		\item
		Montrer qu'une solution réelle $(x;y)$ du système
		vérifie l'égalité vectorielle
			\[ x \cdot u + y\cdot v = w, \]
		où 
			\begin{align*}
				u = \pvec{\AaV}{\AcV}, && v=\pvec{\AbV}{\AdV}, && \text{ et } && w=\pvec{\baV}{\bbV}.
			\end{align*}
		\item
		Montrer que $u$ et $v$ sont colinéaires en donnant le nombre $\lambda\in\R$ tel que $v=\lambda\cdot u$.
		\item 
		Montrer que si $(x;y)$ est une solution quelconque du système, alors $w$ est nécessairement colinéaire à $u$.
		\item 
		Montrer que $u$ et $w$ ne sont pas colinéaires et en déduire qu'il n'existe aucune solution au système.
	\end{enumerate}

}{
	\begin{enumerate}
		\item
		Si un couple $(x;y)$ vérifie le système de départ, alors on a nécessairement
			\[ x \cdot u + y \cdot v = x \pvec{\AaV}{\AcV} + y \pvec{\AbV}{\AdV} = \pvec{\Aav x \AbV y}{\AcV x \AdV y} = \pvec{\baV}{\bbV} = w, \]
		comme requis.

		\item
		On identifie en divisant les coordonnées deux à deux que $\lambda = \lambdaI$ fonctionne car
			\[ \lambdaI \cdot \pvec{\AaV}{\AcV} = \pvec{\AbV}{\AdV}. \]
		
		\item
		Si $(x;y)$ est solution du système, alors $xu + yv = w$, d'après la première question.
		D'après la deuxième question, $v = \lambdaI u$.
		Il suit donc que
			\begin{align*}
				xu + y(\lambdaI u) = w && \iff && \bigl(x + (\lambdaI)y \bigr) u = w.
			\end{align*}
		Par conséquent, $w$ est multiple de $u$ et donc, par définition, $w$ est colinéaire à $u$.

		\item
		On peut raisonner par facteurs multiplicateurs candidats qui ne sont pas égaux en divisant les premières coordonnées ensemble puis les deuxièmes.
		Plus simplement, et d'après le cours, on peut calculer le déterminant
			\[ \det(u, w) = (\AaV)\cdot(\bbV) - (\AcV)(\baV) \neq 0. \]
		Celui-ci étant non nul, $u$ et $w$ ne sont pas colinéaires.

		En conclusion, si on suppose qu'un couple quelconque $(x;y)$ est solution du système initial, alors $u$ et $w$ doivent nécessairement être colinéaires.
		Ceci n'étant pas le cas, on arrive à une contradiction : aucun couple $(x;y)$ ne peut être solution du système.
		En fait, ce système admet deux équations contradictoires.

		Remarquons qu'au lieu d'avoir fait des opérations sur les lignes du système (les équations), on a étudié les colonnes (les coordonnées des vecteurs).
		Ceci est toujours possible, même quand on a $3, 4, 5, \dots$ inconnues. Il s'agira alors d'étudier les vecteurs en 3, 4, 5 dimensions !
	\end{enumerate}

}


\end{document}

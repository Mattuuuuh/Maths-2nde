%!TEX encoding = UTF8

\documentclass[a4paper, 12pt]{extarticle}
%%%%%%%%%%% PACKAGES %%%%%%%%%%%
\usepackage[french]{babel}
\usepackage[T1]{fontenc}
\usepackage[utf8]{inputenc}

\usepackage[
	margin=2cm,
	footskip=1.5cm,
]{geometry}

%ams
\usepackage{amsmath,amsfonts,amsthm,amssymb,mathtools}

\usepackage{fancyhdr}
\usepackage{bookmark}
\usepackage{enumitem}
\usepackage[most,many,breakable]{tcolorbox}
\usepackage{varwidth,etoolbox}
\usepackage[makeroom]{cancel}
\usepackage{xcolor}
\usepackage{multicol,array}
%wider tabulars
\def\arraystretch{1.5}
\setlength\tabcolsep{15pt}
\usepackage[ruled,vlined,linesnumbered]{algorithm2e}

\usepackage{caption, subcaption}

\usepackage{marginnote}

\usepackage{makecell} %pour \thead dans tabular ex3 (aligner verticalement le coeff de proportionnalité)

%virgules
\usepackage{icomma}

% for striked out implies sign (\centernot\implies)
\usepackage{centernot}

% roman numerals for \section
\renewcommand{\thesection}{\Roman{section}} 

% tikz
\usepackage{tikz, pgfplots}

% package systeme pour les systèmes d'équations bien alignés
\usepackage{systeme}
\sysalign{r,r}
\syseqspace{3pt}
\syssignspace{3pt}

\newcommand{\sys}[2]{\systeme{#1{,}, #2.}}

%%%%%%%%%%%%%%%%%%%%%%%%%%%%
% ENVIRONMENTS
%%%%%%%%%%%%%%%%%%%%%%%%%%%%

\theoremstyle{definition}

\newtheorem{theorem}{Théorème}[chapter]
\newtheorem{corollaire}[theorem]{Corollaire}
\newtheorem{lemme}[theorem]{Lemme}
\newtheorem{proposition}[theorem]{Proposition}
\newtheorem{exercice}[theorem]{Exercice}
\newtheorem{exemple}[theorem]{Exemple}
\newtheorem{definition}[theorem]{Définition}
\newtheorem*{question}{Question}
\newtheorem*{preuve}{Preuve}
\newtheorem*{remarque}{Remarque}
\newtheorem*{strategie}{Stratégie}
\newtheorem*{methode}{Méthode}
\newtheorem*{notation}{Notation}
\newtheorem*{nomenclature}{Nomenclature}
\newtheorem{axiome}[theorem]{Axiome}
\newtheorem*{heuristique}{Heuristique}

\newtheorem*{definition*}{Définition}
\newtheorem*{lemme*}{Lemme}
\newtheorem*{proposition*}{Proposition}
\newtheorem*{theorem*}{Théorème}
\newtheorem*{corollaire*}{Corollaire}

% exercices
\usepackage[answerdelayed, lastexercise]{exercise}
\renewcommand{\ExerciseHeader}{
	\textbf{
	\theExercise.
	\theExerciseDifficulty
	}
}
\renewcommand{\DifficultyMarker}{$\star$}
\renewcommand{\AnswerHeader}{
	\textbf{
	\theExercise.
	}
}
\newcommand{\exe}[4]{
	\begin{Exercise}[title=#1, label=#3]
		#2
	\end{Exercise}
	\begin{Answer}[ref=#3]
		#4
	\end{Answer}
}

\newcommand{\exemulticols}[5]{
	\begin{multicols}{2}
	\begin{Exercise}[title=#1, label=#4]
		#2
	\end{Exercise}
	\columnbreak
		#3
	\end{multicols}
	\begin{Answer}[ref=#4]
		#5
	\end{Answer}
}


%%%%%%%%%%% LETTERFONTS %%%%%%%%%%%

\usepackage{libertinus,libertinust1math}
\usepackage[T1]{fontenc}

% for calligraphic C, D, P (important to import this after the font)
\usepackage{calrsfs}
\newcommand{\D}{\mathcal{D}}
\newcommand{\C}{\mathcal{C}}
\renewcommand{\P}{\mathcal{P}}

% Schwartz
\renewcommand{\S}{\mathcal{S}} % \S est le signe paragraphe normalement

% corps
\newcommand{\R}{\mathbb{R}}
\newcommand{\Rnn}{\mathbb{R}^{2n}}
\newcommand{\Z}{\mathbb{Z}}
\newcommand{\N}{\mathbb{N}}
\newcommand{\Q}{\mathbb{Q}}
\newcommand{\E}{\mathbb{E}}
\newcommand{\DD}{\mathbb{D}}

% order notations
\DeclareRobustCommand{\O}{%
  \text{\usefont{OMS}{cmsy}{m}{n}O}%
}

% japanese bracket
\newcommand{\japb}[1]{\langle #1 \rangle}

% arrows over partial derivatives
\newcommand{\lp}{\overleftarrow{\partial}}
\newcommand{\rp}{\overrightarrow{\partial}}

% quantization
\newcommand{\h}{\hbar}
\newcommand{\Opht}{\textrm{Op}_{\h}^{t}}
\newcommand{\Op}[2][\hbar]{\textrm{Op}_{#1}^{#2}}

% omega functions
\newcommand{\omegap}[2][\rho_0]{\omega(\partial_{#1},\partial_{#2})}
\newcommand{\omegar}[2][\rho_0]{\omega(#1,#2)}

% space before semicolon
\mathcode`\;="303B

% for \Lightning
\usepackage{marvosym}

% for \warning
% 66 or 49 idk ; depends on the computer for some reason
\newcommand{\warning}{{\fontencoding{U}\fontfamily{futs}\selectfont\char 66\relax}}

% Q(\sqrt(d)) field
\newcommand{\Qsqrt}[1]{\Q\bigl(\mspace{-3mu}\sqrt{#1}\bigr)}

%%%%%%%%% MACROS %%%%%%%%%

% I prefer the slanted \leq
\let\oldleq\leq % save them in case they're every wanted
\let\oldgeq\geq
\renewcommand{\leq}{\leqslant}
\renewcommand{\geq}{\geqslant}

% tel que
\newcommand{\tqs}{\text{ tels que }}
\newcommand{\tq}{\text{ tq. }}
\newcommand{\et}{\text{ et }}
\newcommand{\ou}{\text{ ou }}
\newcommand{\pourtout}{\text{ pour tout }}
\newcommand{\sct}{\text{ sachant }}

% Lois
\newcommand{\Bern}{\text{Bern}}
\newcommand{\Binom}{\text{Binom}}

% ensemble avec bigl et bigr
\newcommand{\bigset}[1]{\bigl\{ #1 \bigr\}}
\newcommand{\Bigset}[1]{\Bigl\{ #1 \Bigr\}}
\newcommand{\bigpar}[1]{\bigl( #1 \bigr)}
\newcommand{\Bigpar}[1]{\Bigl( #1 \Bigr)}

% PLUS INFTY AND MINUS INFTY WITH NO SPACE
\newcommand{\pinfty}{{+}\infty}
\newcommand{\minfty}{{-}\infty}

% vecteur flèche
\renewcommand{\vec}[1]{\overrightarrow{#1}}

% vecteur pmatrix
\newcommand{\pvec}[2]{\begin{pmatrix} #1 \\ #2 \end{pmatrix}}

% vecteur norme
\newcommand{\norm}[1]{\left\Vert #1 \right\Vert}

% point plan
\newcommand{\point}[3]{
	#1\left(#2 ; #3 \right)
}

% \smash avant \underline pour coller la ligne au mot
\let\oldunderline\underline
\renewcommand{\underline}[1]{\oldunderline{\smash{#1}}}

% emph + index
\newcommand{\emphindex}[1]{\emph{#1}\index{#1}}


% fake sections with no title to move around the merged pdf
\newcommand{\fakesection}[1]{%
  \par\refstepcounter{section}% Increase section counter
  \sectionmark{#1}% Add section mark (header)
  \addcontentsline{toc}{section}{\protect\numberline{\thesection}#1}% Add section to ToC
  % Add more content here, if needed.
}

%%%%%%%%%%%%%%%%%%%%%%%%%%%%%%%%%%%%%%%%%%%%%%%%%%%%%%%



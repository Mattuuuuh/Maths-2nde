%% INPUT PREAMBLE.TEX
%% THEN INPUT .ADR
%% THEN THIS

\SetDate[20/02/2026]

\def\arraystretch{1}
\setlength\tabcolsep{10pt}

\begin{document}
\pagestyle{fancy}
\fancyhead[L]{Seconde}
\fancyhead[C]{\textbf{Devoir Maison -- {\seed} -- Vecteurs}}
\fancyhead[R]{} %no date

\fakesection{Devoir \seed}

%%%%%%%%%%%%%%%%%%%
%%%%%%%%%%%%%%%%%%%
%%%%%%%%%%%%%%%%%%%

\exe{,difficulty=1}{
	Considérons le point $A(\Ox ; \Oy)$ et les vecteurs
		\begin{align*}
			u = \pvec{\ux}{\uy}, && v = \pvec{\vx}{\vy}, && \et && w = \projmult\pvec{\projx}{\projy}.
		\end{align*}
	\begin{enumerate}
		\item
		Calculer les coordonnées des points $B = A+u, C=A+v$, et $D = C + w$.
		\item
		Tracer un repère et y placer les points $A, B, C,$ et $D$.
		\item
		Montrer par le calcul que les droites $(AB)$ et $(CD)$ sont perpendiculaires en montrant que
			\begin{enumerate}
				\item
				les droites $(AB)$ et $(AD)$ sont confondues ; et
				\item
				le triangle $ADC$ est rectangle en $D$.
			\end{enumerate}
	\end{enumerate}
}{exe:1}{
	todo
}

%%%%%%%%%%%%%%%%%%%
%%%%%%%%%%%%%%%%%%%
%%%%%%%%%%%%%%%%%%%

\newpage
\fancyhead[C]{\textbf{Solutions -- {\seed} -- Vecteurs}}
\fakesection{Solution \seed}
\shipoutAnswer

\end{document}


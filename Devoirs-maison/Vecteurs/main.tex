%% INPUT PREAMBLE.TEX
%% THEN INPUT .ADR
%% THEN THIS

\SetDate[20/02/2026]

\def\arraystretch{1}
\setlength\tabcolsep{10pt}

\begin{document}
\pagestyle{fancy}
\fancyhead[L]{Seconde}
\fancyhead[C]{\textbf{Devoir Maison -- {\seed} -- Vecteurs}}
\fancyhead[R]{} %no date

\fakesection{Devoir \seed}

%%%%%%%%%%%%%%%%%%%
%%%%%%%%%%%%%%%%%%%
%%%%%%%%%%%%%%%%%%%

\exe{,difficulty=1}{
	Considérons le point $A(\Ox ; \Oy)$ et les vecteurs
		\begin{align*}
			u = \pvec{\ux}{\uy}, && v = \pvec{\vx}{\vy}, && \et && w = \projmult\pvec{\projx}{\projy}.
		\end{align*}
	\begin{enumerate}
		\item
		Calculer les coordonnées des points $B = A+u, C=A+v$, et $D = C + w$.
		\item
		Placer les points $A, B, C,$ et $D$ dans le repère ci-dessous.
		\item
		Montrer par le calcul que les droites $(AB)$ et $(CD)$ sont perpendiculaires en montrant que
			\begin{enumerate}
				\item
				les droites $(AB)$ et $(AD)$ sont confondues ; et
				\item
				le triangle $ADC$ est rectangle en $D$.
			\end{enumerate}
	\end{enumerate}
}{exe:1}{

	\begin{multicols}{2}
	\begin{enumerate}
		\item
		\begin{align*}
			A\left(\xA ; \yA\right) && B\left(\xB ; \yB\right) \\\\ C\left(\xC ; \yC\right) && D\left(\xD ; \yD\right)
		\end{align*}
		\hfill\,
		\item
		\begin{center}
		\begin{tikzpicture}[>=stealth, scale=1]
			\begin{axis}[xmin = \xymin, 
				xmax=\xymax, 
				ymin=\xymin,
				ymax=\xymax, 
				axis x line=middle, 
				axis y line=middle, 
				axis line style=-,
				clip=true, 
				%xtick distance = 1,
				%ytick distance = 1,
				x =\xypt pt,
				y =\xypt pt,
				minor tick num = 1,
				clip=true]
				% ADC triangle
				\addplot[dashed, thick, black] (\xAreal, \yAreal) -- (axis cs:\xCreal, \yCreal) -- (axis cs:\xDreal, \yDreal) -- (axis cs:\xAreal, \yAreal);
				% points
				\addplot[BLUE_E, mark=*, mark size=1] (\xAreal, \yAreal) node[below]{$A$};
				\addplot[RED_E, mark=*, mark size=1] (\xBreal, \yBreal) node[below]{$B$};
				\addplot[GREEN_E, mark=*, mark size=1] (\xCreal, \yCreal) node[below]{$C$};
				\addplot[GOLD_E, mark=*, mark size=1] (\xDreal, \yDreal) node[left]{$D$};
			\end{axis}
		\end{tikzpicture}
		\end{center}
	\end{enumerate}
	\end{multicols}
	\begin{enumerate}
		\item[3.]
		\begin{enumerate}
			\item
			Il s'agit de montrer que $\vec{AB}$ et $\vec{AD}$ sont colinéaires.
			Comme $\vec{AB} = \pvec{\ux}{\uy}$ et $\vec{AD} = \wmult\pvec{\wx}{\wy}$, on justifie soit en disant que $\vec{AD} = \fourier \vec{AB}$, soit en calculant $\det\bigl(\vec{AD}, \vec{AB}\bigr) = 0$.
			\item
			D'après la réciproque du théorème de Pythagore, il s'agit de vérifier que
				\[ AC^2 \stackrel{?}{=} AD^2 + CD^2. \]
			La formule de la longueur $AB^2 = \Vert \vec{AB} \Vert^2$ permet de conclure.
		\end{enumerate}
	\end{enumerate}
}


	\begin{center}
	\begin{tikzpicture}[>=stealth, scale=1]
		\begin{axis}[xmin = \xymin, 
			xmax=\xymax, 
			ymin=\xymin,
			ymax=\xymax, 
			axis x line=middle, 
			axis y line=middle, 
			axis line style=-,
			clip=true, 
			%xtick distance = 1,
			%ytick distance = 1,
			x =\xypt pt,
			y =\xypt pt,
			minor tick num = 1,
			]
		\end{axis}
	\end{tikzpicture}
	\end{center}

%%%%%%%%%%%%%%%%%%%
%%%%%%%%%%%%%%%%%%%
%%%%%%%%%%%%%%%%%%%

\newpage
\fancyhead[C]{\textbf{Solutions -- {\seed} -- Vecteurs}}
\fakesection{Solution \seed}
\shipoutAnswer

\end{document}


%% INPUT PREAMBLE.TEX
%% THEN INPUT .ADR
%% THEN THIS
%% THEN POSTAMBLE


\begin{document}
\pagestyle{fancy}
\fancyhead[L]{Seconde 13}
\fancyhead[C]{\textbf{Devoir Maison -- {\seed} -- Équations diophantiennes}}

\fancyhead[R]{\today}

\fakesection{Devoir \seed}

%%%%%%%%%%%%%%%%%%%
%%%%%%%%%%%%%%%%%%%
%%%%%%%%%%%%%%%%%%%



\exe{, difficulty=2}{
	On considère la série statistique $X$ suivante, dépendante de deux entiers naturels $a, b\in\N$.
	
		\begin{center}
		\begin{tabular}{|c|c|c|c|}\hline
		Valeur   & \A & \B & \C \\ \hline
		Effectif & $a$ & $b$ & \cI \\ \hline
		\end{tabular}
		\end{center}
		
	\begin{enumerate}
		\item
		Monter que, pour le couple $(a;b) = (\sola;\solb)$, la moyenne de la série est $\overline{X} = \Xbar$.
		\item
		Montrer que la condition $\overline{X} = \Xbar$ est équivalente à la relation
			\begin{align}
				- \p a + \q b = 1. \label{eq:1}
			\end{align}
	\end{enumerate}
	Dans les questions suivantes, on utilisera l'équation \eqref{eq:1}, dite \emph{diophantienne}\footnotemark, pour étudier la structure de ses solutions et simplifier les calculs.
	\begin{enumerate}[resume*]
		\item
		Déduire de \eqref{eq:1} que $\p$ divise $\q b - 1$ et trouver le plus petit entier naturel $b_0\in\N$ vérifiant
			\[ \p \quad\text{ divise }\quad \bigl(\q b_0 - 1\bigr). \]
		\item
		Posons 
			\[ a_0 = \dfrac{\q b_0 - 1}\p. \]
		Montrer que pour le couple d'entiers $(a;b) = (a_0 ; b_0)$, la série $X$ est de moyenne $\Xbar$.
		\item
		Montrer que pour tout $n\in\N$, la série $X$ associée au couple
			\[ (a;b) = (a_0 ; b_0) + n \cdot (\q ; \p) \]
		est de moyenne $\Xbar$. 
		\item
		Représenter ces points $(a;b)$ dans un repère pour $n = 0 ; 1 ; 2; 3$. 
		Que dire sur les points ?
	\end{enumerate}
}{exe:1}
{
	\begin{enumerate}
		\item
		On remplace $a$ et $b$ par $\sola$ et $\solb$ respectivement et on trouve bien une moyenne égale à
			\[ \dfrac{\A\cdot\sola \B\cdot\solb+\C\cdot\cI}{\sola+\solb+\cI} = \Xbar. \]
		
		\item 
		On écrit la suite d'équations équivalentes à $\overline{X} = \Xbar$ suivante.
			\begin{align*}
				\dfrac{\A a \B b+\C\cdot\cI}{a + b+\cI} &= \Xbar \\
				\A a \B b+\C\cdot\cI &= \Xbar  (a+b+\cI) \\
				 - \p a + \q b &= 1.
			\end{align*}
		
		\item
		On réarrange les termes pour trouver que 
			\[ \q b - 1 = \p a, \]
		c'est-à-dire que $\q b - 1$ est un multiple de $\p$, car $a$ est un entier naturel.
		
		Comme $b$ est lui-même un entier naturel (c'est un effectif !), on peut essayer des valeurs $b=0, 1, 2, \dots$ jusqu'a ce que $\q b -1$ soit divisible par $\p$.
		La valeur
			\[ b_0 = \bzero \]
		est la première qui fonctionne.
		
		\item
		On a alors
			\[ a_0 = \dfrac{9 \cdot \bzero - 1}{\p} = \azero. \]
		Presque par construction on trouve bien que 
			\[ -\p \cdot \azero  + \q \cdot \bzero = 1, \]
		condition équivalente à ce que $\overline{X} = \Xbar$ d'après la deuxième question (on peut aussi revérifier en remplaçant dans le tableau comme à la question 1).
		
		\item
		Chaque couple de la forme
			\begin{align*}
				(a;b) &= (\azero; \bzero) + n \cdot (\q ; \p) \\
					&= (\azero+ \q n ; \bzero + \p n )
			\end{align*}
		vérifie bien
			\begin{align*}
				 - \p \cdot a +  \q \cdot b &= - \p (\azero + \q n ) + \q (\bzero + \p n ) \\
				 							&= -\p \cdot \azero +  \q \cdot \bzero + n (- \p \cdot \q + \q \cdot \p) \\
				 							&= 1 + n \cdot 0 \\
				 							&= 1.
			\end{align*}
		En fait, le couple $(\q; \p)$ est une solution du problème dit \emph{homogène} car
			\[ -\p \cdot \q + \q \cdot \p = 0, \]
		et la somme d'une solution du problème initial avec une solution du problème homogène donne une autre solution du problème initial.
		
		\item
		On voit les couples comme les coordonnées de points et on place les quatre premiers ainsi.
			\begin{center}
			\begin{tikzpicture}[>=stealth, scale=1]
			\begin{axis}[xmin = -1, xmax=\athree+5, ymin=-1, ymax=\bthree+5, axis x line=middle, axis y line=middle, axis line style=->, grid=both, clip=false]
				\addplot[black, mark=*, mark size = 1, thick] (\azero,\bzero) node[below right] {$(\azero;\bzero)$};
				\addplot[black, mark=*, mark size = 1, thick] (\sola,\solb) node[below right] {$(\sola;\solb)$};
				\addplot[black, mark=*, mark size = 1, thick] (\atwo,\btwo) node[below right] {$(\atwo;\btwo)$};
				\addplot[black, mark=*, mark size = 1, thick] (\athree,\bthree) node[below right] {$(\athree;\bthree)$};
				\addplot[dashed, red, -] expression[domain=-1:\athree+5] {(1+ \p* x)/ \q};
			\end{axis}
			\end{tikzpicture}
			\end{center}
		On remarque que tous les points sont alignés. 
		Les points trouvés sont en fait tous les points à coordonnées entières de la droite $-\p x + \q y = 1$.
		%Les points trouvés sont en fait tous les points à coordonnées entières d'une certaine droite.
		%On étudiera les équations de droites dans un chapitre dédié plus tard. En attendant, on pourra écrire \texttt{$ -\p x + \q y = 1$} dans Geogebra pour voir la droite.
	\end{enumerate}
}

\footnotetext{De Diophante d'Alexandrie (entre 0 et 400, dates exactes inconnues), mathématicien grec.}

%\exe{, difficulty=2}{
%	On considère la série statistique $X$ suivante, dépendante de deux entiers naturels $a, b\in\N$.
%	
%		\begin{center}
%		\begin{tabular}{|c|c|c|c|}\hline
%		Valeur   & \kmpB & \kpqB & \nvalB \\ \hline
%		Effectif & $a$ & $b$ & \nB \\ \hline
%		\end{tabular}
%		\end{center}
%		
%	\begin{enumerate}
%		\item
%		Montrer que la condition $\overline{X} = \kB$ est équivalente à la relation
%			\begin{align}
%				- \pB a + \qB b = \dmnB. \label{eq:2}
%			\end{align}
%		\item
%		Trouver le plus grand diviseur commun à $\pB$ et $\qB$ et montrer qu'il divise nécessairement 
%			\[ - \pB a + \qB b.\]
%		\item
%		Conclure par contradiction qu'il n'existe pas d'entiers naturels $a, b\in\N$ tels que la série $X$ ait une moyenne égale à $\kB$.
%		\item 
%		Trouver un moyenne possible et deux couples $(a,b)$ distincts qui la réalisent.
%		Y a-t-il une infinité de tels couples ?
%	\end{enumerate}
%}{exe:2}
%{
%	\begin{enumerate}
%		\item
%			Ceci se montre de façon équivalente à la première question de l'exercice 1.
%		\item
%			Le plus grand diviseur commun à $\pB$ et $\qB$ est donné par $\divisor$ :
%				\begin{align*}
%					\pB &= \divisor \cdot \psdB, \\
%					\qB &= \divisor \cdot \qsdB.
%				\end{align*}
%			Ainsi, $\divisor$ divise la combinaison entière $ - \pB \cdot a + \qB \cdot b$ pour n'importe quels entiers $a$ et $b$ car on a bien
%				\[ - \pB \cdot a + \qB \cdot b = \divisor \cdot ( - \psdB \cdot a + \qsdB \cdot b ), \]
%			multiple de $\divisor$.
%		\item
%		Supposons qu'il existe un tel couple $(a;b)$ d'entiers naturels tels que la moyenne de la série soit $\kB$.
%		Dans ce cas, la relation 
%			\[ - \pB \cdot a + \qB \cdot b = \dmnB. \]
%		doit être vérifiée.
%		Or on a montré que le membre de gauche est une multiple de $\divisor$.
%		Cependant, comme le membre de droite, $\dmnB$, n'est évidemment pas un multiple de $\divisor$, on arrive à une contradiction.
%		\item 
%		On peut choisir un couple arbitrairement et calculer la moyenne correspondante.
%		Choisir $(a;b) = (0;0)$ par exemple donne une moyenne de $\overline{X} = \nvalB$.
%		
%		Au regard de l'exercice 1, il existe nécessairement une infinité de couples vérifiant $\overline{X} = \nvalB$, car on peut en créer en ajoutant autant de fois qu'on le souhaite la solution $(\qB ; \pB)$ du probème homogène à $(0;0)$.
%	\end{enumerate}
%
%}



%%%%%%%%%%%%%%%%%%%
%%%%%%%%%%%%%%%%%%%
%%%%%%%%%%%%%%%%%%%

\newpage
\fancyhead[C]{\textbf{Solutions -- {\seed} --  Équations diophantiennes}}
\fakesection{Solution \seed}
\shipoutAnswer

\end{document}



%% INPUT PREAMBLE.TEX
%% THEN INPUT .ADR
%% THEN THIS


\begin{document}
\pagestyle{fancy}
\fancyhead[L]{Seconde}
\fancyhead[C]{\textbf{Devoir Maison -- {\seed} -- Projeté orthogonal}}
\fancyhead[R]{\today}

\fakesection{Devoir \seed}

Considérons la fonction $f$ donnée algébriquement pour tout $x\in\R$ par
	\[ f(x) = x + 1, \]
et le point $A(2 ; 1)$.

En admettant que la courbe $\C_f$ est une droite, le but de ce devoir est de trouver algébriquement le projeté orthogonal du point $A$ sur $\C_f$.

\begin{rpl}
	Pour deux proposition $(p)$ et $(q)$, l'expression « $(p)$ si et seulement si $(q)$ » se note $(p) \iff (q)$ et signifique à la fois que $(p) \implies (q)$, et que $(q) \implies (p)$.
\end{rpl}

\exe{}{
	Construire un repère de domaine $[0 ; 4]$, y grapher $\C_f$, et y placer le point $A$.
}{exe:1}{
	todo
}

\exe{}{
	Le point $A$ appartient-il à $\C_f$ ? Justifier par le calcul.
}{exe:2}{
	todo
}

\exe{}{
	Justifier qu'un point $P$ appartient à $\C_f$ si et seulement si $P$ est de la forme $(x ; x+1)$, avec $x\in\R$.
}{exe:3}{
	todo
}

\exe{, difficulty=1}{
	Montrer que la distance au carré entre $A$ et $P$ est donnée par
		\[ AP^2 = 2(x-1)^2 + 2. \]
}{exe:4}{
	todo
}

\exe{}{
	Justifier soigneusement que 
	\begin{itemize}
		\item $AP^2$ est toujours supérieure ou égale à 2 ; et
		\item $AP^2$ vaut 2 si et seulement si $x=1$.
	\end{itemize}
}{exe:5}{
	todo
}

\exe{}{
	Conclure que le projeté orthogonal de $A$ sur $\C_f$ est donné par $B(1;2)$.
	Placer $B$ dans le repère de la question \ref{exe:1}.
}{exe:6}{
	todo
}

\newpage
\fancyhead[C]{\textbf{Solutions -- {\seed}}}
\fakesection{Solution \seed}
\shipoutAnswer

\end{document}

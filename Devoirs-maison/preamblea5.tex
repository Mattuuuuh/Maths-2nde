%!TEX encoding = UTF8

\documentclass[a5paper, 12pt, landscape]{extarticle}
\usepackage[french]{babel}
\usepackage[margin=2cm]{geometry}
  
\usepackage[utf8x]{inputenc}
%fonts
\usepackage{libertinus,libertinust1math}
\usepackage{amsmath,amsthm,amssymb,mathtools}

%virgules
\usepackage{icomma}

% HEADER, ARRAY, ENUM, MULTIOCL
\usepackage{fancyhdr}
\usepackage{array}
\usepackage{multicol, enumitem}

% theorems
\theoremstyle{theorem}
\newtheorem{thm}{Théorème}
\theoremstyle{plain}
\newtheorem*{sol}{Solution}
\theoremstyle{definition}
\newtheorem{ex}{Exercice}
\newtheorem{dfn}{Définition}
\newtheorem*{dfn*}{Définition}
\newtheorem*{rpl}{Rappel}

% exercices
\usepackage[answerdelayed, lastexercise]{exercise}
\usepackage{ifthen}
\renewcommand{\ExerciseHeader}{
	\tikz[baseline=(R.base)]\node[draw,rectangle, thick, inner sep=2pt](R) {\textbf{\theExercise.}};\!
	\ifnum\ExerciseDifficulty=0
	\else
		(\theExerciseDifficulty)
	\fi
}
\renewcommand{\DifficultyMarker}{$\star$}
\renewcommand{\AnswerHeader}{
	\tikz[baseline=(R.base)]\node[draw,rectangle, thick, inner sep=2pt](R) {\textbf{\theExercise.}};\!
}
\newcommand{\exe}[4]{
	\begin{Exercise}[title=#1, label=#3]
		#2
	\end{Exercise}
	\begin{Answer}[ref=#3]
		#4
	\end{Answer}
}
\newcommand{\exemulticols}[5]{
	\begin{multicols}{2}
	\begin{Exercise}[title=#1, label=#4]
		#2
	\end{Exercise}
	\columnbreak
		#3
	\end{multicols}
	\begin{Answer}[ref=#4]
		#5
	\end{Answer}
}

% date
\usepackage{advdate}

% vabs
\newcommand{\vabs}[1]{
	\left| #1 \right|
}
% plots
\usepackage{pgfplots}
\tikzset{
	every axis/.style = {clip=false, axis lines=center, axis line style=<->, xlabel={}, ylabel={}, grid=both, grid style = {opacity=.5}, domain=-2:2}
}

%subfigures
\usepackage{subcaption}

%hyperlink footnote
\usepackage{hyperref}

% tableaux var, signe
\usepackage{tkz-tab}

%wider tabulars
\def\arraystretch{2}
\setlength\tabcolsep{15pt}
\usepackage{makecell} %pour \thead dans tabular ex3 (aligner verticalement le coeff de proportionnalité)

% package systeme pour les systèmes d'équations bien alignés
\usepackage{systeme}
\sysalign{r,r}
\syseqspace{3pt}
\syssignspace{3pt}

\newcommand{\sys}[2]{\systeme{#1{,}, #2.}}

% for striked out implies sign (\centernot\implies)
\usepackage{centernot}

%%%%%%%%%%%%%%%%%%%%%%%%%%%%%%
% SELF MADE COLORS
%%%%%%%%%%%%%%%%%%%%%%%%%%%%%%

%%%%%%%%%%%%%%%%%%%%%%%%%%%%%%
% SELF MADE COLORS
%%%%%%%%%%%%%%%%%%%%%%%%%%%%%%


\definecolor{myg}{RGB}{56, 140, 70}
\definecolor{myb}{RGB}{45, 111, 177}
\definecolor{myr}{RGB}{199, 68, 64}
\definecolor{mytheorembg}{HTML}{F2F2F9}
\definecolor{mytheoremfr}{HTML}{00007B}
\definecolor{mylenmabg}{HTML}{FFFAF8}
\definecolor{mylenmafr}{HTML}{983b0f}
\definecolor{mypropbg}{HTML}{f2fbfc}
\definecolor{mypropfr}{HTML}{191971}
\definecolor{myexamplebg}{HTML}{F2FBF8}
\definecolor{myexamplefr}{HTML}{88D6D1}
\definecolor{myexampleti}{HTML}{2A7F7F}
\definecolor{mydefinitbg}{HTML}{E5E5FF}
\definecolor{mydefinitfr}{HTML}{3F3FA3}
\definecolor{notesgreen}{RGB}{0,162,0}
\definecolor{myp}{RGB}{197, 92, 212}
\definecolor{mygr}{HTML}{2C3338}
\definecolor{myred}{RGB}{127,0,0}
\definecolor{myyellow}{RGB}{169,121,69}
\definecolor{myexercisebg}{HTML}{F2FBF8}
\definecolor{myexercisefg}{HTML}{88D6D1}
\definecolor{doc}{RGB}{0,60,110}

% manim colors because they're beautiful
% https://docs.manim.community/en/stable/reference/manim.utils.color.manim_colors.html

\definecolor{BLACK}{HTML}{000000}\definecolor{BLUE}{HTML}{58C4DD}\definecolor{BLUE_A}{HTML}{C7E9F1}\definecolor{BLUE_B}{HTML}{9CDCEB}\definecolor{BLUE_C}{HTML}{58C4DD}\definecolor{BLUE_D}{HTML}{29ABCA}\definecolor{BLUE_E}{HTML}{236B8E}\definecolor{DARKER_GRAY}{HTML}{222222}\definecolor{DARKER_GREY}{HTML}{222222}\definecolor{DARK_BLUE}{HTML}{236B8E}\definecolor{DARK_BROWN}{HTML}{8B4513}\definecolor{DARK_GRAY}{HTML}{444444}\definecolor{DARK_GREY}{HTML}{444444}\definecolor{GOLD}{HTML}{F0AC5F}\definecolor{GOLD_A}{HTML}{F7C797}\definecolor{GOLD_B}{HTML}{F9B775}\definecolor{GOLD_C}{HTML}{F0AC5F}\definecolor{GOLD_D}{HTML}{E1A158}\definecolor{GOLD_E}{HTML}{C78D46}\definecolor{GRAY}{HTML}{888888}\definecolor{GRAY_A}{HTML}{DDDDDD}\definecolor{GRAY_B}{HTML}{BBBBBB}\definecolor{GRAY_BROWN}{HTML}{736357}\definecolor{GRAY_C}{HTML}{888888}\definecolor{GRAY_D}{HTML}{444444}\definecolor{GRAY_E}{HTML}{222222}\definecolor{GREEN}{HTML}{83C167}\definecolor{GREEN_A}{HTML}{C9E2AE}\definecolor{GREEN_B}{HTML}{A6CF8C}\definecolor{GREEN_C}{HTML}{83C167}\definecolor{GREEN_D}{HTML}{77B05D}\definecolor{GREEN_E}{HTML}{699C52}\definecolor{GREY}{HTML}{888888}\definecolor{GREY_A}{HTML}{DDDDDD}\definecolor{GREY_B}{HTML}{BBBBBB}\definecolor{GREY_BROWN}{HTML}{736357}\definecolor{GREY_C}{HTML}{888888}\definecolor{GREY_D}{HTML}{444444}\definecolor{GREY_E}{HTML}{222222}\definecolor{LIGHTER_GRAY}{HTML}{DDDDDD}\definecolor{LIGHTER_GREY}{HTML}{DDDDDD}\definecolor{LIGHT_BROWN}{HTML}{CD853F}\definecolor{LIGHT_GRAY}{HTML}{BBBBBB}\definecolor{LIGHT_GREY}{HTML}{BBBBBB}\definecolor{LIGHT_PINK}{HTML}{DC75CD}\definecolor{LOGO_BLACK}{HTML}{343434}\definecolor{LOGO_BLUE}{HTML}{525893}\definecolor{LOGO_GREEN}{HTML}{87C2A5}\definecolor{LOGO_RED}{HTML}{E07A5F}\definecolor{LOGO_WHITE}{HTML}{ECE7E2}\definecolor{MAROON}{HTML}{C55F73}\definecolor{MAROON_A}{HTML}{ECABC1}\definecolor{MAROON_B}{HTML}{EC92AB}\definecolor{MAROON_C}{HTML}{C55F73}\definecolor{MAROON_D}{HTML}{A24D61}\definecolor{MAROON_E}{HTML}{94424F}\definecolor{ORANGE}{HTML}{FF862F}\definecolor{PINK}{HTML}{D147BD}\definecolor{PURE_BLUE}{HTML}{0000FF}\definecolor{PURE_GREEN}{HTML}{00FF00}\definecolor{PURE_RED}{HTML}{FF0000}\definecolor{PURPLE}{HTML}{9A72AC}\definecolor{PURPLE_A}{HTML}{CAA3E8}\definecolor{PURPLE_B}{HTML}{B189C6}\definecolor{PURPLE_C}{HTML}{9A72AC}\definecolor{PURPLE_D}{HTML}{715582}\definecolor{PURPLE_E}{HTML}{644172}\definecolor{RED}{HTML}{FC6255}\definecolor{RED_A}{HTML}{F7A1A3}\definecolor{RED_B}{HTML}{FF8080}\definecolor{RED_C}{HTML}{FC6255}\definecolor{RED_D}{HTML}{E65A4C}\definecolor{RED_E}{HTML}{CF5044}\definecolor{TEAL}{HTML}{5CD0B3}\definecolor{TEAL_A}{HTML}{ACEAD7}\definecolor{TEAL_B}{HTML}{76DDC0}\definecolor{TEAL_C}{HTML}{5CD0B3}\definecolor{TEAL_D}{HTML}{55C1A7}\definecolor{TEAL_E}{HTML}{49A88F}\definecolor{WHITE}{HTML}{FFFFFF}\definecolor{YELLOW}{HTML}{FFFF00}\definecolor{YELLOW_A}{HTML}{FFF1B6}\definecolor{YELLOW_B}{HTML}{FFEA94}\definecolor{YELLOW_C}{HTML}{FFFF00}\definecolor{YELLOW_D}{HTML}{F4D345}\definecolor{YELLOW_E}{HTML}{E8C11C}

%%%%%%%%%%%%%%%%%%%%%%%%%%%%
% LETTERFONTS
%%%%%%%%%%%%%%%%%%%%%%%%%%%%


% Schwartz
\renewcommand{\S}{\mathcal{S}} % \S est le signe paragraphe normalement

% corps
\newcommand{\C}{\mathcal{C}}
\newcommand{\R}{\mathbb{R}}
\newcommand{\Rnn}{\mathbb{R}^{2n}}
\newcommand{\Z}{\mathbb{Z}}
\newcommand{\N}{\mathbb{N}}
\newcommand{\Q}{\mathbb{Q}}

% domain
\newcommand{\D}{\mathcal{D}}

% order notations
\renewcommand{\O}{\mathcal{O}}

% japanese bracket
\newcommand{\japb}[1]{\langle #1 \rangle}

% arrows over partial derivatives
\newcommand{\lp}{\overleftarrow{\partial}}
\newcommand{\rp}{\overrightarrow{\partial}}

% quantization
\newcommand{\h}{\hbar}
\newcommand{\Opht}{\textrm{Op}_{\h}^{t}}
\newcommand{\Op}[2][\hbar]{\textrm{Op}_{#1}^{#2}}

% omega functions
\newcommand{\omegap}[2][\rho_0]{\omega(\partial_{#1},\partial_{#2})}
\newcommand{\omegar}[2][\rho_0]{\omega(#1,#2)}

%%%%%%%%%%%%%%%%%%%%%%%%%%%%
% MACROS
%%%%%%%%%%%%%%%%%%%%%%%%%%%%

%!TEX encoding = UTF8
%!TEX root = 0-notes.tex

%%%%%%%%%%%%%%%%%%%%%%%%%%%%%%
% SELF MADE COMMANDS
%%%%%%%%%%%%%%%%%%%%%%%%%%%%%%


%%
% tcolor environments VS clean environments
%%

\ifclean

\newcommand{\thm}[3]{\begin{theorem}[#1]\label{#3}#2\end{theorem}}
\newcommand{\cor}[3]{\begin{corollaire}[#1]\label{#3}#2\end{corollaire}}
\newcommand{\lem}[3]{\begin{lemme}[#1]\label{#3}#2\end{lemme}}
\newcommand{\mprop}[3]{\begin{proposition}[#1]\label{#3}#2\end{proposition}}
\newcommand{\ex}[3]{\begin{exemple}[#1]\label{#3}#2\end{exemple}}
%\newcommand{\exe}[3]{\begin{exercice}[#1]\label{#3}#2\end{exercice}}
\newcommand{\dfn}[3]{\begin{definition}[#1]\label{#3}#2\end{definition}}
\newcommand{\qs}[2]{\begin{question}[#1]#2\end{question}}
\newcommand{\pf}[2]{\begin{preuve}[#1]#2\end{preuve}}
\newcommand{\nt}[1]{\begin{remarque}#1\end{remarque}}
\newcommand{\str}[1]{\begin{strategie}#1\end{strategie}}
\newcommand{\mth}[1]{\begin{methode}#1\end{methode}}
\newcommand{\ax}[3]{\begin{axiome}[#1]\label{#3}#2\end{axiome}}

\newcommand{\exe}[4]{
	\begin{Exercise}[title=#1, label=#3]
		\marginpar{\mbox{\scriptsize(solution p.\pageref{\ExerciseLabel-Answer})}}
		#2
	\end{Exercise}
	\begin{Answer}[ref=#3]
		#4
	\end{Answer}
}

\else

\newcommand{\thm}[3]{\begin{Theorem}[label=#3]{#1}{}#2\end{Theorem}}
\newcommand{\cor}[3]{\begin{Corollary}[label=#3]{#1}{}#2\end{Corollary}}
\newcommand{\lem}[3]{\begin{Lemma}[label=#3]{#1}{}#2\end{Lemma}}
\newcommand{\mprop}[3]{\begin{Prop}[label=#3]{#1}{}#2\end{Prop}}
\newcommand{\ex}[3]{\begin{Example}[label=#3]{#1}{}#2\end{Example}}
%\newcommand{\exe}[3]{\begin{Exe}[label=#3]{#1}{}#2\end{Exe}}
\newcommand{\dfn}[3]{\begin{Definition}[colbacktitle=red!75!black, label=#3]{#1}{}#2\end{Definition}}
\newcommand{\qs}[2]{\begin{MyQuestion}{#1}{}#2\end{MyQuestion}}
\newcommand{\pf}[2]{\begin{myproof}[#1]#2\end{myproof}}
\newcommand{\nt}[1]{\begin{Note}#1\end{Note}}
\newcommand{\str}[1]{\begin{Strategy}#1\end{Strategy}}
\newcommand{\mth}[1]{\begin{Methode}#1\end{Methode}}
\newcommand{\axiome}[3]{\begin{Axiome}[label=#3]{#1}{}#2\end{Axiome}}

\newcommand{\exe}[4]{
	\begin{Exe}[label=#3]{}{}#2\end{Exe}
	\begin{Answer}[ref=#3]
		#4
	\end{Answer}
}

\fi

\newcommand{\notations}[1]{\begin{notation}#1 \end{notation}}
\newcommand{\nomen}[1]{\begin{nomenclature}#1 \end{nomenclature}}

%%

\newcommand*\circled[1]{\tikz[baseline=(char.base)]{
		\node[shape=circle,draw,inner sep=1pt] (char) {#1};}}
\newcommand\getcurrentref[1]{%
	\ifnumequal{\value{#1}}{0}
	{??}
	{\the\value{#1}}%
}
\newcommand{\getCurrentSectionNumber}{\getcurrentref{section}}
\newenvironment{myproof}[1][\proofname]{%
	\proof[\bfseries #1: ]%
}{\endproof}

\newcommand{\mclm}[2]{\begin{myclaim}[#1]#2\end{myclaim}}
\newenvironment{myclaim}[1][\claimname]{\proof[\bfseries #1: ]}{}

\newcounter{mylabelcounter}

\makeatletter
\newcommand{\setword}[2]{%
	\phantomsection
	#1\def\@currentlabel{\unexpanded{#1}}\label{#2}%
}
\makeatother


\tikzset{
	symbol/.style={
			draw=none,
			every to/.append style={
					edge node={node [sloped, allow upside down, auto=false]{$#1$}}}
		}
}


% deliminators
\DeclarePairedDelimiter{\abs}{\lvert}{\rvert}
%\DeclarePairedDelimiter{\norm}{\lVert}{\rVert}

\DeclarePairedDelimiter{\ceil}{\lceil}{\rceil}
\DeclarePairedDelimiter{\floor}{\lfloor}{\rfloor}
\DeclarePairedDelimiter{\round}{\lfloor}{\rceil}

\newsavebox\diffdbox
\newcommand{\slantedromand}{{\mathpalette\makesl{d}}}
\newcommand{\makesl}[2]{%
\begingroup
\sbox{\diffdbox}{$\mathsurround=0pt#1\mathrm{#2}$}%
\pdfsave
\pdfsetmatrix{1 0 0.2 1}%
\rlap{\usebox{\diffdbox}}%
\pdfrestore
\hskip\wd\diffdbox
\endgroup
}
\newcommand{\dd}[1][]{\ensuremath{\mathop{}\!\ifstrempty{#1}{%
\slantedromand\@ifnextchar^{\hspace{0.2ex}}{\hspace{0.1ex}}}%
{\slantedromand\hspace{0.2ex}^{#1}}}}
\ProvideDocumentCommand\dv{o m g}{%
  \ensuremath{%
    \IfValueTF{#3}{%
      \IfNoValueTF{#1}{%
        \frac{\dd #2}{\dd #3}%
      }{%
        \frac{\dd^{#1} #2}{\dd #3^{#1}}%
      }%
    }{%
      \IfNoValueTF{#1}{%
        \frac{\dd}{\dd #2}%
      }{%
        \frac{\dd^{#1}}{\dd #2^{#1}}%
      }%
    }%
  }%
}
\providecommand*{\pdv}[3][]{\frac{\partial^{#1}#2}{\partial#3^{#1}}}
%  - others
\DeclareMathOperator{\Lap}{\mathcal{L}}
\DeclareMathOperator{\Var}{Var} % variance
\DeclareMathOperator{\Cov}{Cov} % covariance

% Since the amsthm package isn't loaded

% I prefer the slanted \leq
\let\oldleq\leq % save them in case they're every wanted
\let\oldgeq\geq
\renewcommand{\leq}{\leqslant}
\renewcommand{\geq}{\geqslant}

% tel que
\newcommand{\tqs}{\text{ tels que }}
\newcommand{\tq}{\text{ tq. }}
\newcommand{\et}{\text{ et }}
\newcommand{\ou}{\text{ ou }}
\newcommand{\pourtout}{\text{ pour tout }}

% Lois
\newcommand{\Bern}{\text{Bern}}
\newcommand{\Binom}{\text{Binom}}

% ensemble avec bigl et bigr
\newcommand{\bigset}[1]{\bigl\{ #1 \bigr\}}
\newcommand{\Bigset}[1]{\Bigl\{ #1 \Bigr\}}
\newcommand{\bigpar}[1]{\bigl( #1 \bigr)}
\newcommand{\Bigpar}[1]{\Bigl( #1 \Bigr)}

% PLUS INFTY AND MINUS INFTY WITH NO SPACE
\newcommand{\pinfty}{{+}\infty}
\newcommand{\minfty}{{-}\infty}

% vecteur flèche
\renewcommand{\vec}[1]{\overrightarrow{#1}}

% vecteur pmatrix
\newcommand{\pvec}[2]{\begin{pmatrix} #1 \\ #2 \end{pmatrix}}

% vecteur norme
\newcommand{\norm}[1]{\left\Vert #1 \right\Vert}

% point plan
\newcommand{\point}[3]{
	#1\left(#2 ; #3 \right)
}

% \smash avant \underline pour coller la ligne au mot
\let\oldunderline\underline
\renewcommand{\underline}[1]{\oldunderline{\smash{#1}}}

% emph + index
\newcommand{\emphindex}[1]{\emph{#1}\index{#1}}


% fake sections with no title to move around the merged pdf
\newcommand{\fakesection}[1]{%
  \par\refstepcounter{section}% Increase section counter
  \sectionmark{#1}% Add section mark (header)
  \addcontentsline{toc}{section}{\protect\numberline{\thesection}#1}% Add section to ToC
  % Add more content here, if needed.
}

%%%%%%%%%%%%%%%%%%%%%%%%%%%%%%%%%%%%%%%%%%%%%%%%%%%%%%%



%% INPUT PREAMBLE.TEX
%% THEN INPUT .ADR
%% THEN THIS

\SetDate[14/01/2026]

\begin{document}
\pagestyle{fancy}
\fancyhead[L]{Seconde}
\fancyhead[C]{\textbf{Devoir Maison -- {\seed} -- Fonction carré}}
\fancyhead[R]{\today}

\fakesection{Devoir \seed}

%%%%%%%%%%%%%%%%%%%
%%%%%%%%%%%%%%%%%%%
%%%%%%%%%%%%%%%%%%%

\exe{, difficulty=1}{
	Considérons la fonction $f$ définie algébriquement sur $\R$ par
		\[ f(x) = \fa x^2  \fb x \fc. \]
	\begin{enumerate}
		\item
		Tracer la courbe représentative de $f$ dans le repère ci-dessous.
		\item 
		Montrer que $f(x) = \hbeta \hA (\ha x \hb)^2$ pour tout $x\in\R$.
		\item
		Donner l'extremum de $f$ en précisant
			\begin{enumerate}
				\item s'il s'agit d'un minimum ou d'un maximum ; et
				\item l'antécédent $x^\star$ qui réalise cet extremum.
			\end{enumerate}
	\end{enumerate}
	
	% repère général avec transparent du coup?

}{exe:1}{
	2. En développant grâce aux identités remarquables.
	
	\begin{enumerate}[start=3]
		\item
		On part du carré $E^2$ toujours positif avec égalité $E^2 = 0$ quand l'expression $E$ est nulle.
			\begin{align*}
				(\ha x \hb)^2 \geq 0 && \text{avec égalité quand } \ha x \hb = 0
			\end{align*}
		En résolvant $\ha x \hb =0 \iff x= \xstarfrac$, on obtient l'argument de l'extremum.

		Puis, en multipliant par $\hA$ en faisant attention au signe et en ajoutant $\hbeta$, on trouve
		\ifnum\ishApositive=1
			\begin{align*}
				\hA(\ha x \hb)^2 &\geq 0 && \text{ parce que } \hA \geq 0 \\
				\hbeta \hA(\hA x \hb)^2 &\geq \hbeta && \\
				f(x) &\geq \hbeta && \\
			\end{align*}
		Ainsi le minimum de $f(x)$ est $\hbeta$, atteint en $x^\star = \xstarfrac$.
		\else
			\begin{align*}
				\hA(\ha x \hb)^2 &\leq 0 && \text{ parce que } \hA \leq 0 \\
				\hbeta \hA(\hA x \hb)^2 &\leq \hbeta && \\
				f(x) &\leq \hbeta && \\
			\end{align*}
		Ainsi le maximum de $f(x)$ est $\hbeta$, atteint en $x^\star = \xstarfrac$.
		\fi
	\end{enumerate}
	\begin{center}
	\begin{tikzpicture}[>=stealth, scale=.6]
		\begin{axis}[xmin = \xmin, xmax=\xmax, ymin=\ymin, ymax=\ymax, axis x line=middle, axis y line=middle, axis line style=-, clip=false, xtick distance = 1, x=100pt,]
			\addplot[myb, thick, domain =\xmin:\xmax, samples=50] {\faR*x^2+\fbR*x+\fcR}  node[pos=.5, right=15pt] {$\mathcal{C}_\mathcal{f}$};
			
			\addplot[PURPLE, very thick, mark=|, mark size = 1] (\xstar,0) node[above] {$\color{PURPLE} x^\star$};
			\addplot[RED_E, very thick, mark=-, mark size = 1] (0,\hbeta) node[below] {$\color{RED_E} f(x^\star)$};
			\draw[PURPLE, dashed, very thick] (axis cs:\xstar, 0) -- (axis cs:\xstar, \hbeta);
			\draw[RED_E, dashed, very thick] (axis cs:0, \hbeta) -- (axis cs:\xstar, \hbeta);
		\end{axis}
	\end{tikzpicture}
	\end{center}
}

	\begin{center}
	\begin{tikzpicture}[>=stealth, scale=1]
		\begin{axis}[xmin = \xmin, xmax=\xmax, ymin=\ymin, ymax=\ymax, axis x line=middle, axis y line=middle, axis line style=-, clip=true, xtick distance = 1, x=100pt,]
				\addplot[transparent, thick, domain =\xmin:\xmax, samples=50] {\faR*x^2+\fbR*x+\fcR}  node[pos=.5, right=15pt] {$\mathcal{C}_f$};
		\end{axis}
	\end{tikzpicture}
	\end{center}


%%%%%%%%%%%%%%%%%%%
%%%%%%%%%%%%%%%%%%%
%%%%%%%%%%%%%%%%%%%

\newpage
\fancyhead[C]{\textbf{Solutions -- {\seed} -- Fonction carré}}
\fakesection{Solution \seed}
\shipoutAnswer

\end{document}


%% INPUT PREAMBLE.TEX
%% THEN INPUT .ADR
%% THEN THIS

\SetDate[05/11/2025]

\begin{document}
\pagestyle{fancy}
\fancyhead[L]{Seconde}
\fancyhead[C]{\textbf{Devoir Maison -- {\seed} -- Interpolation de Lagrange}}
\fancyhead[R]{\today}

\fakesection{Devoir \seed}

%%%%%%%%%%%%%%%%%%%
%%%%%%%%%%%%%%%%%%%
%%%%%%%%%%%%%%%%%%%

\exe{}{
	On définit algébriquement les fonctions $a, b,$ et $c$ sur tout $\R$ de la façon suivante.
		\begin{align*}
		a(x) = \qI(x-\xII)(x-\xIII),
		&&
		b(x) =  \qII(x-\xI)(x-\xIII),
		&&
		c(x) = \qIII(x-\xI)(x-\xII).
		\end{align*}
	\underline{Sans développer les expressions}, remplir les images de $a, b, c$ dans le tableau figure \ref{fig:1}.
	Justifier par le calcul.
}{exe:1}{
	\begin{center}
	\def\arraystretch{1.5}
	\setlength\tabcolsep{40pt}
	\begin{tabular}{|c|c|c|c|}\hline
		$x$ & 1 & 2 & 3 \\ \hline
		$a(x)$ & 1 & 0 & 0 \\ \hline
		$b(x)$ & 0 & 1 & 0 \\ \hline
		$c(x)$ & 0 & 0 & 1 \\ \hline
	\end{tabular}
	\end{center}
}

\exe{}{
	En gardant les notations de la question \ref{exe:1}, posons désormais la fonction sur $\R$
		\[ f(x) = \fI a(x) + \fII b(x) + \fIII c(x). \]
	\underline{Sans développer l'expression}, remplir les images de $f$ dans le tableau figure \ref{fig:1}.
	Justifier par le calcul.
}{exe:2}{
	\begin{center}
	\def\arraystretch{1.5}
	\setlength\tabcolsep{40pt}
	\begin{tabular}{|c|c|c|c|}\hline
		$x$ & \xI & \xII & \xIII \\ \hline
		$f(x)$ & \fI & \fII & \fIII \\ \hline
	\end{tabular}
	\end{center}
}


\newcommand{\sxIxII}{\number\numexpr \xI+\xII \relax}
\newcommand{\sxIxIII}{\number\numexpr \xI+\xIII \relax}
\newcommand{\sxIIxIII}{\number\numexpr \xII+\xIII \relax}
\newcommand{\pxIxII}{\number\numexpr \xI * \xII \relax}
\newcommand{\pxIxIII}{\number\numexpr \xI * \xIII \relax}
\newcommand{\pxIIxIII}{\number\numexpr \xII * \xIII \relax}
\newcommand{\pxIxIIot}{\number\numexpr \xI * \xII/2 \relax}
\newcommand{\pxIxIIIot}{\number\numexpr \xI * \xIII/2 \relax}
\newcommand{\pxIIxIIIot}{\number\numexpr \xII * \xIII/2 \relax}

\exe{, difficulty=1}{
	Développer et réduire l'expression de $f$ de la question \ref{exe:2} en substituant les expressions de la question \ref{exe:1} pour trouver l'expression algébrique de $f$. 
	
	Vérifier que l'expression obtenue est bien cohérente en comparant avec les images de $f$ du tableau figure \ref{fig:1}.
	Justifier par le calcul.
	%que $f(x) = x \bI$.
}{exe:3}{
	On développe $a, b, c$ séparément puis on combine les résultats pour trouver la forme développée réduite de $f$.
	\[ a(x) = \dfrac12(x-\xII)(x-\xIII) = \dfrac12 (x^2 - \xI x - \xII x + \pxIIxIII) = \dfrac12 x^2 - \dfrac{\sxIIxIII}2 x +  \pxIIxIIIot. \]
	\[ b(x) = -(x-\xI)(x-\xIII) = -(x^2 -\xIII x - \xI x + \pxIxIII) = -x^2 + \sxIxIII x - \pxIxIII. \]
	\[ c(x) = \dfrac12(x-\xI)(x-\xII) = \dfrac12 (x^2 - \xII x - \xI x + \pxIxII) = \dfrac12x^2 - \dfrac{\sxIxII}2 x + \pxIxIIot. \]
	
	D'où, 
	\begin{align*}
		f(x) &= \fI a(x) + \fII b(x) + \fIII c(x), \\
			&= \fI\left( \dfrac12 x^2 - \dfrac{\sxIIxIII}2 x +  \pxIIxIIIot \right) + \fII \left(-x^2 + \sxIxIII x - \pxIxIII \right) + \fIII \left(\dfrac12x^2 - \dfrac{\sxIxII}2 x + \pxIxIIot\right), \\
			&= 0x^2 + 1x \bI, \\
			&= x \bI.
	\end{align*}
	
	Finalement, on vérifie que l'expression de $f$ est cohérentes avec les images du tableau en calculant $f(\xI) = \xI \bI = \fI ; f(\xII)  = \fII ; f(\xIII) = \fIII$.
}

\begin{figure}[h!]
	\begin{center}
	\def\arraystretch{2}
	\setlength\tabcolsep{40pt}
	\begin{tabular}{|c|c|c|c|}\hline
		$x$ & \xI & \xII & \xIII \\ \hline
		$a(x)$ & && \\ \hline
		$b(x)$ & && \\ \hline
		$c(x)$ & && \\ \hline
		$f(x)$ & && \\ \hline
	\end{tabular}
	\end{center}
	\caption{Tableau de valeurs de $a, b, c,$ et $f$.}
	\label{fig:1}
\end{figure}

\subsection*{Questions d'approfondissement (non notées)}

\newcommand{\fIV}{\number\numexpr \fIII+1 \relax}

\exe{, difficulty=1}{
	Donner une fonction $g$ vérifiant $g(\xI) = \fI$, $g(\xII) =\fII,$ et $g(\xIII) = \fIV$.
	Développer et réduire l'expression de $g$.
}{exe:4}{
	Il s'agit ici de développer l'expression $f(x) = \fI a(x) +\fII b(x) + \fIV c(x)$, exercice laissé au lecteur ! (\emph{cf}. question \ref{exe:3})
}


\exe{, difficulty=1}{
	Existe-t-il une fonction $h$ vérifiant $h(\xI) = y_1$, $h(\xII) =y_2,$ et $h(\xIII) = y_3$ pour n'importe quels $y_1, y_2, y_3$ ?
}{exe:5}{
	$f(x) = y_1 a(x) + y_2 b(x) + y_3 c(x)$ fait l'affaire, mais il y en a d'autres...
}

\exe{, difficulty=2}{
	Démontrer qu'une infinité de fonctions différentes vérifient le tableau de valeurs de $f$, figure \ref{fig:1}.
	
	Pour justifier que deux fonctions $F$ et $G$ sont différentes sur $\R$, il suffit de trouver un nombre où elles diffèrent, c'est-à-dire un $x\in\R$ où $F(x) \neq G(x)$.
}{exe:6}{
	Considérons toutes les fonctions de la forme
		\[ f_c(x) = f(x) + c (x-\xI)(x-\xII)(x-\xIII), \]
	où $c\in\R$ est un nombre au choix.
	
	Ainsi, $f_1 (x) =  f(x) + (x-\xI)(x-\xII)(x-\xIII), f_0(x) = f(x), f_{-1}(x) =  f(x) - (x-\xI)(x-\xII)(x-\xIII)$, etc...
	Le paramètre $c$ n'est bien sûr pas restreint aux entiers relatifs.
	
	$f_c$ admet les mêmes valeurs que $f$ en $x=\xI ; \xII ; \xIII$ car la partie ajoutée s'annule en chacuns de ces antécédents, par construction.
	
	De plus, pour $c \neq \tilde{c}$, deux valeurs différentes, les fonctions engendrées sont différentes : $f_c \neq f_{\tilde{c}}$.
	Pour montrer cela, il suffit de comparer deux images lorsque la partie ajoutée ne s'annule pas.
	Par exemple, en $x=-1$, on a
		\begin{align*}
			f_c(-1) = f(-1) + c(-1-\xI)(-1-\xII)(-1-\xIII) && \et && f_{\tilde{c}}(-1) = f(-1) + \tilde{c}(-1-\xI)(-1-\xII)(-1-\xIII).
		\end{align*}
	Inutile de calculer les valeurs ici : seul le fait que le produit soit non nul importe.
	En effet, si on avait $f_c(-1) = f_{\tilde{c}}(-1)$, alors, en simplifiant l'équation, on aurait $c = \tilde{c}$, ce qui est faux.
	(et vice versa, si $c = \tilde{c}$, alors $f_c = f_{\tilde{c}}$ ; c'est donc une équivalence !)
	
	Il suit donc que les fonctions sont distinctes et en nombre infini : chaque $c\in\R$ donne naissance à une nouvelle fonction $f_c$ qui admet le même tableau de valeurs que $f$ en $x=\xI ; \xII ; \xIII$.
}

\exe{, difficulty=1}{
	Existe-t-il une fonction $H$ vérifiant $H(x_1) = y_1$, $H(x_2) =y_2,$ et $H(x_3) = y_3$ pour n'importe quels $x_1, x_2, x_3, y_1, y_2, y_3$ ?
}{exe:7}{
	Non, il suffit de prendre deux $x$ identiques aux images distinctes.
	Par exemple, $x_1=x_2 = 0$, et $y_1 = 0 \neq 1 = y_2$.
	
	Aucun fonction $H$ ne vérifie $H(0) = 0$, et $H(0) =1$, et ce par définition d'une fonction (unicité de l'image).
}







%%%%%%%%%%%%%%%%%%%
%%%%%%%%%%%%%%%%%%%
%%%%%%%%%%%%%%%%%%%

\newpage
\fancyhead[C]{\textbf{Solutions -- {\seed} -- Interpolation de Lagrange}}
\fakesection{Solution \seed}
\shipoutAnswer

\end{document}


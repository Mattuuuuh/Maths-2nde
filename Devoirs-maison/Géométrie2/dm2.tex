%% INPUT PREAMBLE.TEX
%% THEN INPUT VARS_{i}.ADR
%% THEN RUN THIS
%% DYSLEXIA SWITCH
\newif\ifdys
		
				% ENABLE or DISABLE font change
				% use XeLaTeX if true
				\dystrue
				\dysfalse


\ifdys

\documentclass[a4paper, 14pt]{extarticle}
\usepackage{amsmath,amsfonts,amsthm,amssymb,mathtools}

\tracinglostchars=3 % Report an error if a font does not have a symbol.
\usepackage{fontspec}
\usepackage{unicode-math}
\defaultfontfeatures{ Ligatures=TeX,
                      Scale=MatchUppercase }

\setmainfont{OpenDyslexic}[Scale=1.0]
\setmathfont{Fira Math} % Or maybe try KPMath-Sans?
\setmathfont{OpenDyslexic Italic}[range=it/{Latin,latin}]
\setmathfont{OpenDyslexic}[range=up/{Latin,latin,num}]

\else

\documentclass[a4paper, 12pt]{extarticle}

\usepackage[utf8x]{inputenc}
%fonts
\usepackage{amsmath,amsfonts,amsthm,amssymb,mathtools}
% comment below to default to computer modern
\usepackage{libertinus,libertinust1math}

\fi


\usepackage[french]{babel}
\usepackage[
a4paper,
margin=2cm,
nomarginpar,% We don't want any margin paragraphs
]{geometry}
\usepackage{icomma}

\usepackage{fancyhdr}
\usepackage{array}
\usepackage{hyperref}

\usepackage{multicol, enumerate}
\newcolumntype{P}[1]{>{\centering\arraybackslash}p{#1}}


\usepackage{stackengine}
\newcommand\xrowht[2][0]{\addstackgap[.5\dimexpr#2\relax]{\vphantom{#1}}}

% theorems

\theoremstyle{plain}
\newtheorem{theorem}{Th\'eor\`eme}
\newtheorem*{sol}{Solution}
\theoremstyle{definition}
\newtheorem{ex}{Exercice}
\newtheorem*{rpl}{Rappel}
\newtheorem{enigme}{Énigme}

% corps
\usepackage{calrsfs}
\newcommand{\C}{\mathcal{C}}
\newcommand{\R}{\mathbb{R}}
\newcommand{\Rnn}{\mathbb{R}^{2n}}
\newcommand{\Z}{\mathbb{Z}}
\newcommand{\N}{\mathbb{N}}
\newcommand{\Q}{\mathbb{Q}}

% variance
\newcommand{\Var}[1]{\text{Var}(#1)}

% domain
\newcommand{\D}{\mathcal{D}}


% date
\usepackage{advdate}
\AdvanceDate[0]


% plots
\usepackage{pgfplots}

% table line break
\usepackage{makecell}
%tablestuff
\def\arraystretch{2}
\setlength\tabcolsep{15pt}

%subfigures
\usepackage{subcaption}

\definecolor{myg}{RGB}{56, 140, 70}
\definecolor{myb}{RGB}{45, 111, 177}
\definecolor{myr}{RGB}{199, 68, 64}

% fake sections with no title to move around the merged pdf
\newcommand{\fakesection}[1]{%
  \par\refstepcounter{section}% Increase section counter
  \sectionmark{#1}% Add section mark (header)
  \addcontentsline{toc}{section}{\protect\numberline{\thesection}#1}% Add section to ToC
  % Add more content here, if needed.
}


% SOLUTION SWITCH
\newif\ifsolutions
				\solutionstrue
				%\solutionsfalse

\ifsolutions
	\newcommand{\exe}[2]{
		\begin{ex} #1  \end{ex}
		\begin{sol} #2 \end{sol}
	}
\else
	\newcommand{\exe}[2]{
		\begin{ex} #1  \end{ex}
	}
	
\fi


% tableaux var, signe
\usepackage{tkz-tab}


%pinfty minfty
\newcommand{\pinfty}{{+}\infty}
\newcommand{\minfty}{{-}\infty}

\begin{document}
\input{adr/vars_1.adr}

\pagestyle{fancy}
\fancyhead[L]{Seconde 13}
\fancyhead[C]{\textbf{Devoir Maison 4 -- \seed \ifsolutions \, -- Solutions  \fi}}
\fancyhead[R]{\today}

%RATIO = $\ratio$, XMAX = $\xmax$
Dans un jeu de tir, un joueur est placé en $O$ et fait face à un adversaire immobile placé en $A$.
% situé à $\L$ unités de distance.
%Il se situe à $\L$ unités de distance de son adversaire immobile placé en $A$.
$\L$ unités de distance les sépare.

On estime que, étant donné le temps de réaction de l'adversaire, le joueur peut se déplacer de $\l$ unités avant d'être attaqué.
On pose $B$ le point atteint par le joueur après son déplacement et on considère le triangle $OAB$ ainsi formé.

On note $\theta = \widehat{BAO}$, l'angle de correction que doit appliquer l'adversaire avant de tirer.

Le joueur souhaite connaître son angle $\alpha = \widehat{AOB}\leq 90$° de déplacement lui permettant d'optimiser son avantage : il veut se déplacer de sorte que l'angle de correction $\theta$ soit maximal.
%On supposera que $\alpha \leq 90$°.

\begin{center}
\begin{tikzpicture}[scale=1]
	
	% triangle
	\node[black, below] at (0,0) {$O$};
	\node[black, above] at  (0,4) {$A$};
	\node[black, left] at  (-3,1) {$B$};
	
	\draw[-, thick, black] (0,0) -- (0,4);
	\draw[-, thick, black] (0,4) -- (-3,1);
	\draw[-, thick, black] (-3,1) -- (0,0);
	
	% arc de cercle centré en O passant par B
	\draw [black,thick, dashed,domain=120:180] plot ({3.16*cos(\x)}, {3.16*sin(\x)}) ;
	
	% angles alpha, theta
	\draw [black,thick,domain=90:160] plot ({.2*cos(\x)}, {.2*sin(\x)});
	\node[below] at (-.2,.6) {$\alpha$};
	\draw [black,thick,domain=-90:-130] plot ({.2*cos(\x)}, {4+.2*sin(\x)});
	\node[below] at (-.2,3.8) {$\theta$};
	
	% length braces
	\draw [decorate,decoration={brace,amplitude=5pt,mirror, raise=2pt}]
  (-3,1) -- (0,0) node[midway, below = 5pt]{$\l$};
	\draw [decorate,decoration={brace,amplitude=5pt,mirror, raise=2pt}]
  (0,0) -- (0,4) node[midway, right = 5pt]{$\L$};
\end{tikzpicture}
\end{center}

\exe{
	Dessiner le projeté orthogonal $I$ de $B$ sur $(OA)$ qui définit la hauteur du triangle $OAB$ issue de $B$.
	On pose $h = BI$ cette hauteur.
}{}

\exe{
	Montrer qu'on a
		\[ h = \l\sin(\alpha). \]
}{}

\exe{
	Montrer que la base $OI$ est donnée par
		\[ OI = \l\cos(\alpha). \]
}{}

\exe{
	Montrer qu'on a 
		\[ \tan(\theta) = \dfrac{\l \sin(\alpha)}{\L - \l \cos(\alpha)}. \]
}{}

\exe{
	%À l'aide de l'outil \texttt{Table} de la calculatrice ou d'un programme informatique et en approximant à $10^{-4}$, compléter les tableaux de valeurs de la fonction $f$ donnée algébriquement par
	En approximant à $10^{-4}$ , compléter les tableaux de valeurs de la fonction $f$ donnée algébriquement par
		\[ f(\alpha ) = \arctan \left(\dfrac{\l \sin(\alpha)}{\L - \l \cos(\alpha)} \right). \]
	
	\begin{center}
	\hspace{-1cm}
	\begin{tabular}{|c|c|}\hline
		$\alpha$ & $f(\alpha)$ \\ \hline
		$\sampleI$ & \\ \hline
		$\sampleII$ & \\ \hline
		$\sampleIII$ & $\imageIII$\\ \hline
		$\sampleIV$ & \\ \hline
	\end{tabular}
	\hfill
	\begin{tabular}{|c|c|}\hline
		$\alpha$ & $f(\alpha)$ \\ \hline
		$\sampleV$ & \\ \hline
		$\sampleVI$ & \\ \hline
		$\sampleVII$ & \\ \hline
		$\sampleVIII$ & $\imageVIII$ \\ \hline
	\end{tabular}
	\hfill
	\begin{tabular}{|c|c|}\hline
		$\alpha$ & $f(\alpha)$ \\ \hline
		$\sampleIX$ & \imageIX \\ \hline
		$\sampleX$ & \\ \hline
		$\sampleXI$ & \\ \hline
		$\sampleXII$ & \\ \hline
	\end{tabular}
	\hfill
	\begin{tabular}{|c|c|}\hline
		$\alpha$ & $f(\alpha)$ \\ \hline
		$\sampleXIII$ & \\ \hline
		$\sampleXIV$ & \\ \hline
		$\sampleXV$ & $\imageXV$ \\ \hline
		$\sampleXVI$ & \\ \hline
	\end{tabular}
	\hfill
	\begin{tabular}{|c|c|}\hline
		$\alpha$ & $f(\alpha)$ \\ \hline
		$\sampleXVII$ & $\imageXVII$ \\ \hline
		$\sampleXVIII$ & \\ \hline
		$\sampleXIX$ & \\ \hline
		$\sampleXX$ & \\ \hline
	\end{tabular}
	\end{center}
}{}

\exe{
	Tracer $\C_f$ dans un repère de domaine $[0; 90]$ et estimer l'angle $\alpha^\star \in [0;90]$ pour lequel $f(\alpha^\star)$ est maximal.
}{}

\exe{
	Calculer les coordonnées à $10^{-4}$ près du point d'abscisse $\arccos\left(\dfrac{\l}{\L}\right)$ appartenant $\C_f$ puis le placer sur la courbe de la question précédente.
}{}


\end{document}

%% INPUT PREAMBLE.TEX
%% THEN INPUT .ADR
%% THEN THIS

\SetDate[03/11/2025]

\begin{document}
\pagestyle{fancy}
\fancyhead[L]{Seconde}
\fancyhead[C]{\textbf{Devoir Maison -- {\seed} -- Évolution}}
\fancyhead[R]{\today}

\fakesection{Devoir \seed}

%%%%%%%%%%%%%%%%%%%
%%%%%%%%%%%%%%%%%%%
%%%%%%%%%%%%%%%%%%%

\exe{}{
	La taxe sur la valeur ajoutée (TVA) est un impôt indirect sur la consommation.
	On considère dans cet exercice une TVA fictive fixée à $\TVA\%$.
	Tout objet est alors vendu $\TVA\%$ plus cher que son prix initial : c'est la différence entre un prix « hors taxes » (HT) et « toutes taxes comprises » (TTC).
	
	Un prix TTC comprend donc deux parts : l'une qui revient au vendeur, et l'autre versée à l'État.
	
	\begin{enumerate}
		\item
		Quel est le prix TTC d'un objet coûtant $\A$€ HT ?
		\item
		Quel est le prix HT d'un objet coûtant $\B$€ TTC ?		
%		\item 
%		Pour les deux exemples ci-dessus, calculer la proportion \emph{exacte} du prix TTC qui revient à l'État. 
%		C'est-à-dire, calculer le rapport
%			\[ \dfrac{\text{part qui revient à l'état (TVA)}}{\text{prix total TTC}}. \]
%		\item
%		Montrer que la proportion du prix TTC qui revient à l'État est constante et ne dépend pas du prix (si tant est qu'il soit non nul).
%		Est-ce un nombre réel ? rationnel ? décimal ? entier ?
	\end{enumerate}
}{exe:TVA}{
	\begin{enumerate}
		\item
			En arrondissant au centime près, on augmente le prix de $\TVA\%$ en calculant
				\[ \left(1+\dfrac{\TVA}{100} \right) \cdot \A \approx \Asol. \]
			Le prix avec TVA comprise est donc de $\Asol$€.
		\item
			Il s'agit d'appliquer l'évolution réciproque de $+\TVA\%$.
			En arrondissant au centime près, on calcule
				\[ \left(1+ \dfrac{\TVA}{100}\right)^{-1} \cdot \B \approx \Bsol. \]
			Le prix sans TVA est donc de $\Bsol$€.
%		\item
%			Il s'agit ici de calculer la proportion 
%				\[ \dfrac{\text{part qui revient à l'état (TVA)}}{\text{prix total TTC}}. \]
%			Pour le premier objet, on calcule
%				\[ \dfrac{\dfrac{\TVA}{100} \cdot \A}{\left(1+\dfrac{\TVA}{100} \right) \cdot \A} = \dfrac{\rationum}{\ratiodenom} \]
%			Pour le deuxième objet, on calcule
%				\[ \dfrac{\B - \left(1- \dfrac{\TVA}{100}\right) \cdot \B }{\B} = \dfrac{\rationum}{\ratiodenom}. \]
%		\item
%			Soit $P>0$ un prix hors taxe quelconque.
%			La part de TVA par rapport au prix TTC est égale à
%				\begin{align*}
%					& \dfrac{\dfrac{\TVA}{100} \cdot P}{\left(1 + \dfrac{\TVA}{100}\right) \cdot P} \\
%					=& \dfrac{\dfrac{\TVA}{100}}{\left(1 + \dfrac{\TVA}{100}\right)} \\
%					&= \dfrac{\TVA}{\left(100 + \TVA\right)} = \dfrac{\rationum}{\ratiodenom},
%				\end{align*}
%			qui ne dépend bien pas du prix $P$ choisi.
	\end{enumerate}
}



\exe{}{
	En vue des soldes, un magasin augmente frauduleusement ses prix avant d'appliquer une remise de $\REMISE\%$.
	On compare les prix initiaux avant augmentation aux prix finaux après application des deux évolutions.
	
	Répondre aux questions suivantes en arrondissant les pourcentages à l'unité.
	\begin{enumerate}
		\item 
		Quelle augmentation faut-il effectuer pour que les prix ne changent pas, avant et après évolutions ?
		%Quelle augmentation faut-il effectuer pour retrouver le prix initial ?
		\item
		Quelle augmentation faut-il effectuer pour que, au final, les prix aient augmenté de $\FINALaug\%$ ?
		\item
		Quelle augmentation faut-il effectuer pour que, au final, les prix aient diminué de $\FINALdim\%$ ?
	\end{enumerate}
}{exe:magasin}{
	\begin{enumerate}
		\item
		Une diminution de $\REMISE\%$ correspond à une multiplication par le coefficient
			\[ 1 - \dfrac{\REMISE}{100}. \]
		L'évolution réciproque est donc donnée par une multiplication par
			\[ \dfrac{1}{1 - \dfrac{\REMISE}{100}} = \dfrac{100}{100-\REMISE} \approx \reciproque.\]
		Ceci correspond à une augmentation arrondie de $\recsol \%$.
		
		\item
		Le coefficient multiplicateur $x$ à trouver vérifie
			\[ x \cdot \left(1-\dfrac{\REMISE}{100}\right) = 1+\dfrac{\FINALaug}{100}. \]
		On résoud pour trouver $x \approx \evoldeux$, ce qui correspond à une augmentation arrondie de $\FINALaugsol\%$.
	
		 \item 
		Le coefficient multiplicateur $x$ à trouver vérifie
			\[ x \cdot \left(1-\dfrac{\REMISE}{100}\right) = 1-\dfrac{\FINALdim}{100}. \]
		On résoud pour trouver $x \approx \evoltrois$, ce qui correspond à une augmentation arrondie de $\FINALdimsol\%$.
	\end{enumerate}
}



%%%%%%%%%%%%%%%%%%%
%%%%%%%%%%%%%%%%%%%
%%%%%%%%%%%%%%%%%%%

\newpage
\fancyhead[C]{\textbf{Solutions -- {\seed} -- Interpolation de Lagrange}}
\fakesection{Solution \seed}
\shipoutAnswer

\end{document}


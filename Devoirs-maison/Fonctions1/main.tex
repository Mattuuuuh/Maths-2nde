%% INPUT PREAMBLE.TEX
%% THEN INPUT .ADR
%% THEN THIS
%% THEN POSTAMBLE


\begin{document}
\pagestyle{fancy}
\fancyhead[L]{Seconde 13}
\fancyhead[C]{\textbf{Devoir Maison -- {\seed} -- Interpolation de Lagrange}}
\fancyhead[R]{\today}

\fakesection{Devoir \seed}

%%%%%%%%%%%%%%%%%%%
%%%%%%%%%%%%%%%%%%%
%%%%%%%%%%%%%%%%%%%

\exe{}{
	On définit algébriquement les fonctions $a, b,$ et $c$ de la façon suivante.
		\begin{align*}
		a(x) = \qI(x-\xII)(x-\xIII),
		&&
		b(x) =  \qII(x-\xI)(x-\xIII),
		&&
		c(x) = \qIII(x-\xI)(x-\xII).
		\end{align*}
	\underline{Sans développer les expressions}, remplir les images de $a, b, c$ dans le tableau figure \ref{fig:1}.
}{exe:1}{
	\begin{center}
	\def\arraystretch{1.5}
	\setlength\tabcolsep{40pt}
	\begin{tabular}{|c|c|c|c|}\hline
		$x$ & 1 & 2 & 3 \\ \hline
		$a(x)$ & 1 & 0 & 0 \\ \hline
		$b(x)$ & 0 & 1 & 0 \\ \hline
		$c(x)$ & 0 & 0 & 1 \\ \hline
	\end{tabular}
	\end{center}
}

\exe{}{
	En gardant les notations de l'exercice \ref{exe:1}, posons désormais la fonction
		\[ f(x) = \fI a(x) + \fII b(x) + \fIII c(x). \]
	\underline{Sans développer l'expression}, remplir les images de $f$ dans le tableau figure \ref{fig:1}.
}{exe:2}{
	\begin{center}
	\def\arraystretch{1.5}
	\setlength\tabcolsep{40pt}
	\begin{tabular}{|c|c|c|c|}\hline
		$x$ & \xI & \xII & \xIII \\ \hline
		$f(x)$ & \fI & \fII & \fIII \\ \hline
	\end{tabular}
	\end{center}
}

\exe{, difficulty=1}{
	Développer et réduire l'expression de $f$ de l'exercice \ref{exe:2} en substituant les expressions de l'exercice \ref{exe:1} pour trouver que $f(x) = x \bI$.
}{exe:3}{
	On développe $a, b, c$ séparément puis on combine les résultats pour trouver la forme développée réduite de $f$.
%	\[ a(x) = \dfrac12(x-2)(x-3) = \dfrac12 (x^2 - 3x -2x + 6) = \dfrac12 x^2 - \dfrac52 x + 3. \]
%	\[ b(x) = -(x-1)(x-3) = -(x^2 -3x -x + 3) = -x^2 + 4x - 3. \]
%	\[ c(x) = \dfrac12(x-1)(x-2) = \dfrac12 (x^2 - 2x - x + 2) = \dfrac12x^2 - \dfrac32 x + 1. \]
%	
%	D'où, 
%	\begin{align*}
%		f(x) &= 3a(x) + 4b(x) + 5c(x), \\
%			&= \dfrac32 x^2 - \dfrac{15}2 x + 9 -4x^2 + 16x - 12 + \dfrac52x^2 - \dfrac{15}2 x + 5, \\
%			&= 0x^2 + 1x + 2, \\
%			&= x + 2.
%	\end{align*}
}

\begin{figure}[h!]
	\begin{center}
	\def\arraystretch{2}
	\setlength\tabcolsep{40pt}
	\begin{tabular}{|c|c|c|c|}\hline
		$x$ & \xI & \xII & \xIII \\ \hline
		$a(x)$ & && \\ \hline
		$b(x)$ & && \\ \hline
		$c(x)$ & && \\ \hline
		$f(x)$ & && \\ \hline
	\end{tabular}
	\end{center}
	\caption{Tableau de valeurs de $a, b, c,$ et $f$.}
	\label{fig:1}
\end{figure}

\subsection*{Questions ouvertes (non notées)}

\exe{, difficulty=1}{
	En adaptant les exercices précédents, donner une fonction $g$ vérifiant $g(1) = 3$, $g(2) =4,$ et $g(3) = 7$.
	Développer et réduire l'expression de $g$ pour trouver $g(x) = x^2 - 2x + 4$.
}{exe:4}{
	%On considère $g(x) = 3a(x) + 4b(x) + 7c(x)$ et qu'on développera calmement...
}


\exe{, difficulty=2}{
	Existe-t-il une fonction $h$ vérifiant $h(1) = y_1$, $h(2) =y_2,$ et $h(3) = y_3$ pour n'importe quels $y_1, y_2, y_3$ ?
}{exe:5}{

}


\exe{, difficulty=2}{
	Existe-t-il une seule fonction $f$ vérifiant le tableau figure \ref{fig:1} ?
}{exe:5}{
	Non, la fonction $f(x) + (x-\xI)(x-\xII)(x-\xIII)$ admet le même tableau et est différente de $f$ car sa valeur en $x=0$ est différente.
}





%%%%%%%%%%%%%%%%%%%
%%%%%%%%%%%%%%%%%%%
%%%%%%%%%%%%%%%%%%%

\newpage
\fancyhead[C]{\textbf{Solutions -- {\seed} -- Interpolation de Lagrange}}
\fakesection{Solution \seed}
\shipoutAnswer

\end{document}


%% INPUT PREAMBLE.TEX
%% THEN INPUT VARS_{i}.ADR
%% THEN RUN THIS
%% DYSLEXIA SWITCH
\newif\ifdys
		
				% ENABLE or DISABLE font change
				% use XeLaTeX if true
				\dystrue
				\dysfalse


\ifdys

\documentclass[a4paper, 14pt]{extarticle}
\usepackage{amsmath,amsfonts,amsthm,amssymb,mathtools}

\tracinglostchars=3 % Report an error if a font does not have a symbol.
\usepackage{fontspec}
\usepackage{unicode-math}
\defaultfontfeatures{ Ligatures=TeX,
                      Scale=MatchUppercase }

\setmainfont{OpenDyslexic}[Scale=1.0]
\setmathfont{Fira Math} % Or maybe try KPMath-Sans?
\setmathfont{OpenDyslexic Italic}[range=it/{Latin,latin}]
\setmathfont{OpenDyslexic}[range=up/{Latin,latin,num}]

\else

\documentclass[a4paper, 12pt]{extarticle}

\usepackage[utf8x]{inputenc}
%fonts
\usepackage{amsmath,amsfonts,amsthm,amssymb,mathtools}
% comment below to default to computer modern
\usepackage{libertinus,libertinust1math}

\fi


\usepackage[french]{babel}
\usepackage[
a4paper,
margin=2cm,
nomarginpar,% We don't want any margin paragraphs
]{geometry}
\usepackage{icomma}

\usepackage{fancyhdr}
\usepackage{array}
\usepackage{hyperref}

\usepackage{multicol, enumerate}
\newcolumntype{P}[1]{>{\centering\arraybackslash}p{#1}}


\usepackage{stackengine}
\newcommand\xrowht[2][0]{\addstackgap[.5\dimexpr#2\relax]{\vphantom{#1}}}

% theorems

\theoremstyle{plain}
\newtheorem{theorem}{Th\'eor\`eme}
\newtheorem*{sol}{Solution}
\theoremstyle{definition}
\newtheorem{ex}{Exercice}
\newtheorem*{rpl}{Rappel}
\newtheorem{enigme}{Énigme}

% corps
\usepackage{calrsfs}
\newcommand{\C}{\mathcal{C}}
\newcommand{\R}{\mathbb{R}}
\newcommand{\Rnn}{\mathbb{R}^{2n}}
\newcommand{\Z}{\mathbb{Z}}
\newcommand{\N}{\mathbb{N}}
\newcommand{\Q}{\mathbb{Q}}

% variance
\newcommand{\Var}[1]{\text{Var}(#1)}

% domain
\newcommand{\D}{\mathcal{D}}


% date
\usepackage{advdate}
\AdvanceDate[0]


% plots
\usepackage{pgfplots}

% table line break
\usepackage{makecell}
%tablestuff
\def\arraystretch{2}
\setlength\tabcolsep{15pt}

%subfigures
\usepackage{subcaption}

\definecolor{myg}{RGB}{56, 140, 70}
\definecolor{myb}{RGB}{45, 111, 177}
\definecolor{myr}{RGB}{199, 68, 64}

% fake sections with no title to move around the merged pdf
\newcommand{\fakesection}[1]{%
  \par\refstepcounter{section}% Increase section counter
  \sectionmark{#1}% Add section mark (header)
  \addcontentsline{toc}{section}{\protect\numberline{\thesection}#1}% Add section to ToC
  % Add more content here, if needed.
}


% SOLUTION SWITCH
\newif\ifsolutions
				\solutionstrue
				%\solutionsfalse

\ifsolutions
	\newcommand{\exe}[2]{
		\begin{ex} #1  \end{ex}
		\begin{sol} #2 \end{sol}
	}
\else
	\newcommand{\exe}[2]{
		\begin{ex} #1  \end{ex}
	}
	
\fi


% tableaux var, signe
\usepackage{tkz-tab}


%pinfty minfty
\newcommand{\pinfty}{{+}\infty}
\newcommand{\minfty}{{-}\infty}

\begin{document}
\input{adr/vars_45294.adr}

\pagestyle{fancy}
\fancyhead[L]{Seconde 13}
\fancyhead[C]{\textbf{Devoir Maison 1 -- \seed \ifsolutions \, -- Solutions  \fi}}
\fancyhead[R]{\today}


\exe{	
	La taxe sur la valeur ajoutée (TVA) est un impôt indirect sur la consommation.
	On considère dans cet exercice une TVA fictive fixée à $\TVA\%$.
	Tout objet est alors vendu $\TVA\%$ plus cher que son prix initial : c'est la différence entre un prix hors taxes (HT) et toutes taxes comprises (TTC).
	
	Un prix TTC comprend donc deux parts : l'une qui revient au vendeur, et l'autre versée à l'État.
	
	\begin{enumerate}
		\item
		Quel est le prix toutes taxes comprises d'un objet coûtant $\A$€ hors taxes ?
		\item
		Quel est le prix HT d'un objet coûtant $\B$€ TTC ?		
		\item 
		Pour les deux exemples ci-dessus, calculer la proportion \emph{exacte} du prix TTC qui revient à l'État. 
		\item
		Montrer que la proportion du prix TTC qui revient à l'État est constante et ne dépend pas du prix.
	\end{enumerate}
}{
	\begin{enumerate}
		\item
			En arrondissant au centime près, on augmente le prix de $\TVA\%$ en calculant
				\[ \left(1+\dfrac{\TVA}{100} \right) \cdot \A \approx \Asol. \]
			Le prix avec TVA comprise est donc de $\Asol$€.
		\item
			Il s'agit d'appliquer l'évolution réciproque de $+\TVA\%$.
			En arrondissant au centime près, on calcule
				\[ \left(1+ \dfrac{\TVA}{100}\right)^{-1} \cdot \B \approx \Bsol. \]
			Le prix sans TVA est donc de $\Bsol$€.
		\item
			Il s'agit ici de calculer la proportion 
				\[ \dfrac{\text{part qui revient à l'état (TVA)}}{\text{prix total TTC}}. \]
			Pour le premier objet, on calcule
				\[ \dfrac{\dfrac{\TVA}{100} \cdot \A}{\left(1+\dfrac{\TVA}{100} \right) \cdot \A} = \dfrac{\rationum}{\ratiodenom} \]
			Pour le deuxième object, on calcule
				\[ \dfrac{\B - \left(1- \dfrac{\TVA}{100}\right) \cdot \B }{\B} = \dfrac{\rationum}{\ratiodenom}. \]
		\item
			Soit $P\geq0$ un prix hors taxe quelconque.
			La part de TVA par rapport au prix TTC est égale à
				\begin{align*}
					& \dfrac{\dfrac{\TVA}{100} \cdot P}{\left(1 + \dfrac{\TVA}{100}\right) \cdot P} \\
					=& \dfrac{\dfrac{\TVA}{100}}{\left(1 + \dfrac{\TVA}{100}\right)} \\
					&= \dfrac{\TVA}{\left(100 + \TVA\right)} = \dfrac{\rationum}{\ratiodenom},
				\end{align*}
			qui ne dépend bien pas du prix $P$ choisi.
	\end{enumerate}
}

\exe{
	Une cordonnière vend des chaussures de différentes qualité. Chaque catégorie a son prix et sa quantité en vente indiqués dans le tableau suivant.
	
		\begin{center}
		\begin{tabular}{|c|c|c|c|c|}\hline
		Prix TTC (€)  & \P & \PP & \PPP & \PPPP \\ \hline
		Quantité & \Q & \QQ & \QQQ & \QQQQ \\ \hline
		\end{tabular}
		\end{center}

	\begin{enumerate}
		\item
		Quel est le prix TTC moyen d'une chaussure en vente ? Arrondir au centime d'euro.
	\end{enumerate}
	
	On admet que chaque prix comprend une TVA de $\TVA\%$.
	Pour sa comptabilité, la cordonnière souhaite connaître les prix hors taxe (HT).
	En outre, ces prix HT comprennent un coût fixe de $\PFIXE$€, coût des semelles qu'elle souhaite retirer pour qu'il ne reste que le vrai prix des chaussures.
	\begin{enumerate}
		\item[2.]
		Donner le vrai prix moyen HT des chaussures en ventes de deux façons différentes :
			\begin{enumerate}[a)]
				\item en ajoutant une ligne \og Vrai prix HT (€)\fg \, au tableau et en calculant la moyenne ; et 
				\item en utilisant la linéarité de la moyenne.
			\end{enumerate}
	\end{enumerate}
}{
	\begin{enumerate}
		\item
		La moyenne est donnée par 
			\[ \dfrac{\P\cdot\Q+\PP\cdot\QQ+\PPP\cdot\QQQ+\PPPP\cdot\QQQQ}{\Q+\QQ+\QQQ+\QQQQ} \approx \average.\]
	
		\item
		Pour chaque prix, il s'agit d'abord de supprimer la TVA en le multipliant par $\dfrac{1}{1^+\dfrac{\TVA}{100}}$, puis de retirer $\PFIXE$€, coût des semelles.
		On obtient donc le tableau suivant, en arrondissant au centime près.
		
			\begin{center}
			\begin{tabular}{|c|c|c|c|c|}\hline
			Vrai prix HC (€)  & \Psol & \PPsol & \PPPsol & \PPPPsol \\ \hline
			Quantité & \Q & \QQ & \QQQ & \QQQQ \\ \hline
			\end{tabular}
			\end{center}
	
		La nouvelle moyenne est donnée par 
			\[ \dfrac{\Psol\cdot\Q+\PPsol\cdot\QQ+\PPPsol\cdot\QQQ+\PPPPsol\cdot\QQQQ}{\Q+\QQ+\QQQ+\QQQQ} \approx \newaverage.\]
			
		Alternativement, d'après la linéarité de la moyenne et en appliquant les opérations directement à la moyenne calculée précédemment, on trouve
			\[ \average \cdot \dfrac{1}{1+\dfrac{\TVA}{100}} - \PFIXE \approx \newaverage. \]
	
	\end{enumerate}
}

\exe{
		Un magasin augmente ses prix avant d'appliquer une remise de $\REMISE\%$.
		On compare les prix initiaux avant augmentation aux prix finaux après application des deux évolutions.
		
		\begin{enumerate}
			\item 
			Quelle augmentation faut-il effectuer pour que les prix ne changent pas, après et avant évolutions ?
			\item
			Quelle augmentation faut-il effectuer pour qu'au final, les prix aient augmenté de $\FINALaug\%$ ?
			\item
			Quelle augmentation faut-il effectuer pour qu'au final, les prix aient diminué de $\FINALdim\%$ ?
		\end{enumerate}
}{
	\begin{enumerate}
		\item
		Une diminution de $\REMISE\%$ correspond à une multiplication par le coefficient
			\[ 1 - \dfrac{\REMISE}{100}. \]
		L'évolution réciproque est donc donnée par une multiplication par
			\[ \dfrac{1}{1 - \dfrac{\REMISE}{100}} = \dfrac{100}{100-\REMISE} \approx \reciproque.\]
		Ceci correspond à une augmentation arrondie de $\recsol \%$.
		
		\item
		Le coefficient multiplicateur $x$ à trouver vérifie
			\[ x \cdot \left(1-\dfrac{\REMISE}{100}\right) = 1+\dfrac{\FINALaug}{100}. \]
		On résoud pour trouver $x \approx \evoldeux$, ce qui correspond à une augmentation arrondie de $\FINALaugsol\%$.
	
		 \item 
		Le coefficient multiplicateur $x$ à trouver vérifie
			\[ x \cdot \left(1-\dfrac{\REMISE}{100}\right) = 1-\dfrac{\FINALdim}{100}. \]
		On résoud pour trouver $x \approx \evoltrois$, ce qui correspond à une augmentation arrondie de $\FINALdimsol\%$.
	\end{enumerate}
}

\ifsolutions
\else
\newpage
\fi

\exe{
	Écrire un tableau Valeur/Effectif pour chaque histogramme de la figure \ref{fig:hist}.
	On assignera la valeur moyenne à chaque élément d'une classe.
	Par exemple, de l'histogramme \ref{fig:a} on lit $\EFFCONCa$ notes de valeur $\CONCa,5$.
	
	Ensuite, pour chaque série obtenue, calculer
		\begin{multicols}{2}
		\begin{enumerate}[i)]
			\item La moyenne ;
			\item L'écart type ;
			\item La médiane ;
			\item Le premier quartile ;
			\item Le troisième quartile ; et
			\item L'écart interquartile.
		\end{enumerate}
		\end{multicols}
		
	Comparer les séries statistiques en s'appuyant sur les valeurs calculées.
}{
	Pour l'histogramme \ref{fig:a}, on lit le tableau Valeur/Effectif suivant.
			\begin{center}
			\begin{tabular}{|c|c|c|c|c|}\hline
			Note  & \vala & \valb & \valc & \vald \\ \hline
			Effectif & \EFFCONCa & \EFFCONCb & \EFFCONCc & \EFFCONCd \\ \hline
			\end{tabular}
			\end{center}
	
	On en déduit les valeurs suivantes, arrondies au centième.
	\begin{enumerate}[i)]
		\item La moyenne est de $\overline{X} = \Concaverage$ ;
		\item L'écart type est $\sigma(X) = \Concstd$ ;
		\item La médiane est $\Concmedian$ ;
		\item Le premier quartile est $Q_1 = \Concfirstquartile$ ;
		\item Le troisième quartile est $Q_3 = \Concthirdquartile$ ; et
		\item L'écart interquartile est $Q_3 - Q_1 = \Concecart$.
	\end{enumerate}

	Pour l'histogramme \ref{fig:b}, on lit le tableau Valeur/Effectif suivant.
			\begin{center}
			\begin{tabular}{|c|c|c|c|c|c|}\hline
			Note  & \valA & \valB & \valC & \valD & \valE \\ \hline
			Effectif & \EFFDISPa & \EFFDISPb & \EFFDISPc & \EFFDISPd  & \EFFDISPe \\ \hline
			\end{tabular}
			\end{center}
	
	On en déduit les valeurs suivantes, arrondies au centième.
	\begin{enumerate}[i)]
		\item La moyenne est de $\overline{X} = \Dispaverage$ ;
		\item L'écart type est $\sigma(X) = \Dispstd$ ;
		\item La médiane est $\Dispmedian$ ;
		\item Le premier quartile est $Q_1 = \Dispfirstquartile$ ;
		\item Le troisième quartile est $Q_3 = \Dispthirdquartile$ ; et
		\item L'écart interquartile est $Q_3 - Q_1 = \Dispecart$.
	\end{enumerate}


	En conclusion, la première série statistique est bien plus concentrée autour de sa moyenne que la deuxième.
	On le voit qualitativement en comparant les écarts types ou en comparant les écarts interquartiles.
	L'écart $|Q_1 - Q_3|$ donne l'étendue que prend la moitié des valeurs centrales. 
	Plus cet écart est grand, plus il existe des valeurs centrales espacées les unes des autres.
	
	Ce résultat n'est pas surprenant car on distingue un groupe uni de valeurs sur l'histogramme \ref{fig:a}, et deux groupes disjoints sur l'histogramme \ref{fig:b}.
	Dans le contexte des notes, un écart type élevé indique une haute hétérogénéité de résultats dans une classe. 
	L'histrogramme \ref{fig:b} permet dans ce cas de créer des groupes de niveau.
}

%\ifsolutions
%\newpage
%\fi

\begin{figure}[t!]
  \centering
  \begin{subfigure}[b]{.45\textwidth}
    \centering
  \begin{tikzpicture}[scale=1]
    \begin{axis}[
    	ymin=0,
        %ymin=0, ymax=8,
        %minor y tick num = 3,
        xmin = 0, xmax=20,
        xtick = {2, 4, ..., 18},
        area style,
        xlabel = {Note sur $20$},
        ylabel = {Effectif}
      ]
      \addplot+[ybar interval,mark=no, draw=myg, fill=myg!50] plot coordinates {
        (\CONCa, \EFFCONCa) (\CONCb, \EFFCONCb) (\CONCc, \EFFCONCc) (\CONCd, \EFFCONCd) (\CONCd+1, 0)
      };
    \end{axis} 
  \end{tikzpicture}
  \caption{Première série.}
  \label{fig:a}
  \end{subfigure}
  \hfill
  % NOT LINE BREAK!!
  \begin{subfigure}[b]{.45\textwidth}
    \centering
  \begin{tikzpicture}[scale=1]
    \begin{axis}[
    	ymin=0,
        %ymin=0, ymax=8,
        %minor y tick num = 3,
        xmin = 0, xmax=20,
        xtick = {2, 4, ..., 18},
        area style,
        xlabel = {Note sur $20$},
        ylabel = {Effectif}
      ]
      \addplot+[ybar interval,mark=no, draw=myb, fill=myb!50] plot coordinates {
        (\DISPa, \EFFDISPa) (\DISPb, \EFFDISPb) (\DISPc, \EFFDISPc) (\DISPc+1, 0) (\DISPd, \EFFDISPd) (\DISPe, \EFFDISPe) (\DISPe+1, 0)
      };
    \end{axis} 
  \end{tikzpicture}
  \caption{Deuxième série.}
  \label{fig:b}
  \end{subfigure}
  
  
  \caption{Histogrammes de notes (min $0$, max $20$). Les classes sont de la forme $[k;k+1[$ où $k\in\N$ est un entier naturel. 
  La colonne entre $\CONCa$ et $\CONCb$ compte donc toutes les notes appartenant à $[\CONCa;\CONCb[$.}
  \label{fig:hist}
\end{figure}


\end{document}

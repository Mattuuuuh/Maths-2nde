%% INPUT PREAMBLE.TEX
%% THEN INPUT .ADR
%% THEN THIS

\SetDate[30/09/2025]

\begin{document}
\pagestyle{fancy}
\fancyhead[L]{Seconde}
\fancyhead[C]{\textbf{Devoir Maison -- {\seed} -- Plan cartésien}}

\fancyhead[R]{\today}

\fakesection{Devoir \seed}

%%%%%%%%%%%%%%%%%%%
%%%%%%%%%%%%%%%%%%%
%%%%%%%%%%%%%%%%%%%

\exe{, difficulty=0}{
	Considérons les points 
	\begin{align*}
		A\left(\xA ; \yA\right), && B\left(\xB ; \yB\right), && \et && C\left(\xC ; \yC\right).
	\end{align*}
	\begin{enumerate}
		\item 
		Construire un repère et y placer les points $A, B$, et $C$.
		
		\item 
		Dans le triangle $ABC$, calculer
		\begin{multicols}{3}
		\begin{enumerate}[label=\alph*)]
			\item $AB^2$ ;
			\item $AC^2$ ; et
			\item $BC^2$.
		\end{enumerate}
		\end{multicols}
		\item
		Quelles longueurs sont entières ? quelles longueurs ne le sont pas ? Justifier.
		\item 
		Que dire du triangle $ABC$ ?
	\end{enumerate}
}{exe:1}
{
	\begin{center}
	\begin{tikzpicture}
	\begin{axis}[xmin=\xmin, xmax=\xmax, ymin=\ymin, ymax=\ymax, axis x line=middle, axis y line=middle, axis line style=<->, xlabel={}, ylabel={}, grid=both, grid style = {opacity=.5}, unit vector ratio*=1 1]
		\draw[thick] (axis cs:\realxA, \realyA) node[above, BLUE_E] {$A$} -- (axis cs:\realxB, \realyB)  node[left, GREEN_E] {$B$} -- (axis cs:\realxC, \realyC) node[below, RED_E] {$C$} -- (axis cs:\realxA, \realyA);
	\end{axis}	
	\end{tikzpicture}
	\end{center}
	
	\begin{enumerate}		
		\item[2.]
		Dans le triangle $ABC$, calculer
		\begin{enumerate}[label=\alph*)]
			\item $AB^2 = (\ABx)^2 + (\ABy)^2 = \ABsq$ ;
			\item $AC^2 = (\ACx)^2 + (\ACy)^2 = \ACsq$
			\item $BC^2 = (\BCx)^2 + (\BCy)^2 = \BCsq$.
		\end{enumerate}
		\item[3.]
		Que dire du triangle $ABC$ ?
	\end{enumerate}
}

%%%%%%%%%%%%%%%%%%%
%%%%%%%%%%%%%%%%%%%
%%%%%%%%%%%%%%%%%%%

\newpage
\fancyhead[C]{\textbf{Solutions -- {\seed} --  Plan cartésien}}
\fakesection{Solution \seed}
\shipoutAnswer

\end{document}



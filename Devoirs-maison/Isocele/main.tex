%% INPUT PREAMBLE.TEX
%% THEN INPUT .ADR
%% THEN THIS


\begin{document}
\pagestyle{fancy}
\fancyhead[L]{Seconde}
\fancyhead[C]{\textbf{Devoir Maison -- {\seed} -- Plan cartésien}}

\fancyhead[R]{\today}

\fakesection{Devoir \seed}

%%%%%%%%%%%%%%%%%%%
%%%%%%%%%%%%%%%%%%%
%%%%%%%%%%%%%%%%%%%



\exe{, difficulty=0}{
	Considérons les points 
	\begin{align*}
		A\left(\xA ; \yA\right), && B\left(\xB ; \yB\right), && \et && C\left(\xC ; \yC\right).
	\end{align*}
	\begin{enumerate}
		\item 
		Construire un repère et y placer les points $A, B$, et $C$.
		
		\item 
		Dans le triangle $ABC$, calculer
		\begin{multicols}{3}
		\begin{enumerate}[label=\alph*)]
			\item $AB^2$ ;
			\item $AC^2$ ; et
			\item $BC^2$.
		\end{enumerate}
		\end{multicols}
		\item
		Quelles longueurs sont entières ? quelles longueurs ne le sont pas ? Justifier.
		\item 
		Que dire du triangle $ABC$ ?
	\end{enumerate}
}{exe:1}
{

}


%%%%%%%%%%%%%%%%%%%
%%%%%%%%%%%%%%%%%%%
%%%%%%%%%%%%%%%%%%%

\newpage
\fancyhead[C]{\textbf{Solutions -- {\seed} --  Équations diophantiennes}}
\fakesection{Solution \seed}
\shipoutAnswer

\end{document}



%% INPUT PREAMBLE.TEX
%% THEN INPUT .ADR
%% THEN THIS

\SetDate[30/09/2025]

\begin{document}
\pagestyle{fancy}
\fancyhead[L]{Seconde}
\fancyhead[C]{\textbf{Devoir Maison -- {\seed} -- Plan cartésien}}

\fancyhead[R]{\today}

\fakesection{Devoir \seed}

%%%%%%%%%%%%%%%%%%%
%%%%%%%%%%%%%%%%%%%
%%%%%%%%%%%%%%%%%%%

\exe{, difficulty=0}{
	Considérons les points 
	\begin{align*}
		A\left(\xA ; \yA\right), && B\left(\xB ; \yB\right), && \et && C\left(\xC ; \yC\right).
	\end{align*}
	\begin{enumerate}
		\item 
		Construire un repère et y placer les points $A, B$, et $C$.
		
		\item 
		Dans le triangle $ABC$, calculer
		\begin{multicols}{3}
		\begin{enumerate}[label=\alph*)]
			\item $AB^2$ ;
			\item $AC^2$ ; et
			\item $BC^2$.
		\end{enumerate}
		\end{multicols}
		\item
		Quelles longueurs sont entières ? quelles longueurs ne le sont pas ? Justifier.
		\item 
		Que dire du triangle $ABC$ ?
	\end{enumerate}
}{exe:1}
{
	\begin{multicols}{2}
	\begin{enumerate}		
		\item[]
		\begin{center}
		\begin{tikzpicture}
		\begin{axis}[xmin=\xmin, xmax=\xmax, ymin=\ymin, ymax=\ymax, axis x line=middle, axis y line=middle, axis line style=<->, xlabel={}, ylabel={}, grid=both, grid style = {opacity=.5}, unit vector ratio=*1 1, width=20cm, height=5cm]
			\draw[thick] (axis cs:\realxA, \realyA) node[above, BLUE_E] {$A$} -- (axis cs:\realxB, \realyB)  node[left, GREEN_E] {$B$} -- (axis cs:\realxC, \realyC) node[below, RED_E] {$C$} -- (axis cs:\realxA, \realyA);
		\end{axis}	
		\end{tikzpicture}
		\end{center}
		\item[2.]
		D'après la formule $PQ^2 = \norm{P-Q}^2$, on trouve
		\begin{enumerate}[label=\alph*)]
			\item $AB^2 = (\ABx)^2 + (\ABy)^2 = \ABsq$ ;
			\item $AC^2 = (\ACx)^2 + (\ACy)^2 = \ACsq$ ;
			\item $BC^2 = (\BCx)^2 + (\BCy)^2 = \BCsq$.
		\end{enumerate}
	\end{enumerate}
	\end{multicols}
	% [resume] doesn't work. Maybe because of different scope because of multicols?
	\begin{enumerate}\addtocounter{enumi}{2}
		\item
			\ifnum\isABSquare=0
			La longueur $AB$ n'est pas entière car les carrés sont croissants et qu'on a l'encadrement
				\[ \ABlow^2 < AB^2 < \ABhigh^2. \]
			\else
			La longueur $AB$ est entière, de longueur $\ABlow$, car $\ABlow^2 = \ABsq = AB^2$.
			\fi
			\ifnum\isACSquare=0
			Les longueurs $AC$ et $BC$ ne sont pas entières car les carrés sont croissants et qu'on a l'encadrement
				\[ \AClow^2 < AC^2 = BC^2 < \AChigh^2. \]
			\else
			Les longueurs $AC$ et $BC$ sont entières, de longueur $\AClow$, car $\AClow^2 = \ACsq = AC^2 = BC^2$.
			\fi
			% isoceles so useless
			%\ifnum\isBCSquare=0
			%La longueur $BC$ n'est pas entière car les carrés sont croissants et qu'on a l'encadrement
			%	\[ \BClow^2 < BC^2 < \BChigh^2. \]
			%\else
			%La longueur $BC$ est entière, de longueur $\BClow$, car $\BClow^2 = \BCsq = BC^2$.
			%\fi
		\item
		Le triangle $ABC$ est isocèle en $C$ car $AC = BC = \AC \neq AB$.
	\end{enumerate}
}

%%%%%%%%%%%%%%%%%%%
%%%%%%%%%%%%%%%%%%%
%%%%%%%%%%%%%%%%%%%

\newpage
\fancyhead[C]{\textbf{Solutions -- {\seed} --  Plan cartésien}}
\fakesection{Solution \seed}
\shipoutAnswer

\end{document}



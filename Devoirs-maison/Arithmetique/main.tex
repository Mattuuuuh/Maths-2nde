%% INPUT PREAMBLE.TEX
%% THEN INPUT .ADR
%% THEN THIS

\SetDate[16/01/2026]

\begin{document}
\pagestyle{fancy}
\fancyhead[L]{Seconde}
\fancyhead[C]{\textbf{Devoir Maison -- {\seed} -- Arithmétique}}
\fancyhead[R]{\today}

\fakesection{Devoir \seed}

%%%%%%%%%%%%%%%%%%%
%%%%%%%%%%%%%%%%%%%
%%%%%%%%%%%%%%%%%%%

\exe{, difficulty=1}{
	Considérons un entier naturel $n\in\N$ qui est à la fois multiple de 2 et multiple de 3.
	\begin{enumerate}
		\item 
		Donner l'ensemble $A$ des multiples de 2 inférieurs ou égaux à 30.
		\item 
		Donner l'ensemble $B$ des multiples de 3 inférieurs ou égaux à 30.
		\item 
		Considérons $E = A \cap B$. Décrire avec des mots cet ensemble puis donner ses éléments.
		\item
		En déduire quelques valeurs possibles de $n$.
		Que dire sur ces valeurs ?
	\end{enumerate}
}{exe:1}{
	todo
}


\exe{, difficulty=1}{
	Considérons $n = 6120$ et $m=8120$.
	\begin{enumerate}
		\item 
		Sans calculatrice, donner la décomposition en facteurs premiers de $n$ en décrivant votre démarche.
		\item 
		Faire de même avec $m$.
		\item
		En déduire la décomposition en facteurs premiers des trois nombres suivants.
			\begin{multicols}{3}
			\begin{enumerate}	
				\item $6120 \times 8120$
				\item $6120 \times 8120^2$
				\item $6120^3$
			\end{enumerate}
			\end{multicols}
		\item
		Est-ce que $2^2\times3\times5^3$ divise 6120 ?
		Justifier sans calculs à l'aide de la question 1.
	\end{enumerate}
}{exe:2}{
	todo
}


%%%%%%%%%%%%%%%%%%%
%%%%%%%%%%%%%%%%%%%
%%%%%%%%%%%%%%%%%%%

\newpage
\fancyhead[C]{\textbf{Solutions -- {\seed} -- Arithmétique}}
\fakesection{Solution \seed}
\shipoutAnswer

\end{document}


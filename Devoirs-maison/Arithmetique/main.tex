%% INPUT PREAMBLE.TEX
%% THEN INPUT .ADR
%% THEN THIS

\SetDate[16/01/2026]

\begin{document}
\pagestyle{fancy}
\fancyhead[L]{Seconde}
\fancyhead[C]{\textbf{Devoir Maison -- {\seed} -- Arithmétique}}
\fancyhead[R]{\today}

\fakesection{Devoir \seed}

%%%%%%%%%%%%%%%%%%%
%%%%%%%%%%%%%%%%%%%
%%%%%%%%%%%%%%%%%%%

\exe{, difficulty=0}{
	Considérons un entier naturel $n\in\N$ qui est à la fois multiple de $\aI$ et multiple de $\bI$.
	\begin{enumerate}
		\item 
		Donner l'ensemble $A$ des multiples de $\aI$ inférieurs ou égaux à $\bornesup$.
		\item 
		Donner l'ensemble $B$ des multiples de $\bI$ inférieurs ou égaux à $\bornesup$.
		\item 
		Considérons $E = A \cap B$. Décrire avec des mots cet ensemble puis donner ses éléments.
		\item
		En déduire quelques valeurs possibles de $n$.
		Que dire sur ces valeurs ?
	\end{enumerate}
}{exe:1}{
	todo
}



% énoncé
\newcommand{\firstdiv}{
	\ifnum\dIpowtwoII=0 \else \ifnum\dIpowtwoII=1 2 \times \else 2^{\dIpowtwoII} \times \fi\fi
	\ifnum\dIpowthreeII=0 \else \ifnum\dIpowthreeII=1 3 \times \else 3^{\dIpowthreeII} \times \fi\fi
	\ifnum\dIpowfiveII=0 \else \ifnum\dIpowfiveII=1 5 \times \else 5^{\dIpowfiveII} \times \fi\fi
	\ifnum\dIpowhardprimeII=0 \else \ifnum\dIpowhardprimeII=1 \nhardprimeII \else \nhardprimeII^{\dIpowhardprimeII} \fi\fi
}

\newcommand{\seconddiv}{
	\ifnum\dIIpowtwoII=0 \else \ifnum\dIIpowtwoII=1 2 \times \else 2^{\dIIpowtwoII} \times \fi\fi
	\ifnum\dIIpowthreeII=0 \else \ifnum\dIIpowthreeII=1 3 \times \else 3^{\dIIpowthreeII} \times \fi\fi
	\ifnum\dIIpowfiveII=0 \else \ifnum\dIIpowfiveII=1 5 \times \else 5^{\dIIpowfiveII} \times \fi\fi
	\ifnum\dIIpowhardprimeII=0 \else \ifnum\dIIpowhardprimeII=1 \mhardprimeII \else \mhardprimeII^{\dIIpowhardprimeII} \fi\fi
}

% solution


\exe{, difficulty=1}{
	\begin{enumerate}
		\item 
		Sans calculatrice, donner la décomposition en facteurs premiers de $\nII$ et de $\mII$ en décrivant votre démarche.
		\item
		En déduire la décomposition en facteurs premiers des trois nombres suivants.
			\begin{multicols}{3}
			\begin{enumerate}	
				\item $\nII \times \mII$
				\item $\nII \times \mII^2$
				\item $\nII^3$
			\end{enumerate}
			\end{multicols}
		\item
		À l'aide de la question 1 et sans calculs, est-ce que
			\begin{multicols}{2}
			\begin{enumerate}
				\item $\firstdiv$ divise \nII ?
				\item $\seconddiv$ divise \mII ?
			\end{enumerate}
			\end{multicols}
		
	\end{enumerate}
}{exe:2}{
	\begin{enumerate}
		\item 
		On utilise les règles de divisibilité par 2, 3, 5 et on réduit petit à petit les deux nombres pour trouver
		\begin{align*}
		\nII = 
		\ifnum\npowtwoII0 \else 2^{\npowtwoII} \times \fi
		\ifnum\npowthreeII0 \else 3^{\npowthreeII} \times \fi
		\ifnum\npowfiveII0 \else 5^{\npowfiveII} \times  \fi
		\nhardprimeII
		&&
		\et		
		&&
		\mII = 
		\ifnum\mpowtwoII0 \else 2^{\mpowtwoII} \times \fi
		\ifnum\mpowthreeII0 \else 3^{\mpowthreeII} \times \fi
		\ifnum\mpowfiveII0 \else 5^{\mpowfiveII} \times  \fi
		\mhardprimeII
		\end{align*}
		\item
			\begin{enumerate}	
				\item 
					\[
					\nII \times \mII = 
					\ifnum\sumIpowtwoII0 \else 2^{\sumIpowtwoII} \times \fi
					\ifnum\sumIpowthreeII0 \else 3^{\sumIpowthreeII} \times \fi
					\ifnum\sumIpowfiveII0 \else 5^{\sumIpowfiveII} \times  \fi
					\mhardprimeII \times \nhardprimeII
					\]
	
				\item 
					\[
					\nII \times \mII^2 = 
					\ifnum\sumIIpowtwoII0 \else 2^{\sumIIpowtwoII} \times \fi
					\ifnum\sumIIpowthreeII0 \else 3^{\sumIIpowthreeII} \times \fi
					\ifnum\sumIIpowfiveII0 \else 5^{\sumIIpowfiveII} \times  \fi
					\mhardprimeII^2 \times \nhardprimeII
					\]
					
				\item 
					\[
					\nII^3 =
					\ifnum\sumIIIpowtwoII0 \else 2^{\sumIIIpowtwoII} \times \fi
					\ifnum\sumIIIpowthreeII0 \else 3^{\sumIIIpowthreeII} \times \fi
					\ifnum\sumIIIpowfiveII0 \else 5^{\sumIIIpowfiveII} \times  \fi
					\nhardprimeII^3
					\]
			\end{enumerate}
		\item
		Par inspection des puissances et d'après le cours,
		\ifnum\dividenII=0
			\begin{enumerate}
				\item $\firstdiv$ ne divise pas $\nII$ ; et
				\item $\seconddiv$ divise \mII.
			\end{enumerate}
		\else
			\begin{enumerate}
				\item $\firstdiv$ divise $\nII$ ; et
				\item $\seconddiv$ ne divise pas \mII.
			\end{enumerate}
		\fi
	\end{enumerate}
}


%%%%%%%%%%%%%%%%%%%
%%%%%%%%%%%%%%%%%%%
%%%%%%%%%%%%%%%%%%%

\newpage
\fancyhead[C]{\textbf{Solutions -- {\seed} -- Arithmétique}}
\fakesection{Solution \seed}
\shipoutAnswer

\end{document}


				% ENABLE or DISABLE font change
				% use XeLaTeX if true
\newif\ifdys
				\dystrue
				\dysfalse

\newif\ifsolutions
				\solutionstrue
				\solutionsfalse

% DYSLEXIA SWITCH
\newif\ifdys
		
				% ENABLE or DISABLE font change
				% use XeLaTeX if true
				\dystrue
				\dysfalse


\ifdys

\documentclass[a4paper, 14pt]{extarticle}
\usepackage{amsmath,amsfonts,amsthm,amssymb,mathtools}

\tracinglostchars=3 % Report an error if a font does not have a symbol.
\usepackage{fontspec}
\usepackage{unicode-math}
\defaultfontfeatures{ Ligatures=TeX,
                      Scale=MatchUppercase }

\setmainfont{OpenDyslexic}[Scale=1.0]
\setmathfont{Fira Math} % Or maybe try KPMath-Sans?
\setmathfont{OpenDyslexic Italic}[range=it/{Latin,latin}]
\setmathfont{OpenDyslexic}[range=up/{Latin,latin,num}]

\else

\documentclass[a4paper, 12pt]{extarticle}

\usepackage[utf8x]{inputenc}
%fonts
\usepackage{amsmath,amsfonts,amsthm,amssymb,mathtools}
% comment below to default to computer modern
\usepackage{libertinus,libertinust1math}

\fi


\usepackage[french]{babel}
\usepackage[
a4paper,
margin=2cm,
nomarginpar,% We don't want any margin paragraphs
]{geometry}
\usepackage{icomma}

\usepackage{fancyhdr}
\usepackage{array}
\usepackage{hyperref}

\usepackage{multicol, enumerate}
\newcolumntype{P}[1]{>{\centering\arraybackslash}p{#1}}


\usepackage{stackengine}
\newcommand\xrowht[2][0]{\addstackgap[.5\dimexpr#2\relax]{\vphantom{#1}}}

% theorems

\theoremstyle{plain}
\newtheorem{theorem}{Th\'eor\`eme}
\newtheorem*{sol}{Solution}
\theoremstyle{definition}
\newtheorem{ex}{Exercice}
\newtheorem*{rpl}{Rappel}
\newtheorem{enigme}{Énigme}

% corps
\usepackage{calrsfs}
\newcommand{\C}{\mathcal{C}}
\newcommand{\R}{\mathbb{R}}
\newcommand{\Rnn}{\mathbb{R}^{2n}}
\newcommand{\Z}{\mathbb{Z}}
\newcommand{\N}{\mathbb{N}}
\newcommand{\Q}{\mathbb{Q}}

% variance
\newcommand{\Var}[1]{\text{Var}(#1)}

% domain
\newcommand{\D}{\mathcal{D}}


% date
\usepackage{advdate}
\AdvanceDate[0]


% plots
\usepackage{pgfplots}

% table line break
\usepackage{makecell}
%tablestuff
\def\arraystretch{2}
\setlength\tabcolsep{15pt}

%subfigures
\usepackage{subcaption}

\definecolor{myg}{RGB}{56, 140, 70}
\definecolor{myb}{RGB}{45, 111, 177}
\definecolor{myr}{RGB}{199, 68, 64}

% fake sections with no title to move around the merged pdf
\newcommand{\fakesection}[1]{%
  \par\refstepcounter{section}% Increase section counter
  \sectionmark{#1}% Add section mark (header)
  \addcontentsline{toc}{section}{\protect\numberline{\thesection}#1}% Add section to ToC
  % Add more content here, if needed.
}


% SOLUTION SWITCH
\newif\ifsolutions
				\solutionstrue
				%\solutionsfalse

\ifsolutions
	\newcommand{\exe}[2]{
		\begin{ex} #1  \end{ex}
		\begin{sol} #2 \end{sol}
	}
\else
	\newcommand{\exe}[2]{
		\begin{ex} #1  \end{ex}
	}
	
\fi


% tableaux var, signe
\usepackage{tkz-tab}


%pinfty minfty
\newcommand{\pinfty}{{+}\infty}
\newcommand{\minfty}{{-}\infty}

\begin{document}


\AdvanceDate[0]

\begin{document}
\pagestyle{fancy}
\fancyhead[L]{Seconde 13}
\fancyhead[C]{\textbf{Évaluation Cassandra -- Fonctions \ifsolutions -- Solutions \fi}}
\fancyhead[R]{\today}

	\begin{theorem}[Propriété fondamentale]\label{thm:1}
		Soit $f : \D \rightarrow \R$ une fonction réelle sur un domaine $\D$ et $(x;y)$ un point du plan avec $x\in\D$.
		Alors
			\begin{align*}
				(x ; y) \in \C_f && \iff && \underline{\qquad} = \underline{\qquad\qquad}.
			\end{align*}
	\end{theorem}
	
	\exe{[2pts]
		Compléter le théorème \ref{thm:1} vu cours.
	}{}
	
	
	\exe{[4pts]
		Considérons la fonction $f: \R \rightarrow\R$ donnée algébriquement par
			\[ f(x) = x^2. \]
		Pour chaque point suivant, déterminer s'il appartient à $\C_f$ ou non.
		
		\begin{multicols}{2}
		\begin{enumerate}[label=\roman*)]
			\item $\left(0 ; 0\right)$
			\item $\left(4; 8\right)$
			\item $\left(-\dfrac45 ; \dfrac{16}{25}\right)$
			\item $\left(-3 ;-9\right)$
		\end{enumerate}
		\end{multicols}
	
	}
	
	\exe{[3pts]
		Considérons $f:[-2; 4]\rightarrow\R$ donnée algébriquement par
			\[ f(x) = \dfrac74. \]
		\begin{enumerate}
			\item Donner trois points différents appartenant à $\C_f$.
			\item Esquisser la courbe de $f$.
			\item Que dire de $f$ ?
		\end{enumerate}
	}{}
	

	\exe{[4pts]
		\begin{multicols}{2}
		Considérons deux fonctions $f,g: \left]{-\dfrac85} ; {\dfrac85}\right[ \rightarrow \R$ dont les courbes représentatives sont données ci-contre.
		
		Donner approximativement l'\textbf{ensemble} des nombres $x$ du domaine vérifiant les (in)équations suivantes.
		\begin{enumerate}
			\itemsep1em 
			\item $g(x) = h(x)$
			\item $g(x) \leq h(x)$
			\item $h(x) \leq g(x)$
		\end{enumerate}
		\vfill
		
			\begin{center}
			\begin{tikzpicture}[>=stealth, scale=1]
				\begin{axis}[xmin = -1.6, xmax=1.6, ymin=-4, ymax=8.1, axis x line=middle, axis y line=middle, axis line style=->, grid=both]
					\addplot[no marks, blue, -, thick] expression[domain=-2:2, samples=100]{-((x+1)^3 /3 - 2*(x+1) +3)+5}
					node[pos=.3, above]{$\mathcal{C}_g$};
					\addplot[no marks, red, -, thick] expression[domain=-2:1.2, samples=100]{-(-4*(x+.5)^2 + 7)+5}
					node[pos=.4, below left]{$\mathcal{C}_h$};
				\end{axis}
			\end{tikzpicture}
			\end{center}
		\end{multicols}
	}{}

	\newpage

	\exe{[4pts]
		Considérons la représentation graphique suivante d'une fonction $f$ définie sur $\D = \left]{-9,8} ; -\dfrac52\right[$.
		
			\begin{center}
			\begin{tikzpicture}[>=stealth]
				\begin{axis}[xmin = -9.8, xmax=-2.5, ymin=-4.3, ymax=3.5, axis x line=middle, axis y line=middle, axis line style=->, grid=both, ytick={-4,-3,...,2,3}, xtick={-9,-8,...,-4,-3},
    every y tick label/.style={
        anchor=near yticklabel opposite,
        xshift=0.2em,
    }]
					\addplot[no marks, blue, -, thick] expression[domain=-10:-8/1.2, samples=50]{5*((x*1.2)+10)*((x*1.2)^3 - 300*(x*1.2) - 1920)/80};
					\addplot[no marks, blue, -, thick] expression[domain=-8/1.2:-6/1.2, samples=2]{5*((x*1.2)/2 + 3.2)};
					\addplot[no marks, blue, -, thick] expression[domain=-6/1.2:-5/1.2, samples=50]{-5*(((x*1.2)+5.5)^2 + 0.5^2)+3.5};
					\addplot[no marks, blue, -, thick] expression[domain=-5/1.2:-2.5/1.2, samples=2]{5*0.2}
					node[pos=.3, above]{$\mathcal{C}_f$};
				\end{axis}
			\end{tikzpicture}
			\end{center}
		\begin{enumerate}
			\item Donner approximativement les images de $-4$ et de $-7,5$ par $f$.
			\item Énumérer approximativement les antécédents de $-2$ et de $3$ par $f$.
			\item Donner approximativement un réel qui admet exactement un seul antécédent par $f$.
			\item Donner approximativement un réel qui admet une infinité d'antécédents par $f$.
		\end{enumerate}
	}{}
	

	\exe{[3pts]
		Esquisser les courbes des fonctions $F,G:[3;6]\rightarrow\R$ données algébriquement par
			\begin{align*}
				F(x) = \dfrac{12}x + 2, && G(x) = 14-2x.
			\end{align*}
		En déduire approximativement l'ensemble des $x$ de $[3;6]$ vérifiant
			\[ F(x) \leq G(x). \]
	}{}

\section*{Bonus}

	\exe{[2pts]
		Soit $f(x) = ax + b$, où $a,b\in\R$ sont deux paramètres réels.
		
		Déterminer $a$ et $b$ sachant que $\C_f$ contient les points $(0;3)$ et $(3;-1)$.
	}{}

%\exe{
%	Considérons $A(x_A ; y_A)$ et $B(x_B;y_B)$ deux points différents de $O(0;0)$, l'origine du plan.
%	Le théorème de Pythagore énonce que le triangle $OAB$ est rectangle en $O$ si et seulement si
%		\[ OA^2 + OB^2 = AB^2. \]
%	Démontrer que cette condition est équivalente à
%		\[ x_A \cdot x_B + y_A \cdot y_B = 0. \]
%}{}

\end{document}
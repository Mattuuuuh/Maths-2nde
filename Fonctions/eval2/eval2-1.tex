% DYSLEXIA SWITCH
\newif\ifdys
		
				% ENABLE or DISABLE font change
				% use XeLaTeX if true
				\dystrue
				\dysfalse


\ifdys

\documentclass[a4paper, 14pt]{extarticle}
\usepackage{amsmath,amsfonts,amsthm,amssymb,mathtools}

\tracinglostchars=3 % Report an error if a font does not have a symbol.
\usepackage{fontspec}
\usepackage{unicode-math}
\defaultfontfeatures{ Ligatures=TeX,
                      Scale=MatchUppercase }

\setmainfont{OpenDyslexic}[Scale=1.0]
\setmathfont{Fira Math} % Or maybe try KPMath-Sans?
\setmathfont{OpenDyslexic Italic}[range=it/{Latin,latin}]
\setmathfont{OpenDyslexic}[range=up/{Latin,latin,num}]

\else

\documentclass[a4paper, 12pt]{extarticle}
\usepackage{amsmath,amsfonts,amsthm,amssymb,mathtools}

\fi


\usepackage[french]{babel}
\usepackage[
a4paper,
margin=2cm,
nomarginpar,% We don't want any margin paragraphs
]{geometry}
\usepackage{fancyhdr}
\usepackage{array}
\usepackage{amsmath,amsfonts,amsthm,amssymb,mathtools,}
\newcolumntype{P}[1]{>{\centering\arraybackslash}p{#1}}

\usepackage{enumitem, multicol}

\usepackage{stackengine}
\newcommand\xrowht[2][0]{\addstackgap[.5\dimexpr#2\relax]{\vphantom{#1}}}

% theorems

\theoremstyle{plain}
\newtheorem{theorem}{Th\'eor\`eme}
\newtheorem*{sol}{Solution}
\theoremstyle{definition}
\newtheorem{ex}{Exercice}
\newtheorem{definition}{Définition}


% corps
\newcommand{\C}{\mathcal{C}}
\newcommand{\R}{\mathbb{R}}
\newcommand{\Rnn}{\mathbb{R}^{2n}}
\newcommand{\Z}{\mathbb{Z}}
\newcommand{\N}{\mathbb{N}}
\newcommand{\Q}{\mathbb{Q}}

% domain
\newcommand{\D}{\mathcal{D}}



% plots
\usepackage{pgfplots}

% for calligraphic C
\usepackage{calrsfs}

% euro
\usepackage{lmodern,textcomp}

% ensembles tq. 
\newcommand{\xRtq}[1]{
	$\left\{ x \in \R \text{ tq. } #1 \right\}$
}


% ensembles tq. 
\newcommand{\vabs}[1]{
	\left| #1 \right|
}

%virgules
\usepackage{icomma}


%pinfty minfty
\newcommand{\pinfty}{{+}\infty}
\newcommand{\minfty}{{-}\infty}

% SOLUTION SWITCH
\newif\ifsolutions
				\solutionstrue
				\solutionsfalse

\ifsolutions
	\newcommand{\exe}[2]{
		\begin{ex} #1  \end{ex}
		\begin{sol} #2 \end{sol}
	}
\else
	\newcommand{\exe}[2]{
		\begin{ex} #1  \end{ex}
	}
	
\fi



% date
\usepackage{advdate}
\AdvanceDate[0]


\begin{document}
\pagestyle{fancy}
\fancyhead[L]{Seconde 13}
\fancyhead[C]{\textbf{Évaluation blanche -- Fonctions \ifsolutions -- Solutions \fi}}
\fancyhead[R]{\today}

	\begin{theorem}[Propriété fondamentale]\label{thm:1}
		Soit $f : \D \rightarrow \R$ une fonction réelle sur un domaine $\D$ et $(x;y)$ un point du plan avec $x\in\D$.
		Alors
			\begin{align*}
				(x ; y) \in \C_f && \iff && \underline{\qquad} = \underline{\qquad\qquad}.
			\end{align*}
	\end{theorem}
	
	\exe{
		Compléter le théorème \ref{thm:1} vu en cours.
	}{}
	
	
	\exe{
		Considérons la fonction $f: \R \rightarrow\R$ donnée algébriquement par
			\[ f(x) = \dfrac{2}{3x}. \]
		Pour chaque point suivant, déterminer s'il appartient à $\C_f$ ou non.
		
		\begin{multicols}{2}
		\begin{enumerate}[label=\roman*)]
			\item $\left(2; 1\right)$
			\item $\left(-1 ;-\dfrac23\right)$
			\item $\left(\dfrac12 ; \dfrac13\right)$
			\item $\left(-\dfrac23 ; -1\right)$
		\end{enumerate}
		\end{multicols}
	
	}
	
	\exe{
		Considérons $f:[-2; 4]\rightarrow\R$ donnée algébriquement par
			\[ f(x) = 2-x. \]
		\begin{enumerate}
			\item Donner trois points différents appartenant à $\C_f$.
			\item Esquisser la courbe de $f$.
		\end{enumerate}
	}{}
	
	\exe{
		Esquisser les courbes des fonctions $F,G:[-2;2]\rightarrow\R$ données algébriquement par
			\begin{align*}
				F(x) = x^2, && G(x) = -\dfrac{x}2  + 1.
			\end{align*}
		En déduire approximativement l'ensemble des $x$ de $[-2;2]$ vérifiant
			\[ F(x) = G(x). \]
	}{}

	\exe{
		\begin{multicols}{2}
		Considérons deux fonctions $f,g: \left]{-1} ; 3\right[ \rightarrow \R$ dont les courbes représentatives sont données ci-contre.
		
		Donner approximativement l'\textbf{ensemble} des nombres $x$ du domaine vérifiant les (in)équations suivantes.
		\begin{enumerate}
			\itemsep1em 
			\item $g(x) = h(x)$
			\item $g(x) \leq h(x)$
			\item $h(x) \leq g(x)$
		\end{enumerate}
		\vfill
		
			\begin{center}
			\begin{tikzpicture}[>=stealth, scale=1]
				\begin{axis}[xmin = -1, xmax=3, ymin=-4, ymax=8.1, axis x line=middle, axis y line=middle, axis line style=->, grid=both]
					\addplot[no marks, blue, -] expression[domain=-1:3, samples=100]{x^3 -2*x}
					node[pos=.15, below=2pt]{$\mathcal{C}_g$};
					\addplot[no marks, red, -] expression[domain=-1:3, samples=100]{x^2 -1}
					node[pos=.6, right=6pt]{$\mathcal{C}_h$};
				\end{axis}
			\end{tikzpicture}
			\end{center}
		\end{multicols}
	}{}

	\newpage

	\exe{
		Considérons la représentation graphique suivante d'une fonction $f$ définie sur $\D = \left]2,8; 5,6\right[$.
		
			\begin{center}
			\begin{tikzpicture}[>=stealth]
				\begin{axis}[xmin = 2.8, xmax=5.6, ymin=-3, ymax=5, axis x line=middle, axis y line=middle, axis line style=->, grid=both]
					\addplot[no marks, blue, -] expression[domain=2.8:6, samples=100]{2*((x-4)^4 - (x-4)^5 + 2*(x-4))}
					node[pos=.3, above]{$\mathcal{C}_f$};
				\end{axis}
			\end{tikzpicture}
			\end{center}
		\begin{enumerate}
			\item Donner approximativement les images de $3$ et de $4$ par $f$.
			\item Énumérer approximativement les antécédents de $2$ et de $0$ par $f$.
			\item Donner approximativement un réel qui admet exactement un seul antécédent par $f$.
		\end{enumerate}
	}{}

\end{document}
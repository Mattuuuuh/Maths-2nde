% DYSLEXIA SWITCH
\newif\ifdys
		
				% ENABLE or DISABLE font change
				% use XeLaTeX if true
				\dystrue
				%\dysfalse


\ifdys

\documentclass[a4paper, 14pt]{extarticle}
\usepackage{amsmath,amsfonts,amsthm,amssymb,mathtools}

\tracinglostchars=3 % Report an error if a font does not have a symbol.
\usepackage{fontspec}
\usepackage{unicode-math}
\defaultfontfeatures{ Ligatures=TeX,
                      Scale=MatchUppercase }

\setmainfont{OpenDyslexic}[Scale=1.0]
\setmathfont{Fira Math} % Or maybe try KPMath-Sans?
\setmathfont{OpenDyslexic Italic}[range=it/{Latin,latin}]
\setmathfont{OpenDyslexic}[range=up/{Latin,latin,num}]

\else

\documentclass[a4paper, 12pt]{extarticle}

\usepackage[utf8x]{inputenc}
\usepackage{lmodern,textcomp}
\usepackage{amsmath,amsfonts,amsthm,amssymb,mathtools}

\fi


\usepackage[french]{babel}
\usepackage[
a4paper,
margin=2cm,
nomarginpar,% We don't want any margin paragraphs
]{geometry}
\usepackage{icomma}

\usepackage{fancyhdr}
\usepackage{array}

\usepackage{multicol, enumerate}
\newcolumntype{P}[1]{>{\centering\arraybackslash}p{#1}}


\usepackage{stackengine}
\newcommand\xrowht[2][0]{\addstackgap[.5\dimexpr#2\relax]{\vphantom{#1}}}

% theorems

\theoremstyle{plain}
\newtheorem{theorem}{Th\'eor\`eme}
\newtheorem*{sol}{Solution}
\theoremstyle{definition}
\newtheorem{ex}{Exercice}

% corps
\newcommand{\C}{\mathcal{C}}
\newcommand{\R}{\mathbb{R}}
\newcommand{\Rnn}{\mathbb{R}^{2n}}
\newcommand{\Z}{\mathbb{Z}}
\newcommand{\N}{\mathbb{N}}
\newcommand{\Q}{\mathbb{Q}}

% domain
\newcommand{\D}{\mathcal{D}}


% date
\usepackage{advdate}
\AdvanceDate[0]


% plots
\usepackage{pgfplots}

% for calligraphic C
\usepackage{calrsfs}

% SOLUTION SWITCH
\newif\ifsolutions
				\solutionstrue
				\solutionsfalse

\ifsolutions
	\newcommand{\exe}[2]{
		\begin{ex} #1  \end{ex}
		\begin{sol} #2 \end{sol}
	}
\else
	\newcommand{\exe}[2]{
		\begin{ex} #1  \end{ex}
	}
	
\fi

\begin{document}
\pagestyle{fancy}
\fancyhead[L]{Seconde 13}
\fancyhead[C]{\textbf{Fonctions 2 \ifsolutions -- Solutions  \fi}}
\fancyhead[R]{\today}


	\exe{
		Considérons la fonction $f: \left]{-}\dfrac72 ; \dfrac{11}2 \right[ \rightarrow\R$ donnée algébriquement par
			\[ f(x) = \dfrac17-x. \]
		Pour chaque point suivant, déterminer s'il appartient à $\C_f$ ou non.
		
		\begin{multicols}{2}
		\begin{enumerate}[i)]
			\item $\left(0; \dfrac17\right)$
			\item $\left(\dfrac17 ; 0\right)$
			\item $\left(\dfrac27 ; \dfrac37\right)$
			\item $\left(-\dfrac{13}7 ; 2\right)$
			\item $\left(\dfrac67 ; 1\right)$
			\item $\left(\dfrac27 ; -\dfrac17\right)$
		\end{enumerate}
		\end{multicols}
	
	}{}

	
	\exe{
		Considérons deux fonctions $f, g: ]{-}3 ; 3[ \rightarrow\R$ données algébriquement par
			\begin{align*}
				f(x) = x^2 - 2\cdot x && g(x) = (x-1)^2
			\end{align*}
		
		\begin{enumerate}
			\item Esquisser les représentations graphiques de $f$ et de $g$ dans un même repère.
			\item Démontrer que $g(x) - 1^2 = f(x)$ pour tout $x$ du domaine.
			\item En déduire que $(x-1)^2 = x^2 - 2\cdot x + 1$ pour tout $x$ du domaine.
		\end{enumerate}
	}{}
	
	\exe{
		Esquisser la courbe de la fonction $f:[-2; 4]\rightarrow\R$ donnée algébriquement par
			\[ f(x) = 3. \]
		Que dire de $f$ et de $\C_f$ ?
	}{}
	
	\exe{
		Esquisser la courbe de la fonction $f:[-5;3]\rightarrow\R$ donnée algébriquement par
			\[ f(x) = 1-x. \]
		Que dire $\C_f$ ?
	}{}
	
	\exe{
		Esquisser la courbe de la fonction $f:[3;10]\rightarrow\R$ donnée algébriquement par
			\[ f(x) = \dfrac3x + 1. \]
	}{}
	
	\exe{
		Considérons la représentation graphique suivante d'une fonction $f$ définie sur $\D = ]{-}3,4 ; 2,3[$.
		
			\begin{center}
			\begin{tikzpicture}[>=stealth]
				\begin{axis}[xmin = -3.4, xmax=2.3, ymin=-5.1, ymax=5.1, axis x line=middle, axis y line=middle, axis line style=->, grid=both]
					\addplot[no marks, blue, -] expression[domain=-5:3, samples=50]{x^3 /3 - 2*x +3}
					node[pos=.3, right]{$\mathcal{C}_f$};
				\end{axis}
			\end{tikzpicture}
			\end{center}
		\begin{enumerate}
			\item Donner approximativement les images de $-1,5$ et de $-\dfrac{20}7$ par $f$.
			\item Énumérer approximativement les antécédents de $-2$ et de $2$ par $f$.
			\item Donner approximativement un réel qui admet exactement deux antécédents par $f$.
			\item Si $f$ était définie sur $\R$ tout entier, serait-il toujours possible de connaître l'image de $-2$ ? Et tous les antécédents de $-2$ ?
		\end{enumerate}
		Supposons désormais que $f(x) = 3-2\cdot x +\dfrac13 \cdot x^3$ pour tout $x\in\D$ du domaine.
		\begin{enumerate}
			\item[5.] Vérifier à la calculatrice les réponses aux deux premières questions.
			\item[6.] Montrer sans calculatrice que l'image par $f$ de $-3$ est $0$ et que l'image par $f$ de $0$ est $3$.
		\end{enumerate}
	}{}
	
	
	\exe{
		Un fonction $f$ admet une représentation graphique suivante.
			\begin{center}
			\begin{tikzpicture}[>=stealth]
				\begin{axis}[xmin = -3.1, xmax=1.1, ymin=-3.1, ymax=5.1, axis x line=middle, axis y line=middle, axis line style=->, grid=both]
					\addplot[no marks, blue, -] expression[domain=-3:2, samples=2]{-2/3 - 2*x}
					node[pos=.3, right]{$\mathcal{C}_f$};
				\end{axis}
			
			\end{tikzpicture}
			\end{center}
		Parmis les expressions algébriques suivantes, trouver celle qui correspond à $f(x)$.
			\begin{multicols}{2}
			\begin{enumerate}[i)]
				\item $1-x$
				\item $\dfrac{-1-x}3$
				\item $\left(x+\dfrac13\right)^2$
				\item $-2\cdot x - \dfrac23$
			\end{enumerate}
			\end{multicols}
	}{}
	
	
	\subsection*{Exercices supplémentaires}
	
	\exe{
		Donner une fonction réelle et un domaine telle qu'une de ses images admet
			\begin{enumerate}
				\item Exactement un antécédent
				\item Exactement deux antécédents
				\item Exactement trois antécédents
				\item Une infinité d'antécédents
			\end{enumerate}	
	}{}
	
	\exe{
		Donner graphiquement une fonction sur $\R$ non constante telle que toutes les images de $f$ admettent un nombre infini d'antécédents.
	}{}
	
	
	\exe{
		Comparer les représentations graphiques des fonctions suivantes données algébriquement.
			\begin{align*}
				f(x) = x^2 && g(x) = x^2 - 3 && h(x) = (x+4)^2.
			\end{align*}
	}{}

\end{document}

% DYSLEXIA SWITCH
\newif\ifdys
		
				% ENABLE or DISABLE font change
				% use XeLaTeX if true
				\dystrue
				\dysfalse


\ifdys

\documentclass[a4paper, 14pt]{extarticle}
\usepackage{amsmath,amsfonts,amsthm,amssymb,mathtools}

\tracinglostchars=3 % Report an error if a font does not have a symbol.
\usepackage{fontspec}
\usepackage{unicode-math}
\defaultfontfeatures{ Ligatures=TeX,
                      Scale=MatchUppercase }

\setmainfont{OpenDyslexic}[Scale=1.0]
\setmathfont{Fira Math} % Or maybe try KPMath-Sans?
\setmathfont{OpenDyslexic Italic}[range=it/{Latin,latin}]
\setmathfont{OpenDyslexic}[range=up/{Latin,latin,num}]

\else

\documentclass[a4paper, 12pt]{extarticle}

\usepackage[utf8x]{inputenc}
\usepackage{lmodern,textcomp}
\usepackage{amsmath,amsfonts,amsthm,amssymb,mathtools}

\fi


\usepackage[french]{babel}
\usepackage[
a4paper,
margin=2cm,
nomarginpar,% We don't want any margin paragraphs
]{geometry}
\usepackage{icomma}

\usepackage{fancyhdr}
\usepackage{array}

\usepackage{multicol, enumerate}
\newcolumntype{P}[1]{>{\centering\arraybackslash}p{#1}}


\usepackage{stackengine}
\newcommand\xrowht[2][0]{\addstackgap[.5\dimexpr#2\relax]{\vphantom{#1}}}

% theorems

\theoremstyle{plain}
\newtheorem{theorem}{Th\'eor\`eme}
\newtheorem*{theorem*}{Th\'eor\`eme}
\newtheorem*{sol}{Solution}
\theoremstyle{definition}
\newtheorem{ex}{Exercice}

% corps
\newcommand{\C}{\mathcal{C}}
\newcommand{\R}{\mathbb{R}}
\newcommand{\Rnn}{\mathbb{R}^{2n}}
\newcommand{\Z}{\mathbb{Z}}
\newcommand{\N}{\mathbb{N}}
\newcommand{\Q}{\mathbb{Q}}

% domain
\newcommand{\D}{\mathcal{D}}


% date
\usepackage{advdate}
\AdvanceDate[0]


% plots
\usepackage{pgfplots}

% for calligraphic C
\usepackage{calrsfs}

% SOLUTION SWITCH
\newif\ifsolutions
				\solutionstrue
				%\solutionsfalse

\ifsolutions
	\newcommand{\exe}[2]{
		\begin{ex} #1  \end{ex}
		\begin{sol} #2 \end{sol}
	}
\else
	\newcommand{\exe}[2]{
		\begin{ex} #1  \end{ex}
	}
	
\fi

\begin{document}
\pagestyle{fancy}
\fancyhead[L]{Seconde 13}
\fancyhead[C]{\textbf{Exercices 33 et 61 pages 227-232 \ifsolutions -- Solutions  \fi}}
\fancyhead[R]{\today}

\exe{[33 page 227]
	Dans le plan rapporté à un repère orthonormé, représenter les fonctions définies sur $\R$ par :
		\begin{align*}
			f(x) = x^2 + 1 && \text{ et } && g(x)=3x+1.
		\end{align*}
	\begin{enumerate}[a)]
		\item Déterminer graphiquement l'ensemble des solutions de l'équation $f(x) = g(x)$.
		\item Déterminer graphiquement l'ensemble des solutions de l'inéquation $f(x) \leq g(x)$.
	\end{enumerate}
}{}

\begin{center}
\begin{tikzpicture}[>=stealth, scale=2]
	\begin{axis}[xmin = -2.4, xmax=4.9, ymin=-1, ymax=15.1, axis x line=middle, axis y line=middle, axis line style=->, grid=both]
		\addplot[no marks, blue, -, thick] expression[domain=-2:5, samples=100]{x^2 + 1}
		node[pos=.1, above=5pt]{$\mathcal{C}_f$};
		\addplot[no marks, red, -, thick] expression[domain=-2:5, samples=100]{3*x+1}
		node[pos=.8, right=5pt]{$\mathcal{C}_g$};
	\end{axis}
\end{tikzpicture}
\end{center}

\newpage 
\exe{[61 page 232]
	\begin{enumerate}
		\item 
		Soit $f$ la fonction définie sur $\R$ par $f(x) = |x|$.
		\begin{enumerate}[a)]
			\item
			Dans un repère orthonormé du plan, tracer la représentation graphique de $\C_f$ de $f$.
			\item
			Graphiquement, donner les solutions de l'équation $f(x) = 2$.
			\item
			Pour un réel $k$ donné, quelles sont les solutions, si elles existent, de l'équation $f(x) = k$ ?
		\end{enumerate}
		\item 
		Soit $g$ la fonction définie sur $\R$ par $g(x) = -\dfrac12 x + 5$.
		\begin{enumerate}[a)]
			\item
			Tracer dans le repère précédent du plan la courbe représentative $\C_g$ de la fonction $g$.
			\item
			Déterminer l'ensemble des nombres réels pour lesquels $f(x) \leq g(x)$
		\end{enumerate}
	\end{enumerate}
}{}


\begin{center}
\begin{tikzpicture}[>=stealth, scale=2]
	\begin{axis}[xmin = -11, xmax=6, ymin=-1, ymax=12.1, axis x line=middle, axis y line=middle, axis line style=->, grid=both]
		\addplot[no marks, blue, -, thick] expression[domain=-11:6, samples=100]{sqrt(x^2)}
		node[pos=.9, above=5pt]{$\mathcal{C}_f$};
		\addplot[no marks, red, -, thick] expression[domain=-11:6, samples=2]{-x/2 + 5}
		node[pos=.3, above]{$\mathcal{C}_g$};
	\end{axis}
\end{tikzpicture}
\end{center}


\end{document}

%!TEX encoding = UTF8
%!TEX root =notes.tex


%%%%%%%%%%%%%%%%%%%%%%%%%%%%%%%%%
% PACKAGE IMPORTS
%%%%%%%%%%%%%%%%%%%%%%%%%%%%%%%%%


\usepackage[french]{babel}

\usepackage[tmargin=2cm,rmargin=1in,lmargin=1in,margin=0.85in,bmargin=2cm,footskip=.2in]{geometry}
\usepackage{amsmath,amsfonts,amsthm,amssymb,mathtools}
\usepackage[varbb]{newpxmath}
\usepackage{xfrac}
\usepackage[makeroom]{cancel}
\usepackage{mathtools}
\usepackage{bookmark}
\usepackage{enumitem}
\usepackage{hyperref,theoremref}
\hypersetup{
	pdftitle={Assignment},
	colorlinks=true, linkcolor=doc!90,
	bookmarksnumbered=true,
	bookmarksopen=true
}
\usepackage[most,many,breakable]{tcolorbox}
\usepackage{xcolor}
\usepackage{varwidth}
\usepackage{varwidth}
\usepackage{etoolbox}
%\usepackage{authblk}
\usepackage{nameref}
\usepackage{multicol,array}
\usepackage{tikz-cd}
\usepackage[ruled,vlined,linesnumbered]{algorithm2e}
\usepackage{comment} % enables the use of multi-line comments (\ifx \fi) 
\usepackage{import}
\usepackage{xifthen}
\usepackage{pdfpages}
\usepackage{transparent}


\newcommand\mycommfont[1]{\footnotesize\ttfamily\textcolor{blue}{#1}}
\SetCommentSty{mycommfont}
\newcommand{\incfig}[1]{%
    \def\svgwidth{\columnwidth}
    \import{./figures/}{#1.pdf_tex}
}

\usepackage{tikzsymbols}
%\renewcommand\qedsymbol{$\Laughey$}


%\usepackage{import}
%\usepackage{xifthen}
%\usepackage{pdfpages}
%\usepackage{transparent}


%%%%%%%%%%%%%%%%%%%%%%%%%%%%%%
% SELF MADE COLORS
%%%%%%%%%%%%%%%%%%%%%%%%%%%%%%



\definecolor{myg}{RGB}{56, 140, 70}
\definecolor{myb}{RGB}{45, 111, 177}
\definecolor{myr}{RGB}{199, 68, 64}
\definecolor{mytheorembg}{HTML}{F2F2F9}
\definecolor{mytheoremfr}{HTML}{00007B}
\definecolor{mylenmabg}{HTML}{FFFAF8}
\definecolor{mylenmafr}{HTML}{983b0f}
\definecolor{mypropbg}{HTML}{f2fbfc}
\definecolor{mypropfr}{HTML}{191971}
\definecolor{myexamplebg}{HTML}{F2FBF8}
\definecolor{myexamplefr}{HTML}{88D6D1}
\definecolor{myexampleti}{HTML}{2A7F7F}
\definecolor{mydefinitbg}{HTML}{E5E5FF}
\definecolor{mydefinitfr}{HTML}{3F3FA3}
\definecolor{notesgreen}{RGB}{0,162,0}
\definecolor{myp}{RGB}{197, 92, 212}
\definecolor{mygr}{HTML}{2C3338}
\definecolor{myred}{RGB}{127,0,0}
\definecolor{myyellow}{RGB}{169,121,69}
\definecolor{myexercisebg}{HTML}{F2FBF8}
\definecolor{myexercisefg}{HTML}{88D6D1}


%%%%%%%%%%%%%%%%%%%%%%%%%%%%
% TCOLORBOX SETUPS
%%%%%%%%%%%%%%%%%%%%%%%%%%%%

\setlength{\parindent}{1cm}
%================================
% THEOREM BOX
%================================

\tcbuselibrary{theorems,skins,hooks}
\newtcbtheorem[number within=chapter]{Theorem}{Théorème}
{%
	enhanced,
	breakable,
	colback = mytheorembg,
	frame hidden,
	boxrule = 0sp,
	borderline west = {2pt}{0pt}{mytheoremfr},
	sharp corners,
	detach title,
	before upper = \tcbtitle\par\smallskip,
	coltitle = mytheoremfr,
	fonttitle = \bfseries\sffamily,
	description font = \mdseries,
	separator sign none,
	segmentation style={solid, mytheoremfr},
}
{th}


\tcbuselibrary{theorems,skins,hooks}
\newtcolorbox{Theoremcon}
{%
	enhanced
	,breakable
	,colback = mytheorembg
	,frame hidden
	,boxrule = 0sp
	,borderline west = {2pt}{0pt}{mytheoremfr}
	,sharp corners
	,description font = \mdseries
	,separator sign none
}

%================================
% Corollery
%================================
\tcbuselibrary{theorems,skins,hooks}
\newtcbtheorem[use counter=tcb@cnt@Theorem]{Corollary}{Corollaire}
{%
	enhanced
	,breakable
	,colback = myp!10
	,frame hidden
	,boxrule = 0sp
	,borderline west = {2pt}{0pt}{myp!85!black}
	,sharp corners
	,detach title
	,before upper = \tcbtitle\par\smallskip
	,coltitle = myp!85!black
	,fonttitle = \bfseries\sffamily
	,description font = \mdseries
	,separator sign none
	,segmentation style={solid, myp!85!black}
}
{th}

%================================
% LENMA
%================================

\tcbuselibrary{theorems,skins,hooks}
\newtcbtheorem[use counter=tcb@cnt@Theorem]{Lemma}{Lemme}
{%
	enhanced,
	breakable,
	colback = mylenmabg,
	frame hidden,
	boxrule = 0sp,
	borderline west = {2pt}{0pt}{mylenmafr},
	sharp corners,
	detach title,
	before upper = \tcbtitle\par\smallskip,
	coltitle = mylenmafr,
	fonttitle = \bfseries\sffamily,
	description font = \mdseries,
	separator sign none,
	segmentation style={solid, mylenmafr},
}
{th}


%================================
% PROPOSITION
%================================

\tcbuselibrary{theorems,skins,hooks}
\newtcbtheorem[use counter=tcb@cnt@Theorem]{Prop}{Proposition}
{%
	enhanced,
	breakable,
	colback = mypropbg,
	frame hidden,
	boxrule = 0sp,
	borderline west = {2pt}{0pt}{mypropfr},
	sharp corners,
	detach title,
	before upper = \tcbtitle\par\smallskip,
	coltitle = mypropfr,
	fonttitle = \bfseries\sffamily,
	description font = \mdseries,
	separator sign none,
	segmentation style={solid, mypropfr},
}
{th}


%================================
% CLAIM
%================================

\tcbuselibrary{theorems,skins,hooks}
\newtcbtheorem[use counter=tcb@cnt@Theorem]{claim}{Claim}
{%
	enhanced
	,breakable
	,colback = myg!10
	,frame hidden
	,boxrule = 0sp
	,borderline west = {2pt}{0pt}{myg}
	,sharp corners
	,detach title
	,before upper = \tcbtitle\par\smallskip
	,coltitle = myg!85!black
	,fonttitle = \bfseries\sffamily
	,description font = \mdseries
	,separator sign none
	,segmentation style={solid, myg!85!black}
}
{th}



%================================
% Exercise
%================================

\tcbuselibrary{theorems,skins,hooks}
\newtcbtheorem[use counter=tcb@cnt@Theorem]{Exercise}{Exercice}
{%
	enhanced,
	breakable,
	colback = myexercisebg,
	frame hidden,
	boxrule = 0sp,
	borderline west = {2pt}{0pt}{myexercisefg},
	sharp corners,
	detach title,
	before upper = \tcbtitle\par\smallskip,
	coltitle = myexercisefg,
	fonttitle = \bfseries\sffamily,
	description font = \mdseries,
	separator sign none,
	segmentation style={solid, myexercisefg},
}
{th}

%================================
% EXAMPLE BOX
%================================

\newtcbtheorem[use counter=tcb@cnt@Theorem]{Example}{Exemple}
{%
	colback = myexamplebg
	,breakable
	,colframe = myexamplefr
	,coltitle = myexampleti
	,boxrule = 1pt
	,sharp corners
	,detach title
	,before upper=\tcbtitle\par\smallskip
	,fonttitle = \bfseries
	,description font = \mdseries
	,separator sign none
	,description delimiters parenthesis
}
{ex}

%================================
% DEFINITION BOX
%================================

\newtcbtheorem[use counter=tcb@cnt@Theorem]{Definition}{Définition}{enhanced,
	before skip=2mm,after skip=2mm, colback=red!5,colframe=red!80!black,boxrule=0.5mm,
	attach boxed title to top left={xshift=1cm,yshift*=1mm-\tcboxedtitleheight}, varwidth boxed title*=-3cm,
	boxed title style={frame code={
					\path[fill=tcbcolback]
					([yshift=-1mm,xshift=-1mm]frame.north west)
					arc[start angle=0,end angle=180,radius=1mm]
					([yshift=-1mm,xshift=1mm]frame.north east)
					arc[start angle=180,end angle=0,radius=1mm];
					\path[left color=tcbcolback!60!black,right color=tcbcolback!60!black,
						middle color=tcbcolback!80!black]
					([xshift=-2mm]frame.north west) -- ([xshift=2mm]frame.north east)
					[rounded corners=1mm]-- ([xshift=1mm,yshift=-1mm]frame.north east)
					-- (frame.south east) -- (frame.south west)
					-- ([xshift=-1mm,yshift=-1mm]frame.north west)
					[sharp corners]-- cycle;
				},interior engine=empty,
		},
	fonttitle=\bfseries,
	title={#2},#1}{def}

%================================
% Solution BOX
%================================

\makeatletter
\newtcbtheorem[use counter=tcb@cnt@Theorem]{question}{Question}{enhanced,
	breakable,
	colback=white,
	colframe=myb!80!black,
	attach boxed title to top left={yshift*=-\tcboxedtitleheight},
	fonttitle=\bfseries,
	title={#2},
	boxed title size=title,
	boxed title style={%
			sharp corners,
			rounded corners=northwest,
			colback=tcbcolframe,
			boxrule=0pt,
		},
	underlay boxed title={%
			\path[fill=tcbcolframe] (title.south west)--(title.south east)
			to[out=0, in=180] ([xshift=5mm]title.east)--
			(title.center-|frame.east)
			[rounded corners=\kvtcb@arc] |-
			(frame.north) -| cycle;
		},
	#1
}{def}
\makeatother

%================================
% SOLUTION BOX
%================================

\makeatletter
\newtcolorbox{solution}{enhanced,
	breakable,
	colback=white,
	colframe=myg!80!black,
	attach boxed title to top left={yshift*=-\tcboxedtitleheight},
	title=Solution,
	boxed title size=title,
	boxed title style={%
			sharp corners,
			rounded corners=northwest,
			colback=tcbcolframe,
			boxrule=0pt,
		},
	underlay boxed title={%
			\path[fill=tcbcolframe] (title.south west)--(title.south east)
			to[out=0, in=180] ([xshift=5mm]title.east)--
			(title.center-|frame.east)
			[rounded corners=\kvtcb@arc] |-
			(frame.north) -| cycle;
		},
}
\makeatother

%================================
% Question BOX
%================================

\makeatletter
\newtcbtheorem[use counter=tcb@cnt@Theorem]{qstion}{Question}{enhanced,
	breakable,
	colback=white,
	colframe=mygr,
	attach boxed title to top left={yshift*=-\tcboxedtitleheight},
	fonttitle=\bfseries,
	title={#2},
	boxed title size=title,
	boxed title style={%
			sharp corners,
			rounded corners=northwest,
			colback=tcbcolframe,
			boxrule=0pt,
		},
	underlay boxed title={%
			\path[fill=tcbcolframe] (title.south west)--(title.south east)
			to[out=0, in=180] ([xshift=5mm]title.east)--
			(title.center-|frame.east)
			[rounded corners=\kvtcb@arc] |-
			(frame.north) -| cycle;
		},
	#1
}{def}
\makeatother

\newtcbtheorem[number within=chapter]{wconc}{Wrong Concept}{
	breakable,
	enhanced,
	colback=white,
	colframe=myr,
	arc=0pt,
	outer arc=0pt,
	fonttitle=\bfseries\sffamily\large,
	colbacktitle=myr,
	attach boxed title to top left={},
	boxed title style={
			enhanced,
			skin=enhancedfirst jigsaw,
			arc=3pt,
			bottom=0pt,
			interior style={fill=myr}
		},
	#1
}{def}



%================================
% NOTE BOX
%================================

\usetikzlibrary{arrows,calc,shadows.blur}
\tcbuselibrary{skins}
\newtcolorbox{note}[1][]{%
	enhanced jigsaw,
	colback=gray!20!white,%
	colframe=gray!80!black,
	size=small,
	boxrule=1pt,
	title=\colorbox{white!100}{\textbf{ Remarque }},
	halign title=flush center,
	coltitle=black,
	breakable,
	drop shadow=black!50!white,
	attach boxed title to top left={xshift=1cm,yshift=-\tcboxedtitleheight/2,yshifttext=-\tcboxedtitleheight/2},
	minipage boxed title=2.6cm,
	boxed title style={%
			colback=white,
			size=fbox,
			boxrule=1pt,
			boxsep=2pt,
			underlay={%
					\coordinate (dotA) at ($(interior.west) + (-0.5pt,0)$);
					\coordinate (dotB) at ($(interior.east) + (0.5pt,0)$);
					\begin{scope}
						\clip (interior.north west) rectangle ([xshift=3ex]interior.east);
						\filldraw [white, blur shadow={shadow opacity=60, shadow yshift=-.75ex}, rounded corners=2pt] (interior.north west) rectangle (interior.south east);
					\end{scope}
					\begin{scope}[gray!80!black]
						\fill (dotA) circle (2pt);
						\fill (dotB) circle (2pt);
					\end{scope}
				},
		},
	#1,
}

%================================
% STRATÉGIE BOX
%================================

\usetikzlibrary{arrows,calc,shadows.blur}
\tcbuselibrary{skins}
\newtcolorbox{strategy}[1][]{%
	enhanced jigsaw,
	colback=myb!20!white,%
	colframe=gray!80!black,
	size=small,
	boxrule=1pt,
	title=\colorbox{white!100}{\textbf{ Stratégie }},
	halign title=flush center,
	coltitle=black,
	breakable,
	drop shadow=black!50!white,
	attach boxed title to top left={xshift=1cm,yshift=-\tcboxedtitleheight/2,yshifttext=-\tcboxedtitleheight/2},
	minipage boxed title=2.5cm,
	boxed title style={%
			colback=white,
			size=fbox,
			boxrule=1pt,
			boxsep=2pt,
			underlay={%
					\coordinate (dotA) at ($(interior.west) + (-0.5pt,0)$);
					\coordinate (dotB) at ($(interior.east) + (0.5pt,0)$);
					\begin{scope}
						\clip (interior.north west) rectangle ([xshift=3ex]interior.east);
						\filldraw [white, blur shadow={shadow opacity=60, shadow yshift=-.75ex}, rounded corners=2pt] (interior.north west) rectangle (interior.south east);
					\end{scope}
					\begin{scope}[gray!80!black]
						\fill (dotA) circle (2pt);
						\fill (dotB) circle (2pt);
					\end{scope}
				},
		},
	#1,
}

%================================
% MÉTHODE BOX
%================================

\usetikzlibrary{arrows,calc,shadows.blur}
\tcbuselibrary{skins}
\newtcolorbox{methode}[1][]{%
	enhanced jigsaw,
	colback=white,%
	colframe=gray!80!black,
	size=small,
	boxrule=1pt,
	title=\textbf{Méthode},
	halign title=flush center,
	coltitle=black,
	breakable,
	drop shadow=black!50!white,
	attach boxed title to top left={xshift=1cm,yshift=-\tcboxedtitleheight/2,yshifttext=-\tcboxedtitleheight/2},
	minipage boxed title=2.5cm,
	boxed title style={%
			colback=white,
			size=fbox,
			boxrule=1pt,
			boxsep=2pt,
			underlay={%
					\coordinate (dotA) at ($(interior.west) + (-0.5pt,0)$);
					\coordinate (dotB) at ($(interior.east) + (0.5pt,0)$);
					\begin{scope}
						\clip (interior.north west) rectangle ([xshift=3ex]interior.east);
						\filldraw [white, blur shadow={shadow opacity=60, shadow yshift=-.75ex}, rounded corners=2pt] (interior.north west) rectangle (interior.south east);
					\end{scope}
					\begin{scope}[gray!80!black]
						\fill (dotA) circle (2pt);
						\fill (dotB) circle (2pt);
					\end{scope}
				},
		},
	#1,
}

%%%%%%%%%%%%%%%%%%%%%%%%%%%%%%%%%%%%%%%%%%%
% TABLE OF CONTENTS
%%%%%%%%%%%%%%%%%%%%%%%%%%%%%%%%%%%%%%%%%%%

\usepackage{tikz}

\definecolor{doc}{RGB}{0,60,110}
\usepackage{titletoc}
\contentsmargin{0cm}
\titlecontents{chapter}[3.7pc]
{\addvspace{30pt}%
	\begin{tikzpicture}[remember picture, overlay]%
		\draw[fill=doc!60,draw=doc!60] (-7,-.1) rectangle (-0.2,.6);%
		\pgftext[left,x=-3.5cm,y=0.2cm]{\color{white}\Large\sc\bfseries Chapitre\ \thecontentslabel};%
	\end{tikzpicture}\color{doc!60}\large\sc\bfseries}%
{}
{}
{\;\titlerule\;\large\sc\bfseries Page \thecontentspage
	\begin{tikzpicture}[remember picture, overlay]
		\draw[fill=doc!60,draw=doc!60] (2pt,0) rectangle (4,0.1pt);
	\end{tikzpicture}}%
\titlecontents{section}[3.7pc]
{\addvspace{2pt}}
{\contentslabel[\thecontentslabel]{2pc}}
{}
{\hfill\small \thecontentspage}
[]
\titlecontents*{subsection}[3.7pc]
{\addvspace{-1pt}\small}
{}
{}
{\ --- \small\thecontentspage}
[ \textbullet\ ][]

\makeatletter
\renewcommand{\tableofcontents}{%
	\chapter*{%
	  \vspace*{-20\p@}%
	  \begin{tikzpicture}[remember picture, overlay]%
		  \pgftext[right,x=15cm,y=0.2cm]{\color{doc!60}\Huge\sc\bfseries \contentsname};%
		  \draw[fill=doc!60,draw=doc!60] (13,-.75) rectangle (20,1);%
		  \clip (13,-.75) rectangle (20,1);
		  \pgftext[right,x=15cm,y=0.2cm]{\color{white}\Huge\sc\bfseries \contentsname};%
	  \end{tikzpicture}}%
	\@starttoc{toc}}
\makeatother


%%%%%%%%%%%%%%%%%%%%%%%%%%%%%%%%%%%%%%%%%%%
% MINTED FOR PYTHON ALGORITHMS
%%%%%%%%%%%%%%%%%%%%%%%%%%%%%%%%%%%%%%%%%%%

\usepackage{tcolorbox}
\tcbuselibrary{minted,breakable,xparse,skins}
\definecolor{bg}{gray}{0.95}
\DeclareTCBListing{mintedbox}{O{}m!O{}}{%
  breakable=true,
  listing engine=minted,
  listing only,
  minted language=#2,
  minted style=default,
  minted options={%
    linenos,
    gobble=0,
    breaklines=true,
    breakafter=,,
    fontsize=\small,
    numbersep=8pt,
    #1},
  boxsep=0pt,
  left skip=0pt,
  right skip=0pt,
  left=25pt,
  right=0pt,
  top=3pt,
  bottom=3pt,
  arc=5pt,
  leftrule=0pt,
  rightrule=0pt,
  bottomrule=2pt,
  toprule=2pt,
  colback=bg,
  colframe=orange!70,
  enhanced,
  overlay={%
    \begin{tcbclipinterior}
    \fill[orange!20!white] (frame.south west) rectangle ([xshift=20pt]frame.north west);
    \end{tcbclipinterior}},
  #3}
  
  
 % for braces
\usetikzlibrary{decorations.pathreplacing}



				% ENABLE or DISABLE font change
				% use XeLaTeX if true
				\dystrue
				\dysfalse

				\solutionstrue
				\solutionsfalse

\AdvanceDate[0]

\begin{document}
\pagestyle{fancy}
\fancyhead[L]{Seconde 13}
\fancyhead[C]{\textbf{Fonctions 1\ifsolutions -- Solutions  \fi}}
\fancyhead[R]{\today}

\exe{
	Un étudiant jette une balle dans les airs et mesure la hauteur de la balle tous les quarts de seconde.
	Il note ses résultats dans le tableau ci-dessous.
	\begin{center}
		\begin{tabular}{|c|c|c|c|c|c|c|c|}\hline
			Hauteur (cm) & 85 & 145 & 190 & 145 & 85 & 40 & 0 \\ \hline
			Temps (s) & 0 & 0,25 & 0,5 & 0,75 & 1 & 1,25 & 1,5 \\\hline
		\end{tabular}
	\end{center}
	
	\begin{enumerate}
		\item La hauteur est-elle une fonction du temps ? Justifier.
		\item Le temps est-il une fonction de la hauteur ? Justifier.
	\end{enumerate}
}{
	\begin{enumerate}
		\item On peut associer \textbf{une seule} hauteur à chaque temps. La hauteur peut donc être vue comme fonction du temps.
		\item On ne peut pas associer un unique temps à chaque hauteur. Par exemple, il y a deux temps distincts pour lesquels la hauteur est de $85$cm (0s et 1s).
		Le temps ne peut donc pas être vu comme fonction de la hauteur.
	\end{enumerate}
}
	
	
	\exe{
		\begin{enumerate}
			\item
			Montrer que le rayon $R$ d'un cercle est fonction de son périmètre $P$ et écrire la fonction $R(P)$ associée.
			\item
			Montrer que l'aire $A$ d'un cercle est fonction de son rayon $R$ et écrire la fonction $A(R)$ associée.
			\item
			En déduire que l'aire $A$ d'un cercle est fonction de son périmètre $P$ et écrire la fonction $A(P)$ associée.
		\end{enumerate}
	}{
		\begin{enumerate}
			\item
			De la relation 
				\[ P = 2\pi \cdot R, \]
			on déduit que 
				\[ R = \dfrac{1}{2\pi} \cdot P. \]
			Remarquons qu'on a inversé le membre de gauche et le membre de droite pour extraire la forme d'une fonction : une valeur de $P$ donne une unique valeur de $R$.
			
			Ainsi on peut voir $R$ comme une fonction de $P$. On note alors $R=R(P)$, qui vérifie
			\[ R(P) = \dfrac{1}{2\pi} \cdot P. \]
			
			\item 
			La formule de l'aire
				\[ A = \pi \cdot R^2 \]
			exprime directement l'aire comme fonction du rayon : à chaque rayon possible, on peut calculer une unique aire.
			On note alors
				\[ A(R) =  \pi \cdot R^2. \]
			\item
			On souhaite exprimer l'aire $A$ comme fonction du périmètre $P$.
			Or d'après les questions précédentes, l'aire $A$ est fonction du rayon $R$, et le rayon $R$ est lui-même fonction du périmètre $P$.
			
			Ainsi, une valeur de $P$ donne une unique valeur de $R$ qui donne une unique valeur de $A$.
			En suivant le raisonnement, on peut composer les fonctions trouvées ci-dessus en considérant $A(R(P))$.
			Pour ne pas surcharger les notations, appelons plutôt $\text{Aire}$ la fonction prenant un périmètre.
				\begin{align*}
				 \text{Aire}(P) &= A(R(P)) \\ &=A(\dfrac{1}{2\pi} \cdot P) \\ &= \pi \left( \dfrac{1}{2\pi} \cdot P \right)^2 \\ &= \dfrac{\pi}{4\pi^2} \cdot P^2 = \dfrac{1}{4\pi} \cdot P^2.
				 \end{align*}
		·\end{enumerate}
	}
	
	
	\exe{
		Considérons la fonction $f$ donnée par
		\begin{align*}
			f: \R & \longrightarrow \R \\
			x& \longmapsto 3\cdot x+1.
		\end{align*}
		Autrement dit, $f(x) = 3\cdot x + 1$ pour tout $x\in\R$.
		
		\begin{enumerate}
			\item
			Calculer l'image par $f$ de $0$, de $3,1$, de $\frac13$, de $-1$, de $-\frac23$.
			\item
			Donner un antécédent de $1$ par $f$.
			\item
			Déterminer tous les antécédents de $6$ par $f$.	
		\end{enumerate}
	}{
	
		\begin{enumerate}
		\item
		L'image de $x$ est donnée par $f(x)$. On calcule donc
			\begin{align*}
				f(0) &= 3\cdot0 + 1 = 1 \\
				f(3,1) &= 3\cdot3,1 + 1 = 10,3 \\
				f(\frac13) &= 3\cdot\frac13 + 1 = 1 + 1 = 2 \\
				f(-1) &= 3\cdot(-1) + 1 = -2 \\
				f(-\frac23) &= 3 \cdot (- \frac23) + 1 = -1.
			\end{align*}
		\item
		La relation
			\[ f(0) = 1 \]
		implique que l'image de $0$ est $1$, et qu'un antécédent de $1$ est $0$.
		On a donc trouvé $0$, antécédent de $1$ par $f$.
		\item
		Appelons $x$ un antécédent de $6$ par $f$.
		Autrement dit, l'image de $x$ est $6$, et donc 
			\[ f(x) = 6. \]
		En utilisant l'expression algébrique de $f$, on trouve
			\begin{align*}
				f(x) &= 6, \\
				3\cdot x + 1 &= 6, \\
				3\cdot x &= 5, \\
				x &= \dfrac53.
			\end{align*}
		Ainsi $x=\frac53$ est le seul antécédent de $6$ par $f$.
		En général, certaines fonctions admettes plusieurs antécédents.
		Voir par exemple l'exercice suivant.
		\end{enumerate}
	
	
	}
	
	

	\exe{
		Un fonction $f$ admet le tableau de valeurs suivant.
			\begin{center}
			\begin{tabular}{|c|c|c|c|c|}\hline
				$x$ & 0 & -2 & 1 & -1 \\ \hline
				$f(x)$ & 1 & 0 & 0 & 1 \\ \hline
			\end{tabular}
			\end{center}
		Parmis les expressions algébriques suivantes, trouver celle qui correspond à $f(x)$.
			\begin{multicols}{2}
			\begin{enumerate}[i)]
				\item $1-x$
				\item $1+\dfrac{x}2$
				\item $\dfrac{1-x}2$
				\item $\dfrac{-x^2 - x + 2}2$
			\end{enumerate}
			\end{multicols}
	}{
		Il s'agit de discriminer les fonctions possibles en utilisant les images du tableau.
			\begin{enumerate}[i)]
				\item $1-x$ vaut bien $1$ en $x=0$, mais l'expression vaut $3$ en $x=-2$, donc ce n'est pas l'expression de $f$.
				\item  $1+\dfrac{x}2$ vaut bien $1$ en $x=0$ et $0$ en $x=-2$, mais elle vaut $1$ en $x=1$, ce n'est donc pas l'expression de $f$.
				\item $\dfrac{1-x}2$ vaut $\frac12$ en $x=0$, ce n'est donc pas l'expression de $f$.
				\item On déduit $f(x) = \dfrac{-x^2 - x + 2}2$, qu'on vérifiera en calculant les images de $0, -2, 1,$ et $-1$.
			\end{enumerate}
	
		Cette exercice deviendra vite évident après avoir terminé l'étude des fonctions affines. 
		Comme les courbes représentatives des trois premières expressions sont des droites non constantes, toutes leurs images ont un unique antécédent.
		Or ici $0$ et $1$ ont deux antécédents.
	}
	
	
	\begin{figure}[!htb]
		\begin{subfigure}[b]{.45\textwidth}
\begin{mintedbox}{python}
def f(x):
    y = 2*x-3
    return y

f1 = f(1)
f2 = f(-1/2)


print(f1, f2)
\end{mintedbox}
		\caption{Programme 1.}
		\label{python:1}
		\end{subfigure}
		\begin{subfigure}[b]{.45\textwidth}
\begin{mintedbox}{python}
def f(x):
    y = x*x + 1
    z = -2*x
    return y+z

f1 = f(4)
f2 = f(-2)

print(f1, f2)
\end{mintedbox}
		\caption{Programme 2.}
		\label{python:2}
		\end{subfigure}
		\caption{Deux programmes implémentés en Python.}
		\label{python:1-2}
	\end{figure}
	
	\exe{
		Lire les deux programmes Python de la figure \ref{python:1-2}.
		Quelles valeurs impriment-t-ils ?
		
		Écrire la fonction $f(x)$ de chaque programme sous forme algébrique.
	}{}{
	
	
	\ifsolutions
	
	\subsubsection*{Algorithme $1$}
	
		En donnant $1$ à la fonction $f$ du programme $1$, celle-ci calcule
			\[ y = 2\cdot 1 - 3 = -1, \]
		et retourne donc $-1$.
		Ainsi \texttt{f1 = -1}.
		Similairement, $f(-.5) = -4$, car la fonction calcule
			\[ y = 2 \cdot (-.5) - 3 = -4, \]
		valeur qu'elle retourne.
		
		En général, $f$ retourne $2\cdot x-3$ pour une donnée $x$.
		C'est une façon algorithmique de voir
			\[ f(x) = 2\cdot x - 3. \]
		
		On aurait pu d'ailleurs raccourcir la définition de la fonction, chose que le compilateur ferait en secret dans un autre langage.
		On ne gagne cependant pas en mémoire en n'utilisant pas de valeur intermédiaire $y$ car elle est créée implicitement.
\begin{mintedbox}{python}
def f(x):
    return 2*x-3
\end{mintedbox}
		Pour gagner en mémoire, on peut modifier directement la variable si elle est passée par référence. 
		Ce n'est en général pas le cas en Python, qui choisit de passer ses paramètres par affectation.
		Voir le C++ pour un langage plus rigoureux avec ces considérations !
	
	\subsubsection*{Algorithme $2$}
		
		Similairement pour le programme $2$, lorsque $f$ prend la valeur $4$, elle calcule
			\begin{align*}
				y &= 4\cdot4 + 1 = 17, \\
				z &= -2\cdot 4 = -8,
			\end{align*}
		et renvoie la somme $9$.
		Ainsi \texttt{f1 = 9}.
		Idem pour $x=-2$, le programme calcule
			\begin{align*}
				y &= (-2)\cdot(-2) + 1 = 5, \\
				z &= -2\cdot (-2) = 4,
			\end{align*}
		et renvoie la somme $9$. Ainsi \texttt{f2 = 9} aussi.
		
		En général, on a 
			\[ f(x) = y + z = (x^2 + 1) + (-2\cdot x) = x^2 - 2\cdot x + 1. \]
		On vérifiera qu'on trouve bien $f(4) = f(-2) = 9$.
		
		En fait, on a aussi l'écriture 
			\[ f(x) = (x-1)^2, \]
		qui permet de calculer $f(x)$ beaucoup plus rapidement : au lieu de faire $5$ opérations (mise au carré, ajout de $1$, multiplication par $-2$, ajout des résultats), on en fait que $2$ (soustraction de $1$, mise au carré).
		La fonction décrite ci-après est donc beaucoup plus efficace, et ça un compilateur ne peut pas le deviner seul.
		Les mathématiques servent donc aussi à largement optimiser les algorithmes de calcul.
	}
	
\else
\newpage
\fi
\subsection*{Exercices supplémentaires}

	
	\exe{
		On considère la fonction $f:\N\rightarrow\R$ qui à chaque $k\in\N$ associe la variance de la série statistique suivante.
		
		
		\begin{center}
		\begin{tabular}{|c|c|c|c|}\hline
		Valeur   & 0 & 20 & 10 \\ \hline
		Effectif & 17 & 17 & $k$ \\ \hline
		\end{tabular}
		\end{center}
		
		\begin{enumerate}
			\item
			Montrer que la moyenne $\overline{X}$ de la série est constante et ne dépend pas de $k$.
			\item
			Utiliser la formule de la variance pour donner une expression algébrique de $f(k)$ dépendant de $k\in\N$, entier naturel.
			\item
			Calculer $f(10), f(20), f(50), f(100)$ et comparer les variances obtenues.
		\end{enumerate}
	}{
		\begin{enumerate}
			\item
			La moyenne est donnée par la somme des produits valeur $\times$ effectif divisée par l'effectif total :
				\[ \overline{X} = \dfrac{0\cdot17 + 20\cdot17 + 10\cdot k}{17+17+k} =\dfrac{10\cdot(34+k)}{34+k} = 10.
				\]
			\item
			D'après la formule de la variance,
				\begin{align*}
					f(k) &= \dfrac{17 \cdot (0-10)^2 + 17 \cdot (20-10)^2 + k \cdot(10-10^2}{17+17+k} \\
						&= \dfrac{3400}{34+k}.
				\end{align*}
			\item
			On calcule alors facilement les variances de plusieurs séries statistiques, ce qui avant était un processus long et douloureux.
				\begin{align*}
					f(10) &= \dfrac{3400}{44} \approx 77,27, \\
					f(20) &= \dfrac{3400}{55} \approx 61,82, \\
					f(50) &= \dfrac{3400}{84} \approx 40,48, \\
					f(100) &= \dfrac{3400}{134}\approx 25,37
				\end{align*}
			Sans surprise, les variances diminuent (et donc les écarts types aussi !) : les séries deviennent de plus en plus concentrées autour de leur moyenne de $10$ car on ajoute de plus en plus de valeurs moyennes.
		\end{enumerate}
	
	
	}

	\exe{
		Considérons la fonction $f$ donnée par
		\begin{align*}
			f: \R & \longrightarrow \R \\
			x& \longmapsto \dfrac15-x.
		\end{align*}
		Autrement dit, $f(x) = \dfrac15-x$ pour tout $x\in\R$.
		
		\begin{enumerate}
			\item
			Calculer l'image par $f$ de $0$, de $-3,1$, de $-\frac47$, de $0,2$, de $-\frac23$.
			\item
			Donner un antécédent de $0$ par $f$.
			\item
			Déterminer tous les antécédents de $5$ par $f$.	
		\end{enumerate}
	}{
		\begin{enumerate}
			\item
			On calcule
				\begin{align*}
					f(0) &= \dfrac15 - 0 = \dfrac15 \\
					f(-3,1) &= \dfrac15 - (-3,1) = -2,9 \\
					f\left(-\frac47\right) &= \dfrac15 - (-\dfrac47) = \dfrac{7 + 20}{35} = \dfrac{27}{35} \\
					f(0,2) &= \dfrac15 - 0,2 = -\dfrac15 \\
					f\left( -\frac23 \right) &= \dfrac15 - (-\dfrac23) = \dfrac{3 + 10}{15} = \dfrac{13}{15}
				\end{align*}
			\item
			On souhaite obtenir un $x$ tel que $f(x)$ soit nul. Un tel $x$ naturel est $\frac15$, car il annulera le $\frac15$ déjà présent.
			\item
			Pour connaître tous les antécédents de $5$ par $f$, on note $x$ un tel antécédent et on écrit l'égalité qu'il vérifie : l'image de $x$ doit être $5$.
				\[ f(x) = 5. \]
			On résoud pour trouver $x = \dfrac15 - 5 =  -\dfrac{24}{5}$ comme unique solution.
		\end{enumerate}
	
	
	}
	
	\exe{
		Considérons les fonctions $f,g$ données par
		\begin{align*}
			f: \R & \longrightarrow \R &g: \R & \longrightarrow \R \\
			x& \longmapsto x^2 -6\cdot x + 9,  &x& \longmapsto (x-3)^2. 
		\end{align*}
		Autrement dit, $f(x) =x^2 -6\cdot x + 9$ et $g(x)=(x-3)^2$ pour tout $x\in\R$.
		
		\begin{enumerate}
			\item
			Calculer les images par $f$ et $g$ de $0$, de $-1$, de $3$, de $-\frac12$, de $-\frac23$.
			\item
			Donner un antécédent de $9$ par $f$.
			\item
			Déterminer tous les antécédents de $9$ par $g$.	
			\item
			Montrer que $g(x) - 3^2 = x^2 - 6\cdot x$ et en déduire que $f(x) = g(x)$ pour tout $x\in\R$ réel.
			\item 
			Déterminer tous les antécédents de $16$ par $f$.
		\end{enumerate}
	}{
	
		\begin{enumerate}
			\item
			On trouve les images suivantes, en se rappelant que $(-1)^2 = (-1) \cdot (-1) = 1$.
				\begin{align*}
					f(0) &= 9 &= g(0), \\
					f(-1) &= 4 &= g(-1), \\
					f(3) &= 0 &= g(3), \\
					f\left(-\frac12\right) &= \dfrac{25}4 &= g\left(-\frac12\right), \\
					f\left(-\frac23\right) &= \dfrac{49}9 &= g\left(-\frac23\right).
				\end{align*}
			
			\item
			D'après la première question, $f(0) = 9$, et donc $0$ est un antécédent de $9$.
			
			\item
			On cherche les $x\in\R$ vérifiant
				\begin{align*}
					g(x) &= 9, \\
					(x-3)^2 &= 9.
				\end{align*}
			On rappelle la propriété $\sqrt{x^2} = |x|$ vue en cours : la racine carré d'un nombre au carré vaut sa valeur absolue.
			On en déduit que
				\begin{align*}
					|x-3| &= 3 \\
					x-3=3 \qquad&\text{ou}\qquad x-3 = -3 \\
					x = 6  \qquad&\text{ou}\qquad  x=0
				\end{align*}
 			Il y a donc deux solutions, $0$ et $6$ qu'on vérifiera bien sûr : $g(0) = (-3)^2 = 9$, et $g(6) = (6-3)^2 = 9$.
	
			\item
			La différence des carrés donne, d'après l'identité remarquable $(a+b)(a-b) = a^2 - b^2$, 
				\begin{align*}
					g(x) - 3^2 &= (x-3)^2 - 3^2 \\
								&= (x-3+3) \cdot (x-3-3) \\
								&= x \cdot (x-6) = x^2 - 6\cdot x.
				\end{align*}
			Il suit naturellement que
				\[ g(x) = x^2 - 6\cdot x + 3^2 = f(x), \]
			et ceci pour tout $x\in\R$ nombre réel.
			
			\item
			On souhaite résoudre
				\[ f(x) = 16, \]
			mais la forme factorisée de $g$ rend la tâche plus facile.
			Comme on a montré que $f(x) = g(x)$, on se permet de calculer
				\begin{align*}
					g(x) &= 16 \\
					(x-3)^2 &= 16 \\
					|x-3| &= 4
				\end{align*}
			Il suit donc que $x=-1$ et $x=7$ sont les deux seuls antécédents de $16$ par $g$ et donc $f$.
		\end{enumerate}
	
	}

\end{document}

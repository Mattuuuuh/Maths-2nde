% DYSLEXIA SWITCH
\newif\ifdys
		
				% ENABLE or DISABLE font change
				% use XeLaTeX if true
				\dystrue
				%\dysfalse


\ifdys

\documentclass[a4paper, 14pt]{extarticle}
\usepackage{amsmath,amsfonts,amsthm,amssymb,mathtools}

\tracinglostchars=3 % Report an error if a font does not have a symbol.
\usepackage{fontspec}
\usepackage{unicode-math}
\defaultfontfeatures{ Ligatures=TeX,
                      Scale=MatchUppercase }

\setmainfont{OpenDyslexic}[Scale=1.0]
\setmathfont{Fira Math} % Or maybe try KPMath-Sans?
\setmathfont{OpenDyslexic Italic}[range=it/{Latin,latin}]
\setmathfont{OpenDyslexic}[range=up/{Latin,latin,num}]

\else

\documentclass[a4paper, 12pt]{extarticle}

\usepackage[utf8x]{inputenc}
\usepackage{lmodern,textcomp}
\usepackage{amsmath,amsfonts,amsthm,amssymb,mathtools}

\fi


\usepackage[french]{babel}
\usepackage[
a4paper,
margin=2cm,
nomarginpar,% We don't want any margin paragraphs
]{geometry}
\usepackage{icomma}

\usepackage{fancyhdr}
\usepackage{array}

\usepackage{multicol, enumerate}
\newcolumntype{P}[1]{>{\centering\arraybackslash}p{#1}}


\usepackage{stackengine}
\newcommand\xrowht[2][0]{\addstackgap[.5\dimexpr#2\relax]{\vphantom{#1}}}

% theorems

\theoremstyle{plain}
\newtheorem{theorem}{Th\'eor\`eme}
\newtheorem*{sol}{Solution}
\theoremstyle{definition}
\newtheorem{ex}{Exercice}

% corps
\newcommand{\C}{\mathbb{C}}
\newcommand{\R}{\mathbb{R}}
\newcommand{\Rnn}{\mathbb{R}^{2n}}
\newcommand{\Z}{\mathbb{Z}}
\newcommand{\N}{\mathbb{N}}
\newcommand{\Q}{\mathbb{Q}}

% domain
\newcommand{\D}{\mathbb{D}}


% date
\usepackage{advdate}
\AdvanceDate[0]


% plots
\usepackage{pgfplots}

%subfigures
\usepackage{subcaption}

% python
\usepackage{tcolorbox}
\tcbuselibrary{minted,breakable,xparse,skins}
\definecolor{bg}{gray}{0.95}
\DeclareTCBListing{mintedbox}{O{}m!O{}}{%
  breakable=true,
  listing engine=minted,
  listing only,
  minted language=#2,
  minted style=default,
  minted options={%
    linenos,
    gobble=0,
    breaklines=true,
    breakafter=,,
    fontsize=\small,
    numbersep=8pt,
    #1},
  boxsep=0pt,
  left skip=0pt,
  right skip=0pt,
  left=25pt,
  right=0pt,
  top=3pt,
  bottom=3pt,
  arc=5pt,
  leftrule=0pt,
  rightrule=0pt,
  bottomrule=2pt,
  toprule=2pt,
  colback=bg,
  colframe=orange!70,
  enhanced,
  overlay={%
    \begin{tcbclipinterior}
    \fill[orange!20!white] (frame.south west) rectangle ([xshift=20pt]frame.north west);
    \end{tcbclipinterior}},
  #3}

% SOLUTION SWITCH
\newif\ifsolutions
				\solutionstrue
				\solutionsfalse

\ifsolutions
	\newcommand{\exe}[2]{
		\begin{ex} #1  \end{ex}
		\begin{sol} #2 \end{sol}
	}
\else
	\newcommand{\exe}[2]{
		\begin{ex} #1  \end{ex}
	}
	
\fi

\begin{document}
\pagestyle{fancy}
\fancyhead[L]{Seconde 13}
\fancyhead[C]{\textbf{Fonctions 1\ifsolutions -- Solutions  \fi}}
\fancyhead[R]{\today}

	\exe{
		Un étudiant jette une balle dans les airs et mesure la hauteur de la balle tous les quarts de seconde.
		Il note ses résultats dans le tableau ci-dessous.
		\begin{center}
			\begin{tabular}{|c|c|c|c|c|c|c|c|}\hline
				Hauteur (cm) & 85 & 145 & 190 & 145 & 85 & 40 & 0 \\ \hline
				Temps (s) & 0 & 0,25 & 0,5 & 0,75 & 1 & 1,25 & 1,5 \\\hline
			\end{tabular}
		\end{center}
		
		\begin{enumerate}
			\item La hauteur est-elle une fonction du temps ? Justifier.
			\item Le temps est-il une fonction de la hauteur ? Justifier.
		\end{enumerate}
	}{}
	
	
	\exe{
		\begin{enumerate}
			\item
			Montrer que le rayon $R$ d'un cercle est fonction de son périmètre $P$ et écrire la fonction $R(P)$ associée.
			\item
			Montrer que l'aire $A$ d'un cercle est fonction de son rayon $R$ et écrire la fonction $A(R)$ associée.
			\item
			En déduire que l'aire $A$ d'un cercle est fonction de son périmètre $P$ et écrire la fonction $A(P)$ associée.
		\end{enumerate}
	}{}
	
	
	\exe{
		Considérons la fonction $f$ donnée par
		\begin{align*}
			f: \R & \longrightarrow \R \\
			x& \longmapsto 3\cdot x+1.
		\end{align*}
		Autrement dit, $f(x) = 3\cdot x + 1$ pour tout $x\in\R$.
		
		\begin{enumerate}
			\item
			Calculer l'image par $f$ de $0$, de $3,1$, de $\frac13$, de $-1$, de $-\frac23$.
			\item
			Donner un antécédent de $1$ par $f$.
			\item
			Déterminer tous les antécédents de $6$ par $f$.	
		\end{enumerate}
	}{}
	
	
	\exe{
		On considère la fonction $f:\N\rightarrow\R$ qui à chaque $k\in\N$ associe la variance de la série statistique suivante.
		
		
		\begin{center}
		\begin{tabular}{|c|c|c|c|}\hline
		Valeur   & 0 & 20 & 10 \\ \hline
		Effectif & 17 & 17 & k \\ \hline
		\end{tabular}
		\end{center}
		
		\begin{enumerate}
			\item
			Montrer que la moyenne $\overline{X}$ de la série est constante et ne dépend pas de $k$.
			\item
			Utiliser la formule de la variance pour donner une expression algébrique de $f(k)$ dépendant de $k\in\N$, entier naturel.
			\item
			Calculer $f(10), f(20), f(50), f(100)$ et comparer les variances obtenues.
		\end{enumerate}
	}{exe:vark}
	

	\exe{
		Un fonction $f$ admet le tableau de valeurs suivant.
			\begin{center}
			\begin{tabular}{|c|c|c|c|c|}\hline
				$x$ & 0 & -2 & 1 & -1 \\ \hline
				$f(x)$ & 1 & 0 & 0 & 1 \\ \hline
			\end{tabular}
			\end{center}
		Parmis les expressions algébriques suivantes, trouver celle qui correspond à $f(x)$.
			\begin{multicols}{2}
			\begin{enumerate}[i)]
				\item $1-x$
				\item $1+\dfrac{x}2$
				\item $\dfrac{1-x}2$
				\item $\dfrac{-x^2 - x + 2}2$
			\end{enumerate}
			\end{multicols}
	}{}
	
	
	
	
	
		
	\begin{figure}[!htb]
		\begin{subfigure}[b]{.45\textwidth}
\begin{mintedbox}{python}
def f(x):
    y = 2*x-3
    return y

f1 = f(1)
f2 = f(-1/2)


print(f1, f2)
\end{mintedbox}
		\caption{Programme 1.}
		\label{python:1}
		\end{subfigure}
		\begin{subfigure}[b]{.45\textwidth}
\begin{mintedbox}{python}
def f(x):
    y = x*x + 1
    z = -2*x
    return y+z

f1 = f(4)
f2 = f(-2)

print(f1, f2)
\end{mintedbox}
		\caption{Programme 2.}
		\label{python:2}
		\end{subfigure}
		\caption{Deux programmes implémentés en Python.}
		\label{python:1-2}
	\end{figure}
	
	\exe{
		Lire les deux programmes Python de la figure \ref{python:1-2}.
		Quelles valeurs impriment-t-ils ?
		
		Écrire la fonction $f(x)$ de chaque programme sous forme algébrique.
	}{}

\newpage
\subsection*{Exercices supplémentaires}

	\exe{
		Considérons la fonction $f$ donnée par
		\begin{align*}
			f: \R & \longrightarrow \R \\
			x& \longmapsto \dfrac15-x.
		\end{align*}
		Autrement dit, $f(x) = \dfrac15-x$ pour tout $x\in\R$.
		
		\begin{enumerate}
			\item
			Calculer l'image par $f$ de $0$, de $-3,1$, de $-\frac47$, de $0,2$, de $-\frac23$.
			\item
			Donner un antécédent de $0$ par $f$.
			\item
			Déterminer tous les antécédents de $5$ par $f$.	
		\end{enumerate}
	}{}
	
	\exe{
		Considérons les fonctions $f,g$ données par
		\begin{align*}
			f: \R & \longrightarrow \R &g: \R & \longrightarrow \R \\
			x& \longmapsto x^2 -6\cdot x + 9,  &x& \longmapsto (x-3)^2. 
		\end{align*}
		Autrement dit, $f(x) =x^2 -6\cdot x + 9$ et $g(x)=(x-3)^2$ pour tout $x\in\R$.
		
		\begin{enumerate}
			\item
			Calculer les images par $f$ et $g$ de $0$, de $-1$, de $3$, de $-\frac12$, de $-\frac23$.
			\item
			Donner un antécédent de $9$ par $f$.
			\item
			Déterminer tous les antécédents de $9$ par $g$.	
			\item
			Montrer que $g(x) - 3^2 = x^2 - 6\cdot x$ et en déduire que $f(x) = g(x)$ pour tout $x\in\R$ réel.
			\item 
			Déterminer tous les antécédents de $16$ par $f$.
		\end{enumerate}
	}{}

\end{document}

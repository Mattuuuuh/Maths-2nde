% DYSLEXIA SWITCH
\newif\ifdys
		
				% ENABLE or DISABLE font change
				% use XeLaTeX if true
				\dystrue
				\dysfalse


\ifdys

\documentclass[a4paper, 14pt]{extarticle}
\usepackage{amsmath,amsfonts,amsthm,amssymb,mathtools}

\tracinglostchars=3 % Report an error if a font does not have a symbol.
\usepackage{fontspec}
\usepackage{unicode-math}
\defaultfontfeatures{ Ligatures=TeX,
                      Scale=MatchUppercase }

\setmainfont{OpenDyslexic}[Scale=1.0]
\setmathfont{Fira Math} % Or maybe try KPMath-Sans?
\setmathfont{OpenDyslexic Italic}[range=it/{Latin,latin}]
\setmathfont{OpenDyslexic}[range=up/{Latin,latin,num}]

\else

\documentclass[a4paper, 12pt]{extarticle}

\usepackage[utf8x]{inputenc}
\usepackage{lmodern,textcomp}
\usepackage{amsmath,amsfonts,amsthm,amssymb,mathtools}

\fi


\usepackage[french]{babel}
\usepackage[
a4paper,
margin=2cm,
nomarginpar,% We don't want any margin paragraphs
]{geometry}
\usepackage{icomma}

\usepackage{fancyhdr}
\usepackage{array}

\usepackage{multicol, enumerate}
\newcolumntype{P}[1]{>{\centering\arraybackslash}p{#1}}


\usepackage{stackengine}
\newcommand\xrowht[2][0]{\addstackgap[.5\dimexpr#2\relax]{\vphantom{#1}}}

% theorems

\theoremstyle{plain}
\newtheorem{theorem}{Th\'eor\`eme}
\newtheorem*{theorem*}{Th\'eor\`eme}
\newtheorem*{sol}{Solution}
\theoremstyle{definition}
\newtheorem{ex}{Exercice}

% corps
\newcommand{\C}{\mathcal{C}}
\newcommand{\R}{\mathbb{R}}
\newcommand{\Rnn}{\mathbb{R}^{2n}}
\newcommand{\Z}{\mathbb{Z}}
\newcommand{\N}{\mathbb{N}}
\newcommand{\Q}{\mathbb{Q}}

% domain
\newcommand{\D}{\mathcal{D}}


% date
\usepackage{advdate}
\AdvanceDate[0]


% plots
\usepackage{pgfplots}

% for calligraphic C
\usepackage{calrsfs}

% SOLUTION SWITCH
\newif\ifsolutions
				\solutionstrue
				\solutionsfalse

\ifsolutions
	\newcommand{\exe}[2]{
		\begin{ex} #1  \end{ex}
		\begin{sol} #2 \end{sol}
	}
\else
	\newcommand{\exe}[2]{
		\begin{ex} #1  \end{ex}
	}
	
\fi

\begin{document}
\pagestyle{fancy}
\fancyhead[L]{Seconde 13}
\fancyhead[C]{\textbf{Fonctions 3 \ifsolutions -- Solutions  \fi}}
\fancyhead[R]{\today}
	
	\exe{
		\begin{multicols}{2}
		Considérons trois fonctions $f,g,h:]{-}3,4 ; 2,3[ \rightarrow \R$ données graphiquement ci-contre.
		
		Donner approximativement l'ensemble des nombres $x$ du domaine vérifiant les (in)équations suivantes.
		\begin{enumerate}
			\itemsep1em 
			\item $f(x) = 4$
			\item $f(x) = 1,5$
			\item $f(x) \leq 2$
			\item $f(x) \leq 0$
			\item $g(x) = h(x)$
			\item $g(x) \leq h(x)$
			\item $h(x) \leq g(x)$
		\end{enumerate}
		\vfill
		
			\begin{center}
			\begin{tikzpicture}[>=stealth, scale=1]
				\begin{axis}[xmin = -3.4, xmax=2.3, ymin=-5.1, ymax=5.1, axis x line=middle, axis y line=middle, axis line style=->, grid=both, ytick={-4,-3, ..., 3, 4}]
					\addplot[no marks, blue, -] expression[domain=-4:3, samples=100]{x^3 /3 - 2*x +3}
					node[pos=.55, above=5pt]{$\mathcal{C}_f$};
				\end{axis}
			\end{tikzpicture}
			\end{center}
		
			\begin{center}
			\begin{tikzpicture}[>=stealth, scale=1]
				\begin{axis}[xmin = -3.4, xmax=2.3, ymin=-4, ymax=8.1, axis x line=middle, axis y line=middle, axis line style=->, grid=both]
					\addplot[no marks, blue, -] expression[domain=-4:3, samples=100]{x^3 /3 - 2*x +3}
					node[pos=.4, above=15pt]{$\mathcal{C}_g$};
					\addplot[no marks, red, -] expression[domain=-4:3, samples=100]{-4*x^2 + 7}
					node[pos=.68, above=15pt]{$\mathcal{C}_h$};
				\end{axis}
			\end{tikzpicture}
			\end{center}
		\end{multicols}·
	}{}
	
	
	\begin{theorem*}[Identités remarquables]
		Pour tous les $a,b\in\R$ réels, on a les identités suivantes.
			\begin{align*}
				a^2 - b^2 &= (a+b)(a-b) \\
				(a+b)^2 &= a^2 + 2ab + b^2 \\
				(a-b)^2 &= a^2 - 2ab + b^2
			\end{align*}
	\end{theorem*}


	\exe{
		Développer les expressions algébriques suivantes.
			\begin{multicols}{2}
			\begin{enumerate}[$\bullet$]
				\item $f(x) = (1+x)^2$
				\item $g(x) = (x-3)^2$
				\item $h(x) = (3-x)^2$
				\item $F(x) = (3 + 2x)^2$
				\item $G(x) = (3x - 7)^2$
				\item $H(x) = (-7x - 2)^2$
			\end{enumerate}
			\end{multicols}
	}{}
	
	\exe{
		Pour chaque équation suivante, factoriser le membre de gauche à l'aide des identités remarquables et donner l'ensemble des solutions $x\in\R$.
			\begin{multicols}{2}
			\begin{enumerate}[a)]
				\item $x^2 + 2x + 1 = 16$
				\item $4x^2 - 4x + 1 = 9$
				\item $9 - 18x + 9x^2 = 0$
				\item $x^2 - 2x + 1 = 5$
				\item $x^2 +6x + 9 = -1$
				\item $16x^2 - 64 = 0$
			\end{enumerate}
			\end{multicols}
	}{}
	
	
	\newpage
	\subsection*{Exercices supplémentaires}
	
	\exe{
		On appelle \emph{triplet pythagoricien} un triplet $(a ; b ; c)$, liste de $3$ nombres entiers naturels $a,b,c\in\N$ vérifiant
			\[ a^2 + b^2 = c^2. \]
		Nommons deux paramètres entiers $k,\ell\in\N$ avec $k\geq \ell$.
		Vérifier que le triplet
			\[ (k^2 - \ell^2 ; 2k\ell ; k^2 + \ell^2) \]
		est pythagoricien.
		
		Créer quelques triplets en prenant des valeurs de $k \geq \ell$ et les vérifier à la calculatrice.
		Par exemple, pour $k=4$ et $\ell=2$, on trouve le triplet $(12 ; 16 ; 20)$, qui vérifie bien
			\[ 12^2 + 16^2 = 20^2. \] 
	}{}
	
	\exe{
		Démontrer que, pour tout $n\in\N$ entier naturel, on a l'égalité
			\[ (n+1)^2 -n^2 = 2n+1. \]
		En déduire, en prenant $n\in\{0;1;2; \dots\}$ qu'on a
			\begin{align*}
				1 &= 1 \\
				2^2 - 1 &= 3 \\
				3^2 - 2^2 &= 5 \\
				4^2 - 3^2 &= 7 \\
				5^2 - 4^2 &= 9 \\
					\vdots\quad \, &= \ \ \vdots
			\end{align*}
		En sommant les membres de gauche et de droite de chaque équation, montrer d'abord que
			\[ 5^2 = 1 + 3 + 5 + 7 + 9, \]
		puis généraliser pour montrer que, pour tout $n\in\N$ entier naturel,
			\[ (n+1)^2 = 1 + 3 + 5 + \dots + (2n+1). \]
	}{}
	
	\exe{
		On rappelle qu'un nombre est impair si et seulement s'il s'écrit de la forme $2k+1$ où $k\in\N$ est un entier naturel.
		Montrer et élargir le résultat vu en cours : si $n$ est impair, alors $n^2$ est impair et $n^3$ est aussi impair.
		
		Adapter la démonstration par contradiction de l’irrationalité de $\sqrt{2}$ pour montrer que $\sqrt[3]{2}$ est irrationnel. ($\sqrt[3]{2}$ est l'unique $x\in\R$ vérifiant $x^3 = 2$.)
	}{}


\end{document}

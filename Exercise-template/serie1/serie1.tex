% DYSLEXIA SWITCH
\newif\ifdys
		
				% ENABLE or DISABLE font change
				% use XeLaTeX if true
				\dystrue
				\dysfalse


\ifdys

\documentclass[a4paper, 14pt]{extarticle}
\usepackage{amsmath,amsfonts,amsthm,amssymb,mathtools}

\tracinglostchars=3 % Report an error if a font does not have a symbol.
\usepackage{fontspec}
\usepackage{unicode-math}
\defaultfontfeatures{ Ligatures=TeX,
                      Scale=MatchUppercase }

\setmainfont{OpenDyslexic}[Scale=1.0]
\setmathfont{Fira Math} % Or maybe try KPMath-Sans?
\setmathfont{OpenDyslexic Italic}[range=it/{Latin,latin}]
\setmathfont{OpenDyslexic}[range=up/{Latin,latin,num}]

\else

\documentclass[a4paper, 12pt]{extarticle}

\usepackage[utf8x]{inputenc}
%fonts
\usepackage{amsmath,amsfonts,amsthm,amssymb,mathtools}
% comment below to default to computer modern
\usepackage{libertinus,libertinust1math}

\fi


\usepackage[french]{babel}
\usepackage[
a4paper,
margin=2cm,
nomarginpar,% We don't want any margin paragraphs
]{geometry}
\usepackage{icomma}

\usepackage{fancyhdr}
\usepackage{array}
\usepackage{hyperref}

\usepackage{multicol, enumerate}
\newcolumntype{P}[1]{>{\centering\arraybackslash}p{#1}}


\usepackage{stackengine}
\newcommand\xrowht[2][0]{\addstackgap[.5\dimexpr#2\relax]{\vphantom{#1}}}

% theorems

\theoremstyle{plain}
\newtheorem{theorem}{Th\'eor\`eme}
\newtheorem*{sol}{Solution}
\theoremstyle{definition}
\newtheorem{ex}{Exercice}
\newtheorem*{rpl}{Rappel}
\newtheorem{enigme}{Énigme}

% corps
\usepackage{calrsfs}
\newcommand{\C}{\mathcal{C}}
\newcommand{\R}{\mathbb{R}}
\newcommand{\Rnn}{\mathbb{R}^{2n}}
\newcommand{\Z}{\mathbb{Z}}
\newcommand{\N}{\mathbb{N}}
\newcommand{\Q}{\mathbb{Q}}

% variance
\newcommand{\Var}[1]{\text{Var}(#1)}

% domain
\newcommand{\D}{\mathcal{D}}


% date
\usepackage{advdate}
\AdvanceDate[0]


% plots
\usepackage{pgfplots}

% table line break
\usepackage{makecell}
%tablestuff
\def\arraystretch{2}
\setlength\tabcolsep{15pt}

%subfigures
\usepackage{subcaption}

\definecolor{myg}{RGB}{56, 140, 70}
\definecolor{myb}{RGB}{45, 111, 177}
\definecolor{myr}{RGB}{199, 68, 64}

% fake sections with no title to move around the merged pdf
\newcommand{\fakesection}[1]{%
  \par\refstepcounter{section}% Increase section counter
  \sectionmark{#1}% Add section mark (header)
  \addcontentsline{toc}{section}{\protect\numberline{\thesection}#1}% Add section to ToC
  % Add more content here, if needed.
}


% SOLUTION SWITCH
\newif\ifsolutions
				\solutionstrue
				%\solutionsfalse

\ifsolutions
	\newcommand{\exe}[2]{
		\begin{ex} #1  \end{ex}
		\begin{sol} #2 \end{sol}
	}
\else
	\newcommand{\exe}[2]{
		\begin{ex} #1  \end{ex}
	}
	
\fi


% tableaux var, signe
\usepackage{tkz-tab}


%pinfty minfty
\newcommand{\pinfty}{{+}\infty}
\newcommand{\minfty}{{-}\infty}

\begin{document}


\AdvanceDate[0]

\begin{document}
\pagestyle{fancy}
\fancyhead[L]{Seconde 13}
\fancyhead[C]{\textbf{Systèmes d'équations linéaires }}
\fancyhead[R]{\today}

\exe{, difficulty=1}{
	Pour chaque paire d'équations, ajouter entre elles le symbole
		\begin{enumerate}[itemindent=1.5cm]
			\item[$\implies$] si la première équation implique la seconde ;
			\item[$\impliedby$] si la deuxième équation implique la première ;
			\item[$\iff$] si les deux équations sont équivalentes ;
			\item[$\centernot\iff$] si les deux équations sont indépendantes.
		\end{enumerate}
	Spécifier l'opération appliquée pour obtenir la deuxième équation à partir de la première le cas échéant.
	
	Par abus de notation et pour gagner du temps, l'absence de symbole signifie en général que les équations sont équivalentes, mais il faut toujours faire attention à ce que cela soit bien le cas !
	
	\begin{center}
	\begin{tabular}{ccc|c|c}
		Équation 1 & Symbole & Équation 2 & Opération & \thead{Les équations sont-\\ elles équivalentes ?} \\ \hline
		$3x + 2y = 1$ &  & $6x  + 4y = 2$ & &  \\ \hline
		$-x + 2y = 1$ &  & $-x + 2y - 1 = 0$ & &  \\ \hline
		$3x + 2y = 1$ & & $-3x - 5y = -3$ & &  \\ \hline
		$x^2 = -3x$ &  & $x = -3$ & &  \\ \hline
		$-10x - y = 3$ &  & $0=0$ & &  \\ \hline
		$x-y= 3$ & & $-x + y = -3$ & &  \\ \hline
		$x -y = 0$ & & $x=y$ & &  \\ \hline
		$-x + 2y = 3$ & & $0=0$ & &  \\ \hline
		$2x + 3y + 4 = 2x + 3y + 5$ & & $1=0$ & &  \\ \hline
		$-x - y = 1$ & & $-x - y = 1$ & &  \\ \hline
		$0=0$ & & $23x - 10y = 6$ & &  \\ \hline
		$x + 2y = 3$ & & $x^2 + 2yx = 3x$ & &  \\ \hline
		$x-3y  =0$ & & $-2x+6y = 0$ & &  \\ \hline
		$-2x + 3y = -3$ & & $4x - 6y = -6$ & &  \\ \hline
		$x^2 = x$ & & $x = 1$ & &  \\ \hline
		$x = 3$ & & $x^2 = 9$ & &  \\ \hline
	\end{tabular}
	\end{center}
}{exe:1}{

	
	\begin{center}
	\begin{tabular}{ccc|c|c}
		Équation 1 & Symbole & Équation 2 & Opération & \thead{Les équations sont-\\ elles équivalentes ?} \\ \hline
		$3x + 2y = 1$ & {$\iff$} & $6x  + 4y = 2$ & {$\times2$} & {Oui} \\ \hline
		$-x + 2y = 1$ & {$\iff$} & $-x + 2y - 1 = 0$ & {$-1$} & {Oui} \\ \hline
		$3x + 2y = 1$ & {$\centernot\iff$} & $-3x - 5y = -3$ & & {Non} \\ \hline
		$x^2 = -3x$ & {$\impliedby$} & $x = -3$ & & {Non} \\ \hline
		$-10x - y = 3$ & {$\implies$} & $0=0$ & {$\times0$} & {Non} \\ \hline
		$x-y= 3$ & {$\iff$} & $-x + y = -3$ & {$\times(-1)$} & {Oui} \\ \hline
		$x -y = 0$ &{$\iff$} & $x=y$ & {$+y$} & {Oui} \\ \hline
		$-x + 2y = 3$ & {$\implies$} & $0=0$ & {$\times0$} & {Non} \\ \hline
		$2x + 3y + 4 = 2x + 3y + 5$ & {$\iff$} & $1=0$ & {$+(-2x-3y)$} & {Oui} \\ \hline
		$-x - y = 1$ & {$\centernot\iff$} & $-x - y = 1$ & & {Non} \\ \hline
		$0=0$ & {$\impliedby$} & $23x - 10y = 6$ & & {Non} \\ \hline
		$x + 2y = 3$ & {$\implies$} & $x^2 + 2yx = 3x$ & {$\times x$} & {Non} \\ \hline
		$x-3y  =0$ & {$\iff$} & $-2x+6y = 0$ & {$\times(-2)$} & {Oui} \\ \hline
		$-2x + 3y = -3$ & {$\centernot\iff$} & $4x - 6y = -6$ & & {Non} \\ \hline
		$x^2 = x$ & {$\impliedby$} & $x = 1$ & & {Non} \\ \hline
		$x = 3$ & {$\implies$} & $x^2 = 9$ & {Mise au carré} & {Non} \\ \hline
	\end{tabular}
	\end{center}
	
}

\newpage

\exe{}{
	Pour chaque système d'équations d'inconnues $x, y \in \R$, donner un système équivalent tel que les coefficients multipliant $x$ soient opposés (c'est-à-dire leur somme soit nulle).
	\setlength{\columnsep}{1cm}
	\begin{multicols}{2}
	\begin{enumerate}[label=\roman*), itemsep=20pt]
		\item $\systeme{x + 3y = 3{,}, -2x + 2y = 2.} \iff \systeme{,}$
		\item $\systeme{x - 7y = 1{,}, -3x + 5y = 13.} \iff \systeme{,}$
		\item $\systeme{2x - y = 3{,}, -x + 2y = -5.} \iff \systeme{,}$
		\item $\systeme{4x  - y = -4{,},  x + 5y = 2.} \iff  \systeme{,}$
		\item $\systeme{-2x - 2y = -5{,}, 3x + 7y = 10.} \iff\systeme{,}$
		\item $\systeme{7x - y = 10{,}, 5x + 2y = 1.} \iff\systeme{,}$
	\end{enumerate}
	\end{multicols}
}{exe:2}{
	\setlength{\columnsep}{1cm}
	\begin{multicols}{2}
	\begin{enumerate}[label=\roman*), itemsep=20pt]
		\item $\systeme{x + 3y = 3{,}, -2x + 2y = 2.} \iff \systeme{2x+6y=6{,},-2x+2y=2.}$
		\item $\systeme{x - 7y = 1{,}, -3x + 5y = 13.} \iff \systeme{3x - 21y = 3{,}, -3x + 5y = 13.}$
		\item $\systeme{2x - y = 3{,}, -x + 2y = -5.} \iff \systeme{2x - y = 3{,}, -2x + 4y = -10.}$
		\item $\systeme{4x  - y = -4{,},  x + 5y = 2.} \iff  \systeme{4x  - y = -4{,},  -4x - 20y = -8.}$
		\item $\systeme{-2x - 2y = -5{,}, 3x + 7y = 10.} \iff\systeme{-6x - 6y = -15{,}, 6x + 14y = 20.}$
		\item $\systeme{7x - y = 10{,}, 5x + 2y = 1.} \iff \systeme{35x - 5y = 50{,}, -35x - 14y = -7.}$
	\end{enumerate}
	\end{multicols}
}

\exe{}{
	Pour chaque système de l'exercice \ref{exe:2}, 
		\begin{enumerate}
			\item combiner les équations pour trouver $y$ ; 
			\item trouver $x$ ; 
			\item vérifier que le couple $(x ;y)$ obtenu soit bien solution du système initial.
		\end{enumerate}
}{exe:3}{ \, \\
	\begin{enumerate}[label=Système \roman*) :, itemsep=20pt, leftmargin=80pt]
		\item La somme des deux équations donne
			\[ 8y = 8 \iff y = 1. \]
		En remplaçant $y$ par $1$ dans la première équation, on trouve $x + 3 = 3$, et donc $x =0$.
			\[ (x ; y ) = (0; 1). \]
		
		\item La somme des deux équations donne
			\[ -16y = 16 \iff y = -1. \]
		En remplaçant $y$ par $-1$ dans la première équation, on trouve $x + 7 = 1$, et donc $x =-6$.
			\[ (x ; y ) = (-6; -1). \]
		
		\item La somme des deux équations donne
			\[ 3y = -7 \iff y = -\dfrac73. \]
		En remplaçant $y$ par $-\dfrac73$ dans la première équation, on trouve $2x +\dfrac73 = 3$, et donc $x =\dfrac13$.
			\[ (x ; y ) = \left(\dfrac13; -\dfrac73 \right). \]
		
		\item La somme des deux équations donne
			\[ -21y = -12 \iff y = \dfrac{12}{21} = \dfrac47. \]
		En remplaçant $y$ par $ \dfrac47$ dans la deuxième équation, on trouve $x +\dfrac{20}7 = 2$, et donc $x =-\dfrac67$.
			\[ (x ; y ) = \left(-\dfrac67; \dfrac47 \right). \]
		
		\item La somme des deux équations donne
			\[ 8y = 5 \iff y = \dfrac58. \]
		En remplaçant $y$ par $\dfrac58$ dans la deuxième équation, on trouve $3x +\dfrac{35}8 = 10 \iff x =\dfrac{45}{24} = \dfrac{15}8$.
			\[ (x ; y ) = \left(\dfrac{15}8; \dfrac58 \right). \]
		
		\item La somme des deux équations donne
			\[ -19y = 43 \iff y = -\dfrac{43}{19}. \]
		En remplaçant $y$ par $-\dfrac{43}{19}$ dans la première équation, on trouve $7x +\dfrac{43}{19}= 10 \iff x =\dfrac{147}{133} = \dfrac{21}{19}$.
			\[ (x ; y ) = \left( \dfrac{21}{19};  -\dfrac{43}{19} \right). \]
	\end{enumerate}
}

\exe{}{
	Donner les solutions réelles $(x;y)$ des systèmes suivantes.
	
	\begin{multicols}{2}
	\begin{enumerate}[label=\roman*), itemsep=20pt]
		\item $\sys{2x+4y=0}{-2x -2y = 2}$
		\item $\sys{-6x+3y=-3}{x -3y = -2}$
		\item $\sys{x+y = 2}{x+y=2}$
		\item $\sys{8x-4y = 6+2x-y}{7x+12y=-5+y}$
		\item $\sys{x-y = 1}{-x+y=10}$
		\item $\sys{\dfrac12 x - \dfrac23 y= -1}{ ,\dfrac15 x + \dfrac72 y= 5}$
		\item $\sys{2x - 8y = 2}{-4x+16y=-1}$
		\item $\sys{5x + y = -2x-1-y}{8x = -1+2y}$
		\item $\systeme[yx]{2y + 12x = -3{,}, , {-6}x=y + \dfrac32}$
	\end{enumerate}
	\end{multicols}
}{exe:4}{\, \\
	
	\begin{enumerate}[label=\roman*), itemsep=20pt]
		\item $(x ; y) = (-2 ; 1)$
		\item $(x ; y) = (1;1)$
		\item Les deux équations sont redondantes : on a en fait qu'une seule contrainte pour deux variables et donc un degré liberté. Pour chaque choix de $x$, on peut trouver un $y$ tel que $(x;y)$ soit solution.
		L'ensemble des solutions est
			\[ \{ (x ; y) \text{ tq. } x+y=2, x, y \in \R \} = \{ (x ; y) \text{ tq. } y = -x + 2, \text{ où $x$ parcourt $\R$} \} = \C_f, \]
		où $f(x) = -x+2$.
		\item $(x ; y) = \left(\dfrac{17}{29} ; -\dfrac{24}{29}\right)$
		\item Les deux équations sont contradictoires et aucune solution n'existe.
		\item $(x ; y) = \left(-\dfrac{10}{113} ; \dfrac{162}{113}\right)$
		\item Les deux équations sont contradictoires et aucune solution n'existe.
		\item $(x ; y) = \left(-\dfrac{2}{15} ; -\dfrac{1}{30}\right)$
		\item Les deux équations sont redondantes : on a en fait qu'une seule contrainte pour deux variables et donc un degré liberté. Pour chaque choix de $x$, on peut trouver un $y$ tel que $(x;y)$ soit solution.
		L'ensemble des solutions est
			\[ \left\{ (x ; y) \text{ tq. } -6x -y = \dfrac32, x, y \in \R \right\} = \left\{ (x ; y) \text{ tq. } y= -6x - \dfrac32 \text{ où $x$ parcourt $\R$} \right\} = \C_f, \]
		où $f(x) =-6x - \dfrac32$.
	\end{enumerate}
}



\newpage
\fancyhead[C]{\textbf{Solutions }}
\shipoutAnswer

\end{document}

				% ENABLE or DISABLE font change
				% use XeLaTeX if true
\newif\ifdys
				\dystrue
				\dysfalse

\newif\ifsolutions
				\solutionstrue
				\solutionsfalse

% DYSLEXIA SWITCH
\newif\ifdys
		
				% ENABLE or DISABLE font change
				% use XeLaTeX if true
				\dystrue
				\dysfalse


\ifdys

\documentclass[a4paper, 14pt]{extarticle}
\usepackage{amsmath,amsfonts,amsthm,amssymb,mathtools}

\tracinglostchars=3 % Report an error if a font does not have a symbol.
\usepackage{fontspec}
\usepackage{unicode-math}
\defaultfontfeatures{ Ligatures=TeX,
                      Scale=MatchUppercase }

\setmainfont{OpenDyslexic}[Scale=1.0]
\setmathfont{Fira Math} % Or maybe try KPMath-Sans?
\setmathfont{OpenDyslexic Italic}[range=it/{Latin,latin}]
\setmathfont{OpenDyslexic}[range=up/{Latin,latin,num}]

\else

\documentclass[a4paper, 12pt]{extarticle}

\usepackage[utf8x]{inputenc}
%fonts
\usepackage{amsmath,amsfonts,amsthm,amssymb,mathtools}
% comment below to default to computer modern
\usepackage{libertinus,libertinust1math}

\fi


\usepackage[french]{babel}
\usepackage[
a4paper,
margin=2cm,
nomarginpar,% We don't want any margin paragraphs
]{geometry}
\usepackage{icomma}

\usepackage{fancyhdr}
\usepackage{array}
\usepackage{hyperref}

\usepackage{multicol, enumerate}
\newcolumntype{P}[1]{>{\centering\arraybackslash}p{#1}}


\usepackage{stackengine}
\newcommand\xrowht[2][0]{\addstackgap[.5\dimexpr#2\relax]{\vphantom{#1}}}

% theorems

\theoremstyle{plain}
\newtheorem{theorem}{Th\'eor\`eme}
\newtheorem*{sol}{Solution}
\theoremstyle{definition}
\newtheorem{ex}{Exercice}
\newtheorem*{rpl}{Rappel}
\newtheorem{enigme}{Énigme}

% corps
\usepackage{calrsfs}
\newcommand{\C}{\mathcal{C}}
\newcommand{\R}{\mathbb{R}}
\newcommand{\Rnn}{\mathbb{R}^{2n}}
\newcommand{\Z}{\mathbb{Z}}
\newcommand{\N}{\mathbb{N}}
\newcommand{\Q}{\mathbb{Q}}

% variance
\newcommand{\Var}[1]{\text{Var}(#1)}

% domain
\newcommand{\D}{\mathcal{D}}


% date
\usepackage{advdate}
\AdvanceDate[0]


% plots
\usepackage{pgfplots}

% table line break
\usepackage{makecell}
%tablestuff
\def\arraystretch{2}
\setlength\tabcolsep{15pt}

%subfigures
\usepackage{subcaption}

\definecolor{myg}{RGB}{56, 140, 70}
\definecolor{myb}{RGB}{45, 111, 177}
\definecolor{myr}{RGB}{199, 68, 64}

% fake sections with no title to move around the merged pdf
\newcommand{\fakesection}[1]{%
  \par\refstepcounter{section}% Increase section counter
  \sectionmark{#1}% Add section mark (header)
  \addcontentsline{toc}{section}{\protect\numberline{\thesection}#1}% Add section to ToC
  % Add more content here, if needed.
}


% SOLUTION SWITCH
\newif\ifsolutions
				\solutionstrue
				%\solutionsfalse

\ifsolutions
	\newcommand{\exe}[2]{
		\begin{ex} #1  \end{ex}
		\begin{sol} #2 \end{sol}
	}
\else
	\newcommand{\exe}[2]{
		\begin{ex} #1  \end{ex}
	}
	
\fi


% tableaux var, signe
\usepackage{tkz-tab}


%pinfty minfty
\newcommand{\pinfty}{{+}\infty}
\newcommand{\minfty}{{-}\infty}

\begin{document}


\AdvanceDate[0]

\begin{document}
\pagestyle{fancy}
\fancyhead[L]{Seconde 13}
\fancyhead[C]{\textbf{Systèmes d'équations linéaires : approfondissements \ifsolutions \\ Solutions  \fi}}
\fancyhead[R]{\today}

\exe{
	On considère le système de trois équations à trois inconnues $x, y, z \in \R$ suivant.
		\[ \begin{cases} -x-2y+2z &= -2, \\ 3x - 5y + 2z &= 5, \\ 2x - 2y + 2z &= 10. \end{cases} \]
	\begin{enumerate}
		\item Ajouter aux équations 2 et 3 un multiple de la première équation pour annuler leur coefficient en $x$.
		\item Écrire le système équivalent obtenu et vérifier que les deux dernières équations forment le sous-système de deux équations à deux inconnues suivant.
		\[ \begin{cases} - 11y + 8z &= -1, \\ - 6y + 6z &= 6. \end{cases} \]
		\item Résoudre ce sous-système pour trouver $y=3$ et $z=4$.
		\item En déduire que $x=4$.
		\item Vérifier que $(x ; y ; z) = (4 ; 3 ;4)$ est bien solution du système initial.
	\end{enumerate}
}{}

% ils vont pas le faire pfff
%\exe{
%	On considère le système de quatre équations à quatre inconnues réelles, $x, y, z,$ et $t$.
%		\[ \begin{cases} x+2y-2z &= 2, \\ 3x - 5y + 2z &= 5, \\ 2x - 2y + 2z &= 10. \end{cases} \]
%	\begin{enumerate}
%		\item Comme à l'exercice précédent, annuler tous les coefficients en $x$ des trois dernières équations.
%		\item Écrire le sous-système de trois équations à trois inconnues $y, z,$ et $t$.
%		\item Résoudre le système comme à l'exercice précédent pour trouver $y, z,$ et $t$.
%		\item Trouver $x$.
%		\item Vérifier que les valeurs obtenues
%	\end{enumerate}
%}{}

\exe{
	Donner un système le plus simple possible de deux équations linéaires à inconnues $x, y \in \R$ qui admet uniquement $(x ; y) = (2 ; 5)$ comme solution.
	Modifier le système pour qu'aucun coefficient multipliant $x$ ou $y$ ne soit nul.
}{}

\exe{
	En sommant les équations, montrer que le système suivant d'inconnues $x, y \in \R$ admet exactement quatre solutions : $(x ;y) = (\pm 2 ; \pm 3)$. \footnote{On rappelle la notation $\pm$ qui signifie \og plus ou moins \fg. Par exemple, $\pm3 = 3$ ou $-3$.}
	\[ \begin{cases} x^2 - y^2 &= -5, \\ x^2 + y^2 &= 13. \end{cases} \]

	Donner un système de deux équations non linéaires à inconnues $x, y \in \R$ admettant exactement quatre solutions : $(\pm 3; \pm4)$.
}{}

\exe{\label{ex:5}
	Montrer que, pour tout $y\in\R$,
		\[ (y+1)^2 + y^2 - 13 = 2(y-2)(y+3). \]

	En combinant les équations, montrer que le système suivant d'inconnues $x, y \in \R$ admet exactement trois solutions : $(x ;y) = (0; 2),$ et $(x ;y) = \left( \pm \dfrac12 \sqrt{10} ; -3\right)$.
	\[ \begin{cases} 2x^2 + (y+1)^2 &= 9, \\ 2x^2 - y^2 &=-4. \end{cases} \]
}{}

\newpage

\exe{[$\star$]
	On considère le système suivant qu'on ne cherche pas à résoudre.
		\[ \begin{cases} 3x - 2y &= 7, \\ -x - 3y &= -10. \end{cases} \]
	Supposons qu'on ait deux paires de solutions : $(x ; y)$ et $(x' ; y')$.
	
	Montrer que la différence $(a; b) = (x - x' ; y - y')$ est solution du système 
		\[ \begin{cases} 3a - 2b &= 0, \\ -a - 3b &= 0. \end{cases} \]
	On appelle ce système \emph{homogène} car les coefficients constants à droite sont tous nuls.
	
	Montrer que $(a ; b) = (0 ; 0)$, et en déduire qu'on a nécessairement $(x ; y) = (x' ; y')$ : le système initial admet au plus une solution.
	
	Montrer plus généralement que n'importe quel système linéaire admet au plus une solution dès que l'unique solution du système homogène est la solution nulle.
}{}


\def\arraystretch{1} % for pmatrix
\exe{
	On étudie le système homogène suivant qu'on ne cherche pas à résoudre.
		\[ \systeme{12x - 5y = 0{,}, -7x + 2y = 0.} \]
	\begin{enumerate}
		\item 
		Montrer que le système est équivalent à l'égalité de vecteurs
			\[ x \cdot u + y \cdot v = \pvec{0}{0}, \]
		pour les deux vecteurs $u = \pvec{12}{-7}$ et $v = \pvec{-5}{2}$.
	
		\item
		Montrer qu'une solution $(x ;y)$ non nulle existe si et seulement si $u$ et $v$ sont colinéaires, et donc si et seulement si $\det(u, v) = 0$.
		
		\item	
		Déduire dans ce cas que l'unique solution du système est la solution nulle $(x ;y) = (0;0)$.
	\end{enumerate}
}{}

\exe{[$\star$]
	
		
	\begin{enumerate}
		\item
		Si $u$ et $v$ sont colinéaires, et $w$ un troisième vecteur quelconque, montrer que le système $x \cdot u + y \cdot v = w$ admet une solution si et seulement si $w$ est colinéaire à $u$ et $v$.

		\item
		À partir des vecteurs colinéaires $u = \pvec23$ et $v = \pvec{-6}{-9}$ et d'un vecteur $w$ à poser, construire un système linéaire $x \cdot u + y \cdot v = w$ n'ayant aucune solution.
		\item
		À partir des vecteurs colinéaires $u = \pvec1{-4}$ et $v = \pvec{-2}{8}$ et d'un vecteur $w$ à poser, construire un système linéaire $x \cdot u + y \cdot v = w$ ayant un nombre infini de solutions.
		

	\end{enumerate}
}{}


\end{document}

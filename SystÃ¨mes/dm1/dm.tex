%% INPUT PREAMBLE.TEX
%% THEN INPUT VARS_{i}.ADR
%% THEN RUN THIS
%% DYSLEXIA SWITCH
\newif\ifdys
		
				% ENABLE or DISABLE font change
				% use XeLaTeX if true
				\dystrue
				\dysfalse


\ifdys

\documentclass[a4paper, 14pt]{extarticle}
\usepackage{amsmath,amsfonts,amsthm,amssymb,mathtools}

\tracinglostchars=3 % Report an error if a font does not have a symbol.
\usepackage{fontspec}
\usepackage{unicode-math}
\defaultfontfeatures{ Ligatures=TeX,
                      Scale=MatchUppercase }

\setmainfont{OpenDyslexic}[Scale=1.0]
\setmathfont{Fira Math} % Or maybe try KPMath-Sans?
\setmathfont{OpenDyslexic Italic}[range=it/{Latin,latin}]
\setmathfont{OpenDyslexic}[range=up/{Latin,latin,num}]

\else

\documentclass[a4paper, 12pt]{extarticle}

\usepackage[utf8x]{inputenc}
%fonts
\usepackage{amsmath,amsfonts,amsthm,amssymb,mathtools}
% comment below to default to computer modern
\usepackage{libertinus,libertinust1math}

\fi


\usepackage[french]{babel}
\usepackage[
a4paper,
margin=2cm,
nomarginpar,% We don't want any margin paragraphs
]{geometry}
\usepackage{icomma}

\usepackage{fancyhdr}
\usepackage{array}
\usepackage{hyperref}

\usepackage{multicol, enumerate}
\newcolumntype{P}[1]{>{\centering\arraybackslash}p{#1}}


\usepackage{stackengine}
\newcommand\xrowht[2][0]{\addstackgap[.5\dimexpr#2\relax]{\vphantom{#1}}}

% theorems

\theoremstyle{plain}
\newtheorem{theorem}{Th\'eor\`eme}
\newtheorem*{sol}{Solution}
\theoremstyle{definition}
\newtheorem{ex}{Exercice}
\newtheorem*{rpl}{Rappel}
\newtheorem{enigme}{Énigme}

% corps
\usepackage{calrsfs}
\newcommand{\C}{\mathcal{C}}
\newcommand{\R}{\mathbb{R}}
\newcommand{\Rnn}{\mathbb{R}^{2n}}
\newcommand{\Z}{\mathbb{Z}}
\newcommand{\N}{\mathbb{N}}
\newcommand{\Q}{\mathbb{Q}}

% variance
\newcommand{\Var}[1]{\text{Var}(#1)}

% domain
\newcommand{\D}{\mathcal{D}}


% date
\usepackage{advdate}
\AdvanceDate[0]


% plots
\usepackage{pgfplots}

% table line break
\usepackage{makecell}
%tablestuff
\def\arraystretch{2}
\setlength\tabcolsep{15pt}

%subfigures
\usepackage{subcaption}

\definecolor{myg}{RGB}{56, 140, 70}
\definecolor{myb}{RGB}{45, 111, 177}
\definecolor{myr}{RGB}{199, 68, 64}

% fake sections with no title to move around the merged pdf
\newcommand{\fakesection}[1]{%
  \par\refstepcounter{section}% Increase section counter
  \sectionmark{#1}% Add section mark (header)
  \addcontentsline{toc}{section}{\protect\numberline{\thesection}#1}% Add section to ToC
  % Add more content here, if needed.
}


% SOLUTION SWITCH
\newif\ifsolutions
				\solutionstrue
				%\solutionsfalse

\ifsolutions
	\newcommand{\exe}[2]{
		\begin{ex} #1  \end{ex}
		\begin{sol} #2 \end{sol}
	}
\else
	\newcommand{\exe}[2]{
		\begin{ex} #1  \end{ex}
	}
	
\fi


% tableaux var, signe
\usepackage{tkz-tab}


%pinfty minfty
\newcommand{\pinfty}{{+}\infty}
\newcommand{\minfty}{{-}\infty}

\begin{document}

%\input{adr/vars_44284.adr}
%\newcommand{\seed}{TEST}

\pagestyle{fancy}
\fancyhead[L]{Seconde 13}
\fancyhead[C]{\textbf{Devoir Maison 5 --- \seed \ifsolutions \, --- Solutions  \fi}}
\fancyhead[R]{\today}

\fakesection{Devoir \seed}

\exe{
	On considère le système de deux équations à deux inconnues $x, y \in \R$ suivant.

	\[\systeme[xy]{
		\Aa x \Ab y = \ba{,},
		\Ac x \Ad y = \bb.
	}\]
	
	%\[ (x ; y) = (\x ; \y). \]

	\begin{enumerate}
		\item Trouver la solution exacte $(x;y)$ du système en explicitant sa démarche.
		\item Vérifier que le couple $(x;y)$ trouvé est bien solution du système.
	\end{enumerate}
}{}

\exe{
	
	On considère le système de trois équations à trois inconnues $x, y, z \in \R$ suivant.

	\[\systeme{
		\AaII x \AbII y \AcII z = \baII{,}, 
		\AdII x \AeII y \AfII z = \bbII{,}, 
		\AgII x \AhII y \AiII z = \bcII.
	}\]
	
	%\[ (x ; y; z) = (\xII ; \yII ; \zII). \]
	
	\begin{enumerate}
		\item Ajouter aux équations 2 et 3 un multiple de la première équation pour annuler leur coefficient en $x$.
		\item Écrire le système équivalent obtenu et vérifier que les deux dernières équations forment le sous-système de deux équations à deux inconnues suivant.
			\[\systeme[yz]{
				\AaIII y \AbIII z = \baIII{,},
				\AcIII y \AdIII z = \bbIII.
			}\]
		\item Résoudre ce sous-système pour trouver $y$ et $z$.
		\item Trouver $x$.
		\item Vérifier que le triplet $(x;y;z)$ trouvé est bien solution du système initial.
	\end{enumerate}

}{}

\exe{
	On considère le système d'inconnues $x, y\in\R$ suivant qu'on ne cherche pas à résoudre.
		\[ \systeme{
			\AaIV x \AbIV y = \baIV {,},
			\AcIV x \AdIV y = \bbIV.
		}\]
	\begin{enumerate}
		\item
		Montrer qu'une solution réelle $(x;y)$ du système
		vérifie l'égalité vectorielle
			\[ x \cdot u + y\cdot v = w, \]
		où 
			\begin{align*}
				u = \pvec{\AaV}{\AcV}, && v=\pvec{\AbV}{\AdV}, && \text{ et } && w=\pvec{\baV}{\bbV}.
			\end{align*}
		\item
		Montrer que $u$ et $v$ sont colinéaires en donnant le nombre $\lambda\in\R$ tel que $v=\lambda\cdot u$.
		\item 
		Montrer que si $(x;y)$ est une solution quelconque du système, alors $w$ est nécessairement colinéaire à $u$.
		\item 
		Montrer que $u$ et $w$ ne sont pas colinéaires et en déduire qu'il n'existe aucune solution au système.
	\end{enumerate}

}{}


\end{document}

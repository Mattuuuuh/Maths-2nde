\documentclass[tikz]{standalone}

% tikz
\usepackage{tikz, pgfplots}
% i wish external worked but idk it sucks
%\usetikzlibrary{external}
%\tikzexternalize[prefix=figures/]

% for function graph
\usetikzlibrary{positioning}
\usetikzlibrary{shapes.geometric}
\usetikzlibrary{positioning}
\usetikzlibrary{shapes.misc}
\tikzset{
dot/.style = {circle, fill=#1, minimum size=5pt,
              inner sep=0pt, outer sep=0pt},
dot/.default = black % size of the circle diameter
}
\tikzset{cross/.style={cross out, draw=black, minimum size=2*(#1-\pgflinewidth), inner sep=0pt, outer sep=0pt},
%default radius will be 1pt. 
cross/.default={1pt}}

 % for braces
\usetikzlibrary{decorations.pathreplacing}
% for hashing area
\usetikzlibrary{patterns}
% tableaux var, signe
% source https://www.sqlpac.com/fr/documents/latex-package-tkz-tab-tikz-tableaux-de-signes-et-de-variations-de-fonctions.html
\usepackage{tkz-tab}
\input{../colors}
% Schwartz
\renewcommand{\S}{\mathcal{S}} % \S est le signe paragraphe normalement

% corps
\newcommand{\C}{\mathcal{C}}
\newcommand{\R}{\mathbb{R}}
\newcommand{\Rnn}{\mathbb{R}^{2n}}
\newcommand{\Z}{\mathbb{Z}}
\newcommand{\N}{\mathbb{N}}
\newcommand{\Q}{\mathbb{Q}}

% domain
\newcommand{\D}{\mathcal{D}}

% order notations
\renewcommand{\O}{\mathcal{O}}

% japanese bracket
\newcommand{\japb}[1]{\langle #1 \rangle}

% arrows over partial derivatives
\newcommand{\lp}{\overleftarrow{\partial}}
\newcommand{\rp}{\overrightarrow{\partial}}

% quantization
\newcommand{\h}{\hbar}
\newcommand{\Opht}{\textrm{Op}_{\h}^{t}}
\newcommand{\Op}[2][\hbar]{\textrm{Op}_{#1}^{#2}}

% omega functions
\newcommand{\omegap}[2][\rho_0]{\omega(\partial_{#1},\partial_{#2})}
\newcommand{\omegar}[2][\rho_0]{\omega(#1,#2)}
% PLUS INFTY AND MINUS INFTY WITH NO SPACE
\newcommand{\pinfty}{{+}\infty}
\newcommand{\minfty}{{-}\infty}

\begin{document}
%
	% page 1
  \begin{tikzpicture}[scale=1]
    \begin{axis}[
        ymin=0, ymax=8,
        %minor y tick num = 3,
        xtick = {2, 4, ..., 16, 18},
        area style,
        xlabel = {Note sur $20$},
        ylabel = {Effectif}
      ]
      \addplot+[ybar interval,mark=no, draw=GREEN_E, thick, fill=GREEN_A] plot coordinates {
        (2,1) (3,0) (4,0) (5,1) (6,1) (7,3) (8,1) (9,5) (10,1) (11,0) (12,3) (13,4) (14,3) (15,3) (16,5) (17,1) (18,2) (19,0)
      };
    \end{axis} 
  \end{tikzpicture}
  % page 2
  \begin{tikzpicture}[scale=1]
    \begin{axis}[
        ymin=0, ymax=8,
        %minor y tick num = 3,
        %xtick = {4, 5, ..., 18, 19},
        xtick = {2, 4, ..., 16, 18},
        area style,
        xlabel = {Note sur $20$},
        ylabel = {Effectif}
      ]
      \addplot+[ybar interval,mark=no, draw=BLUE_E, thick, fill=BLUE_A] plot coordinates {
        (4,1) (5,0) (6,1) (7,0) (8,2) (9,1) (10,4) (11,2) (12,7) (13,7) (14,3) (15,0) (16,2) (17,2) (18,1) (19,0)
      };
    \end{axis} 
  \end{tikzpicture}
  % page 3
  \begin{tikzpicture}[scale=1]
    \begin{axis}[
        ymin=0, ymax=8,
        %minor y tick num = 3,
        xtick = {2, 4, ..., 16, 18},
        area style,
        xlabel = {Note sur $20$},
        ylabel = {Effectif}
      ]
      \addplot+[ybar interval,mark=no, draw=RED_E, thick, fill=RED_A] plot coordinates {
      		(2,1) (3,0) (8,2) (9,0) (10, 1) (11,4) (12,1) (13,4) (14,4) (15,7) (16,7) (17,1) (18,2) (19,0)
      };
    \end{axis} 
  \end{tikzpicture}
  % page 4
  % j'aime pas les pie chart c'est inutile
\def\angle{0}
\def\radius{3}
\def\cyclelist{{"myg","gray","myr","myb"}}
\newcount\cyclecount \cyclecount=-1
\newcount\ind \ind=-1
  \begin{tikzpicture}
      \foreach \percent/\name in {
        75/{préoccupation mineure (LC)},
        6/{données insufficantes (DD)},
        14/{éteintes ou menacées (EX à VU)},
        5/{quasi menacées (NT)}
    } {
      \ifx\percent\empty\else               % If \percent is empty, do nothing
        \global\advance\cyclecount by 1     % Advance cyclecount
        \global\advance\ind by 1            % Advance list index
        \ifnum3<\cyclecount                 % If cyclecount is larger than list
          \global\cyclecount=0              %   reset cyclecount and
          \global\ind=0                     %   reset list index
        \fi
        \pgfmathparse{\cyclelist[\the\ind]} % Get color from cycle list
        \edef\color{\pgfmathresult}         %   and store as \color
        % Draw angle and set labels
        \draw[fill={\color!50},draw={\color}] (0,0) -- (\angle:\radius)
          arc (\angle:\angle+\percent*3.6:\radius) -- cycle;
        \node at (\angle+0.5*\percent*3.6:0.7*\radius) {\percent\,\%};
        \node[pin=\angle+0.5*\percent*3.6:\name]
          at (\angle+0.5*\percent*3.6:\radius) {};
        \pgfmathparse{\angle+\percent*3.6}  % Advance angle
        \xdef\angle{\pgfmathresult}         %   and store in \angle
      \fi
    };
  \end{tikzpicture}
\end{document}
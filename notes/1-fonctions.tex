%!TEX encoding = UTF8
%!TEX root = 0-notes.tex

\chapter{Fonctions}
\label{chap:fonctions}

%\section{Fonction comme boite noire}
%
%\section{Définition algébrique d'une fonction}
%
%\section{Courbe représentative d'une fonction}
%
%\section{Implémentation}


\section{Introduction}

\dfn{fonction, variable, constante}{
	On dit qu'une quantité $y\in\R$ s'exprime en \emphindex{fonction} d'une quantité $x\in\R$ lorsque, à chaque nombre $x$ possible, on peut associer \underline{une seule} valeur $y$.
	On note alors $x \mapsto y$ et $x$ est appelé la \emphindex{variable}.
	Si une quantité ne dépend pas de $x$, on l'appelle \emphindex{constante}, ou \emphindex{indépendante} de $x$.
}{}

\nt{
	Pour faire signifier la dépendance de $y$ comme fonction de $x$, on utilise la notation fonctionnelle $y(x)$, lu \og $y$ de $x$ \fg.
	Ainsi, lorsque $x$ vaut $2$, la valeur associée sera $y(2)$, lu \og $y$ de $2$ \fg.
	Ceci permet de différencier les valeurs qui émanent de différents $x\in\R$ : $y(1), y(-3), y(\frac34), \dots$.
	La notation $y(x)$ est remplacée par $f(x)$ ($f$ pour fonction) pour se libérer la lettre $y$.
}

\ex{}{
	Le périmètre d'un cercle de rayon $R$ est donné par $2\pi \cdot R$.
	Un rayon détermine donc un unique périmètre, et la fonction périmètre, dépendant de la variable rayon, est donnée par
		\[ R\mapsto 2\pi\cdot R. \]
	En appelant $P$ cette fonction, on peut aussi noter 
		\[ P(R) = 2\pi \cdot R. \]	
}{ex:perimètre}

\exe{1}{
	Un lynx prend la fuite : il court à 15 mètres par seconde.
	\begin{enumerate}
		\item La distance parcourue par le lynx est-elle une fonction du temps écoulé depuis son départ ?
		\item Le temps écoulé depuis le départ du lynx est-il une fonction de la distance parcourue ?
	\end{enumerate}
}{exe:lynx}{
	TODO
}

\exe{}{
	Un étudiant jette une balle dans les airs et mesure la hauteur de la balle tous les quarts de seconde.
	Il note ses résultats dans le tableau ci-dessous.
	\begin{center}
		\begin{tabular}{|c|c|c|c|c|c|c|c|}\hline
			Hauteur (cm) & 85 & 145 & 190 & 145 & 85 & 40 & 0 \\ \hline
			Temps (s) & 0 & 0,25 & 0,5 & 0,75 & 1 & 1,25 & 1,5 \\\hline
		\end{tabular}
	\end{center}
	
	\begin{enumerate}
		\item La hauteur est-elle une fonction du temps ? Justifier.
		\item Le temps est-il une fonction de la hauteur ? Justifier.
	\end{enumerate}
}{exe:table}{
	TODO
}

\exe{, difficulty=1}{
	\begin{enumerate}
		\item
		Montrer que le rayon d'un cercle est fonction de son périmètre et écrire la fonction associée.
		\item
		Montrer que l'aire d'un cercle est fonction de son rayon et écrire la fonction associée.
		\item
		En déduire que l'aire d'un cercle est fonction de son périmètre et écrire la fonction associée.
	\end{enumerate}
}{exe:rayon-périmètre-aire}{
	TODO
}

\dfn{fonction réelle, domaine}{
	Soit $\D \subset \R$ un ensemble de nombres qu'on appelle \emphindex{domaine}.
	
	Une fonction $f$ de $\D$ dans $\R$ associe à chaque élément $x\in\D$ du domaine un nombre $f(x) \in \R$.
	
	On note alors
		\begin{align*}
			f: \D & \longrightarrow \R \\
			x& \longmapsto f(x).
		\end{align*}
	
	\begin{center}
	\includegraphics[page=1]{figures/fig-fonctions.pdf}
	\end{center}
}{}

\ex{}{
	Une constante est en particulier une fonction d'une variable $x\in\R$ réelle.
	Ainsi la fonction $g$ donnée par
		\begin{align*}
		g: \R & \longrightarrow \R \\
		x& \longmapsto 42
		\end{align*}
	vaut constamment $42$, peut importe la valeur de $x$. 
	
	On voit dans l'expression $g(x) = 42$ que l'image de $x$ ne dépend aucunement de la valeur de $x$ :
	$g$ est une \emphindex{fonction constante}.
}{ex:fonction-constante}

\exe{}{
	On considère la fonction $f$ qui à chaque triangle associe la somme de ses angles.
	Que vaut $f(T)$ pour tout triangle $T$ ? Que dire de $f$ ?
}{exe:somme-angles}{
	Dans le plan, la somme des angles d'un triangle vaut toujours 180°.
	Donc $f(T) = 180^\circ$ pour tout triangle $T$ : c'est une fonction constante.
}

\exe{, difficulty=2}{
	On considère la fonction $f$ qui à chaque triangle associe l'angle le plus grand de $T$.
	Montrer que $f(T) \leq 180^\circ$ pour n'importe quel triangle $T$.
	Quel est le minimum de $f$ ? Donner un triangle $T$ tel que $f(T)$ est minimal.
}{exe:max-angles}{
	L'angle d'un triangle $T$ est au plus l'angle plat, de valeur 180°. Son angle le plus grand, $f(T)$, est donc aussi borné supérieurement par 180°.
	
	Remarquons que $f(T) \geq 60^\circ$ pour tout triangle $T$, car si tous les angles d'un triangle sont inférieurs à 60°, alors leur somme ne peut pas être 180°.
	Le triangle équilatéral vérifie $f(T) = 60^\circ$, le minimum de $f$.
}
	

\dfn{antécédent, image}{
	Considérons une fonction $f$ et deux nombres réels $x, y\in\R$ vérifiants
		\[ y = f(x). \]
	On lit \og $y$ égal $f$ de $x$ \fg, et on dit alors que
		\begin{enumerate}
			\item
			$y$ est l'\emphindex{image} de $x$ par $f$ ; et
			\item
			$x$ est \underline{un} \emphindex{antécédent} de $y$ par $f$.
		\end{enumerate}
}{}

\nt{
	\begin{enumerate}
		\item 
		Chaque $x\in\D$, élément du domaine, a \underline{exactement une} image par $f$ qui est $f(x)$.
		\item
		Un nombre $y\in\R$ peut avoir \underline{un}, \underline{aucun}, ou même \underline{plusieurs} antécédents. C'est le cas par exemple de $y=0, -1, 1$ pour la fonction carré $f(x) = x^2$.
	\end{enumerate}
}

\dfn{forme algébrique}{
	On appelle 
		\[ f(x) = 3\cdot x+1 \]
	la forme \emphindex{algébrique} de $f$.
	On lit dans ce cas \og $f$ de $x$ égal trois $x$ plus 1 \fg.
	
	C'est une forme entièrement générale qui permet de déduire l'image $f(x)$ à partir de n'importe quel réel $x\in\D$ du domaine de $f$.
}{}

\exe{}{
	On reprend l'exercice \ref{exe:lynx}, dans lequel un lynx court à 15 mètres par seconde.
	Donner la forme algébrique de la fonction $f$ donnant la distance parcourue (mètres) au temps $t$ (secondes).
	Quel est le domaine de $f$ ?
}{exe:lynx2}{
	La distance est la vitesse multipliée par le temps : $f(t) = 15\cdot t = 15t$.
	Comme $f$ prend un temps, son domaine est tous les nombres réels positifs ou nuls.
}

\nt{
	Suite à l'exercice \ref{exe:lynx2} ci-dessus, on dira alors qu'on a écrit la distance \emphindex{en fonction} du temps $t$.
	L'expression algébrique permet de calculer facilement plusieurs images sans refaire le raisonnement « distance = vitesse fois temps » à chaque fois.
}

\notations{
	Pour n'importe quel nombre $x$ on dénote $x^2 = x \cdot x$ le produit de $x$ par lui-même.
	On lit « $x$ au carré » ou « $x$ carré ».
}

\ex{}{
	Considérons la \emphindex{fonction carré} $f(x) = x^2$ pour tout $x\in\R$.
	Calculons les images de $1; 4; -1; -0,31; \frac23 ; -\frac23;$ et $0$ par $f$.
		\begin{multicols}{3}
		\begin{enumerate}%[leftmargin=50pt]
			\item $f(1) = 1$
			\item $f(-1) = 1$
			\item $f( -0,31) = 0,0961$
			\item $f\left(\dfrac23\right) = \dfrac49$
			\item $f\left( -\dfrac23\right) = \dfrac49$
			\item $f(0) = 0$
		\end{enumerate}
		\end{multicols}
	\noindent
	On en déduit les propositions suivantes.
		\begin{multicols}{2}
		\begin{enumerate}[label=--]%, leftmargin=50pt]
			\item
			L'image de $1$ par $f$ est $1$.
			\item
			L'image de $-1$ par $f$ est $1$.
			\item
			$1$ et $-1$ sont deux antécédents de $1$ par $f$.
			\item
			Un antécédent de $0,0961$ par $f$ est $-0,31$.
			\item
			Un antécédent de $\dfrac49$ par $f$ est $\dfrac23$.
			\item
			Un antécédent de $\dfrac49$ par $f$ est $-\dfrac23$.
			\item
			$0$ est l'image et un antécédent de $0$ par $f$. %C'est d'ailleurs le seul antécédent.
		\end{enumerate}
		\end{multicols}
}{ex:carré0}

\exe{}{
	Considérons la fonction $f(x) = 3\cdot x + 1$ pour tout $x\in\R$.
	
	\begin{enumerate}
		\item
		Calculer l'image par $f$ de $0$ ; de $3,1$ ; de $\frac13$ ; de $-1$ ; de $-\frac23$.
		\item
		Donner un antécédent de $1$ par $f$.
		\item
		Déterminer tous les antécédents de $6$ par $f$.	
	\end{enumerate}
}{exe:images-antécédents}{
	TODO
}

\exe{}{
	Un fonction $f$ définie sur tout $\R$ admet le tableau de valeurs suivant.
		\begin{center}
		\begin{tabular}{|c|c|c|c|c|}\hline
			$x$ & 0 & -2 & 1 & -1 \\ \hline
			$f(x)$ & 1 & 0 & 0 & 1 \\ \hline
		\end{tabular}
		\end{center}
	Parmis les expressions algébriques suivantes, lesquelles \underline{ne peuvent pas} correspondre à $f(x)$ ?
		\begin{multicols}{3}
		\begin{enumerate}[label=\roman*)]
			\item $1-x$
			\item $1+\dfrac{x}2$
			\item $\dfrac{1-x}2$
			\item $\dfrac{-x^2 - x + 2}2$
			\item $\dfrac{x^3 - 3x + 2}2$
			\item $\dfrac{-x^4 - 2x^3 + x + 2}2$
		\end{enumerate}
		\end{multicols}
}{exe:fonctions-QCM}{
	TODO
}

\exe{}{
	Montrer que $g(x) = x(x+2)(x-1)(x+1)$ vérifie $g(0) = g(-2) = g(1) = g(-1) = 0$.
	Conclure que $f(x)$ et $h(x) = f(x) + g(x)$ vérifient toutes les deux le tableau de valeurs de l'exercice \ref{exe:fonctions-QCM}.
}{exe:fonctions-QCM2}{
	Lors du calcul de chacune des images, au moins un facteur est nul : le produit vaut donc toujours zéro.
}

\exe{, difficulty=1}{
	Montrer qu'il y a une infinité de fonctions $h$ différentes vérifiant le tableau de valeurs de l'exercice \ref{exe:fonctions-QCM}-
}{exe:fonctions-QCM3}{
	On prend par exemple $h(x) = f(x) + g(x), f(x) + 2g(x), f(x) + 3g(x), \dots$.
}

\nt{
	Lorsqu'on a un nombre fini d'images, on peut invalider un candidat d'expression algébrique mais jamais le valider pour tout $x\in\R$.
}

\exe{}{
	Considérons la fonction $f(x) = \frac15-x$ pour tout $x\in\R$.
	
	\begin{enumerate}
		\item
		Calculer l'image par $f$ de $0$ ; de $0,2$ ; de $\frac47$ ; de $-\frac23$.
		\item
		Donner un antécédent de $0$ par $f$.
		\item
		Déterminer tous les antécédents de $5$ par $f$.	
	\end{enumerate}
}{exe:images-antécédents2}{
	TODO
}

\exe{}{
	Considérons les fonctions $f(x) =x^2 -6\cdot x + 9$ et $g(x)=(x-3)^2$ pour tout $x\in\R$.
	%\begin{multicols}{2}
	\begin{enumerate}
		\item
		Calculer les images par $f$ et $g$ de $0$ ; de $6$ ; de $3$ ; de $-\frac23$ ; de $\frac{20}3$.
		\item
		Donner un antécédent de $9$ par $f$.
		\item
		Déterminer tous les antécédents de $9$ par $g$.	
		\item
		Montrer que $f(x) = g(x)$ pour tout $x\in\R$ réel.
		\item 
		En déduire tous les antécédents de $9$ par $f$.
	\end{enumerate}
	%\end{multicols}
}{exe:forme-canonique}{
	TODO
}

\section{Intervalles}

\dfn{intervalle borné}{
	Un \emphindex{intervalle} borné est un segment de la droite réelle $\R$. C'est donc un ensemble de nombres.
	Il est donné par une \emphindex{borne inférieure} $a \in \R$ et une \emphindex{borne supérieure} $b\in\R$ et peut contenir ou non ses bornes.
	
	\begin{enumerate}
		\item Si $a$ et $b$ sont contenues dans l'intervalle, on le note $[a ; b]$.
		\item Si $a$  est contenue dans l'intervalle mais $b$ ne l'est pas, on le note $[a ; b [$.
		\item Si $a$ n'est pas contenue dans l'intervalle mais $b$ l'est, on le note $] a ; b]$.
		\item Si ni $a$ ni $b$ ne sont contenues dans l'intervalle, on le note $] a ; b [$.
	\end{enumerate}
}{}

\ex{}{
	Les intervalles suivants sont bornés.
	\begin{multicols}{2}
	\begin{enumerate}[label=$\bullet$]
		\item $[-1 ; 1]$
		\item $[-3 ; 1[$
		\item $\left]-10{,}341 ; \pi\right]$
		\item $\left]\sqrt{2} ; 130\right[$
	\end{enumerate}
	\end{multicols}

}{}

\dfn{intervalle non borné}{
	Un intervalle n'est pas forcément borné : une ou les deux bornes peuvent être infinies ($\pinfty$ ou $\minfty$). 
	Dans ce cas, l'intervalle n'inclut jamais l'infini car ce n'est pas un nombre.
}{}

\ex{}{
	Les intervalles suivants ne sont pas bornés.
	\begin{multicols}{2}
	\begin{enumerate}[label=$\bullet$]
		\item $]\minfty; 2]$
		\item $]\minfty ; 3[$
		\item $]0; \pinfty[$
		\item $\R = ]\minfty; \pinfty[$
	\end{enumerate}
	\end{multicols}
}{ex:3.5}

\nt{
	Un intervalle n'a de sens que si la borne inférieure est plus petite que la borne supérieure.
	De plus, un élément appartient à l'ensemble dès qu'il est plus petit que la borne supérieure, et plus grand que la borne inférieure.
	Il faut donc pouvoir noter simplement les relations \og plus petit que \fg et \og plus grand que \fg.
}

\notations{
	On définit les signes suivants correspondant à des inégalités \emphindex{strictes} et \emphindex{larges}.
	\begin{multicols}{2}
	\begin{enumerate}[leftmargin=50pt]
		\item[$<$ :] strictement inférieur à
		\item[$\leq$ :] inférieur ou égal à
		\item[$>$ :] strictement supérieur à
		\item[$\geq$ :] supérieur ou égal à
	\end{enumerate}
	\end{multicols}
	On attire l'attention sur le fait que le « ou » est inclusif : $x \leq x$ est toujours vrai. 
}

\ex{}{
	Les inégalités suivantes sont vraies. 
	\begin{multicols}{3}
	\begin{enumerate}[label=\roman*)]
		\item $1 \leq 2$
		\item $-3 \leq -2$
		\item $0 \leq 0$
		\item $1{,}02 < 1{,}1$
		\item $7{,}391 > 7{,}30001$
		\item $-4{,}001 > -4{,}0001$
	\end{enumerate}
	\end{multicols}
}{}

\nt{
	Appartenir à un intervalle est équivalent à être inférieur à la borne supérieure, et être supérieur à la borne inférieure.
	On a donc l'équivalence suivante.
		\[ x \in [a ; b] \qquad \iff \qquad x \in \R, x \geq a, \textbf{ et } x \leq b. \]
}

\notations{
	On peut noter deux inégalités sur la même lignes dès qu'elles vont dans le même sens.
	Par exemple, les deux inégalités ci-dessus peuvent être condensées en une :
		\[ x \in \R, \text{ et } a \leq x \leq b, \]
	ou encore
		\[ x \in \R, \text{ et } b \geq x \geq a. \]
}

\ex{}{
	\begin{enumerate}
		\item Prendre $x \in [-3 ; 4]$ est équivalent à prendre $x \in \R$ vérifiant $-3 \leq x \leq 4$.
		\item Prendre $x \in [-4 ; 3[$ est équivalent à prendre $x \in \R$ vérifiant $-4 \leq x < 3$.
		\item Prendre $x \in ]\minfty ; 0[$ est équivalent à prendre $x \in \R$ vérifiant $x < 0$.
		
		On dit alors que $x$ est strictement négatif.
		\item Prendre $x \in [0; \pinfty [$ est équivalent à prendre $x \in \R$ vérifiant $x \geq 0$.
		
		On dit alors que $x$ est positif ou nul.
		\item Prendre $x \in ]\minfty; \pinfty[$ est équivalent à prendre $x \in \R$.	
		
		Ceci est en fait tautologique car $]\minfty; \pinfty[ = \R$, comme vu dans l'exemple \ref{ex:3.5}.
	\end{enumerate}
}{}

\section{Représentation graphique}

\subsection{Courbe représentative}

On souhaite représenter graphiquement une fonction $f$ afin de pouvoir rapidement trouver toutes les images de $f$.
Une bonne représentation graphique nous permet de lire
	%\begin{multicols}{2}
	\begin{enumerate}[label=$\bullet$]
		\item les images $f(x)$;
		\item les antécédents $x$ ;
		%\item les variations de $f$ (croissante, décroissante, constante) ;
		%\item le signe de $f(x)$ (positif, négatif, nul).
	\end{enumerate}
	%\end{multicols}

Pour chaque $x\in\D\subset\R$, élément du domaine de $f$, il faut donc pouvoir facilement lire son image $f(x)$ par $f$.
À cette fin et pour chaque $x\in\D$, on crée un point d'abscisse $x$ et d'ordonnée $f(x)$.
L'ensemble de ces points est la courbe représentative de $f$.
Pour lire l'image de $x$ par $f$, il suffit alors de trouver l'ordonnée de l'unique point de la courbe d'abscisse $x$.

\dfn{courbe représentative}{
	Considérons $f : \D \rightarrow \R$ une fonction.
	
	La \emphindex{courbe représentative} de $f$, notée $\C_f$, est donnée par l'ensemble de points
		\[ \C_f = \Bigset{ \bigpar{x ; f(x)} \text{ où $x$ parcourt } \D }. \]
	On lit \og la courbe représentative de $f$ est l'ensemble des points $\bigl(x,f(x)\bigr)$ du plan où $x$ parcourt le domaine de $f$ \fg.
}{dfn:Cf}	

\nt{
	Connaître $f$ sur son domaine $\D$ c'est connaître $\C_f$, et vice versa.
	Cependant, il n'est possible de dessiner $\C_f$ si le domaine $\D$ est infini, ou si $f$ prend des valeurs toujours plus grandes.
}

\cor{propriété fondamentale}{
	On a donc la propriété fondamentale suivante valable pour tout $x, y\in\R$.
		\begin{align*}
			(x;y) \in \C_f && \iff && y = f(x).
		\end{align*}
		
		\begin{center}
		 \includegraphics[page=2, scale=1.5]{figures/fig-fonctions.pdf}
		\end{center}
	%\emph{Note : $f(x) = \dfrac1{200}(x+3,5)(x+2,5)(x-3,5)(x-4,5)(x-5,5) + 2$, polynôme de degré $5$ dans l'exemple ci-dessus.}
}{cor:prop-fond}

\nomen{
	On appelle \emphindex{graphe} la représentation graphique d'une fonction.
}

\ex{}{
	Soit la fonction $f : [-1,5; 2,5] \rightarrow \R$ donnée algébriquement par
		\[ f(x) = 3\cdot x^3 -x - 3, \]
	et considérons les points $P(0;-3), Q(1; 1), R(-1; 1),$ et $S(2; 19)$.
	On se demande si les points appartiennent à la courbe représentative de $f$ ou non.
	
	D'après la propriété fondamentale \ref{cor:prop-fond}, un point $(x;y)$ appartient à la courbe de $f$ si et seulement si l'équation $y=f(x)$ est vérifiée.
	On a donc les (non) appartenances suivantes.
	\begin{multicols}{2}
		\begin{enumerate}
			\item $f(0) = -3$, et donc $P \in \C_f$.
			\item $f(1) = -1 \neq 1$, donc $Q \not\in \C_f$.
			\item $f(-1) = -5 \neq 1$, donc $R \not\in\C_f$.
			\item $f(2) = 19$, et donc $S\in\C_f$.
		\end{enumerate}
		
		\begin{center}
		 \includegraphics[page=3, scale=1.1]{figures/fig-fonctions.pdf}
		\end{center}
	\end{multicols}
}{}


\exe{}{
	Considérons la fonction $f: \left]{-}\dfrac72 ; \dfrac{11}2 \right[ \rightarrow\R$ donnée algébriquement par
		\[ f(x) = \dfrac17-x. \]
	Pour chaque point suivant, déterminer s'il appartient à $\C_f$ ou non.
	
	\begin{multicols}{2}
	\begin{enumerate}[label=\roman*)]
		\item $\left(0; \dfrac17\right)$
		\item $\left(\dfrac17 ; 0\right)$
		\item $\left(\dfrac27 ; \dfrac37\right)$
		\item $\left(-\dfrac{13}7 ; 2\right)$
		\item $\left(\dfrac67 ; 1\right)$
		\item $\left(\dfrac27 ; -\dfrac17\right)$
	\end{enumerate}
	\end{multicols}

}{exe:Cf}{
	TODO
}

\exe{}{
	Considérons deux fonctions $f, g: ]{-}3 ; 3[ \rightarrow\R$ données algébriquement par
		\begin{align*}
			f(x) = x^2 - 2\cdot x && g(x) = (x-1)^2
		\end{align*}
	
	\begin{enumerate}
		\item Esquisser les représentations graphiques de $f$ et de $g$ dans un même repère.
		\item Démontrer que $g(x) - 1 = f(x)$ pour tout $x$ du domaine.
		\item En déduire que $(x-1)^2 = x^2 - 2\cdot x + 1$ pour tout $x$ du domaine.
	\end{enumerate}
}{exe:id-rem-graph}{
	TODO
}

\exe{}{
	Esquisser la courbe de la fonction $f:[-2; 4]\rightarrow\R$ donnée algébriquement par
		\[ f(x) = 3. \]
	Que dire de $f$ et de $\C_f$ ?
}{exe:graph-const}{
	TODO
}

\exe{}{
	Esquisser la courbe de la fonction $f:[-5;3]\rightarrow\R$ donnée algébriquement par
		\[ f(x) = 1-x. \]
	Que dire $\C_f$ ?
}{exe:graph-droite}{
	TODO
}

\exe{}{
	Esquisser la courbe de la fonction $f:[3;10]\rightarrow\R$ donnée algébriquement par
		\[ f(x) = \dfrac3x + 1. \]
}{exe:graph-droite2}{
	TODO
}


\exe{}{
	Considérons la représentation graphique suivante d'une fonction $f$ définie sur $\D = ]{-}3,4 ; 2,3[$.
	
	\begin{center}
	\includegraphics[page=9]{figures/fig-fonctions.pdf}
	\end{center}
	\begin{enumerate}
		\item Donner approximativement les images de $-1,5$ et de $-\dfrac{20}7$ par $f$.
		\item Énumérer approximativement les antécédents de $-2$ et de $2$ par $f$.
		\item Donner approximativement un réel qui admet exactement deux antécédents par $f$.
		\item Si $f$ était définie sur $\R$ tout entier, serait-il toujours possible de connaître l'image de $-2$ ? Et tous les antécédents de $-2$ ?
	\end{enumerate}
	Supposons désormais que $f(x) = 3-2\cdot x +\dfrac13 \cdot x^3$ pour tout $x\in\D$ du domaine.
	\begin{enumerate}
		\item[5.] Vérifier à la calculatrice les réponses aux deux premières questions.
		\item[6.] Montrer sans calculatrice que l'image par $f$ de $-3$ est $0$ et que l'image par $f$ de $0$ est $3$.
	\end{enumerate}
}{exe:deg3}{
	TODO
}


\exe{}{
	Un fonction $f$ admet une représentation graphique suivante.
		\begin{center}
		 \includegraphics[page=4]{figures/fig-fonctions.pdf}
		\end{center}
	Parmis les expressions algébriques suivantes, trouver celle qui correspond à $f(x)$.
		\begin{multicols}{2}
		\begin{enumerate}[label=\roman*)]
			\item $1-x$
			\item $\dfrac{-1-x}3$
			\item $\left(x+\dfrac13\right)^2$
			\item $-2\cdot x - \dfrac23$
		\end{enumerate}
		\end{multicols}
}{exe:expr-from-graph}{
	TODO
}

\exe{}{
	Comparer les représentations graphiques des fonctions suivantes données algébriquement.
		\begin{align*}
			f(x) = x^2 && g(x) = x^2 - 3 && h(x) = (x+4)^2.
		\end{align*}
}{exe:y-shift}{
	TODO
}

\exe{, difficulty=2}{
	Donner une fonction réelle et un domaine telle qu'une de ses images admet
		\begin{enumerate}
			\item Exactement un antécédent
			\item Exactement deux antécédents
			\item Exactement trois antécédents
			\item Une infinité d'antécédents
		\end{enumerate}	
}{exe:nb-antécédents}{
	TODO
}

\exe{, difficulty=2}{
	Donner graphiquement une fonction sur $\R$ non constante telle que toutes les images de $f$ admettent un nombre infini d'antécédents.
}{exe:infinité-antécédents}{
	TODO
}

\newpage

\subsection{Résolution graphique d'(in)équations}

\subsection*{Résoudre graphiquement $f(x) = k$}


On cherche à résoudre graphiquement une équation du type $f(x)=k$ pour un $k\in\R$ donné, c'est-à-dire trouver l'ensemble des $x\in\R$ pour lesquels l'équation est vérifiée.
En reformulant, il s'agit de trouver l'ensemble des antécédents de $k$ par $f$. 

En reprenant la solution de l'exercice \ref{exe:deg3}, on peut généraliser la stratégie à employer.
Supposons qu'on souhaite résoudre $f(x) = 2$ où $x\in]{-}3,4 ; 2,3[$ et $\C_f$ est donnée graphiquement.
Comme chaque point de $\C_f$ est de la forme $\bigl(x; f(x)\bigr)$, on souhaite trouver les points de la forme $(x;2)$ pour ensuite lire leur abscisse et enfin obtenir les différents $x$ du domaine.
Pour trouver tous les points d'ordonnée $2$, on peut s'aider en créant une droite horizontale d'ordonnée $2$, comme ci-dessous.
	\begin{center}
	\includegraphics[page=8, scale=1.5]{figures/fig-fonctions.pdf}
	\end{center}
Les coordonnées des points $A,B,C$ trouvés en intersectant la droite horizontale avec $\C_f$ sont
	\begin{align*}
		A \approx (-2,67 ; 2) && B \approx (0,52 ; 2) && C \approx (2,14 ; 2)
	\end{align*}
L'ensemble de solutions de l'équation $f(x)=2$ est donc approximativement donné par 
	\[ \bigset{ -2,67 ; 0,52 ; 2,14 }. \]

\thm{}{
	Soit $f:\D \rightarrow \R$ une fonction réelle définie sur son domaine $\D$.
	Pour tout réel $k$, les solutions dans $\D$ de l'équation $f(x)=k$ sont les abscisses des points d'intersection de $\C_f$ et de la droite d'équation $y=k$.
}{}


\exemulticols{}{
	Donner l'ensemble des $x$ vérifiant $f(x) = -2$ à l'aide du graphe de $f$ ci-contre.
}{
	TODO graph
}{exe:fx=k}{
	TODO
}


\newpage
\subsection*{Résoudre graphiquement $f(x) < k$}


On cherche à résoudre graphiquement une équation du type $f(x)\leq k$ pour un $k\in\R$ donné, c'est-à-dire trouver l'ensemble des $x\in\R$ pour lesquels l'inéquation est vérifiée.

On étudie à nouveau la courbe de l'exercice \ref{exe:deg3} pour généraliser la stratégie à employer.
Supposons qu'on souhaite résoudre $f(x) \leq 2$ où $x\in]{-}3,4 ; 2,3[$ et $\C_f$ est donnée graphiquement.
Comme chaque point de $\C_f$ est de la forme $\bigl(x; f(x)\bigr)$, on souhaite trouver les points d'ordonnée inférieure à $2$ pour ensuite lire leur abscisse et enfin obtenir les différents $x$ du domaine.
Pour trouver tous les points d'ordonnée inférieure à $2$, on peut s'aider en créant une droite horizontale d'ordonnée $2$, comme ci-dessous.
	\begin{center}
	 \includegraphics[page=5, scale=1.5]{figures/fig-fonctions.pdf}
	\end{center}
Les points de $\C_f$ d'ordonnée inférieure à $2$ ont leur abscisse appartenant à l'union d'intervalles
	\[ ] {-}3,4 ; -2,67 ] \cup [0,52 ; 2,14]. \]
C'est notre ensemble de solutions de l'équation $f(x) \leq 2$.

\thm{}{
	Soit $f:\D \rightarrow \R$ une fonction réelle définie sur son domaine $\D$.
	Pour tout réel $k$, les solutions dans $\D$ de l'équation $f(x) \leq k$ sont les abscisses des points de $\C_f$ situés en dessous de la droite d'équation $y=k$.
}{}

\exemulticols{}{
	Donner l'ensemble des $x$ vérifiant $f(x) < -2$ à l'aide du graphe de $f$ ci-contre.
}{
	TODO graph
}{exe:fx=k}{
	TODO
}

\newpage
\subsection*{Résoudre graphiquement $f(x) = g(x)$}

On cherche à résoudre graphiquement une équation du type $f(x)=g(x)$, c'est-à-dire trouver l'ensemble des $x\in\R$ pour lesquels l'équation est vérifiée.

Soient $f,g : ]{-}3,4 ; 2,3[ \rightarrow \R$ deux fonctions réelles sur le domaine $]{-}3,4 ; 2,3[$ dont les courbes représentatives sont données ci-dessous.

	\begin{center}
	 \includegraphics[page=6, scale=1.5]{figures/fig-fonctions.pdf}
	\end{center}
	
Imaginons un nombre réel $x\in\R$ parcourant $]{-}3,4 ; 2,3[$ en partant de la gauche et allant vers la droite.
Pour chaque $x$, le point de $\C_f$ d'abscisse $x$ est d'ordonnée $f(x)$, par définition.
Idem pour le point de $\C_g$ d'abscisse $x$ : il est d'ordonnée $g(x)$.

Ainsi quand $x$ traverse le domaine, ces deux points sont confondus si et seulement si ces deux points sont de même ordonnée, c'est-à-dire si et seulement si $f(x) = g(x)$.		
	
Les solutions $x\in]{-}3,4 ; 2,3[$ de l'équation $f(x) = g(x)$ sont donc données approximativement par l'ensemble
	\[ \bigset{ -0,8 ; 1,2 }. \]

\thm{}{
	Soient $f,g$ deux fonctions réelles définies sur un même domaine $\D$.
	Les solutions dans $\D$ de l'équation $f(x) = g(x)$ sont les abscisses des points d'intersection de $\C_f$ et $\C_g$.
}{}

\exemulticols{}{
	Donner l'ensemble des $x$ vérifiant $f(x) = g(x)$ à l'aide des graphes de $f$ et $g$ ci-contre.
}{
	TODO graph
}{exe:fx=k}{
	TODO
}

\newpage
\subsection*{Résoudre graphiquement $f(x) \leq g(x)$}

On cherche à résoudre graphiquement une équation du type $f(x)=g(x)$, c'est-à-dire trouver l'ensemble des $x\in\R$ pour lesquels l'inéquation est vérifiée.

Soient $f,g : ]{-}3,4 ; 2,3[ \rightarrow \R$ deux fonctions réelles sur le domaine $]{-}3,4 ; 2,3[$ dont les courbes représentatives sont données ci-dessous.

\begin{center}
\includegraphics[page=7, scale=1.5]{figures/fig-fonctions.pdf}
\end{center}

Pour chaque $x$ parcourant $]{-}3,4 ; 2,3[$, la hauteur (l'ordonnée) du point de $\C_f$ d'abscisse $x$ est donnée par $f(x)$, et celle du point de $\C_g$ d'abscisse $x$ est donnée par $g(x)$.
Ainsi, le premier point est en dessous du second dès que $f(x) \leq g(x)$.

L'ensemble des solutions de l'inéquation $f(x) \leq g(x)$ est donc donné approximativement par
	\[ [{-}0,8 ; 1,2 ]. \]

\thm{}{
	Soient $f,g$ deux fonctions réelles définies sur un même domaine $\D$.
	Les solutions dans $\D$ de l'équation $f(x) \leq g(x)$ sont les abscisses des points de $\C_f$ situés en dessous de $\C_g$.
}{}


\exemulticols{}{
	Donner l'ensemble des $x$ vérifiant $f(x) \leq g(x)$ à l'aide des graphes de $f$ et $g$ ci-contre.
}{
	TODO graph
}{exe:fx=k}{
	TODO
}


\newpage
\subsection{Fonctions parentes : transformations des ordonnées}

À partir de la courbe d'une fonction qu'on connaît bien, on peut déduire les courbes de toute une famille de fonctions apparentées.

\ex{ajout d'une constante}{
	Considérons $f$ une fonction quelconque sur $\R$ et $g$ définie par
		\[ g(x) = f(x)+1. \]
	Pour calculer $g(0)$, on utilise donc la définition $g(0) = f(0) + 1$.
	De façon identique, $g(-3) = f(-3) + 1$, $g(12) = f(12) + 1$, etc…
	
	D'un point $\bigpar{x; f(x)}$ de la courbe $\C_f$, on peut donc en déduire le point $\bigpar{x ; g(x)} = \bigpar{ x ; f(x)+1 }$ de la courbe $\C_g$.
	Graphiquement, la courbe $\C_f$ est translatée de $1$ unité vers le haut pour obtenir $\C_g$
	
	Pour $h(x) = f(x) - 2$, on obtient les graphes suivants.
	\begin{center}
	\includegraphics[page=10, scale=1.5]{figures/fig-fonctions.pdf}
	\end{center}
}{}

\ex{multiplication par une constante}{
	Considérons $f$ une fonction quelconque sur $\R$ et $g$ définie par
		\[ g(x) = 2f(x). \]
	Pour calculer $g(0)$, on utilise donc la définition $g(0) = 2f(0)$.
	De façon identique, $g(-3) = 2f(-3)$, $g(12) = 2f(12)$, etc…
	
	D'un point $(x ; f(x))$ de $\C_f$, on double son ordonnée (sa hauteur) pour obtenir $(x ; 2f(x)) = (x ; g(x))$, le point de $\C_g$ d'abscisse $x$.
	C'est ce qu'on appelle une homothétie : on agrandit ou réduit l'ordonnée de chaque point d'un même facteur.
	Si ce facteur est négatif, on fait en plus une symétrie par rapport à l'axe des abscisses.
	
	Pour $h(x) = -f(x)$, on obtient les graphes suivants.
	\begin{center}
	\includegraphics[page=11, scale=1.5]{figures/fig-fonctions.pdf}
	\end{center}
}{}

\exe{}{
	Tracer les graphes des fonctions $g(x) = f(x) + 2$ et $h(x) = f(x) - 1$ dans le repère contenant $\C_f$ ci-dessous.
	TODO
}{exe:draw-plusc}{
	TODO
}



%\section{Implémentation}
%
%
%
%
%\exe{}{
%	Lire les deux programmes Python de la figure \ref{python:1-2}.
%	Quelles valeurs impriment-t-ils ?
%	
%	Écrire la fonction $f(x)$ de chaque programme sous forme algébrique.
%}{exe:fonction-python}{
%	TODO
%}
%
%\begin{figure}[!htb]
%	\begin{subfigure}[b]{.45\textwidth}
%\begin{mintedbox}{python}
%def f(x):
%y = 2*x-3
%return y
%
%f1 = f(1)
%f2 = f(-1/2)
%
%
%print(f1, f2)
%\end{mintedbox}
%	\caption{Programme 1.}
%	\label{python:1}
%	\end{subfigure}
%	\begin{subfigure}[b]{.45\textwidth}
%\begin{mintedbox}{python}
%def f(x):
%y = x*x + 1
%z = -2*x
%return y+z
%
%f1 = f(4)
%f2 = f(-2)
%
%print(f1, f2)
%\end{mintedbox}
%	\caption{Programme 2.}
%	\label{python:2}
%	\end{subfigure}
%	\caption{Deux fonctions implémentées en Python.}
%	\label{python:1-2}
%\end{figure}
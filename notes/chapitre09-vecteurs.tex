%!TEX encoding = UTF8
%!TEX root =notes.tex

\chapter{Vecteurs}

Le but de ce chapitre est de traiter la section \og Manipuler les vecteurs du plan \fg du bulletin officiel.
Un partie du contenu a déjà été étudiée au chapitre \ref{chap:3}, section \ref{sec:geom-plane} : géométrie plane.
Il s'agira d'une part de revoir des concepts : représentation de points dans un repère orthonormé, opérations sur les points, longueur de segment.
D'autre part, on ajoutera les notions de vecteur comme translation, de norme, de colinéarité, et de déterminant.

Le contenu du chapitre est le suivant.
	\begin{itemize}
		\item
			Vecteur $\vec{MM'}$ associé à la translation qui transforme $M$ en $M'$. Direction, sens, et norme.
		\item
			Égalité de deux vecteurs. Notation $\vec{u}$. Vecteur nul.
		\item
			Somme de deux vecteurs en lien avec l'enchaînement des translations. Relation de Chasles.
		\item
			Base orthonormée. Coordonnées d'un vecteur. Expression de la norme d'un vecteur.
		\item
			Expression des coordonnées de $\vec{AB}$ en fonction de celles de $A$ et de $B$.
		\item
			Produit d'un vecteur par un nombre réel. Colinéarité de deux vecteurs.
		\item
			Déterminant de deux vecteurs dans une base orthonormée, critère de colinéarité.
			Application à l'alignement et au parallélisme.
	\end{itemize}

Les capacités attendues sont les suivantes.
	\begin{itemize}
		\item 
			Représenter géométriquement des vecteurs.
		\item
			Construire géométriquement la somme de deux vecteurs.
		\item
			Représenter un vecteurs dont on connaît les coordonnées.
			Lire les coordonnées d'un vecteur.
		\item
			Calculer les coordonnées d'une somme de vecteurs, d'un produit d'un vecteur par un nombre réel.
		\item
			Calculer la distance entre deux points. Calculer les coordonnées du milieu d'un segment.
		\item
			Résoudre des problèmes en utilisation la représentation la plus adaptée des vecteurs.
	\end{itemize}
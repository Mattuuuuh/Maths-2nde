%!TEX encoding = UTF8
%!TEX root = 0-notes.tex

\chapter{Statistiques descriptives}
\label{chap:statistiques}


On considère dans cette section une \emphindex{série statistique}, c'est-à-dire une liste de valeurs d'une même nature qu'on souhaite étudier.
Les exemples de séries statistiques sont nombreux : une liste de notes à une évaluation, une liste de notes d'un élève unique à plusieurs évaluations, une liste de salaires bruts, une liste de températures dans un même lieu, etc…

On parle ici de \emphindex{liste} et non d'\emph{ensemble} car une valeur peut se répéter plusieurs fois, ce qui change alors la nature de la série statistique.
La moyenne de la liste $(8; 10; 12)$ est différente de la moyenne de $(8; 10; 12; 12)$.

Dans toute la section suivante, $N, p \in \N$ sont deux entiers naturels non nuls qui désignent un cardinal (de liste ou d'ensemble). 
Les valeurs $x_1, x_2, \dots$ sont toutes des nombres réels.

\dfn{Étendue}{
	Considérons $X$ une série statistique.
	L'\emphindex{étendue} de la série $X$ est égale à la distance entre la plus petite et la plus grande valeur.
		\[ \bigl|\min(X) - \max(X) \bigr|. \]
}{}

\section{Moyennes pondérées}

\dfn{moyenne}{
	Considérons $X = (x_1, x_2, \dots, x_N)$ une série statistique.
	On note $\overline{X}$ sa moyenne, donnée par
		\[ \overline{X} = \dfrac{x_1 + \dots + x_N}N. \]
}{}


\exe{1}{
	Calculer la moyenne de chaque série statistique suivante.
		\begin{multicols}{2}
		\begin{enumerate}
			\item $(0; 20)$
			\item $(0; 0; 20)$
			\item $(0; 20; 20)$
			\item $(1;2;3; \dots; 18; 19; 20)$
		\end{enumerate}
		\end{multicols}
}{exe:stats1}{
	TODO
}


\dfn{moyenne pondérée}{
	Soit $X = (x_1, x_2, \dots, x_N)$ une série statistique à coefficients $(c_1, \dots, c_N)$.
	
	On note $\overline{X}$ sa moyenne, donnée par
		\[ \overline{X} = \dfrac{c_1 \cdot x_1 + \dots + c_N \cdot x_N}{c_1 + \cdots + c_N}. \]
}{def:moyenne-pond}


\exe{}{
	Un élève reçoit $5$ notes listées ainsi : $(4,25 ; 10; 18,5 ; 15,5 ; 11,25)$.
	À ces notes sont associés $5$ coefficients, donnés dans le même ordre par $(0,5 ; 1,4 ; 0,1 ; 0,5; 1,1)$.
	
	Calculer la moyenne pondérée des notes.
}{exe:stats2}{
	TODO
}


\nt{
	Remarquons que la moyenne de la définition \ref{def:moyenne-pond} s'écrit aussi 
		\begin{align}
			\overline{X} &=  \dfrac{c_1 \cdot x_1 + \dots + c_N \cdot x_N}{c_1 + \cdots c_N} \nonumber\\
						&= \dfrac{c_1}{c_1 + \cdots c_N} \cdot x_1 + \dfrac{c_2}{c_1 + \cdots c_N} \cdot x_2 + \dots + \dfrac{c_N}{c_1 + \cdots c_N} \cdot x_n \label{expr:poids}
		\end{align}
	En divisant par la somme des coefficients, on a \emphindex{normalisé} les coefficients en poids, qu'on multiplie à chaque valeur.
	
	% to fix
	Une discussion dans le cas de deux valeurs ($N=2$) est donnée dans la section \ref{sec:longueur-milieu} du chapitre \ref{chap:3}.
	Celle-ci lie la notion de milieu avec celle de moyenne, puis la notion d'intervalle avec celle de moyenne pondérée.
}

\lem{}{
	Les poids $\dfrac{c_1}{c_1 + \cdots c_N}, \dfrac{c_2}{c_1 + \cdots c_N}, \dots, \dfrac{c_N}{c_1 + \cdots c_N}$ de l'expression \eqref{expr:poids} ci-dessus vérifient
		\begin{enumerate}
			\item Chaque poids est un nombre appartenant à $[0;1]$ ; et
			\item La somme des poids vaut $1$.
		\end{enumerate}
}{lem:poids-normalisés}

\nt{
	Certaines séries statistiques sont données sous la forme d'un ensemble de couples $(x, n)$ où $n$ est le nombre d'appartitions de $x$ dans la liste.
	Chaque valeur $x$ est alors distincte et on peut bien parler d'ensemble.
}{}

\dfn{moyenne avec effectifs}{
	Considérons $X = \left\{ (x_1, n_1), (x_2, n_2), \dots, (x_p, n_p) \right\}$ une série statistique avec effectifs.
	
	On note $\overline{X}$ sa moyenne, donnée par
		\begin{align}\label{eq:moyenne-effectifs}
			\overline{X} = \dfrac{n_1 \cdot x_1 + \dots + n_p \cdot x_p}{n_1 + \dots + n_p}.
		\end{align}
}{}

\nt{
	Notons $n = n_1 + \dots + n_p$. On peut réécrire la moyenne $\overline{X}$ donnée en \eqref{eq:moyenne-effectifs} comme
		\begin{align*}
			\overline{X} &= \dfrac{n_1 \cdot x_1 + \dots + n_p \cdot x_p}{n} \\
						&= \dfrac{n_1}n \cdot x_1 + \dfrac{n_2}n x_2 + \dots + \dfrac{n_p}n x_p \\
						&= f_1 \cdot x_1 + f_2 \cdot x_2 + \dots f_p \cdot x_p,
		\end{align*}
	où $f_1 = \dfrac{n_1}n$, $f_ 2 = \dfrac{n_2}n$, etc... sont les fréquences d'apparition des valeurs dans la série statistique.
}

\dfn{Fréquence d'apparition}{
	Soit $X = \left\{ (x_1, n_1), (x_2, n_2), \dots, (x_p, n_p) \right\}$ une série statistique avec effectifs.

	On pose $n = n_1 + \dots + n_p$. 
	Alors
		\begin{align*}
			f_1 = \dfrac{n_1}n, && f_ 2 = \dfrac{n_2}n, && \dots && f_ p = \dfrac{n_p}n,
		\end{align*}
	désignent les \emphindex{fréquences d'apparition} de chaque valeur $x_1, x_2, \dots, x_p$.
}
{def:freq-app}

\exe{}{
	Pour chaque série statistique suivante, remplir la ligne \og Fréquence \fg et calculer la moyenne.
		\begin{multicols}{2}
		\begin{enumerate}
			\item 
				\begin{tabular}{|c|c|c|}\hline
				Valeur   & 0 & 20 \\ \hline
				Effectif & 1 & 2  \\ \hline
				Fréquence & &  \\ \hline
				\end{tabular}
				
			\item 
				\begin{tabular}{|c|c|c|}\hline
				Valeur   & 0 & 20 \\ \hline
				Effectif & 2 & 1  \\ \hline
				Fréquence & &  \\ \hline
				\end{tabular}
				
			\item 
				\begin{tabular}{|c|c|c|c|c|}\hline
				Valeur   & 11 & 13 & 9 & 7 \\ \hline
				Effectif & 2 & 1 & 5 & 10 \\ \hline
				Fréquence & &&&  \\ \hline
				\end{tabular}
				
			\item 
				\begin{tabular}{|c|c|c|c|c|}\hline
				Valeur   & -23 & -1 & 1 & 23 \\ \hline
				Effectif & 176 & 304 &304 & 176 \\ \hline
				Fréquence & &&&  \\ \hline
				\end{tabular}

		\end{enumerate}
		\end{multicols}
}{exe:stats3}{
	TODO
}

\lem{}{
	Les fréquences $f_1, \dots, f_p$ de la définition \ref{def:freq-app} vérifient
		\begin{align*}
			f_1 \in [0;1], && f_2 \in [0;1], && \dots && f_p \in [0;1],
		\end{align*}
	ainsi que
		\[ f_1 + \dots + f_p = 1. \]
}{lem:fréquences-normalisées}

\nt{
	La remarque suivante est en dehors du champ d'application du cours.
	
	Du lemme \ref{lem:fréquences-normalisées}, on déduit qu'une moyenne est en fait une \emphindex{combinaison convexe} des valeurs d'une série statistique. On parle aussi de \emphindex{barycentre}. 
	Un intervalle est donné par l'ensemble de telles combinaisons des bornes inférieure et supérieure : c'est l'enveloppe convexe de ses bornes.
}

\mprop{Linéarité de la moyenne}{
	Soit $X=(x_1 ; \dots ; x_N)$ une série statistique, et $\overline{X}$ sa moyenne.
	On étudie deux opérations faites sur la série statistique entière : multiplier chaque valeur par un nombre $a\in\R$, et additionner un nombre $b\in\R$ à chaque valeur.	
	
	Pour n'importe quel $a \in \R$, la série statistique
		\[ (a\cdot x_1; a\cdot x_2; \dots ; a\cdot x_N) \]
	a pour moyenne $a\cdot \overline{X}$.
	Autrement dit, $\overline{aX} = a \overline{X}$.
	
	Pour n'importe quel $b\in\R$, la série statistique
		\[ (x_1 + b; x_2 + b; \dots ; x_N + b) \]
	a pour moyenne $\overline{X} + b$.
	Autrement dit, $\overline{X+b} = \overline{X} + b$.
	
	La combinaison des deux opérations dites \emphindex{opérations linéaires} donne
		\[ \overline{aX + b} = a\overline{X} + b \]
	pour n'importe quel $a,b \in \R$.
}{}


\exe{Inflation et euro fixe}{
	La moyenne des salaires mensuels d'une entreprise est de $1860$€.
	L'inflation désigne une augmentation générale des prix. Ainsi, $6\%$ d'inflation signifie que, en moyenne, les prix augmentent de $6\%$.
	Réciproquement et afin de pouvoir comparer les pouvoirs d'achats, on peut fixer les prix et voir l'inflation comme une diminution du salaire (en l'occurrence de $1-\frac{1}{1,06} \approx 5,6 \%$, d'après le théorème \ref{thm:ev-rec}).
	
	L'entreprise, après une année où l'inflation se mesurait à $6\%$ décide d'ajouter des primes aux salaires : tous les employés recevront $50$€ mensuellement en plus. Ces $50$€ sont, eux aussi, soumis à l'inflation.
	
	Quelle est la nouvelle moyenne des salaires mensuels de l'entreprise ?

}{exe:stats4}{
	TODO
}


\section{Écart type}

La moyenne seule ne suffit pas à elle seule à caractériser une série statistique.
L'exercice suivant considère plusieurs séries différentes ayant une même moyenne.
Le but de la section est d'introduire l'\emphindex{écart type}, noté $\sigma$ et lu \og sigma \fg, qui permet d'indiquer la \emphindex{dispersion} d'une série : son caractère plus ou moins concentré autour de sa moyenne.

\exe{}{
	Pour chaque série statistique suivante, calculer la moyenne.
		\begin{multicols}{2}
		\begin{enumerate}
			\item 
				\begin{tabular}{|c|c|c|c|c|}\hline
				Valeur   & 9 & 13 & 11 & 7 \\ \hline
				Effectif & 5 & 12 & 5 & 12 \\ \hline
				\end{tabular}
				
			\item 
				\begin{tabular}{|c|c|c|c|c|}\hline
				Valeur   & 2 & 13 & 18 & 7 \\ \hline
				Effectif & 5 & 12 & 5 & 12 \\ \hline
				\end{tabular}
				
			\item 
				\begin{tabular}{|c|c|c|c|c|}\hline
				Valeur   & 0 & 20 & 2 & 5 \\ \hline
				Effectif & 5 & 14 & 10 & 2  \\ \hline
				\end{tabular}

			\item 
				\begin{tabular}{|c|c|c|c|}\hline
				Valeur   & 0 & 20 & 10 \\ \hline
				Effectif & 17 & 17 & 1 \\ \hline
				\end{tabular}


		\end{enumerate}
		\end{multicols}
}{exe:stats5}{
	TODO
}

\nt{
	Soit $X=(x_1 ; \dots ;x_N)$ une série statistique.
	
	En plus de la moyenne $\overline{X}$, il est aussi intéressant d'étudier la variance $\Var(X)$.
	Elle mesure la distance moyenne des $x_1, \dots, x_N$ à $\overline{X}$.
	Plus cette distance est petite, plus la valeur est proche de la moyenne.
	Globalement, si toutes les distances sont petites, la série est concentrée autour de sa moyenne, au lieu d'être dispersée.
	
	Il est tentant de calculer les distances
		\[ | x - \overline{X}| \]
	pour chaque valeur $x$, mais il est préférable de considérer le carré de la distance, c'est-à-dire
		\[ | x - \overline{X}|^2. \]
	La raison profonde sera étudiée plus tard, dans le chapitre des fonctions.
	Simplement, la valeur absolue n'est \emph{pas} lisse, alors que la fonction carré, elle, l'est.
	Comme $|x|^2 = x^2$, la valeur absolue disparaît lorsque mise au carré, et la fonction devient lisse.
}

\dfn{variance}{
	Soit $X=(x_1 ; \dots ;x_N)$ une série statistique, $\overline{X}$ sa moyenne.
	
	On dénote par $\Var(X)$ la \emphindex{variance} de $X$, donnée par
		\[ \Var(X) = \dfrac{\left(x_1 - \overline{X}\right)^2 + \dots + \left(x_N - \overline{X}\right)^2}N .\]
	La variance est toujours positive.
	
	Si $X = \left\{ (x_1, n_1), (x_2, n_2), \dots, (x_p, n_p) \right\}$ est une série statistique donnée avec effectifs, on a alors 
		\begin{align}
			\Var(X) = \dfrac{n_1 \cdot \left(x_1 - \overline{X}\right)^2 + \dots + n_p \cdot \left(x_p - \overline{X}\right)^2}{n_1 + \dots + n_p}. \label{eq:var}
		\end{align}
	
}{}

\dfn{Écart type}{
	Soit $X$ une série statistique.
	L'\emphindex{écart type} $\sigma = \sigma(X)$  (lu \og sigma \fg)  est donné par
		\[ \sigma(X) = \sqrt{\Var(X) }. \]
}{}

\nt{
	Soit $X=(x_1 ; \dots ;x_N)$ une série statistique, $\overline{X}$ sa moyenne.
	
	Si on pose $Y = \left( \left(x_1 - \overline{X}\right)^2  ; \dots ; \left(x_N - \overline{X}\right)^2 \right)$, nouvelle série statistique, on a
		\[ \overline{Y} = \Var(X). \]
		
	La variance est donc bien une moyenne des distances au carré à $\overline{X}$.
}

\exe{}{
	Calculer et comparer les écart types des séries de l'exercice \ref{exe:stats5}.
}{exe:stats6}{
	TODO
}

\exe{}{
	\begin{multicols}{2}
	On considère la série ci-contre tirée de l'exercice \ref{exe:stats5}.
	
		\begin{center}
		\begin{tabular}{|c|c|c|c|}\hline
		Valeur   & 0 & 20 & 10 \\ \hline
		Effectif & 17 & 17 & 1 \\ \hline
		\end{tabular}
		\end{center}
	\end{multicols}
		
	Modifier l'effectif de la valeur $10$ en $10, 20, 50, 100$, et calculer la moyenne et l'écart type de la série obtenue.
	Comment évolue $\sigma$ ? La série devient-elle plus ou moins concentrée autour de sa moyenne ?
}{exe:stats7}{
	TODO
}

\section{Quartiles}

Dans ce qui suit, on considère $X = (x_1 ; \dots ; x_N)$ une série statistique qu'on \textbf{ordonne} : c'est-à-dire dont les valeurs sont rangées par ordre croissant.
	\[ \underbrace{x_1 \leq x_2 \leq x_3 \leq \dots}_{\text{proportion } p} \leq \dots \leq x_{N-2} \leq x_{N-1} \leq x_N. \]
La notion de quartile (et plus généralement de décile, centile, ...) permet de répondre aux questions du type : \og Quelle proportion $p$ de valeurs de $X$ sont inférieures à un seuil donné ? \fg, et \og Pour une proportion donnée $p$, quelle est le seuil qui lui correspond ? \fg.

Dans cette section, on considère les proportions $p=0,5 ; \dfrac14 ; et \dfrac34$ qui correspondent à la médiane, au premier et au troisième quartile.

\dfn{Médiane}{
	La médiane d'une série $X = (x_1 ; \dots ; x_N)$ rangée par ordre croissant est donné par 
		\begin{enumerate}
			\item sa valeur centrale si $N$ est impair ; ou
			\item la moyenne de ses valeurs centrales si $N$ est pair.
		\end{enumerate}
		
	Ainsi $50\%$ des valeurs de $X$ sont inférieure à sa médiane.
}{}

\dfn{Quartiles}{
	Le premier quartile $Q_1$ d'une série $X = (x_1 ; \dots ; x_N)$ rangée par ordre croissant est donné par la plus petite valeur de la série de rang supérieur ou égal à $\dfrac{N}4$.
	
	Ainsi $25\%$ des valeurs de $X$ sont inférieure à son premier quartile.
	
	Le troisième quartile $Q_3$ d'une série $X = (x_1 ; \dots ; x_N)$ rangée par ordre croissant est donné par la plus petite valeur de la série de rang supérieur ou égal à $\dfrac{3N}4$.
	
	Ainsi $75\%$ des valeurs de $X$ sont inférieure à son troisième quartile.
}{}

\dfn{Écart interquartile}{
	L'écart interquartile d'une série statistique $X$ est donné par la distance entre ses deux quartiles $Q_1$ et $Q_3$, c'est-à-dire
		\[ |Q_1 - Q_3|. \]
	En supposant que $4|N$ pour faciliter les notations, on a la situation suivante.
	\[ x_1 \leq \dots \leq \underbrace{ \overbrace{x_{N/4}}^{Q_1} \leq \dots \leq \overbrace{x_{3N/4}}^{Q_3}}_{\frac{N}2 \text{ valeurs centrales}} \leq \dots \leq x_N. \]
	Les $\frac{N}2$ valeurs centrales sont à distance $|Q_1 - Q_3|$ les unes des autres.
}{}

\begin{figure}[t!]
  \centering
  \begin{subfigure}[b]{.45\textwidth}
    \centering
    \includegraphics[page=1]{figures/fig-statistiques.pdf}
  \caption{Évaluation ``Droite réelle''.}
  \label{fig:a}
  \end{subfigure}
  \hfill
  % NOT LINE BREAK!!
  \begin{subfigure}[b]{.45\textwidth}
    \centering
    \includegraphics[page=2]{figures/fig-statistiques.pdf}
  \caption{Évaluation ``Droite et plan''.}
  \end{subfigure}
    \begin{subfigure}[b]{.45\textwidth}
    \centering
    \includegraphics[page=3]{figures/fig-statistiques.pdf}
  \caption{Automatismes 1 à 5.}
  \end{subfigure}
  
  
  \caption{Histogrammes de notes (min $0$, max $20$). Les classes sont de la forme $[k;k+1[, k\in\N$. La colonne entre $12$ et $13$ compte donc toutes les notes appartenant à $[12;13[$.}
  \label{fig:hist}
\end{figure}

\exe{}{
	Écrire un tableau Valeur/Effectif pour chaque histogramme de la figure \ref{fig:hist}.
	On assignera la valeur moyenne $k+\frac12$ à chaque élément d'une classe $[k; k+1[, k\in\N$.
	Par exemple, de l'histogramme \ref{fig:a} on lit $4$ notes de valeur $16,5$.
	
	Pour chaque série obtenue, calculer
		\begin{multicols}{2}
		\begin{enumerate}
			\item La moyenne ;
			\item L'écart type ;
			\item La médiane ;
			\item Le premier quartile ;
			\item Le troisième quartile ; et
			\item L'écart interquartile.
		\end{enumerate}
		\end{multicols}
}{exe:stats8}{
	TODO
}









% j'aime pas les pie chart c'est inutile
%\def\angle{0}
%\def\radius{3}
%\def\cyclelist{{"myg","gray","myr","myb"}}
%\newcount\cyclecount \cyclecount=-1
%\newcount\ind \ind=-1
%\begin{figure}
%  \begin{tikzpicture}
%      \foreach \percent/\name in {
%        75/{préoccupation mineure (LC)},
%        6/{données insufficantes (DD)},
%        14/{éteintes ou menacées (EX à VU)},
%        5/{quasi menacées (NT)}
%    } {
%      \ifx\percent\empty\else               % If \percent is empty, do nothing
%        \global\advance\cyclecount by 1     % Advance cyclecount
%        \global\advance\ind by 1            % Advance list index
%        \ifnum3<\cyclecount                 % If cyclecount is larger than list
%          \global\cyclecount=0              %   reset cyclecount and
%          \global\ind=0                     %   reset list index
%        \fi
%        \pgfmathparse{\cyclelist[\the\ind]} % Get color from cycle list
%        \edef\color{\pgfmathresult}         %   and store as \color
%        % Draw angle and set labels
%        \draw[fill={\color!50},draw={\color}] (0,0) -- (\angle:\radius)
%          arc (\angle:\angle+\percent*3.6:\radius) -- cycle;
%        \node at (\angle+0.5*\percent*3.6:0.7*\radius) {\percent\,\%};
%        \node[pin=\angle+0.5*\percent*3.6:\name]
%          at (\angle+0.5*\percent*3.6:\radius) {};
%        \pgfmathparse{\angle+\percent*3.6}  % Advance angle
%        \xdef\angle{\pgfmathresult}         %   and store in \angle
%      \fi
%    };
%  \end{tikzpicture}
%  \label{fig:pie-chart}
%  \caption{Répartition par niveau de menace des espèces présentes en France et évaluées dans la liste rouge mondiale de l'UICN ($15579$ espèces). Source : \href{naturefrance.fr}{naturefrance.fr}, $2023$.}
%\end{figure}


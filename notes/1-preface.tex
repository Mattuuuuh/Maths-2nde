%!TEX encoding = UTF8
%!TEX root = 0-notes.tex

\chapter*{Préface}

Construction d'un cours de mathématiques : Exemple -- Remarque -- Définition -- Lemme -- Proposition -- Théorème -- Démonstration -- Corollaire.

Voir \emph{Mechanica, volume 1}, Leonhard Euler, 1736 (\href{https://scholarlycommons.pacific.edu/euler-works/15/}{E15}), pour un exemple de cours de mathématiques en latin.

Wolframalpha permet de faire du calcul exact (voir chapitre nombres rationnels pour la nécessité).

\href{https://eduscol.education.fr/document/24553/download}{Programme de 2nde} comme figurant sur 
\href{https://eduscol.education.fr/1723/programmes-et-ressources-en-mathematiques-voie-gt}{Eduscol}.

\begin{definition*}
Associe un mot de vocabulaire à un objet mathématique.
Un mot de vocabulaire peut avoir un sens différent en mathématiques qu'en langage vernaculaire.
C'est par exemple le cas de « ensemble », « couple », « rationnel », « réel », ... pour ne citer que le chapitre 1 !
\end{definition*}

\begin{lemme*}
	Résultat préliminaire, parfois technique, utile à la démonstration d'une proposition plus importante.
\end{lemme*}

\begin{proposition*}
	Proposition importante.
\end{proposition*}

\begin{theorem*}
	Résultat fondamental.
\end{theorem*}

\begin{corollaire*}
	Résultat particulier mais important, découlant quasi directement d'un théorème ou d'une proposition.
\end{corollaire*}

\newcounter{preface}
\begin{Exercise}[counter=preface]
	Exemple de problème d'application directe.
\end{Exercise}
\begin{Exercise}[difficulty=1, counter=preface]
	Exemple de problème plus théorique.
\end{Exercise}
\begin{Exercise}[difficulty=2, counter=preface]
	Exemple de problème d'approfondissement.
\end{Exercise}

\newpage
\begin{figure}
	\centering
	\includegraphics[page=1, scale=1.5]{figures/fig-preface.pdf}
	\caption{Dépendances des chapitres.}
\end{figure}
	
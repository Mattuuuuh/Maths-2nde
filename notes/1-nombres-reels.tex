%!TEX encoding = UTF8
%!TEX root =0-notes.tex

\chapter{Nombres rationnels, nombres réels}

\section{Nombres rationnels}

\notations{
	Un collection \emphindex{non ordonnée} d'éléments est un \emphindex{ensemble}, noté avec des accolades et des points-virgules entre les éléments.
	
		\[ E = \bigset{ 2 ; 3 ; 5 ; 7 ; 11 }. \]
	
	Un ensemble peut contenir un nombre infini d'éléments.
	
		\[ F = \bigset{ 0 ; 2 ; 4 ; 6 ; 8 ; 10 ; \dots }. \]
}

\exe{1}{
	Donner l'ensemble $A$ des entiers supérieurs ou égaux à 0 et inférieurs ou égaux à 7.
	
	Donner l'ensemble $B$ des éléments de $A$ qui sont pairs.
}{exe:set1}{
	\begin{multicols}{2}
		$A = \bigset{ 0 ; 1 ; 2 ; 3 ; 4 ; 5 ; 6 ; 7}$
		
		$B = \bigset{ 0 ; 2 ; 4 ; 6 }$
	\end{multicols}
}

\dfn{nombres entiers}{
	On pose les nombres \emphindex{entiers naturels}
		\[ \N =  \bigset{ 0 ; 1 ; 2 ; \dots }, \]
	et les nombres \emphindex{entiers relatifs}
		\[ \Z =  \bigset{ \dots ; -2 ; -1 ; 0 ; 1 ; 2 ; \dots }. \]
}{dfn:nombres-entiers}
	
	
\notations{
	Il est aussi possible de décrire l'ensemble $F$
		\[ F =  \bigset{ 0 ; 2 ; 4 ; 6 ; 8 ; 10 ; \dots } \]
	sous la forme
		\[ F =  \bigset{ n \in \N \tqs \text{$n$ est pair} }. \]
	La deuxième forme est plus précise et évite au lecteur d'avoir à deviner la structure de l'ensemble.
}

\exe{}{
	Décrire l'ensemble $A$ de l'exercice \ref{exe:set1} sous les formes $\bigset{ n \in \N \tq \dots }$ et $\bigset{ n \in \Z \tq \dots }$.
	
	Décrire l'ensemble $B$ de l'exercice \ref{exe:set1} sous la forme $\bigset{ n \in A \tq \dots }$.
}{exe:set2}{
	\begin{multicols}{2}
		$A = \bigset{ n \in \N \tq n \leq 7}$
		
		$A = \bigset{ n \in \Z \tq 0 \leq n \leq 7}$
		
		$B = \bigset{ n \in A \tq \text{$n$ est pair}}$
	\end{multicols}
}

\dfn{nombres rationnels}{
	Soit $a, b \in \Z$ deux entiers relatifs, avec $b$ non nul.
	
	Alors le \emphindex{rationnel} $\frac{a}{b}$ est l'\underline{unique} nombre tel que
		\[ \dfrac{a}b \times b = a. \]
	
	On note l'ensemble de tous les rationnels
		\[ \Q = \Bigset{ \dfrac{a}{b} \tq a \in \Z, b \in \Z, b \neq 0}. \]
}{dfn:nombres-rationnels}

\ex{}{
	Le nombre rationnel $\frac12$ est l'unique nombre vérifiant
		\[ \dfrac12 \times 2 = 1. \]
	Comme $0,5 \times 2 = 1$, on a $\frac12  = 0,5$. C'est le \emphindex{développement décimal} de $\frac12$.
}{ex:nombres-rationnels1}

\ex{}{
	Le nombre rationnel $\frac43$ est l'unique nombre vérifiant
		\[ \dfrac43 \times 3 = 4. \]
}{ex:nombres-rationnels2}

\exe{}{
	Montrer que $\frac13 = \frac26$.
}{exe:reduction-fraction}{
	Par définition, $\frac13$ est l'unique $x\in\Q$ vérifiant $3x = 1$.
	Cependant, on a l'équivalence suivante :
		\[ 3x = 1 \iff 6x = 2. \]
	$x$ est donc égal à $\frac26$, par définition.
}

\exe{}{
	Montrer que $\frac{a}{a} = 1$ pour n'importe que $a\in\Z$ non nul.
}{exe:a-div-a}{
	Par définition, $\frac{a}{a}$ est l'unique $x\in\Q$ vérifiant $ax = a$.
	Comme $x=1$ vérifie cette équation, l'unicité conclut.
}

\exe{, difficulty=1}{
	Montrer que $\frac12 + \frac13 = \frac56$.
}{exe:somme-rationnels}{
	Posons $x = \frac12 + \frac13$.
	On a alors
		\begin{align*}
			6x &= 6 \times \dfrac12 + 6 \times \dfrac13, \\
				&= 3(2 \times \dfrac12) + 2(3 \times \dfrac13), \\
				&= 3 + 2 = 5.
		\end{align*}
	On conclut que $x = \frac56$.
}

\exe{, difficulty=1}{
	Montrer que $\frac12 \times \frac23 = \frac26$.
}{exe:produit-rationnels}{
	Posons $x = \frac12 \times \frac23$.
	On a alors, par associativité,
		\begin{align*}
			6x &= 6 \times \left(\dfrac12 \times \dfrac23\right), \\
				&= (2\times3) \times \left(\dfrac12 \times \dfrac23\right), \\
				&= \left(2\times\dfrac12\right)\times \left(3\times\dfrac23\right), \\
				&= 1 \times 2 = 2.
		\end{align*}
	On conclut que $x = \frac26$.
}

\thm{}{
	On a les identités suivantes, pour $a, b, c, d \in \Z$, non nuls lorsques dénominateurs.
	
	\begin{align*}
		\dfrac{a}{b} \cdot \dfrac{c}{d} = \dfrac{ac}{bd}
		&& \text{ et } &&
		\dfrac{a}{b} + \dfrac{c}{d} = \dfrac{ad + bc}{bd}.
	\end{align*}
}{thm:stabilité-Q}

\exe{, difficulty=2}{
	Démontrer le théorème \ref{thm:stabilité-Q}.
}{exe:stabilité-Q}{
	Soit $r = \dfrac{a}{b}$ et $s = \dfrac{c}{d}$.
	Alors
		\begin{align*}
			br = a, && \text{ et } && ds = c.
		\end{align*}
	En multipliant la première équation par $d$ et la deuxième par $b$, on pourra factoriser et obtenir une équation pour $r+s$ :
		\begin{align*}
			(db)r = da, && \text{ et } && (bd)s = bc,
		\end{align*}
	implique $db(r + s) = da + bc$.
	Il suit que $r+s = \dfrac{ad + bc}{bd} \in \Q$.
	
	Similairement, en multipliant les équations, on obtient $(br)(ds)=ac \iff (bd)(rs) = ac$, et donc $r \cdot s = \dfrac{ac}{bd} \in \Q$.
}

\nomen{
	On appelle « corollaire » une proposition qui découle immédiatement d'un théorème mais qui mérite d'être énoncée seule.
	Le corollaire couronne ainsi le théorème.
}

\cor{}{
	Pour $a, b, c \in \Z$, non nuls lorsques dénominateurs, on a 
		\[ \dfrac{a}{b} = \dfrac{ac}{bc}. \]
}{cor:simplification-Q}

\exe{, difficulty=2}{
	Démontrer le corollaire \ref{cor:simplification-Q}.
}{exe:simplification-Q}{
	D'après l'exercice \ref{exe:a-div-a}, $1 = \dfrac{c}{c}$.
	Comme multiplier par 1 n'a pas d'effet, on a
		\[ \dfrac{a}b = \dfrac{a}b \times 1 = \dfrac{a}b \times \dfrac{c}c = \dfrac{ac}{bc}, \]
	où on a utilisé le théorème \ref{thm:stabilité-Q} pour la dernière égalité.
}

\qs{}{
	Est-il possible d'écrire le développement décimal de $\frac13$ sur un feuille de papier A4 ?
	
	Le but de la prochaine section est de répondre à cette question par la négative.
}

\section{Développements décimaux et écriture scientifique}

\dfn{nombres décimaux}{
	On dit d'un nombre $x$ qu'il est \emphindex{décimal} si on peut écrire son développement décimal sur un feuille de papier A4 (possiblement en écrivant très petit).
	Autrement dit, son développement décimal est fini.
	
	On note 
		\[ \DD = \bigset{ x \in \Q \tq \text{ le développement décimal de $x$ est fini} }. \]
}{dfn:nombres-décimaux}

\mprop{puissances de 10}{
	Soit $n\in\N$ un entier naturel. Comme
		\begin{enumerate}[label=$\bullet$]
			\item multiplier par $10$ décale la virgule d'une position vers la droite ; et
			\item diviser par $10$ décale la virgule d'une position vers la gauche ;
		\end{enumerate}
	on a nécessairement
		\begin{flalign*}
			&&
			10^n = 1\underbrace{00\dots00}_{ \text{$n$ fois} }
			&&
			\et
			&&
			10^{-n} = \dfrac1{10^n} = \underbrace{0,00\dots00}_{ \text{$n$ fois} }1.
			&&
		\end{flalign*}
}{prop:puissances-10}

\ex{}{
	L'égalité $\frac{321}{5000} = \frac{642}{10000} = 0,0642$ donne le développement décimal de la fraction $\frac{321}{5000}$.
}{ex:dev-decimal}

\exe{}{
	Sans calculatrice, donner le développement décimal des fractions suivantes.
	\begin{multicols}{3}
	\begin{enumerate}[label=\roman*), leftmargin=60pt]
		\item $\dfrac1{10}$
		\item $\dfrac1{5}$
		\item $\dfrac3{10^{5}}$
		\item $\dfrac7{20}$
		\item $\dfrac{395}{50}$
		\item $\dfrac{11}{200}$
	\end{enumerate}
	\end{multicols}
}{exe:dev-decimaux}{
	\begin{multicols}{3}
	\begin{enumerate}[label=\roman*)]
		\item $\dfrac1{10} = 0,1.$
		\item $\dfrac1{5} = \dfrac2{10} = 0,2.$
		\item $\dfrac3{10^{5}} = 0,00003.$
		\item $\dfrac7{20} = \dfrac{3,5}{10} = 0,35.$
		\item $\dfrac{395}{50} = \dfrac{790}{100} = 7,9.$
		\item $\dfrac{11}{200} = \dfrac{5,5}{100} = 0,055.$
	\end{enumerate}
	\end{multicols}
}

\thm{caractérisation de $\DD$}{
		\[ \DD = \Bigset{ \dfrac{a}{10^n} \tq a \in \Z, n \in \N }. \]
}{thm:caractérisation-D}

\pf{}{
	Posons $E = \Bigset{ \frac{a}{10^n} \tq a \in \Z, n \in \N }$.
	Pour montrer l'égalité d'ensembles $\DD = E$, il faut toujours montrer la double inclusion $\DD \subseteq E$ et $E \subseteq \DD$.
	Pour montrer une inclusion $\subseteq$ d'ensembles, il faut montrer que n'importe quel élément de l'ensemble de gauche (le plus petit) appartient à l'ensemble de droite (le plus grand).
	
	\begin{enumerate}[leftmargin=100pt]
		\item[\underline{Inclusion $\DD \subseteq E$} :]
		Un élément $x$ de $\DD$ admet un développement décimal fini, posons $n\in\N$ sa longueur.
		Alors $10^n x = a$ est un entier relatif. Il suit que $x = \frac{a}{10^n} \in E$ comme requis.
		
		\item[\underline{Inclusion $E \subseteq \DD$} :]
		Un élément $x$ de $E$ est de la forme $x = \frac{a}{10^n}$. Comme $a$ est entier, le développement décimal de $x$ est de longueur au plus $n$, et est donc fini.
	\end{enumerate} 
}{}

\cor{}{
	On a l'inclusion d'ensembles
		\[ \DD \subseteq \Q. \]
}{cor:D-in-Q}

\exe{}{
	Montrer le corollaire \ref{cor:D-in-Q}.
}{exe:caractérisation-D}{
	Tout élément de $\DD$ a la forme $\dfrac{a}{10^n}$ où $a \in \Z$ et $n \in \N$.
	Comme $10^n \in \N$, l'élément est bien un ratio de deux entiers et il appartient aux rationnels.
}


\dfn{notation scientifique}{
	Pour $x\in\D$ décimal, en choisissant $1 \leq a < 10$, et en permettant $n\in\Z$ entier relatif, l'expression
		\[ x = a \times 10^n, \]
	est la \emphindex{notation scientifique} de $x$.
}{dfn:notation-scientifique}

\ex{}{
	D'après l'exemple \ref{ex:dev-decimal}, on a $\frac{321}{5000} = 0,0642 = 6,42 \times 10^{-2}$.
}{ex:notation-scientifique}

\exe{}{
	Sans calculatrice, écrire les nombres suivants en notation scientifique.
		
	\begin{multicols}{3}
	\begin{enumerate}[label=\alph*)]
		\item 201
		\item 10
		\item 123 400 000
		\item 0,8
		\item 0,000 327
		\item 0,009 000 1
	\end{enumerate}
	\end{multicols}
}{exe:notation-scientifique}{
	TODO
}

\thm{}{
	Le nombre rationnel $\frac13$ n'est pas décimal : $\frac13 \not\in \DD$.
}{}

\pf{}{
	Par l'absurde, supposons que $\frac13 = \frac{a}{10^n}$.
	Alors on a nécessairement 
		\[ 10^n = 3a. \]
	Comme le membre de droite est un multiple de $3$, le membre de gauche aussi.
	
	On arrive ici à une contradiction car une puissance de $10$ ne peut pas être un multiple de $3$.
	En effet, les multiples de $3$ adviennent tous les $3$ entiers ($\{0 ; 3 ; 6 ; 9 ; 12 ; \dots\}$).
	Or l'entier juste avant $10^n$ est un multiple de $3$, car c'est 
		\[ 10^n - 1 = \underbrace{99{\dots}99}_{\text{n fois}} = 3 \times \underbrace{33{\dots}33}_{\text{n fois}} . \]
	$10^n$ ne peut donc pas être multiple de $3$, une contradiction ! \Large\Lightning
}

\exe{}{
	Montrer que $\frac19$ n'est pas décimal.
}{exe:nondecimal0}{
	Si $\dfrac19 = \frac{a}{10^n}$, alors $10^n = 9a$ et est multiple de $9$.
	Or l'entier d'avant est multiple de $9$ car
		\[ 10^n - 1 = \underbrace{99{\dots}99}_{\text{n fois}} = 9 \times \underbrace{11{\dots}11}_{\text{n fois}} . \]
	$10^n$ ne peut donc pas être multiple de $9$, une contradiction ! \Large\Lightning
}

\exe{}{
	Montrer que $\frac1{k}$ n'est pas décimal dès qu'aucune puissance de 10 n'est multiple entier de $k$.
}{exe:nondecimal1}{
	Si $\frac1k = \frac{a}{10^n}$, alors $10^n = ka$ et est multiple de $k$, une contradiction ! \Large\Lightning
}

\exe{}{
	Montrer que si une certaine puissance de 10 est multiple entier de $k$, alors $\frac1k$ est décimal.
}{exe:nondecimal2}{
	Si $10^n$ est multiple entier de $k$, alors il existe un $a \in \Z$ tel que
		\[ ak = 10^n. \]
	Ainsi $\frac1k = \frac{a}{10^n} \in \DD$.
}

\mprop{}{
	Le développement décimal de $\frac13$ est donné par
		\[ \dfrac13 = 0,333{\dots}.\]
	Il est \emphindex{infini} et \emphindex{périodique}.
}{}

\pf{}{
	Posons $x = 0,333\dots$ et montrons que $x=\frac13$.
	On a 
		\[ 10x = 3,333{\dots} = 3 + x. \]
	Il suit donc que $9x = 3$, et donc que $x=\frac13$.
}{}

\exe{}{
	Montrer que les nombres suivants sont rationnels en les exprimant sous forme de fraction d'entiers.
	\[
	\begin{aligned}
		A &= 0,666{\dots} \\
		B &= 9,999{\dots} \\
		C &= 0,121212{\dots} \\
	\end{aligned}
	\hspace{5cm}
	\begin{aligned}
		D &= 1,666{\dots} \\
		E &= 0,34777{\dots} \\
		F &= 0,123123123{\dots}
	\end{aligned}
	\]
	\[
	G = 0,123456789123456789123{\dots} \text{ (nombre d'Amandine) }
	\]
}{exe:dev-to-fraction}{
	\begin{enumerate}[label=\Alph*.]
		\item 
		La relation $10A = 6 + A$ donne $A = \frac23$.
		\item
		La relation $10B = 90 + B$ donne $A = 10$. Il s'avère donc que $10$ admet deux écritures décimales différentes.
		Pour se convaincre encore que $B=10$, on peut remarquer la chose suivante : il n'existe pas de nombre entre $B$ et $10$ strictement différent des deux. Or il existe toujours un nombre situé entre deux nombres distincts (leur moyenne, par exemple).
		\item
		La relation $100C = 12 + C$ donne $C = \frac{12}{99} = \frac{4}{33}$.
		\item
		La relation $10D = 15 + D$ donne $D = \frac{15}{9} = \frac53$.
		\item
		En posant $E' = 100E = 34,777{\dots}$, on obtient $10E' = 313 + E'$, d'où $E' = \frac{313}{9}$, et donc $E = \frac{1}{100}E' = \frac{313}{900}$.
		\item
		La relation $1000F = 123 + F$ donne $F = \frac{123}{999} = \frac{41}{333}$.
		\item 
		La relation $10^9 G = 123456789 + G$ donne $G = \frac{123456789}{999999999}$.
	\end{enumerate}
}

\dfn{développement périodique}{
	On dit que le développement décimal d'un nombre est éventuellement périodique dès qu'il se répète à partir d'un certain point.
}{dfn:périodique}

\ex{}{
	Le nombre $142,748210505050505\dots$ (les 05 se répètent indéfiniment) est périodique à partir de la sixième décimale.
	La période est de 2 nombres décimaux (0 et 5).
}{ex:période}

\lem{}{
	Le développement décimal de $\frac17$ est périodique.
}{lem:17periode}

\pf{}{
	On décrit ici un algorithme permettant de calculer le développement décimal de $\frac17$.
	Comme $\frac17 < 1$, écrivons 
		\[ \dfrac17 = 0,n_1 n_2 n_3 \dots \]
	le développement décimal de $\frac17$, où chaque $n_i$ est un chiffre.
	En multipliant par 10, on trouve
		\[ \dfrac{10}7 = n_1, n_2 n_3 n_4 \dots = 1 + \dfrac37, \]
	donc $n_1 = 1$ et
		\[ \dfrac37 = 0,n_2 n_3 n_4 \dots. \]
	On répète le raisonnement pour connaître $n_2$ :
		\[ \dfrac{30}7 = n_2,n_3n_4n_5\dots  = 4 + \dfrac27,\] 
	donc $n_2 = 4$ et
		\[ \dfrac27 = 0,n_3n_4n_5\dots,\] 
	
	Remarquons que la partie décimale est toujours de la forme $\frac{r}7$, avec $0 \leq r \leq 6$.
	De plus, elle détermine tout le développement décimal de $\frac17$ à partir d'un certain rang.
	En $8$ étapes, deux parties décimales identiques sont rencontrées, et donc le développement décimal de $\frac17$ est périodique.
}

\exe{, difficulty=1}{
	Déterminer le développement décimal de $\frac17$ sans calculatrice.
}{exe:17-dec}{
	On continue la preuve du lemme \ref{lem:17periode}, où on a déjà trouvé $n_1 = 1, n_2 = 4$, et $\frac27 = 0,n_3 n_4 n_5 \dots$.
	En multipliant par $10$, on trouve $\frac{20}7 = 2 + \frac67$, et donc $n_3 = 2$ et $\frac67 = 0,n_4 n_5n_6 \dots$.
	
	Ensuite, on a $\frac{60}7 = 8 + \frac47$, et donc $n_4 = 8$.
	Puis $\frac{40}7 = 5 + \frac57$, et $n_5 = 5$.
	Enfin, $\frac{50}7 = 7 + \frac17$, et $n_6 = 7$.
	
	Il suit que $\frac17 = 0,n_7 n_8 n_9 \dots$.
	Le développement de $\frac17$ se répète donc à partir de la septième décimale :
		\[ \dfrac17 = 0,142857~142857~142857~\cdots. \]
}

\exe{, difficulty=1}{
	Déterminer le développement décimal de $\frac1{11}$ sans calculatrice.
}{exe:111-dec}{
	Par multplication par 10 successives et étude des parties décimales, on obtient $n_1 = 0$, $\dfrac{100}{11} = 9 + \dfrac1{11}$ et $n_2=9$.
	D'où
		\[ \dfrac1{11} = 0,09~09~09~09~\cdots. \]
}

\notations{
	L'expression « si et seulement si » est notée $\iff$ et traduit l'équivalence de deux propositions.
	Deux propositions $(p)$ et $(q)$ sont \emphindex{équivalentes} si
		\begin{enumerate}
			\item dès que $(p)$ est vraie, alors $(q)$ est vraie. On note $(p) \implies (q)$ ; et
			\item dès que $(q)$ est vraie, alors $(p)$ est vraie. On note $(q) \implies (p)$.
		\end{enumerate}
}

\thm{}{
	Soit $x$ un nombre. $x$ est rationnel si et seulement si $x$ admet un développement décimal éventuellement périodique.
	Autrement dit :
	
		Pour tout nombre $x$, 
		$\Big( x$ est rationnel $\Big)$ 
		$\iff$ 
		$\Bigg($ \parbox{.4\textwidth}{
			\centering	$x$ admet un développement décimal éventuellement périodique
		} $\Bigg)$
}{thm:décimal-périodique}
	
\nt{
	On a donc l'égalité d'ensembles
		\[ \Bigg\{ 
			\parbox{.4\textwidth}{
			\centering nombres $x$ tels que $x$ admet un développement décimal éventuellement périodique
			}
			\Bigg\}
		= \Q. \]
}

\exe{, difficulty=2}{
	Démontrer le théorème \ref{thm:décimal-périodique}.
}{exe:décimal-périodique}{
	On montre les deux implications séparément.
	\begin{enumerate}
		\item[\underline{$\implies$} :]
		Si $x$ est rationnel, alors $x=\dfrac{a}b$ avec $a, b$ entiers ($b\neq0$).
		L'algorithme décrit dans la preuve du lemme \ref{lem:17periode} implique le fait suivant :
		il existe nécessairement $m < n\in\N$ tels que $10^m x$ et $10^n x$ admettent la même partie décimale, car c'est un nombre de la forme $\dfrac{r}{b}$ avec $0 \leq r \leq b$.
		
		Ainsi, le développement décimal de $x$ est périodique à partir du rang $m$ et de période $n-m$.
		\item[\underline{$\impliedby$} :]
		Soit $x$ admettant un développement décimal éventuellement périodique.
		Notons $n$ le début de la partie périodique, et $p$ la période.
		Alors $10^{n-1} x$ et $10^{n-1+p}x$ admettent le même développement décimal.
		La différence $10^{n-1+p} x - 10^{n-1} x = \bigl(10^{n-1+p} - 10^{n-1}\bigr)x$ est donc un entier.
		Par division par $10^{n-1+p}-10^{n-1}$, $x$ est rationnel.
	\end{enumerate}
}

\exe{, difficulty=2}{
	Montrer que si $\frac1k$ est périodique de période $p$ à partir de la $n$-ième décimale, alors $10^{(n-1)+p}-10^{n-1}$ est multiple entier de $k$.
	
	En déduire que $10^6 - 1 = 999~999$ fait partie de la table de multiplication de 7.
}{exe:10np10n}{
	Notons $x= \frac1k$. 
	Par hypothèses, le développement décimal de $10^{n-1}x$ est périodique de période $p$.
	Il suit que $10^p 10^{n-1}x - 10^{n-1}x = \bigl(10^{n-1+p}-10^{n-1}\bigr) x $ est un nombre entier, ce qui conclut.
	
	Pour $k=7$, l'exercice \ref{exe:17-dec} implique que $\frac17$ est périodique à partir de la première décimale de période 6.
	En spécialisant le résultat démontré pour $n=1$ et $p=6$, on obtient bien que $10^6 - 1$ est un multiple de 7.
}

\qs{}{
	Existe-t-il un nombre irrationnel ? C'est-à-dire, existe-t-il un nombre tel que son développement décimal ne soit pas périodique ?
}

\exe{, difficulty=2}{
	Donner un nombre irrationnel.
}{exe:nombre-R}{
	Voir la démonstration du théorème \ref{thm:nombre-R}.
}

\section{Nombres réels}

\thm{}{
	Il existe un nombre non rationnel.
}{thm:nombre-R}

\pf{}{
	On considère le nombre $x$ dont on construit le développement décimal de la façon suivante.
		\[ x = 0,10100100010000100000100{\dots} . \]
	Entre chaque paire de $1$, on place un nombre croissant de zéros (un, puis deux, puis trois etc…).
	
	Supposons, par l'absurde, que $x$ devienne éventuellement périodique de période $p$.
	Il existe nécessairement une série de $2p$ zéros dans sa partie périodique.
	Cette série contient une période entière, qui doit donc être entièrement nulle. 
	Le développement décimal de $x$ est ainsi éventuellement nul ($x$ est décimal), ce qui n'est pas vrai.	
	\Large\Lightning
	% un dessin serait bien ici mais je sais pas encore comment en faire un qui éclaire.
}

\dfn{nombres réels}{
	L'ensemble de tous les nombres est noté $\R$.
	On les appelle les \emphindex{nombres réels}.
	
	Ils sont continus et sont représentés par une droite qu'on appelle la \emphindex{droite réelle}.
}{dfn:nombres-réels}

\nt{
	On a ainsi le \emphindex{drapeau d'ensembles} suivant.
		\[ \N \subseteq \Z \subseteq \DD \subseteq \Q \subseteq \R. \]
}

\thm{}{
	Quelque soit $x\in\R$ réel et quelque soit $n\in\N$, il existe $r, s \in\Q$ rationnels tels que
		\begin{align*}
			r \leq x \leq s, && \text { et } && s - r \leq \dfrac{1}{10^{n}}.
		\end{align*}
	Ainsi les rationnels sont \emphindex{denses} dans les réels car il peuvent s'approcher d'aussi près que souhaité (ordre $10^{-n}$).
}{thm:densité-Q}

\nomen{
	On appelle la suite d'inégalités $r \leq x \leq s$ un \emphindex{encadrement} de $x$.
	La différence $s - r$ est nommée l'\emphindex{amplitude} de l'encadrement.
}

\exe{, difficulty=1}{
	Démontrer le théorème \ref{thm:densité-Q}.
}{exe:densité-Q}{
	Considérons le nombre $10^n x$, qu'on encadre par deux entiers consécutifs $a$ et $a+1$ en tronquant les chiffres après la virgule :
		\[ a \leq 10^n x \leq (a+1). \]
	En divisant par $10^n$, on obtient l'encadrement
		\[ \dfrac{a}{10^n} \leq x \leq \dfrac{a+1}{10^n}, \]
	qui vérifie bien les propriétés souhaitées car $\dfrac{a+1}{10^n} - \dfrac{a}{10^n} = \dfrac{a+1-a}{10^n} = \dfrac{1}{10^n}$.
}

\exe{}{
	L'encadrement 
	
	\def\arraystretch{2}
	\setlength\tabcolsep{15pt}
	\begin{tabular}{c c c}
		\hspace{10cm} & Vrai & Faux \\
		$2,6 < \sqrt{7} < 2,8$ est à $10^{-1}$ près & $\square$ & $\square$  \\
		$3,14 < \pi < 3,15$ est à $10^{-2}$ près & $\square$ & $\square$  \\
		$-4,473 < -2\sqrt5 < -4,472$ est à $10^{-3}$ près & $\square$ & $\square$  \\
		$3,3 \times 10^{-4} < 3,3931 \times 10^{-4} < 3,4 \times 10^{-4}$ est à $10^{-5}$ près & $\square$ & $\square$  \\
	\end{tabular}
}{exe:v-f-encadrement}{
	TODO
}

\exe{}{
	Encadrer les nombres suivants à l'amplitude demandée.
	\begin{multicols}{2}
	\begin{enumerate}[label=\alph*)]
		\item $3,605 \times 10^{-2}$ à $10^{-4}$ près.
		\item $9 854,698 \times 10^3$ à $10^4$ près.
		\item $-31,45$ à $10^{-1}$ près.
		\item $-0,0125$ à $10^{-4}$ près.
	\end{enumerate}
	\end{multicols}
}{exe:encadrement}{
	TODO
}

\exe{, difficulty=1}{
	On donne l'encadrement du nombre d'or $\phi =  \frac{1+\sqrt5}2$ suivant.
		\[ 1,61803 < \dfrac{1+\sqrt5}2 < 1,618035 \]
	\begin{enumerate}
		\item Donner l'amplitude de l'encadrement en notation scientifique.
		\item Trouver un encadrement de $\sqrt{5}$ et donner son amplitude.
	\end{enumerate}
}{exe:encadrement-phi}{
	TODO
}

\thm{de Cantor}{
	Les nombres rationnels sont énumérables, alors que les réels ne le sont pas.
}{thm:Q-dénombrable}

\nt{
	Cantor\footnote{Georg Cantor (1845-1918), mathématicien allemand.} démontre ce théorème grâce à son désormais célèbre \emphindex{argument de la diagonale}.
	Celui-ci en déduit que, en un certain sens (dont la définition précise est en dehors du champ d'application du cours), 
	\begin{enumerate}
		\item il y a autant de rationnels que de nombres entiers ; et
		\item il y a davantage de réels entre 0 et 1 que de rationnels dans leur intégralité.
	\end{enumerate}
}
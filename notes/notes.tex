\documentclass[a4paper, 12pt]{report}

%!TEX encoding = UTF8
%!TEX root =notes.tex


%%%%%%%%%%%%%%%%%%%%%%%%%%%%%%%%%
% PACKAGE IMPORTS
%%%%%%%%%%%%%%%%%%%%%%%%%%%%%%%%%


\usepackage[french]{babel}

\usepackage[tmargin=2cm,rmargin=1in,lmargin=1in,margin=0.85in,bmargin=2cm,footskip=.2in]{geometry}
\usepackage{amsmath,amsfonts,amsthm,amssymb,mathtools}
\usepackage[varbb]{newpxmath}
\usepackage{xfrac}
\usepackage[makeroom]{cancel}
\usepackage{mathtools}
\usepackage{bookmark}
\usepackage{enumitem}
\usepackage{hyperref,theoremref}
\hypersetup{
	pdftitle={Assignment},
	colorlinks=true, linkcolor=doc!90,
	bookmarksnumbered=true,
	bookmarksopen=true
}
\usepackage[most,many,breakable]{tcolorbox}
\usepackage{xcolor}
\usepackage{varwidth}
\usepackage{varwidth}
\usepackage{etoolbox}
%\usepackage{authblk}
\usepackage{nameref}
\usepackage{multicol,array}
\usepackage{tikz-cd}
\usepackage[ruled,vlined,linesnumbered]{algorithm2e}
\usepackage{comment} % enables the use of multi-line comments (\ifx \fi) 
\usepackage{import}
\usepackage{xifthen}
\usepackage{pdfpages}
\usepackage{transparent}


\newcommand\mycommfont[1]{\footnotesize\ttfamily\textcolor{blue}{#1}}
\SetCommentSty{mycommfont}
\newcommand{\incfig}[1]{%
    \def\svgwidth{\columnwidth}
    \import{./figures/}{#1.pdf_tex}
}

\usepackage{tikzsymbols}
%\renewcommand\qedsymbol{$\Laughey$}


%\usepackage{import}
%\usepackage{xifthen}
%\usepackage{pdfpages}
%\usepackage{transparent}


%%%%%%%%%%%%%%%%%%%%%%%%%%%%%%
% SELF MADE COLORS
%%%%%%%%%%%%%%%%%%%%%%%%%%%%%%



\definecolor{myg}{RGB}{56, 140, 70}
\definecolor{myb}{RGB}{45, 111, 177}
\definecolor{myr}{RGB}{199, 68, 64}
\definecolor{mytheorembg}{HTML}{F2F2F9}
\definecolor{mytheoremfr}{HTML}{00007B}
\definecolor{mylenmabg}{HTML}{FFFAF8}
\definecolor{mylenmafr}{HTML}{983b0f}
\definecolor{mypropbg}{HTML}{f2fbfc}
\definecolor{mypropfr}{HTML}{191971}
\definecolor{myexamplebg}{HTML}{F2FBF8}
\definecolor{myexamplefr}{HTML}{88D6D1}
\definecolor{myexampleti}{HTML}{2A7F7F}
\definecolor{mydefinitbg}{HTML}{E5E5FF}
\definecolor{mydefinitfr}{HTML}{3F3FA3}
\definecolor{notesgreen}{RGB}{0,162,0}
\definecolor{myp}{RGB}{197, 92, 212}
\definecolor{mygr}{HTML}{2C3338}
\definecolor{myred}{RGB}{127,0,0}
\definecolor{myyellow}{RGB}{169,121,69}
\definecolor{myexercisebg}{HTML}{F2FBF8}
\definecolor{myexercisefg}{HTML}{88D6D1}


%%%%%%%%%%%%%%%%%%%%%%%%%%%%
% TCOLORBOX SETUPS
%%%%%%%%%%%%%%%%%%%%%%%%%%%%

\setlength{\parindent}{1cm}
%================================
% THEOREM BOX
%================================

\tcbuselibrary{theorems,skins,hooks}
\newtcbtheorem[number within=chapter]{Theorem}{Théorème}
{%
	enhanced,
	breakable,
	colback = mytheorembg,
	frame hidden,
	boxrule = 0sp,
	borderline west = {2pt}{0pt}{mytheoremfr},
	sharp corners,
	detach title,
	before upper = \tcbtitle\par\smallskip,
	coltitle = mytheoremfr,
	fonttitle = \bfseries\sffamily,
	description font = \mdseries,
	separator sign none,
	segmentation style={solid, mytheoremfr},
}
{th}


\tcbuselibrary{theorems,skins,hooks}
\newtcolorbox{Theoremcon}
{%
	enhanced
	,breakable
	,colback = mytheorembg
	,frame hidden
	,boxrule = 0sp
	,borderline west = {2pt}{0pt}{mytheoremfr}
	,sharp corners
	,description font = \mdseries
	,separator sign none
}

%================================
% Corollery
%================================
\tcbuselibrary{theorems,skins,hooks}
\newtcbtheorem[use counter=tcb@cnt@Theorem]{Corollary}{Corollaire}
{%
	enhanced
	,breakable
	,colback = myp!10
	,frame hidden
	,boxrule = 0sp
	,borderline west = {2pt}{0pt}{myp!85!black}
	,sharp corners
	,detach title
	,before upper = \tcbtitle\par\smallskip
	,coltitle = myp!85!black
	,fonttitle = \bfseries\sffamily
	,description font = \mdseries
	,separator sign none
	,segmentation style={solid, myp!85!black}
}
{th}

%================================
% LENMA
%================================

\tcbuselibrary{theorems,skins,hooks}
\newtcbtheorem[use counter=tcb@cnt@Theorem]{Lemma}{Lemme}
{%
	enhanced,
	breakable,
	colback = mylenmabg,
	frame hidden,
	boxrule = 0sp,
	borderline west = {2pt}{0pt}{mylenmafr},
	sharp corners,
	detach title,
	before upper = \tcbtitle\par\smallskip,
	coltitle = mylenmafr,
	fonttitle = \bfseries\sffamily,
	description font = \mdseries,
	separator sign none,
	segmentation style={solid, mylenmafr},
}
{th}


%================================
% PROPOSITION
%================================

\tcbuselibrary{theorems,skins,hooks}
\newtcbtheorem[use counter=tcb@cnt@Theorem]{Prop}{Proposition}
{%
	enhanced,
	breakable,
	colback = mypropbg,
	frame hidden,
	boxrule = 0sp,
	borderline west = {2pt}{0pt}{mypropfr},
	sharp corners,
	detach title,
	before upper = \tcbtitle\par\smallskip,
	coltitle = mypropfr,
	fonttitle = \bfseries\sffamily,
	description font = \mdseries,
	separator sign none,
	segmentation style={solid, mypropfr},
}
{th}


%================================
% CLAIM
%================================

\tcbuselibrary{theorems,skins,hooks}
\newtcbtheorem[use counter=tcb@cnt@Theorem]{claim}{Claim}
{%
	enhanced
	,breakable
	,colback = myg!10
	,frame hidden
	,boxrule = 0sp
	,borderline west = {2pt}{0pt}{myg}
	,sharp corners
	,detach title
	,before upper = \tcbtitle\par\smallskip
	,coltitle = myg!85!black
	,fonttitle = \bfseries\sffamily
	,description font = \mdseries
	,separator sign none
	,segmentation style={solid, myg!85!black}
}
{th}



%================================
% Exercise
%================================

\tcbuselibrary{theorems,skins,hooks}
\newtcbtheorem[use counter=tcb@cnt@Theorem]{Exercise}{Exercice}
{%
	enhanced,
	breakable,
	colback = myexercisebg,
	frame hidden,
	boxrule = 0sp,
	borderline west = {2pt}{0pt}{myexercisefg},
	sharp corners,
	detach title,
	before upper = \tcbtitle\par\smallskip,
	coltitle = myexercisefg,
	fonttitle = \bfseries\sffamily,
	description font = \mdseries,
	separator sign none,
	segmentation style={solid, myexercisefg},
}
{th}

%================================
% EXAMPLE BOX
%================================

\newtcbtheorem[use counter=tcb@cnt@Theorem]{Example}{Exemple}
{%
	colback = myexamplebg
	,breakable
	,colframe = myexamplefr
	,coltitle = myexampleti
	,boxrule = 1pt
	,sharp corners
	,detach title
	,before upper=\tcbtitle\par\smallskip
	,fonttitle = \bfseries
	,description font = \mdseries
	,separator sign none
	,description delimiters parenthesis
}
{ex}

%================================
% DEFINITION BOX
%================================

\newtcbtheorem[use counter=tcb@cnt@Theorem]{Definition}{Définition}{enhanced,
	before skip=2mm,after skip=2mm, colback=red!5,colframe=red!80!black,boxrule=0.5mm,
	attach boxed title to top left={xshift=1cm,yshift*=1mm-\tcboxedtitleheight}, varwidth boxed title*=-3cm,
	boxed title style={frame code={
					\path[fill=tcbcolback]
					([yshift=-1mm,xshift=-1mm]frame.north west)
					arc[start angle=0,end angle=180,radius=1mm]
					([yshift=-1mm,xshift=1mm]frame.north east)
					arc[start angle=180,end angle=0,radius=1mm];
					\path[left color=tcbcolback!60!black,right color=tcbcolback!60!black,
						middle color=tcbcolback!80!black]
					([xshift=-2mm]frame.north west) -- ([xshift=2mm]frame.north east)
					[rounded corners=1mm]-- ([xshift=1mm,yshift=-1mm]frame.north east)
					-- (frame.south east) -- (frame.south west)
					-- ([xshift=-1mm,yshift=-1mm]frame.north west)
					[sharp corners]-- cycle;
				},interior engine=empty,
		},
	fonttitle=\bfseries,
	title={#2},#1}{def}

%================================
% Solution BOX
%================================

\makeatletter
\newtcbtheorem[use counter=tcb@cnt@Theorem]{question}{Question}{enhanced,
	breakable,
	colback=white,
	colframe=myb!80!black,
	attach boxed title to top left={yshift*=-\tcboxedtitleheight},
	fonttitle=\bfseries,
	title={#2},
	boxed title size=title,
	boxed title style={%
			sharp corners,
			rounded corners=northwest,
			colback=tcbcolframe,
			boxrule=0pt,
		},
	underlay boxed title={%
			\path[fill=tcbcolframe] (title.south west)--(title.south east)
			to[out=0, in=180] ([xshift=5mm]title.east)--
			(title.center-|frame.east)
			[rounded corners=\kvtcb@arc] |-
			(frame.north) -| cycle;
		},
	#1
}{def}
\makeatother

%================================
% SOLUTION BOX
%================================

\makeatletter
\newtcolorbox{solution}{enhanced,
	breakable,
	colback=white,
	colframe=myg!80!black,
	attach boxed title to top left={yshift*=-\tcboxedtitleheight},
	title=Solution,
	boxed title size=title,
	boxed title style={%
			sharp corners,
			rounded corners=northwest,
			colback=tcbcolframe,
			boxrule=0pt,
		},
	underlay boxed title={%
			\path[fill=tcbcolframe] (title.south west)--(title.south east)
			to[out=0, in=180] ([xshift=5mm]title.east)--
			(title.center-|frame.east)
			[rounded corners=\kvtcb@arc] |-
			(frame.north) -| cycle;
		},
}
\makeatother

%================================
% Question BOX
%================================

\makeatletter
\newtcbtheorem[use counter=tcb@cnt@Theorem]{qstion}{Question}{enhanced,
	breakable,
	colback=white,
	colframe=mygr,
	attach boxed title to top left={yshift*=-\tcboxedtitleheight},
	fonttitle=\bfseries,
	title={#2},
	boxed title size=title,
	boxed title style={%
			sharp corners,
			rounded corners=northwest,
			colback=tcbcolframe,
			boxrule=0pt,
		},
	underlay boxed title={%
			\path[fill=tcbcolframe] (title.south west)--(title.south east)
			to[out=0, in=180] ([xshift=5mm]title.east)--
			(title.center-|frame.east)
			[rounded corners=\kvtcb@arc] |-
			(frame.north) -| cycle;
		},
	#1
}{def}
\makeatother

\newtcbtheorem[number within=chapter]{wconc}{Wrong Concept}{
	breakable,
	enhanced,
	colback=white,
	colframe=myr,
	arc=0pt,
	outer arc=0pt,
	fonttitle=\bfseries\sffamily\large,
	colbacktitle=myr,
	attach boxed title to top left={},
	boxed title style={
			enhanced,
			skin=enhancedfirst jigsaw,
			arc=3pt,
			bottom=0pt,
			interior style={fill=myr}
		},
	#1
}{def}



%================================
% NOTE BOX
%================================

\usetikzlibrary{arrows,calc,shadows.blur}
\tcbuselibrary{skins}
\newtcolorbox{note}[1][]{%
	enhanced jigsaw,
	colback=gray!20!white,%
	colframe=gray!80!black,
	size=small,
	boxrule=1pt,
	title=\colorbox{white!100}{\textbf{ Remarque }},
	halign title=flush center,
	coltitle=black,
	breakable,
	drop shadow=black!50!white,
	attach boxed title to top left={xshift=1cm,yshift=-\tcboxedtitleheight/2,yshifttext=-\tcboxedtitleheight/2},
	minipage boxed title=2.6cm,
	boxed title style={%
			colback=white,
			size=fbox,
			boxrule=1pt,
			boxsep=2pt,
			underlay={%
					\coordinate (dotA) at ($(interior.west) + (-0.5pt,0)$);
					\coordinate (dotB) at ($(interior.east) + (0.5pt,0)$);
					\begin{scope}
						\clip (interior.north west) rectangle ([xshift=3ex]interior.east);
						\filldraw [white, blur shadow={shadow opacity=60, shadow yshift=-.75ex}, rounded corners=2pt] (interior.north west) rectangle (interior.south east);
					\end{scope}
					\begin{scope}[gray!80!black]
						\fill (dotA) circle (2pt);
						\fill (dotB) circle (2pt);
					\end{scope}
				},
		},
	#1,
}

%================================
% STRATÉGIE BOX
%================================

\usetikzlibrary{arrows,calc,shadows.blur}
\tcbuselibrary{skins}
\newtcolorbox{strategy}[1][]{%
	enhanced jigsaw,
	colback=myb!20!white,%
	colframe=gray!80!black,
	size=small,
	boxrule=1pt,
	title=\colorbox{white!100}{\textbf{ Stratégie }},
	halign title=flush center,
	coltitle=black,
	breakable,
	drop shadow=black!50!white,
	attach boxed title to top left={xshift=1cm,yshift=-\tcboxedtitleheight/2,yshifttext=-\tcboxedtitleheight/2},
	minipage boxed title=2.5cm,
	boxed title style={%
			colback=white,
			size=fbox,
			boxrule=1pt,
			boxsep=2pt,
			underlay={%
					\coordinate (dotA) at ($(interior.west) + (-0.5pt,0)$);
					\coordinate (dotB) at ($(interior.east) + (0.5pt,0)$);
					\begin{scope}
						\clip (interior.north west) rectangle ([xshift=3ex]interior.east);
						\filldraw [white, blur shadow={shadow opacity=60, shadow yshift=-.75ex}, rounded corners=2pt] (interior.north west) rectangle (interior.south east);
					\end{scope}
					\begin{scope}[gray!80!black]
						\fill (dotA) circle (2pt);
						\fill (dotB) circle (2pt);
					\end{scope}
				},
		},
	#1,
}

%================================
% MÉTHODE BOX
%================================

\usetikzlibrary{arrows,calc,shadows.blur}
\tcbuselibrary{skins}
\newtcolorbox{methode}[1][]{%
	enhanced jigsaw,
	colback=white,%
	colframe=gray!80!black,
	size=small,
	boxrule=1pt,
	title=\textbf{Méthode},
	halign title=flush center,
	coltitle=black,
	breakable,
	drop shadow=black!50!white,
	attach boxed title to top left={xshift=1cm,yshift=-\tcboxedtitleheight/2,yshifttext=-\tcboxedtitleheight/2},
	minipage boxed title=2.5cm,
	boxed title style={%
			colback=white,
			size=fbox,
			boxrule=1pt,
			boxsep=2pt,
			underlay={%
					\coordinate (dotA) at ($(interior.west) + (-0.5pt,0)$);
					\coordinate (dotB) at ($(interior.east) + (0.5pt,0)$);
					\begin{scope}
						\clip (interior.north west) rectangle ([xshift=3ex]interior.east);
						\filldraw [white, blur shadow={shadow opacity=60, shadow yshift=-.75ex}, rounded corners=2pt] (interior.north west) rectangle (interior.south east);
					\end{scope}
					\begin{scope}[gray!80!black]
						\fill (dotA) circle (2pt);
						\fill (dotB) circle (2pt);
					\end{scope}
				},
		},
	#1,
}

%%%%%%%%%%%%%%%%%%%%%%%%%%%%%%%%%%%%%%%%%%%
% TABLE OF CONTENTS
%%%%%%%%%%%%%%%%%%%%%%%%%%%%%%%%%%%%%%%%%%%

\usepackage{tikz}

\definecolor{doc}{RGB}{0,60,110}
\usepackage{titletoc}
\contentsmargin{0cm}
\titlecontents{chapter}[3.7pc]
{\addvspace{30pt}%
	\begin{tikzpicture}[remember picture, overlay]%
		\draw[fill=doc!60,draw=doc!60] (-7,-.1) rectangle (-0.2,.6);%
		\pgftext[left,x=-3.5cm,y=0.2cm]{\color{white}\Large\sc\bfseries Chapitre\ \thecontentslabel};%
	\end{tikzpicture}\color{doc!60}\large\sc\bfseries}%
{}
{}
{\;\titlerule\;\large\sc\bfseries Page \thecontentspage
	\begin{tikzpicture}[remember picture, overlay]
		\draw[fill=doc!60,draw=doc!60] (2pt,0) rectangle (4,0.1pt);
	\end{tikzpicture}}%
\titlecontents{section}[3.7pc]
{\addvspace{2pt}}
{\contentslabel[\thecontentslabel]{2pc}}
{}
{\hfill\small \thecontentspage}
[]
\titlecontents*{subsection}[3.7pc]
{\addvspace{-1pt}\small}
{}
{}
{\ --- \small\thecontentspage}
[ \textbullet\ ][]

\makeatletter
\renewcommand{\tableofcontents}{%
	\chapter*{%
	  \vspace*{-20\p@}%
	  \begin{tikzpicture}[remember picture, overlay]%
		  \pgftext[right,x=15cm,y=0.2cm]{\color{doc!60}\Huge\sc\bfseries \contentsname};%
		  \draw[fill=doc!60,draw=doc!60] (13,-.75) rectangle (20,1);%
		  \clip (13,-.75) rectangle (20,1);
		  \pgftext[right,x=15cm,y=0.2cm]{\color{white}\Huge\sc\bfseries \contentsname};%
	  \end{tikzpicture}}%
	\@starttoc{toc}}
\makeatother


%%%%%%%%%%%%%%%%%%%%%%%%%%%%%%%%%%%%%%%%%%%
% MINTED FOR PYTHON ALGORITHMS
%%%%%%%%%%%%%%%%%%%%%%%%%%%%%%%%%%%%%%%%%%%

\usepackage{tcolorbox}
\tcbuselibrary{minted,breakable,xparse,skins}
\definecolor{bg}{gray}{0.95}
\DeclareTCBListing{mintedbox}{O{}m!O{}}{%
  breakable=true,
  listing engine=minted,
  listing only,
  minted language=#2,
  minted style=default,
  minted options={%
    linenos,
    gobble=0,
    breaklines=true,
    breakafter=,,
    fontsize=\small,
    numbersep=8pt,
    #1},
  boxsep=0pt,
  left skip=0pt,
  right skip=0pt,
  left=25pt,
  right=0pt,
  top=3pt,
  bottom=3pt,
  arc=5pt,
  leftrule=0pt,
  rightrule=0pt,
  bottomrule=2pt,
  toprule=2pt,
  colback=bg,
  colframe=orange!70,
  enhanced,
  overlay={%
    \begin{tcbclipinterior}
    \fill[orange!20!white] (frame.south west) rectangle ([xshift=20pt]frame.north west);
    \end{tcbclipinterior}},
  #3}
  
  
 % for braces
\usetikzlibrary{decorations.pathreplacing}

%%%%%%%%%%%%%%%%%%%%%%%%%%%%%%
% SELF MADE COMMANDS
%%%%%%%%%%%%%%%%%%%%%%%%%%%%%%


\newcommand{\thm}[3]{\begin{Theorem}[label=#3]{#1}{}#2\end{Theorem}}
\newcommand{\cor}[3]{\begin{Corollary}[label=#3]{#1}{}#2\end{Corollary}}
\newcommand{\lem}[3]{\begin{Lemma}[label=#3]{#1}{}#2\end{Lemma}}
\newcommand{\mprop}[3]{\begin{Prop}[label=#3]{#1}{}#2\end{Prop}}
\newcommand{\clm}[3]{\begin{claim}{#1}{#2}#3\end{claim}}
\newcommand{\wc}[2]{\begin{wconc}{#1}{}\setlength{\parindent}{1cm}#2\end{wconc}}
\newcommand{\thmcon}[2]{\begin{Theoremcon}[label=#2]{#1}\end{Theoremcon}}
\newcommand{\ex}[3]{\begin{Example}[label=#3]{#1}{}#2\end{Example}}
\newcommand{\exe}[2]{\begin{Exercise}{#1}{}#2\end{Exercise}}
\newcommand{\dfn}[2]{\begin{Definition}[colbacktitle=red!75!black]{#1}{}#2\end{Definition}}
\newcommand{\dfnc}[2]{\begin{definition}[colbacktitle=red!75!black]{#1}{}#2\end{definition}}
\newcommand{\qs}[2]{\begin{question}{#1}{}#2\end{question}}
\newcommand{\pf}[2]{\begin{myproof}[#1]#2\end{myproof}}
\newcommand{\nt}[1]{\begin{note}#1\end{note}}
\newcommand{\str}[1]{\begin{strategy}#1\end{strategy}}



\newcommand*\circled[1]{\tikz[baseline=(char.base)]{
		\node[shape=circle,draw,inner sep=1pt] (char) {#1};}}
\newcommand\getcurrentref[1]{%
	\ifnumequal{\value{#1}}{0}
	{??}
	{\the\value{#1}}%
}
\newcommand{\getCurrentSectionNumber}{\getcurrentref{section}}
\newenvironment{myproof}[1][\proofname]{%
	\proof[\bfseries #1: ]%
}{\endproof}

\newcommand{\mclm}[2]{\begin{myclaim}[#1]#2\end{myclaim}}
\newenvironment{myclaim}[1][\claimname]{\proof[\bfseries #1: ]}{}

\newcounter{mylabelcounter}

\makeatletter
\newcommand{\setword}[2]{%
	\phantomsection
	#1\def\@currentlabel{\unexpanded{#1}}\label{#2}%
}
\makeatother




\tikzset{
	symbol/.style={
			draw=none,
			every to/.append style={
					edge node={node [sloped, allow upside down, auto=false]{$#1$}}}
		}
}


% deliminators
\DeclarePairedDelimiter{\abs}{\lvert}{\rvert}
\DeclarePairedDelimiter{\norm}{\lVert}{\rVert}

\DeclarePairedDelimiter{\ceil}{\lceil}{\rceil}
\DeclarePairedDelimiter{\floor}{\lfloor}{\rfloor}
\DeclarePairedDelimiter{\round}{\lfloor}{\rceil}

\newsavebox\diffdbox
\newcommand{\slantedromand}{{\mathpalette\makesl{d}}}
\newcommand{\makesl}[2]{%
\begingroup
\sbox{\diffdbox}{$\mathsurround=0pt#1\mathrm{#2}$}%
\pdfsave
\pdfsetmatrix{1 0 0.2 1}%
\rlap{\usebox{\diffdbox}}%
\pdfrestore
\hskip\wd\diffdbox
\endgroup
}
\newcommand{\dd}[1][]{\ensuremath{\mathop{}\!\ifstrempty{#1}{%
\slantedromand\@ifnextchar^{\hspace{0.2ex}}{\hspace{0.1ex}}}%
{\slantedromand\hspace{0.2ex}^{#1}}}}
\ProvideDocumentCommand\dv{o m g}{%
  \ensuremath{%
    \IfValueTF{#3}{%
      \IfNoValueTF{#1}{%
        \frac{\dd #2}{\dd #3}%
      }{%
        \frac{\dd^{#1} #2}{\dd #3^{#1}}%
      }%
    }{%
      \IfNoValueTF{#1}{%
        \frac{\dd}{\dd #2}%
      }{%
        \frac{\dd^{#1}}{\dd #2^{#1}}%
      }%
    }%
  }%
}
\providecommand*{\pdv}[3][]{\frac{\partial^{#1}#2}{\partial#3^{#1}}}
%  - others
\DeclareMathOperator{\Lap}{\mathcal{L}}
\DeclareMathOperator{\Var}{Var} % varience
\DeclareMathOperator{\Cov}{Cov} % covarience
\DeclareMathOperator{\E}{E} % expected

% Since the amsthm package isn't loaded

% I prefer the slanted \leq
\let\oldleq\leq % save them in case they're every wanted
\let\oldgeq\geq
\renewcommand{\leq}{\leqslant}
\renewcommand{\geq}{\geqslant}

% PLUS INFTY AND MINUS INFTY WITH NO SPACE

\newcommand{\pinfty}{{+}\infty}
\newcommand{\minfty}{{-}\infty}


% Schwartz
\renewcommand{\S}{\mathcal{S}} % \S est le signe paragraphe normalement

% corps
\newcommand{\C}{\mathcal{C}}
\newcommand{\R}{\mathbb{R}}
\newcommand{\Rnn}{\mathbb{R}^{2n}}
\newcommand{\Z}{\mathbb{Z}}
\newcommand{\N}{\mathbb{N}}
\newcommand{\Q}{\mathbb{Q}}

% domain
\newcommand{\D}{\mathcal{D}}

% order notations
\renewcommand{\O}{\mathcal{O}}

% japanese bracket
\newcommand{\japb}[1]{\langle #1 \rangle}

% arrows over partial derivatives
\newcommand{\lp}{\overleftarrow{\partial}}
\newcommand{\rp}{\overrightarrow{\partial}}

% quantization
\newcommand{\h}{\hbar}
\newcommand{\Opht}{\textrm{Op}_{\h}^{t}}
\newcommand{\Op}[2][\hbar]{\textrm{Op}_{#1}^{#2}}

% omega functions
\newcommand{\omegap}[2][\rho_0]{\omega(\partial_{#1},\partial_{#2})}
\newcommand{\omegar}[2][\rho_0]{\omega(#1,#2)}

\title{\Huge{Mathématiques en classe de Seconde}\\ Année 2024-2025}
\author{\huge{Matthieu Haeberle}}
\date{\today}

\begin{document}

\maketitle
\newpage% or \cleardoublepage
% \pdfbookmark[<level>]{<title>}{<dest>}
\pdfbookmark[section]{\contentsname}{toc}
\tableofcontents
\pagebreak

%!TEX encoding = UTF8
%!TEX root =notes.tex

\chapter{Ensembles dénombrables}

	\dfn{Ensemble de nombres}{
		On note un ensemble de nombres par des accolades \{  $\dots$ \}  entourant une liste de nombres.
	}
	\ex{}{
		Les ensembles suivants sont des ensembles de nombres.
		\begin{multicols}{3}
			$E_1 = \{ 1 ;  2 ; 3\}$, \\
			$E_2 = \left\{ -1,2 ; 4 ; \dfrac{7}{12} \right\}$, \\
			$E_3 = \{ 2\pi \}$.
		\end{multicols}
		Les nombres appartenant à un ensembles sont appelé les \emph{éléments} de l'ensemble.
		Tous les éléments sont distincts.
	}

	\dfn{Entiers et rationnels}{
		On définit les ensembles infinis suivants.
		\begin{multicols}{2}
		\begin{enumerate}
			\item $\N = \{ 0 ; 1 ; 2 ; 3 ; \dots \}$,
			\item $\Z = \{ 0 ; 1 ; -1 ; 2 ; -2 ; 3 ; -3 ; \dots \}$, 
			\item $\D = \left\{ \dfrac{a}{10^n} \text{ tel que : } a \in \Z, n \in \N \right\}$,
			\item $\Q = \left\{ \dfrac{a}b \text{ tel que : } a, b \in \Z \right\}$.
		\end{enumerate}
		\end{multicols}
		Ils sont respectivement : les entiers \emph{naturels}, les entiers \emph{relatifs}, les nombres décimaux, et les nombres \emph{rationnels}.
	}
	
	\nt{
		Les décimaux $\D$ ont un nombre de décimales fini. En effet, la fraction $\dfrac{a}{10^n}$ pour $a \in \Z$ admet au plus $n$ décimales après la virgules.
	}
	
	\dfn{Appartenance, inclusion}{
		On pose les symboles suivants pour signifier l'appartenance et l'inclusion.
		\begin{itemize}
			\item $a \in E$ : l'élément $a$ appartient à l'ensemble $E$.
			\item $E \subset S$ : l'ensemble $E$ est inclus dans l'ensemble $S$.
		\end{itemize}	
	}
	\nt{
		Pour que $E \subset S$, tous les éléments de $E$ doivent aussi appartenir à $S$.
	}
	
	\ex{Appartenances et inclusions}{
		\begin{multicols}{3}
		\begin{itemize}
			\item  $2 \in \N$,
			\item $2/3 \in \Q$,
			\item $0{,}7 \in \D$,
			\item $ \left\{ 0 ; -3 ; \dfrac{32}{8} ; 1 \right\} \subset \Z$,
			\item $\N \subset \Z$,
			\item $\Z \subset \Z$.
		\end{itemize}
		\end{multicols}		
		On a la suite d'inclusions $\N \subset \Z \subset \D \subset \Q$.
	}
	
	\ex{Non appartenances et non inclusions}{
		\begin{multicols}{3}
		\begin{itemize}
			\item  $-2 \notin \N$,
			\item $\dfrac13 \notin \D$, 
			\item $2 \pi \notin \Q$,
			\item $ \{ 0 ; -1,2 ; 4 \}  \not\subset \Z$,
			\item $ \left\{ \dfrac{4}{13}; -1,5 ; 2,75 \right\} \not\subset \D$.
		\end{itemize}
		\end{multicols}	
	}
	
	
	
	\qs{}{
		Le rationnel $\dfrac13$ n'appartient pas à $\D$ car ses décimales se répètent infiniment. Qu'en est-il de $\dfrac17$ ou $\dfrac{5}{12}$ ?
		
		
		Le but du prochain chapitre est de démontrer rigoureusement que les fractions du type $\dfrac{5}{12}$ ne sont pas décimales : leur écriture est infinie.
	}
	\qs{}{
		Soit $\sqrt{2}$ le nombre positif qui vérifie $\left(\sqrt{2}\right)^2 = 2$. Est-ce que $\sqrt{2}$ est rationnel ?
		
		Nous démontrerons également que $\sqrt{2}$ n'est pas un nombre rationnel.
	}

%!TEX encoding = UTF8
%!TEX root =notes.tex

\chapter{Arithmétique sur $\N$}

	\section{Diviseurs et multiples}

	\dfn{Diviseur, multiple}{
		Pour $d , n\in \N$ deux entiers naturels, on dit que 
			\[ d \textbf{ divise } n \in \N \]
		dès que $n$ s'écrit de la forme
			\[ n = d \cdot k \]
		pour un entier naturel $k \in \N$.
		On écrit alors 
			\[ d \ | \ n, \]
		et on dit également que $n$ est un \textbf{multiple} de $d$.
	}
	
	\ex{Nombre pair, impair}{
		 Les nombres $n$ \textbf{pairs} sont les multiples de $2$. Ils s'écrivent donc 
		 	\[ n = 2 k \]
		 pour un entier naturel $k \in \N$.
		 
		 Les nombres $n$ \textbf{impairs} se situent juste après un nombre pair et s'écrivent alors
		 	\[ n = 2 k + 1 \]
		pour un entier naturel $k \in \N$.
	}{}
	
	\qs{}{
		Soit $a \in \N$ un entier naturel.
		L'entier $ (2a + 1)^2 - 1$ est-il toujours pair ?
		
		\emph{Rappel : on a l'identité remarquable $(a+b)(a-b) = a^2 - b^2$, pour $a, b$ des nombres quelconques.}
	}
	
	\ex{Ensembles de diviseurs}{
		L'ensemble des diviseurs de $24$ est donné par
			\[ \mathcal{D}_{24} = \left\{ 1 ; 2 ; 3 ; 4 ; 6 ; 8 ; 12 ; 24 \right\}. \]
		L'ensemble des multiples de $17$ inférieurs ou égaux à $100$ est donné par
			\[ \mathcal{M} = \left\{ 0 ; 17 ; 34 ; 51 ; 68 ; 85 \right\}. \]
	}{}
	
	\nt{
		Les diviseurs se regoupent par paires : en effet, si $d$ divise $n$ et que $n = d \cdot k$, alors $k$ divise aussi $n$.
	}
	
	\qs{}{
		Étant donné que $37$ divise $111$. Montrer que $37$ divise alors aussi $555$.
		
		Plus généralement, montrer que si $a | b$, alors $a$ divise aussi tous les multiples de $b$.
	}
	
\section{Nombres premiers}

	\dfn{Nombre premier}{
		Pour $p \in \N$, $p \geq 2$, un entier naturel. On dit que $p$ est \textbf{premier} si ses seuls diviseurs sont $1$ et lui-même.
	}
	
	\ex{}{
		Les premiers nombres premiers sont
			\[ \{ 2 ; 3 ; 5 ; 7 ; 11 ; 13 ; \dots \}. \]
		Il y en a une infinité.
	}{}
	
	
	\thm{Théorème fondamental de l'arithmétique}{
		Tout entier $n \in \N$, $n \geq 2$ peut s'écrire de façon unique comme produit de nombres premiers.
	}{}
	
	
	\ex{}{
		\begin{multicols}{2}
		\begin{itemize}[label=$\bullet$]
			\item $32 = 2^5$
			\item $9 = 3^2$
			\item $24 = 2^3 \cdot 3$
			\item $110 = 2 \cdot 5 \cdot 11$
			\item $10^n = 2^n \cdot 5^n$
			\item $10^n \cdot 30^m = 3^m \cdot 2^{m+n} \cdot 5^{m+n}$
		\end{itemize}
		\end{multicols}	
	}{}
	
	\exe{}{
		Écrire la décomposition en produit de facteurs premiers des entiers suivants.
		
		\begin{multicols}{4}
		\begin{itemize}[label=$\bullet$]
			\item $33$
			\item $48$
			\item $110 \times 55$
			\item $35 \times 90$
		\end{itemize}
		\end{multicols}
	}{}
	
	\mprop{ }{ 
		Le rationnel $\dfrac17 \in \Q$ n'est pas un nombre décimal : $\dfrac17 \notin \D$.
	}{}
	
	\pf{Démonstration par l'absurde }{
		La preuve se décline comme suit.
		
		\begin{enumerate}
			\item Supposons, par l'absurde, que $\dfrac17 \in \D$.
			\item Par définition, $\D  = \left\{ \dfrac{a}{10^n} \text{ tel que : } a \in \Z, n \in \N \right\}$.
			\item Donc $\dfrac17$ s'écrit $\dfrac{a}{10^n}$ pour certains $a \in Z$ et $n \in \N$.
			\item D'où $\dfrac17 =  \dfrac{a}{10^n}$, et par suite $10^n = 7 \cdot a$. C'est une égalité de deux entiers naturels.
			\item À droite : $a$ étant un entier et $7$ étant premier, la décomposition en produit de premiers de $7 \cdot a$ contient $7$.
			\item Cependant, à gauche, $7$ n'apparaît pas dans la décomposition en produit de premiers de $10^n = 2^5 \cdot 5^n$.
			\item L'égalité obtenue ne peut donc pas être vraie : ceci est une contradition, et $\dfrac17 \notin \D$.
		\end{enumerate}
	}

	\qs{}{
		Refaire la démonstration pour $\dfrac5{12} \notin \D$.
	}
	
	\nt{
		Si un entier naturel $d \in \N$ divise un autre entier naturel $n \in \N$, décomposition en facteurs premiers de la relation
			\[ n = d \cdot k \]
		où $k\in\N$ permet de dire la chose suivante.
		
		La puissance d'un premier $p$ dans la décompoistion de $d$ est inférieure ou égale à sa puissance dans la décomposition de $n$.
	}
	
	\mprop{}{
		Considérons un $n \in \N$.
		
		Si $n$ est pair, alors tous les multiples de$n$ sont pairs.
	}{prop:1}
	
	\exe{}{
		Cette proposition est le cas $a=2$ de la proposition suivante à démontrer.
		
		Si $a|b$, alors $a$ divise tous les multiples de $b$.
	}{}
	
\section{Coprimalité}
	
	\nt{
		Si le numérateur et le dénominateur partagent un diviseur commun, il peut s'annuler
			$\dfrac{a \cdot d}{b \cdot d} = \dfrac{a}b$.
		On dit alors qu'on réduit la fraction.
	}
	
	\dfn{Coprimalité et fractions irréductibles}{
		Deux entiers naturels $a$ et $b$ de $\N$ sont \textbf{premiers entre eux} si leur seul diviseur commun est $1$.
		
		La fraction $\dfrac{a}b$ est alors \textbf{irréductible}.
	}
	
	\exe{}{
		Donner l'ensemble des diviseurs communs à $100 \times 121$ et $44 \times 55$.
		Réduire la fraction $\dfrac{100 \times 121}{44 \times 55}$.
	}{}
	
	\thm{}{
		Soit $\sqrt{2}$ le nombre positif qui vérifie $\left(\sqrt{2}\right)^2 = 2$.
		Alors $\sqrt{2}$ est irrationnel : $\sqrt{2} \notin \Q$.
	}{thm:1}
	
	
	\lem{}{
		Considérons un entier naturel $n\in\N$.
		
		Si $n$ est impair, alors $n^2$ est impair.
	}{lem:1}
	
	\pf{Démonstration du lemme \ref{lem:1}}{
		Si $n$ est impair, et pour montrer que $n^2$ est impair, il suffit de montrer que $n^2 -1$ est pair.
		
		Or 
			\[ n^2 - 1 = (n+1)(n-1), \]
		Comme $n$ est impair, $n-1$ est pair, et $n^2 - 1$ est un multiple d'un nombre pair.
		
		D'après la proposition \ref{prop:1}, $n^2 -1 $ est pair, et donc $n^2$ est impair.
	}
	
	\lem{Constraposition du lemme \ref{lem:1}}{
		Considérons un entier naturel $n\in\N$.
		
		Si $n^2$ est pair, alors $n$ est pair.
	}{lem:1bis}
	
	\pf{Preuve du lemme \ref{lem:1bis}}{
		C'est la \emph{contraposition} du lemme \ref{lem:1}.
	}
	
	\pf{Preuve du théorème \ref{thm:1}  }{
		La démonstration est à nouveau par l'absurde.
		
		\begin{enumerate}
			\item Supposons, par l'absurde, que $\sqrt{2} \in \Q$ soit rationnel.
			
			\item Par définition $\Q = \left\{ \dfrac{a}b \text{ tq. } a \in \Z, b\in\Z, b \neq 0 \right\}.$
			
			\item Donc $\sqrt{2}$ s'écrit $\dfrac{a}{b}$ pour certains $a,b \in \Z$, $b \neq 0$.

			\item Comme $\sqrt{2}$ est positif, et en simplifiant la fraction, on peut écrire $\sqrt{2} = \dfrac{a}{b} = \dfrac{p}{q}$ fraction irréductible avec $p, q \in \N$, $q \neq 0$.
			
			\item Par définition de $\sqrt{2}$, 
				\[ \sqrt{2}^2 = 2 = \left( \dfrac{p}q \right)^2 = \dfrac{p^2}{q^2}. \]
				
			\item D'où l'égalité d'entiers naturels $p^2 = 2 q^2$. 
				L'entier $p^2$ est donc pair, et $p$ doit l'être aussi d'après le lemme \ref{lem:1bis}.
			
			\item $p$ est multiple de $2$ et s'écrit alors comme 
				\[ p = 2 k, \]
				pour un entier naturel $k \in \N$.
				
				En substituant dans l'équation $p^2 = 2 q^2$, on trouve
					\[ (2k)^2 = 2 q^2 \qquad \iff \qquad q^2 = 2 k^2.\]
			\item D'après le lemme \ref{lem:1bis}, $q$ est pair.
			\item Ceci est une contradiction, car la fraction $\dfrac{p}q$ a été choisie irréductible, alors que $p$ et $q$ sont tous les deux pairs. 
				Finalement, $\sqrt{2} \notin \Q$.
		\end{enumerate}
	}

%!TEX encoding = UTF8
%!TEX root =notes.tex

\chapter{Droite réelle et géométrie plane}

\section{Droite réelle}
	

On considère la droite graduée suivante.

\begin{center}
	
	\begin{tikzpicture}
		
		% real line
		\draw[<->, thick] (-7,0) node[below] {$-\infty$} -- (7,0) node[below] {$+\infty$};
		
		\foreach \x in {-2,...,2}
			\draw[-] (1.3*\x,-.1) node[below] {$\x$} -- (1.3*\x,.1) node{};
		
		\foreach \x in {-3, 3}
			\draw[-] (1.3*\x,-.1) node[below=3pt] {$\dots$} -- (1.3*\x,.1) node{};
		
	
	\end{tikzpicture}
\end{center}

On peut construire à la règle et au compas le nombre $\sqrt{2}$ sur la droite grâce au théorème de Pythagore.

\exe{}{
	Construire, en utilisant uniquement la règle et le compas, tous les nombres rationnels, c'est-à-dire toutes les fractions $\dfrac{p}q$ où $p, q \in \Z$ et $q \neq 0$.
	
	\emph{Indication : les couples $(1, p/q)$ et $(q, p)$ sont proportionnels.}
}{}

L'ensemble des points de la droite est donc strictement plus grand que celui des rationnels, car $\sqrt{2}$ n'est pas rationnel d'après le théorème \ref{thm:1}.

\begin{center}
	
	\begin{tikzpicture}
		
		% real line
		\draw[<->, thick] (-7,0) node[below] {$-\infty$} -- (7,0) node[below] {$+\infty$};
		\foreach \x in {-1, 0, 1}
			\draw[-] (1.3*\x,-.1) node[below] {$\x$} -- (1.3*\x,.1);
		
		% right angled triangle
		\draw[-, thick] (1.3,0) -- (1.3,1);
		\draw[-, thick] (0,0)-- (1.3,1);
		
		\draw[-, thick] (1.1,0)-- (1.1,.2);
		\draw[-, thick] (1.1,.2)-- (1.3,.2);
		
		% lengths
		\draw[decoration={brace,raise=5pt},decorate]
  			(0,0) -- node[above=12pt, left=4pt] {$\sqrt{2}$} (1.3,1);
		\draw[decoration={brace,mirror,raise=5pt},decorate]
  			(1.3,0) -- node[right=6pt] {$1$} (1.3,1);
  			
  			
		% circle arc
		 \draw [red,thick,domain=-5:50] plot ({sqrt(2.69)*cos(\x)}, {sqrt(2.69)*sin(\x)});
		 
		 % sqrt{2} mark
		\draw[-, red] (1.640,-.1) node[below, red] {$\sqrt{2}$} -- (1.640,.1);
				
	\end{tikzpicture}
\end{center}

Tous les points de la droite ne peuvent en fait pas être construit à la règle et au compas. Ceci est étroitement lié au théorème de Gauss-Wantzel sur les	polygônes réguliers constructibles.
En particulier, l'heptagone régulier n'est pas constructible à la règle et au compas.
	
Ce théorème est en dehors du champ d'application du cours.

\dfn{Nombres réels $\R$}{
	À chaque point de la droite on associe un nombre, appelé \emph{réel}.
	L'ensemble des nombres réels est noté $\R$.
}

\nt{
	On a la suite d'inclusions suivantes :
		\[ \N \subset \Z \subset \D \subset \Q \subset \Q. \]
}

\subsection{Intervalles}

\dfn{Intervalle borné}{
	Un intervalle borné est un segment de la droite réelle $\R$. C'est donc un ensemble de nombres.
	Il est donné par une \emph{borne inférieure} $a \in \R$ et une \emph{borne supérieure} $b\in\R$ et peut contenir ou non ses bornes.
	
	\begin{enumerate}
		\item Si $a$ et $b$ sont contenues dans l'intervalle, on le note $[a ; b]$.
		\item Si $a$  est contenue dans l'intervalle mais $b$ ne l'est pas, on le note $[a ; b [$.
		\item Si $a$ n'est pas contenue dans l'intervalle mais $b$ l'est, on le note $] a ; b]$.
		\item Si ni $a$ ni $b$ ne sont contenues dans l'intervalle, on le note $] a ; b [$.
	\end{enumerate}
}
\ex{}{
	Les intervalles suivants sont bornés.
	\begin{multicols}{2}
	\begin{enumerate}[label=$\bullet$]
		\item $[-1 ; 1]$
		\item $[-3 ; 1[$
		\item $]-10{,}341 ; \pi]$
		\item $]\sqrt{2} ; 130[$
	\end{enumerate}
	\end{multicols}

}{}

\dfn{Intervalle non borné}{
	Un intervalle n'est pas forcément borné : une ou les deux bornes peuvent être infinies ($\pinfty$ ou $\minfty$). 
	Dans ce cas, l'intervalle n'inclut jamais l'infini car ce n'est pas un nombre.
}

\ex{}{
	Les intervalles suivants ne sont pas bornés.
	\begin{multicols}{2}
	\begin{enumerate}[label=$\bullet$]
		\item $]\minfty; 2]$
		\item $]\minfty ; 3[$
		\item $]0; \pinfty[$
		\item $\R = ]\minfty; \pinfty[$
	\end{enumerate}
	\end{multicols}
}{ex:3.5}

\subsection{Intersection et union}

\dfn{Intersection, union}{	
	Pour deux ensembles $E, F$ on définit les ensembles suivants.
		\begin{enumerate}
			\item $E \cap F$ : l'intersection des deux ensembles.
			
			Un élément appartient à $E \cap F$ dès qu'il appartient à $E$ \textbf{et} à $F$.
			\item $E \cup F$ : l'union des deux ensembles.
			
			Un élément appartient à $E \cup F$ dès qu'il appartient à $E$ \textbf{ou} à $F$.
		\end{enumerate}
}


\exe{}{
	Exprimer les intersections et unions suivantes sous forme d'intervalle.
	\begin{multicols}{2}
	\begin{enumerate}
		\item $[-1 ; 1] \cap [-3 ; 1]$
		\item $] {-}\infty ; 2] \cap [-3 ; -2]$
		\item $]{-}\infty ; 4 [ \cup [2 ; +\infty[$
		\item $]-2 ; 4 [  \cup [4 ; 8[$
	\end{enumerate}
	\end{multicols}
}{}

\subsection{Inégalités}

Un intervalle n'a de sens que si la borne inférieure est plus petite que la borne supérieure.
De plus, un élément appartient à l'ensemble dès qu'il est plus petit que la borne supérieure, et plus grand que la borne inférieure.

Il faut donc pouvoir noter simplement les relations \og plus petit que \fg et \og plus grand que \fg.

\dfn{Inégalités}{
	On définit les signes suivants correspondant à des inégalités \emph{strictes} et \emph{larges}.
		\begin{enumerate}
			\item $<$ : strictement inférieur à
			\item $\leq$ : inférieur ou égal à
			\item $>$ : strictement supérieur à
			\item $\geq$ : supérieur ou égal à
		\end{enumerate}
}

\ex{}{
	\begin{multicols}{3}
	\begin{enumerate}
		\item $1 \leq 2$
		\item $-3 \leq -2$
		\item $0 \leq 0$
		\item $1{,}02 < 1{,}1$
		\item $7{,}391 > 7{,}30001$
		\item $-4{,}001 > -4{,}0001$
	\end{enumerate}
	\end{multicols}
}{}

\nt{
	Appartenir à un intervalle est donc équivalent à respecter certaines inégalités.
	En particulier : être inférieur à la borne supérieure, et être supérieur à la borne inférieure.
	
	On a donc l'équivalence suivante.
		\[ x \in [a ; b] \qquad \iff \qquad x \in \R, x \geq a, \textbf{ et } x \leq b. \]
		
	On peut noter deux inégalités sur la même lignes dès qu'elles vont dans le même sens.
	Par exemple, les deux inégalités ci-dessus peuvent être condensées en une :
		\[ x \in \R, \text{ et } a \leq x \leq b, \]
	ou encore
		\[ x \in \R, \text{ et } b \geq x \geq a. \]
}

\ex{}{
	\begin{enumerate}
		\item Prendre $x \in [-3 ; 4]$ est équivalent à prendre $x \in \R$ vérifiant $-3 \leq x \leq 4$.
		\item Prendre $x \in [-4 ; 3[$ est équivalent à prendre $x \in \R$ vérifiant $-4 \leq x < 3$.
		\item Prendre $x \in ]\minfty ; 0[$ est équivalent à prendre $x \in \R$ vérifiant $x < 0$.
		
		On dit alors que $x$ est strictement négatif.
		\item Prendre $x \in [0; \pinfty [$ est équivalent à prendre $x \in \R$ vérifiant $x \geq 0$.
		
		On dit alors que $x$ est positif ou nul.
		\item Prendre $x \in ]\minfty; \pinfty[$ est équivalent à prendre $x \in \R$.	
		
		Ceci est en fait tautologique car $]\minfty; \pinfty[ = \R$, comme vu dans l'exemple \ref{ex:3.5}.
	\end{enumerate}
}{}

\thm{}{
	Soient $x, y, a \in \R$ vérifiant
		\[ x \leq y. \]
	Alors l'inégalité ci-dessus est équivalente aux inégalités suivantes.
		\begin{enumerate}
			\item $x + a \leq y + a$, sans condition sur $a$.
			\item $a x \leq a y$ si $a > 0$ est strictement positif.
			\item $ax \geq a y$ si $a < 0$ est strictement négatif.
		\end{enumerate}
}{thm:ineg}

\nt{
	Le théorème \ref{thm:ineg} reste vrai si on remplace l'inégalité large $\leq$ par une inégalité stricte $<$.
}

\str{
	On décrit ci-après la stratégie de résolutions d'inéquations du type
		\[ a x + b \leq c x + d, \]
	où $a,b,c,d \in \R$ sont des réels quelconques et $x\in\R$ est l'inconnue à placer dans un intervalle.
	
	Comme pour les équations, on procède en deux étapes
		\begin{enumerate}
			\item Isoler les multiples de $x$ d'un côté et les constantes de l'autre.
			\item Diviser pour trouver l'intervalle de $x$.
		\end{enumerate}
	Pour cela, on peut par exemple soustraire $cx$ et $b$ des deux côtés à l'aide du Théorème \ref{thm:ineg}.
		\[ (a-c) x \leq d - b.\]
	Si $a-c \neq 0$, on divise par $a-c$ en faisant attention au signe de celui-ci :
		\[ \begin{cases*} 
						x \leq \dfrac{d-b}{a-c} & si $a-c > 0$, \\
					 	x \geq \dfrac{d-b}{a-c} & si $a-c < 0$.
			\end{cases*} \]
}

\ex{}{
	Essayons de déterminer quels réels $x\in\R$ vérifient l'inégalité suivante.
		\[ 3 x + 1 \geq  8 \]
	Cette inégalité est équivalente à
		\begin{align*}
			& 3x \geq 7,  & \text{puis à} && x\geq \dfrac73,
		\end{align*}			
	car $3$ est strictement positif.
	
	On a donc trouvé que
		\[ \{ x \in \R \text{ tq. } 3x + 1 \geq 8 \} = [\dfrac73 ; \pinfty[. \]
}{}

\ex{}{
	On détermine les réels $x \in \R$ vérifiant la double inégalité suivante.
		\[ 14 > - 7x + 10 \geq -1.\]
	On vérifie d'abord qu'on ait bien $14 > -1$ pour s'assurer que de tels réels $x$ existent bien. 
	On peut par exemple résoudre l'équation $-7x+10 = 2$ pour avoir un exemple d'un tel $x$, car on a bien $14 > 2 \geq -1$.
	
	L'inégalité est équivalente à
		\[ 4 > -7x \geq -11. \]
	On divise ensuite par $-7$, ce qui échange le sens des inégalités car $\dfrac{1}{-7}$ est strictement négatif.
	D'où
		\[ \dfrac{4}{-7} < x \leq  \dfrac{-11}{-7}. \]
	Finalement, on a obtenu que
		\[ \{ x \in \R \text{ tq. } 14 > - 7x + 10 \geq -1 \} = ]-\dfrac47 ; \dfrac{11}7]. \]
}{}

\exe{}{
	Trouver quels réels $x\in\R$ vérifient les deux inégalités suivantes
		\begin{align*} & -\dfrac32x - 2 \geq -4 & \textbf{et} && -\dfrac32x -2 \leq 13 & \end{align*}
	en prenant l'intersection des intervalles obtenus pour chaque inégalité.
	
	Faire de même en manipulant directement la double inégalité
		\[ -4 \leq -\dfrac32x -2 \leq 13, \]
	et vérifier qu'on trouve le même intervalle final.
}{}



\subsection{Valeur absolue}

La valeur absolue correspond à la notion de \emph{distance} parmis les réels.
Celle-ci intervient notamment lorsqu'on souhaite calculer la longueur d'un intervalle borné afin de mesure sa précision.
Par exemple, l'intervalle $[1 ; 3]$, dans lequel $2$ appartient, est moins précis que l'intervalle $[1{,}99 ; 2{,}01]$ car sa longueur est plus grande.

Les intervalles de confiance seront étudiés plus tard dans le chapitre dédié aux statistiques.

\exe{}{
	Quelle est la distance dans les réels 
		\begin{multicols}{2}
		\begin{enumerate}
			\item $3$ et $2$ ?
			\item $2$ et $4$ ?
			\item $-1$ et $1$ ?
			\item $-4{,}2$ et $-4{,}12$ ?
		\end{enumerate}
		\end{multicols}
}{}

\dfn{Valeur absolue}{
	La valeur absolue d'un $x \in \R$ général est égale à $x$ si celui-ci est positif, et à son opposé $-x$ sinon.
	La valeur absolue est donc toujours positive et peut être définie comme suit.
	
		\[ |x| = \begin{cases} x \text{ si $x \geq 0$}, \\ -x \text{ sinon.} \end{cases}. \]

	Soient $a, b \in \R$ deux réels quelconques.
	La distance de $a$ à $b$ est égale à
		\[ \text{Distance entre $a$ et $b$} = \begin{cases} a-b \text{ si $a \geq b$} , \\ b-a \text{ sinon.} \end{cases}. \]
	La condition $a \geq b$ peut être réécrite comme $a-b \geq 0$.
	
	En remarquant que $b-a = -(a-b)$, on voit que la distance entre $a$ et $b$ est exactement donnée par la valeur absolue de $a-b$.
		\[ \text{Distance entre $a$ et $b$} = | a- b|. \]
}

\nt{
	On retrouve d'ailleurs $|x| = |x - 0|$, la distance de $x$ à $0$.
}

\exe{}{
		Quels réels $x\in\R$ vérifient les conditions suivantes ? Donner des ensembles ou des intervalles.
		\begin{multicols}{2}
		\begin{enumerate}
			\item $|x| = 1$
			\item $|2x -4| = 5$
			\item $| x | \leq 3{,}14$
			\item $| 7 + 3x | \leq 10$
		\end{enumerate}
		\end{multicols}
}{}





\section{Géométrie plane}
	

%\chapter{Calcul littéral et logique}
%
%\section{Calcul littéral}
%
%	\dfn{Distributivité des réels}{
%		Tout triplet $(a,b,c)$ de réels vérifie
%		
%		$a \cdot (b + c)  = ab + ac$.
%		
%		$(b+c) \cdot a = ba + ca$.
%	}
%	
%	\thm{Identités remarquables}{
%		On a les identités suivantes, pour $a,b,c$ réels.
%		
%		$(a+b)^2 = a^2 + b^2 + 2ab$
%		
%		$(a - b)^2 = a^2 + b^2 - 2ab$
%		
%		$(a+b)(a-b) = a^2 - b^2$.
%	}
%
%\section{Logique}
%
%	\dfn{Contraposition logique}{
%		$p \implies q$ est équivalent à $ \text{non } q \implies \text{non } p$.
%	}
%	
%	\ex{}{
%		Voir théorème \ref{thm:1}.
%	}
%	
%	\dfn{Preuve par l'absurde}{
%		Pour montrer $p \implies q$, on suppose $p$ vraie et $q$ fausse pour arriver à une contradiction logique.
%	}
%
%	\ex{}{
%		Voir théorème \ref{thm:2}.
%	}
%	
%
%
%\chapter{Algorithmique}
%
%\begin{mintedbox}{python}
%from math import *
%import numpy as np
%from matplotlib import pyplot as plt
%
%a = 3
%
%def fonction(x):
%	return x**2
%\end{mintedbox}
%	
%\chapter{Nombres réels}
%
%\section{Intervalles réels}
%
%
%\section{Manipulations d'inégalités}
%
%
%\section{Valeur absolue}
%
%
%\chapter{Fonctions de référence}
%
%
%\chapter{Vecteurs dans le plan}

\end{document}

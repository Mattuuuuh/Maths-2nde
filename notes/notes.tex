\documentclass[a4paper, 12pt]{report}

% DYSLEXIA SWITCH
\newif\ifdys
		
				% ENABLE or DISABLE font change
				% use XeLaTeX if true
				\dystrue
				\dysfalse


\ifdys

\documentclass[a4paper, 14pt]{extarticle}
\usepackage{amsmath,amsfonts,amsthm,amssymb,mathtools}

\tracinglostchars=3 % Report an error if a font does not have a symbol.
\usepackage{fontspec}
\usepackage{unicode-math}
\defaultfontfeatures{ Ligatures=TeX,
                      Scale=MatchUppercase }

\setmainfont{OpenDyslexic}[Scale=1.0]
\setmathfont{Fira Math} % Or maybe try KPMath-Sans?
\setmathfont{OpenDyslexic Italic}[range=it/{Latin,latin}]
\setmathfont{OpenDyslexic}[range=up/{Latin,latin,num}]

\else

\documentclass[a4paper, 12pt]{extarticle}

\usepackage[utf8x]{inputenc}
%fonts
\usepackage{amsmath,amsfonts,amsthm,amssymb,mathtools}
% comment below to default to computer modern
\usepackage{libertinus,libertinust1math}

\fi


\usepackage[french]{babel}
\usepackage[
a4paper,
margin=2cm,
nomarginpar,% We don't want any margin paragraphs
]{geometry}
\usepackage{icomma}

\usepackage{fancyhdr}
\usepackage{array}
\usepackage{hyperref}

\usepackage{multicol, enumerate}
\newcolumntype{P}[1]{>{\centering\arraybackslash}p{#1}}


\usepackage{stackengine}
\newcommand\xrowht[2][0]{\addstackgap[.5\dimexpr#2\relax]{\vphantom{#1}}}

% theorems

\theoremstyle{plain}
\newtheorem{theorem}{Th\'eor\`eme}
\newtheorem*{sol}{Solution}
\theoremstyle{definition}
\newtheorem{ex}{Exercice}
\newtheorem*{rpl}{Rappel}
\newtheorem{enigme}{Énigme}

% corps
\usepackage{calrsfs}
\newcommand{\C}{\mathcal{C}}
\newcommand{\R}{\mathbb{R}}
\newcommand{\Rnn}{\mathbb{R}^{2n}}
\newcommand{\Z}{\mathbb{Z}}
\newcommand{\N}{\mathbb{N}}
\newcommand{\Q}{\mathbb{Q}}

% variance
\newcommand{\Var}[1]{\text{Var}(#1)}

% domain
\newcommand{\D}{\mathcal{D}}


% date
\usepackage{advdate}
\AdvanceDate[0]


% plots
\usepackage{pgfplots}

% table line break
\usepackage{makecell}
%tablestuff
\def\arraystretch{2}
\setlength\tabcolsep{15pt}

%subfigures
\usepackage{subcaption}

\definecolor{myg}{RGB}{56, 140, 70}
\definecolor{myb}{RGB}{45, 111, 177}
\definecolor{myr}{RGB}{199, 68, 64}

% fake sections with no title to move around the merged pdf
\newcommand{\fakesection}[1]{%
  \par\refstepcounter{section}% Increase section counter
  \sectionmark{#1}% Add section mark (header)
  \addcontentsline{toc}{section}{\protect\numberline{\thesection}#1}% Add section to ToC
  % Add more content here, if needed.
}


% SOLUTION SWITCH
\newif\ifsolutions
				\solutionstrue
				%\solutionsfalse

\ifsolutions
	\newcommand{\exe}[2]{
		\begin{ex} #1  \end{ex}
		\begin{sol} #2 \end{sol}
	}
\else
	\newcommand{\exe}[2]{
		\begin{ex} #1  \end{ex}
	}
	
\fi


% tableaux var, signe
\usepackage{tkz-tab}


%pinfty minfty
\newcommand{\pinfty}{{+}\infty}
\newcommand{\minfty}{{-}\infty}

\begin{document}

%!TEX encoding = UTF8
%!TEX root = 0-notes.tex

%%%%%%%%%%%%%%%%%%%%%%%%%%%%%%
% SELF MADE COMMANDS
%%%%%%%%%%%%%%%%%%%%%%%%%%%%%%


%%
% tcolor environments VS clean environments
%%

\ifclean

\newcommand{\thm}[3]{\begin{theorem}[#1]\label{#3}#2\end{theorem}}
\newcommand{\cor}[3]{\begin{corollaire}[#1]\label{#3}#2\end{corollaire}}
\newcommand{\lem}[3]{\begin{lemme}[#1]\label{#3}#2\end{lemme}}
\newcommand{\mprop}[3]{\begin{proposition}[#1]\label{#3}#2\end{proposition}}
\newcommand{\ex}[3]{\begin{exemple}[#1]\label{#3}#2\end{exemple}}
%\newcommand{\exe}[3]{\begin{exercice}[#1]\label{#3}#2\end{exercice}}
\newcommand{\dfn}[3]{\begin{definition}[#1]\label{#3}#2\end{definition}}
\newcommand{\qs}[2]{\begin{question}[#1]#2\end{question}}
\newcommand{\pf}[2]{\begin{preuve}[#1]#2\end{preuve}}
\newcommand{\nt}[1]{\begin{remarque}#1\end{remarque}}
\newcommand{\str}[1]{\begin{strategie}#1\end{strategie}}
\newcommand{\mth}[1]{\begin{methode}#1\end{methode}}
\newcommand{\ax}[3]{\begin{axiome}[#1]\label{#3}#2\end{axiome}}

\newcommand{\exe}[4]{
	\begin{Exercise}[title=#1, label=#3]
		\marginpar{\mbox{\scriptsize(solution p.\pageref{\ExerciseLabel-Answer})}}
		#2
	\end{Exercise}
	\begin{Answer}[ref=#3]
		#4
	\end{Answer}
}

\else

\newcommand{\thm}[3]{\begin{Theorem}[label=#3]{#1}{}#2\end{Theorem}}
\newcommand{\cor}[3]{\begin{Corollary}[label=#3]{#1}{}#2\end{Corollary}}
\newcommand{\lem}[3]{\begin{Lemma}[label=#3]{#1}{}#2\end{Lemma}}
\newcommand{\mprop}[3]{\begin{Prop}[label=#3]{#1}{}#2\end{Prop}}
\newcommand{\ex}[3]{\begin{Example}[label=#3]{#1}{}#2\end{Example}}
%\newcommand{\exe}[3]{\begin{Exe}[label=#3]{#1}{}#2\end{Exe}}
\newcommand{\dfn}[3]{\begin{Definition}[colbacktitle=red!75!black, label=#3]{#1}{}#2\end{Definition}}
\newcommand{\qs}[2]{\begin{MyQuestion}{#1}{}#2\end{MyQuestion}}
\newcommand{\pf}[2]{\begin{myproof}[#1]#2\end{myproof}}
\newcommand{\nt}[1]{\begin{Note}#1\end{Note}}
\newcommand{\str}[1]{\begin{Strategy}#1\end{Strategy}}
\newcommand{\mth}[1]{\begin{Methode}#1\end{Methode}}
\newcommand{\axiome}[3]{\begin{Axiome}[label=#3]{#1}{}#2\end{Axiome}}

\newcommand{\exe}[4]{
	\begin{Exe}[label=#3]{}{}#2\end{Exe}
	\begin{Answer}[ref=#3]
		#4
	\end{Answer}
}

\fi

\newcommand{\notations}[1]{\begin{notation}#1 \end{notation}}
\newcommand{\nomen}[1]{\begin{nomenclature}#1 \end{nomenclature}}

%%

\newcommand*\circled[1]{\tikz[baseline=(char.base)]{
		\node[shape=circle,draw,inner sep=1pt] (char) {#1};}}
\newcommand\getcurrentref[1]{%
	\ifnumequal{\value{#1}}{0}
	{??}
	{\the\value{#1}}%
}
\newcommand{\getCurrentSectionNumber}{\getcurrentref{section}}
\newenvironment{myproof}[1][\proofname]{%
	\proof[\bfseries #1: ]%
}{\endproof}

\newcommand{\mclm}[2]{\begin{myclaim}[#1]#2\end{myclaim}}
\newenvironment{myclaim}[1][\claimname]{\proof[\bfseries #1: ]}{}

\newcounter{mylabelcounter}

\makeatletter
\newcommand{\setword}[2]{%
	\phantomsection
	#1\def\@currentlabel{\unexpanded{#1}}\label{#2}%
}
\makeatother


\tikzset{
	symbol/.style={
			draw=none,
			every to/.append style={
					edge node={node [sloped, allow upside down, auto=false]{$#1$}}}
		}
}


% deliminators
\DeclarePairedDelimiter{\abs}{\lvert}{\rvert}
%\DeclarePairedDelimiter{\norm}{\lVert}{\rVert}

\DeclarePairedDelimiter{\ceil}{\lceil}{\rceil}
\DeclarePairedDelimiter{\floor}{\lfloor}{\rfloor}
\DeclarePairedDelimiter{\round}{\lfloor}{\rceil}

\newsavebox\diffdbox
\newcommand{\slantedromand}{{\mathpalette\makesl{d}}}
\newcommand{\makesl}[2]{%
\begingroup
\sbox{\diffdbox}{$\mathsurround=0pt#1\mathrm{#2}$}%
\pdfsave
\pdfsetmatrix{1 0 0.2 1}%
\rlap{\usebox{\diffdbox}}%
\pdfrestore
\hskip\wd\diffdbox
\endgroup
}
\newcommand{\dd}[1][]{\ensuremath{\mathop{}\!\ifstrempty{#1}{%
\slantedromand\@ifnextchar^{\hspace{0.2ex}}{\hspace{0.1ex}}}%
{\slantedromand\hspace{0.2ex}^{#1}}}}
\ProvideDocumentCommand\dv{o m g}{%
  \ensuremath{%
    \IfValueTF{#3}{%
      \IfNoValueTF{#1}{%
        \frac{\dd #2}{\dd #3}%
      }{%
        \frac{\dd^{#1} #2}{\dd #3^{#1}}%
      }%
    }{%
      \IfNoValueTF{#1}{%
        \frac{\dd}{\dd #2}%
      }{%
        \frac{\dd^{#1}}{\dd #2^{#1}}%
      }%
    }%
  }%
}
\providecommand*{\pdv}[3][]{\frac{\partial^{#1}#2}{\partial#3^{#1}}}
%  - others
\DeclareMathOperator{\Lap}{\mathcal{L}}
\DeclareMathOperator{\Var}{Var} % variance
\DeclareMathOperator{\Cov}{Cov} % covariance

% Since the amsthm package isn't loaded

% I prefer the slanted \leq
\let\oldleq\leq % save them in case they're every wanted
\let\oldgeq\geq
\renewcommand{\leq}{\leqslant}
\renewcommand{\geq}{\geqslant}

% tel que
\newcommand{\tqs}{\text{ tels que }}
\newcommand{\tq}{\text{ tq. }}
\newcommand{\et}{\text{ et }}
\newcommand{\ou}{\text{ ou }}
\newcommand{\pourtout}{\text{ pour tout }}

% Lois
\newcommand{\Bern}{\text{Bern}}
\newcommand{\Binom}{\text{Binom}}

% ensemble avec bigl et bigr
\newcommand{\bigset}[1]{\bigl\{ #1 \bigr\}}
\newcommand{\Bigset}[1]{\Bigl\{ #1 \Bigr\}}
\newcommand{\bigpar}[1]{\bigl( #1 \bigr)}
\newcommand{\Bigpar}[1]{\Bigl( #1 \Bigr)}

% PLUS INFTY AND MINUS INFTY WITH NO SPACE
\newcommand{\pinfty}{{+}\infty}
\newcommand{\minfty}{{-}\infty}

% vecteur flèche
\renewcommand{\vec}[1]{\overrightarrow{#1}}

% vecteur pmatrix
\newcommand{\pvec}[2]{\begin{pmatrix} #1 \\ #2 \end{pmatrix}}

% vecteur norme
\newcommand{\norm}[1]{\left\Vert #1 \right\Vert}

% point plan
\newcommand{\point}[3]{
	#1\left(#2 ; #3 \right)
}

% \smash avant \underline pour coller la ligne au mot
\let\oldunderline\underline
\renewcommand{\underline}[1]{\oldunderline{\smash{#1}}}

% emph + index
\newcommand{\emphindex}[1]{\emph{#1}\index{#1}}

% Schwartz
\renewcommand{\S}{\mathcal{S}} % \S est le signe paragraphe normalement

% corps
\newcommand{\C}{\mathcal{C}}
\newcommand{\R}{\mathbb{R}}
\newcommand{\Rnn}{\mathbb{R}^{2n}}
\newcommand{\Z}{\mathbb{Z}}
\newcommand{\N}{\mathbb{N}}
\newcommand{\Q}{\mathbb{Q}}

% domain
\newcommand{\D}{\mathcal{D}}

% order notations
\renewcommand{\O}{\mathcal{O}}

% japanese bracket
\newcommand{\japb}[1]{\langle #1 \rangle}

% arrows over partial derivatives
\newcommand{\lp}{\overleftarrow{\partial}}
\newcommand{\rp}{\overrightarrow{\partial}}

% quantization
\newcommand{\h}{\hbar}
\newcommand{\Opht}{\textrm{Op}_{\h}^{t}}
\newcommand{\Op}[2][\hbar]{\textrm{Op}_{#1}^{#2}}

% omega functions
\newcommand{\omegap}[2][\rho_0]{\omega(\partial_{#1},\partial_{#2})}
\newcommand{\omegar}[2][\rho_0]{\omega(#1,#2)}

\title{\Huge{Mathématiques en classe de Seconde}\\ Année 2024-2025}
\author{\huge{Matthieu Haeberle}}
\date{\today}

\begin{document}

\maketitle
\newpage% or \cleardoublepage
% \pdfbookmark[<level>]{<title>}{<dest>}
\pdfbookmark[section]{\contentsname}{toc}
\tableofcontents
\pagebreak

%!TEX encoding = UTF8
%!TEX root =notes.tex

\chapter{Ensembles dénombrables}

	\dfn{Ensemble de nombres}{
		On note un ensemble de nombres par des accolades \{  $\dots$ \}  entourant une liste de nombres.
	}{}
	\ex{}{
		Les ensembles suivants sont des ensembles de nombres.
		\begin{multicols}{3}
			$E_1 = \{ 1 ;  2 ; 3\}$, \\
			$E_2 = \left\{ -1,2 ; 4 ; \dfrac{7}{12} \right\}$, \\
			$E_3 = \{ 2\pi \}$.
		\end{multicols}
		Les nombres appartenant à un ensembles sont appelé les \emph{éléments} de l'ensemble.
		Tous les éléments sont distincts.
	}{}

	\dfn{Entiers et rationnels}{
		On définit les ensembles infinis suivants.
		\begin{multicols}{2}
		\end{multicols}
		Ils sont respectivement : les entiers \emph{naturels}, les entiers \emph{relatifs}, les nombres décimaux, et les nombres \emph{rationnels}.
	}{}
	
	\nt{
		Les décimaux $\D$ ont un nombre de décimales fini. En effet, la fraction $\dfrac{a}{10^n}$ pour $a \in \Z$ admet au plus $n$ décimales après la virgules.
	}
	
	\dfn{Appartenance, inclusion}{
		On pose les symboles suivants pour signifier l'appartenance et l'inclusion.
		\begin{itemize}
			\item $a \in E$ : l'élément $a$ appartient à l'ensemble $E$.
			\item $E \subset S$ : l'ensemble $E$ est inclus dans l'ensemble $S$.
		\end{itemize}	
	}{}
	\nt{
		Pour que $E \subset S$, tous les éléments de $E$ doivent aussi appartenir à $S$.
	}
	
	\ex{Appartenances et inclusions}{
		\begin{multicols}{3}
		\begin{itemize}
			\item  $2 \in \N$,
			\item $2/3 \in \Q$,
			\item $0{,}7 \in \D$,
			\item $ \left\{ 0 ; -3 ; \dfrac{32}{8} ; 1 \right\} \subset \Z$,
			\item $\N \subset \Z$,
			\item $\Z \subset \Z$.
		\end{itemize}
		\end{multicols}		
		On a la suite d'inclusions $\N \subset \Z \subset \D \subset \Q$.
	}{}
	
	\ex{Non appartenances et non inclusions}{
		\begin{multicols}{2}
		\begin{itemize}
			\item  $-2 \notin \N$,
			\item $\dfrac13 \notin \D$, 
			\item $2 \pi \notin \Q$,
			\item $ \{ 0 ; -1,2 ; 4 \}  \not\subset \Z$,
			\item $ \left\{ \dfrac{4}{13}; -1,5 ; 2,75 \right\} \not\subset \D$.
		\end{itemize}
		\end{multicols}	
	}{}
	
	
	
	\qs{}{
		Le rationnel $\dfrac13$ n'appartient pas à $\D$ car ses décimales se répètent infiniment. Qu'en est-il de $\dfrac17$ ou $\dfrac{5}{12}$ ?
		
		
		Le but du prochain chapitre est de démontrer rigoureusement que les fractions du type $\dfrac{5}{12}$ ne sont pas décimales : leur écriture est infinie.
	}
	\qs{}{
		Soit $\sqrt{2}$ le nombre positif qui vérifie $\left(\sqrt{2}\right)^2 = 2$. Est-ce que $\sqrt{2}$ est rationnel ?
		
		Nous démontrerons également que $\sqrt{2}$ n'est pas un nombre rationnel.
	}

%!TEX encoding = UTF8
%!TEX root =notes.tex

\chapter{Arithmétique sur $\N$}

	\section{Diviseurs et multiples}

	\dfn{Diviseur, multiple}{
		Pour $d , n\in \N$ deux entiers naturels, on dit que 
			\[ d \textbf{ divise } n \in \N \]
		dès que $n$ s'écrit de la forme
			\[ n = d \cdot k \]
		pour un entier naturel $k \in \N$.
		On écrit alors 
			\[ d \ | \ n, \]
		et on dit également que $n$ est un \textbf{multiple} de $d$.
	}
	
	\ex{Nombre pair, impair}{
		 Les nombres $n$ \textbf{pairs} sont les multiples de $2$. Ils s'écrivent donc 
		 	\[ n = 2 k \]
		 pour un entier naturel $k \in \N$.
		 
		 Les nombres $n$ \textbf{impairs} se situent juste après un nombre pair et s'écrivent alors
		 	\[ n = 2 k + 1 \]
		pour un entier naturel $k \in \N$.
	}{}
	
	\qs{}{
		Soit $a \in \N$ un entier naturel.
		L'entier $ (2a + 1)^2 - 1$ est-il toujours pair ?
		
		\emph{Rappel : on a l'identité remarquable $(a+b)(a-b) = a^2 - b^2$, pour $a, b$ des nombres quelconques.}
	}
	
	\ex{Ensembles de diviseurs}{
		L'ensemble des diviseurs de $24$ est donné par
			\[ \mathcal{D}_{24} = \left\{ 1 ; 2 ; 3 ; 4 ; 6 ; 8 ; 12 ; 24 \right\}. \]
		L'ensemble des multiples de $17$ inférieurs ou égaux à $100$ est donné par
			\[ \mathcal{M} = \left\{ 0 ; 17 ; 34 ; 51 ; 68 ; 85 \right\}. \]
	}{}
	
	\nt{
		Les diviseurs se regoupent par paires : en effet, si $d$ divise $n$ et que $n = d \cdot k$, alors $k$ divise aussi $n$.
	}
	
	\qs{}{
		Étant donné que $37$ divise $111$. Montrer que $37$ divise alors aussi $555$.
		
		Plus généralement, montrer que si $a | b$, alors $a$ divise aussi tous les multiples de $b$.
	}
	
\section{Nombres premiers}

	\dfn{Nombre premier}{
		Pour $p \in \N$, $p \geq 2$, un entier naturel. On dit que $p$ est \textbf{premier} si ses seuls diviseurs sont $1$ et lui-même.
	}
	
	\ex{}{
		Les premiers nombres premiers sont
			\[ \{ 2 ; 3 ; 5 ; 7 ; 11 ; 13 ; \dots \}. \]
		Il y en a une infinité.
	}{}
	
	
	\thm{Théorème fondamental de l'arithmétique}{
		Tout entier $n \in \N$, $n \geq 2$ peut s'écrire de façon unique comme produit de nombres premiers.
	}{}
	
	
	\ex{}{
		\begin{multicols}{2}
		\begin{itemize}[label=$\bullet$]
			\item $32 = 2^5$
			\item $9 = 3^2$
			\item $24 = 2^3 \cdot 3$
			\item $110 = 2 \cdot 5 \cdot 11$
			\item $10^n = 2^n \cdot 5^n$
			\item $10^n \cdot 30^m = 3^m \cdot 2^{m+n} \cdot 5^{m+n}$
		\end{itemize}
		\end{multicols}	
	}{}
	
	\exe{}{
		Écrire la décomposition en produit de facteurs premiers des entiers suivants.
		
		\begin{multicols}{4}
		\begin{itemize}[label=$\bullet$]
			\item $33$
			\item $48$
			\item $110 \times 55$
			\item $35 \times 90$
		\end{itemize}
		\end{multicols}
	}
	
	\mprop{ }{ 
		Le rationnel $\dfrac17 \in \Q$ n'est pas un nombre décimal : $\dfrac17 \notin \D$.
	}{}
	
	\pf{Démonstration par l'absurde }{
		La preuve se décline comme suit.
		
		\begin{enumerate}
			\item Supposons, par l'absurde, que $\dfrac17 \in \D$.
			\item Par définition, $\D  = \left\{ \dfrac{a}{10^n} \text{ tel que : } a \in \Z, n \in \N \right\}$.
			\item Donc $\dfrac17$ s'écrit $\dfrac{a}{10^n}$ pour certains $a \in Z$ et $n \in \N$.
			\item D'où $\dfrac17 =  \dfrac{a}{10^n}$, et par suite $10^n = 7 \cdot a$. C'est une égalité de deux entiers naturels.
			\item À droite : $a$ étant un entier et $7$ étant premier, la décomposition en produit de premiers de $7 \cdot a$ contient $7$.
			\item Cependant, à gauche, $7$ n'apparaît pas dans la décomposition en produit de premiers de $10^n = 2^5 \cdot 5^n$.
			\item L'égalité obtenue ne peut donc pas être vraie : ceci est une contradition, et $\dfrac17 \notin \D$.
		\end{enumerate}
	}

	\qs{}{
		Refaire la démonstration pour $\dfrac5{12} \notin \D$.
	}
	
	\nt{
		Si un entier naturel $d \in \N$ divise un autre entier naturel $n \in \N$, décomposition en facteurs premiers de la relation
			\[ n = d \cdot k \]
		où $k\in\N$ permet de dire la chose suivante.
		
		La puissance d'un premier $p$ dans la décompoistion de $d$ est inférieure ou égale à sa puissance dans la décomposition de $n$.
	}
	
	\mprop{}{
		Considérons un $n \in \N$.
		
		Si $n$ est pair, alors tous les multiples de$n$ sont pairs.
	}{prop:1}
	
	\exe{}{
		Cette proposition est le cas $a=2$ de la proposition suivante à démontrer.
		
		Si $a|b$, alors $a$ divise tous les multiples de $b$.
	}
	
\section{Coprimalité}
	
	\nt{
		Si le numérateur et le dénominateur partagent un diviseur commun, il peut s'annuler
			$\dfrac{a \cdot d}{b \cdot d} = \dfrac{a}b$.
		On dit alors qu'on réduit la fraction.
	}
	
	\dfn{Coprimalité et fractions irréductibles}{
		Deux entiers naturels $a$ et $b$ de $\N$ sont \textbf{premiers entre eux} si leur seul diviseur commun est $1$.
		
		La fraction $\dfrac{a}b$ est alors \textbf{irréductible}.
	}
	
	\exe{}{
		Donner l'ensemble des diviseurs communs à $100 \times 121$ et $44 \times 55$.
		Réduire la fraction $\dfrac{100 \times 121}{44 \times 55}$.
	}
	
	\thm{}{
		Soit $\sqrt{2}$ le nombre positif qui vérifie $\left(\sqrt{2}\right)^2 = 2$.
		Alors $\sqrt{2}$ est irrationnel : $\sqrt{2} \notin \Q$.
	}{thm:1}
	
	
	\lem{}{
		Considérons un entier naturel $n\in\N$.
		
		Si $n$ est impair, alors $n^2$ est impair.
	}{lem:1}
	
	\pf{Démonstration du lemme \ref{lem:1}}{
		Si $n$ est impair, et pour montrer que $n^2$ est impair, il suffit de montrer que $n^2 -1$ est pair.
		
		Or 
			\[ n^2 - 1 = (n+1)(n-1), \]
		Comme $n$ est impair, $n-1$ est pair, et $n^2 - 1$ est un multiple d'un nombre pair.
		
		D'après la proposition \ref{prop:1}, $n^2 -1 $ est pair, et donc $n^2$ est impair.
	}
	
	\lem{Constraposition du lemme \ref{lem:1}}{
		Considérons un entier naturel $n\in\N$.
		
		Si $n^2$ est pair, alors $n$ est pair.
	}{lem:1bis}
	
	\pf{Preuve du lemme \ref{lem:1bis}}{
		C'est la \emph{contraposition} du lemme \ref{lem:1}.
	}
	
	\pf{Preuve du théorème \ref{thm:1}  }{
		La démonstration est à nouveau par l'absurde.
		
		\begin{enumerate}
			\item Supposons, par l'absurde, que $\sqrt{2} \in \Q$ soit rationnel.
			
			\item Par définition $\Q = \left\{ \dfrac{a}b \text{ tq. } a \in \Z, b\in\Z, b \neq 0 \right\}.$
			
			\item Donc $\sqrt{2}$ s'écrit $\dfrac{a}{b}$ pour certains $a,b \in \Z$, $b \neq 0$.

			\item Comme $\sqrt{2}$ est positif, et en simplifiant la fraction, on peut écrire $\sqrt{2} = \dfrac{a}{b} = \dfrac{p}{q}$ fraction irréductible avec $p, q \in \N$, $q \neq 0$.
			
			\item Par définition de $\sqrt{2}$, 
				\[ \sqrt{2}^2 = 2 = \left( \dfrac{p}q \right)^2 = \dfrac{p^2}{q^2}. \]
				
			\item D'où l'égalité d'entiers naturels $p^2 = 2 q^2$. 
				L'entier $p^2$ est donc pair, et $p$ doit l'être aussi d'après le lemme \ref{lem:1bis}.
			
			\item $p$ est multiple de $2$ et s'écrit alors comme 
				\[ p = 2 k, \]
				pour un entier naturel $k \in \N$.
				
				En substituant dans l'équation $p^2 = 2 q^2$, on trouve
					\[ (2k)^2 = 2 q^2 \qquad \iff \qquad q^2 = 2 k^2.\]
			\item D'après le lemme \ref{lem:1bis}, $q$ est pair.
			\item Ceci est une contradiction, car la fraction $\dfrac{p}q$ a été choisie irréductible, alors que $p$ et $q$ sont tous les deux pairs. 
				Finalement, $\sqrt{2} \notin \Q$.
		\end{enumerate}
	}

%!TEX encoding = UTF8
%!TEX root =notes.tex

\chapter{Droite réelle et géométrie plane}\label{chap:3}

\section{Droite réelle}
	

On considère la droite graduée suivante.

\begin{center}
	
	\begin{tikzpicture}
		
		% real line
		\draw[<->, thick] (-7,0) node[below] {$-\infty$} -- (7,0) node[below] {$+\infty$};
		
		\foreach \x in {-2,...,2}
			\draw[-] (1.3*\x,-.1) node[below] {$\x$} -- (1.3*\x,.1) node{};
		
		\foreach \x in {-3, 3}
			\draw[-] (1.3*\x,-.1) node[below=3pt] {$\dots$} -- (1.3*\x,.1) node{};
		
	
	\end{tikzpicture}
\end{center}

On peut construire à la règle et au compas le nombre $\sqrt{2}$ sur la droite grâce au théorème de Pythagore.

\exe{}{
	Construire, en utilisant uniquement la règle et le compas, tous les nombres rationnels, c'est-à-dire toutes les fractions $\dfrac{p}q$ où $p, q \in \Z$ et $q \neq 0$.
	
	\emph{Indication : les couples $(1, p/q)$ et $(q, p)$ sont proportionnels.}
}{}

L'ensemble des points de la droite est donc strictement plus grand que celui des rationnels, car $\sqrt{2}$ n'est pas rationnel d'après le théorème \ref{thm:1}.

\begin{center}
	
	\begin{tikzpicture}[scale=1.3]
		
		% real line
		\draw[<->, thick] (-5,0) node[below] {$-\infty$} -- (5,0) node[below] {$+\infty$};
		\foreach \x in {-1, 0, 1}
			\draw[-] (1.3*\x,-.1) node[below] {$\x$} -- (1.3*\x,.1);
		
		% right angled triangle
		\draw[-, thick] (-1.3,0) -- (-1.3,1);
		\draw[-, thick] (0,0)-- (-1.3,1);
		
		\draw[-, thick] (-1.1,0)-- (-1.1,.2);
		\draw[-, thick] (-1.1,.2)-- (-1.3,.2);
		
  			
		% circle arc
		 \draw [red,thick,domain=-5:143] plot ({sqrt(2.69)*cos(\x)}, {sqrt(2.69)*sin(\x)});
		 
		 % sqrt{2} mark
		\draw[-, red] (1.640,-.1) node[below, red] {$\sqrt{2}$} -- (1.640,.1);
		
		% lengths
		\draw[decoration={brace,raise=5pt, mirror},decorate]
  			(0,0) -- node[above=12pt, right] {$\sqrt{2}$} (-1.3,1);
		\draw[decoration={brace,raise=5pt},decorate]
  			(-1.3,0) -- node[left=6pt] {$1$} (-1.3,1);
  			
				
	\end{tikzpicture}
\end{center}

Tous les points de la droite ne peuvent en fait pas être construit à la règle et au compas. Ceci est étroitement lié au théorème de Gauss-Wantzel sur les	polygônes réguliers constructibles.
En particulier, l'heptagone régulier n'est pas constructible à la règle et au compas.
	
Ce théorème est en dehors du champ d'application du cours.

\dfn{Nombres réels $\R$}{
	À chaque point de la droite on associe un nombre, appelé \emph{réel}.
	L'ensemble des nombres réels est noté $\R$.
}{}

\nt{
	On a la suite d'inclusions suivantes :
		\[ \N \subset \Z \subset \D \subset \Q \subset \Q. \]
}

\subsection{Intervalles}

\dfn{Intervalle borné}{
	Un intervalle borné est un segment de la droite réelle $\R$. C'est donc un ensemble de nombres.
	Il est donné par une \emph{borne inférieure} $a \in \R$ et une \emph{borne supérieure} $b\in\R$ et peut contenir ou non ses bornes.
	
	\begin{enumerate}
		\item Si $a$ et $b$ sont contenues dans l'intervalle, on le note $[a ; b]$.
		\item Si $a$  est contenue dans l'intervalle mais $b$ ne l'est pas, on le note $[a ; b [$.
		\item Si $a$ n'est pas contenue dans l'intervalle mais $b$ l'est, on le note $] a ; b]$.
		\item Si ni $a$ ni $b$ ne sont contenues dans l'intervalle, on le note $] a ; b [$.
	\end{enumerate}
}{}
\ex{}{
	Les intervalles suivants sont bornés.
	\begin{multicols}{2}
	\begin{enumerate}[label=$\bullet$]
		\item $[-1 ; 1]$
		\item $[-3 ; 1[$
		\item $\left]-10{,}341 ; \pi\right]$
		\item $\left]\sqrt{2} ; 130\right[$
	\end{enumerate}
	\end{multicols}

}{}

\dfn{Intervalle non borné}{
	Un intervalle n'est pas forcément borné : une ou les deux bornes peuvent être infinies ($\pinfty$ ou $\minfty$). 
	Dans ce cas, l'intervalle n'inclut jamais l'infini car ce n'est pas un nombre.
}{}

\ex{}{
	Les intervalles suivants ne sont pas bornés.
	\begin{multicols}{2}
	\begin{enumerate}[label=$\bullet$]
		\item $]\minfty; 2]$
		\item $]\minfty ; 3[$
		\item $]0; \pinfty[$
		\item $\R = ]\minfty; \pinfty[$
	\end{enumerate}
	\end{multicols}
}{ex:3.5}

\subsection{Intersection et union}

\dfn{Intersection, union}{	
	Pour deux ensembles $E, F$ on définit les ensembles suivants.
		\begin{enumerate}
			\item $E \cap F$ : l'intersection des deux ensembles.
			
			Un élément appartient à $E \cap F$ dès qu'il appartient à $E$ \textbf{et} à $F$.
			\item $E \cup F$ : l'union des deux ensembles.
			
			Un élément appartient à $E \cup F$ dès qu'il appartient à $E$ \textbf{ou} à $F$.
		\end{enumerate}
}{}


\exe{}{
	Exprimer les intersections et unions suivantes sous forme d'intervalle.
	\begin{multicols}{2}
	\begin{enumerate}
		\item $[-1 ; 1] \cap [-3 ; 1]$
		\item $] {-}\infty ; 2] \cap [-3 ; -2]$
		\item $]{-}\infty ; 4 [ \cup [2 ; +\infty[$
		\item $]-2 ; 4 [  \cup [4 ; 8[$
	\end{enumerate}
	\end{multicols}
}{ex:inter-union}

\nt{
	Lorsqu'on prend l'intersection de deux intervalles, seules les bornes les plus restrictives subsistent.
	En effet, un nombre appartient à un intervalle si et seulement s'il est à la fois supérieur à la borne inférieure, et inférieur à la borne supérieure (inégalités larges ou strictes à préciser).
	
	Dès lors, pour qu'un élément appartienne à l'intersection de deux intervalles, il doit
		\begin{enumerate}
			\item être supérieur aux deux bornes inférieures, et
			\item être inférieur aux deux bornes supérieures.
		\end{enumerate}
	La plus restrictive des bornes inférieures est donc la plus grande, et la plus restrictive borne supérieure est la plus petite.
	On en déduit une règle générale
		\[ [a ; b] \cap [c ; d] = [ \max\{a, c\} ; \min\{b, d\}]. \]
}

\nt{
	Pour l'union de deux intervalles, les bornes les moins restrictives restent, et on a
		\[ [a ; b] \cup [c ; d] = [ \min\{a, c\} ; \max\{b, d\}]. \]
}

\exe{}{
	Refaire l'exercice \ref{ex:inter-union} à l'aide des règles des remarques ci-dessus.
	Comparer les résultats.
}{}

\subsection{Inégalités}

Un intervalle n'a de sens que si la borne inférieure est plus petite que la borne supérieure.
De plus, un élément appartient à l'ensemble dès qu'il est plus petit que la borne supérieure, et plus grand que la borne inférieure.

Il faut donc pouvoir noter simplement les relations \og plus petit que \fg et \og plus grand que \fg.

\dfn{Inégalités}{
	On définit les signes suivants correspondant à des inégalités \emph{strictes} et \emph{larges}.
		\begin{enumerate}
			\item $<$ : strictement inférieur à
			\item $\leq$ : inférieur ou égal à
			\item $>$ : strictement supérieur à
			\item $\geq$ : supérieur ou égal à
		\end{enumerate}
}{}

%TEMPORARY
%\newpage

\ex{}{
	\begin{multicols}{3}
	\begin{enumerate}
		\item $1 \leq 2$
		\item $-3 \leq -2$
		\item $0 \leq 0$
		\item $1{,}02 < 1{,}1$
		\item $7{,}391 > 7{,}30001$
		\item $-4{,}001 > -4{,}0001$
	\end{enumerate}
	\end{multicols}
}{}

\nt{
	Appartenir à un intervalle est donc équivalent à respecter certaines inégalités.
	En particulier : être inférieur à la borne supérieure, et être supérieur à la borne inférieure.
	
	On a donc l'équivalence suivante.
		\[ x \in [a ; b] \qquad \iff \qquad x \in \R, x \geq a, \textbf{ et } x \leq b. \]
		
	On peut noter deux inégalités sur la même lignes dès qu'elles vont dans le même sens.
	Par exemple, les deux inégalités ci-dessus peuvent être condensées en une :
		\[ x \in \R, \text{ et } a \leq x \leq b, \]
	ou encore
		\[ x \in \R, \text{ et } b \geq x \geq a. \]
}

\ex{}{
	\begin{enumerate}
		\item Prendre $x \in [-3 ; 4]$ est équivalent à prendre $x \in \R$ vérifiant $-3 \leq x \leq 4$.
		\item Prendre $x \in [-4 ; 3[$ est équivalent à prendre $x \in \R$ vérifiant $-4 \leq x < 3$.
		\item Prendre $x \in ]\minfty ; 0[$ est équivalent à prendre $x \in \R$ vérifiant $x < 0$.
		
		On dit alors que $x$ est strictement négatif.
		\item Prendre $x \in [0; \pinfty [$ est équivalent à prendre $x \in \R$ vérifiant $x \geq 0$.
		
		On dit alors que $x$ est positif ou nul.
		\item Prendre $x \in ]\minfty; \pinfty[$ est équivalent à prendre $x \in \R$.	
		
		Ceci est en fait tautologique car $]\minfty; \pinfty[ = \R$, comme vu dans l'exemple \ref{ex:3.5}.
	\end{enumerate}
}{}

\thm{}{
	Soient $x, y, a \in \R$ vérifiant
		\[ x \leq y. \]
	Alors l'inégalité ci-dessus est équivalente aux inégalités suivantes.
		\begin{enumerate}
			\item $x + a \leq y + a$, sans condition sur $a$.
			\item $a x \leq a y$ si $a > 0$ est strictement positif.
			\item $ax \geq a y$ si $a < 0$ est strictement négatif.
		\end{enumerate}
}{thm:ineg}
\pf{Justifications du théorème \ref{thm:ineg}}{
	Le point $1.$ se justifie sans peine avec un dessin : ajouter $a$ revient à décaler les points $x$ et $y$ de la droite réelle vers la droite (si $a >0$) ou la gauche (si $a<0$) d'une même distance. 
	Leur relation d'ordre reste donc inchangée.
	
	Étant donné $1.$, on a que $x \leq y$ est équivalent à $y-x \geq 0$.
	Ainsi $y-x$ est un nombre réel positif.
	Or, multiplier deux positifs entre eux donne un nombre positif, et multiplier un nombre négatif avec un positif donne un nombre négatif.
	
	Par conséquent, si $a > 0$, on a bien
		\[ a \cdot (y-x) \geq 0 \iff ay \geq ax \iff ax \leq ay, \]
	et si $a < 0$, on a
		\[ a \cdot (y-x) \leq 0 \iff ay \leq ax \iff ax \geq ay, \]
	ce qui conclut.

}

\nt{
	Le théorème \ref{thm:ineg} reste vrai si on remplace l'inégalité large $\leq$ par une inégalité stricte $<$.
	
	En outre, échanger le rôle de $x$ et de $y$ revient à inverse le sens de l'inégalité.
}

\str{
	On décrit ci-après la stratégie de résolutions d'inéquations du type
		\[ a x + b \leq c x + d, \]
	où $a,b,c,d \in \R$ sont des réels quelconques et $x\in\R$ est l'inconnue à placer dans un intervalle.
	
	Comme pour les équations, on procède en deux étapes
		\begin{enumerate}
			\item Isoler les multiples de $x$ d'un côté et les constantes de l'autre.
			\item Diviser pour trouver l'intervalle de $x$.
		\end{enumerate}
	Pour cela, on peut par exemple soustraire $cx$ et $b$ des deux côtés à l'aide du Théorème \ref{thm:ineg}.
		\[ (a-c) x \leq d - b.\]
	Si $a-c \neq 0$, on divise par $a-c$ en faisant attention au signe de celui-ci :
		\[ \begin{cases*} 
						x \leq \dfrac{d-b}{a-c} & si $a-c > 0$, \\
					 	x \geq \dfrac{d-b}{a-c} & si $a-c < 0$.
			\end{cases*} \]
}

\ex{}{
	Essayons de déterminer quels réels $x\in\R$ vérifient l'inégalité suivante.
		\[ 3 x + 1 \geq  8 \]
	Cette inégalité est équivalente à
		\begin{align*}
			 3x \geq 7,  && \text{puis à} && x\geq \dfrac73,
		\end{align*}			
	car $\frac13 > 0$ est strictement positif et d'après le théorème \ref{thm:ineg}.
	
	On a donc trouvé que
		\[ \{ x \in \R \text{ tq. } 3x + 1 \geq 8 \} = \left[\dfrac73 ; \pinfty\right[. \]
}{}

\ex{}{
	On détermine les réels $x \in \R$ vérifiant la double inégalité suivante.
		\[ 14 > - 7x + 10 \geq -1.\]
	On vérifie d'abord qu'on ait $14 > -1$ pour s'assurer que de tels réels $x$ existent bien. 
	On peut par exemple résoudre l'équation $-7x+10 = 2$ pour avoir un exemple d'un tel $x$ (car $14 > 2 \geq -1$).
	
	L'inégalité est équivalente à
		\[ 4 > -7x \geq -11. \]
	On divise ensuite par $-7$, ce qui échange le sens des inégalités car $\dfrac{1}{-7}$ est strictement négatif.
	D'où
		\[ \dfrac{4}{-7} < x \leq  \dfrac{-11}{-7}. \]
	Finalement, on a obtenu que
		\[ \{ x \in \R \text{ tq. } 14 > - 7x + 10 \geq -1 \} = \left]{-}\dfrac47 ; \dfrac{11}7\right]. \]
}{}

\exe{}{
	Trouver quels réels $x\in\R$ vérifient les deux inégalités suivantes
		\begin{align*} -\dfrac32x - 2 \geq -4 && \textbf{et} && -\dfrac32x -2 \leq 13 & \end{align*}
	en prenant l'intersection des intervalles obtenus pour chaque inégalité.
	
	Faire de même en manipulant directement la double inégalité
		\[ -4 \leq -\dfrac32x -2 \leq 13, \]
	et vérifier qu'on trouve le même intervalle final.
}{}



\subsection{Valeur absolue}

La valeur absolue correspond à la notion de \emph{distance} parmis les réels.
Celle-ci intervient notamment lorsqu'on souhaite calculer la longueur d'un intervalle borné afin de mesure sa précision.
Par exemple, l'intervalle $[1 ; 3]$, dans lequel $2$ appartient, est moins précis que l'intervalle $[1{,}99 ; 2{,}01]$ car sa longueur est plus grande.

Les intervalles de confiance seront étudiés plus tard dans le chapitre dédié aux statistiques.

\exe{}{
	Quelle est la distance dans les réels 
		\begin{multicols}{2}
		\begin{enumerate}
			\item $3$ et $2$ ?
			\item $2$ et $4$ ?
			\item $-1$ et $1$ ?
			\item $-4{,}2$ et $-4{,}12$ ?
		\end{enumerate}
		\end{multicols}
}{}

\dfn{Valeur absolue}{
	La valeur absolue d'un $x \in \R$ général est égale à $x$ si celui-ci est positif, et à son opposé $-x$ sinon.
	La valeur absolue est donc toujours positive et peut être définie comme suit.
	
		\[ |x| = \begin{cases} x \text{ si $x \geq 0$}, \\ -x \text{ sinon.} \end{cases}. \]

	Soient $a, b \in \R$ deux réels quelconques.
	La distance de $a$ à $b$ est égale à
		\[ \text{Distance entre $a$ et $b$} = \begin{cases} a-b \text{ si $a \geq b$} , \\ b-a \text{ sinon.} \end{cases}. \]
	La condition $a \geq b$ peut être réécrite comme $a-b \geq 0$.
	
	En remarquant que $b-a = -(a-b)$, on voit que la distance entre $a$ et $b$ est exactement donnée par la valeur absolue de $a-b$.
		\[ \text{Distance entre $a$ et $b$} = | a- b|. \]
}{}

\nt{
	On retrouve d'ailleurs $|x| = |x - 0|$, la distance de $x$ à $0$.
}

\str{
	Pour résoudre une équation du type
		\[ |E| = a, \]
	où $E$ est une certaine expression, et $a\in\R$ est un nombre réel positif ou nul, on remarque que l'égalité est équivalente à
		\begin{align*}
			E = a, && \text{ ou } && E = -a.
		\end{align*}
}

\ex{}{
	On cherche un réel $x\in\R$ vérifiant l'égalité
		\[ |3x-8| = 5. \]
	En comprenant $3x-8$ comme l'expression $E$ et $5$ comme le réel $a$ de la stratégie ci-dessus, on se retrouve à résoudre les deux équations linéaires suivantes.
		\begin{align*}
			3x-8 = 5, && \text{ ou } && 3x-8 = -5.
		\end{align*}
	On les résoud calmement pour trouver
		\begin{align*}
			x = \dfrac{13}3, && \text{ ou } && x = 1.
		\end{align*}
	On pourra alors vérifier les valeurs obtenues en substituant dans l'équation d'origine.
	On a bien
	
		\begin{align*}
			|3 \times \dfrac{13}3 - 8| = |5| =5, && \text{ et } && |3 \times 1 - 8| = |-5| =5,
		\end{align*}
	ce qui nous rassure.
}{}

\str{
	Pour résoudre une inéquation du type
		\[ |E| \leq a, \]
	où $E$ est une certaine expression, et $a\in\R$ est un nombre réel positif ou nul, on remarque que l'inégalité est équivalente à
		\begin{align*}
			- a \leq E \leq a.
		\end{align*}
	Le résultat donnera donc un intervalle.
		
	On fera attention au caractère stricte ou non des inégalités en question.
}

\ex{}{
	On cherche quels réels $x\in\R$ vérifient l'inégalité
		\[ |2x-3| < 1. \]
	En comprenant $2x-3$ comme l'expression $E$ et $1$ comme le réel $a$ de la stratégie ci-dessus, on se retrouve à résoudre les inégalités strictes suivantes. 
		\[
			-1 < 2x-3 < 1.
		\]
	On les résoud pour trouver
		\[
			\{ x \in \R \text{ tq. } |2x-3| < 1 \} = \{ x \in \R \text{ tq. } 1 < x < 2 \} = ]1; 2[.
		\]
	On pourra alors s'assurer que n'importe quel $x\in\R$ réel de l'intervalle $]1;2[$ vérifie bien que $|2x-3|<1$, par exemple en testant quelques valeurs (dont les bornes).
}{}

\exe{}{
		Quels réels $x\in\R$ vérifient les conditions suivantes ? Donner des ensembles finis ou des intervalles.
		\begin{multicols}{2}
		\begin{enumerate}
			\item $|x| = 1$
			\item $|2x -4| = 5$
			\item $| x | \leq 3{,}14$
			\item $| 7 + 3x | \leq 10$
		\end{enumerate}
		\end{multicols}
}{}

\str{
	Pour résoudre une inéquation du type
		\[ |E| \geq a, \]
	où $E$ est une certaine expression, et $a\in\R$ est un nombre réel positif ou nul, on remarque que l'inégalité est équivalente à
		\begin{align*}
			E \leq -a, && \text{ ou } && E \geq a.
		\end{align*}
	Le résultat donnera donc une union de deux intervalles.
		
	On fera également attention au caractère stricte ou non des inégalités en question.
}

\ex{}{
	On cherche quels réels $x\in\R$ vérifient l'inégalité
		\[ \left|5 - \dfrac12x \right| \geq 3. \]
	En comprenant $5 - \dfrac12x$ comme l'expression $E$ et $3$ comme le réel $a$ de la stratégie ci-dessus, on se retrouve à résoudre les inégalités strictes suivantes. 
		\begin{align*}
			5 - \dfrac12x \leq -3, && \text{ ou } && 5 - \dfrac12x \geq 3.
		\end{align*}
	On les résoud pour trouver
		\[
			\{ x \in \R \text{ tq. }  \left|5 - \dfrac12x \right| \geq 3 \} = \{ x \in \R \text{ tq. } x \geq 16 \text{ ou } x \leq -4\} = ]\minfty ; -4] \cup [16; \pinfty[.
		\]
}{}

\nt{
	Il sera pratique à partir de maintenant de noter la multiplication de deux nombres par le point médian $\cdot$ à la place de la croix $\times$ pour la distinguer plus facilement du réel $x$.
	
	La notation $\times$ prendra d'autres sens en mathématiques plus tard : produit cartésien, produit vectoriel (notation anglosaxone), dimensions de matrices, etc…
}

\thm{Propriétés de la valeur absolue}{
	Soient $x,y\in\R$ deux nombres réels quelconques.
	Alors
		\begin{enumerate}
			\item $|x| = \left|-x\right|$
			\item $|x \cdot y| = |x| \cdot |y|$
			\item $\sqrt{x^2} = |x|$
		\end{enumerate}
}{thm:prop-vabs}
%!TEX encoding = UTF8
%!TEX root =notes.tex

\section{Géométrie plane}

Sur la droite réelle, on associe chaque point avec un nombre, appelé nombre réel : chaque point correspond à un nombre, et chaque nombre correspond à un point.

Dans cette section, on augmente la notion de droite réelle en ajoutant une dimension ; la droite devient le plan.
On associera alors à chaque point du plan \emph{deux}\footnote{Il s'avère qu'il est possible en théorie d'associer chaque point du plan à un unique nombre, mais la construction dépasse le champ d'application de ce cours.} nombres réel : le premier représentant sa position \og gauche/droite \fg et le deuxième sa position \og haut/bas \fg.

Un point sera alors un couple $(x,y)$ où $x,y \in \R$ sont deux nombres réels appelés respectivement l'abscisse et l'ordonnée du point.

\subsection{Repère orthonormé}

\dfn{Repère}{
	
	\begin{multicols}{2}

	Un repère est determiné par trois points $(O; I,J)$ comme ci-contre.
	\begin{itemize}
		\item Le point $O$ est appelé l'\textbf{origine}.
		\item La droite $(OI)$ est appelée l'axe des \textbf{abscisses}.
		\item La droite $(OJ)$ est appelée l'axe des \textbf{ordonées}.
		\item[]
		\item[]
	\end{itemize}
	
	
	\begin{tikzpicture}[>=stealth, scale=1]
		\begin{axis}[xmin = -2.9, xmax=4.9, xtick={ -3, ..., 5}, ymin=-2.9, ymax=4.9, ytick={-3, ..., 5}, axis x line=middle, axis y line=middle, axis line style=<->, xlabel={}, ylabel={}, grid=both]
		
			\addplot[violet, mark=*, mark size = 1] (0,0) node[below=8pt, left] {$O$};
			\addplot[violet, mark=*, mark size = 1] (1,0) node[above] {$I$};
			\addplot[violet, mark=*, mark size = 1] (0,1) node[right] {$J$};
		\end{axis}
	\end{tikzpicture}
	\end{multicols}

}{}

\dfn{Repère orthonormé}{
	Un repère $(O; I, J)$ est
	\begin{enumerate}
		\item \textbf{orthogonal} si ses axes $(OI)$ et $(OJ)$ sont perpendiculaires ;
		\item \textbf{normé} si les segments $[OI]$ et $[OJ]$ sont de même longueur (fixée à $1$) ;
		\item \textbf{orthonormé} s'il est orthogonal et normé.
	\end{enumerate}´
}{def:repere-orthonorme}

\nt{
	L'ordre des points $I$ et $J$ est important car il donne l'ordre de lecture des coordonnées.
	Traditionnellement, l'axe des abscisse est horizontale, et celle des ordonnée verticale.
	
	Ainsi, le point $I$ admet pour coordonnée $(1;0)$, et le point $J$ $(0;1)$.
	On notera alors :
		\begin{align*}
			I(1;0), && \text{ et } && J(0;1).
		\end{align*}
}

\ex{}{
	\begin{multicols}{2}
		\begin{tikzpicture}[>=stealth, scale=1]
		\begin{axis}[xmin = -2.9, xmax=4.9, xtick={ -3, ..., 5}, ymin=-2.9, ymax=4.9, ytick={-3, ..., 5}, axis x line=middle, axis y line=middle, axis line style=<->, xlabel={}, ylabel={}, grid=both]
		
			\addplot[darkgray, mark=*, mark size = 1] (0,0) node[below=8pt, left] {$O$};
			\addplot[darkgray, mark=*, mark size = 1] (1,0) node[above] {$I$};
			\addplot[darkgray, mark=*, mark size = 1] (0,1) node[right] {$J$};
			
			\addplot[black, mark=*, mark size = 1] (2,3) node[above, right] {$A$};
			\addplot[black, mark=*, mark size = 1] (-1,2) node[above, left] {$B$};
			\addplot[black, mark=*, mark size = 1] (3,-1) node[below, right] {$C$};
		\end{axis}
	\end{tikzpicture}
	
	Les points $A$, $B$, et $C$ admettent pour coordonnées, dans le repère $(O; I, J)$,
		\begin{align*}
			A(2;3), && B(-1;2), && C(3;-1).
		\end{align*} 
	\end{multicols}
}{}

\nt{
	On omettera parfois de préciser que le repère est orthonormé.
	
	On voit donc le plan comme l'ensemble des couples de réels :
		\[ \{ (x, y) \text{ tq. } x \in \R, \text{ et } y\in\R \}. \]
}


\subsection{Segments du plan : longueur et milieu}

\dfn{Segment}{
	Soient $E, F$ deux points du plan. Le segment $[EF]$ est l'ensemble des points de la droite $(AB)$ situés entre $A$ et $B$ (tous les deux inclus).
}{deg:segment}

\nt{
	Un segment du plan est une généralisation d'un intervalle sur la droite.
	Tous les segments du plans qui seront considérés contiennent leur bornes (contrairement aux intervalles !).
}

\mprop{}{
	Soient $a,b \in \R$ deux nombres réels tels que $a < b$.
	
	Alors le milieu de l'intervalle $[a,b]$ est égal à $\dfrac{a+b}2$.
}{prop:milieu-inter}

\pf{Démonstration de la proposition \ref{prop:milieu-inter}}{
	Notons $x\in\R$ le milieu de l'intervalle.
	Celui-ci est à équidistance de $a$ et de $b$ et vérifie donc
		\[ |x-a| = |x-b|. \]
	Comme $x$ vérifie $a \leq x \leq b$, et par définition de la valeur absolue, on a
		\begin{align*}
			|x-a| &= x-a, \text{ et} \\
			|x-b| &= -(x-b) = b-x.
		\end{align*}
	On peut alors résoudre pour $x\in\R$ :
		\begin{align*}
			x-a = b-x && \iff && 2x = a+b && \iff && x = \dfrac{a+b}2
		\end{align*}
}

\nt{
	En voyant $\dfrac{a+b}2$ comme $\dfrac12 a + \dfrac12 b$, on peut comprendre le milieu comme une moyenne de deux valeurs ($a$ et $b$).
	C'est d'ailleurs comme ça qu'on calcule une moyenne de notes ayant le même coefficient -- leur poids est alors $\sfrac12$.
	Remarquons que la somme des poids vaut toujours $1$ : ils correspondent aux coefficients divisés par la somme de tous les coefficients.
	
	En changeant les poids, on peut créer d'autres points de l'intervalle. 
	Par exemple, si $a$ compte deux fois plus que $b$, les coefficients seront (par exemple) $2$ et $1$, et les poids $\sfrac23$ et $\sfrac13$.
	 La formule de la moyenne pondérée sera alors
		\[ \dfrac23a + \dfrac13b, \]
	qui est plus proche de $a$ que de $b$.
	
}

\exe{}{
	Considérons l'intervalle $[15;20]$ qui correspond à deux notes : $15$ et $20$.
	\begin{enumerate}
		\item Si les deux notes ont le même coefficient fixé à $1$, quel est le poids de chaque note et quelle est la moyenne pondérée ? Comparer avec le milieu de l'intervalle.
		\item Posons $\lambda \in [0;1]$ le poids de la première note, et $1-\lambda$ celui de la deuxième.
			Calculer la moyenne pondérée $15\lambda + 20(1-\lambda)$ pour certaines valeurs de $\lambda$ dans son intervalle $[0;1]$. 
			Vérifier que toutes les valeurs obtenues appartiennent bien à l'intervalle $[15;20]$.
		\item Quel $\lambda \in [0;1]$ choisir pour avoir une moyenne de $18{,}5$ ?
	\end{enumerate}
}{exe:milieu-inter}

\mprop{}{
	Soient $A(x_A, y_B)$ et $B(x_A,y_B)$ deux points du plan.
	
	Alors le milieu du segment $[AB]$ est le point $M$ de coordonnées
		\begin{align}\label{expr:milieu}
			M\left(\dfrac{x_A+x_B}2, \dfrac{y_A + y_B}2\right).
		\end{align}
}{prop:milieu-segment}

\pf{Démonstration de la proposition \ref{prop:milieu-segment}}{

	Le point $M$ est au centre du rectangle dont les coins sont $A$ et $B$.
	Ainsi ses coordonnées sont les moyennes des coordonnées et $A$ et de $B$.
	
	\centering
	\resizebox{8cm}{!}{
	\begin{tikzpicture}[>=stealth, scale=1.2]
		\begin{axis}[xmin = -6.9, xmax=22.9, xticklabel=\empty, ymin=-2.9, ymax=4.9, yticklabel=\empty, axis x line=middle, axis y line=middle, axis line style=<->, xlabel={}, ylabel={}]
		
			% A and B
			\addplot[black, mark=*, mark size = 1] (1,3) node[above, right] {$A$};
			\addplot[black, mark=*, mark size = 1] (20,-2) node[below, right] {$B$};
			
			
			\addplot[black, mark=+, mark size = 1] (1,0) node[below] {$x_A$};
			\addplot[black, mark=+, mark size = 1] (20,0) node[above] {$x_B$};
			\addplot[black, mark=+, mark size = 1] (0,3) node[left] {$y_A$};
			\addplot[black, mark=+, mark size = 1] (0,-2) node[left] {$y_B$};
			
			% midpoint
			\addplot[red, mark=*, mark size = 1] (10.5,.5) node[above=10pt, right] {$M$};	
			\addplot[red, mark=+, mark size = 2] (10.5,0) node[below=3pt] {$\dfrac{x_A+x_B}2$};
			\addplot[red, mark=+, mark size = 2] (0,.5) node[above=3pt, left] {$\dfrac{y_A+y_B}2$};
			
			
			\draw[red, dotted, thick] (axis cs:10.5, 0) -- (axis cs:10.5,.5);
			\addplot[red, dotted, thick, domain = 0:10.5, samples=2] {.5};
			
			% dotted rectangle
			\draw[black, dotted, thick] (axis cs:1, 0) -- (axis cs:1,3);
			\draw[black, dotted, thick] (axis cs:20, 0) -- (axis cs:20,-2);
			\addplot[black, dotted, thick, domain = 0:1, samples=2] {3};
			\addplot[black, dotted, thick, domain = 0:20, samples=2] {-2};
		
			
		\end{axis}
	\end{tikzpicture}
	}
	
}

\nt{
	La formule \eqref{expr:milieu} du milieu du segment $[AB]$ ressemble de près à celle du milieu d'un intervalle.
	Pour simplifier les expressions, il sera alors pratique d'écrire
		\[ M = \dfrac{A+B}2 = \dfrac12 A + \dfrac12 B.\]
	Pour cela, il faut pouvoir additionner les points du plans, et les multiplier par des nombres réels.
}

\dfn{Manipulations des points}{
	Soient $A(x_A, y_A)$, $B(x_B, y_B)$ deux points du plan et $\kappa \in \R$ un nombre réel quelconque.
	
	Les opérations suivantes sont légales.
		\begin{enumerate}
			\item L'addition de deux points : $A+B$, de coordonées $(x_A+x_B, y_A+y_B)$.
			\item La multiplication d'un point par un réel : $\kappa A$, de coordonnées $(\kappa x_A; \kappa y_A)$.
		\end{enumerate}
	\textbf{Attention} : on ne multiplie jamais les points ensemble ! Le produit $A$ par $B$ n'a pas de sens.
}{}


\ex{}{
	Considérons $A(1;3)$ et $B(-3;2)$ deux points du plan.
	Alors 
		\begin{multicols}{3}
		\begin{enumerate}
			\item $A+B = (-2 ; 5)$
			\item $·2B = (-6 ; 4)$
			\item $-A = (-1 ; -3)$
			\item $B - A = (-4 ; -1)$
			\item $-2A - B = (1 ; -8)$
			\item $\dfrac{A-2B}{2} = \left(\dfrac72 ; \dfrac{-1}2\right)$
		\end{enumerate}
		\end{multicols}
}{}

\mprop{Reformulation de la proposition \ref{prop:milieu-segment}}{
	Le milieu $M$ du segment $[AB]$ est égal à
		\[ M = \dfrac{A+B}{2}. \]
}{}

\nt{
	On peut retrouver la formule du milieu d'un intervalle en plongeant celui-ci dans le plan.
	Par exemple, l'intervalle $[-3;1]$ correspond au segment $[AB]$, où $A(-3;0)$ et $B(1;0)$.
	
	Le milieu $M$ du segment est alors $\dfrac{A+B}2 = (-1;0)$, qui correspond bien au milieu $\dfrac{-3+1}2 = -1$ de l'intervalle.
}

\exe{}{
	Représenter les points $A(1;1)$ et $B(3;-1)$ dans un repère orthonormé.
	Par analogie à avec l'exercice \ref{exe:milieu-inter}, représenter le point
		\[ \lambda A + (1-\lambda)B, \]
	pour certaines valeurs de $\lambda \in [0;1]$.
	
	Quel $\lambda$ choisir pour obtenir le point $C\left(\dfrac32; \dfrac12\right)$ ?
}{exe:milieu-segment}

\thm{Longueur d'un segment}{
	Soient $A(x_A, y_A)$, $B(x_B, y_B)$ deux points du plan dans un repère orthonormé.
	La longueur $\ell$ du segment $[AB]$ est donnée par
		\[ \ell = \sqrt{|x_A - x_B|^2 + |y_A - y_B|^2}. \]
}{thm:long-segment}

\pf{Démonstration du théorème \ref{thm:long-segment}}{
	Comme le repère est normé, on peut lire les distances sur les coordonnées.
	En particulier, la distance en la première coordonnée entre $A$ et $B$ est $|x_A - x_B|$.
	La distance en la deuxième coordonnée est $|y_A - y_B|$.
	On a donc le dessin suivant.
	
	\begin{center}
	\begin{tikzpicture}[>=stealth, scale=1.2]
		\begin{axis}[xmin = -1.9, xmax=15.9, xticklabel=\empty, ymin=-1.9, ymax=10.9, yticklabel=\empty, axis x line=middle, axis y line=middle, axis line style=<->, xlabel={}, ylabel={}]
		
			% A and B
			\addplot[black, mark=*, mark size = 1] (1,3) node[below] {$A$};
			\addplot[black, mark=*, mark size = 1] (10,8) node[below, right] {$B$};
			
			
			\addplot[black, mark=+, mark size = 1] (1,0) node[below] {$x_A$};
			\addplot[black, mark=+, mark size = 1] (10,0) node[below] {$x_B$};
			\addplot[black, mark=+, mark size = 1] (0,3) node[left] {$y_A$};
			\addplot[black, mark=+, mark size = 1] (0,8) node[left] {$y_B$};
			
			% triangle
			\draw[black, thick, decorate, decoration = {brace, mirror}] (axis cs:10,3) -- (axis cs:10,8) node [midway, right]{$|y_A-y_B|$};
			\draw[black, thick, decorate, decoration = {brace}] (axis cs:1, 3) -- (axis cs:10,8) node [midway, above=10pt, left]{$\ell$};
			\draw[black, thick, decorate, decoration = {brace, mirror}] (axis cs:1, 3) -- (axis cs:10,3) node [midway, below]{$|x_A-x_B|$};
			
		\end{axis}
	\end{tikzpicture}
	\end{center}
	
	Comme le repère est orthogonal, les axes sont perpendiculaires et le triangle est donc rectangle.
	Le théorème de Pythagore s'applique donc, et implique que
		\[ \ell^2 = |x_A-x_B|^2 + |y_A-y_B|^2, \]
	ce qui conclut.

}
	

%\chapter{Calcul littéral et logique}
%
%\section{Calcul littéral}
%
%	\dfn{Distributivité des réels}{
%		Tout triplet $(a,b,c)$ de réels vérifie
%		
%		$a \cdot (b + c)  = ab + ac$.
%		
%		$(b+c) \cdot a = ba + ca$.
%	}
%	
%	\thm{Identités remarquables}{
%		On a les identités suivantes, pour $a,b,c$ réels.
%		
%		$(a+b)^2 = a^2 + b^2 + 2ab$
%		
%		$(a - b)^2 = a^2 + b^2 - 2ab$
%		
%		$(a+b)(a-b) = a^2 - b^2$.
%	}
%
%\section{Logique}
%
%	\dfn{Contraposition logique}{
%		$p \implies q$ est équivalent à $ \text{non } q \implies \text{non } p$.
%	}
%	
%	\ex{}{
%		Voir théorème \ref{thm:1}.
%	}
%	
%	\dfn{Preuve par l'absurde}{
%		Pour montrer $p \implies q$, on suppose $p$ vraie et $q$ fausse pour arriver à une contradiction logique.
%	}
%
%	\ex{}{
%		Voir théorème \ref{thm:2}.
%	}
%	
%
%
%\chapter{Algorithmique}
%
%\begin{mintedbox}{python}
%from math import *
%import numpy as np
%from matplotlib import pyplot as plt
%
%a = 3
%
%def fonction(x):
%	return x**2
%\end{mintedbox}
%	
%\chapter{Nombres réels}
%
%\section{Intervalles réels}
%
%
%\section{Manipulations d'inégalités}
%
%
%\section{Valeur absolue}
%
%
%\chapter{Fonctions de référence}
%
%
%\chapter{Vecteurs dans le plan}

\end{document}

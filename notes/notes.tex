\documentclass[a4paper, 12pt]{report}

% DYSLEXIA SWITCH
\newif\ifdys
		
				% ENABLE or DISABLE font change
				% use XeLaTeX if true
				\dystrue
				\dysfalse


\ifdys

\documentclass[a4paper, 14pt]{extarticle}
\usepackage{amsmath,amsfonts,amsthm,amssymb,mathtools}

\tracinglostchars=3 % Report an error if a font does not have a symbol.
\usepackage{fontspec}
\usepackage{unicode-math}
\defaultfontfeatures{ Ligatures=TeX,
                      Scale=MatchUppercase }

\setmainfont{OpenDyslexic}[Scale=1.0]
\setmathfont{Fira Math} % Or maybe try KPMath-Sans?
\setmathfont{OpenDyslexic Italic}[range=it/{Latin,latin}]
\setmathfont{OpenDyslexic}[range=up/{Latin,latin,num}]

\else

\documentclass[a4paper, 12pt]{extarticle}

\usepackage[utf8x]{inputenc}
%fonts
\usepackage{amsmath,amsfonts,amsthm,amssymb,mathtools}
% comment below to default to computer modern
\usepackage{libertinus,libertinust1math}

\fi


\usepackage[french]{babel}
\usepackage[
a4paper,
margin=2cm,
nomarginpar,% We don't want any margin paragraphs
]{geometry}
\usepackage{icomma}

\usepackage{fancyhdr}
\usepackage{array}
\usepackage{hyperref}

\usepackage{multicol, enumerate}
\newcolumntype{P}[1]{>{\centering\arraybackslash}p{#1}}


\usepackage{stackengine}
\newcommand\xrowht[2][0]{\addstackgap[.5\dimexpr#2\relax]{\vphantom{#1}}}

% theorems

\theoremstyle{plain}
\newtheorem{theorem}{Th\'eor\`eme}
\newtheorem*{sol}{Solution}
\theoremstyle{definition}
\newtheorem{ex}{Exercice}
\newtheorem*{rpl}{Rappel}
\newtheorem{enigme}{Énigme}

% corps
\usepackage{calrsfs}
\newcommand{\C}{\mathcal{C}}
\newcommand{\R}{\mathbb{R}}
\newcommand{\Rnn}{\mathbb{R}^{2n}}
\newcommand{\Z}{\mathbb{Z}}
\newcommand{\N}{\mathbb{N}}
\newcommand{\Q}{\mathbb{Q}}

% variance
\newcommand{\Var}[1]{\text{Var}(#1)}

% domain
\newcommand{\D}{\mathcal{D}}


% date
\usepackage{advdate}
\AdvanceDate[0]


% plots
\usepackage{pgfplots}

% table line break
\usepackage{makecell}
%tablestuff
\def\arraystretch{2}
\setlength\tabcolsep{15pt}

%subfigures
\usepackage{subcaption}

\definecolor{myg}{RGB}{56, 140, 70}
\definecolor{myb}{RGB}{45, 111, 177}
\definecolor{myr}{RGB}{199, 68, 64}

% fake sections with no title to move around the merged pdf
\newcommand{\fakesection}[1]{%
  \par\refstepcounter{section}% Increase section counter
  \sectionmark{#1}% Add section mark (header)
  \addcontentsline{toc}{section}{\protect\numberline{\thesection}#1}% Add section to ToC
  % Add more content here, if needed.
}


% SOLUTION SWITCH
\newif\ifsolutions
				\solutionstrue
				%\solutionsfalse

\ifsolutions
	\newcommand{\exe}[2]{
		\begin{ex} #1  \end{ex}
		\begin{sol} #2 \end{sol}
	}
\else
	\newcommand{\exe}[2]{
		\begin{ex} #1  \end{ex}
	}
	
\fi


% tableaux var, signe
\usepackage{tkz-tab}


%pinfty minfty
\newcommand{\pinfty}{{+}\infty}
\newcommand{\minfty}{{-}\infty}

\begin{document}

%!TEX encoding = UTF8
%!TEX root = 0-notes.tex

%%%%%%%%%%%%%%%%%%%%%%%%%%%%%%
% SELF MADE COMMANDS
%%%%%%%%%%%%%%%%%%%%%%%%%%%%%%


%%
% tcolor environments VS clean environments
%%

\ifclean

\newcommand{\thm}[3]{\begin{theorem}[#1]\label{#3}#2\end{theorem}}
\newcommand{\cor}[3]{\begin{corollaire}[#1]\label{#3}#2\end{corollaire}}
\newcommand{\lem}[3]{\begin{lemme}[#1]\label{#3}#2\end{lemme}}
\newcommand{\mprop}[3]{\begin{proposition}[#1]\label{#3}#2\end{proposition}}
\newcommand{\ex}[3]{\begin{exemple}[#1]\label{#3}#2\end{exemple}}
%\newcommand{\exe}[3]{\begin{exercice}[#1]\label{#3}#2\end{exercice}}
\newcommand{\dfn}[3]{\begin{definition}[#1]\label{#3}#2\end{definition}}
\newcommand{\qs}[2]{\begin{question}[#1]#2\end{question}}
\newcommand{\pf}[2]{\begin{preuve}[#1]#2\end{preuve}}
\newcommand{\nt}[1]{\begin{remarque}#1\end{remarque}}
\newcommand{\str}[1]{\begin{strategie}#1\end{strategie}}
\newcommand{\mth}[1]{\begin{methode}#1\end{methode}}
\newcommand{\ax}[3]{\begin{axiome}[#1]\label{#3}#2\end{axiome}}

\newcommand{\exe}[4]{
	\begin{Exercise}[title=#1, label=#3]
		\marginpar{\mbox{\scriptsize(solution p.\pageref{\ExerciseLabel-Answer})}}
		#2
	\end{Exercise}
	\begin{Answer}[ref=#3]
		#4
	\end{Answer}
}

\else

\newcommand{\thm}[3]{\begin{Theorem}[label=#3]{#1}{}#2\end{Theorem}}
\newcommand{\cor}[3]{\begin{Corollary}[label=#3]{#1}{}#2\end{Corollary}}
\newcommand{\lem}[3]{\begin{Lemma}[label=#3]{#1}{}#2\end{Lemma}}
\newcommand{\mprop}[3]{\begin{Prop}[label=#3]{#1}{}#2\end{Prop}}
\newcommand{\ex}[3]{\begin{Example}[label=#3]{#1}{}#2\end{Example}}
%\newcommand{\exe}[3]{\begin{Exe}[label=#3]{#1}{}#2\end{Exe}}
\newcommand{\dfn}[3]{\begin{Definition}[colbacktitle=red!75!black, label=#3]{#1}{}#2\end{Definition}}
\newcommand{\qs}[2]{\begin{MyQuestion}{#1}{}#2\end{MyQuestion}}
\newcommand{\pf}[2]{\begin{myproof}[#1]#2\end{myproof}}
\newcommand{\nt}[1]{\begin{Note}#1\end{Note}}
\newcommand{\str}[1]{\begin{Strategy}#1\end{Strategy}}
\newcommand{\mth}[1]{\begin{Methode}#1\end{Methode}}
\newcommand{\axiome}[3]{\begin{Axiome}[label=#3]{#1}{}#2\end{Axiome}}

\newcommand{\exe}[4]{
	\begin{Exe}[label=#3]{}{}#2\end{Exe}
	\begin{Answer}[ref=#3]
		#4
	\end{Answer}
}

\fi

\newcommand{\notations}[1]{\begin{notation}#1 \end{notation}}
\newcommand{\nomen}[1]{\begin{nomenclature}#1 \end{nomenclature}}

%%

\newcommand*\circled[1]{\tikz[baseline=(char.base)]{
		\node[shape=circle,draw,inner sep=1pt] (char) {#1};}}
\newcommand\getcurrentref[1]{%
	\ifnumequal{\value{#1}}{0}
	{??}
	{\the\value{#1}}%
}
\newcommand{\getCurrentSectionNumber}{\getcurrentref{section}}
\newenvironment{myproof}[1][\proofname]{%
	\proof[\bfseries #1: ]%
}{\endproof}

\newcommand{\mclm}[2]{\begin{myclaim}[#1]#2\end{myclaim}}
\newenvironment{myclaim}[1][\claimname]{\proof[\bfseries #1: ]}{}

\newcounter{mylabelcounter}

\makeatletter
\newcommand{\setword}[2]{%
	\phantomsection
	#1\def\@currentlabel{\unexpanded{#1}}\label{#2}%
}
\makeatother


\tikzset{
	symbol/.style={
			draw=none,
			every to/.append style={
					edge node={node [sloped, allow upside down, auto=false]{$#1$}}}
		}
}


% deliminators
\DeclarePairedDelimiter{\abs}{\lvert}{\rvert}
%\DeclarePairedDelimiter{\norm}{\lVert}{\rVert}

\DeclarePairedDelimiter{\ceil}{\lceil}{\rceil}
\DeclarePairedDelimiter{\floor}{\lfloor}{\rfloor}
\DeclarePairedDelimiter{\round}{\lfloor}{\rceil}

\newsavebox\diffdbox
\newcommand{\slantedromand}{{\mathpalette\makesl{d}}}
\newcommand{\makesl}[2]{%
\begingroup
\sbox{\diffdbox}{$\mathsurround=0pt#1\mathrm{#2}$}%
\pdfsave
\pdfsetmatrix{1 0 0.2 1}%
\rlap{\usebox{\diffdbox}}%
\pdfrestore
\hskip\wd\diffdbox
\endgroup
}
\newcommand{\dd}[1][]{\ensuremath{\mathop{}\!\ifstrempty{#1}{%
\slantedromand\@ifnextchar^{\hspace{0.2ex}}{\hspace{0.1ex}}}%
{\slantedromand\hspace{0.2ex}^{#1}}}}
\ProvideDocumentCommand\dv{o m g}{%
  \ensuremath{%
    \IfValueTF{#3}{%
      \IfNoValueTF{#1}{%
        \frac{\dd #2}{\dd #3}%
      }{%
        \frac{\dd^{#1} #2}{\dd #3^{#1}}%
      }%
    }{%
      \IfNoValueTF{#1}{%
        \frac{\dd}{\dd #2}%
      }{%
        \frac{\dd^{#1}}{\dd #2^{#1}}%
      }%
    }%
  }%
}
\providecommand*{\pdv}[3][]{\frac{\partial^{#1}#2}{\partial#3^{#1}}}
%  - others
\DeclareMathOperator{\Lap}{\mathcal{L}}
\DeclareMathOperator{\Var}{Var} % variance
\DeclareMathOperator{\Cov}{Cov} % covariance

% Since the amsthm package isn't loaded

% I prefer the slanted \leq
\let\oldleq\leq % save them in case they're every wanted
\let\oldgeq\geq
\renewcommand{\leq}{\leqslant}
\renewcommand{\geq}{\geqslant}

% tel que
\newcommand{\tqs}{\text{ tels que }}
\newcommand{\tq}{\text{ tq. }}
\newcommand{\et}{\text{ et }}
\newcommand{\ou}{\text{ ou }}
\newcommand{\pourtout}{\text{ pour tout }}

% Lois
\newcommand{\Bern}{\text{Bern}}
\newcommand{\Binom}{\text{Binom}}

% ensemble avec bigl et bigr
\newcommand{\bigset}[1]{\bigl\{ #1 \bigr\}}
\newcommand{\Bigset}[1]{\Bigl\{ #1 \Bigr\}}
\newcommand{\bigpar}[1]{\bigl( #1 \bigr)}
\newcommand{\Bigpar}[1]{\Bigl( #1 \Bigr)}

% PLUS INFTY AND MINUS INFTY WITH NO SPACE
\newcommand{\pinfty}{{+}\infty}
\newcommand{\minfty}{{-}\infty}

% vecteur flèche
\renewcommand{\vec}[1]{\overrightarrow{#1}}

% vecteur pmatrix
\newcommand{\pvec}[2]{\begin{pmatrix} #1 \\ #2 \end{pmatrix}}

% vecteur norme
\newcommand{\norm}[1]{\left\Vert #1 \right\Vert}

% point plan
\newcommand{\point}[3]{
	#1\left(#2 ; #3 \right)
}

% \smash avant \underline pour coller la ligne au mot
\let\oldunderline\underline
\renewcommand{\underline}[1]{\oldunderline{\smash{#1}}}

% emph + index
\newcommand{\emphindex}[1]{\emph{#1}\index{#1}}

% Schwartz
\renewcommand{\S}{\mathcal{S}} % \S est le signe paragraphe normalement

% corps
\newcommand{\C}{\mathcal{C}}
\newcommand{\R}{\mathbb{R}}
\newcommand{\Rnn}{\mathbb{R}^{2n}}
\newcommand{\Z}{\mathbb{Z}}
\newcommand{\N}{\mathbb{N}}
\newcommand{\Q}{\mathbb{Q}}

% domain
\newcommand{\D}{\mathcal{D}}

% order notations
\renewcommand{\O}{\mathcal{O}}

% japanese bracket
\newcommand{\japb}[1]{\langle #1 \rangle}

% arrows over partial derivatives
\newcommand{\lp}{\overleftarrow{\partial}}
\newcommand{\rp}{\overrightarrow{\partial}}

% quantization
\newcommand{\h}{\hbar}
\newcommand{\Opht}{\textrm{Op}_{\h}^{t}}
\newcommand{\Op}[2][\hbar]{\textrm{Op}_{#1}^{#2}}

% omega functions
\newcommand{\omegap}[2][\rho_0]{\omega(\partial_{#1},\partial_{#2})}
\newcommand{\omegar}[2][\rho_0]{\omega(#1,#2)}

\title{\Huge{Mathématiques en classe de Seconde}\\ Année 2024-2025}
\author{\huge{Matthieu Haeberle}}
\date{\today}

\begin{document}

\maketitle
\newpage% or \cleardoublepage
% \pdfbookmark[<level>]{<title>}{<dest>}
\pdfbookmark[section]{\contentsname}{toc}
\tableofcontents
\pagebreak

\chapter{Ensembles dénombrables}
\section{Introduction}

	\dfn{Ensemble de nombres}{
		On note un ensemble de nombres par des accolades \{  $\dots$ \}  entourant une liste de nombres.
	}
	\ex{}{
		Les ensembles suivants sont des ensembles de nombres.
		\begin{multicols}{3}
			$E_1 = \{ 1 ;  2 ; 3\}$, \\
			$E_2 = \left\{ -1,2 ; 4 ; \dfrac{7}{12} \right\}$, \\
			$E_3 = \{ 2\pi \}$.
		\end{multicols}
		Les nombres appartenant à un ensembles sont appelé les \emph{éléments} de l'ensemble.
		Tous les éléments sont distincts.
	}

	\dfn{Entiers et rationnels}{
		On définit les ensembles infinis suivants.
		\begin{multicols}{2}
		\begin{enumerate}
			\item $\N = \{ 0 ; 1 ; 2 ; 3 ; \dots \}$,
			\item $\Z = \{ 0 ; 1 ; -1 ; 2 ; -2 ; 3 ; -3 ; \dots \}$, 
			\item $\D = \left\{ \dfrac{a}{10^n} \text{ tel que : } a \in \Z, n \in \N \right\}$,
			\item $\Q = \left\{ \dfrac{a}b \text{ tel que : } a, b \in \Z \right\}$.
		\end{enumerate}
		\end{multicols}
		Ils sont respectivement : les entiers \emph{naturels}, les entiers \emph{relatifs}, les nombres décimaux, et les nombres \emph{rationnels}.
	}
	
	\nt{
		Les décimaux $\D$ ont un nombre de décimales fini. En effet, la fraction $\dfrac{a}{10^n}$ pour $a \in \Z$ admet au plus $n$ décimales après la virgules.
	}
	
	\dfn{Appartenance, inclusion}{
		On pose les symboles suivants pour signifier l'appartenance et l'inclusion.
		\begin{itemize}
			\item $a \in E$ : l'élément $a$ appartient à l'ensemble $E$.
			\item $E \subset S$ : l'ensemble $E$ est inclus dans l'ensemble $S$.
		\end{itemize}	
	}
	\nt{
		Pour que $E \subset S$, tous les éléments de $E$ doivent aussi appartenir à $S$.
	}
	
	\ex{Appartenances et inclusions}{
		\begin{multicols}{3}
		\begin{itemize}
			\item  $2 \in \N$,
			\item $2/3 \in \Q$,
			\item $0{,}7 \in \D$,
			\item $ \left\{ 0 ; -3 ; \dfrac{32}{8} ; 1 \right\} \subset \Z$,
			\item $\N \subset \Z$,
			\item $\Z \subset \Z$.
		\end{itemize}
		\end{multicols}		
		On a la suite d'inclusions $\N \subset \Z \subset \D \subset \Q$.
	}
	
	\ex{Non appartenances et non inclusions}{
		\begin{multicols}{3}
		\begin{itemize}
			\item  $-2 \notin \N$,
			\item $\dfrac13 \notin \D$, 
			\item $2 \pi \notin \Q$,
			\item $ \{ 0 ; -1,2 ; 4 \}  \not\subset \Z$,
			\item $ \left\{ \dfrac{4}{13}; -1,5 ; 2,75 \right\} \not\subset \D$.
		\end{itemize}
		\end{multicols}	
	}
	
	
	
	\qs{}{
		Le rationnel $\dfrac13$ n'appartient pas à $\D$ car ses décimales se répètent infiniment. Qu'en est-il de $\dfrac17$ ou $\dfrac{5}{12}$ ?
		
		
		Le but du prochain chapitre est de démontrer rigoureusement que les fractions du type $\dfrac{5}{12}$ ne sont pas décimales : leur écriture est infinie.
	}
	\qs{}{
		Soit $\sqrt{2}$ le nombre positif qui vérifie $\left(\sqrt{2}\right)^2 = 2$. Est-ce que $\sqrt{2}$ est rationnel ?
		
		Nous démontrerons également que $\sqrt{2}$ n'est pas un nombre rationnel.
	}


\chapter{Arithmétique sur $\Z$}

	\section{Diviseurs et multiples}

	\dfn{Diviseur, multiple}{
		Pour $d , n\in \N$ deux entiers naturels, on dit que 
			\[ d \textbf{ divise } n \in \N \]
		dès que $n$ s'écrit de la forme
			\[ n = d \cdot k \]
		pour un entier naturel $k \in \N$.
		On écrit alors 
			\[ d \ | \ n, \]
		et on dit également que $n$ est un \textbf{multiple} de $d$.
	}
	
	\ex{Nombre pair, impair}{
		 Les nombres $n$ \textbf{pairs} sont les multiples de $2$. Ils s'écrivent donc 
		 	\[ n = 2 k \]
		 pour un entier naturel $k \in \N$.
		 
		 Les nombres $n$ \textbf{impairs} se situent juste après un nombre pair et s'écrivent alors
		 	\[ n = 2 k + 1 \]
		pour un entier naturel $k \in \N$.
	}
	
	\qs{}{
		Soit $a \in \N$ un entier naturel.
		L'entier $ (2a + 1)^2 - 1$ est-il toujours pair ?
		
		\emph{Rappel : on a l'identité remarquable $(a+b)(a-b) = a^2 - b^2$, pour $a, b$ des nombres quelconques.}
	}
	
	\ex{Ensembles de diviseurs}{
		L'ensemble des diviseurs de $24$ est donné par
			\[ \mathcal{D}_{24} = \left\{ 1 ; 2 ; 3 ; 4 ; 6 ; 8 ; 12 ; 24 \right\}. \]
		L'ensemble des multiples de $17$ inférieurs ou égaux à $100$ est donné par
			\[ \mathcal{M} = \left\{ 0 ; 17 ; 34 ; 51 ; 68 ; 85 \right\}. \]
	}
	
	\nt{
		Les diviseurs se regoupent par paires : en effet, si $d$ divise $n$ et que $n = d \cdot k$, alors $k$ divise aussi $n$.
	}
	
	\qs{}{
		Étant donné que $37$ divise $111$. Montrer que $37$ divise alors aussi $555$.
		
		Plus généralement, montrer que si $a | b$, alors $a$ divise aussi tous les multiples de $b$.
	}
	
\section{Nombres premiers}

	\dfn{Nombre premier}{
		Pour $p \in \N$, $p \geq 2$, un entier naturel. On dit que $p$ est \textbf{premier} si ses seuls diviseurs sont $1$ et lui-même.
	}
	
	\ex{}{
		Les premiers nombres premiers sont
			\[ \{ 2 ; 3 ; 5 ; 7 ; 11 ; 13 ; \dots \}. \]
		Il y en a une infinité.
	}
	
	
	\thm{Théorème fondamental de l'arithmétique}{
		Tout entier $n \in \N$, $n \geq 2$ peut s'écrire de façon unique comme produit de nombres premiers.
	}{}
	
	
	\ex{}{
		\begin{multicols}{2}
		\begin{itemize}[label=$\bullet$]
			\item $32 = 2^5$
			\item $9 = 3^2$
			\item $24 = 2^3 \cdot 3$
			\item $110 = 2 \cdot 5 \cdot 11$
			\item $10^n = 2^n \cdot 5^n$
			\item $10^n \cdot 30^m = 3^m \cdot 2^{m+n} \cdot 5^{m+n}$
		\end{itemize}
		\end{multicols}	
	}
	
	\exe{}{
		Écrire la décomposition en produit de facteurs premiers des entiers suivants.
		
		\begin{multicols}{4}
		\begin{itemize}[label=$\bullet$]
			\item $33$
			\item $48$
			\item $110 \times 55$
			\item $35 \times 90$
		\end{itemize}
		\end{multicols}
	}
	
	\mprop{ }{ 
		Le rationnel $\dfrac17 \in \Q$ n'est pas un nombre décimal : $\dfrac17 \notin \D$.
	}{}
	
	\pf{Démonstration par l'absurde }{
		La preuve se décline comme suit.
		
		\begin{enumerate}
			\item Supposons, par l'absurde, que $\dfrac17 \in \D$.
			\item Par définition, $\D  = \left\{ \dfrac{a}{10^n} \text{ tel que : } a \in \Z, n \in \N \right\}$.
			\item Donc $\dfrac17$ s'écrit $\dfrac{a}{10^n}$ pour certains $a \in Z$ et $n \in \N$.
			\item D'où $\dfrac17 =  \dfrac{a}{10^n}$, et par suite $10^n = 7 \cdot a$. C'est une égalité de deux entiers naturels.
			\item À droite : $a$ étant un entier et $7$ étant premier, la décomposition en produit de premiers de $7 \cdot a$ contient $7$.
			\item Cependant, à gauche, $7$ n'apparaît pas dans la décomposition en produit de premiers de $10^n = 2^5 \cdot 5^n$.
			\item L'égalité obtenue ne peut donc pas être vraie : ceci est une contradition, et $\dfrac17 \notin \D$.
		\end{enumerate}
	}

	\qs{}{
		Refaire la démonstration pour $\dfrac5{12} \notin \D$.
	}
	
	\nt{
		Si un entier naturel $d \in \N$ divise un autre entier naturel $n \in \N$, décomposition en facteurs premiers de la relation
			\[ n = d \cdot k \]
		où $k\in\N$ permet de dire la chose suivante.
		
		La puissance d'un premier $p$ dans la décompoistion de $d$ est inférieure ou égale à sa puissance dans la décomposition de $n$.
	}
	
	\mprop{}{
		Considérons un $n \in \N$.
		
		Si $n$ est pair, alors tous les multiples de$n$ sont pairs.
	}{prop:1}
	
	\exe{}{
		Cette proposition est le cas $a=2$ de la proposition suivante à démontrer.
		
		Si $a|b$, alors $a$ divise tous les multiples de $b$.
	}
	
\section{Coprimalité}
	
	\nt{
		Si le numérateur et le dénominateur partagent un diviseur commun, il peut s'annuler
			$\dfrac{a \cdot d}{b \cdot d} = \dfrac{a}b$.
		On dit alors qu'on réduit la fraction.
	}
	
	\dfn{Coprimalité et fractions irréductibles}{
		Deux entiers naturels $a$ et $b$ de $\N$ sont \textbf{premiers entre eux} si leur seul diviseur commun est $1$.
		
		La fraction $\dfrac{a}b$ est alors \textbf{irréductible}.
	}
	
	\exe{}{
		Donner l'ensemble des diviseurs communs à $100 \times 121$ et $44 \times 55$.
		Réduire la fraction $\dfrac{100 \times 121}{44 \times 55}$.
	}
	
	\thm{}{
		Soit $\sqrt{2}$ le nombre positif qui vérifie $\left(\sqrt{2}\right)^2 = 2$.
		Alors $\sqrt{2}$ est irrationnel : $\sqrt{2} \notin \Q$.
	}{thm:1}
	
	
	\lem{}{
		Considérons un entier naturel $n\in\N$.
		
		Si $n$ est impair, alors $n^2$ est impair.
	}{lem:1}
	
	\pf{Démonstration du lemme \ref{lem:1}}{
		Si $n$ est impair, et pour montrer que $n^2$ est impair, il suffit de montrer que $n^2 -1$ est pair.
		
		Or 
			\[ n^2 - 1 = (n+1)(n-1), \]
		Comme $n$ est impair, $n-1$ est pair, et $n^2 - 1$ est un multiple d'un nombre pair.
		
		D'après la proposition \ref{prop:1}, $n^2 -1 $ est pair, et donc $n^2$ est impair.
	}
	
	\lem{Constraposition du lemme \ref{lem:1}}{
		Considérons un entier naturel $n\in\N$.
		
		Si $n^2$ est pair, alors $n$ est pair.
	}{lem:1bis}
	
	\pf{Preuve du lemme \ref{lem:1bis}}{
		C'est la \emph{contraposition} du lemme \ref{lem:1}.
	}
	
	\pf{Preuve du théorème \ref{thm:1}  }{
		La démonstration est à nouveau par l'absurde.
		
		\begin{enumerate}
			\item Supposons, par l'absurde, que $\sqrt{2} \in \Q$ soit rationnel.
			
			\item Par définition $\Q = \left\{ \dfrac{a}b \text{ tq. } a \in \Z, b\in\Z, b \neq 0 \right\}.$
			
			\item Donc $\sqrt{2}$ s'écrit $\dfrac{a}{b}$ pour certains $a,b \in \Z$, $b \neq 0$.

			\item Comme $\sqrt{2}$ est positif, et en simplifiant la fraction, on peut écrire $\sqrt{2} = \dfrac{a}{b} = \dfrac{p}{q}$ fraction irréductible avec $p, q \in \N$, $q \neq 0$.
			
			\item Par définition de $\sqrt{2}$, 
				\[ \sqrt{2}^2 = 2 = \left( \dfrac{p}q \right)^2 = \dfrac{p^2}{q^2}. \]
				
			\item D'où l'égalité d'entiers naturels $p^2 = 2 q^2$. 
				L'entier $p^2$ est donc pair, et $p$ doit l'être aussi d'après le lemme \ref{lem:1bis}.
			
			\item $p$ est multiple de $2$ et s'écrit alors comme 
				\[ p = 2 k, \]
				pour un entier naturel $k \in \N$.
				
				En substituant dans l'équation $p^2 = 2 q^2$, on trouve
					\[ (2k)^2 = 2 q^2 \qquad \iff \qquad q^2 = 2 k^2.\]
			\item D'après le lemme \ref{lem:1bis}, $q$ est pair.
			\item Ceci est une contradiction, car la fraction $\dfrac{p}q$ a été choisie irréductible, alors que $p$ et $q$ sont tous les deux pairs. 
				Finalement, $\sqrt{2} \notin \Q$.
		\end{enumerate}
	}
	
	
%\chapter{Calcul littéral et logique}
%
%\section{Calcul littéral}
%
%	\dfn{Distributivité des réels}{
%		Tout triplet $(a,b,c)$ de réels vérifie
%		
%		$a \cdot (b + c)  = ab + ac$.
%		
%		$(b+c) \cdot a = ba + ca$.
%	}
%	
%	\thm{Identités remarquables}{
%		On a les identités suivantes, pour $a,b,c$ réels.
%		
%		$(a+b)^2 = a^2 + b^2 + 2ab$
%		
%		$(a - b)^2 = a^2 + b^2 - 2ab$
%		
%		$(a+b)(a-b) = a^2 - b^2$.
%	}
%
%\section{Logique}
%
%	\dfn{Contraposition logique}{
%		$p \implies q$ est équivalent à $ \text{non } q \implies \text{non } p$.
%	}
%	
%	\ex{}{
%		Voir théorème \ref{thm:1}.
%	}
%	
%	\dfn{Preuve par l'absurde}{
%		Pour montrer $p \implies q$, on suppose $p$ vraie et $q$ fausse pour arriver à une contradiction logique.
%	}
%
%	\ex{}{
%		Voir théorème \ref{thm:2}.
%	}
%	
%
%
%\chapter{Algorithmique}
%
%\begin{mintedbox}{python}
%from math import *
%import numpy as np
%from matplotlib import pyplot as plt
%
%a = 3
%
%def fonction(x):
%	return x**2
%\end{mintedbox}
%	
%\chapter{Nombres réels}
%
%\section{Intervalles réels}
%
%
%\section{Manipulations d'inégalités}
%
%
%\section{Valeur absolue}
%
%
%\chapter{Fonctions de référence}
%
%
%\chapter{Vecteurs dans le plan}

\end{document}

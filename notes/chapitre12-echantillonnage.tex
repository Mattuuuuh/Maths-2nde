%!TEX encoding = UTF8
%!TEX root =notes.tex

\chapter{Échantillonnage}

Le but de ce chapitre est de traiter la partie \og Échantillonnage \fg~ du bulletin officiel.

Le contenu du chapitre est le suivant.
	\begin{itemize}
		\item Échantillon aléatoire de taille $n$ pour une expérience à deux issues.
		\item Version vulgarisée de la loi des grands nombres : \og Lorsque $n$ est grand, la fréquence observée est proche de la probabilité. \fg
		\item Principe de l'estimation d'une probabilité, ou d'une proportion dans une population par une fréquence observée sur un échantillon.
		\item Fonctions renvoyant un nombre aléatoire. Série statistique obtenue par la répétition de l'appel d'une telle fonction.
	\end{itemize}

Les capacités attendues sont les suivantes.
	\begin{itemize}
		\item Lire et comprnedre un fonction Python renvoyant le nombre ou la fréquence de succès dans un échantillon de taille $n$ pour une expérience aléatoire à deux issues.
		\item Observer la loi des grands nombres à l'aide d'une simulation Python ou tableau.
		\item Simuler $N$ échantillons de taille $n$ d'une expérience aléatoire à deux issues. Si $p$ est la probabilité d'une issue et $f$ sa fréquence observée dans un échantillon, calculer la proportion des vas où l'écart entre $p$ et $f$ est inférieur ou égal à $\frac1{\sqrt{n}}$.
		\item Lire et comprendre une fonction renvoyant une moyenne, un écart type.
		\item Écrire des fonctions renvoyant le résultat numérique d'une expérience aléatoire, d'une répétition d'expériences aléatoires indépendantes.
	\end{itemize}

\section{Introduction}

\section{Loi des grands nombres}

Estimation empirique de $P(|X - \mu| \geq k) \leq \dfrac1{k^2} \sigma$.
Avec $k = \frac1{\sqrt{n}}$ et $\sigma$ fixé, on peut tracer les différentes fonctions et voir laquelle s'approche le plus.
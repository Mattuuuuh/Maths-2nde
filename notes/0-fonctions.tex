%!TEX encoding = UTF8
%!TEX root = 0-notes.tex

\chapter{Fonctions}
\label{chap:fonctions}

%\section{Fonction comme boite noire}
%
%\section{Définition algébrique d'une fonction}
%
%\section{Courbe représentative d'une fonction}
%
%\section{Implémentation}


\section{Introduction}

\dfn{Fonction, variable, constante}{
	On dit qu'une quantité $y\in\R$ s'exprime en \emph{fonction} d'une quantité $x\in\R$ lorsque, à chaque nombre $x$ possible, on peut associer \underline{une seule} valeur $y$.
	On note alors $x \mapsto y$ et $x$ est appelé la \emph{variable}.
	Si une quantité ne dépend pas de $x$, on l'appelle \emph{constante}, ou \emph{indépendante} de $x$.
}{}

\nt{
	Pour faire signifier la dépendance de $y$ comme fonction de $x$, on utilise la notation fonctionnelle $y(x)$, lu \og $y$ de $x$ \fg.
	Ainsi, lorsque $x$ vaut $2$, la valeur associée sera $y(2)$, lu \og $y$ de $2$ \fg.
	Ceci permet de différencier les valeurs qui émanent de différents $x\in\R$ : $y(1), y(-3), y(\frac34), \dots$.
	La notation $y(x)$ est remplacée par $f(x)$ ($f$ pour fonction) pour se libérer la lettre $y$.
}

\ex{}{
	Le périmètre d'un cercle de rayon $R$ est donné par $2\pi \cdot R$.
	Un rayon détermine donc un unique périmètre, et la fonction périmètre, dépendant de la variable rayon, est donnée par
		\[ R\mapsto 2\pi\cdot R. \]
	En appelant $P$ cette fonction, on peut aussi noter 
		\[ P(R) = 2\pi \cdot R. \]	
}{ex:perimètre}

\exe{}{
	Un étudiant jette une balle dans les airs et mesure la hauteur de la balle tous les quarts de seconde.
	Il note ses résultats dans le tableau ci-dessous.
	\begin{center}
		\begin{tabular}{|c|c|c|c|c|c|c|c|}\hline
			Hauteur (cm) & 85 & 145 & 190 & 145 & 85 & 40 & 0 \\ \hline
			Temps (s) & 0 & 0,25 & 0,5 & 0,75 & 1 & 1,25 & 1,5 \\\hline
		\end{tabular}
	\end{center}
	
	\begin{enumerate}
		\item La hauteur est-elle une fonction du temps ? Justifier.
		\item Le temps est-il une fonction de la hauteur ? Justifier.
	\end{enumerate}
}{exe:table}{
	TODO
}

\exe{}{
	\begin{enumerate}
		\item
		Montrer que le rayon d'un cercle est fonction de son périmètre et écrire la fonction associée.
		\item
		Montrer que l'aire d'un cercle est fonction de son rayon et écrire la fonction associée.
		\item
		En déduire que l'aire d'un cercle est fonction de son périmètre et écrire la fonction associée.
	\end{enumerate}
}{exe:rayon-périmètre-aire}{
	TODO
}

\dfn{Fonction réelle, domaine}{
	Soit $\D \subset \R$ un ensemble de nombres qu'on appelle \emph{domaine}.
	
	Une fonction $f$ de $\D$ dans $\R$ associe à chaque élément $x\in\D$ du domaine un nombre $f(x) \in \R$.
	
	On note alors
		\begin{align*}
			f: \D & \longrightarrow \R \\
			x& \longmapsto f(x).
		\end{align*}
	
	\begin{center}
	\includegraphics[page=1]{figures/fig-fonctions.pdf}
	\end{center}
}{}

\ex{Fonctions vues en cours jusqu'ici}{

	\begin{enumerate}
		\item 
		La valeur absolue $|x|$ peut être formulée sous forme de fonction notée $|\cdot|$ : elle associe à chaque nombre réel sa valeur absolue.
		\begin{align*}
			|\cdot|: \R & \longrightarrow \R \\
			x& \longmapsto |x|.
		\end{align*}	
		
		\item
		Considérons la série statistiques $X$ suivante dépendant d'un entier naturel $k\in\N$.
			\begin{center}
			\begin{tabular}{|c|c|c|c|}\hline
				Valeur   & 0 & 20 & 10 \\ \hline
				Effectif & 17 & 17 & k \\ \hline
			\end{tabular}
			\end{center}
		La quantité $\Var(X)$ peut être vue comme une fonction qui à tout entier naturel $k\in\N$ associe la variance de la série $X$.
		Nommons $f$ cette fonction pour ne pas utiliser surcharger la notation $\Var$.
		\begin{align}
		\begin{split}
			f: \N & \longrightarrow \R \\
			k& \longmapsto \Var(X)
		\end{split}\label{func:vark}
		\end{align}
		Remarquons que le domaine de $f$ est l'ensemble des entiers naturels $\N$ car un effectif est forcément un nombre entier et positif.
		
		\item
		Pour définir l'écart type, il a fallu utiliser la fonction racine carrée.
		Comme la racine carrée d'un nombre négatif n'existe pas, le domaine de cette fonction est $[0;\pinfty[$, et non $\R$ tout entier.
			\begin{align*}
			\sqrt{\cdot}: [0;\pinfty[ & \longrightarrow \R \\
			x& \longmapsto \sqrt{x}
			\end{align*}

		\item 
		Une constante est en particulier une fonction d'une variable $x\in\R$ réelle.
		Ainsi la fonction $g$ donnée par
			\begin{align*}
			g: \R & \longrightarrow \R \\
			x& \longmapsto 42
			\end{align*}
		vaut constamment $42$, peut importe la valeur de $x$. 
		
		On voit dans l'expression $g(x) = 42$ que l'image de $x$ ne dépend aucunement de la valeur de $x$ :
		$g$ est une fonction constante.
					
	\end{enumerate}

}{ex:fonctions-deja-vues}

\dfn{Antécédent, image}{
	Considérons une fonction $f$ et deux nombres réels $x,y\in\R$ vérifiants
		\[ y = f(x). \]
	On lit \og $y$ égal $f$ de $x$ \fg, et on dit alors que
		\begin{enumerate}
			\item
			$y$ est l'\emph{image} de $x$ par $f$ ; et
			\item
			$x$ est \underline{un} \emph{antécédent} de $y$ par $f$.
		\end{enumerate}
}{}

\nt{
	\begin{enumerate}
		\item 
		Chaque $x\in\D$, élément du domaine, a \underline{exactement une} image par $f$ qui est $f(x)$.
		\item
		Un nombre $y\in\R$ peut avoir \underline{un, aucun, ou même plusieurs} antécédents. C'est le cas par exemple de $y=-1$ pour la fonction valeur absolue $f(x) = |x|$.
	\end{enumerate}
}

\ex{}{
	Considérons la fonction valeur absolue.
	On peut l'appeler $f$ en posant $f(x) = |x|$ pour tout $x\in\R$.
	
	Calculons les images de $1; 4; -1; -0,31; \frac23 ; -\frac23;$ et $0$.
		\begin{multicols}{2}
		\begin{enumerate}
			\item $f(1) = 1$
			\item $f(-1) = 1$
			\item $f( -0,31) = 0,31$
			\item $f\left(\dfrac23\right) = \dfrac23$
			\item $f\left( -\dfrac23\right) = \dfrac23$
			\item $f(0) = 0$
		\end{enumerate}
		\end{multicols}
		
	On en déduit les propositions suivantes.
		\begin{enumerate}[label=$\bullet$]
			\item
			L'image de $1$ par $f$ est $1$.
			\item
			L'image de $-1$ par $f$ est $1$.
			\item
			$1$ et $-1$ sont donc deux antécédents de $1$ par $f$.
			\item
			Un antécédent de $0,31$ par $f$ est $-0,31$.
			\item
			Un antécédent de $\dfrac23$ par $f$ est $\dfrac23$.
			\item
			Un antécédent de $\dfrac23$ par $f$ est $-\dfrac23$.
			\item
			$0$ est l'image et un antécédent de $0$ par $f$. C'est d'ailleurs le seul antécédent.
		\end{enumerate}
}{ex:v-abs}

\exe{}{
	Considérons la fonction $f$ donnée par
	\begin{align*}
		f: \R & \longrightarrow \R \\
		x& \longmapsto 3\cdot x+1.
	\end{align*}
	Autrement dit, $f(x) = 3\cdot x + 1$ pour tout $x\in\R$.
	
	\begin{enumerate}
		\item
		Calculer l'image par $f$ de $0$, de $3,1$, de $\frac13$, de $-1$, de $-\frac23$.
		\item
		Donner un antécédent de $1$ par $f$.
		\item
		Déterminer tous les antécédents de $6$ par $f$.	
	\end{enumerate}
}{exe:images-antécédents}{
	TODO
}

\dfn{Forme algébrique}{
	On appelle 
		\[ f(x) = 3\cdot x+1 \]
	la forme \emph{algébrique} de $f$.
	On lit dans ce cas \og $f$ de $x$ égal trois $x$ plus 1 \fg.
	
	C'est une forme entièrement générale qui permet de déduire l'image $f(x)$ à partir de n'importe quel réel $x\in\D$ du domaine de $f$.
}{}

%\newpage %temp


\exe{}{
	Lire les deux programmes Python de la figure \ref{python:1-2}.
	Quelles valeurs impriment-t-ils ?
	
	Écrire la fonction $f(x)$ de chaque programme sous forme algébrique.
}{exe:fonction-python}{
	TODO
}

\begin{figure}[!htb]
	\begin{subfigure}[b]{.45\textwidth}
\begin{mintedbox}{python}
def f(x):
y = 2*x-3
return y

f1 = f(1)
f2 = f(-1/2)


print(f1, f2)
\end{mintedbox}
	\caption{Programme 1.}
	\label{python:1}
	\end{subfigure}
	\begin{subfigure}[b]{.45\textwidth}
\begin{mintedbox}{python}
def f(x):
y = x*x + 1
z = -2*x
return y+z

f1 = f(4)
f2 = f(-2)

print(f1, f2)
\end{mintedbox}
	\caption{Programme 2.}
	\label{python:2}
	\end{subfigure}
	\caption{Deux fonctions implémentées en Python.}
	\label{python:1-2}
\end{figure}

% exe -> ex car suppression des exercices et celui-ci est référencé
\ex{}{
	On reconsidère la fonction $f$ de l'exemple \ref{ex:fonctions-deja-vues} donnée en $\eqref{func:vark}$.
	
	\begin{enumerate}
		\item
		Montrer que la moyenne $\overline{X}$ de la série est constante et ne dépend pas de $k$.
		\item
		Utiliser la formule \eqref{eq:var} de la variance pour donner une expression algébrique de $f(k)$ dépendant de $k\in\N$, entier naturel.
		\item
		Calculer $f(10), f(20), f(50), f(100)$ et comparer les variances obtenues.
	\end{enumerate}
}{exe:vark}

\nt{
	Suite à l'exemple \ref{exe:vark} ci-dessus, on dira alors qu'on a écrit la variance \emph{en fonction de}  l'entier naturel $k\in\N$.
	
	L'expression algébrique permet de calculer facilement plusieurs images sans refaire le raisonnement à chaque fois.
}

\exe{}{
	Un fonction $f$ admet le tableau de valeurs suivant.
		\begin{center}
		\begin{tabular}{|c|c|c|c|c|}\hline
			$x$ & 0 & -2 & 1 & -1 \\ \hline
			$f(x)$ & 1 & 0 & 0 & 1 \\ \hline
		\end{tabular}
		\end{center}
	Parmis les expressions algébriques suivantes, trouver celle qui correspond à $f(x)$.
		\begin{multicols}{2}
		\begin{enumerate}[label=\roman*)]
			\item $1-x$
			\item $1+\dfrac{x}2$
			\item $\dfrac{1-x}2$
			\item $\dfrac{-x^2 - x + 2}2$
		\end{enumerate}
		\end{multicols}
}{exe:fonctions-QCM}{
	TODO
}

\exe{}{
	Considérons la fonction $f$ donnée par
	\begin{align*}
		f: \R & \longrightarrow \R \\
		x& \longmapsto \dfrac15-x.
	\end{align*}
	Autrement dit, $f(x) = \frac15-x$ pour tout $x\in\R$.
	
	\begin{enumerate}
		\item
		Calculer l'image par $f$ de $0$, de $-3,1$, de $-\frac47$, de $0,2$, de $-\frac23$.
		\item
		Donner un antécédent de $0$ par $f$.
		\item
		Déterminer tous les antécédents de $5$ par $f$.	
	\end{enumerate}
}{exe:images-antécédents2}{
	TODO
}

\exe{}{
	Considérons les fonctions $f,g$ données par
	\begin{align*}
		f: \R & \longrightarrow \R &g: \R & \longrightarrow \R \\
		x& \longmapsto x^2 -6\cdot x + 9,  &x& \longmapsto (x-3)^2. 
	\end{align*}
	Autrement dit, $f(x) =x^2 -6\cdot x + 9$ et $g(x)=(x-3)^2$ pour tout $x\in\R$.
	
	\begin{enumerate}
		\item
		Calculer les images par $f$ et $g$ de $0$, de $-1$, de $3$, de $-\frac12$, de $-\frac23$.
		\item
		Donner un antécédent de $9$ par $f$.
		\item
		Déterminer tous les antécédents de $9$ par $g$.	
		\item
		Montrer que $g(x) - 3^2 = x^2 - 6\cdot x$ pour tout $x\in\R$ réel, et en déduire que $f(x) = g(x)$ pour tout $x\in\R$ réel.
		\item 
		Déterminer tous les antécédents de $16$ par $f$.
	\end{enumerate}
}{exe:forme-canonique}{
	TODO
}


\section{Représentation graphique}

On souhaite représenter graphiquement une fonction $f$ afin de pouvoir rapidement trouver plusieurs informations sur $f$.
Une bonne représentation graphique nous permet de lire
	%\begin{multicols}{2}
	\begin{enumerate}[label=$\bullet$]
		\item les images $f(x)$;
		\item les antécédents $x$ ;
		\item les variations de $f$ (croissante, décroissante, constante) ;
		\item le signe de $f(x)$ (positif, négatif, nul).
	\end{enumerate}
	%\end{multicols}

Pour chaque $x\in\D\subset\R$, élément du domaine de $f$, il faut donc pouvoir facilement lire son image $f(x)$ par $f$.
À cette fin et pour chaque $x\in\D$, on crée un point d'abscisse $x$ et d'ordonnée $f(x)$.
L'ensemble de ces points est la courbe représentative de $f$.
Pour lire l'image de $x$ par $f$, il suffit alors de trouver l'ordonnée de l'unique point de la courbe d'abscisse $x$.

\dfn{Courbe représentative}{
	Considérons $f : \D \rightarrow \R$ une fonction.
	
	La \emph{courbe représentative} de $f$, notée $\C_f$, est donnée par l'ensemble de points
		\[ \C_f = \left\{ \left(x ; f(x) \right) \text{ où } x\in\D \right\}. \]
	On lit \og la courbe représentative de $f$ est l'ensemble des points $(x,f(x))$ du plan où $x$ parcourt le domaine de $f$ \fg.
}{}	

\cor{Propriété fondamentale}{
	On a donc la propriété fondamentale suivante valable pour tout $x,y\in\R$.
		\begin{align*}
			(x;y) \in \C_f && \iff && y = f(x).
		\end{align*}
		
		
		\begin{center}
		 \includegraphics[page=2, scale=1.5]{figures/fig-fonctions.pdf}
		\end{center}
	Note : $f(x) = \dfrac1{200}(x+3,5)(x+2,5)(x-3,5)(x-4,5)(x-5,5) + 2$, polynôme de degré $5$ dans l'exemple ci-dessus.
}{cor:prop-fond}

\ex{}{
	Soit la fonction $f : [-1,5; 2,5] \rightarrow \R$ donnée algébriquement par
		\[ f(x) = 3\cdot x^3 -x - 3, \]
	et considérons les points $P(0;-3), Q(1; 1), R(-1; 1),$ et $S(2; 19)$.
	On se demande si les points appartiennent à la courbe représentative de $f$ ou non.
	
	D'après la propriété fondamentale \ref{cor:prop-fond}, un point $(x;y)$ appartient à la courbe de $f$ si et seulement si l'équation $y=f(x)$ est vérifiée.
	On a donc les (non) appartenances suivantes.
	\begin{multicols}{2}
		\begin{enumerate}
			\item $f(0) = -3$, et donc $P \in \C_f$.
			\item $f(1) = -1 \neq 1$, donc $Q \not\in \C_f$.
			\item $f(-1) = -5 \neq 1$, donc $R \not\in\C_f$.
			\item $f(2) = 19$, et donc $S\in\C_f$.
		\end{enumerate}
		
		\begin{center}
		 \includegraphics[page=3, scale=1.1]{figures/fig-fonctions.pdf}
		\end{center}
	\end{multicols}
}{}


\exe{}{
	Considérons la fonction $f: \left]{-}\dfrac72 ; \dfrac{11}2 \right[ \rightarrow\R$ donnée algébriquement par
		\[ f(x) = \dfrac17-x. \]
	Pour chaque point suivant, déterminer s'il appartient à $\C_f$ ou non.
	
	\begin{multicols}{2}
	\begin{enumerate}[label=\roman*)]
		\item $\left(0; \dfrac17\right)$
		\item $\left(\dfrac17 ; 0\right)$
		\item $\left(\dfrac27 ; \dfrac37\right)$
		\item $\left(-\dfrac{13}7 ; 2\right)$
		\item $\left(\dfrac67 ; 1\right)$
		\item $\left(\dfrac27 ; -\dfrac17\right)$
	\end{enumerate}
	\end{multicols}

}{exe:Cf}{
	TODO
}

\exe{}{
	Considérons deux fonctions $f, g: ]{-}3 ; 3[ \rightarrow\R$ données algébriquement par
		\begin{align*}
			f(x) = x^2 - 2\cdot x && g(x) = (x-1)^2
		\end{align*}
	
	\begin{enumerate}
		\item Esquisser les représentations graphiques de $f$ et de $g$ dans un même repère.
		\item Démontrer que $g(x) - 1 = f(x)$ pour tout $x$ du domaine.
		\item En déduire que $(x-1)^2 = x^2 - 2\cdot x + 1$ pour tout $x$ du domaine.
	\end{enumerate}
}{exe:id-rem-graph}{
	TODO
}

\exe{}{
	Esquisser la courbe de la fonction $f:[-2; 4]\rightarrow\R$ donnée algébriquement par
		\[ f(x) = 3. \]
	Que dire de $f$ et de $\C_f$ ?
}{exe:graph-const}{
	TODO
}

\exe{}{
	Esquisser la courbe de la fonction $f:[-5;3]\rightarrow\R$ donnée algébriquement par
		\[ f(x) = 1-x. \]
	Que dire $\C_f$ ?
}{exe:graph-droite}{
	TODO
}

\exe{}{
	Esquisser la courbe de la fonction $f:[3;10]\rightarrow\R$ donnée algébriquement par
		\[ f(x) = \dfrac3x + 1. \]
}{exe:graph-droite2}{
	TODO
}

% exe -> ex car suppression des exe et celui-ci est référencé	
\ex{}{
	Considérons la représentation graphique suivante d'une fonction $f$ définie sur $\D = ]{-}3,4 ; 2,3[$.
	
	\begin{center}
	\includegraphics[page=9]{figures/fig-fonctions.pdf}
	\end{center}
	\begin{enumerate}
		\item Donner approximativement les images de $-1,5$ et de $-\dfrac{20}7$ par $f$.
		\item Énumérer approximativement les antécédents de $-2$ et de $2$ par $f$.
		\item Donner approximativement un réel qui admet exactement deux antécédents par $f$.
		\item Si $f$ était définie sur $\R$ tout entier, serait-il toujours possible de connaître l'image de $-2$ ? Et tous les antécédents de $-2$ ?
	\end{enumerate}
	Supposons désormais que $f(x) = 3-2\cdot x +\dfrac13 \cdot x^3$ pour tout $x\in\D$ du domaine.
	\begin{enumerate}
		\item[5.] Vérifier à la calculatrice les réponses aux deux premières questions.
		\item[6.] Montrer sans calculatrice que l'image par $f$ de $-3$ est $0$ et que l'image par $f$ de $0$ est $3$.
	\end{enumerate}
}{exe:deg3}{
	TODO
}


\exe{}{
	Un fonction $f$ admet une représentation graphique suivante.
		\begin{center}
		 \includegraphics[page=4]{figures/fig-fonctions.pdf}
		\end{center}
	Parmis les expressions algébriques suivantes, trouver celle qui correspond à $f(x)$.
		\begin{multicols}{2}
		\begin{enumerate}[label=\roman*)]
			\item $1-x$
			\item $\dfrac{-1-x}3$
			\item $\left(x+\dfrac13\right)^2$
			\item $-2\cdot x - \dfrac23$
		\end{enumerate}
		\end{multicols}
}{exe:expr-from-graph}{
	TODO
}

\exe{}{
	Comparer les représentations graphiques des fonctions suivantes données algébriquement.
		\begin{align*}
			f(x) = x^2 && g(x) = x^2 - 3 && h(x) = (x+4)^2.
		\end{align*}
}{exe:y-shift}{
	TODO
}

\exe{, difficulty=2}{
	Donner une fonction réelle et un domaine telle qu'une de ses images admet
		\begin{enumerate}
			\item Exactement un antécédent
			\item Exactement deux antécédents
			\item Exactement trois antécédents
			\item Une infinité d'antécédents
		\end{enumerate}	
}{exe:nb-antécédents}{
	TODO
}

\exe{, difficulty=2}{
	Donner graphiquement une fonction sur $\R$ non constante telle que toutes les images de $f$ admettent un nombre infini d'antécédents.
}{exe:infinité-antécédents}{
	TODO
}

\section{Identités remarquables}

\ex{}{
	Considérons les fonctions $f,g$ données par
	\begin{align*}
		f: \R & \longrightarrow \R &g: \R & \longrightarrow \R \\
		x& \longmapsto x^2 -6\cdot x + 9,  &x& \longmapsto (x-3)^2. 
	\end{align*}
	Autrement dit, $f(x) =x^2 -8\cdot x + 16$ et $g(x)=(x-4)^2$ pour tout $x\in\R$.
	
	On se demande quels sont les antécédents d'une certaine valeur, mettons $64$ par exemple, par $f$.
	En nommant un tel antécédent $x$, on a l'équation
		\begin{align*}
			f(x) &= 64, \\
			x^2 - 8\cdot x + 16 &= 64,
		\end{align*}
	qui n'est pas simple du tout à résoudre car non linéaire.
	Cependant, si on arrive à démontrer que
		\begin{align*}
			(x-4)^2 = x^2 - 8\cdot x + 16 && \text{c'est-à-dire} && g(x) = f(x),
		\end{align*}
	et ceci pour tout $x\in\R$ nombre réel, on pourra utiliser la forme plus agréable de $g$ afin d'obtenir les antécédents recherchés.
	
	Pour développer $(x-4)^2 = (x-4)\cdot(x-4)$, on utilise la distributivité qui est rappelée ci-après par le théorème \ref{thm:distr}.
	On voit l'un des $(x-4)$ comme un tout, et l'autre comme un somme de deux quantités, $x$ et $-4$.
	En découle que, par distributivité utilisée trois fois,
		\begin{align*}
			(x-4)^2 &= (x-4) \cdot (x-4) \\
					&= (x-4) \cdot x + (x-4) \cdot (-4) \\
					&= x\cdot x + (-4) \cdot x + x \cdot(-4) + (-4) \cdot (-4) \\
					&= x^2 - 8 \cdot x + 16
		\end{align*}
	comme recherché.
	
	En revenant à notre situation, on se permet donc de résoudre
		\begin{align*}
			g(x) &= 64 \\
			(x-4)^2 &= 64 \\
			|x-4| &= 8,
		\end{align*}
	en utilisant la propriété de la valeur absolue vue au théorème \ref{thm:prop-vabs}, chapitre \ref{chap:3}.
	Le chapitre en question nous a enseigné à résoudre ce genre d'équations :
		\begin{align*}
			|x-4| &= 8 \\
			x-4=8 \qquad&\text{ou}\qquad x-4 = -8 \\
			x = 12  \qquad&\text{ou}\qquad  x=-4
		\end{align*}
	
	La fonction $f(x) = x^2 -8\cdot x + 16$ admet donc exactement deux antécédents de $64$ : $-4$ et $12$.
	Il conviendra de vérifier ceci en calculant $f(-4)$ et $f(12)$.
}{ex:id-rem1}{}

\thm{Distributivité}{
	Pour $a,b,c\in\R$ nombres réels, on a
		\[ c \cdot (a+b) = (a+b) \cdot c = a\cdot c + b \cdot c. \]
}{thm:distr}

Certaines expressions de fonctions sont donc plus agréables à étudier que d'autres.
Il sera donc utile de savoir quelques identités dites \emph{remarquables} qui permettront à la fois de développer une forme factorisée, et de factoriser une forme développée.

\thm{Identités remarquables}{
	Pour tous les $a,b\in\R$ réels, on a les identités
		\begin{align*}
			a^2 - b^2 &= (a+b)(a-b) \\
			(a+b)^2 &= a^2 + 2ab + b^2 \\
			(a-b)^2 &= a^2 - 2ab + b^2
		\end{align*}
	Remarquons que les points médians $\cdot$ de la multiplication disparaissent et sont sous-entendus.
	En lisant $2ab$, il faut comprendre \og 2 fois $a$ fois $b$ \fg, aussi notée $2\cdot a \cdot b$.
}{thm:id-rem}

\pf{Démonstration du théorème \ref{thm:id-rem}}{
	Toutes les identités remarquables découlent de la distributivité.
	La première se déduit en développant le produit $(a+b)\cdot (a-b)$.
		\begin{align*}
			(a+b)\cdot(a-b) &= a \cdot(a-b) + b \cdot (a-b) \\
							&= a \cdot a + a \cdot (-b) + b\cdot a + b \cdot(-b)\\
							&= a^2 - ab + ba - b^2 \\
							&= a^2 - b^2
		\end{align*}
	La deuxième s'obtient similairement.
		\begin{align*}
			(a+b)^2 &= (a+b)\cdot(a+b) \\
						&= a \cdot(a+b) + b \cdot (a+b) \\
						&= a \cdot a + a \cdot b + b\cdot a + b \cdot b \\
						&= a^2 + ab + ba + b^2 \\
						&= a^2 +2ab + b^2
		\end{align*}
	Pour obtenir la dernière, on peut remplacer $-b$ par $b$ de la façon suivante.
		\begin{align*}
			(a-b)^2 &= \left( a + (-b) \right)^2 \\
					&= a^2 + 2\cdot a\cdot (-b) + (-b)^2. 
		\end{align*}
	Or comme $-b = (-1)\cdot b$, on trouve que $2a(-b) = -2ab$, et que $(-b)^2 = b^2$, ce qui conclut.
}{}

\nt{
	Il sera utile de connaître quelques carrés parfaits pour développer tranquillement les expressions factorisées.
	On énumère alors
		\begin{align*}
			0^2 &= 0 & 7^2 &= 49 \\
			1^2 &= 1 & 8^2 &= 64 \\
			2^2 &= 4  & 9^2 &= 81  \\
			3^2 &= 9  & 10^2 &= 100 \\
			4^2 &= 16  &  11^2 &= 121 \\
			5^2 &= 25  & 12^2 &= 144 \\
			6^2 &= 36 & 13^2 &= 169 
		\end{align*}
	Remarquons qu'on a toujours $(-n)^2 = n^2$, donc qu'on peut en déduire aussi les carrés d'entiers négatifs. 
	Par exemple, $(-9)^2 = 9^2 = 81$.
}

\exe{}{
	Développer les expressions algébriques suivantes.
		\begin{multicols}{2}
		\begin{enumerate}[label=$\bullet$]
			\item $f(x) = (1+x)^2$
			\item $g(x) = (x-3)^2$
			\item $h(x) = (3-x)^2$
			\item $F(x) = (3 + 2x)^2$
			\item $G(x) = (3x - 7)^2$
			\item $H(x) = (-7x - 2)^2$
		\end{enumerate}
		\end{multicols}
}{exe:développement}{
	TODO
}

\exe{}{
	Factoriser les membres de gauche à la façon de l'exemple \ref{ex:id-rem1} et trouver toutes les solutions $x\in\R$ des équations suivantes.
		\begin{multicols}{2}
		\begin{enumerate}[label=\alph*)]
			\item $x^2 + 2x + 1 = 16$
			\item $4x^2 - 4x + 1 = 9$
			\item $9 - 18x + 9x^2 = 0$
			\item $3x^2 - 6x + 3 = 0$
			\item $x^2 +6x + 9 = -1$
			\item $-4x^2 + 8x - 1 = 3$
		\end{enumerate}
		\end{multicols}
}{exe:factorisation}{
	TODO
}

\exe{, difficulty=2}{
	On appelle \emph{triplet pythagoricien} un triplet $(a ; b ; c)$, liste de $3$ nombres entiers naturels $a, b, c\in\N$ vérifiants
		\[ a^2 + b^2 = c^2. \]
	Nommons deux paramètres entiers $k, \ell\in\N$ avec $k\geq \ell$.
	Vérifier que le triplet
		\[ (k^2 - \ell^2 ; 2k\ell ; k^2 + \ell^2) \]
	est pythagoricien.
	
	Créer quelques triplets en prenant des valeurs de $k \geq \ell$ et les vérifier à la calculatrice.
	Par exemple, pour $k=4$ et $\ell=2$, on trouve le triplet $(12 ; 16 ; 20)$, qui vérifie bien
		\[ 12^2 + 16^2 = 20^2. \] 
}{}{}

\section{Résolutions graphiques d'équations et d'inéquations}

Soient $f, g$ deux fonctions admettant un domaine $\D$ commun.
Le but de cette partie est d'étudier graphiquement les solutions $x\in\D$ des (in)équations du type
	\begin{multicols}{2}
	\begin{enumerate}[label=$\bullet$]
		\item $f(x) = k$
		\item $f(x) \leq k$
		\item $f(x) = g(x)$
		\item $f(x) \leq g(x)$
	\end{enumerate}
	\end{multicols}
\noindent où $k\in\R$ est un réel quelconque.
C'est-à-dire qu'on cherche l'ensemble des nombres réels $x$ tels que les (in)équations ci-dessus soient vraies.

\subsection{Résoudre graphiquement $f(x) = k$}


On cherche à résoudre graphiquement une équation du type $f(x)=k$ pour un $k\in\R$ donné, c'est-à-dire trouver l'ensemble des $x\in\R$ pour lesquels l'équation est vérifiée.
En reformulant, il s'agit de trouver l'ensemble des antécédents de $k$ par $f$. 

\str{
	En reprenant l'exemple \ref{ex:5.24}, on peut généraliser la stratégie à employer.
	Supposons qu'on souhaite résoudre $f(x) = 2$ où $x\in]{-}3,4 ; 2,3[$ et $\C_f$ est donnée graphiquement.
	Comme chaque point de $\C_f$ est de la forme $(x; f(x))$, on souhaite trouver les points de la forme $(x;2)$ pour ensuite lire leur abscisse et enfin obtenir les différents $x$ du domaine.
	Pour trouver tous les points d'ordonnée $2$, on peut s'aider en créant une droite horizontale d'ordonnée $2$, comme ci-dessous.
		\begin{center}
		\includegraphics[page=8, scale=2]{figures/fig-fonctions.pdf}
		\end{center}
	Les coordonnées des points $A,B,C$ trouvés en intersectant la droite horizontale avec $\C_f$ sont
		\begin{align*}
			A \approx (-2,67 ; 2) && B \approx (0,52 ; 2) && C \approx (2,14 ; 2)
		\end{align*}
	L'ensemble de solutions de l'équation $f(x)=2$ est donc approximativement donné par 
		\[ \{ -2,67 \ ; 0,52 \ ; 2,14 \}. \]
}{}

\thm{}{
	Soit $f:\D \rightarrow \R$ une fonction réelle définie sur son domaine $\D$.
	Pour tout réel $k$, les solutions dans $\D$ de l'équation $f(x)=k$ sont les abscisses des points d'intersection de $\C_f$ et de la droite d'équation $y=k$.
}{}


\subsection{Résoudre graphiquement $f(x) < k$}


On cherche à résoudre graphiquement une équation du type $f(x)\leq k$ pour un $k\in\R$ donné, c'est-à-dire trouver l'ensemble des $x\in\R$ pour lesquels l'inéquation est vérifiée.

\str{
	En reprenant l'exemple \ref{ex:5.24}, on peut généraliser la stratégie à employer.
	Supposons qu'on souhaite résoudre $f(x) \leq 2$ où $x\in]{-}3,4 ; 2,3[$ et $\C_f$ est donnée graphiquement.
	Comme chaque point de $\C_f$ est de la forme $(x; f(x))$, on souhaite trouver les points d'ordonnée inférieure à $2$ pour ensuite lire leur abscisse et enfin obtenir les différents $x$ du domaine.
	Pour trouver tous les points d'ordonnée inférieure à $2$, on peut s'aider en créant une droite horizontale d'ordonnée $2$, comme ci-dessous.
		\begin{center}
		 \includegraphics[page=5, scale=2]{figures/fig-fonctions.pdf}
		\end{center}
	Les points de $\C_f$ d'ordonnée inférieure à $2$ ont leur abscisse appartenant à l'union d'intervalles
		\[ ] {-}3,4 ; -2,67 ] \cup [0,52 ; 2,14]. \]
	C'est notre ensemble de solutions de l'équation $f(x) \leq 2$.
}{}

\thm{}{
	Soit $f:\D \rightarrow \R$ une fonction réelle définie sur son domaine $\D$.
	Pour tout réel $k$, les solutions dans $\D$ de l'équation $f(x) \leq k$ sont les abscisses des points de $\C_f$ situés en dessous de la droite d'équation $y=k$.
}{}


\subsection{Résoudre graphiquement $f(x) = g(x)$}

On cherche à résoudre graphiquement une équation du type $f(x)=g(x)$, c'est-à-dire trouver l'ensemble des $x\in\R$ pour lesquels l'équation est vérifiée.

\str{
	Soient $f,g : ]{-}3,4 ; 2,3[ \rightarrow \R$ deux fonctions réelles sur le domaine $]{-}3,4 ; 2,3[$ dont les courbes représentatives sont données ci-dessous.

		\begin{center}
		 \includegraphics[page=6, scale=2]{figures/fig-fonctions.pdf}
		\end{center}
		
	Imaginons un nombre réel $x\in\R$ parcourant $]{-}3,4 ; 2,3[$ en partant de la gauche et allant vers la droite.
	Pour chaque $x$, le point de $\C_f$ d'abscisse $x$ est d'ordonnée $f(x)$, par définition.
	Idem pour le point de $\C_g$ d'abscisse $x$ : il est d'ordonnée $g(x)$.
	
	Ainsi quand $x$ traverse le domaine, ces deux points sont confondus si et seulement si ces deux points sont de même ordonnée, c'est-à-dire si et seulement si $f(x) = g(x)$.		
		
	Les solutions $x\in]{-}3,4 ; 2,3[$ de l'équation $f(x) = g(x)$ sont donc données approximativement par l'ensemble
		\[ \{ -0,8 ; 1,2 \}. \]

}{}

\thm{}{
	Soient $f,g$ deux fonctions réelles définies sur un même domaine $\D$.
	Les solutions dans $\D$ de l'équation $f(x) = g(x)$ sont les abscisses des points d'intersection de $\C_f$ et $\C_g$.
}{}


\subsection{Résoudre graphiquement $f(x) \leq g(x)$}

On cherche à résoudre graphiquement une équation du type $f(x)=g(x)$, c'est-à-dire trouver l'ensemble des $x\in\R$ pour lesquels l'inéquation est vérifiée.

\ex{}{
	Soient $f,g : ]{-}3,4 ; 2,3[ \rightarrow \R$ deux fonctions réelles sur le domaine $]{-}3,4 ; 2,3[$ dont les courbes représentatives sont données ci-dessous.

	\begin{center}
	\includegraphics[page=7, scale=2]{figures/fig-fonctions.pdf}
	\end{center}
	
	Pour chaque $x$ parcourant $]{-}3,4 ; 2,3[$, la hauteur (l'ordonnée) du point de $\C_f$ d'abscisse $x$ est donnée par $f(x)$, et celle du point de $\C_g$ d'abscisse $x$ est donnée par $g(x)$.
	Ainsi, le premier point est en dessous du second dès que $f(x) \leq g(x)$.
	
	L'ensemble des solutions de l'inéquation $f(x) \leq g(x)$ est donc donné approximativement par
		\[ [{-}0,8 ; 1,2 ]. \]
}{}

\thm{}{
	Soient $f,g$ deux fonctions réelles définies sur un même domaine $\D$.
	Les solutions dans $\D$ de l'équation $f(x) \leq g(x)$ dont les abscisses des points de $\C_f$ situés en dessous de $\C_g$.
}{}
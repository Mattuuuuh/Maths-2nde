%!TEX encoding = UTF8
%!TEX root =notes.tex

\chapter{Fonctions de référence}

Le but de ce chapitre est de traiter la partie \og Se constituer un répertoire de fonctoins de référence \fg~ du bulletin officiel.

Le contenu du chapitre est le suivant.
	\begin{itemize}
		\item Fonctions carré, inverse, racine carrée, cube.
		\item Définitions, domaines, courbes représentatives, variations, extrema.
	\end{itemize}

Les capacités attendues sont les suivantes.
	\begin{itemize}
		\item Pour deux nombres $a$ et $b$ donnés et une fonction de référence $f$, comparer $f(a)$ et $f(b)$ numériquement ou graphiquement.
		\item Pour les fonctions affines, carré, inverse, racine carrée et cube, résoudre graphiquement ou algébriquement une équation ou une inéquation du type $f(x) = k$, $f(x) < k$.
	\end{itemize}

\section{Fonctions de référence}

\section{Position relative des courbes, parité}
%!TEX encoding = UTF8
%!TEX root =0-notes.tex

\chapter*{Préface}

Construction d'un cours de mathématiques : Exemple -- Remarque -- Définition -- Lemme -- Proposition -- Théorème -- Démonstration.

Les problèmes sont de la forme
\newcounter{preface}
\begin{Exercise}[counter=preface]
	Exemple de problème d'application directe.
\end{Exercise}
\begin{Exercise}[difficulty=1, counter=preface]
	Exemple de problème plus théorique.
\end{Exercise}
\begin{Exercise}[difficulty=2, counter=preface]
	Exemple de problème d'approfondissement.
\end{Exercise}


Wolframalpha permet de faire du calcul exact (voir chapitre nombres rationnels pour la nécessité).



\href{https://eduscol.education.fr/1723/programmes-et-ressources-en-mathematiques-voie-gt}{Eduscol}

\href{https://eduscol.education.fr/document/24553/download}{Programme PDF}
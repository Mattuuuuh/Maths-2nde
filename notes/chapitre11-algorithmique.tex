%!TEX encoding = UTF8
%!TEX root =notes.tex

\chapter{Algorithmique}

Le but de ce chapitre est de traiter la partie \og Algorithmique et programmation \fg~ du bulletin officiel.

Le contenu du chapitre est le suivant.
	\begin{itemize}
		\item Variables \texttt{int}, \texttt{bool}, \texttt{float}, \texttt{string} ; affectation ; instructions ; conditions.
		\item Boucles \texttt{for} et \texttt{while}.
		\item Fonctions à un ou plusieurs arguments.
	\end{itemize}

Les capacités attendues sont les suivantes.
	\begin{itemize}
		\item Choisir ou déterminer le type d'une variable ; concevoir et écrire une instruction d'affectation, une séquence d'instructions, une instruction conditionnelle ; écrire une formule permettant un calcul combinant des variables.
		\item Programmer, dans des cas simples, une boucle bornée, une boucle non bornée.
		\item Dans des cas plus complexes : lire, comprendre, modifier ou compléter un algorithme ou un programme.
		\item Écrire des fonctions simples ; lire, comprendre, modifier, compléter des fonctions plus complexes. Appeler une fonction.
	\end{itemize}

%\section{Variables, boucles, fonctions}
% est-ce bien nécessaire ? complexifie pour aucain gain je trouve.
%\subsection{Synthaxe en pseudo-code} 

\section{Variables, boucles, fonctions : synthaxe en Python}

%\subsection{Variables et boucles}
%
%Revoir les cours de NSI pour réviser les instructions de base de Python.

\subsection{Fonctions}

Une fonction est un objet qui prend un ou plusieurs arguments.
Une fonction peut être un outil qui ne renvoie rien (par exemple une fonction qui vérifie un résultat), ou elle peut renvoyer un ou plusieurs nombres, des listes, du texte, etc...

Dès que \texttt{return} est atteint, la valeur est retournée et rien d'autre n'est lu.
Les fonctions suivantes sont donc équivalentes.

\begin{mintedbox}{python}
def premiere(a, b):
	if a <= b:
		return b
	else:
		return a
		
def seconde(a,b):
	if a <= b:
		return b
	return a
\end{mintedbox}

\subsection{Algorithmes récursifs}

Un algorithme récursif est un algorithme qui fait appel à lui-même.
Implémenté, il prend la forme d'une fonction qui s'appelle elle-même.
Pour ne pas tomber dans une boucle infinie, il faut s'assurer de deux choses.
	\begin{enumerate}
		\item La fonction s'appelle elle-même en changeant au moins un argument.
		\item L'argument changé atteint une condition de sortie en temps fini.
	\end{enumerate}

\ex{Factorielle}{
	Supposons qu'on veuille calculer la \emph{factorielle} de $n$, notée
		\[ n! = 1 \times 2 \times 3 \times \cdots \times (n-1) \times n. \]
	 Nous souhaitons créer une fonction \texttt{factorielle(n)} qui retourne $n!$.
	 On reconnaît que 
		\[ n! = n \times (n-1)!, \]
	et donc qu'on peut calculer \texttt{factorielle(n)} à l'aide de \texttt{factorielle(n-1)}.
	Pour ne pas tomber dans une récursion infinie, on pose la condition de sortie qui est que $1! = 1$, et donc $\texttt{factorielle(1)}=1$.		
		

}{}
\begin{mintedbox}{python}
def factorielle(n):
	if n==1:
		return 1
	return n*factorielle(n-1)
\end{mintedbox}

\ex{Coefficient binomial}{
	On souhaite calculer le nombre de façon d'obtenir $4$ piles en $10$ lancers d'une pièce.
	
	On généralise ce problème à obtenir $k$ piles en $n$ lancers, où $n\geq k \geq 0$, et on appelle \texttt{binome(k,n)} cette valeur. \footnote{Voir l'article \emph{Coefficient binomial} sur Wikipédia.}
	
	On distingue deux cas après le premier lancer :
		\begin{itemize}
			\item soit on obtient pile, et il reste $k-1$ piles en $n-1$ lancers à obtenir ;
			\item soit on obtient face, et il reste $k$ piles en $n-1$ lancers à obtenir.
		\end{itemize}
	Par conséquent, $\texttt{binome(k,n)} = \texttt{binome(k-1,n-1)} + \texttt{binome(k,n-1)}$.
	
	Comme $k$ et $n$ diminuent en parallèle, on définit deux conditions de sortie : $\texttt{binome(n,n)} = \texttt{binome(0,n)} = 1$.
	
	Un appel \texttt{binome(4,10)} permet d'obtenir de répondre à la question de départ : il y a $210$ façons d'obtenir $4$ piles en $10$ lancers d'une pièce.
	Ce résultat est obtenu en $5,8 \times 10^{-5}$ secondes.
}{}
		
\begin{mintedbox}{python}
def binome(k, n):
	if k==n or k == 0:
		return 1
	return binome(k-1, n-1) + binome(k,n-1)
\end{mintedbox}



\section{Coût d'un algorithme : complexité en temps et en espace}

Pour mesurer le coût des algorithmes étudiés, nous nous mettons dans le modèle RAM (pour \emph{random-access machine}).
Dans ce modèle, un ordinateur dispose : d'une mémoire infinie non ordonnée ; des opérations arithmétiques de base (addition, multiplication) ; et de certaines instructions de base (désaffectation de variable, copie, saut au sein du code, conditions).

\dfn{Complexité en temps d'un algorithme (modèle RAM)}{
	Dans ce modèle, chaque opération arithmétiques \emph{élémentaire} sur des nombres réels est supposée prendre une même unité de temps.
	
	On compte alors les opérations arithmétiques élémentaires comme fonction des paramètres d'entrée.
	Celles-ci sont
	\begin{enumerate}
		\item l'addition de deux nombres réels ; et
		\item la multiplication de deux nombres réels.
	\end{enumerate}
}{}

\nt{
	Le modèle de Turing prend en compte la taille des nombres en question et est plus fidèle à la réalité. En effet, calculer $3\times7$ est plus rapide que $302194\times8491023$ !
}{}

\dfn{Complexité en espace d'un algorithme (modèle RAM)}{
	Pour mesurer l'espace que prend un algorithme dans le modèle RAM, on compte le plus grand nombre de réels qui doivent être stockés à la fois.
	Ce nombre dépend des paramètres d'entrée de l'algorithme en question.
}{}

\ex{Exponentiation rapide}{
	Pour calculer $5^{18}$, on a plusieurs façons de faire.
	\begin{enumerate}	
		\item Calculer toutes les puissances de $5$, de $1$ à $18$.
		On fait alors $17$ multiplications, et on garde $18$ nombres en mémoires.
		\item Calculer toutes les puissances de $5$, de $1$ à $18$, en ne gardant en mémoire que la dernière puissance.
		On fait alors $17$ multiplications, et on garde $2$ nombres en mémoire ($5$ et la dernière puissance de $5$).
		\item Calculer $5, 5^2, 5^4, 5^8, 5^{16}$ avec mises au carré successives, en ne gardant en mémoire que $5^2$ et $5^{16}$.
		Multiplier les deux résultats pour obtenir $5^{18}$.
		On fait alors $5$ multiplications, et on garde $2$ nombres en mémoire.
	\end{enumerate}
	
	La troisième façon de calculer une puissance est optimale car elle prend peu de temps et peu de mémoire.
	C'est l'\emph{exponentiation rapide}.
}{}

\section{Application : système de deux équations linéaires à deux inconnues}

\dfn{Système de deux équations}{
	Un \emph{système} de deux équations linéaires à deux inconnues réelles $x$ et $y$ s'écrit sous la forme
		\[ \begin{cases*} 2x - 3y &= 5, \\ x - y &= 12. \end{cases*} \]
	L'accolade dénote le système : les deux équations sont valables \emph{simultanément} et donnent donc deux informations qui peuvent nous permettre de résoudre le système.
	
	Les nombres $2, -3, 5, \dots$ du système sont appelés les \emph{coefficients} : ils multiplient $x$ et $y$ à gauche et sont des coefficients dits \emph{constants} à droite.
}{}

\nt{
	Une règle heuristique qui est valable généralement en mathématiques est la suivante :
		\begin{center}
			(nombre de variables) - (nombre d'équations) = degrés de liberté.
		\end{center}
	Par exemple, une équation linéaire $3x + 1 = 2$ n'offre aucun degré de liberté car elle force la valeur de $x$.
	
	Une équation à deux inconnues, par exemple $y= 2x - 5$ offre un degré de liberté : chaque valeur de $x$ donne une unique valeur de $y$, et on a une infinité de couples $(x ; y)$ qui vérifient cette équation.
	La théorie des fonctions affines nous donne en plus que l'ensemble de ces couples forment une droite.
	
	Une équation à trois inconnues, par exemples $2x + 3y - z = 10$ offre deux degrés de liberté : chaque choix de $x$ et $y$ force une unique valeur de $z$.
	Il s'avère que cette équation est celle d'un plan en trois dimensions.
	
	Inversement, s'il y a trop de contraintes et pas assez de variables, il n'y a généralement pas de solutions.
	Par exemple, $2x + 1 = 3$ et $2-x = 0$ ne peuvent pas être valables simultanéments : les équations sont contradictoires.
}{}

\dfn{Équations équivalentes, systèmes équivalents}{
	Deux équations sont \emph{équivalentes} si on peut obtenir l'une à partir de l'autre, \underline{et inversement}.
	On note l'équivalence par le signe $\iff$, amalgame des signes $\implies$ et $\impliedby$.
	
	Un système d'équations est \emph{équivalent} à un autre si on peut obtenir l'un à partir de l'autre, et inversement.
}{}

\nt{
	Si un nombre est solution d'une équation d'inconnue $x$, alors il est nécessairement solution d'une équation équivalente.
	C'est d'ailleurs comme ça qu'on résoud une équation linéaire simple, car $2x + 3 = 0$ est équivalente à $2x = -3$, et à $x = -\frac32$.
	La seule solution de la deuxième équation est bien sûr $-\frac32$, unique solution de la première.
	
	Les équations 
		\begin{align*}
			x = 1 && \text{ et } && x^2 = x
		\end{align*}
	ne sont pas équivalentes car l'ensemble des solutions de la première est $\{ 1 \}$, alors que l'ensemble des solutions de la seconde est $\{ 0 ; 1 \}$.
	En effet, la seconde est équivalente à $x(x-1) = 0$, et le théorème du produit nul conclut.

	Les équations semblent cependant équivalentes, à première vue : on a multiplé la première par un nombre, $x$, pour obtenir la deuxième.
	Il suit que multiplier par un nombre est parfois une opération irréversible : c'est le cas si ce nombre est nul !
	En effet, de $3x + 1 = 3$, on peut déduire que $0 = 0$ en multipliant par zéro.
	Mais l'implication inverse n'est bien sûr pas possible.
	
	En multipliant une équation par une variable $x$, on ajoute systématiquement la solution $x=0$.
}{}

\ex{Ajout d'un nombre}{
	Pour $x, y \in \R$ et de l'équation
		\[ 2x - y = 5, \]
	on peut obtenir
		\[ 2x - y + 3 = 8 \]
	en ajoutant $3$.
	En ajoutant $-3$ à la deuxième, on obtient la première, et les équations sont donc équivalentes.
		\begin{align*}
			2x - y = 5 && \iff && 2x - y + 3 = 8
		\end{align*}
	
	En ajoutant $x$ et $-x$, on obtient aussi
		\begin{align*}
			2x - y = 5 && \iff && 3x - y = 5 + x
		\end{align*}
}{}

\ex{Multiplication par un nombre non nul}{
	De l'équation 
		\[ x + 3y = -2, \]
	on peut obtenir
		\[ 4x + 12y = -8 \]
	en multipliant par $4$.
	En multipliant la deuxième équation par $\dfrac14$, on obtient la première, et les équations sont donc équivalentes.
		\begin{align*}
			x + 3y = -2 && \iff && 4x + 12y = -8
		\end{align*}
	
	Attention : multiplier par zéro est irréversible, on a donc une implication 
		\begin{align*}
			x + 3y = -2 && \implies && 0 = 0,
		\end{align*}
	mais pas l'autre :
		\begin{align*}
			0 = 0 && \centernot\implies && x + 3y = -2,
		\end{align*}
}{}

\thm{}{
	Un système de deux équations linéaires à deux inconnues prend l'une de ces trois formes.	
	\begin{enumerate}
		\item Les équations sont redondantes (l'une implique l'autre), et il existe une infinité de solutions.
		\item Les équations sont contradictoires, et il n'existe aucune solution.
		\item Il existe une unique solution.
	\end{enumerate}
}{}

Le but de la suite est de créer un algorithme qui, étant donné un système de deux équations linéaires à deux inconnues, renvoie la forme du système : y a-t-il aucune solution, une seule solution, ou une infinité ?
De plus, s'il n'existe qu'une seule solution, l'algorithme la donnera.

\mprop{Combinaisons d'équations}{
	Considérons le système suivant.
		\[ \begin{cases} A &= B, \\ C &= D. \end{cases}\]
	Alors ce système est équivalent à
		\[ \begin{cases} A &= B, \\ C+A &= D+B. \end{cases}\]
}{}

\str{
	Considérons, par exemple, le système d'inconnues $x, y \in \R$ suivant.
		\[ \begin{cases*} 2x - 3y &= 5, \\ x - y &= 12. \end{cases*} \]
	On souhaite, tout en gardant un système équivalent, simplifier le système pour le résoudre.
	Or, on ne sait que résoudre les équations à une seule inconnue : la stratégie est donc de modifier les équations puis de les combiner afin d'obtenir un système dont une équation n'a qu'une seule inconnue.
	
	Supposons qu'on veuille une équation en $y$ uniquement : on souhaite donc annuler le coefficient devant $x$.
	On a l'équivalence des systèmes suivante.
		\[ \begin{cases*} 2x - 3y &= 5, \\ x - y &= 12. \end{cases*} \iff \begin{cases*} 2x - 3y &= 5, \\ -2x +  2y &= -24. \end{cases*} \]
	Puis, en combinant, on a
		\[ \begin{cases*} 2x - 3y &= 5, \\ -2x +  2y &= -24. \end{cases*} \iff \begin{cases*} 2x - 3y &= 5, \\ -2x +  2y + 2x - 3y &= -24 + 5. \end{cases*} \iff \begin{cases*} 2x - 3y &= 5, \\ -y &= -19. \end{cases*} \]
	Il suit immédiatement que $y = 19$, et que, comme $2x - 3y = 5$, on a $2x - 3\times19 = 5$, et $x = 31$.
	
	On vérifie finalement que $(x ; y) = (31 ; 19)$ résoud bien le système initial :
		\[ \begin{cases} 2x - 3y = 2\times31 - 3\times19 = 62 - 57 = 5, \\ x -y = 31 - 19 = 12. \end{cases} \] 
}{}

\section{Application : dichotomie pour l'approximation de racines}

On dispose d'une fonction $f$ dont on souhaite trouver une racine, c'est-à-dire trouver un antécédent $x_0\in\D_f$ tel que $f(x_0)=0$.
Lorsque la factorisation de $f$ en produit de facteurs linéaires n'est pas connue (et d'ailleurs, elle n'existe pas en général, voir le théorème d'Abel-Ruffini), il faut se contenter d'une approximation de $x_0$.

\ex{Racine carrée de $7$}{
	Par exemple, pour calculer approximativement $\sqrt{7}$, on pourra considérer la fonction 
		\[ f(x) = x^2 - 7, \]
	et lui chercher une racine $x_0$.
	Il existe deux racines : une positive et une négative. 
	Pour approximer la racine positive, on cherchera $x_0$ entre $2$ et $3$, car 
		\[ 2 = \sqrt{4} < \sqrt{7} < \sqrt{9} = 3 \]
	puisque la fonction racine carrée est croissante.
	
	Le but du jeu est d'entretenir un intervalle $[a ; b]$ dans lequel $x_0 = \sqrt{7}$ appartient.
	Pour réduire la taille de l'intervalle, on évalue calcule l'image du milieu $m = \dfrac{a+b}2$ de l'intervalle par $f$.
	\begin{itemize}
		\item Si cette image $f(m)$ est supérieure à $0$, la racine $x_0$ se situe entre $a$ et $m$, et on met à jour l'intervalle qui devient $[a ; m]$.
		\item Si l'image $f(m)$ est inférieure à $0$, la racine $x_0$ si situe entre $m$ et $b$, et on met à jour l'intervalle qui devient $[m ; b]$.
	\end{itemize}
	Cet algorithme est donc récursif, avec condition de sortie généralement donnée par la longueur de l'intervalle : dès que l'intervalle est de longueur inférieur à $10^{-5}$ (par exemple), on s'arrête.
	
	Les quatre premières étapes sont décrites dans la figure \ref{fig:dichotomie}.
	On en déduit que $\sqrt{7} \in [2,625 ; 2,6875]$.
	
	L'implémentation de l'algorithme en Python est donnée figure \ref{algo:dichotomie-python}.
	Celle-ci nous donne 
		\[ \sqrt{7} \in [2,6457443237304688 ; 2,645751953125]. \]
	En continuant ainsi, on obtient l'approximation de la calculatrice :
		\[ \sqrt{7} \approx 2,645751311. \]
}{}

	\begin{figure}
	\begin{subfigure}{.5\textwidth}
	%\begin{center}
	\begin{tikzpicture}[>=stealth, scale=1]
	\begin{axis}[xmin = 1.9, xmax=3.1, ymin=-3.5, ymax=2.9, axis x line=middle, axis y line=middle, axis line style=<->, xlabel={}, ylabel={}, xtick = {2, 3}, ytick = {-4, -3, ..., 4}]
		
		% (f)
		\addplot[myb, very thick, domain =1.9:3.1, samples=50] {x^2-7}  node[pos = .7, above=15pt] {$\C_f$};
		
		% a
		\addplot[myg, thick, mark=|, mark size = 3] (2,0) node[above] {$a$};
		% b
		\addplot[myg, thick, mark=|, mark size = 3] (3,0) node[above] {$b$};
		% m1
		\addplot[myr, thick, mark=|, mark size = 3] (2.5,0) node[above] {$m_1$};
	\end{axis}
	\end{tikzpicture}
	\caption{Première étape. $a=2 ; m_1 = 2,5 ; b=3$.}
	\end{subfigure}
	\begin{subfigure}{.5\textwidth}
	\begin{tikzpicture}[>=stealth, scale=1]
	\begin{axis}[xmin = 2.4, xmax=3.1, ymin=-3.5, ymax=2.9, axis x line=middle, axis y line=middle, axis line style=<->, xlabel={}, ylabel={}, xtick = {2, 3}, ytick = {-4, -3, ..., 4}]
		
		% (f)
		\addplot[myb, very thick, domain =1.9:3.1, samples=50] {x^2-7}  node[pos = .7, above=15pt] {$\C_f$};
	
		% b
		\addplot[myg, thick, mark=|, mark size = 3] (3,0) node[above] {$b$};
		% m1
		\addplot[myg, thick, mark=|, mark size = 3] (2.5,0) node[above] {$m_1$};
		% m2
		\addplot[myr, thick, mark=|, mark size = 3] (2.75,0) node[below] {$m_2$};
	\end{axis}
	\end{tikzpicture}
	\caption{Troisième étape. $m_1=2,5 ; m_2=2,75 ; b = 3$.}
	\end{subfigure}
	
	\begin{subfigure}{.5\textwidth}
	\begin{tikzpicture}[>=stealth, scale=1]
	\begin{axis}[xmin = 2.4, xmax=2.85, ymin=-3.5, ymax=2.9, axis x line=middle, axis y line=middle, axis line style=<->, xlabel={}, ylabel={}, xtick = {2, 3}, ytick = {-4, -3, ..., 4}]
		
		% (f)
		\addplot[myb, very thick, domain =1.9:3.1, samples=50] {x^2-7}  node[pos = .7, above=15pt] {$\C_f$};
	
		% m1
		\addplot[myg, thick, mark=|, mark size = 3] (2.5,0) node[above] {$m_1$};
		% m2
		\addplot[myg, thick, mark=|, mark size = 3] (2.75,0) node[below] {$m_2$};
		% m3
		\addplot[myr, thick, mark=|, mark size = 3] (2.625,0) node[above] {$m_3$};
	\end{axis}
	\end{tikzpicture}
	\caption{Troisième étape. $m_1=2,5 ; m_2=2,75 ; m_3 = 2,6875$.}
	\end{subfigure}
	\begin{subfigure}{.5\textwidth}
	\begin{tikzpicture}[>=stealth, scale=1]
	\begin{axis}[xmin = 2.525, xmax=2.85, ymin=-3.5, ymax=2.9, axis x line=middle, axis y line=middle, axis line style=<->, xlabel={}, ylabel={}, xtick = {2, 3}, ytick = {-4, -3, ..., 4}]
		
		% (f)
		\addplot[myb, very thick, domain =1.9:3.1, samples=50] {x^2-7}  node[pos = .7, above=15pt] {$\C_f$};
	
		% m2
		\addplot[myg, thick, mark=|, mark size = 3] (2.75,0) node[below] {$m_2$};
		% m3
		\addplot[myg, thick, mark=|, mark size = 3] (2.625,0) node[above] {$m_3$};
		% m4
		\addplot[myr, thick, mark=|, mark size = 3] (2.6875,0) node[below] {$m_4$};
	\end{axis}
	\end{tikzpicture}
	\caption{Quatrième étape. $m_3 = 2,625 ; m_4 =2,6875 ; m_2=2,75$.}
	\end{subfigure}
	%\end{center}
	\caption{Les quatre premières étapes de l'algorithme de recherche de la racine positive de $f(x) = x^2 - 7$ en commençant avec l'intervalle $[2 ; 3]$. On en déduit que $\sqrt{7} \in [m_3 ; m_4] = [2,625 ; 2,6875]$. }
	\label{fig:dichotomie}
	\end{figure}


\begin{figure}[h]
\begin{mintedbox}{python}
def f(x):
	return x**2 - 7

precision = 1e-5 # précision souhaitée

def dichotomie(a,b):
	if b-a < precision: # condition de sortie
		return a, b
	m = (a+b)/2 # milieu de l'intervalle
	if f(m) > 0:
		return dichotomie(a,m)
	return dichotomie(m, b)
	
dichotomie(2,3) # car f(2) < 0 et f(3) > 0

precision = 1e-10 # meilleure précision
dichotomie(2,3)
\end{mintedbox}

	\caption{Programme de recherche de racine de $f(x)=x^2 - 7$ par dichotomie.}
	\label{algo:dichotomie-python}
\end{figure}

\IncMargin{1em}
\begin{algorithm}
\renewcommand{\algorithmcfname}{Algorithme}%
\SetKwInput{KwRes}{retourner}%
\SetKwIF{Si}{SinonSi}{Sinon}{si}{alors}{sinon si}{sinon}{fin si}%
\SetKwFor{Tq}{tant que}{faire}{fin tq}%
\SetKwInOut{Input}{entrée}\SetKwInOut{Output}{sortie}
	\Input{
	Fonction $f$ bien définie sur l'intervalle de départ $I=[a ;b]$ tel que $f(a) < 0 < f(b)$.
	Précision souhaitée $\epsilon > 0$.
	}
	\Output{Intervalle de taille inférieur à $\epsilon$ contenant une racine de $f$.}
	\BlankLine
	\Tq{$b-a > \epsilon$}{
		$m = \dfrac{a+b}2$ 		\hfill \emph{$m$ est le milieu de l'intervalle $I$}\\
		\Si{ $G(m) \geq 0$}{
			$I = [a ; m]$
		}
		\Sinon{
			$I = [m ; b]$
		}
	}
	\KwRes{L'intervalle $I$.}
\caption{Dichotomie autour d'une racine.}\label{algo:dichotomie}
\end{algorithm}\DecMargin{1em} 










%%%%%%%%%% IDÉES %%%%%%%%%%
%On compte les opérations ($+$ et $\times$) ce qui est une simplification de la réalité (Machine de Turning vs. arithmetic model : real RAM)
%En effet, fait $\times 2$ en binaire est plus facile que faire $\times3$, au même titre que faire $\times10$ est plus facile que faire $\times11$ en base $10$.
%Méoïzation? l'exemple de Fibonacci est un algo récursif pas hyper simple? Faisable quand même.
%Peut-être avec les deux algos et $t_1, t_2$ les temps au début et à la fin. Remplir $print(t2-t1)$ et comparer le résultat (tous les $F_{\leq n}$ vs juste $F_n$) et les temps d'execution des deux programmes pour $n=3, 4, 10, 20, 30$.
%Comparer le résultat et le timing avec la forme close encore (inexact sans rounding mais plus rapide?)
%Complexité simple exacte : $S = 1 + ... + n$ ($n-1$ opérations), vs $n(n+1)/2$ (4 opérations)
%Bien sûr généralisable avec des formules pour $S_3 = 1+2³+...+n³$ p.ex. [$2(n-1) + (n-1)$] opérations vs expression de forme fermée vs $S^2$.
%Questions : à partir de quel n vaut-il mieux utiliser la formule plutôt qu'une boucle for?
%Simplification de complexité avec erreur. P.ex. $\sin(x) \approx x$ ou $\cos(x) \approx 1 - \dfrac{x^2}2$.
%Algorithmes par dichotomie pour trouver une racine.
%Fast exponentiation : comment calculer $q^n$ le plus vite possible ?


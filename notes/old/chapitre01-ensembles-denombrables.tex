%!TEX encoding = UTF8
%!TEX root =notes.tex

\chapter{Ensembles dénombrables}

	\dfn{Ensemble de nombres}{
		On note un ensemble de nombres par des accolades \{  $\dots$ \}  entourant une liste de nombres.
	}{}
	\ex{}{
		Les ensembles suivants sont des ensembles de nombres.
		\begin{multicols}{3}
			$E_1 = \{ 1 ;  2 ; 3\}$, \\
			$E_2 = \left\{ -1,2 ; 4 ; \dfrac{7}{12} \right\}$, \\
			$E_3 = \{ 2\pi \}$.
		\end{multicols}
		Les nombres appartenant à un ensembles sont appelé les \emph{éléments} de l'ensemble.
		Tous les éléments sont distincts.
	}{}

	\dfn{Entiers et rationnels}{
		On définit les ensembles infinis suivants.
		\begin{multicols}{2}
		\end{multicols}
		Ils sont respectivement : les entiers \emph{naturels}, les entiers \emph{relatifs}, les nombres décimaux, et les nombres \emph{rationnels}.
	}{}
	
	\nt{
		Les décimaux $\D$ ont un nombre de décimales fini. En effet, la fraction $\dfrac{a}{10^n}$ pour $a \in \Z$ admet au plus $n$ décimales après la virgules.
	}
	
	\dfn{Appartenance, inclusion}{
		On pose les symboles suivants pour signifier l'appartenance et l'inclusion.
		\begin{itemize}
			\item $a \in E$ : l'élément $a$ appartient à l'ensemble $E$.
			\item $E \subset S$ : l'ensemble $E$ est inclus dans l'ensemble $S$.
		\end{itemize}	
	}{}
	\nt{
		Pour que $E \subset S$, tous les éléments de $E$ doivent aussi appartenir à $S$.
	}
	
	\ex{Appartenances et inclusions}{
		\begin{multicols}{3}
		\begin{itemize}
			\item  $2 \in \N$,
			\item $2/3 \in \Q$,
			\item $0{,}7 \in \D$,
			\item $ \left\{ 0 ; -3 ; \dfrac{32}{8} ; 1 \right\} \subset \Z$,
			\item $\N \subset \Z$,
			\item $\Z \subset \Z$.
		\end{itemize}
		\end{multicols}		
		On a la suite d'inclusions $\N \subset \Z \subset \D \subset \Q$.
	}{}
	
	\ex{Non appartenances et non inclusions}{
		\begin{multicols}{2}
		\begin{itemize}
			\item  $-2 \notin \N$,
			\item $\dfrac13 \notin \D$, 
			\item $2 \pi \notin \Q$,
			\item $ \{ 0 ; -1,2 ; 4 \}  \not\subset \Z$,
			\item $ \left\{ \dfrac{4}{13}; -1,5 ; 2,75 \right\} \not\subset \D$.
		\end{itemize}
		\end{multicols}	
	}{}
	
	
	
	\qs{}{
		Le rationnel $\dfrac13$ n'appartient pas à $\D$ car ses décimales se répètent infiniment. Qu'en est-il de $\dfrac17$ ou $\dfrac{5}{12}$ ?
		
		
		Le but du prochain chapitre est de démontrer rigoureusement que les fractions du type $\dfrac{5}{12}$ ne sont pas décimales : leur écriture est infinie.
	}
	\qs{}{
		Soit $\sqrt{2}$ le nombre positif qui vérifie $\left(\sqrt{2}\right)^2 = 2$. Est-ce que $\sqrt{2}$ est rationnel ?
		
		Nous démontrerons également que $\sqrt{2}$ n'est pas un nombre rationnel.
	}
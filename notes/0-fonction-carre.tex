%!TEX encoding = UTF8
%!TEX root = 0-notes.tex

\chapter{Fonction carré, fonction racine carrée, et valeur absolue}
\label{chap:fonction-carré}


\section{Fonction carré}

\notations{
	Pour n'importe quel nombre $x$ on dénote $x^2 = x \cdot x$ le produit de $x$ par lui-même.
	On lit « $x$ au carré » ou « $x$ carré ».
}

\dfn{Fonction carré}{
	La \emph{fonction carré} est la fonction qui prend un nombre $x$ et renvoie son carré, $x^2 = x \cdot x$ :
		\[ f(x) = x^2 \qquad \pourtout x\in\R. \]
}{dfn:fonction-carré}

\exe{}{
	Donner $\bigset{k^2 \text{ où } k = 0, 1, 2, \dots, 12}$.
}{exe:carrés-parfaits}{
	Les carrés de 0 à 12 sont
		\[ \bigset{0, 1, 4, 9, 16, 25, 36, 49, 64, 81, 100, 121, 144}. \]
}

\exe{}{
	Tracer la courbe représentation de la fonction carré sur le domaine $\D = [-4 ; 4]$.
}{exe:courbe-carré}{
	Les images $f(-4) = f(4) = 16, f(-3) = f(3) = 9, f(-2) = f(2) = 4, f(-1)=f(1) = 1$, et $f(0) =0$ permettent de tracer la courbe bleue suivante. La courbe verte est la version avec 101 points choisis uniformément.
	\begin{center}
	\begin{tikzpicture}[>=stealth, scale=1]
		\begin{axis}[xmin = -4.1, xmax=4.1, ymin=-.1, ymax=16.1, axis x line=middle, axis y line=middle, axis line style=->, grid=both]
			\addplot[no marks, BLUE_E, -, very thick] expression[domain=-4:4, samples=9]{x^2} node[pos=.92, left]{$y=x^2$};
		\end{axis}
	\end{tikzpicture}
	\begin{tikzpicture}[>=stealth, scale=1]
		\begin{axis}[xmin = -4.1, xmax=4.1, ymin=-.1, ymax=16.1, axis x line=middle, axis y line=middle, axis line style=->, grid=both]
			\addplot[no marks, GREEN_E, -, very thick] expression[domain=-4:4, samples=101]{x^2} node[pos=.92, left]{$y=x^2$};
		\end{axis}
	\end{tikzpicture}
	\end{center}
}

\exe{, difficulty=1}{
	Montrer que si $0 < x < 1$, alors $x^2 < x$, et que si $x>1$, alors $x^2 > x$.
}{exe:sq-id}{
	On peut multiplier chaque inégalité par $x$ sans changer l'ordre dès que celui-ci est strictement positif.
	Ainsi $x < 1 \implies x^2 < x$, et $x>1 \implies x^2 > x$ pour n'importe quel $x>0$.
}

%\thm{Signe du produit}{
%	Soit $A, B \in \R$ deux réels.
%	On distingue trois cas sur le signe du produit $A \cdot B$.
%		\begin{enumerate}
%			\item Si $A \cdot B = 0$, alors $A=0$ ou $B=0$. (le « ou » est inclusif)
%			\item Si $A \cdot B > 0$, alors $A$ et $B$ ont le même signe :
%				\begin{enumerate}[label=\roman*)]
%					\item soit $A, B > 0$ ;
%					\item soit $A, B < 0$.
%				\end{enumerate}
%			\item Si $A \cdot B < 0$, alors $A$ et $B$ sont de signes opposés :
%				\begin{enumerate}[label=\roman*)]
%					\item soit $A> 0$ et $B < 0$ ;
%					\item  soit $A < 0$ et $B > 0$.
%				\end{enumerate}
%		\end{enumerate}
%}{thm:signe-produit}

\thm{Propriétés du carré}{
	Soient $a, b \in\R$ deux réels, $b$ non nul lorsque dénominateur.
	Alors
		\begin{align*}
			(a \cdot b)^2 = a^2 \cdot b^2, && \et && \left(\dfrac{a}{b}\right)^2 = \dfrac{a^2}{b^2}.
		\end{align*}
}{thm:prop-carré}


\thm{Signe du produit}{
	Le signe d'un produit suit les règles suivantes, où la positivité et la négativité sont strictes :
	\begin{itemize}
		\item $\oplus \cdot \oplus = \oplus$
		\item $\ominus \cdot \ominus = \oplus$
		\item $\oplus \cdot\ominus = \ominus$
	\end{itemize}
}{thm:signe-produit}

\thm{Positivité du carré}{
	Pour n'importe quel $x\in\R$, $x^2 \geq 0$ avec égalité uniquement en $x=0$.
}{thm:carré-positif}

\pf{}{
	On a d'abord $0^2 = 0$.
	Ensuite, si $x\neq0$ est non nul on distingue les deux cas suivants.
	\begin{itemize}
		\item Si $x>0$, alors $x^2 = x\cdot x >0$ comme produit de deux positifs.
		\item Si $x<0$, alors $x^2 = x\cdot x > 0$ comme produit de deux négatifs.
	\end{itemize}
}

\mprop{Symétrie du carré}{
	La fonction $f(x) = x^2$ vérifie $f(-x)=f(x)$ pour tout $x\in\R$.
	Graphiquement, la courbe représentative $\C_f$ est symétrie par rapport à l'axe des ordonnées.
}{prop:sym-carré}

\pf{}{
	$f(-x) = (-x)^2 = \bigl( (-1)x \bigr)^2 = (-1)^2 x^2 = x^2 = f(x)$ pour tout $x\in\R$.
}

\dfn{Fonction paire}{
	Une fonction quelconque $f$ définie sur $\R$ est dite \emph{paire} dès que $f(x) = f(-x)$ pour tout $x\in\R$.
	Sa courbe représentative $\C_f$ est symétrie par rapport à l'axe des ordonnées.
}{dfn:fonction-paire}

\exe{}{
	Montrer que les fonctions suivantes sont paires.
	\begin{multicols}{2}
	\begin{enumerate}
		\item $f(x) = 2x^2$
		\item $g(x) = 4x^2 + 7$
		\item $h(x) = \dfrac{1}{4x^2 + 7}$
		\item $k(x) = x^4$
	\end{enumerate}
	\end{multicols}
}{exe:paires}{
	Pour tout $x\in\R$ on a bien 
	\begin{multicols}{2}
	\begin{enumerate}
		\item $f(-x) = 2(-x)^2 = 2x^2 = f(x)$.
		\item $g(-x) = 4(-x)^2 + 7 = 4x^2 + 7 = g(x)$
		\item $h(-x) = \dfrac{1}{4(-x)^2 + 7} = \dfrac{1}{4x^2 + 7} = h(x)$
		\item $k(-x) = (-x)^4 = \bigl( (-x)^2 \bigr)^2 = \bigl( x^2 \bigr)^2 = x^4 = k(x)$
	\end{enumerate}
	\end{multicols}
}

\exe{, difficulty=1}{
	Montrer que la fonction $f(x) = x^2 - x^3$ n'est pas paire.
}{exe:non-paire}{
	Supposons que $f$ soit paire.
	Par définition, $f(-x) = f(x)$ pour tout $x\in\R$.
	
	Comme $(-x)^3 = (-x)(-x)^2 = -x^3$, on a $f(-x) = x^2 + x^3$.
	Par conséquence, $f(x) = f(-x) \iff x^2 - x^3 = x^2 + x^3 \iff x^3 = 0$.
	Or ceci n'est pas vrai pour tout $x\in\R$ : prendre n'importe quel $x$ non nul fournit un contre-exemple.
	Ainsi pour $x=1$, $f(1) = 0$ et $f(-1) = 2$ et $f$ n'est en effet pas paire.
}

\exe{, difficulty=2}{
	Montrer que si $g$ est une fonction de $x^2$, alors $g$ est paire.
	Plus précisément, on suppose que $g(x) = h(x^2)$ pour une certaine fonction $h$.
}{exe:gpaire}{
	$g(-x) = h\bigl( (-x)^2 \bigr) = h(x^2) = g(x)$ pour tout $x\in\R$, donc $g$ est paire.
}

\exe{, difficulty=2}{
	Montrer que si $k$ est une fonction quelconque, alors la fonction $l(x) = k(x) + k(-x)$ est paire.
}{exe:symétrisation}{
	$l(-x) = k(-x) + k(-(-x)) = k(-x) + k(x) = k(x) + k(-x) = l(x)$ pour tout $x\in\R$, donc $k$ est paire.
}

\section{Fonction racine carrée}
\label{sec:racine-carrée}

\dfn{Racine carrée}{
	Soit $a \in \R, a\geq0$ un réel positif ou nul.
	La quantité $\sqrt{a}$ est l'unique nombre réel positif ou nul vérifiant
		\[ \sqrt{a}^2 = a. \]
}{}

\exe{}{
	Compléter les pointillés.
	\begin{multicols}{2}
	\begin{enumerate}
		\item $\sqrt{25} = \dots$
		\item $\sqrt{81} = \dots$
		\item $\sqrt{121} = \dots$
		\item $\sqrt{\quad\dots\quad} = 25$
		\item $\sqrt{\quad\dots\quad} = 12$
		\item $\sqrt{\quad\dots\quad} = 10^3$
	\end{enumerate}
	\end{multicols}
}{exe:sqrt1}{
	TODO
}

\exe{}{
	Calculer les racines carrées suivantes.
	\begin{multicols}{2}
	\begin{enumerate}
		\item $\sqrt{7^2}$
		\item $\sqrt{17}^2$
		\item $\sqrt{(-9)^2}$
		\item $\sqrt{10^4}$
		\item $\left(-\sqrt{4}\right)^2$
		\item $-\sqrt{15^2}$
	\end{enumerate}
	\end{multicols}
}{exe:sqrt2}{
	TODO
}

% pas hyper intéressant sans notion de croissance de sqrt je trouve
%\exe{}{
%	Donner un encadrement des nombres à l'unité.
%		\begin{multicols}{2}
%		\begin{enumerate}
%			\item $\sqrt{43}$
%			\item $\sqrt{70,8}$
%			\item $\sqrt{\dfrac{61}7}$
%			\item $\sqrt{101,204}$
%		\end{enumerate}
%		\end{multicols}
%}{}

\exe{}{
	Montrer que la notation $x = \sqrt{-1}$ n'a pas de sens pour $x\in\R$ réel.
}{exe:sqrt-undef}{
	À supposer qu'il existe un tel $x\in\R$, alors il vérifierait $x^2 = -1$.
	Ce n'est pas possible car la fonction carré est toujours positive. \Large\Lightning
}

\dfn{Domaine de définition}{
	Soit $f$ une fonction sur $\R$.
	Le plus grand domaine de $\R$ sur lequel $f$ est bien définie est le \emph{domaine de définition} de $f$.
}{}

\notations{
	Le domaine de définition de $f$ est noté $\D_f$.
}

\mprop{}{
	Le domaine de définition de la fonction racine carrée est $[0 ; \pinfty[$.
}{prop:domaine-sqrt}

\notations{
	On note également $\R_+ = [0 ; \pinfty[$ l'ensemble des réels positifs ou nuls.
}

\thm{Propriétés de la racine carrée}{
	Soient $a, b \in\R$ deux réels positifs, $b$ non nul lorsque dénominateur.
	Alors
		\begin{align*}
			\sqrt{a \cdot b} = \sqrt{a} \cdot \sqrt{b}, && \et && \sqrt{\dfrac{a}{b}} = \dfrac{\sqrt{a}}{\sqrt{b}}.
		\end{align*}
}{thm:prop-sqrt}

\exe{}{
	Calculer les produits suivants.
	\begin{multicols}{2}
	\begin{enumerate}
		\item $\sqrt{169} \cdot \sqrt{81}$
		\item $\sqrt{169 \cdot 81}$
		\item $\sqrt{0,16} \cdot \sqrt{900}$
		\item $\sqrt{0,16 \cdot 900}$
	\end{enumerate}
	\end{multicols}
}{exe:sqrt3}{
	TODO
}

\exe{}{
	Écrire les nombres suivants sous la forme $a\sqrt{b}$ où $a\in\N$ et $b\in\N$ est le plus petit entier possible.

	\begin{multicols}{2}
	\begin{enumerate}
		\item $\sqrt{12}$
		\item $\sqrt{150}$
		\item $5\sqrt{96}$
		\item $2\sqrt{300}$
		\item $\dfrac{12}{\sqrt{3}}$
		\item $\dfrac{18}{\sqrt{6}}$
	\end{enumerate}
	\end{multicols}
}{exe:sqrt4}{
	TODO
}


\exe{}{
	Écrire les nombres suivants sous la forme $\sqrt{a}$ où $a\in\N$.
	\begin{multicols}{2}
	\begin{enumerate}
		\item $3\sqrt{2}$
		\item $50\sqrt{0,5}$
	\end{enumerate}
	\end{multicols}
}{exe:sqrt5}{
	TODO
}

\exe{}{
	Calculer les fractions suivantes.
	\begin{multicols}{2}
	\begin{enumerate}
		\item $\dfrac{\sqrt{64}}{\sqrt{4}}$
		\item $\sqrt{\dfrac{64}4}$
		\item $\dfrac{\sqrt{0,81}}{\sqrt{0,09}}$
		\item $\sqrt{\dfrac{0,81}{0,09}}$
	\end{enumerate}
	\end{multicols}
}{exe:sqrt6}{
	TODO
}

\section{Fonction valeur absolue}

\dfn{Valeur absolue}{
	On nomme la fonction $f(x) = \sqrt{x^2}$ la fonction \emph{valeur absolue}.
	On note $|x|$ la valeur absolue de $x$, image de $x$ par $f$.
}{dfn:valeur-absolue}

\exe{}{
	Écrire les nombres suivants sans les barres de valeur absolue.
	\begin{multicols}{3}
	\begin{enumerate}[label=\roman*)]
		\item $|-7|$
		\item $|8|$
		\item $|-13-8|$
		\item $|\pi - 4|$
		\item $|-5+3| + |-7+4|$
		\item $|\sqrt{2} - \sqrt{3}|$
	\end{enumerate}
	\end{multicols}
}{exe:valeur-absolue}{
	\begin{multicols}{3}
	\begin{enumerate}[label=\roman*)]
		\item $7$
		\item $8$
		\item $21$
		\item $4 - \pi$
		\item $5$
		\item $\sqrt{3}- \sqrt{2}$
	\end{enumerate}
	\end{multicols}
}

\mprop{}{
	La valeur absolue est toujours positive ou nulle.
	
	On a en fait
		%\[ |x| =  \begin{cases} x \text{ si $x$ \geq 0}, \\ -x \text{ si $x$ \leq 0}. \end{cases} \]
}{prop:caractérisation-valeur-absolue}

\pf{}{
	Si $x\geq0$, alors $x$ est le nombre positif tel que, mis au carré, il vaut $x^2$.
	C'est donc la racine carrée de $x^2$.

	Si $x\leq0$, le même raisonnement ne tient pas, car la racine carrée de $x^2$ est le nombre \emph{positif} tel que, mis au carré, il vaut $x^2$.
	Le nombre $-x$ fonctionne ici : il est positif, et $(-x)^2 = (-1)^2 x^2 = x^2$.
	Par conséquent, $\sqrt{x^2} = -x$ dans ce cas.
}

\exe{}{
	Montrer que pour tout $x\in\R$, $|x| = |-x|$.
}{exe:moinsabs}{
	Comme $x^2 = (-x)^2$, appliquer la racine carrée conclut.
	
	On aurait aussi pû étudier les cas $x\geq0$ et $x\leq0$ séparément :
	\begin{itemize}
		\item Si $x\geq0$, alors $-x\leq0$ et $|-x| = -(-x)= x = |x|$.
		\item Si $x\leq0$, alors $-x\geq0$ et $|x| = -x= |-x|$.
	\end{itemize}
}

\exe{}{
	Montrer que pour tout $x\in\R$, $|2x| = 2|x|$.
}{exe:2abs}{
	Comme $(2|x|)^2 = 4x^2 = (2x)^2$, appliquer la racine carrée conclut.
	
	On aurait aussi pû étudier les cas $x\geq0$ et $x\leq0$ séparément :
	\begin{itemize}
		\item Si $x\geq0$, $2x\geq0$ et $|2x| = 2x = 2|x|$.
		\item Si $x\leq0$, alors $2x\leq0$ et $|2x| = -2x= 2(-x) = 2|x|$.
	\end{itemize}
}

\exe{}{
	Montrer que pour tout $x\in\R$, $|x|^2 = x^2$.
}{exe:abscarre}{
	$|x| = x$ ou $-x$. Dans le premier cas, il n'y a rien à montrer, et dans le second, $(-x)^2 = (-1)^2 x^2 = x^2$.
}

\thm{}{
	Pour tout $x, y\in\R$, on a
		\[ |x\cdot y| = |x|\cdot|y|. \]
}{thm:produit-valeur-absolue}

\pf{}{
	D'après l'exercice \ref{exe:abscarre}, on a $(|x|\cdot|y|)^2 = |x|^2 |y|^2 = x^2 y^2 = (xy)^2$.
	En appliquant la racine carrée, on trouve
		\[ |x| \cdot |y| = \big| |x|\cdot|y| \big| = |xy|. \]
}

\thm{}{
	Soit $x, y \in \R$. Alors la distance entre $x$ et $y$ sur la droite réelle est donnée par 
		\[ \text{distance}(x,y) = |x-y| = |y-x|. \]
}{thm:valeurs-absolue-distance}

\pf{}{
	Remarquons d'abord que $|x-y| = |y-x|$ d'après l'exercice \ref{exe:abs} et car $x-y$ est l'opposé de $y-x$.	

	Si $x \leq y$, alors la distance entre $x$ et $y$ sur la droite réelle est donnée par $y-x$ qui vaut $|y-x|$ par positivité.
	Sinon, $x \geq y$, alors la distance entre $x$ et $y$ sur la droite réelle est donnée par $x-y$ qui vaut $|x-y|$, à nouveau par positivité.

	Dans tous les cas, la distance vaut donc bien $|x-y| = |y-x|$.
}

\exe{}{
	Donner l'ensemble des $x\in\R$ vérifiant $|x| = 4$.
}{exe:vabs-4}
{
	Comme $|x|$ vaut soit $x$, soit $-x$, on a $x=4$ ou $-x=4 \iff x = -4$.
	Ainsi $\bigset{ x \in \R \tq |x| = 4 } = \bigset{-4 ; 4}$.
}

\exe{}{
	Donner l'ensemble des $x\in\R$ vérifiant $x^2 = 16$.
}{exe:carre-16}
{
	En appliquant la racine carrée, on a $|x| = \sqrt{x^2} = \sqrt{16} = 4$.
	L'exercice \ref{exe:vabs-4} permet de conclure que $\bigset{ x \in \R \tq x^2 = 16 } = \bigset{-4 ; 4}$.
}

\exe{}{
	Donner l'ensemble des $x\in\R$ vérifiant $|x| \leq 9$.
}{exe:vabs-9}
{
	Comme $|x|$ donne la distance de $x$ à 0, on a 
		\[ \bigset{ x \in \R \tq |x|\leq9 } = [-9; 9]. \]
}

\exe{}{
	Donner l'ensemble des $x\in\R$ vérifiant $x^2 \leq 81$.
}{exe:carre-81}
{
	En appliquant la racine carrée, et comme la fonction racine est croissante, on a $x^2 \leq 81 \iff |x| \leq 9$.
	L'exercice \ref{exe:vabs-9} permet de conclure que $\bigset{ x \in \R \tq x^2 \leq 81 } = [-9 ; 9]$.
}

\cor{Manipulation d'égalités, d'inégalités avec valeurs absolues}{
	Soit $E\in\R$ une expression quelconque et $a\geq0$ un réel positif ou nul.
	Alors 
		\begin{align*}
			| E | = a && \iff && E = a \text{ ou } E = -a, \\
			|E| \leq a && \iff && E \in [-a ; a], \\
			%|E| \geq a && \iff && E \in ]\minfty; -a] \text{ ou } E\in [a ; \pinfty[.
		\end{align*}
}{cor:manipulations-valeurs-absolues}

\exe{, difficulty=1}{
	Démontrer le corollaire \ref{cor:manipulations-valeurs-absolues}.
}{exe:manipulations-valeurs-absolues}{
	TODO
}

\exe{, difficulty=2}{
	Donner l'ensemble des $x\in\R$ vérifiant $|x| \geq 21$.
}{exe:vabs-9}
{
	Comme $|x|$ donne la distance de $x$ à 0, on a 
		\[ \bigset{ x \in \R \tq |x|\geq21 } = ]\minfty; -21] \cup [21 ; \pinfty[. \]
}


\cor{}{
	Pour $r, c \in\R$ et $r\geq0$, on a l'égalité d'ensembles
		\[ \bigset{ x \in \R \tq |x - c| \leq r} = [c - r ; c+r]. \]
}{cor:boule-rc}

\exe{}{
	Démontrer le corollaire \ref{cor:boule-rc}.
}{exe:boule-rc}{
	L'ensemble $\bigset{ x \in \R \tq |x - c| \leq r}$ contient tous les nombres réels $x$ à distance inférieure à $r$ de $c$.
	C'est donc un segment centré en $c$ et de rayon $r$, l'équivalent du disque en une dimension.
	
	Purement algébriquement, on a $|E| \leq r \iff E\in[-r ; r] \iff -r \leq E \leq r$ d'après le corollaire \ref{cor:manipulations-valeurs-absolues}.
	Avec $E = x-c$, on a $-r \leq x-c \leq r \iff c-r \leq x \leq c+r \iff x \in [c-r ; c+r]$.
}

\exe{, difficulty=2}{
	Montrer que pour tout $x,y\in\R$, on a 
		\begin{align*}
			\min\{x,y\} = \dfrac{x+y}2 - \dfrac{|x-y|}2, && \text{ et } && \min\{x,y\} = \dfrac{x+y}2 + \dfrac{|x-y|}2.
		\end{align*}
}{exe:valeur-absolue-minmax}{
	Quels que soient $x, y\in\R$, le segment borné par $x$ et $y$ est de milieu $\dfrac{x+y}2$ et de longueur $|x-y|$.
	Ses extrémités s'expriment donc comme $c - r$ et $c+r$ où $c$ est le centre du segment et $r$ son rayon, c'est-à-dire la moitié de sa longueur.

	Algébriquement, si $x<y$, alors $\dfrac{x+y}2 - \dfrac{|x-y|}2 = \dfrac{x+y}2 - \dfrac{y-x}2 = \dfrac{(x+y)-(y-x)}2 = \dfrac{2x}2 = x = \min\{x ,y\}$.
	On peut faire de même pour le maximum puis dans le cas $x>y$ (ou se convaincre que le cas $x>y$ n'est pas à faire par symétrie des expressions).
}

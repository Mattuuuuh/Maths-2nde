%!TEX encoding = UTF8
%!TEX root =notes.tex

\setcounter{maincounter}{0}
\chapter{Nombres rationnels, nombres réels}

\section{Ensembles de nombres}

\notations{
	Un collection \emph{non ordonnée} d'éléments est un \emph{ensemble}, noté avec des accolades et des points-virgules entre les éléments.
	
		\[ E = \bigset{ 2 ; 3 ; 5 ; 7 ; 11 }. \]
	
	Un ensemble peut contenir un nombre infini d'éléments.
	
		\[ F = \bigset{ 0 ; 2 ; 4 ; 6 ; 8 ; 10 ; \dots }. \]
}

\dfn{Nombres entiers}{
	On pose les nombres entiers \emph{naturels}
		\[ \N =  \bigset{ 0 ; 1 ; 2 ; \dots }, \]
	et les nombres entiers \emph{relatifs}
		\[ \Z =  \bigset{ \dots ; -2 ; -1 ; 0 ; 1 ; 2 ; \dots }. \]
}{dfn:nombres-entiers}
	
	
\notations{
	Il est aussi possible de décrire l'ensemble $F$
		\[ F =  \bigset{ 0 ; 2 ; 4 ; 6 ; 8 ; 10 ; \dots } \]
	sous la forme
		\[ F =  \bigset{ n \in \N \tqs \text{$n$ est pair} }. \]
	La deuxième forme est plus précise et évite au lecteur d'avoir à deviner la structure de l'ensemble.
}

\dfn{Nombres rationnels}{
	Soit $a, b \in \Z$ deux entiers relatifs, avec $b$ non nul.
	
	Alors le \emph{rationnel} $\dfrac{a}{b}$ est l'unique nombre tel que
		\[ \dfrac{a}b \times b = a. \]
	
	On note l'ensemble de tous les rationnels
		\[ \Q = \Bigset{ \dfrac{a}{b} \tq a \in \Z, b \in \Z, b \neq 0}. \]
}{dfn:nombres-rationnels}

\ex{}{
	Le nombre rationnel $\dfrac12$ est l'unique nombre vérifiant
		\[ \dfrac12 \times 2 = 1. \]
	Comme $0,5 \times 2 = 1$, on a $\dfrac12  = 0,5$. C'est le \emph{développement décimal} de $\dfrac12$.
	
	Le nombre rationnel $\dfrac13$ est l'unique nombre vérifiant
		\[ \dfrac13 \times 3 = 1. \]
}{ex:nombres-rationnels}

\qs{}{
	Est-il possible d'écrire le développement décimal de $\dfrac13$ sur un feuille de papier A4 ?
	
	Le but de la prochaine section est de répondre à cette question par la négative.
}

\section{Développements décimaux}

\dfn{Nombres décimaux}{
	On dit d'un nombre $A$ qu'il est \emph{décimal} si on peut écrire son développement décimal.
	Autrement dit, son développement décimal est fini.
	
	On note 
		\[ \D = \bigset{ A \in \Q \tq \text{ le développement décimal de $A$ est fini} }. \]
}{dfn:nombres-décimaux}

\thm{Caractérisation de $\D$}{
	On a 
		\[ \D = \Bigset{ \dfrac{a}{10^n} \tq a \in \Z, n \in \N }. \]
}{}

\thm{}{
	Le nombre rationnel $\dfrac13$ n'est pas décimal : $\dfrac13 \not\in \D$.
}{}

\pf{}{
	Par l'absurde, supposons que $\dfrac13 = \dfrac{a}{10^n}$.
	Alors on a nécessairement 
		\[ 10^n = 3a. \]
	Comme le membre de droite est un multiple de $3$, le membre de gauche aussi.
	
	On arrive ici à une contradiction car une puissance de $10$ ne peut pas être un multiple de $3$.
	En effet, les multiples de $3$ adviennent tous les $3$ entiers ($\{0 ; 3 ; 6 ; 9 ; 12 ; \dots\}$).
	Or l'entier avant $10^n$ est un multiple de $3$, car c'est 
		\[ 10^n - 1 = \underbrace{99\cdots99}_{\text{n fois}} = 3 \times \underbrace{33\cdots33}_{\text{n fois}} . \]
}

\mprop{}{
	Le développement décimal de $\dfrac13$ est donné par
		\[ \dfrac13 = 0,333\dots.\]
	Il est \emph{infini} et \emph{périodique}.
}{}

\pf{}{
	Posons $x = 0,333\dots$ et montrons que $x=\dfrac13$.
	On a 
		\[ 10x = 3,333\dots = 3 + x. \]
	Il suit donc que $9x = 3$, et donc que $x=\dfrac13$.
}{}

\dfn{Développement périodique}{
	On dit que le développement décimal d'un nombre est périodique dès qu'il se répète à partir d'un certain point.
}{}

\thm{}{
	Si $x$ est rationnel, alors $x$ admet un développement décimal périodique.
}{}

\thm{Contraposée}{
	Si $x$ admet un développement décimal périodique, alors $x$ est rationnel.
}{}

\nt{
	On a donc l'égalité d'ensembles
		\[ \bigset{ \text{nombres $x$ tels que le développement décimal de $x$ est périodique} } = \Q. \]
}

\qs{}{
	Existe-t-il un nombre $x$ tel que son développement décimal ne soit pas périodique ?
}

\section{Nombres réels}

\thm{}{
	Il existe un nombre $x$ non rationnel.
}{}



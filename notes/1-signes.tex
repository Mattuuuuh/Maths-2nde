%!TEX encoding = UTF8
%!TEX root = 0-notes.tex

\chapter{Signes et fonction inverse}
\label{chap:signes}

\section{Étude de signes}

\subsection{Fonction affine}

\dfn{signe d'un nombre réel}{
	Soit $x\in\R$ un nombre réel quelconque.
	On dit que
		\begin{itemize}
			\item $x$ est \emphindex{strictement positif} si $x>0$ ;
			\item $x$ est \emphindex{strictement négatif} si $x<0$ ;
			\item $x$ est \emphindex{nul} si $x=0$.
		\end{itemize}
	En français, dire \og $x$ est positif \fg~signifie que $x$ est positif ou nul ($x\geq0$).
	Il est préférable de dire \emphindex{positif ou nul} pour éviter toute ambiguïté.
}{}

\nt{
	Pour simplifier la notation, au lieu d'écrire \og $A >0$ et $B > 0$ \fg, on écrira juste \og$A, B > 0$\fg.
}{}

\thm{signe du produit}{
	Soit $A, B \in \R$ deux réels.
	On distingue trois cas sur le signe du produit $A \cdot B$.
		\begin{enumerate}
			\item Si $A \cdot B = 0$, alors $A=0$ ou $B=0$. (le \og ou \fg~est inclusif)
			\item Si $A \cdot B > 0$, alors $A$ et $B$ ont le même signe :
				\begin{enumerate}[label=\roman*)]
					\item soit $A, B > 0$ ;
					\item soit $A, B < 0$.
				\end{enumerate}
			\item Si $A \cdot B < 0$, alors $A$ et $B$ sont de signes opposés :
				\begin{enumerate}[label=\roman*)]
					\item soit $A> 0$ et $B < 0$ ;
					\item  soit $A < 0$ et $B > 0$.
				\end{enumerate}
		\end{enumerate}
}{thm:signe-produit}


\cor{signe du quotient}{
	Soit $A, B \in \R$ deux réels. On suppose que $B \neq 0$.
	
	Alors, le signe de $\dfrac1B$ est le même que celui de $B$, et donc le signe du quotient
		\[ \dfrac{A}{B} = A \cdot \dfrac1B \]
	suit les mêmes règles que celui du produit $A \cdot B$.
}{}

\exe{1}{
	Pour chacune des fonctions affines $f$ suivantes, donner l'intervalle 
		\[ \{ x \in \R \tq f(x) \geq 0 \}. \]
	
	\begin{multicols}{2}
	\begin{enumerate}
		\item $f(x) = x+3$
		\item $f(x) = 3x-10$
		\item $f(x) = x$
		\item $f(x) = -4x-12$
		\item $f(x) = -10x$
		\item $f(x) = 1$
	\end{enumerate}
	\end{multicols}
}{exe:signes1}{
	\begin{enumerate}
		\item 
		L'inégalité $x+3 \geq 0$ est équivalente à $x \geq -3$.
		On obtient donc 
			\[ \{ x \in \R \tq x+ 3 \geq 0 \} = \{ x \in \R \tq x \geq -3 \} = [-3 ; \pinfty[. \]
		\item 
			\[ \left[ \dfrac{10}3 ; \pinfty \right[ \]
		\item
			\[ [0 ; \pinfty[ \]
		\item
		L'inégalité $-4x - 12 \geq 0$ est équivalente à $-4x \geq 12$.
		Pour continuer, on multiplie par $-\dfrac14$ qui est négatif, ce qui change donc le sens de l'inégalité : $x \leq -3$.
		On obtient donc
			\[ ]\minfty ; -3]. \]
		\item
			\[ ]\minfty ; 0] \]
		\item 
			Tous les $x\in\R$ vérifie que $f(x) \geq 0$ car $1 \geq 0$ est toujours vrai.
			Donc 
			\[ \{ x \in \R \tq 1 \geq 0 \} = \R. \]
	\end{enumerate}
}

\exe{}{
	\begin{multicols}{2}
	Exprimer par lecture graphique les ensembles suivants sous forme d'union d'intervalles.
		\begin{enumerate}
			\item $\bigset{ x \in [-3 ; 3] \tq f(x) \geq 0 }$
			\item $\bigset{ x \in [-3 ; 3] \tq g(x) \leq 0 }$
			\item $\bigset{ x \in [-3 ; 3] \tq f(x) \cdot g(x) \geq 0 }$
		\end{enumerate}
	
	\begin{center}
	\includegraphics[page=2]{figures/fig-signes.pdf}
	\end{center}
	\end{multicols}
}{exe:signes2}{
	L'ordonnée du point d'abscisse $x$ appartenant à $\C_f$ est $f(x)$.
	On lit donc son signe en regardant s'il est au-dessus ou en-dessous de l'axe des abscisses.
	
	\begin{enumerate}
		\item On a environ 
			\[ \bigset{ x \in [-3 ; 3] \tq f(x) \geq 0 } \approx [-1,5 ; 1,5]. \]
		\item On a environ 
			\[ \bigset{ x \in [-3 ; 3] \tq g(x) \leq 0 } \approx [-3 ; -1,2] \cup [1,9 ; 3]. \]
		On est en droit de se demander : comment sait-on que $g(-2) \leq 0$ si on ne peut pas lire sa valeur graphiquement ?
		On suppose en fait ici que $\C_g$ est une courbe continue (tracée sans lever le crayon). 
		Ainsi, si $g(-2)$ était positif, la courbe devrait nécessairement couper l'axe des abscisses, ce qui n'est visiblement pas le cas.
		\item Il y a deux situations dans lesquelles le produit $f(x) \cdot g(x)$ peut être positif : soit $f(x)$ et $g(x)$ sont positifs, soit ils sont tous deux négatifs.
		Graphiquement, on lit que
			\[ \bigset{ x \in [-3 ; 3] \tq f(x) \geq 0 \text{ et } g(x) \geq 0 } \approx [-1,2 ; 0,2], \]
		et que (en faisant à nouveau une hypothèse de continuité)
			\[ \bigset{ x \in [-3 ; 3] \tq f(x) \leq 0 \text{ et } g(x) \leq 0 } \approx [-3 ; -1,6] \cup [1,6 ; 1,9]. \]
		On prend l'union des deux pour obtenir
			\[ \bigset{ x \in [-3 ; 3] \tq f(x) \cdot g(x)  \geq 0 } \approx [-1,2 ; 0,2] \cup [-3 ; -1,6] \cup [1,6 ; 1,9]. \]
	\end{enumerate}
}

\exe{}{
	Pour chaque propriété, donner une fonction $f$ sur $\R$ non identiquement nulle la vérifiant.
	\begin{enumerate}
		\item $f$ s'annule en $1$.
		\item $f$ s'annule en $-10$.
		\item $f$ s'annule en $0$.
	\end{enumerate}
}{exe:signes3}{
	La fonction identiquement nulle $f(x) = 0$ n'est bien sûr pas autorisée sinon l'exercice n'a pas d'intérêt !
	\begin{enumerate}
		\item $f(x) = x-1$ fonctionne ainsi que $f(x) = 3(x-1)$, ou $f(x) = -(x-1) = 1-x$.
		\item $f(x) = x+10$ fonctionne, ainsi que $f(x) = (x+10)(x-1)$, et tous ses multiples.
		\item $f(x) = x$ ou $f(x) = 500x$.
	\end{enumerate}

}

\subsection{Produits de fonctions affines}

\ex{problème de signe}{
	On souhaite connaître pour quels $x\in\D$ l'inégalité suivante est vérifée.
		\[ f(x) = \dfrac{6x^2 - 5x - 4}{3-x} \geq 0. \]
	En premier lieu, remarquons que certaines valeurs sont interdites : ce sont les valeurs pour lesquelles le dénominateur s'annule.
	Le domaine de $f$ est donc tout $\R$ perforé en $x=3$ :
		\[ \D = \R - \{3 \} = ]\minfty ; 3[ \cup ]3 ; \pinfty[. \]
	
	Ensuite, on vérifie qu'on a bien 
		\[ (3x-4)(2x+1) = 6x^2 - 5x - 4, \]
	factorisation qui nous est donnée par le professeur.
	
	On obtient alors
		\[ f(x) = \dfrac{(3x-4)(2x+1)}{3-x}, \]
	et connaître le signe de chacun des facteurs permettra de connaitre le signe de $f$.
	
	On a les équivalences suivantes.
		\begin{align*}
			3x-4 \geq 0 && 2x+1  \geq 0 && 3-x \geq 0 \\
			\iff\quad && \iff\quad && \iff\quad \\
			x \geq \dfrac43 && x \geq -\dfrac12 && x \leq 3
		\end{align*}
	On synthétise les signes obtenus dans le tableau suivant qu'on appelle \emphindex{tableau de signes}.
	
	\begin{center}
	\includegraphics[page=1]{figures/fig-signes.pdf}
	\end{center}
	
	Le signe de $f(x)$ dans la première case est déduit du fait qu'on fait \og moins $\times$ moins $\times$ plus \fg, qui donne \og plus \fg, car le produit d'un positif avec deux négatifs est positif. On fait idem pour le reste.
	
	De plus, on note avec les doubles barres $||$ la valeur interdite $x=3$.
	
	En conclusion, l'ensemble des $x \in \D$ vérifiant $f(x) \geq 0$ est donné par
		\[ \bigset{ x \in \R-\{3\} \tq f(x) \geq 0 } = \left]\minfty ; -\dfrac12 \right] \bigcup \left[ \dfrac43 ; 3 \right[. \]
}{}


\ex{problème de racines}{
	On souhaite connaître les solutions de l'équation du deuxième degré d'inconnue $x\in\R$ :
		\[ 21 x^2 - 11 x - 2  = 0. \]
	Ce problème est un large domaine d'étude en mathématiques, c'est le problème des \emphindex{racines} d'une fonction. Ici, la fonction est $f(x) = 21 x^2 - 11 x - 2 $, et ses racines sont les antécédents qui annulent $f$.
	
	Plusieurs stratégies s'offrent à nous pour résoudre $f(x)=0$ : 
		\begin{itemize}
			\item Tracer $\C_f$ et lire les racines graphiquement. Cette méthode est utile pour approximer, mais ne permet pas d'obtenir les solutions exactes (ni le nombre de solutions, d'ailleurs).
			\item Tester des valeurs à partir du tracé. On pourrait démarrer vers une valeur approximative lue graphiquement puis se diriger vers la racine en affinant le pas au fur et à mesure. Cette méthode est très utilisée en pratique (\emph{cf.} la  dichotomie au chapitre d'algorithmique) mais ne permet parfois pas non plus d'obtenir une solution exacte.
			\item Essayer de factoriser $f(x)$ en un produit de deux expressions linéaire et utiliser le théorème \ref{thm:signe-produit}. L'avantage est qu'on obtient des solutions exactes. Un (gros) inconvénient est que cela n'est pas faisable pour toutes les fonctions $f$ (voir le théorème d'Abel-Ruffini). Pour les fonctions du deuxième degré, ça fonctionne cependant tout le temps : la résolution générale est au programme de $1$ère.
		\end{itemize}
	
	Ici, on vérifie que
		\begin{align*}
			(3x-2)(7x+1) &= 3x(7x + 1) - 2(7x+1) \\
							&= 21x^2 + 3x - 14x - 2 \\
							&= 21x^2 - 11x - 2 \\
							&= f(x).
		\end{align*}
	D'après le théorème  \ref{thm:signe-produit}, on a
		\begin{align*}
			&& f(x) = 0  && \\
			&& (3x-2)(7x+1) = 0 && \\
			3x-2 = 0 && \text{ ou }&& 7x + 1 = 0 \\
			x = \dfrac23 && \text{ ou } && x = -\dfrac17
		\end{align*}
}{}

\exe{}{
	À l'aide de l'exercice \ref{exe:signes1} remplir les tableaux de signes ci-dessous et donner
		\begin{align*}
			\{ x \in \R \tq f(x) \geq 0 \}, && \{ x \in \R \tq g(x) \leq 0 \}, && \{ x \in \R \tq h(x) \geq 0 \}
		\end{align*}
	sous forme d'union d'intervalles.
	
	\begin{center}
	\includegraphics[page=3]{figures/fig-signes.pdf}
	\includegraphics[page=4]{figures/fig-signes.pdf}
	\includegraphics[page=5]{figures/fig-signes.pdf}
	\end{center}
}{exe:signes4}{
	On remplit le tableau de signe en calculant d'abord quand chaque fonction affine s'annule : c'est exactement quand elle change de signe (du positif vers le négatif, ou l'inverse).
	
	Par exemple, $x+3$ s'annule en $x=-3$, et l'exercice \ref{exe:signes1} donne que $x+3$ est positif après $-3$, est donc nécessairement négatif avant.
	On fait idem pour $3x-10$, qui s'annule en $\dfrac{10}3$, et qui est positif après, et négatif avant.
	
	Le signe du produit $f(x) = (x+3)(3x-10)$ est déduit du signe des facteurs, à l'aide des règles ``positif $\times$ positif = positif ; négatif $\times$ négatif = positif ; positif $\times$ négatif = négatif''.
	De plus, lorsqu'un des deux facteurs des nul, le produit est forcément nul, multiplier par zéro donne toujours zéro.
	
	Le deuxième tableau est similaire, en faisant attention au fait que le signe de $-4x-12$ est positif puis négatif (cf. exercice \ref{exe:signes1}).
	Ceci est en fait dû au fait qu'on ait multiplié par $-\dfrac14$ lors de la résolution de $-4x-12 \geq 0$.
	Le signe du coefficient directeur $a$ de la fonction affine donne donc l'allure de la droite associée : si $a>0$, la droite monte du négatif vers le positif en passant par $0$, et si $a <0$, la droite descend du positif vers le négatif en passant par $0$.
	
	Le cas $a=0$ est le cas de la fonction constante, dont l'étude de signe n'est pas très intéressante.
	
	\begin{center}
	\includegraphics[page=6]{figures/fig-signes.pdf}
	\includegraphics[page=7]{figures/fig-signes.pdf}
	\includegraphics[page=8, scale=.9]{figures/fig-signes.pdf}
	\end{center}
}


\exe{}{
	Pour chaque propriété, donner une fonction $f$ sur $\R$ non identiquement nulle la vérifiant.
	\begin{enumerate}
		\item $f$ s'annule en $-10$ et en $1$.
		\item Les racines de $f$ sont $2, -3, \dfrac27$, et $0$.
	\end{enumerate}
}{exe:signes5}{
	La fonction identiquement nulle $f(x) = 0$ n'est bien sûr pas autorisée sinon l'exercice n'a pas d'intérêt !
	\begin{enumerate}
		\item $f(x) = (x+10)(x-1) = x^2 +9x -10$.
		\item $f(x) = (x-2)(x+3)(x-\frac27)x$.
	\end{enumerate}
}


\exe{}{
	On souhaite connaître les solutions de l'équation du deuxième degré d'inconnue $x\in\R$ :
		\begin{align}
			22x^2 - 125x + 22  = 0. \label{eq:1}
		\end{align}
	\begin{enumerate}
		\item Montrer que $22x^2 - 125x + 22 = (2x-11)(11x-2)$.
		\item En déduire l'ensemble des solutions de l'équation \eqref{eq:1}.
	\end{enumerate}
}{exe:signes6}{
	\begin{enumerate}
		\item On part toujours de la forme factorisée qu'on développe par double distributivité :
			\[ (2x - 11)(11x - 2) = 22x^2 - 4x - 121x + 22 = 22x^2 - 125x + 22. \]
		\item On utilise que le produit est nul si est seulement si un des deux facteurs est nul.
		L'ensemble des solutions de \eqref{eq:1} est donc $\left\{ \dfrac{11}2 ; \dfrac2{11} \right\}$.
	\end{enumerate}
}

\exe{}{
	On souhaite connaître les solutions de l'équation du troisième degré d'inconnue $x\in\R$ :
		\begin{align}
			4x^3 - 6x^2 - 2x + 3  = 0. \label{eq:2}
		\end{align}
	\begin{enumerate}
		\item Montrer que $4x^3 - 6x^2 - 2x + 3 = (2x-3) \left(2x^2-1\right)$.
		\item En déduire l'ensemble des solutions de l'équation \eqref{eq:2}.
	\end{enumerate}
}{exe:signes7}{

	\begin{enumerate}
		\item On part toujours de la forme factorisée qu'on développe par double distributivité.
		On utilise aussi que $x^2 \cdot x = x \cdot x \cdot x = x^3$.
			\[ (2x - 3)(2x^2 - 1) = 4x^3 - 2x -6x^2 + 3. \]
		\item On utilise que le produit est nul si est seulement si un des deux facteurs est nul.
		Or l'équation $2x^2 - 1 = 0$ est équivalente à $x^2 = \dfrac12$.
		En prenant la racine et en utilisant que $\sqrt{x^2} = |x|$, on obtient
			\[ |x| = \sqrt{\dfrac12} = \dfrac1{\sqrt2}. \]
		Les deux valeurs $\pm \dfrac1{\sqrt2}$ (notation $\pm$ pour \og plus ou moins \fg) sont donc solutions.
		On en déduit que l'ensemble des solutions de \eqref{eq:2} est $\left\{ \dfrac32 ; \pm \dfrac1{\sqrt2} \right\} = \left\{ \dfrac32 ; \dfrac1{\sqrt2} ; - \dfrac1{\sqrt2} \right\}$
	\end{enumerate}


}

\exe{}{
	On souhaite connaître les solutions de l'équation du quatrième degré d'inconnue $x\in\R$ :
		\begin{align}
			16x^4 - 24x^2 + 9  = 0. \label{eq:3}
		\end{align}
	\begin{enumerate}
		\item Montrer que $16x^4 - 24x^2 + 9 = \left(4x^2-3\right)^2$.
		\item En déduire l'ensemble des solutions de l'équation \eqref{eq:3}.
	\end{enumerate}
}{exe:signes8}{

	\begin{enumerate}
		\item On part toujours de la forme factorisée qu'on développe à l'aide de l'identité remarquable $(a-b)^2 = a^2 + b^2 -2ab$.
		On utilise aussi que $(x^2)^2 = x^2 \cdot x^2 = x\cdot x\cdot x\cdot x = x^4$.
			\[ (4x^2 - 3)^2 = (4x^2)^2 + 3^2 - 2\cdot(4x^2)\cdot3 = 16x^4 + 9 -24x^2. \]
		\item On utilise que le produit est nul si est seulement si un des deux facteurs est nul.
		Ici, les deux facteurs sont les mêmes, car $(4x^2 - 3)^2 = (4x^2 - 3)(4x^2 - 3)$, donc on se remet à résoudre $(4x^2 - 3) = 0$.
		Ceci est équivalent à $x^2 = \dfrac34$, et donc l'ensemble des $x$ vérifiant \eqref{eq:3} est donné par $\left\{ \pm \sqrt{\dfrac34} \right\} = \left\{ \pm \dfrac12\sqrt{3} \right\}= \left\{ \dfrac12\sqrt{3} ;  -\dfrac12\sqrt{3} \right\}$.
	\end{enumerate}

}

\exe{}{
	Remplir approximativement les tableaux ci-dessous à l'aide des graphes de $\C_f$ et $\C_g$ sur le domaine $\D = [-11; -2,5]$.
	
	\begin{center}
	\includegraphics[page=9]{figures/fig-signes.pdf}
	
	\includegraphics[page=10]{figures/fig-signes.pdf}
	
	\includegraphics[page=11]{figures/fig-signes.pdf}
	\end{center}
	
}{exe:signes9}{
	
	\begin{center}
	\includegraphics[page=12]{figures/fig-signes.pdf}
	\includegraphics[page=13]{figures/fig-signes.pdf}
	\end{center}
}


\section{Fonction inverse}

TODO

\section{Domaine de définition}

\dfn{domaine de définition, valeurs interdites}{
	Soit $f$ une fonction quelconque. On souhaite connaître le domaine le plus grand possible sur lequel définir $f$.
	On l'appelle le \emphindex{domaine de définition}, qu'on note $\D_f$.
	
	Pour cela, on étudie la fonction sur $\R$ dont on supprime les \emphindex{valeurs interdites}, c'est-à-dire les antécédents pour lesquels $f$ est mal définie.
	On étudiera dans ce chapitre deux conditions qui permettront d'identifier ces valeurs interdites :
		\begin{enumerate}
			\item l'expression sous une racine carrée doit toujours être positive ou nulle ; et
			\item un dénominateur ne doit jamais être nul.
		\end{enumerate}
}{}

\ex{valeurs interdites}{
	Soit $f$ la fonction donnée par
		\[ f(x) = \dfrac{\sqrt{7-2x}}{x}. \]
	On identifie deux conditions qui doivent être vérifées pour que calculer $f(x)$ ait un sens :
		\begin{enumerate}
			\item $7-2x \geq 0$ ; et
			\item $x \neq 0$.
		\end{enumerate}
	La première inégalité nous impose que $x \leq \dfrac72$, et la deuxième condition supprime $0$.
	On obtient donc
		\[ \D_f = \left] \minfty; \frac72 \right] - \{0\} =  \left] \minfty; 0 \right[ \cup  \left] 0; \frac72 \right]. \]
}{}

\exe{}{
	À l'aide de l'exercice \ref{exe:signes1}, donner le domaine de définition de chaque fonction suivante.
	
	\begin{enumerate}
		\item $f(x) = \sqrt{x+3}$
		\item $g(x) = \sqrt{3x-10}$
		\item $h(x) = \sqrt{-4x-12}$
	\end{enumerate}

}{exe:signes10}{
	Le domaine de définition est le plus grand domaine sur lequel une fonction est bien définie.
	Ici, la seule chose qui peut se mal passer est qu'on demande à $f$ de calculer une racine carrée d'un nombre négatif.
	Ceci n'est pas possible car $x^2 \geq 0$ pour tout $x\in\R$, donc si on souhaite calculer $\sqrt{a}$ vérifiant $\sqrt{a}^2 = a$ par définition, on a nécessairement $a\geq0$.
	Par exemple, $\sqrt{-1}$ ne peut pas être un nombre réel, sinon $\sqrt{-1}^2 = -1$, et un carré serait négatif !
	
	On impose donc que l'expression sous une racine carrée soit positive ou nulle, et on prend tous les réels vérifiant cette propriété.
	\begin{enumerate}
		\item
			\[ \D_f = \{ x \in \R \tq x + 3 \geq 0 \} = [-3; \pinfty[, \]
		d'après l'exercice \ref{exe:signes1}.
		\item
			\[ \D_f = \{ x \in \R \tq 3x- 10 \geq 0 \} = \left[ \dfrac{10}3 ; \pinfty \right[, \]
		d'après l'exercice \ref{exe:signes1}.
		\item
			\[ \D_f = \{ x \in \R \tq -4x-12 \geq 0 \} = ]\minfty; -3], \]
		d'après l'exercice \ref{exe:signes1}.
	\end{enumerate}
}

\exe{}{
	À l'aide de l'exercice \ref{exe:signes4}, donner le domaine de définition de la fonction
		\[ f(x) = \sqrt{-10x(x+3)(3x-10)}. \]
}{exe:signes11}{
	La seul contrainte à poser sur $x$ est que l'expression sous la racine soit positive ou nulle.
		\[ \D_f = \{ x \in \R \tq -10x(x+3)(3x-10) \geq 0 \}. \]
	Or la dernière ligne du dernier tableau de signes de l'exercice \ref{exe:signes4} nous donne exactement les intervalles où $h(x) = -10x(x+3)(3x-10)$ est positive ou nulle :
		\[ \D_f = ]\minfty ; -3] \cup \left[0 ; \dfrac{10}3 \right]. \]
}

\exe{}{
	On souhaite connaître pour quels $x\in\D_f$ l'inégalité suivante est vérifée.
		\[ f(x) = \dfrac{2x+1}{7x^2 - 20x - 3} \geq 0 \]
	
	\begin{enumerate}
		\item Montrer que $7x^2 - 20x - 3 = (x-3)(7x+1)$.
		\item En déduire les valeurs interdites à $f$ et donc le domaine de définition $\D_f$.
		\item Remplir le tableau de signes ci-dessous.
		\item Exprimer $\{ x \in \D_f \tq f(x) \geq 0 \}$ sous forme d'union d'intervalles.
	\end{enumerate}
	
	
	\begin{center}
	\includegraphics[page=14]{figures/fig-signes.pdf}
	\end{center}
}{exe:signes12}{
	\begin{enumerate}
		\item 
		Par double distributivité,
			\[ (x-3)(7x+1) = x^2 \cdot (7) + x \cdot (1 - 21) + (-3) = 7x^2 - 20x - 3. \]
		\item
		Il n'y a pas de racine carrée dans l'expression de $f$ : la seule opération illégale est la division par zéro.
		Le dénominateur $(x-3)(7x+1)$ est nul lorsqu'un des deux facteur est nul, c'est-à-dire lorsque $x=3$ ou $x= -\dfrac17$.
		Par conséquent, $\D_f = \R - \left\{ 3 ; -\dfrac17 \right\}$.
		
		\item 
		Dans la dernière ligne, on note par les doubles barres les valeurs interdites à $f$.
		Lorsque le numérateur $2x+1$ est nul, $f$ est bien nulle, mais $f$ n'est pas définie aux $x$ pour lesquels le dénominateur s'annule.
		\item 
			\[ \{ x \in \D_f \tq f(x) \geq 0 \} = \left[ -\dfrac12 ; -\dfrac17 \right[ \cup ]3 ; \pinfty[. \]
		Les valeurs interdites $-\frac17$ et $3$ ne sont pas incluses car elles n'appartiennent pas à $\D_f$, ensemble dans lequel on pioche nos $x$.
	\end{enumerate}
	
	\begin{center}
	\includegraphics[page=15]{figures/fig-signes.pdf}
	\end{center}
}

\exe{}{
	Donner le domaine de définition $\D_f$ de la fonction
		\[ f(x) = \dfrac{\sqrt{7+2x}}{(3x+1) \sqrt{2-x}}, \]
	et trouver l'ensemble des $x\in\D_f$ vérifiant $f(x) \leq 0$.
}{exe:signes13}{
	On identifie quatre contraintes sur $x$ :
		\begin{enumerate}
			\item $7+2x \geq 0$ ;
			\item $3x+1 \neq 0$ ;
			\item $2-x \geq 0$ ; et
			\item $2-x \neq 0$.
		\end{enumerate}
	On traite d'abord les inégalités qui imposent $x \geq -\dfrac72$, et $x\leq 2$.
	Ainsi, $x$ appartient nécessairement à l'intervalle $\left[ -\dfrac72 ; 2 \right]$.
	
	Les autres contraintes imposent que $x \neq -\dfrac13$ et $x\neq 2$, ce qui perfore l'intervalle obtenu :
		\[ \D_f = \left[ -\dfrac72 ; 2 \right] - \left\{ -\dfrac13 ; 2 \right\} = \left[ -\dfrac72 ; -\dfrac13 \right[ \cup \left] -\dfrac13; 2 \right[. \]
	
	Pour obtenir le signe, remarquons qu'il n'est que nécessaire d'étudier le signe de $3x+1$, car les racines carrées sont toujours positives.
	Or $3x+1$ est négatif avant sa racine et positif après, ce qui implique que
		\[ \{ x \in \D_f \tq f(x) \leq 0 \} = \D_f \cap \left]\minfty ; -\dfrac13 \right] = \left[ -\dfrac72 ; -\dfrac13 \right[. \]
}

\exe{}{
	Soit une fonction $f$ donnée par
		\[ f(x) = \dfrac{(4x-8)(x-3)}{-2x^2 + 5x + 12}. \]
	\begin{enumerate}
		\item Montrer que $-2x^2 + 5x + 12 = -2\left(x+\dfrac32\right)(x-4)$.
		\item En déduire $\D_f$.
		\item Donner $\{ x \in \D_f \tq f(x) \leq 0 \}$ sous forme d'union d'intervalles.
	\end{enumerate}
}{exe:signes14}{
	\begin{enumerate}
		\item 
		Par double distributivité, on obtient
			\[ -2\left(x+\dfrac32\right)(x-4) = -2 \left( x^2 + \dfrac32 x - 4x - \dfrac32 \cdot 4 \right) = -2x^2 + 5x + 12. \]
		\item 
		Les seules contraintes sont que le dénominateur ne s'annule pas :
			\[ \D_f = \R - \left\{ - \dfrac32 ; 4 \right \}. \]
		\item
		On crée un tableau de signes utilisant que
			\[ f(x) = \dfrac{(4x-8)(x-3)}{-2\left(x+\dfrac32\right)(x-4)} = -\dfrac12 \cdot \dfrac{(4x-8)(x-3)}{\left(x+\dfrac32\right)(x-4)}. \]
		Remarquons, de manière importante, que le signe de $f$ est donc l'opposé du signe de $\dfrac{(4x-8)(x-3)}{\left(x+\dfrac32\right)(x-4)}$ à cause du $-2$ !
		On l'ajoute donc au tableau pour ne pas l'oublier...
	\end{enumerate}


	\begin{center}
	\includegraphics[page=15]{figures/fig-signes.pdf}
	\end{center}
	
	D'où 
		\[  \{ x \in \D_f \tq f(x) \leq 0 \} = \left] \minfty ; -\dfrac32 \right[ \cup  \left[ 2 ; 3 \right]  \cup \left]4;\pinfty \right[. \]
}

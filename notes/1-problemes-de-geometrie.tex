%!TEX encoding = UTF8
%!TEX root = 0-notes.tex

\chapter{Problèmes de géométrie}
\label{chap:pb-geom}

\section{Projeté orthogonal}

	Dans cette section, et en mathématiques en général, les adjectifs \og orthogonal \fg \, et \og perpendiculaire \fg \, sont synonymes.
	On parlera alors de projeté orthogonal d'un point sur une droite lorsque les droites engendrées sont perpendiculaires.

\dfn{Projeté orthogonal}{
	Le \emphindex{projeté orthogonal} d'un point $M$ sur une droite $(d)$ est l'\emph{unique} point $M'$ tel que les droites $(MM')$ et $(d)$ soient perpendiculaires.
}{}

	\begin{center}
	\includegraphics[page=8]{figures/fig-geom.pdf}
	\end{center}

\thm{Distance minimale}{
	Le point $M'$, projeté orthogonal de $M$ sur $(d)$, est l'\emph{unique} point de $(d)$ le plus proche de $M$.
	
	Autrement dit, $M'$ est l'\emph{unique} minimiseur de la distance à $M$ en restant sur $(d)$.
}{thm:proj-min}



	\begin{center}
	\includegraphics[page=9]{figures/fig-geom.pdf}
	\end{center}

\pf%{Démonstration du théorème \ref{thm:proj-min}}
{}{
	Considérons un autre point $P \in (d)$ sur la droite $(d)$.
	Le triangle $MPM'$ est rectangle en $M'$, et le théorème de Pythagore implique donc que
		\[ MP^2 = MM'^2 + M'P^2. \]
	Or, comme un carré est toujours positif, on a les inégalités suivantes.
		\begin{align*}
			M'P^2 &\geq 0 \\
			M'P^2 + MM'^2 &\geq MM'^2 \\
			MP^2 &\geq MM'^2
		\end{align*}
	La distance de $M$ à $P$ est donc toujours plus grande ou égale à celle de $M$ à $M'$, ce qui démontre que $M'$ minimise sa distance à $M$ en restant sur $(d)$.
	
	Pour l'unicité, remarquons que l'égalité $MP^2 = MM'^2$ n'est vraie que lorsque $M'P^2 = 0$, c'est-à-dire lorsque $P=M'$. 
	$M'$ est donc bien l'unique point de $(d)$ le plus proche de $M$.
}

\ex{Hauteur d'un triangle}{
	Pour calculer l'aire d'un triangle on crée un rectangle grâce à sa hauteur et, par symétrie, on déduit que
		\[ \text{Aire}(\text{triangle}) = \dfrac{\text{base} \cdot \text{hauteur}}2. \]
	
	\begin{center}
	\includegraphics[page=1, scale=1]{figures/fig-geom.pdf}
	\end{center}
	\warning\textbf{La hauteur est une fonction de la base. Si un autre côté est pris pour base, alors la hauteur sera différente.}
	La formule de l'aire ne dépend pas de la base choisie, mais il faut impérativement que la hauteur soit perpendiculaire à cette base, et qu'elle aille jusqu'au point culminant du triangle. 
	Ce point est le sommet du triangle qui n'appartient pas à la base.
	
	Autrement dit, la hauteur suit le projeté orthogonal du sommet n'appartenant pas à la base sur la droite engendrée par cette base.
	
}{}

\section{Trigonométrie}

Les angles sont généralement dénotés par des lettres grecques minuscules.
	\begin{align*}
		\alpha &: \text{\og alpha \fg} & \beta &: \text{\og beta \fg} & \gamma &: \text{\og gamma \fg} \\
		\theta &: \text{\og theta \fg} &  \delta &: \text{\og delta \fg} & \omega &: \text{\og omega \fg}
	\end{align*}

On rappelle les formules de trigonométrie uniquement applicables dans un triangle rectangle.
\warning\textbf{Les côtés opposé et adjacent dépendent de l'angle choisi.}

	\begin{align*}
		\text{sinus} = \dfrac{\text{côté opposé}}{\text{hypoténuse}}
		&& 
		\text{cosinus} = \dfrac{\text{côté adjacent}}{\text{hypoténuse}}
		&&
		\text{tangente} = \dfrac{\text{côté opposé}}{\text{côté adjacent}}
	\end{align*}
	
	\begin{center}
	\includegraphics[page=10, scale=1]{figures/fig-geom.pdf}
	\hspace{30pt}
	\includegraphics[page=11, scale=1]{figures/fig-geom.pdf}
	\end{center}
Moyens mnémotechniques : 
	\begin{center}
	SOH CAH TOA ou CAH SOH TOA.
	\end{center}

Remarquons immédiatement qu'on a la relation
	\begin{align*}
		\dfrac{\text{sinus}}{\text{cosinus}}
		= \dfrac{\text{opposé}}{\text{hypoténuse}} \cdot \dfrac{\text{hypoténuse}}{\text{adjacent}}
		= \dfrac{\text{opposé}}{\text{adjacent}}
		 = \text{tangente}.
	\end{align*}


\thm{}{
	Soit $\theta \in [0; 90\degree]$ un angle aigu ou droit.
	Alors 
		\[ \tan(\theta) = \dfrac{\sin(\theta)}{\cos(\theta)}. \]
}{}

\subsection{Théorème de Thalès}
	
Ces formules sont en fait une conséquence du théorème de Thalès : en connaissant deux angles et un côté d'un triangle, on peut l'agrandir ou le rapetisser pour maintenir les proportions et les angles et se ramener à un triangle rectangle étalon d'hypoténuse de longueur $1$.

	\begin{center}
	\includegraphics[page=2, scale=1]{figures/fig-geom.pdf}
	\end{center}

En effet, les deux triangles ci-dessus sont semblables (on obtient l'un en (dé)zoomant l'autre), et en \emph{définissant} les longueur des côtés du deuxième par $\cos(\alpha)$ et $\sin(\alpha)$, le théorème de Thalès nous donne, par exemple,
	\[ \dfrac{\cos(\alpha)}{1} = \dfrac{\text{opposé}}{\text{hypoténuse}}. \]

En conclusion, le cosinus, le sinus, et la tangente sont donc en fait une banque de données de longueurs de côtés d'un triangle particulier qu'on a mesuré à l'avance.
Certaines formules mathématiques permettent de calculer exactement ces valeurs, mais en général elle sont approximatives (mais aussi précises qu'on le souhaite !) et il n'existe pas de formule compacte pour les exprimer.

\subsection{Quart de cercle trigonométrique}

Considérons, dans le quadrant supérieur-droit d'un repère d'origine $O$, un quart de cercle de rayon $1$.
On choisit un angle $\theta \in [0, 90\degree]$, et on nomme le point $P$ du quart de cercle tel que l'angle entre $(OP)$ et l'axe des abscisse soit $\theta$.

	\begin{center}
	\includegraphics[page=3, scale=1.25]{figures/fig-geom.pdf}
	\end{center}

Le triangle rectangle ainsi créé est d'hypoténuse $1$ et vérifie donc les propriétés du triangle étalon considéré dans la section précédente.
On a donc, par définition,
	\begin{align*}
		x = \cos(\theta), && \text{ et } && y = \sin(\theta).
	\end{align*}
Si on a pas peur des raisonnements circulaires, on peut revérifier les égalités avec les formules apprises.
Par exemple,
	\[
		\cos(\theta) = \dfrac{\text{adjacent}}{\text{hypoténuse}} = \dfrac{x}{1}.
	\]
En conclusion, les coordonnées du point $P$ sont exactement
	\[ P = (x ; y) = \left( \cos(\theta) ; \sin(\theta) \right). \]
Dans ce contexte, on déduit naturellement le théorème suivant, après une introductions aux notations.

\notations{
	Comme il est assez fastidieux d'écrire $ \bigl(\cos\left(\theta\right)\bigr)^2$,
	on optera pour la notation plus allégée suivante :
		\[\cos^2 (\theta) =  \bigl(\cos\left(\theta\right)\bigr)^2. \]
}{}

\thm{}{
	Soit $\theta \in [0; 90\degree]$ un angle aigu ou droit.
	Alors, 
		\begin{align*}
			0 \leq \cos(\theta) \leq 1, &&
			0 \leq \sin(\theta) \leq 1,
		\end{align*}
	et
		\begin{gather*}
			\cos^2(\theta) + \sin^2(\theta) = 1.
		\end{gather*}
}{}




\section{Calculs exacts}

\subsection{Calculs exacts de $\cos(\theta), \sin(\theta), \tan(\theta)$ pour $\theta=30^\circ, 60^\circ$}

Considérons le triangle équilatéral suivant.
Par symétrie, tous les angles sont égaux à 60°.
En prenant une hauteur quelconque, on obtient un triangle rectangle 30-60-90 comme suit.

	\begin{center}
	\includegraphics[page=12]{figures/fig-geom.pdf}
	\hspace{4cm}
	\includegraphics[page=13]{figures/fig-geom.pdf}
	\end{center}

Le théorème de Pythagore donne la longueur $h$ de la hauteur : 
	\[ 1^2 = \bigl(\dfrac12\bigr)^2 + h^2 \iff h^2 = \dfrac34 \iff h = \dfrac{\sqrt3}2. \]
On peut donc connaître les valeurs de $\cos(\theta), \sin(\theta), \tan(\theta)$ pour $\theta\in\bigset{ 30^\circ ; 60^\circ}$.

\mprop{}{
	% moche de faire comme ça mais ça donne le meilleur rendu imo
		\[ \sin(30^\circ) = \dfrac12, \qquad \cos(30^\circ) = \dfrac{\sqrt{3}}2, \qquad \tan(30^\circ) = \dfrac1{\sqrt{3}}. \]
	
		\[ \cos(60^\circ) = \dfrac12, \qquad \sin(60^\circ) = \dfrac{\sqrt{3}}2, \qquad \tan(60^\circ) =\sqrt{3}.\]
}{prop:données}

\exe{}{
	Vérifier qu'on a bien $\cos^2(\theta) + \sin^2(\theta) = 1$ pour $\theta = 30^\circ, 60^\circ$.
}{exe:c2s2-vérif}{
	TODO
}

\subsection{Calculs exacts de longueurs : manipulation des racines carrées}

Le calcul exact de longueurs implique de pouvoir manipuler les racines carrées dans les équations.
La racine et ses propriétés ont été étudiées dans la section \ref{sec:racine-carrée} du chapitre \ref{chap:fonction-carré}.

\ex{}{
	Pour résoudre pour $x\in\R$, $x\neq0$, l'équation 
		\[ \dfrac8x  =\dfrac{\sqrt{2}}2, \]
	on souhaite ramener $x$ au numérateur afin d'obtenir quelque chose de la forme $x = \dots$.
	Pour ça, on multiplie par $x$ les deux membres de l'égalité pour trouver
		\[ 8 = \dfrac{\sqrt{2}}2 x. \]
	On multiplie ensuite par $2$ et par $\dfrac1{\sqrt{2}}$ pour obtenir ce qu'on voulait :
		\[ 8 \cdot 2 \cdot \dfrac1{\sqrt{2}} = x. \]
	Reste à simplifier le résultat obtenu : 
		\[ x = \dfrac{16}{\sqrt{2}}. \]
	On peut manipuler davantage l'expression en multipliant le numérateur et le dénominateur par $\sqrt{3}$ pour trouver
		\[ x = \dfrac{16 \sqrt{2}}{\sqrt{2}^2} = 8 \sqrt{2}. \]
	Les deux nombres $\dfrac{16}{\sqrt{2}}$ et $8 \sqrt{2}$ sont égaux, au même titre que $\dfrac36 = \dfrac12$.
	La deuxième expression est cependant plus agréable à lire et à écrire.
}{ex:eq-sqrt}

\qs{}{
	Peut-on simplifier encore davantage l'expression $8 \sqrt{2}$ ?
	Autrement dit, peut-on écrire $\sqrt2$ comme fraction de deux entier à multiplier par $8$ ?
	$\sqrt{2}$ est-il un nombre rationnel ou irrationnel ?
}

\thm{}{
	Soit $d \in\N$ un entier naturel qui n'est pas un carré parfait ($d\neq k^2, k\in\N$).
	Alors $\sqrt{d}$ n'est pas rationnel.
}{thm:sqrt-irrational}

\cor{}{
	La racine carrée de $2$ est irrationnelle : $\sqrt2 \notin \Q.$
	On écriture décimale est infinie est apériodique.
}{cor:sqrt2-irrational}

\pf{démonstration du corrolaire \ref{cor:sqrt2-irrational}}{
	Par l'absurde, supposons que $\sqrt2 \in \Q$ soit rationnel.
	
	Par définition, $\sqrt2 = \dfrac{a}{b}$ avec $a, b \in \N$ deux entiers.
	Ainsi, par mise au carré,
		\begin{align*}
			2 = \left(\dfrac{a}b\right)^2 = \dfrac{a^2}{b^2} && \iff && a^2 = 2b^2.
		\end{align*}
	Étudions cette égalité d'entiers à l'aide du corollaire \ref{cor:carré-paire}.
	À gauche, $a^2$ est un carré parfait, et donc la puissance de $2$ dans sa décomposition en facteurs premiers est nécessairement paire.
	À droite, la même chose tient pour $b^2$. Il suit que, dans la décomposition en facteurs premiers de $2b^2$, la puissance de $2$ est impaire.
	
	L'entier $a^2$ admet donc deux décompositions en facteurs premiers différentes, ce qui contredit le théorème fondamental de l'arithmétique. {\Large\Lightning}
	
	Concernant l'écriture décimale d'un nombre irrationnel, voir le théorème \ref{thm:décimal-périodique}.
}

\exe{1 , difficulty=2}{
	Démontrer le théorème \ref{thm:sqrt-irrational}.
}{exe:sqrt-irrational}{
	Si $\sqrt{d} = \dfrac{a}{b}$, alors $a^2 = d\cdot b^2$.
	Comme $d$ n'est pas un carré parfait, un des premiers de sa décomposition apparaît à puissance impaire.
	Ce premier apparaît à puissance paire dans $b^2$ et donc impaire dans $a^2$, ce qui est une contradiction. \Large\Lightning
}

\exe{, difficulty=2}{
	Soit $x\in\R$ un nombre vérifiant $x^3 = 2$.
	Montrer que $x$ n'est pas rationnel : $x\not\in\Q$.
}{exe:cbrt2-irr}{
	Supposons que $x = \dfrac{a}b$. En mettant au cube, on obtient $a^3 = 2b^3$, égalité d'entiers naturels. 
	Par unicité de la décomposition de facteurs premiers, les deux nombres admettent la même décomposition.
	À gauche, le nombre premier 2 apparaît à une puissance qui est un multiple de 3.
	À droite, le premier 2 apparaît à une puissance qui est un multiple de 3 plus 1. \Large\Lightning
}

\exe{, difficulty=2}{
	En musique, la gamme classique contient douze notes avec un rapport de fréquence de 2 sur l'octave.
	La gamme est dite à tempérament égal si le rapport de fréquence entre deux notes consécutives est toujours le même.
	Notons ce rapport $r\in\R$.
	Justifier que $r^{12} = 2$ puis démontrer que $r$ est un nombre irrationnel : $r \not\in\Q$.
}{exe:temp-égal}{
	Le rapport de fréquence entre deux notes consécutives étant toujours $r$, on multiplie $r$ par lui-même 12 fois pour obtenir le rapport de fréquence entre deux notes de distance $12$, c'est-à-dire deux notes séparée par un octave.
	Ainsi $r^{12} = 2$ car l'octave correspond à un rapport de fréquence de 2.
	
	Supposons désormais que $r$ soit rationnel. Alors $r^{6}$ est aussi rationnel.
	Cependant, $(r^6)^2 = r^{12} = 2$, donc $r^{6} = \sqrt2$, qui n'est pas rationnel. \Large\Lightning
}

\exe{}{
	Résoudre les équations suivantes pour $x\in\R$, $x\neq0$. Exprimer $x$ sous la forme $q \sqrt{b}$ avec $q\in\Q$ rationnel et $b\in\N$ le plus petit possible.
	\begin{multicols}{2}
	\begin{enumerate}
		\item $\dfrac3x = \dfrac12.$
		\item $ \dfrac7x  =\dfrac12$
		\item $\dfrac8x  =\dfrac{\sqrt{3}}2$
		\item $\dfrac2x =\sqrt{3}$
		\item $\dfrac9x  =\dfrac{\sqrt{6}}5$
		\item $-\dfrac3x =\sqrt{7}$
		\item $-\dfrac7{2x}  =\dfrac{\sqrt{11}}3$
		\item $\dfrac3{5x} =-\sqrt{5}$
	\end{enumerate}
	\end{multicols}
}{exe:équations-sqrt}{
	TODO
}

\exe{}{
	Calculer la longueur \textbf{exacte} de tous les côtés des triangles rectangles suivants à l'aide des valeurs exactes de la proposition \ref{prop:données}.

	\begin{multicols}{2}
	\includegraphics[page=4]{figures/fig-geom.pdf}
	
	\includegraphics[page=5]{figures/fig-geom.pdf}
	
	\includegraphics[page=6]{figures/fig-geom.pdf}
	
	\includegraphics[page=7]{figures/fig-geom.pdf}
	\end{multicols}
}{exe:trigo}{
	TODO
}

\ex{}{
	Exemple sur expression conjuguée pour rationnalisation de dénominateur.
	
	TODO
}{ex:conjugué}

\mprop{}{
	Soit $d\in\R_+$, $d\neq1$. Alors
		\[ \dfrac1{1+\sqrt{d}} = \dfrac{1-\sqrt{d}}{1-d}. \]
}{prop:conjugué}
\pf{}{
	On utilise l'identité remarquable $(a+b)(a-b) = a^2 - b^2$ pour supprimer la racine du dénominateur.
	\begin{align*}
		\dfrac1{1+\sqrt{d}} &= \dfrac{1-\sqrt{d}}{(1+\sqrt{d})(1-\sqrt{d})}, \\
							&= \dfrac{1-\sqrt{d}}{1-d}.
	\end{align*}
}

\exe{, difficulty=2}{
	Montrer que $\sqrt2 = 1 + \dfrac1{1+\sqrt2}$.
	En déduire que $\sqrt2 = 1 + \cfrac1{2 + \cfrac1{1+\sqrt2}}$ et que
		\[\sqrt2 = 1 + \cfrac1{2 + \cfrac1{2 + \cfrac1{2 + \cfrac1{1+\sqrt2}}}}. \]
	Exprimer le rationnel $1 + \cfrac1{2 + \cfrac1{2 + \cfrac12}}$ sous la form $\dfrac{p}{q}$ avec $p$ et $q$ premiers entre eux et calculer $\left| \sqrt2 - \dfrac{p}q \right|$ au millième près.
	Faire idem avec $1 + \cfrac1{2 + \cfrac1{2 + \cfrac1{2 + \cfrac12}}}$.
}{exe:frac-continue-sqrt2}{
	En manipulant l'équation on se retrouve à vouloir montrer que $\sqrt2 - 1 = \dfrac1{\sqrt2+1}$, ce qui est équivalent à $(\sqrt2 - 1)(\sqrt2 + 1) = 1$, vrai par identité remarquable (ou par distributivité simple).
	
	Comme $\sqrt2 = 1 + \dfrac1{1+\sqrt2}$, on peut remplacer le $\sqrt2$ dans l'expression de droite par toute l'expression de droite.
	On obtient donc $\sqrt2 = 1 + \cfrac1{1 + 1 + \cfrac1{1+\sqrt2}} = 1 + \cfrac1{2 + \cfrac1{1+\sqrt2}}$ comme voulu.
	On obtient l'expression d'après en répétant le même raisonnement.
	
	Par simplifications simples, on trouve $\dfrac{p}{q} = \dfrac{17}{12}$.
	La calculatrice nous donne alors $\left| \sqrt2 - \dfrac{17}{12} \right| \approx 0,002$.
	
	Finalement, remarquons que 
	\begin{align*}
		1 + \cfrac1{2 + \cfrac1{2 + \cfrac1{2 + \cfrac12}}} &= 1 + \cfrac1{1+ 1 + \cfrac1{2 + \cfrac1{2 + \cfrac12}}}, \\
			&= 1 + \cfrac1{1+ \dfrac{17}{12}} = \dfrac{41}{29}.
	\end{align*}
	Le rationnel $\dfrac{41}{29}$ vérifie $\left| \sqrt2 - \dfrac{41}{29} \right| \approx 0,0004$, soit 5 fois plus proche de $\sqrt2$ que $\dfrac{17}{12}$ l'est.
}



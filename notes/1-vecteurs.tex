%!TEX encoding = UTF8
%!TEX root = 0-notes.tex

\chapter{Vecteurs}
\label{chap:vecteurs}


\section{Introduction}

\dfn{translation du plan}{
	Une \emphindex{translation} $T$ du plan est le déplacement de tous les points du plan selon la même direction, la même distance, et dans le même sens.
	
	Pour un point $A$ quelconque, on définit l'addition
		\[ A + T \]
	comme le \emphindex{translaté} de $A$ par $T$.
}{}

\nt{
	Une translation $T$ déplace tous les points du plan selon la même direction, la même distance, et dans le même sens.
	On peut donc décomposer $T$ en une translation \og gauche/droite \fg~ selon l'axe des abscisses, et une translation \og haut/bas \fg selon les ordonnées.
	
	Chaque point $A(x_A ; y_A)$ voit ses coordonées modifiées de la même manière :
	celui-ci se déplace systématiquement d'une certaine quantité $x_T$ en abscisse, et $y_T$ en ordonnée.
	Ainsi, une fois déplacé, le point $A$ admet pour nouvelles coordonnées
		\[ A + T = (x_A + x_T ; y_A + y_T). \]
	Remarquons que c'est exactement comme cela qu'on a défini l'addition de deux points, définition \ref{def:manip-points}.
	On pose donc $T = (x_T ; y_T)$ en réutilisant la notation $(x ; y)$ des points du plan.
	
	Il n'y a en fait pas de différence mathématique entre une translation et un point : à chaque translation on peut associer un point, et vice versa (voir corollaire \ref{cor:OAisA}).
}{}

\begin{figure}
	\centering
	\includegraphics[page=4]{figures/fig-vecteurs.pdf}
	\caption[argument for footnotemark]{Le canard Hilbert\footnotemark translaté de $x_T$ unités horizontalement puis $y_T$ unités verticalement.}
\end{figure}

\dfn{vecteur $\vec{AB}$}{
	Soient $A, B$ deux points du plan.
	On nomme 
		\[ T = \vec{AB} \]
	la translation $T$ du plan qui envoie le point $A$ sur le point $B$.
	On a alors la relation
		\[ A + \vec{AB} = B. \]
	
	En décomposant $T$ selon les abscisses et les ordonnées, on écrit ses coordonnées sous la forme $T = \begin{pmatrix} x_T \\ y_T \end{pmatrix}$.
}{}

\ex{}{
	Soient $A(2,5; 3), B(1,5; -4)$ deux points.
	On trace les points dans un repère, ainsi qu'une flèche envoyant $A$ sur $B$, qui correspond au vecteur $\vec{AB}$.
	
	On décompose la translation selon les deux axes : translater de $-1$ en abscisse, puis translater de $-7$ en ordonnée.
	On en déduit que $\vec{AB} = \pvec{-1}{-7}$.
	
	\begin{center}
	\includegraphics[page=1]{figures/fig-vecteurs.pdf}
	\end{center}
}{ex:vec-1}

\footnotetext{David Hilbert (1862-1943), mathématicien allemand. \emph{ « Wir müssen wissen — wir werden wissen. »}}

\mprop{calcul de $\vec{AB}$}{
	Par définition, on a 
		\[ A + \vec{AB} = B, \]
	qui permet de calculer la translation $\vec{AB}$ à l'aide de la relation
		\[ \vec{AB} = B - A = \pvec{x_B - x_A}{y_B-y_A}. \]
}{prop:calcul-AB}

\cor{}{
	Pour connaître les coordonnées d'une translation, il suffit de l'appliquer à l'origine $O$ et de lire les coordonnées du translaté.
	
	En d'autres termes,
		\[ \vec{OA} = A. \]
}{cor:OAisA}

\dfn{opérations sur les vecteurs}{
	Les vecteurs héritent des opérations légales sur les points (définition \ref{def:manip-points}).
	Soient $u = \pvec{x}{y}, v = \pvec{x'}{y'}$ deux vecteurs et $\kappa\in\R$ un réel. 
	Alors
	\begin{multicols}{2}
	\begin{enumerate}
		\item $u+v = \pvec{x+x'}{y+y'}$ ; et
		\item $\kappa \cdot u = \pvec{\kappa \cdot x}{\kappa\cdot y}$.
	\end{enumerate}
	\end{multicols}
}{}

\section{Composition de translations : somme de vecteurs}

\thm{relation de Chasles}{
	Soient $A, B, C$ trois points du plans. 
	Alors
		\[ \vec{AB} + \vec{BC} = \vec{AC}. \]
		\begin{center}
		\og L'addition de la translation qui envoie $A$ sur $B$ à celle qui envoie $B$ sur $C$ est égale à la translation qui envoie $A$ sur $C$ \fg.
		\end{center}
}{thm:chasles}

%\pf{Démonstration du théorème \ref{thm:chasles}}{
\pf{}{
	Par calcul direct à l'aide de la proposition \ref{prop:calcul-AB},
	\begin{align*}
		\vec{AB} + \vec{BC} &= \left(B-A\right) + \left(C - B\right), \\
								&= B - A + C - B, \\
								&= C - A, \\
								&= \vec{AC}.
	\end{align*}
}{}

\ex{Somme de vecteurs géométriquement}{
	Soient ${\color{RED_E} u} = \pvec{2}{-1}$ et ${\color{RED_E} v}= \pvec{-3}{-2}$ deux vecteurs.
	
	Pour construire géométrique la somme ${\color{RED_E} u}+{\color{BLUE_E} v} =\pvec{2-3}{-1-2} = \pvec{-1}{-3}$, on colle bout à bout les vecteurs ${\color{RED_E} u}$ et ${\color{BLUE_E} v}$ comme ci-dessous.
	
	On démarre à l'origine $O$ pour pouvoir lire les coordonnées du vecteur, d'après le corollaire \ref{cor:OAisA}.
	
	\begin{center}
	\includegraphics[page=2]{figures/fig-vecteurs.pdf}
	\end{center}
}{}

\section{Norme vectorielle}

On souhaite désormais parler de « longueur » de translation.
Pour cela, considérons $\vec{AB}$ le vecteur translatant le point $A$ sur le point $B$.
Un définition naturelle de la « longueur » de $\vec{AB}$ est la distance entre les points $A$ et $B$, soit $\norm{A - B}$, comme décrit dans la définition \ref{dfn:norme-carré}.
En mathématiques, on parlera alors de \emphindex{norme vectorielle}.

\dfn{norme d'un vecteur}{
	Soit $u = \pvec{x}{y}$ un vecteur quelconque.
	On définit sa \emphindex{norme} $\norm{u}$ par
		\[ \norm{u} = \sqrt{x^2 + y^2}. \]
		
	Il suit, par construction, que $\norm{\vec{AB}} = AB$.
}{dfn:norme}

\ex{}{
	Soient $A(2,5; 3), B(1,5; -4)$ les deux points de l'exemple \ref{ex:vec-1}.
	On a $\vec{AB} = B - A = \pvec{-1}{-7}$, et donc
		\[ \norm{\vec{AB}} = \norm{\pvec{-1}{-7}} = \sqrt{\left(-1\right)^2 + \left(-7\right)^2} = \sqrt{50} = \sqrt{25 \times 2} = 5 \sqrt{2}. \]
		
	En comparant avec la longueur $AB = BA = \sqrt{(2,5 - 1,5)^2 + (3 - (-4))^2} = 5 \sqrt{2}$, on obtient bien le même résultat.
	Notons qu'il y a moins de chance de faire une erreur avec la norme vectorielle car on calcule la longueur en deux étapes, avec d'abord les différences de coordonnées.
	
%	\begin{center}
%	\includegraphics[page=3]{figures/fig-vecteurs.pdf}
%	\end{center}
}{}

\exe{1}{
	Calculer la norme des vecteurs suivants. L'exprimer sous la forme $a \sqrt{b}$, où $a, b \in\N$ sont des entiers naturels et $b$ est le plus petit possible.
	\begin{multicols}{2}
	\begin{enumerate}	
		\item $u = \pvec34$
		\item $v = -\pvec34$
		\item $w = \pvec{-1}2$
		\item $u' = \pvec5{-7}$
		\item $v' = \pvec20$
		\item $w' = \pvec0{-5}$.
	\end{enumerate}
	\end{multicols}
}{exe:norme-vectorielle}{
	TODO
}

\thm{homogénéité\index{homogénéité} de la norme}{
	Soit $u$ un vecteur et $\kappa\in\R$ un nombre réel.
	Alors
		\[ \norm{\kappa\cdot u} = |\kappa| \cdot \norm{u}. \]
}{}

\pf{}{
	Soit $u = \pvec{x}y$ et $\kappa\in\R$.
	Alors
	\begin{align*}
		\norm{\kappa\cdot u}^2 &= \norm{\pvec{\kappa x}{\kappa y}}^2, \\
								&= (\kappa x)^2 + (\kappa y)^2, \\
								&= \kappa^2 (x^2 + y^2) = \kappa^2 \norm{\pvec{x}y}.
	\end{align*}
	La racine carré conclut en se remémorant que $\sqrt{\kappa^2} = |\kappa|$.
}


\exe{}{
	Calculer la norme des vecteurs suivants. L'exprimer sous la forme $a \sqrt{b}$, où $a, b \in\N$ sont des entiers naturels et $b$ est le plus petit possible.
	\begin{align*}
		u = \pvec{-2}{4}, && v = 2u, && w = -\dfrac32u, && z = -\dfrac12u,
	\end{align*}
}{exe:norme-homogènre}{
	TODO
}

\exe{}{
	Pour quels réels $x\in\R$ a-t-on $\norm{x \cdot \pvec12} = 5$ ?
}{exe:homogénéité}
{
	TODO
}


\section{Colinéarité, alignement, et déterminant}

\dfn{vecteurs colinéaires}{
	Soient $u$ et $v$ deux vecteurs.
	On dit de $u$ et $v$ qu'ils sont \emphindex{colinéaires} dès qu'il existe un réel $\kappa\in\R$ tel que
		\[ u = \kappa \cdot v. \]
	Deux vecteurs colinéaires admettent la même direction, mais pas forcément le même sens ni la même norme.
}{}

\nt{
	Tous les vecteurs sont colinéaires au vecteur nul (en prenant $\kappa = 0$ ci-dessus).
}{}


\lem{}{
	Soit $(d)$ une droite non verticale et $A, B \in (d)$ deux points distincts de celle-ci.
	
	Alors $\vec{AB}$ est colinéaire à un vecteur de la forme $\pvec{1}{a}$ où $a$ est le coefficient directeur de la fonction affine $f$ telle que $\C_f = (d)$.
}{lem:vecteur-dir-coeff-dir}

%\pf{Preuve du lemme \ref{lem:vecteur-dir-coeff-dir}}{
\pf{}{
	En écrivant $A(x_A;y_B)$ et $B(x_B;y_B)$ les coordonnées des points $A$ et $B$, on a $x_A \neq x_B$ car les points sont distincts et la droite est non verticale.
	
	Ainsi, $\vec{AB} = B-A$ est égal à
		\[\vec{AB}= \pvec{x_B - x_A}{y_B-y_A} = \left(x_B - x_A\right) \pvec{1}{\frac{y_B-y_A}{x_B-x_A}}. \]
	On reconnaît bien ici le coefficient directeur de la fonction affine associée à $(d)$.
}{}

\thm{}{
	Soit $A, B, C, D$ quatre points quelconques.
	
	Les vecteurs $\vec{AB}$ et $\vec{CD}$ sont colinéaires si et seulement si les droites $(AB)$ et $(CD)$ sont parallèles.
}{thm:alignement-parallélisme}

\exe{,difficulty=2}{
	Démontrer le théorème \ref{thm:alignement-parallélisme}.
}{exe:alignement-parallélisme}{
	TODO
}

\cor{}{
	Trois points $A, B, C$ quelconques sont alignés si et seulement si les vecteurs $\vec{AB}$ et $\vec{AC}$ sont colinéaires.
}{prop:alignement-colinéarité}

\dfn{vecteur directeur}{
	Soit $(d)$ une droite quelconque.
	On appelle $v$ \emphindex{vecteur directeur} de $(d)$ si $v$ et $(d)$ admettent la même direction.
	
	Si $(d)$ est non verticale, $v$ est colinéaire à $\pvec{1}{a}$, ce qui donne immediatement le coefficient directeur de la fonction affine associée.
}{}

\nt{
	Soit $A$ un point, $v$ un vecteur, et $(d)$ la droite passant par $A$ et dirigée par $v$.
	
	Alors pour $B$ un point quelconque de $(d)$, le vecteur $\vec{AB}$ est nécessairement colinéaire à $v$.
	Par conséquent, il existe un nombre réel $k\in\R$ tel que $\vec{AB} = k \cdot v$.
	
	Or comme $\vec{AB} = B-A$, on a nécessairement
		\[ B = A + k \cdot v. \]
	Un choix arbitraire de $k$ donnera donc naissance à un nouveau point de la droite.
	Par exemple, les points $A+v, A-v, A+2v, A-\frac5{13}v, A - \sqrt{5} \cdot v$, etc... sont tous des points de la droite $(d)$.
	
	On a en fait
		\[ (d) = \left\{ A + k \cdot v \text{ où $k$ parcourt $\R$} \right\}. \]
}{}

\dfn{déterminant de deux vecteurs}{
	Soient $u = \pvec{a}{b}$ et $v=\pvec{c}{d}$ deux vecteurs.
	On définit le \emphindex{déterminant} $\det(u, v)$ par le nombre réel
		\begin{align*}
			\det(u,v) = \begin{vmatrix} a & c \\ b & d \end{vmatrix} = ad - bc.
		\end{align*}
}{}

\thm{}{
	Soient $u, v$ deux vecteurs.
	Alors $\det(u, v) = 0$ si et seulement si les vecteur $u$ et $v$ sont colinéaires.
	
	Le déterminant détermine donc si deux vecteurs sont colinéaires ou non.
}{}

%!TEX encoding = UTF8
%!TEX root = 0-notes.tex

\chapter{Fonctions carré, racine carrée, \\ et valeur absolue}
\label{chap:fonction-carré}


\section{Fonction carré}

\subsection{Définition et propriétés}

\notations{
	Pour n'importe quel nombre $x$ on dénote $x^2 = x \cdot x$ le produit de $x$ par lui-même.
	On lit « $x$ au carré » ou « $x$ carré ».
}

\nt{
	La notation $xx$ prévalait jusqu'au 18è siècle, comme on peut le lire dans les travaux d'Euler\footnotemark. 
	Voir par exemple \emph{Constructio linearum isochronarum in medio quocunque resistente}, 1725 (\href{https://scholarlycommons.pacific.edu/euler-works/1/}{E1}).
	En 1733, Euler utilise déjà la notation actuelle, comme on peut le lire dans \emph{Constructio aequationis differentialis $ax^n dx = dy + y^2dx$} (\href{https://scholarlycommons.pacific.edu/euler-works/31/}{E31}).
}

\footnotetext{Leonhard Euler (1707-1783), mathématicien et physicien suisse.}

\dfn{fonction carré}{
	La \emphindex{fonction carré} est la fonction qui prend un nombre $x$ et renvoie son carré, $x^2 = x \cdot x$ :
		\[ f(x) = x^2 \qquad \pourtout x\in\R. \]
}{dfn:fonction-carré}

\exe{1}{
	Donner $\bigset{k^2 \text{ où } k = 0, 1, 2, \dots, 12}$.
}{exe:carrés-parfaits}{
	Les carrés de 0 à 12 sont
		\[ \bigset{0, 1, 4, 9, 16, 25, 36, 49, 64, 81, 100, 121, 144}. \]
}

\exe{}{
	Tracer la courbe représentation de la fonction carré sur le domaine $\D = [-4 ; 4]$.
}{exe:courbe-carré}{
	Les images $f(-4) = f(4) = 16, f(-3) = f(3) = 9, f(-2) = f(2) = 4, f(-1)=f(1) = 1$, et $f(0) =0$ permettent de tracer la courbe bleue suivante. La courbe verte est la version avec 101 points choisis uniformément.
	\begin{center}
	\includegraphics[page=1]{figures/fig-exe.pdf}
	\includegraphics[page=2]{figures/fig-exe.pdf}
	\end{center}
}

\exe{}{
	À l'aide du tracé de l'exercice \ref{exe:courbe-carré}, décrire l'ensemble $\bigset{ x\in\R \tq x^2 = 4}$.
}{exe:carré-eg}{
	Un point de la courbe $y=x^2$ prend la forme $(x ; x^2)$. 
	On cherche donc les points de la courbe d'ordonnée $y=4$ et on lit leur abscisse $x$.
	Le graphe ci-dessous donne $\bigset{ x\in\R \tq x^2 = 4} = \bigset{ -2 ; 2}$.
	\begin{center}
	\includegraphics[page=3]{figures/fig-exe.pdf}
	\end{center}
}

\exe{}{
	À l'aide du tracé de l'exercice \ref{exe:courbe-carré}, décrire l'ensemble $\bigset{ x\in\R \tq x^2 \leq 9}$.
}{exe:carré-ineg}{
	Un point de la courbe $y=x^2$ prend la forme $(x ; x^2)$. 
	On cherche donc les points de la courbe d'ordonnée $y\leq4$ et on lit leur abscisse $x$.
	Le graphe ci-dessous donne $\bigset{ x\in\R \tq x^2 = 4} = [-3 ; 3]$.
	\begin{center}
	\includegraphics[page=4]{figures/fig-exe.pdf}
	\end{center}
}

\exe{, difficulty=1}{
	Montrer que si $0 < x < 1$, alors $x^2 < x$, et que si $x>1$, alors $x^2 > x$.
}{exe:sq-id}{
	On peut multiplier chaque inégalité par $x$ sans changer l'ordre dès que celui-ci est strictement positif.
	Ainsi $x < 1 \implies x^2 < x$, et $x>1 \implies x^2 > x$ pour n'importe quel $x>0$.
}

\thm{propriétés du carré}{
	Soient $a, b \in\R$ deux réels, $b$ non nul lorsque dénominateur.
	Alors
		\begin{align*}
			(a \cdot b)^2 = a^2 \cdot b^2, && \et && \left(\dfrac{a}{b}\right)^2 = \dfrac{a^2}{b^2}.
		\end{align*}
}{thm:prop-carré}


\thm{signe du produit}{
	Le signe d'un produit suit les règles suivantes, où la positivité et la négativité sont strictes :
	\begin{itemize}
		% j'aime pas ces signes plus et moins.
		% je crois que je veux pas que le signe touche le cercle
		\item $\oplus \cdot \oplus = \oplus$
		\item $\ominus \cdot \ominus = \oplus$
		\item $\oplus \cdot\ominus = \ominus$
	\end{itemize}
}{thm:signe-produit}

\thm{positivité du carré}{
	Pour n'importe quel $x\in\R$, $x^2 \geq 0$ avec égalité uniquement en $x=0$.
}{thm:carré-positif}

\pf{}{
	On a d'abord $0^2 = 0$.
	Ensuite, si $x\neq0$ est non nul on distingue les deux cas suivants.
	\begin{itemize}
		\item Si $x>0$, alors $x^2 = x\cdot x >0$ comme produit de deux positifs.
		\item Si $x<0$, alors $x^2 = x\cdot x > 0$ comme produit de deux négatifs.
	\end{itemize}
}

\mprop{symétrie du carré}{
	La fonction $f(x) = x^2$ vérifie $f(-x)=f(x)$ pour tout $x\in\R$.
	Graphiquement, la courbe représentative $\C_f$ est symétrie par rapport à l'axe des ordonnées.
}{prop:sym-carré}

\pf{}{
	$f(-x) = (-x)^2 = \bigl( (-1)x \bigr)^2 = (-1)^2 x^2 = x^2 = f(x)$ pour tout $x\in\R$.
}

\dfn{fonction paire}{
	Une fonction quelconque $f$ définie sur $\R$ est dite \emphindex{fonction paire} dès que $f(x) = f(-x)$ pour tout $x\in\R$.
	Sa courbe représentative $\C_f$ est symétrique par rapport à l'axe des ordonnées.
}{dfn:fonction-paire}

\exe{}{
	En supposant que $f$ et $g$ soient paires, continuer leur graphe sur tout le domaine.
		TODO
}{exe:draw-paire}{
	TODO
}

\exe{}{
	Montrer que les fonctions suivantes sont paires.
	\begin{multicols}{2}
	\begin{enumerate}
		\item $f(x) = 2x^2$
		\item $g(x) = 4x^2 + 7$
		\item $h(x) = \dfrac{1}{4x^2 + 7}$
		\item $k(x) = x^4$
	\end{enumerate}
	\end{multicols}
}{exe:paires}{
	Pour tout $x\in\R$ on a bien 
	\begin{multicols}{2}
	\begin{enumerate}
		\item $f(-x) = 2(-x)^2 = 2x^2 = f(x)$.
		\item $g(-x) = 4(-x)^2 + 7 = 4x^2 + 7 = g(x)$
		\item $h(-x) = \dfrac{1}{4(-x)^2 + 7} = \dfrac{1}{4x^2 + 7} = h(x)$
		\item $k(-x) = (-x)^4 = \bigl( (-x)^2 \bigr)^2 = \bigl( x^2 \bigr)^2 = x^4 = k(x)$
	\end{enumerate}
	\end{multicols}
}

\exe{, difficulty=1}{
	Montrer que la fonction $f(x) = x^2 - x^3$ n'est pas paire.
}{exe:non-paire}{
	Supposons que $f$ soit paire.
	Par définition, $f(-x) = f(x)$ pour tout $x\in\R$.
	
	Comme $(-x)^3 = (-x)(-x)^2 = -x^3$, on a $f(-x) = x^2 + x^3$.
	Par conséquent, $f(x) = f(-x) \iff x^2 - x^3 = x^2 + x^3 \iff x^3 = 0$.
	Or ceci n'est pas vrai pour tout $x\in\R$ : prendre n'importe quel $x$ non nul fournit un contre-exemple.
	Ainsi pour $x=1$, $f(1) = 0$ et $f(-1) = 2$ et $f$ n'est en effet pas paire.
}

\exe{, difficulty=2}{
	Montrer que si $g$ est une fonction de $x^2$ (c'est-à-dire $g(x) = h(x^2)$ pour une certaine fonction $h$) alors $g$ est paire.
}{exe:gpaire}{
	$g(-x) = h\bigl( (-x)^2 \bigr) = h(x^2) = g(x)$ pour tout $x\in\R$, donc $g$ est paire.
}

\exe{, difficulty=2}{
	Montrer que si $k$ est une fonction quelconque, alors la fonction $l(x) = k(x) + k(-x)$ est paire.
}{exe:symétrisation}{
	$l(-x) = k(-x) + k\bigl(-(-x)\bigr) = k(-x) + k(x) = k(x) + k(-x) = l(x)$ pour tout $x\in\R$, donc $l$ est paire.
}


\subsection{Identités remarquables}

\ax{distributivité}{
	Pour $a,b,c\in\R$ nombres réels, on a
		\[ c \cdot (a+b) = (a+b) \cdot c = a\cdot c + b \cdot c. \]
}{ax:distr}

Certaines expressions de fonctions sont donc plus agréables à étudier que d'autres.
Il sera donc utile de savoir quelques identités dites \emphindex{remarquables} qui permettront à la fois de développer une forme factorisée, et de factoriser une forme développée.

\thm{identités remarquables}{
	Pour tous les $a,b\in\R$ réels, on a les identités
		\begin{align*}
			a^2 - b^2 &= (a+b)(a-b) \\
			(a+b)^2 &= a^2 + 2ab + b^2 \\
			(a-b)^2 &= a^2 - 2ab + b^2
		\end{align*}
	Remarquons que les points médians $\cdot$ de la multiplication disparaissent et sont sous-entendus.
	En lisant $2ab$, il faut comprendre \og 2 fois $a$ fois $b$ \fg, aussi noté $2\cdot a \cdot b$.
}{thm:id-rem}

%\pf{Démonstration du théorème \ref{thm:id-rem}}{
\pf{}{
	Toutes les identités remarquables découlent de l'axiome de distributivité.
	La première se déduit en développant le produit $(a+b)\cdot (a-b)$.
		\begin{align*}
			(a+b)\cdot(a-b) &= a \cdot(a-b) + b \cdot (a-b) \\
							&= a \cdot a + a \cdot (-b) + b\cdot a + b \cdot(-b)\\
							&= a^2 - ab + ba - b^2 \\
							&= a^2 - b^2
		\end{align*}
	La deuxième s'obtient similairement.
		\begin{align*}
			(a+b)^2 &= (a+b)\cdot(a+b) \\
						&= a \cdot(a+b) + b \cdot (a+b) \\
						&= a \cdot a + a \cdot b + b\cdot a + b \cdot b \\
						&= a^2 + ab + ba + b^2 \\
						&= a^2 +2ab + b^2
		\end{align*}
	Pour obtenir la dernière, on peut remplacer $-b$ par $b$ de la façon suivante.
		\begin{align*}
			(a-b)^2 &= \left( a + (-b) \right)^2 \\
					&= a^2 + 2\cdot a\cdot (-b) + (-b)^2 \\
					&= a^2 - 2ab + b^2.
		\end{align*}
}{}

\nt{
	Il sera utile de connaître quelques carrés parfaits pour développer tranquillement les expressions factorisées.
	On énumère alors
		\begin{align*}
			0^2 &= 0 & 7^2 &= 49 \\
			1^2 &= 1 & 8^2 &= 64 \\
			2^2 &= 4  & 9^2 &= 81  \\
			3^2 &= 9  & 10^2 &= 100 \\
			4^2 &= 16  &  11^2 &= 121 \\
			5^2 &= 25  & 12^2 &= 144 \\
			6^2 &= 36 & 13^2 &= 169 
		\end{align*}
	Remarquons qu'on a toujours $(-n)^2 = n^2$, donc qu'on peut en déduire aussi les carrés d'entiers négatifs. 
	Par exemple, $(-9)^2 = 9^2 = 81$.
}

\nomen{
	On appelle \emphindex{polynôme} une expression de la forme
		\[ a_d x^d + a_{d-1}x^{d-1} + \cdots + a_1 x + a_0, \]
	où chaque $a_i\in\R$ est un coefficient fixé.
	La plus haute puissance, $d$, est appelée le \emphindex{degré} du polynôme.
}

\ex{}{
	La fonction $p(x) = x^3 - 2x^2 + 100x$ est un polynôme de degré 3.
}{}
\ex{}{
	La fonction $q(x) = 2x^2 - 1$ est un \emphindex{polynôme du second degré}.
}{}

\exe{}{
	Montrer que $x$ et $x^2$ sont non miscibles, c'est-à-dire montrer que l'identité
		\[ x^2 = x \]
	ne peut pas être vraie pour tout $x\in\R$.
}{exe:x-x2-nonmiscibles}{
	TODO
}

\exe{, difficulty=1}{
	Montrer que, pour $m, n\in\N$ non nuls, l'identité
		\[ x^m = x^n \]
	ne peut être vraie pour tout $x\in\R$ que lorsque $m=n$.
}{exe:xm-xn-nonmiscibles}{
	TODO
}

\nomen{
	Soit $p$ un polynôme de degré 2.
	La forme $p(x) = ax^2 + bx + c$ est la \emphindex{forme développée}.
	La forme $p(x) = a(x-\lambda)(x-\gamma)$ est la \emphindex{forme factorisée}.
	La forme $p(x) = a(x-\alpha)^2 + \beta$ est \emphindex{forme canonique} (hors programme).
}

\ex{}{
	Par calcul littéral, on peut vérifier les trois égalités suivantes pour le polynôme $p(x) = 2x^2 - 18x + 90$.
		\[ p(x) = 2x^2 - 18x + 90 = 2(x-9)(x-5) = 2(x-7)^2 - 8. \]
}{}

\notations{
	On écrira l'identité $2(x-9)(x-5) = 2(x-7)^2 - 8$ sans préciser qu'elle est valable pour tout $x\in\R$.
	Cette égalité n'est pas à résoudre : elle montre l'identité de deux expressions de formes différentes.
}

\exe{}{
	Développer les expressions algébriques suivantes.
		\begin{multicols}{2}
		\begin{enumerate}[label=$\bullet$]
			\item $f(x) = (1+x)^2$
			\item $g(x) = (x-3)^2$
			\item $h(x) = (3-x)^2$
			\item $F(x) = (3 + 2x)^2$
			\item $G(x) = (3x - 7)^2$
			\item $H(x) = (-7x - 2)^2$
		\end{enumerate}
		\end{multicols}
}{exe:développement}{
	TODO
}

\exe{, difficulty=2}{
	On appelle \emphindex{triplet pythagoricien} un triplet $(a ; b ; c)$, liste de $3$ nombres entiers naturels $a, b, c\in\N$ vérifiants
		\[ a^2 + b^2 = c^2. \]
	Nommons deux paramètres entiers $k, \ell\in\N$ avec $k\geq \ell$.
	Vérifier que le triplet
		\[ (k^2 - \ell^2 ; 2k\ell ; k^2 + \ell^2) \]
	est pythagoricien.
	
	Créer quelques triplets en prenant des valeurs de $k \geq \ell$ et les vérifier à la calculatrice.
	Par exemple, pour $k=4$ et $\ell=2$, on trouve le triplet $(12 ; 16 ; 20)$, qui vérifie bien
		\[ 12^2 + 16^2 = 20^2. \] 
}{exe:triplet-pythagoricien}{
	TODO
}

\mprop{}{
	Deux polynômes en sont égaux sur $\R$ tout entier si et seulement s'il admettent la même expression développée.
}{prop:eg-poly}

\exe{, difficulty=2}{
	Supposons que le polynôme du second degré $p(x) = ax^2 + bx + c$ s'écrive de la forme $p(x) = a'(x-\lambda)(x-\gamma)$.
	Montrer, par identification des coefficients en $x^2, x, 1$ permise par la proposition \ref{prop:eg-poly}, que $a'=a$, puis que $a = -\lambda - \gamma$, et enfin que $c = \lambda\cdot\gamma$.
}{exe:déterminant-2nddeg}{
	TODO
}

\exe{, difficulty=2}{
	Supposons que le polynôme du second degré $p(x) = ax^2 + bx + c$ s'écrive de la forme $p(x) = a'(x-\alpha)^2 - \beta$.
	Montrer, par identification des coefficients en $x^2, x, 1$ permise par la proposition \ref{prop:eg-poly}, que $a'=a$, puis que $\alpha = \dfrac{-b}{2a}$, et enfin que $\beta = \dfrac{b^2 - 4ac}{4a}$.
}{exe:déterminant-2nddeg}{
	TODO
}

\exe{, difficulty=2}{
	Montrer la proposition \ref{prop:eg-poly}.
}{exe:eg-poly}{
	Je me demande quelle est la façon la plus simple.
	Sans doute identifier les coefficients constants en prenant $x=0$, puis soustraire, diviser par $x$, et continuer.
	Mais il faut le lemme $xf(x) = xg(x) \forall x \iff f(x) = g(x) \forall x$ qui n'est pas vrai en 0 sans hypothèse hors programme donc ça marche po.
}


\section{Fonction racine carrée}
\label{sec:racine-carrée}

\dfn{racine carrée}{
	Soit $a \in \R, a\geq0$ un réel positif ou nul.
	La quantité $\sqrt{a}$ est l'unique nombre réel positif ou nul vérifiant
		\[ \sqrt{a}^2 = a. \]
}{}

\exe{}{
	Compléter les pointillés.
	\begin{multicols}{2}
	\begin{enumerate}
		\item $\sqrt{25} = \dots$
		\item $\sqrt{81} = \dots$
		\item $\sqrt{121} = \dots$
		\item $\sqrt{\quad\dots\quad} = 25$
		\item $\sqrt{\quad\dots\quad} = 12$
		\item $\sqrt{\quad\dots\quad} = 10^3$
	\end{enumerate}
	\end{multicols}
}{exe:sqrt1}{
	TODO
}

\exe{}{
	Calculer les racines carrées suivantes.
	\begin{multicols}{2}
	\begin{enumerate}
		\item $\sqrt{7^2}$
		\item $\sqrt{17}^2$
		\item $\sqrt{(-9)^2}$
		\item $\sqrt{10^4}$
		\item $\left(-\sqrt{4}\right)^2$
		\item $-\sqrt{15^2}$
	\end{enumerate}
	\end{multicols}
}{exe:sqrt2}{
	TODO
}

% pas hyper intéressant sans notion de croissance de sqrt je trouve
%\exe{}{
%	Donner un encadrement des nombres à l'unité.
%		\begin{multicols}{2}
%		\begin{enumerate}
%			\item $\sqrt{43}$
%			\item $\sqrt{70,8}$
%			\item $\sqrt{\dfrac{61}7}$
%			\item $\sqrt{101,204}$
%		\end{enumerate}
%		\end{multicols}
%}{}

\exe{}{
	Montrer que la notation $x = \sqrt{-1}$ n'a pas de sens pour $x\in\R$ réel.
}{exe:sqrt-undef}{
	À supposer qu'il existe un tel $x\in\R$, alors il vérifierait $x^2 = -1$.
	Ce n'est pas possible car la fonction carré est toujours positive. \Large\Lightning
}

% domaine de def sera mieux travaillé quand on aura la fonction inverse 
%\dfn{Domaine de définition}{
%	Soit $f$ une fonction sur $\R$.
%	Le plus grand domaine de $\R$ sur lequel $f$ est bien définie est le \emphindex{domaine de définition} de $f$.
%}{}
%
%\notations{
%	Le domaine de définition de $f$ est noté $\D_f$.
%}
%
%\mprop{}{
%	Le domaine de définition de la fonction racine carrée est $[0 ; \pinfty[$.
%}{prop:domaine-sqrt}

\mprop{}{
	La racine carrée $\sqrt{x}$ n'a de sens que pour $x\in[0; \pinfty[$.
}{prop:domaine-sqrt}

\notations{
	On note $\R_+ = [0 ; \pinfty[$ l'ensemble des réels positifs ou nuls.
}

\thm{propriétés de la racine carrée}{
	Soient $a, b \in\R_+$ deux réels positifs, $b$ non nul lorsque dénominateur.
	Alors
		\begin{align*}
			\sqrt{a \cdot b} = \sqrt{a} \cdot \sqrt{b}, && \et && \sqrt{\dfrac{a}{b}} = \dfrac{\sqrt{a}}{\sqrt{b}}.
		\end{align*}
}{thm:prop-sqrt}

\exe{}{
	Calculer les produits suivants.
	\begin{multicols}{2}
	\begin{enumerate}
		\item $\sqrt{169} \cdot \sqrt{81}$
		\item $\sqrt{169 \cdot 81}$
		\item $\sqrt{0,16} \cdot \sqrt{900}$
		\item $\sqrt{0,16 \cdot 900}$
	\end{enumerate}
	\end{multicols}
}{exe:sqrt3}{
	TODO
}

\exe{}{
	Écrire les nombres suivants sous la forme $a\sqrt{b}$ où $a\in\N$ et $b\in\N$ est le plus petit entier possible.

	\begin{multicols}{2}
	\begin{enumerate}
		\item $\sqrt{12}$
		\item $\sqrt{150}$
		\item $5\sqrt{96}$
		\item $2\sqrt{300}$
		\item $\dfrac{12}{\sqrt{3}}$
		\item $\dfrac{18}{\sqrt{6}}$
	\end{enumerate}
	\end{multicols}
}{exe:sqrt4}{
	TODO
}


\exe{}{
	Écrire les nombres suivants sous la forme $\sqrt{a}$ où $a\in\N$.
	\begin{multicols}{2}
	\begin{enumerate}
		\item $3\sqrt{2}$
		\item $50\sqrt{0,5}$
	\end{enumerate}
	\end{multicols}
}{exe:sqrt5}{
	TODO
}

\exe{}{
	Calculer les fractions suivantes.
	\begin{multicols}{2}
	\begin{enumerate}
		\item $\dfrac{\sqrt{64}}{\sqrt{4}}$
		\item $\sqrt{\dfrac{64}4}$
		\item $\dfrac{\sqrt{0,81}}{\sqrt{0,09}}$
		\item $\sqrt{\dfrac{0,81}{0,09}}$
	\end{enumerate}
	\end{multicols}
}{exe:sqrt6}{
	TODO
}

\section{Fonction valeur absolue}

\dfn{valeur absolue}{
	On nomme la fonction $f(x) = \sqrt{x^2}$ la fonction \emphindex{valeur absolue}.
	On note $|x|$ la valeur absolue de $x$, image de $x$ par $f$.
}{dfn:valeur-absolue}

\exe{}{
	Écrire les nombres suivants sans les barres de valeur absolue.
	\begin{multicols}{3}
	\begin{enumerate}[label=\roman*)]
		\item $|-7|$
		\item $|8|$
		\item $|-13-8|$
		\item $|\pi - 4|$
		\item $|-5+3| + |-7+4|$
		\item $|\sqrt{2} - \sqrt{3}|$
	\end{enumerate}
	\end{multicols}
}{exe:valeur-absolue}{
	\begin{multicols}{3}
	\begin{enumerate}[label=\roman*)]
		\item $7$
		\item $8$
		\item $21$
		\item $4 - \pi$
		\item $5$
		\item $\sqrt{3}- \sqrt{2}$
	\end{enumerate}
	\end{multicols}
}

\mprop{}{
	La valeur absolue est toujours positive ou nulle.
	
	On a en fait
		\[ |x| =  \begin{cases*} x & si $x \geq 0$, \\ -x  & si $x \leq 0$. \end{cases*}. \]
}{prop:caractérisation-valeur-absolue}

\pf{}{
	Si $x\geq0$, alors $x$ est le nombre positif tel que, mis au carré, il vaut $x^2$.
	C'est donc la racine carrée de $x^2$.

	Si $x\leq0$, le même raisonnement ne tient pas, car la racine carrée de $x^2$ est le nombre \emphindex{positif} tel que, mis au carré, il vaut $x^2$.
	Le nombre $-x$ fonctionne ici : il est positif, et $(-x)^2 = (-1)^2 x^2 = x^2$.
	Par conséquent, $\sqrt{x^2} = -x$ dans ce cas.
}

\nt{
	La fonction de la proposition \ref{prop:caractérisation-valeur-absolue} est définie algébriquement \emphindex{par morceaux}.
	Ce type de définition a été utilisé notamment par Dirichlet\footnotemark lors de l'étude générale de fonctions non lisses comme
		\[ 1_{|\Q}(x) = \begin{cases*} 1 & si $x\in\Q$, \\ 0 & sinon. \end{cases*}. \]
}

\footnotetext{Johann Peter Gustav Lejeune Dirichlet (1805 - 1859), mathématicien prussien né à Düren (allemagne de l'ouest actuelle).}

\exe{, difficulty=2}{
	Donner la courbe représentative de la fonction $1_{|\Q}$ de la remarque ci-dessus sur le domaine $\D=[0;1]$.
}{exe:ind-Q}{
	TODO
}

\exe{}{
	Montrer que, pour tout $x\in\R$, $\bigl| |x| \bigr| = |x|$.
}{exe:abs-abs}{
	todo
}

\exe{}{
	Montrer que, pour tout $x\in\R$, $|x| = |-x|$.
}{exe:moinsabs}{
	Comme $x^2 = (-x)^2$, appliquer la racine carrée conclut.
	
	On aurait aussi pû étudier les cas $x\geq0$ et $x\leq0$ séparément :
	\begin{itemize}
		\item Si $x\geq0$, alors $-x\leq0$ et $|-x| = -(-x)= x = |x|$.
		\item Si $x\leq0$, alors $-x\geq0$ et $|x| = -x= |-x|$.
	\end{itemize}
}

\exe{}{
	Montrer que, pour tout $x\in\R$, $|2x| = 2|x|$.
}{exe:2abs}{
	Comme $(2|x|)^2 = 4x^2 = (2x)^2$, appliquer la racine carrée conclut.
	
	On aurait aussi pû étudier les cas $x\geq0$ et $x\leq0$ séparément :
	\begin{itemize}
		\item Si $x\geq0$, $2x\geq0$ et $|2x| = 2x = 2|x|$.
		\item Si $x\leq0$, alors $2x\leq0$ et $|2x| = -2x= 2(-x) = 2|x|$.
	\end{itemize}
}

\exe{}{
	Montrer que, pour tout $x\in\R$, $|x|^2 = x^2$.
}{exe:abscarre}{
	$|x| = x$ ou $-x$. Dans le premier cas, il n'y a rien à montrer, et dans le second, $(-x)^2 = (-1)^2 x^2 = x^2$.
}

\exe{, difficulty=2}{
	Trouver tous les réels $x\in\R$ vérifiant $|x-2| = |2 + 3x|$.
}{exe:eq-vabs}{
	On souhaite enlever les barres de valeur absolue afin de résoudre une équation linéaire simple. Comme $|E|$ vaut soit $E$ soit $-E$, on obtient quatre équations possibles.
	Deux paires d'entre elles sont en fait équivalentes par multiplication par $-1$.
	Les deux alternatives sont $x-2 = 2+3x \iff -4 = 2x \iff x = -2$ et $2-x = 2+3x \iff 0 = 4x \iff x=0$.
	On vérifie bien que $x=0$ et $x=-2$ sont solutions : $|-2| = |2|$ et $|-4| = |-4|$.
}

\thm{}{
	Pour tout $x, y\in\R$, on a
		\[ |x\cdot y| = |x|\cdot|y|. \]
}{thm:produit-valeur-absolue}

\pf{}{
	D'après l'exercice \ref{exe:abscarre}, on a $\bigl(|x|\cdot|y|\bigr)^2 = |x|^2 |y|^2 = x^2 y^2 = (xy)^2$.
	En appliquant la racine carrée, on trouve
		\[ |x| \cdot |y| = \big| |x|\cdot|y| \big| = |xy|. \]
}

\thm{}{
	Soit $x, y \in \R$. Alors la distance entre $x$ et $y$ sur la droite réelle est donnée par 
		\[ \text{distance}(x,y) = |x-y| = |y-x|. \]
}{thm:valeurs-absolue-distance}

% j'aime pas trop cette démo
\pf{}{
	Remarquons d'abord que $|x-y| = |y-x|$ d'après l'exercice \ref{exe:moinsabs} et car $x-y$ est l'opposé de $y-x$.	

	Si $x \leq y$, alors la distance entre $x$ et $y$ sur la droite réelle est donnée par $y-x$ qui vaut $|y-x|$ par positivité.
	Sinon, $x \geq y$, alors la distance entre $x$ et $y$ sur la droite réelle est donnée par $x-y$ qui vaut $|x-y|$, à nouveau par positivité.

	Dans tous les cas, la distance vaut donc bien $|x-y| = |y-x|$.
}

\exe{}{
	Donner l'ensemble des $x\in\R$ vérifiant $|x| = 4$.
}{exe:vabs-4}
{
	Comme $|x|$ vaut soit $x$, soit $-x$, on a $x=4$ ou $-x=4 \iff x = -4$.
	Ainsi $\bigset{ x \in \R \tq |x| = 4 } = \bigset{-4 ; 4}$.
}

\exe{}{
	Donner l'ensemble des $x\in\R$ vérifiant $x^2 = 16$.
}{exe:carre-16}
{
	En appliquant la racine carrée, on a $|x| = \sqrt{x^2} = \sqrt{16} = 4$.
	L'exercice \ref{exe:vabs-4} permet de conclure que $\bigset{ x \in \R \tq x^2 = 16 } = \bigset{-4 ; 4}$.
}

\exe{}{
	Donner l'ensemble des $x\in\R$ vérifiant $|x| \leq 9$.
}{exe:vabs-9}
{
	Comme $|x|$ donne la distance de $x$ à 0, on a 
		\[ \bigset{ x \in \R \tq |x|\leq9 } = [-9; 9]. \]
}

\exe{}{
	Donner l'ensemble des $x\in\R$ vérifiant $x^2 \leq 81$.
}{exe:carre-81}
{
	En appliquant la racine carrée, et comme la fonction racine est croissante, on a $x^2 \leq 81 \iff |x| \leq 9$.
	L'exercice \ref{exe:vabs-9} permet de conclure que $\bigset{ x \in \R \tq x^2 \leq 81 } = [-9 ; 9]$.
}

\cor{Manipulation d'égalités, d'inégalités avec valeurs absolues}{
	Soit $E\in\R$ une expression quelconque et $a\geq0$ un réel positif ou nul.
	Alors 
		\begin{align*}
			| E | = a && \iff && E = a \text{ ou } E = -a, \\
			|E| \leq a && \iff && E \in [-a ; a], \\
			%|E| \geq a && \iff && E \in ]\minfty; -a] \text{ ou } E\in [a ; \pinfty[.
		\end{align*}
}{cor:manipulations-valeurs-absolues}

\exe{, difficulty=1}{
	Démontrer le corollaire \ref{cor:manipulations-valeurs-absolues}.
}{exe:manipulations-valeurs-absolues}{
	TODO
}

\exe{, difficulty=2}{
	Donner l'ensemble des $x\in\R$ vérifiant $|x| \geq 21$.
}{exe:vabs-9}
{
	Comme $|x|$ donne la distance de $x$ à 0, on a 
		\[ \bigset{ x \in \R \tq |x|\geq21 } = ]\minfty; -21] \cup [21 ; \pinfty[. \]
}


\ex{}{
	Considérons les fonctions $f, g$ données par $f(x) =x^2 -8\cdot x + 16$ et $g(x)=(x-4)^2$ pour tout $x\in\R$.
	
	On se demande quels sont les antécédents d'une certaine valeur, mettons $64$ par exemple, par $f$.
	En nommant un tel antécédent $x$, on a l'équation
		\begin{align*}
			f(x) &= 64, \\
			x^2 - 8\cdot x + 16 &= 64,
		\end{align*}
	qui n'est pas simple du tout à résoudre car non linéaire.
	Cependant, si on arrive à démontrer que
		\begin{align*}
			(x-4)^2 = x^2 - 8\cdot x + 16 && \text{c'est-à-dire} && g(x) = f(x),
		\end{align*}
	et ceci pour tout $x\in\R$ nombre réel, on pourra utiliser la forme plus agréable de $g$ afin d'obtenir les antécédents recherchés.
	
	Pour développer $(x-4)^2 = (x-4)\cdot(x-4)$, on utilise la distributivité qui est rappelée ci-après par l'axiome \ref{ax:distr}.
	On voit l'un des $(x-4)$ comme un tout, et l'autre comme un somme de deux quantités, $x$ et $-4$.
	En découle que, par distributivité utilisée quatre fois,
		\begin{align*}
			(x-4)^2 &= (x-4) \cdot (x-4) \\
					&= (x-4) \cdot x + (x-4) \cdot (-4) \\
					&= x\cdot x + (-4) \cdot x + x \cdot(-4) + (-4) \cdot (-4) \\
					&= x^2 - 8 \cdot x + 16
		\end{align*}
	comme recherché.
	
	En revenant à notre situation, on se permet donc de résoudre
		\begin{align*}
			g(x) &= 64 \\
			(x-4)^2 &= 64 \\
			|x-4| &= 8,
		\end{align*}
	en appliquant la racine carrée. D'après le corollaire \ref{cor:manipulations-valeurs-absolues},	
		\begin{align*}
			|x-4| &= 8 \\
			x-4=8 \qquad&\text{ou}\qquad x-4 = -8 \\
			x = 12  \qquad&\text{ou}\qquad  x=-4
		\end{align*}
	
	La fonction $f(x) = x^2 -8\cdot x + 16$ admet donc exactement deux antécédents de $64$ : $-4$ et $12$.
	Il conviendra de vérifier ceci en calculant $f(-4)$ et $f(12)$.
}{ex:id-rem1}

\exe{, difficulty=1}{
	Factoriser les membres de gauche pour obtenir le carré d'une expression et trouver toutes les solutions $x\in\R$ des équations suivantes.
		\begin{multicols}{2}
		\begin{enumerate}[label=\alph*)]
			\item $x^2 + 2x + 1 = 16$
			\item $4x^2 - 4x + 1 = 9$
			\item $9 - 18x + 9x^2 = 0$
			\item $3x^2 - 6x + 3 = 0$
			\item $x^2 +6x + 9 = -1$
			\item $-4x^2 + 8x - 1 = 3$
		\end{enumerate}
		\end{multicols}
}{exe:factorisation}{
	TODO
}


\cor{}{
	Pour $r, c \in\R$ et $r\geq0$, on a l'égalité d'ensembles
		\[ \bigset{ x \in \R \tq |x - c| \leq r} = [c - r ; c+r]. \]
}{cor:boule-rc}

\exe{}{
	Démontrer le corollaire \ref{cor:boule-rc}.
}{exe:boule-rc}{
	L'ensemble $\bigset{ x \in \R \tq |x - c| \leq r}$ contient tous les nombres réels $x$ à distance inférieure à $r$ de $c$.
	C'est donc un segment centré en $c$ et de rayon $r$, l'équivalent du disque en une dimension.
	
	Purement algébriquement, on a $|E| \leq r \iff E\in[-r ; r] \iff -r \leq E \leq r$ d'après le corollaire \ref{cor:manipulations-valeurs-absolues}.
	Avec $E = x-c$, on a $-r \leq x-c \leq r \iff c-r \leq x \leq c+r \iff x \in [c-r ; c+r]$.
}

\exe{, difficulty=2}{
	Montrer que pour tout $x,y\in\R$, on a 
		\begin{align*}
			\min\{x,y\} = \dfrac{x+y}2 - \dfrac{|x-y|}2, && \text{ et } && \max\{x,y\} = \dfrac{x+y}2 + \dfrac{|x-y|}2.
		\end{align*}
}{exe:valeur-absolue-minmax}{
	Quels que soient $x, y\in\R$, le segment borné par $x$ et $y$ est de milieu $\dfrac{x+y}2$ et de longueur $|x-y|$.
	Ses extrémités s'expriment donc comme $c - r$ et $c+r$ où $c$ est le centre du segment et $r$ son rayon, c'est-à-dire la moitié de sa longueur.

	Algébriquement, si $x<y$, alors $\dfrac{x+y}2 - \dfrac{|x-y|}2 = \dfrac{x+y}2 - \dfrac{y-x}2 = \dfrac{(x+y)-(y-x)}2 = \dfrac{2x}2 = x = \min\{x ,y\}$.
	On peut faire de même pour le maximum puis dans le cas $x>y$ (ou se convaincre que le cas $x>y$ n'est pas à faire par symétrie des expressions).
}


\mprop{}{
	Soient $a, b \in \R$ deux nombres réels tels que $a < b$.
	
	Alors le milieu de l'intervalle $[a,b]$ est égal à $\dfrac{a+b}2$.
}{prop:milieu-inter}

\pf%{Démonstration de la proposition \ref{prop:milieu-inter}}{
{}{
	Notons $x\in\R$ le milieu de l'intervalle.
	Celui-ci est à équidistance de $a$ et de $b$ et vérifie donc
		\[ |x-a| = |x-b|. \]
	Comme $x$ vérifie $a \leq x \leq b$, et par définition de la valeur absolue, on a
		\begin{align*}
			|x-a| &= x-a, \text{ et} \\
			|x-b| &= -(x-b) = b-x.
		\end{align*}
	On peut alors résoudre pour $x\in\R$ :
		\begin{align*}
			x-a = b-x && \iff && 2x = a+b && \iff && x = \dfrac{a+b}2
		\end{align*}
}

\nt{
	En voyant $\dfrac{a+b}2$ comme $\dfrac12 a + \dfrac12 b$, on peut comprendre le milieu comme une moyenne de deux valeurs ($a$ et $b$).
	C'est d'ailleurs comme ça qu'on calcule une moyenne de notes ayant le même coefficient --- leur poids est alors $\dfrac12$.
	Remarquons que la somme des poids vaut toujours $1$ : ils correspondent aux coefficients divisés par la somme de tous les coefficients.
	
	En changeant les poids, on peut créer d'autres points de l'intervalle. 
	Par exemple, si $a$ compte deux fois plus que $b$, les coefficients seront (par exemple) $2$ et $1$, et les poids $\dfrac23$ et $\dfrac13$.
	 La formule de la moyenne pondérée sera alors
		\[ \dfrac23a + \dfrac13b, \]
	qui est plus proche de $a$ que de $b$.
}

\exe{, difficulty=2}{
	Considérons l'intervalle $[15;20]$ qui correspond à deux notes : $15$ et $20$.
	\begin{enumerate}
		\item Si les deux notes ont le même coefficient fixé à $1$, quel est le poids de chaque note et quelle est la moyenne pondérée ? Comparer avec le milieu de l'intervalle.
		\item Posons $\lambda \in [0;1]$ le poids de la première note, et $1-\lambda$ celui de la deuxième.
			Calculer la moyenne pondérée $15\lambda + 20(1-\lambda)$ pour certaines valeurs de $\lambda$ dans son intervalle $[0;1]$. 
			Vérifier que toutes les valeurs obtenues appartiennent bien à l'intervalle $[15;20]$.
		\item Quel $\lambda \in [0;1]$ choisir pour avoir une moyenne de $18{,}5$ ?
	\end{enumerate}
}{exe:milieu-inter}{
	TODO
}

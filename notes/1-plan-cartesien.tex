%!TEX encoding = UTF8
%!TEX root = 0-notes.tex

\chapter{Plan cartésien}
\label{sec:geom-plane}\label{chap:plan-cartésien}

Sur la droite réelle, on associe chaque point avec un nombre, appelé nombre réel : chaque point correspond à un nombre, et chaque nombre correspond à un point.

Dans cette section, on augmente la notion de droite réelle en ajoutant une dimension ; la droite devient le plan.
On associera alors à chaque point du plan \emph{deux}\footnote{Il s'avère qu'il est possible en théorie d'associer chaque point du plan à un unique nombre, mais la construction dépasse le champ d'application de ce cours.} nombres réel : le premier représentant sa position \og gauche/droite \fg et le deuxième sa position \og haut/bas \fg.

Un point sera alors un couple $(x,y)$ où $x,y \in \R$ sont deux nombres réels appelés respectivement l'abscisse et l'ordonnée du point.

\section{Repère orthonormé}

\dfn{Repère}{
	
	\begin{multicols}{2}

	Un repère est determiné par trois points $(O; I,J)$ comme ci-contre.
	\begin{itemize}
		\item Le point $O$ est appelé l'\textbf{origine}.
		\item La droite $(OI)$ est appelée l'axe des \textbf{abscisses}.
		\item La droite $(OJ)$ est appelée l'axe des \textbf{ordonées}.
		\item[]
		\item[]
	\end{itemize}
	
	
	\includegraphics[page=1, scale=1.3]{figures/fig-plan.pdf}
	\end{multicols}

}{}

\dfn{Repère orthonormé}{
	Un repère $(O; I, J)$ est
	\begin{enumerate}
		\item \textbf{orthogonal} si ses axes $(OI)$ et $(OJ)$ sont perpendiculaires ;
		\item \textbf{normé} si les segments $[OI]$ et $[OJ]$ sont de même longueur (fixée à $1$) ;
		\item \textbf{orthonormé} s'il est orthogonal et normé.
	\end{enumerate}
}{def:repere-orthonorme}

\nt{
	L'ordre des points $I$ et $J$ est important car il donne l'ordre de lecture des coordonnées.
	Traditionnellement, l'axe des abscisse est horizontale, et celle des ordonnée verticale.
	
	Ainsi, le point $I$ admet pour coordonnée $(1;0)$, et le point $J$ $(0;1)$.
	On notera alors :
		\begin{align*}
			I(1;0), && \et && J(0;1).
		\end{align*}
}

\ex{}{
	Les points $A$, $B$, et $C$ admettent pour coordonnées, dans le repère $(O; I, J)$,
		\begin{align*}
			A(2;3), && B(-1;2), && C(3;-1).
		\end{align*} 

	\centering
	\includegraphics[page=2,scale=2]{figures/fig-plan.pdf}
}{}

\nt{
	On omettera parfois de préciser que le repère est orthonormé.
}

\dfn{Plan cartésien}{
	Le plan cartésien est l'ensemble des couples $(x,y)$ de réels :
		\[ \bigset{ (x, y) \text{ tq. } x, y\in\R }. \]
}{dfn:plan-cartésien}

\nomen{
	On appelle un élément $(x;y)$ du plan un \emph{point} et $x, y$ ses \emph{coordonnées}.
}

\exe{}{
	Tracer un repère et y placer les points suivants. Il est recommandé de calculer les coordonnées des points à placer avant de tracer le repère.
	\begin{multicols}{2}
	\begin{enumerate}
		\item $\point{A}{2}{3}$
		\item $\point{B}{-1}{3}$
		\item $C = A+B$
		\item $D=A-B$
		\item $E=\frac12A$
		\item $F=-3B$
	\end{enumerate}
	\end{multicols}
}{exe:points-à-placer}{
	\, \\
	\centering
	\includegraphics[page=5]{figures/fig-exe.pdf}
}

\section{Opérations sur les points et norme}\label{sec:longueur-milieu}

\dfn{Segment}{
	Soient $E, F$ deux points du plan. Le segment $[EF]$ est l'ensemble des points de la droite $(AB)$ situés entre $A$ et $B$ (tous les deux inclus).
}{deg:segment}

\nt{
	Un segment du plan est une généralisation d'un intervalle sur la droite.
	Tous les segments du plans qui seront considérés contiennent leur bornes (contrairement aux intervalles !).
}

\thm{admis}{
	Soient $A(x_A, y_B)$ et $B(x_A,y_B)$ deux points du plan.
	
	Alors le milieu du segment $[AB]$ est le point $M$ de coordonnées
		\begin{align}\label{expr:milieu}
			M\left(\dfrac{x_A+x_B}2, \dfrac{y_A + y_B}2\right).
		\end{align}
}{thm:milieu-segment}

\pf
%{Démonstration de la proposition \ref{prop:milieu-segment}}{
{justification}{
	Le point $M$ est au centre du rectangle dont les coins sont $A$ et $B$.
	Ainsi ses coordonnées sont les moyennes des coordonnées et $A$ et de $B$.
	
	\centering
	\includegraphics[page=3, scale=1.1]{figures/fig-plan.pdf}
	
}

\nt{
	La formule \eqref{expr:milieu} du milieu du segment $[AB]$ ressemble de près à celle du milieu d'un intervalle.
	Pour simplifier les expressions, il sera alors pratique d'écrire
		\[ M = \dfrac{A+B}2 = \dfrac12 A + \dfrac12 B.\]
	Pour cela, il faut pouvoir additionner les points du plans, et les multiplier par des nombres réels.
}

\dfn{Manipulations des points}{
	Soient $A(x_A, y_A)$, $B(x_B, y_B)$ deux points du plan et $\kappa \in \R$ un nombre réel quelconque.
	
	Les opérations suivantes sont légales.
		\begin{enumerate}
			\item L'addition de deux points : $A+B$, de coordonées $(x_A+x_B, y_A+y_B)$.
			\item La multiplication d'un point par un réel : $\kappa A$, de coordonnées $(\kappa x_A; \kappa y_A)$.
		\end{enumerate}
	\textbf{\warning On ne multiplie jamais les points ensemble ! Le produit $A$ par $B$ n'a pas de sens.}
}{def:manip-points}


\ex{}{
	Considérons $A(1;3)$ et $B(-3;2)$ deux points du plan.
	Alors 
		\begin{multicols}{3}
		\begin{enumerate}
			\item $A+B = (-2 ; 5)$
			\item $·2B = (-6 ; 4)$
			\item $-A = (-1 ; -3)$
			\item $B - A = (-4 ; -1)$
			\item $-2A - B = (1 ; -8)$
			\item $\dfrac{A-2B}{2} = \left(\dfrac72 ; \dfrac{-1}2\right)$
		\end{enumerate}
		\end{multicols}
}{}

\mprop{reformulation du théorème \ref{thm:milieu-segment}}{
	Le milieu $M$ du segment $[AB]$ est égal à
		\[ M = \dfrac{A+B}{2}. \]
}{}

\exe{}{
	Considérons les points
		\begin{align*}
			\point{A}{0}{1}, && \point{B}{-3}{0}, && \point{C}{1}{-2}, && \point{D}{-2}{-3}.
		\end{align*}
	Démontrer que le quadrilatère $BACD$ est un parallélogramme en comparant le milieu de ses deux diagonales.
	
	\emph{Rappel : un parallélogramme est un quadrilatère dont les diagonales se coupent en leur milieu}.
}{exe:parallélogramme}{
	Le milieu du segment $[BC]$ est donné par
		\[ M = \dfrac12 (B+C) = (-1;-1),\]
	et le milieu du segment $[AD]$ est donné par
		\[ M' = \dfrac12 (A+D) = (-1;-1) = M.\]
	Le quadrilatère est donc bien un parallélogramme.

	\centering
	\includegraphics[page=6]{figures/fig-exe.pdf}
}

\exe{, difficulty=1}{
	Soient $\point{A}12$ et $\point{M}3{-1}$ deux points du plan.
	Quel point $C$ faut-il choisir pour que $M$ soit le milieu du segment $[AC]$ ? Donner ses coordonnées.
}{exe:milieu2}{
	La contrainte que $M$ soit le milieu de $[AC]$ s'écrit
		\begin{align*}
			M = \dfrac12(A+C) && \iff && 2M = A+C && \iff && C = 2M-A
		\end{align*}
	Par suite,
		\[ C= 2M-A = 2\cdot\left(-\dfrac53;-1\right) - \left(-\dfrac32;\dfrac53\right) = \left(-\dfrac{11}6; -\dfrac{11}3\right). \]
	
}

\exe{, difficulty=2}{
	Soient $\point{A}{-\dfrac32}{\dfrac53}$ et $\point{M}{-\dfrac53}{-1}$ deux points du plan.
	Quel point $C$ faut-il choisir pour que $M$ soit le milieu du segment $[AC]$ ? Donner ses coordonnées.
}{exe:milieu2}{
	La contrainte que $M$ soit le milieu de $[AC]$ s'écrit
		\begin{align*}
			M = \dfrac12(A+C) && \iff && 2M = A+C && \iff && C = 2M-A
		\end{align*}
	Par suite,
		\[ C= 2M-A = 2\cdot\left(-\dfrac53;-1\right) - \left(-\dfrac32;\dfrac53\right) = \left(-\dfrac{11}6; -\dfrac{11}3\right). \]
	
}

\exe{}{
	Représenter les points $A(1;1)$ et $B(3;-1)$ dans un repère orthonormé.
	Représenter le point
		\[ \lambda A + (1-\lambda)B, \]
	pour certaines valeurs de $\lambda \in [0;1]$.
	
	Quel $\lambda$ choisir pour obtenir 
		\begin{multicols}{2}
		\begin{itemize}
			\item le point $A$ ?
			\item le point $B$ ?
			\item le milieu du segment $[AB]$ ?
			\item le point $C\left(\dfrac32; \dfrac12\right)$ ?
		\end{itemize}
		\end{multicols}
}{exe:milieu-segment}{
	TODO
}

\notations{
	Pour n'importe quel nombre $x$ on dénote $x^2 = x \cdot x$ le produit de $x$ par lui-même.
	On lit « $x$ au carré » ou « $x$ carré ».
}

\thm{Longueur d'un segment}{
	Soient $A(x_A, y_A)$, $B(x_B, y_B)$ deux points du plan dans un repère orthonormé.
	La longueur $\ell$ du segment $[AB]$ vérifie
		\[ \ell^2 = (x_A - x_B)^2 + (y_A - y_B)^2. \]
}{thm:long-segment}

\pf%{Démonstration du théorème \ref{thm:long-segment}}{
{}{
	Comme le repère est normé, on peut lire les distances sur les coordonnées.
	En particulier, la distance en la première coordonnée entre $A$ et $B$ est soit $(x_A - x_B)$, soit son opposé $(x_B - x_A)$..
	De la même façon, la distance en la deuxième coordonnée est $(y_A - y_B)$ ou $(y_B - y_A)$.
	On a donc le dessin suivant.
	
	\begin{center}
	\includegraphics[page=4, scale=1.1]{figures/fig-plan.pdf}
	\end{center}
	
	Comme le repère est orthogonal, les axes sont perpendiculaires et le triangle est donc rectangle.
	Le théorème de Pythagore s'applique donc. 
	Remarquons que le carré ignore le signe ($(-x)^2 = x^2$), et donc que $(x_A-x_B)^2 = (x_B - x_A)^2$.
	Il suit que
		\[ \ell^2 = (x_A - x_B)^2 + (y_A - y_B)^2, \]
	ce qui conclut.
}

\dfn{Norme}{
	Soit $u = (x ; y)$ un point quelconque.
	On définit le carré de sa \emph{norme}, $\norm{u}^2$, par
		\[ \norm{u}^2 = x^2 + y^2. \]
}{}

\mprop{reformulation du théorème \ref{thm:long-segment}}{
	La longueur du segment $[AB]$ vérifie
		\[ AB^2 = \norm{A - B}^2 = \norm{B - A}^2. \]
}{prop:long-segment}



\exe{}{
	Considérons les points $\point{A}{1}{1}, \point{B}{3}{1}, \point{C}{2}{\sqrt{3}+1}$.
	Démontrer que le triangle $ABC$ est équilatéral en calculant le carré de la longueur de chaque côté.
	
	\emph{Rappel : un triangle équilatéral est un triangle dont les trois côtés ont la même longueur}.
}{exe:équilatéral}{
	On calcule 
		\begin{align*}
			AB^2 = {2^2 + 0^2} = 4, && AC^2 = {1^2 + \sqrt{3}^2} = 4, && BC^2 = 4.
		\end{align*}
}

\exe{, difficulty=1}{
	Soient $\point{G}{-4}{-1}$ et $\point{D}{-1}{3}$ et $x\in\R$ un paramètre réel.
	Pour quel(s) $x\in\R$ est-ce que la longueur du segment entre les points $xG$ et $xD$ est-elle égale à $5$ ?
}{exe:milieu-x}{
	La longueur du segment est donnée par
		\[ 5 = \sqrt{(-4x + x)^2 + (-x - 3x)^2} = \sqrt{9x^2 + 16x^2} = \sqrt{25x^2}. \]
	En mettant l'équation au carré, on trouve
		\begin{align*}
			25 = 25x^2 && \iff &&  x^2 = 1.
		\end{align*}
	D'où on trouve les solutions $x=1$ ou $x=-1$.
}

\qs{}{
	Peut-on connaître exactement une longueur $\ell$ en connaissant son carré $\ell^2$ ?
	Plus généralement, comment connaître exactement un nombre en connaissant son carré ?
	L'étude de la fonction carré fait l'objet de chapitre \ref{chap:fonction-carré}.
}

\exe{}{
	Montrer que $(-2)^2 = 4$ et en déduire qu'il n'existe pas qu'une seule solution réelle à l'équation $x^2 = 4$.
}{exe:carre4}{
	Par définition, $(-2)^2 = (-2) \cdot (-2) = 4$.
	Comme $2^2 = 4$, il existe au moins deux solutions réelles à l'équation $x^2 = 4$ : 2 et $-2$.
}

\exe{, difficulty=2}{
	Montrer que $x^2 - 4 = (x-2)(x+2)$ pour tout $x\in\R$ et en déduire les deux seules solutions réelles de l'équation $x^2 = 4$.
}{exe:carre4-all}{
	Par distributivité, $(x-2)(x+2) = x^2 - 4$.
	Il suit que $x^2 = 4 \iff x^2 - 4 = 0 \iff (x-2)(x+2) = 0$.
	Or le produit de deux nombre n'est nul que si l'un des deux est nul.
	On a donc soit $x-2=0 \iff x=2$, soit $x+2=0 \iff x=-2$.
}

\exe{, difficulty=2}{
	Soient $A(1;1), B(3;1)$ deux points du plan, et $C(2;x)$ un point dépendant d'un paramètre réel $x\in\R$.
	Pour quel(s) $x\in\R$ le triangle $ABC$ est-il équilatéral ? 
	On supposera l'existence d'un nombre noté $\sqrt3$ dont le carré vaut 3.
}{exe:équilatéral2}{
	On souhaite que $AB = AC = BC$ soit vérifié, c'est-à-dire que
		\[ 2 = \sqrt{1 + (1-x)^2} = \sqrt{1 + (1-x)^2}. \]
	La deuxième égalité est redondante et, en mettant au carré, on trouve
		\begin{align*}
		4 = 1 + (1-x)^2 && \iff && (1-x)^2 = 3.
		\end{align*}
	En reprenant l'exercice \ref{exe:carre4-all}, on trouve deux solutions :
	\vspace{-20pt}
		\begin{multicols}{2}
		\begin{align*}
			1-x &= \sqrt{3}, \\
			x &= 1-\sqrt{3}.
		\end{align*}
			
		\begin{align*}
			1-x &= -\sqrt{3}, \\
			x &= 1 +\sqrt{3}.
		\end{align*}
		\end{multicols}
}{}

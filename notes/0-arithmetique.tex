%!TEX encoding = UTF8
%!TEX root =0-notes.tex

\chapter{Arithmétique}

\section{Parité}

\notations{
	Pour noter la multiplication entre deux valeurs numériques, on note $2\times3$.
	Si une lettre rentre en jeu, on notera la multiplication par un point médian ($2\cdot b$) ou simplement rien ($2b$).
}

\dfn{Multiple de 2}{
	La table de multiplication de $2$ donne tous les \emph{multiples} de $2$ :
		\setlength{\columnseprule}{0.4pt}
		\begin{multicols}{3}
			$2\times0 = 0$
			
			$2\times1 = 2$
			
			$2\times2 = 4$
			
			$2\times3 = 6$
			
			$2\times4 = 8$
			
			$2\times5 = 10$
			
			$2\times6 = 12$
			
			$2\times7 = 14$
			
			$2\times8 = 16$
			
			$2\times9 = 18$
			
			$2\times10 = 20$
			
			\qquad\vdots
		\end{multicols}
	Un multiple de $2$ est donc un nombre de la forme $2 \cdot k$, où $k\in\N$ est un entier naturel.
}{}

\dfn{Parité}{
	Soit $n\in\N$ un nombre entier.
	
	On dit que $n$ est \emph{pair} si $n$ est un multiple de 2.
	Sinon, on dit que $n$ est \emph{impair}.
}{}

\thm{}{
	Si $n\in\N$ est impair, alors
		\[ n = 2\cdot k + 1 \]
	pour un certain entier naturel $k\in\N$.
}{thm:impair}

\exe{}{
	Démontrer le théorème \ref{thm:impair}.
}{exe:thm-impair}{
	Considérons l'algorithme suivant : en partant de $n$ et tant que le résultat est postiif ou nul, on soustrait $2$ au nombre en mémoire.
	En notant $k$ le nombre de fois que 2 a été soustrait, deux résultats sont possibles : 
		\begin{enumerate}[label=\roman*)]
			\item soit on obtient 0, auquel cas $n=2k$ ;
			\item soit on obtient 1, auquel cas $n=2k+1$.
		\end{enumerate}
	Si le nombre est impair, la première situation ne peut pas advenir, et $n  = 2k+1$.
}

\notations{
	Lorsque deux éléments $a, b$ appartiennent au même ensemble $E$, on notera « $a, b \in E$ » pour signifier « $a \in E, b \in E$ ».
}

\thm{}{
	Soient $m, n\in\N$ deux entiers naturels.
		\begin{enumerate}
			\item Si $m$ et $n$ sont pairs, alors $m+n$ est également pair.
			\item Si $m$ est pair et $n$ est impair, alors $m+n$ est impair.
			\item Si $m$ et $n$ sont impairs, alors $m+n$ est pair.
		\end{enumerate}
}{thm:parité-somme}

\exe{}{
	Démontrer le théorème \ref{thm:parité-somme}.
}{exe:thm-parité-somme}{
	Si $m$ et $n$ sont pairs, alors $m=2k$ et $n=2\ell$ pour deux entiers $k, \ell \in \N$.
	Il suit que 
		\[ m + n = 2k + 2\ell = 2(k+\ell), \]
	et donc que $m+n$ est pair, car $k+\ell \in \N$.
	
	Si $m$ est pair et $n$ est impair, alors $m=2k$ et $n=2\ell+1$ pour deux entiers $k, \ell \in \N$.
	Il suit que 
		\[ m + n = 2k + 2\ell + 1 = 2(k+\ell) + 1, \]
	et donc que $m+n$ est impair, car $k+\ell \in \N$.
	
	Si $m$ et $n$ sont impairs, alors $m=2k+1$ et $n=2\ell+1$ pour deux entiers $k, \ell \in \N$.
	Il suit que 
		\[ m + n = 2k +1 + 2\ell + 1 = 2(k+\ell + 1), \]
	et donc que $m+n$ est pair, car $k+\ell+1 \in \N$.
}

\section{Diviseurs}

\dfn{}{
	Soient $a, n \in \N$ deux entiers naturels.
	
	On dit que « $a$ divise $n$ » ou que « $n$ est un multiple de $a$ » dès que
		\[ n = a \cdot k, \]
	où $k\in\N$ est un entier naturel.
}{}

\notations{
	On note $a \big| n$ la relation « $a$ divise $n$ ».
}

\exe{, difficulty=1}{
	Montrer que si $a \big| n$, alors $(2a) \big| (2n)$.
}{exe:aknk}{
	Par définition, $n = ak$.
	En multipliant par $2$, on obtient $2n = (2a)k$, ce qui conclut.
}

\exe{, difficulty=1}{
	Montrer que si $a \big| n$, alors $(am) \big| (nm)$ pour tout $m\in\Z$.
}{exe:aknk}{
	Par définition, $n = ak$.
	En multipliant par $m$, on obtient $nm = (am)k$, ce qui conclut.
}

\dfn{}{
	Pour un $n\in\N$, on pose $\D_n$ l'ensemble des diviseurs de $n$.
		\[ D_n = \bigset{ a \in \N \tq a \big| n }. \]
}{dfn:ensemble-diviseurs}

\nt{
	Les diviseurs se regroupent toujours par paires, car si $a$ divise $n$ et $n=a \cdot b$, alors $b$ divise $n$ aussi.
}

\exe{}{
	Donner $\D_2, \D_3, \D_4, \D_5, \D_6, \D_7, \D_8$.
}{exe:diviseurs}{
	\begin{multicols}{2}
	$D_2 = \{ 1 ; 2\}$
	
	$D_3 = \{ 1 ; 3\}$
	
	$D_4 = \{ 1 ; 4 ; 2 \}$
	
	$D_5 = \{ 1 ; 5\}$
	
	$D_6 = \{ 1 ; 6 ; 2 ; 3\}$
	
	$D_7 = \{ 1 ; 7\}$
	
	$D_8 = \{ 1 ; 8 ; 2 ; 4\}$
	\end{multicols}
}

\exe{, difficulty=1}{
	Montrer que le nombre de diviseurs de $n$ n'est pas pair si et seulement si $n$ est un carré parfait ($n = k^2$ pour un $k\in\N$).
}{exe:parité-diviseurs}{
	Les diviseurs se regroupent par paires $(a ; b)$ où $a\cdot b = n$, sauf éventuellement si $a$ et $b$ sont égaux, auquel cas $n = a^2$.
}

\thm{}{
	Considérons $n\in\N$ un entier naturel tel que $2 | n$ et $3 | n$ : $n$ est pair et multiple de $3$.
	
	Alors $6 | n$ : $n$ est multiple de $6$.
}{}

\pf{}{
	Par hypothèses, $n=2k$ et $n=3\ell$, pour certains $k,\ell\in\N$.
	Or $n = 3n - 2n = 3(2k) - 2(3\ell) = 6(k-\ell)$, ce qui conclut.
}

\exe{}{
	Montrer que si $5 \big| n$ et $6 \big| n$, alors $30 \big| n$.
}{exe:coprimalité1}{
	Comme $n = 5k$ et $n = 6 \ell$, on peut écrire
		\[ n = 6n - 5n = 6(5k) - 5(6\ell) = 30(k - \ell). \]
}

\exe{, difficulty=1}{
	Montrer que si $a \big| n$ et $(a+1) \big| n$, alors $a(a+1) \big| n$.
}{exe:coprimalité2}{
	On a bien sûr $(a+1) - a = 1$, qui implique
		\[ n = (a+1)n - an. \]
	Le terme $(a+1)n$ est divisible par $a(a+1)$ car $n$ est divisible par $a$.
	Le terme $an$ est aussi divisible par $a(a+1)$ car $n$ est divisible par $a+1$.
	La somme, $n$, l'est donc également.
}

\exe{}{
	Montrer qu'en général, si $a\big|n$ et $b\big|n$, on a pas nécessairement $(ab)\big|n$.
}{exe:non-coprimalité}{
	Trivialement, 2 divise 2, mais 4 ne divise pas 2.
	Autrement, on peut également citer $3 \big|  24$ et $6 \big| 24$ mais 18 ne divise pas 24.
}

\thm{}{
	Considérons $a \big| n$. Alors
		\[ \bigl( m \text{ divisible par $a$} \bigr) \iff \bigl( m+n \text{ divisible par $a$} \bigr). \]
	
	Ajouter un multiple de $a$ ne change pas la divisibilité par $a$ d'un nombre : s'il l'était avant, il le sera alors, et s'il ne l'était pas, il ne le sera pas.
}{thm:divisibilité-combinaison-entière}


\pf{}{
	On a d'abord $n=a\cdot k$, par définition.
	Pour montrer une équivalence $\iff$, on sépare les deux implications à démontrer.
	\begin{enumerate}[leftmargin=120pt]
		\item[\underline{Direction $\implies$ :}]
		Si $a \big| m$, alors $m=a \cdot \ell$, et $n+m = a(k + \ell)$, qui est divisible par $a$ comme requis.
		
		\item[\underline{Direction $\impliedby$ :}]
		Si $a \big| (n+m)$, alors $n+m = a \cdot \ell$.
		Il suit que $m = a \cdot \ell - n = a (\ell - k)$, qui est multiple de $a$ comme requis.
	\end{enumerate}
}

\nt{
	Le même argument peut s'étendre de la façon suivante : s'il existe $u, v \in\Z$ tels que
		\[ ua + vb = 1, \]
	alors $\bigl( a\big|n \text{ et } b\big|n \bigr) \implies \bigl( (ab) \big| n \bigr)$.
}

\exe{, difficulty=2}{
	Démontrer la remarque ci-dessus.
}{ex:coprimalité2}{
	Si $ua + vb = 1$ et que $a\big|n \text{ et } b\big|n$, alors
		\[ u(an) + v(bn) = n, \]
	avec $ab$ qui divise $bn$ et $an$ car $a$ divise $n$ et $b$ divise $n$ (voir exercice \ref{exe:aknk}).
	
	Ainsi $ab$ divise aussi $uan$ et $vbn$ ainsi que leur somme, valant $n$.
}

\thm{Règle de divisibilité par 3, 9}{
	Un nombre est divisible par 3 (respectivement 9) si et seulement si la somme de ses chiffres est divisible par 3 (resp. 9).
}{thm:règle-divisibilité-3-9}

\pf{Démonstration du théorème \ref{thm:règle-divisibilité-3-9} pour la divisibilité par 3}{
	Considérons un nombre à deux chiffres pour commencer.
		\begin{align*}
			n = [ab] && \iff && n = 10a + b.
		\end{align*}
	Or comme $-9a = 3(-3a)$ est divisible par 3, on peut soustraire $9a$ sans changer la divisibilité par $3$ de $n$ d'après le théorème \ref{thm:divisibilité-combinaison-entière}.
	On obtient ainsi notre critère de divisibilité car
		\[ \bigl( n = 10a+b \text{ divisible par 3} \bigr) \iff \bigl( n-9a = a+b \text{ divisible par 3} \bigr). \]
	
	Continuons pour un nombre à trois chiffres :
		\begin{align*}
			n = [abc] && \iff && n = 100a + 10b + c.
		\end{align*}
	On résonne de la même façon en soustrayant $99a$ et $9b$ à $n$ sans changer sa divisibilité par $3$ :
		\[ \bigl( n = 100a + 10b + c \text{ divisible par 3} \bigr) \iff \bigl( n-99a-9b = a+b+c \text{ divisible par 3} \bigr). \]
	
	En général, $10^n \cdot a$ est divisible par 3 si et seulement si $a$ l'est, ce qui conclut.
}{}

\exe{}{
	Démontrer la règle de divisibilité par 9 énoncée dans le théorème \ref{thm:règle-divisibilité-3-9}.
}{exe:règle-divisibilité-3-9}{
	$10^n \cdot a$ est divisible par $9$ si et seulement si $a$ l'est, car 
		\[ 10^n \cdot a = (10^n - 1)a + a, \]
	et puisque $10^n - 1 = \underbrace{99{\dots}99}_{\text{n fois}}$ est un multiple de 9.
}

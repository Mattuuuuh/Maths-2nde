%!TEX encoding = UTF8
%!TEX root =notes.tex

\setcounter{maincounter}{0}
\chapter{Arithmétique}

\section{Parité}

\notations{
	Pour noter la multiplication entre deux valeurs numériques, on note $2\times3$.
	Si une lettre rentre en jeu, on notera la multiplication par un point médian ($2\cdot b$) ou simplement rien ($2b$).
}

\dfn{Multiple de 2}{
	La table de multiplication de $2$ donne tous les \emph{multiples} de $2$ :
		\setlength{\columnseprule}{0.4pt}
		\begin{multicols}{3}
			$2\times0 = 0$
			
			$2\times1 = 2$
			
			$2\times2 = 4$
			
			$2\times3 = 6$
			
			$2\times4 = 8$
			
			$2\times5 = 10$
			
			$2\times6 = 12$
			
			$2\times7 = 14$
			
			$2\times8 = 16$
			
			$2\times9 = 18$
			
			$2\times10 = 20$
			
			\qquad\vdots
		\end{multicols}
	Un multiple de $2$ est donc un nombre de la forme $2 \cdot k$, où $k\in\N$ est un entier naturel.
}{}

\dfn{Parité}{
	Soit $n\in\N$ un nombre entier.
	
	On dit que $n$ est \emph{pair} si $n$ est un multiple de 2.
	Sinon, on dit que $n$ est \emph{impair}.
}{}

\thm{}{
	Si $n\in\N$ est impair, alors
		\[ n = 2\cdot k + 1 \]
	pour un certain entier naturel $k\in\N$.
}{}

\notations{
	Lorsque deux éléments $a, b$ appartiennent au même ensemble $E$, on notera « $a, b \in E$ » pour signifier « $a \in E, b \in E$ ».
}

\thm{}{
	Soient $m, n\in\N$ deux entiers naturels.
		\begin{enumerate}
			\item Si $m$ et $n$ sont pairs, alors $m+n$ est également pair.
			\item Si $m$ est pair et $n$ est impair, alors $m+n$ est impair.
			\item Si $m$ et $n$ sont impairs, alors $m+n$ est pair.
		\end{enumerate}
}{}

\section{Diviseurs}

\dfn{}{
	Soient $a, b \in \N$ deux entiers naturels.
	
	On dit que « $a$ divise $b$ » ou que « $b$ est un multiple de $a$ » dès que
		\[ b = a \cdot k, \]
	où $k\in\N$ est un entier naturel.
}{}

\notations{
	On note $a | b$ la relation « $a$ divise $b$ ».
}

\thm{}{
	Considérons $n\in\N$ un entier naturel tel que $2 | n$ et $3 | n$ : $n$ est pair et multiple de $3$.
	
	Alors $6 | n$ : $n$ est multiple de $6$.
}{}

\pf{}{
	Par hypothèses, $n=2k$ et $n=3\ell$, pour certains $k,\ell\in\N$.
	Or $n = 3n - 2n = 3(2k) - 2(3\ell) = 6(k-\ell)$, ce qui conclut.
}

\nt{
	Le même argument peut s'étendre de la façon suivante : s'il existe $u, v \in\Z$ tels que
		\[ ua + vb = 1, \]
	alors $\bigl( a|n \text{ et } b|n \bigr) \implies \bigl( (ab) | n \bigr)$.
}



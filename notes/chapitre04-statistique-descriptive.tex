%!TEX encoding = UTF8
%!TEX root =notes.tex
\chapter{Statistiques descriptives}

\section{Sous-populations}

On considère dans cette section des populations qu'on représente par un ensemble fini non vide car l'ordre des individus n'importe pas.

\ex{Ensemble d'élèves}{
  On représente l'ensemble des $34$ élèves d'une classe par l'ensemble
  \[ E = \{ 1 ; 2 ; 3 ; \dots ; 32 ; 33 ; 34 \}. \]
  Chaque élève est associé à un nombre (par exemple sa place dans l'ordre alphabétique de la liste d'appel).
}{ex:pop-eleves}

Une sous-population de l'ensemble $E$ est une partie de l'ensemble : c'est un ensemble $F$ inclus dans $E$.
\[ F \subseteq E. \]
On note $|E| \in \N$ le cardinal d'un ensemble\footnote{Il existe d'autres notations : $\text{Card}(E)$ ou $\#E$ par exemple.}, c'est-à-dire le nombre d'éléments distincts qu'il contient.
C'est un entier naturel.

\ex{Ensembles d'élèves pairs}{
  Soit $E$ donné à l'exemple \ref{ex:pop-eleves}.
  L'ensemble
  \[ F = \{ 2 ; 4; 6 ; \dots ; 32 ; 34 \} \]
  vérifie $F\subseteq E$ et peut donc être compris comme une sous-population des élèves de l'exemple \ref{ex:pop-eleves}.
  Ce sont en fait tous les élèves ayant une position paire dans le classement utilisé.

  Les cardinaux des ensembles sont donnés pas
  \begin{align*}
    |E| = 34 && \text{et} && |F| = 17.
  \end{align*}
}{ex:pop-eleves2}

\dfn{Proportion, pourcentage de sous-population}{
  Considérons deux ensembles finis $F \subseteq E$ avec $E$ non vide.
  La proportion d'éléments de $E$ appartenant à $F$ est donnée par
  \[ p = \dfrac{|F|}{|E|}. \]
  Lorsqu'elle est exprimée sous la forme d'une fraction de $100$, on parle alors de \emph{pourcentage} d'éléments de $E$ appartenant à $F$.
}{}

\ex{Proportion d'élèves pairs}{
  En reprenant les notations de l'exemple \ref{ex:pop-eleves2}, on trouve
  \[ p = \dfrac{17}{34} = \dfrac12 = \dfrac{50}{100} = 50\%. \]
  Ainsi la proportion d'élèves de $E$ appartenant à $F$ est $\frac12 = 50\%$.
}{}

\nt{
  Soient $F \subseteq E$ deux ensembles finis avec $E$ non vide.
  On a alors forcément
  \[ 0 \leq |F| \leq |E|, \]
  et donc
  \begin{align*}
    0 \leq p \leq 1 && \iff && p \in [0 ; 1] && \iff 100\cdot p \in [0 ; 100].
  \end{align*}
  La proportion d'une sous-population est forcément inférieure à $1$ (et le pourcentage inférieur à $100$).
}

\nt{
  On convertit une proportion $p \in [0;1]$ en un pourcentage en la multipliant par $100$.
  Ainsi, un pourcentage peut s'approximer simplement à l'aide des deux premières décimales de $p$.
  \begin{multicols}{2}
  \begin{enumerate}
  \item $1 = 100\%$
  \item $\dfrac12 =0{,}5 = 50\%$
  \item $\dfrac14 = 0{,}25 = 25\%$
  \item $\dfrac15 = 0{,}2 = 20\%$
  \item $\dfrac1{10}=0{,}1 = 10\%$
  \item $\dfrac1{100} = 0{,}01 = 1\%$
  \item $\dfrac1{50} = 0{,}02 = 2\%$
  \item $0 = 0\%$
  \end{enumerate}
  \end{multicols}

  Les valeurs suivantes sont approximatives mais très utiles lorsqu'on souhaite estimer des pourcentages.

  \begin{multicols}{2}
  \begin{enumerate}
  \item $\dfrac13 \approx 33\%$
  \item $\dfrac23 \approx 66\%$
  \item $\dfrac17 \approx 14\%$
  \item $\dfrac19 \approx 11\%$
  \end{enumerate}
  \end{multicols}
  \vspace{3pt} % fractions goes out of bounds
}

\ex{Proportion de fourmis}{
  Au total sur Terre on dénombre actuellement $16 600$ espèces de fourmis parmis les $1{,}3$ millions d'espèces d'insectes déjà décrites.

  Ainsi, la proportion de fourmis parmis les insectes est la suivante.
  \[ \dfrac{16 600}{1{,}3\times10^{6}} \approx 0{,}013 = \dfrac{1{,}3}{100} = 1{,}3 \%. \]
}{ex:fourmis}

\nt{
  Considérons $F \subseteq E$ deux ensembles finis avec $E$ non vide.
  Supposons qu'on connaisse la proportion $p = \dfrac{|F|}{|E|}$.
  Alors
  \begin{align*}
    |F| = p \cdot |E| && \text{et} && |E| = p^{-1} |F| = \dfrac1p |F|
  \end{align*}  
  En connaissant la proportion de $F$ dans $E$, on peut donc déduire la taille de $F$ à partir de celle de $E$ ou vice-versa.

  \textbf{Attention} : on ne peut pas déduire l'ensemble en lui-même, uniquement son cardinal !
}

\ex{}{
  En reprenant les notations de l'exemple \ref{ex:pop-eleves2}, extraire $50\%$ de l'ensemble $E$ donne un ensemble de taille $\dfrac12 \cdot 34 = 17$.
  Cependant, il n'est pas possible de connaître ce sous-ensemble, car l'ensemble $F$ des élèves pairs et l'ensemble
  \[ G = \{ 1 ; 3 ; 5 ; \dots ; 29 ; 31 ; 33 \} \]
  des élèves impairs sont disjoints et tous les deux de cardinal $17$.
}{}

\ex{}{
  On continue l'exemple \ref{ex:fourmis}.
  En extrayant $1{,}3\%$ d'une population de $1{,}3$ millions d'individus, on obtient $16900$ individus.
  \[ \dfrac{1{,}3}{100} \cdot 1{,}3\times10^{6} = 0{,}013 \cdot 1{,}3\times10^{6} = 16900. \]
  L'approximation ($\approx$) de l'exemple \ref{ex:fourmis} explique la différence entre les valeurs trouvées.

  Réciproquement, en sachant que $16600$ espèces constituent $1{,}3\%$ du total sur Terre, on calcule
  \[ 16 600 \cdot \left(\dfrac{1{,}3}{100}\right)^{-1} = \dfrac{16 600 \cdot 100}{1{,}3} \approx 1 276 923. \]
}{ex:fourmis-2}

\ex{Proportions de populations imbriquées}{
  En 2023 en France, $13\%$ des espèces (faune et flore) sont considérées comme menacées à l'échelle mondiale (catégories ``danger critique'' à ``vulnérable'' de l'UICN).
  Parmis celles-ci, $23\%$ sont en danger critique. \footnote{\href{https://naturefrance.fr/indicateurs/proportion-en-france-despeces-menacees-lechelle-mondiale}{https://naturefrance.fr/indicateurs/proportion-en-france-despeces-menacees-lechelle-mondiale}}

  On cherche à calculer le pourcentage d'espèce en danger critique par rapport au nombre total d'espèces.
  Notons $N$ le nombre total d'espèces. Le nombre d'espèces menacées et donc donné par
  \[ 0{,}13 \cdot N. \]
  Parmis ces espèces, le nombre en danger critique est donné par
  \[ 0{,}23 \cdot \left(0{,}13 \cdot N \right) = ( 0{,}23 \cdot 0{,}13 ) \cdot N \approx 0{,}03 \cdot N. \]
  Ainsi, $3\%$ des espèces sont en danger critique.
}{}

\nt{
  Les proportions sont multiplicatives.
  Au même titre que prendre la moitié de la moitié revient à prendre un quart,
  prendre $50\%$ de $50\%$ revient à multiplier par $0{,}5 \cdot 0{,}5 = \left(\dfrac12\right)^2 = \dfrac14$.

  Lorsqu'on prend la moitié du tiers d'une quantité, on en prend en fait $\dfrac12 \cdot \dfrac13 = \dfrac16$.
  On peut estimer la fraction $\dfrac16$ en prenant la moitié de celle de $\dfrac13 \approx 0{,}33$.
  D'où $\dfrac16 \approx 16,6\%$.
}

\section{Évolution}

On généralise la notion de pourcentage de valeur à tous les nombres réels positifs ou nuls et à tous les pourcentages possibles (en permettant ceux supérieurs à $100\%$).

\dfn{}{
  Soit $x, p\in\R$ une valeur et une proporition positives ou nulles.
  On rappelle que $100p$ est le pourcentage associé à $p$.

  On dira \og $100p \%$ de $x$ \fg pour parler de la valeur
  \[ p \cdot x. \]
}
{}


\ex{}{
  Un manteau est mis en vente à un prix initial de $150$€ auquel une remise de $45\%$ est appliquée.
  Celui coûte donc $55\%$ de $150$ euros, ce qui vaut
  \[ 0{,}55 \cdot 150 = 82{,}5.\]
}{}

\mprop{}{
  Soient $A$ et $B$ deux nombres réels positifs ou nuls.
  Les quantités \og $A\%$ de $B$ \fg et \og $B\%$ de $A$ \fg sont égales.
}{}

\exe{}{
  Calculer sans calculatrice les pourcentages suivants.
  \begin{multicols}{2}
    \begin{enumerate}
    \item $50\%$ de $60$
    \item $60\%$ de $50$
    \item $68\%$ de $25$
    \item $77\%$ de $20$
    \end{enumerate}
  \end{multicols}
}{}

\exe{}{
  On estime la biomasse totale des fourmis sur Terre à $12$ millions de tonnes.
  Ceci représenterait $20\%$ de la biomasse humaine.

  Estimer la biomasse totale des humains sur Terre en tonnes ($1$T $= 1000$kg).
}{}


Le pourcentage peut être également marqueur d'une évolution d'une même quantité au fil du temps.
On calculera alors généralement une proportion $p$ donnant l'évolution entre deux valeurs.

\begin{align}\label{eq:evolution}
  p = \dfrac{\text{valeur initiale}}{\text{nouvelle valeur}}.
\end{align}

\mprop{}{
  Soient $a,b$ deux réels strictement positifs et $p = \dfrac{a}b$ leur rapport.
  On distingue trois cas.
  \begin{enumerate}
    \item si $a=b$, alors $p=1$ ;
    \item si $a >b$, alors $p>1$ ; et
    \item si $a < b$, alors $p<1$.
  \end{enumerate}
}{}

\nt{
  La proportion $p$ calculée en \eqref{eq:evolution} peut donc être supérieure à $1$ si la valeur initiale est plus petite que la nouvelle valeur.
  Ceci correspond à un pourcentage supérieur à $100\%$.
}

\exe{Intérêts au long terme}{
  Une jeune femme dépose $10$€ à la banque. Celle-ci lui promet un taux d'intérêt à l'année de $1{,}5\%$.
  Ainsi, après la première année, il y aura $1{,}015 \cdot 10 = 10{,}15$€ sur son compte.
  La deuxième année, il y aura $1{,}015 \cdot 10{,}15 = 10{,}30225$€, etc...
  
  À l'aide de la calculatrice, répondre aux questions suivantes.
  \begin{enumerate}
  \item Combien d'argent aura-t-elle après $5$ ans ?
  \item Combien d'argent aura-t-elle après $50$ ans ?
  \item Combien d'argent y aura-t-il sur son compte après $1000$ ans ?
  \end{enumerate}
}
{}

\dfn{Augmentation, diminution}{
  Une \emph{augmentation} d'une valeur $N$ d'une proportion $p\geq0$ correspond à la somme
  \[ N + p\cdot N = (1+p)\cdot N.\]
  Une \emph{diminution} d'une valeur $N$ d'une proportion $0\leq p \leq 1$ correspond à la différence
  \[ N - p\cdot N = (1-p)\cdot N.\]
}{}

\ex{}{
  En $2023$, le prix moyen du gaz naturel facturé aux ménages français s'élève à $115$€ par MWh, toutes taxes comprises (TTC).\footnote{\href{https://www.statistiques.developpement-durable.gouv.fr/media/7469}{https://www.statistiques.developpement-durable.gouv.fr/media/7469}}
  En $2022$, le prix était de $96$€.

  On calcule $\dfrac{115}{96} \approx 1{,}2$.
  Le prix de $2023$ est donc $120\%$ celui de $2022$, ce qui correspond à une augmentation de $20\%$.
}{}

\exe{}{
  En $2022$, la consommation de gaz naturel s'établit à $463$ TWh.
  En $2021$, celle-ci s'élvait plutôt à $475{,}85$ TWh. \footnote{\href{https://www.statistiques.developpement-durable.gouv.fr/edition-numerique/chiffres-cles-energie-2023/14-gaz-naturel}{https://www.statistiques.developpement-durable.gouv.fr/edition-numerique/chiffres-cles-energie-2023/14-gaz-nature}}

  Calculer le pourcentage de diminution de la consommation entre l'année $2021$ et l'année $2022$.
}{}

\ex{}{
  Une augmentation de $50\%$ revient à multiplier une valeur initiale $a$ par $1{,}5$ pour obtenir une valeur finale $b$.
  \[b = 1{,}5\cdot a.\]
  En changeant de référentiel, on peut se demander quel pourcentage de diminution doit-on appliquer à la valeur finale $b$ pour obtenir la valeur initiale $a$ ?
  \[ a = \dfrac{1}{1,5} \cdot b = \dfrac23 b \approx 0{,}66 \cdot b. \]
  Pour obtenir $a$, il faut donc réduire $b$ d'environ $34\%$.
}{}

\exe{}{
  Si on augmente le prix d'un objet de $150\%$, quel rabais faut-il appliquer pour retrouver le prix initial de l'objet ?
}{}

\exe{}{
  Si on augmente le prix d'un objet de $100p\%$ ($p\geq0$ réel), quel rabais faut-il appliquer (en fonction de $p$) pour retrouver le prix initial de l'objet ?
}{}

\exe{}{
  Considérons deux tailleurs, l'un à $250$€ et l'autre à $360$€.
  \begin{enumerate}
  \item Quelle augmentation de prix faut-il appliquer au premier tailleur pour qu'il ait le prix du second ?
  \item Quel rabais faut-il appliquer au second tailleur pour qu'il ait le prix du premier ?
  \end{enumerate}
}{}


\section{Moyennes et écart type}


\section{Lecture graphique}


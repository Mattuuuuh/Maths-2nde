%!TEX encoding = UTF8
%!TEX root =notes.tex

\chapter{Variations et extrema}

Le but de ce chapitre est de traiter la partie \og Étudier les variations et les extremums d'une fonction \fg~du bulletin officiel.

Le contenu du chapitre est le suivant.
	\begin{enumerate}
		\item Croissance, décroissance, monotonie d'une fonction définie sur un intervalle. Tableau de variations.
		\item Maximum, minimum d'une fonction sur un intervalle.
		\item Pour une fonction affine, interprétation du coefficient directeur comme taux d'accroissement, variations selon le signe.
		\item Variations des fonctions carré, inverse, racine carrée, cube.
	\end{enumerate}

Les capacités attendues sont les suivantes.
	\begin{itemize}
		\item Relier représentation graphique et tableau de variations.
		\item Déterminer graphiquement les extremums d0une fonctions sur un intervalle.
		\item Exploiter un logiciel de géométrie dynamique ou de calcul formel, la calculatrice ou Python pour décrire les variations d'une fonction donnée par une formule.
		\item Relier sens de variation, signe, et droite représentative d'une fonction affine.
	\end{itemize}
	
\section{Introduction}

Rappel de manipulation d'inégalités, déjà étudiés dans la section \ref{subsec:ineg} du chapitre \ref{chap:3}.

\dfn{Inégalités}{
	On définit les signes suivants correspondant à des inégalités \emph{strictes} et \emph{larges}.
		\begin{enumerate}
			\item $<$ : strictement inférieur à
			\item $\leq$ : inférieur ou égal à
			\item $>$ : strictement supérieur à
			\item $\geq$ : supérieur ou égal à
		\end{enumerate}
}{}

\thm{}{
	Soient $x, y, a \in \R$ vérifiant
		\[ x \leq y. \]
	Alors l'inégalité ci-dessus est équivalente aux inégalités suivantes.
		\begin{enumerate}
			\item $x + a \leq y + a$, sans condition sur $a$.
			\item $a x \leq a y$ si $a > 0$ est strictement positif.
			\item $ax \geq a y$ si $a < 0$ est strictement négatif.
		\end{enumerate}
}{thm:ineg}
% important : 
\pf{Justifications du théorème \ref{thm:ineg}}{
	Le point $1.$ se justifie sans peine avec un dessin : ajouter $a$ revient à décaler les points $x$ et $y$ de la droite réelle vers la droite (si $a >0$) ou la gauche (si $a<0$) d'une même distance. 
	Leur relation d'ordre reste donc inchangée.
	
	Étant donné $1.$, on a que $x \leq y$ est équivalent à $y-x \geq 0$.
	Ainsi $y-x$ est un nombre réel positif.
	Or, multiplier deux positifs entre eux donne un nombre positif, et multiplier un nombre négatif avec un positif donne un nombre négatif.
	
	Par conséquent, si $a > 0$, on a bien
		\[ a \cdot (y-x) \geq 0 \iff ay \geq ax \iff ax \leq ay, \]
	et si $a < 0$, on a
		\[ a \cdot (y-x) \leq 0 \iff ay \leq ax \iff ax \geq ay, \]
	ce qui conclut.
}



\section{Variations et tableaux de variations}


On distingue trois cas différents de droites parmis les exemples de la section \ref{sec:aff-1} : les deux premières sont croissantes, la suivante est constante, et la dernière est décroissante.

\dfn{Variations}{
	Soit $f : \D \rightarrow \R$ une fonction $f$ quelconque sur un intervalle $I\subseteq\R$.
	Alors
		\begin{itemize}
			\item On dit que $f$ est \emph{croissante} si, pour tous les $x,y\in I$ du domaine,
				\begin{align*}
					x < y && \implies && f(x) \leq f(y).
				\end{align*}	
			On interprète l'implication ainsi :
			\begin{center}
				\og lorsqu'on augmente l'abscisse $x$, l'ordonnée $f(x)$ augmente \fg.
			\end{center}
				
			\item On dit que $f$ est \emph{décroissante} si, pour tous les $x,y\in I$ du domaine,
				\begin{align*}
					x < y && \implies && f(x) \geq f(y).
				\end{align*}
			On interprète l'implication ainsi :
			\begin{center}
				\og lorsqu'on augmente l'abscisse $x$, l'ordonnée $f(x)$ diminue \fg.
			\end{center}
				
			\item On dit que $f$ est \emph{constante} si, pour tous les $x\in I$ du domaine, et pour une certaine constante $K\in\R$,
				\begin{align*}
					f(x) = K.
				\end{align*}
		\end{itemize}
	On définiera de la même façon \og strictement croissante \fg ~ (respectivement décroissante) en remplaçant l'inégalité large $\leq$ (resp. $\geq$) par l'inégalité stricte $<$ (resp. $>$).
}{}

\thm{Variations}{
	Soit $f$ une fonction affine où $a, b \in\R$ sont ses deux paramètres réels.
		\begin{align*}
			f(x) = a x + b && (x\in\R)
		\end{align*}
	On distingue trois cas de figure.
		\begin{itemize}
			\item Si $a < 0$, alors $f$ est strictement décroissante.
			\item Si $a=0$, alors $f$ est constante.
			\item Si $a>0$, alors $f$ est strictement croissante.
		\end{itemize}
}{thm:affine-var}
%%%%%%%% MAYBE SKIP THIS? 
%%%%%%%% je pense que dy/dx est plus clair : si x monte et y descend, ça fait negatif/positif = négatif, pex.
%%%%%%%% après faut qu'ils soient convaincus de a=dy/dx et personne n'a réussi à le démontrer dans le bonus :((
\pf{Démonstration du théorème \ref{thm:affine-var}}{
	Si $a=0$, $f$ est clairement constante car
		\begin{align*}
			f(x) = b \qquad \text{ pour tout $x\in\R$}.
		\end{align*}
	
	Sinon, on utilise les règles de manipulation des inégalités du théorème \ref{thm:ineg} vues au chapitre \ref{chap:3}.
	
	Si $a>0$, alors, pour tous les $x,y \in \R$ tels que $x < y$, on a
		\begin{align*}
			x &< y \\
			a \cdot x &< a \cdot y \\
			a \cdot x + b &< a \cdot y + b \\
			f(x) &< f(y),
		\end{align*}
	et $f$ est donc strictement croissante.
	
	Si $a<0$, alors, pour tous les $x,y \in \R$ tels que $x < y$, on a
		\begin{align*}
			x &< y \\
			a \cdot x &> a \cdot y \\
			a \cdot x + b &> a \cdot y + b \\
			f(x) &> f(y),
		\end{align*}
	et $f$ est donc strictement décroissante.
}{}

\dfn{Tableau de variations}{
	todo
}{}

\section{Extrema}

\dfn{Minimum, maximum d'une fonction sur un intervalle}{
	Soit $f : I \rightarrow \R$ une fonction réelle $f$ sur un intervalle $I$.
	
	On dit que, pour $x^\star \in I$, $f(x^\star)$ est le minimum de $f$ sur $I$ dès que
		\[ f(x^\star) \leq f(x), \]
	pour tout $x\in I$.
	$x^\star$ est l'antécédent qui \emph{réalise} le minimum. Il n'est pas nécessairement unique.
	
	On dit que, pour $x^\star \in I$, $f(x^\star)$ est le maximum de $f$ sur $I$ dès que
		\[ f(x^\star) \geq f(x), \]
	pour tout $x\in I$.
	$x^\star$ est l'antécédent qui \emph{réalise} le maximum. Il n'est pas nécessairement unique.
}{}

\ex{}{
	La fonction carré $f(x)=x^2$ sur $\R$ tout entier n'admet pas de maximum (à démontrer rigoureusement !).
	Cependant, $f$ admet son minimum $0$ en $x^\star = 0$, car on a 
		\[ f(x) = x^2 \geq 0 = f(0), \]
	pour tout $x\in\R$.
	C'est d'ailleurs \emph{le} minimum car seul $0$ vérifie $f(x) = 0$.
}{}

\exe{}{
	Soit $f(x) = x^2$ définie sur $I \subset \R$ un intervalle borné de $R$.
	Montrer que $f$ atteint son maximum en l'une des deux bornes de $I$.
}{}


\dfn{Tableau de variations cont'd}{
	todo
}{}



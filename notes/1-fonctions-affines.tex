%!TEX encoding = UTF8
%!TEX root = 0-notes.tex

\chapter{Droites et fonctions affines}


%%%%%%%
%%%%%%%

% possibilité de faire comme suit : définir uniquement l'identité y=x et s'assurer que c'est une droite.
% ensuite y = ax est une fonction parente, ce qui donne l'aspect selon le signe de a d'ailleurs
% puis y = ax+b est un shift d'ordonnées. ce qui donne l'ordonnée à l'origine d'ailleurs

%%%%%%%
%%%%%%%
%%%%%%%

\section*{Préliminaire}

\begin{multicols}{2}

Soit $(d)$ une droite \emph{non verticale} qu'on plonge dans un repère orthonormé.
On suppose $(d)$ non verticale pour qu'elle soit la courbe représentative d'une certaine fonction $f$ qu'on souhaite déterminer : \[(d) = \C_f. \]
%Posons en outre $\D = \R$ le domaine de $f$, soit la droite réelle toute entière.

	\begin{center}
	%\includegraphics[page=1, scale=.9]{figures/fig-affines.pdf}
	\includegraphics[page=2, scale=1]{figures/fig-affines.pdf}
	\end{center}

%On a supposé $(d)$ non verticale car dans le cas contraire, un certain antécédent $x$ aurait plusieurs images, ce qui n'est pas possible d'après la définition d'une fonction.

Comme on a supposé $(d)$ non verticale, elle coupe nécessairement l'axe des ordonnées.
Posons $(0; b)$ le point d'intersection, où $b\in\R$ est un réel désormais fixé.

%On souhaite trouver la fonction $f$ telle que $(d) = \C_f$, sa courbe représentative.
On rappelle la propriété fondamentale (voir corollaire \ref{cor:prop-fond}) : 
	\begin{align*}
		(x;y) \in \C_f && \iff && y = f(x).
	\end{align*}
Prenons donc un $x \in \R$ non nul, et essayons de déterminer l'ordonnée du point de $(d)$ d'abscisse $x$ afin d'obtenir $f(x)$.

\vspace{30pt}

On se restreint au triangle rectangle dont les sommets sont $(0;b), (x ; b)$, et $(x;y)$ pour appliquer les résultats de trigonométrie du chapitre \ref{chap:pb-geom}. %de géométrie.

	\begin{center}
	\includegraphics[page=3, scale=1]{figures/fig-affines.pdf}
	\end{center}

L'angle $\alpha$ comme défini ci-dessus est constant et mesure la pente de la droite.

	\begin{align*}
		\tan(\alpha) = \dfrac{|y-b|}{|x|} && \iff && |y-b| = \tan(\alpha) \cdot |x|.
	\end{align*}
Comme la valeur absolue ne fait que supprimer le signe, on en déduit que
	\begin{align*}
		y - b = \pm \tan(\alpha) \cdot x && \iff && y = \pm \tan(\alpha) \cdot x + b,
	\end{align*}
où $\pm$ signifie ``plus ou moins'', c'est-à-dire qu'on ne connait pas \emph{a priori} le signe et que cela n'importe pas pour l'instant.
En posant $a = \pm \tan(\alpha)$, on a trouvé la forme de la fonction $f$ telle que $(d) = \C_f$ :
	\[ f(x) = a\cdot x + b, \]
où $a$ et $b$ sont des nombres réels déterminés entièrement par la droite $(d)$.


	
\end{multicols}

\newpage

\nt{
	De la discussion précédente, on déduit des interprétations préliminaires pour les paramètres $a$ et $b$.
		\begin{itemize}
			\item Le réel $b$ est l'ordonnée du point d'intersection de la droite avec l'axe des ordonnées.
			\item Le réel $a$ est associé à la tangente de l'angle $\alpha$ que forme la droite avec l'axe des abscisses : il mesure donc la pente de la droite. En effet, le pourcentage qu'on voit écrit sur les panneaux de signalisation de descente routière est égal la tangente de l'angle de l'inclinaison vis-à-vis de l'horizontale. Par exemple, un angle de $9$° correspond à une pente de $\tan(9) \approx 15,8\%$.
		\end{itemize}
	On verra dans la suite que le signe de $a$ correspond au caractère croissant ou décroissant de la fonction affine ; et que plus $|a|$ est grand, plus la pente est raide.
}

\section{Lecture graphique et définitions}\label{sec:aff-1}

\dfn{}{
	Un \emphindex{fonction affine} est une fonction $f$ vérifiant pour tout $x\in\R$
		\begin{align}
			f(x) = ax + b, \label{eq:affine}
		\end{align}
	où $a$ et $b$ sont deux paramètres : le \emphindex{coefficient directeur} et l'\emphindex{ordonnée à l'origine}.
}{def:affine}

\nomen{
	Une \emphindex{fonction linéaire} est une fonction affine avec $b=0$.
}

\exe{, difficulty=1}{
	Montrer que deux fonctions affines sont égales si et seulement si les coefficients directeurs sont égaux et les ordonnées à l'origines sont égales.
}{exe:f=g-affines}{
	TODO
}

\mprop{}{
	La courbe représentative d'une fonction affine est une droite non verticale.
}{prop:Cf-affine}


\ex{}{
	Considérons la fonction $id$, dite \emphindex{fonction identité}, donnée par
		\[ id(x) = x \qquad \text{pour tout $x \in \R$.} \]
	Cette fonction est affine, car c'est un cas spécial de la forme \ref{eq:affine} où $a=1$ et $b=0$.
	La droite $(d) = \C_{id}$ vérifie donc
		\begin{align*}
			(x;y) \in (d) && \iff && y = x.
		\end{align*}
		
	\begin{multicols}{2}
	Un point appartient donc à $(d)$ si et seulement si sont abscisse est égale à son ordonnée.
	
	Ainsi, $(3;3) \in (d)$ et $\bigl(-\sqrt{2}; -\sqrt{2}\bigr) \in (d)$, mais $(3;4) \notin (d)$.
	La droite restreinte à \mbox{$\D = [-4 ; 4]$} est donc l'ensemble des points en bleu ci-contre.
	
	\vfill
	
	\begin{center}
	\includegraphics[page=4]{figures/fig-affines.pdf}
	\end{center}
	
	\end{multicols}
}{}

%\ex{}{
%	Considérons la fonction $f$ donnée par
%		\[ f(x) = x+6 \qquad \text{pour tout $x \in \R$.} \]
%	Cette fonction est affine, car c'est un cas spécial de la forme \ref{eq:affine} où $a=1$ et $b=6$.
%	La droite $(d) = \C_{f}$ vérifie donc
%		\begin{align*}
%			(x;y) \in (d) && \iff && y = x+6.
%		\end{align*}
%	Ainsi, $(3;9) \in (d)$ et $\bigl(-\sqrt{2}; -\sqrt{2}+6\bigr) \in (d)$, mais $(6;0) \notin (d)$.
%	La droite est donc l'ensemble des points en bleu ci-dessous.
%	
%		\begin{center}
%	\includegraphics[page=5]{figures/fig-affines.pdf}
%	\end{center}
%}{ex:droite}
%
%\ex{}{
%	Considérons la fonction $f$ donnée par
%		\[ f(x) = 2,2 \qquad \text{pour tout $x \in \R$.} \]
%	Cette fonction est affine, car c'est un cas spécial de la forme \ref{eq:affine} où $a=0$ et $b=2,2$.
%	La droite $(d) = \C_{f}$ vérifie donc
%		\begin{align*}
%			(x;y) \in (d) && \iff && y = 2,2.
%		\end{align*}
%	Ainsi, $(3;2,2) \in (d)$ et $\bigl(-\sqrt{2}; 2,2\bigr) \in (d)$, mais $(0;0) \notin (d)$.
%	La droite est donc l'ensemble des points en bleu ci-dessous.
%	
%	\begin{center}
%	\includegraphics[page=6]{figures/fig-affines.pdf}
%	\end{center}
%}{}
%
%\ex{}{
%	Considérons la fonction $f$ donnée par
%		\[ f(x) = -2x + 1 \qquad \text{pour tout $x \in \R$.} \]
%	Cette fonction est affine, car c'est un cas spécial de la forme \ref{eq:affine} où $a=-2$ et $b=1$.
%	La droite $(d) = \C_{f}$ vérifie donc
%		\begin{align*}
%			(x;y) \in (d) && \iff && y = -2x+1.
%		\end{align*}
%	Ainsi, $(-1;3) \in (d)$ et $(-\sqrt{2}; 2\sqrt{2}+1) \in (d)$, mais $(1;1) \notin (d)$.
%	La droite est donc l'ensemble des points en bleu ci-dessous.
%	
%	\begin{center}
%	\includegraphics[page=7]{figures/fig-affines.pdf}
%	\end{center}
%}{}

\nt{
	On peut réécrire la courbe représentative de $f$ de la façon suivante.
		\[ \C_f = \bigset{ \bigl(x ; f(x) \bigr) \text{ où $x$ parcourt $\R$} }
		= \bigset{ (x ; y ) \text{ où $x$ parcourt $\R$ et où $y = f(x)$} } \]
}{}

% à mettre plus haut non ?
\dfn{équation de droite}{
	Pour une fonction affine $f(x) = ax+b$, on parle de $\C_f$ comme la « \emphindex{droite d'équation $y=ax+b$} ».
}{dfn:eq-droite}

\notations{
	Pour se défaire de la fonction $f$, on peut noter
		\[ (d) : y = ax+b. \]
}

\exe{1}{
	Donner deux points distincts appartenant à la droite d'équation $y = 2x-1$.
	Représenter la droite dans un repère.
}{exe:droite}{
	TODO
}


\nomen{
	On dit d'un point $(x, y)$ du plan qu'il est à \emphindex{coordonnées entières} dès que $x, y \in \Z$ sont deux entiers relatifs.
}

\exe{, difficulty=1}{
	On considère la fonction affine $f(x) = \sqrt2 x$.
	Donner tous les points $(x,y)$ à coordonnées entières de $\C_f$.
}{exe:sqrt2x-integer}{
	Tout d'abord, $(x,y) \in \C_f \iff y = \sqrt2 x$.
	On distingue donc deux cas : si $x=0$, on trouve le couple $(0 ; 0) \in \C_f$ à coordonnées entières.
	Sinon, $\sqrt2 = \dfrac{y}x$ pour deux entiers relatifs $x, y\in\Z$.
	Comme $\sqrt2$ est irrationnel, aucun couple ne peut vérifier cette égalité.
	
	Le seul couple est donc $(0 ; 0)$.
}

\exe{, difficulty=1}{
	On considère la fonction affine $f(x) = \frac12 x$.
	Donner tous les points $(x,y)$ à coordonnées entières de $\C_f$.
}{exe:12x-integer}{
	Tout d'abord, $(x,y) \in \C_f \iff y = \dfrac12  x \iff x = 2y$.
	On retrouve là la définition d'un nombre pair : $x$ est multiple de 2.
	
	L'ensemble recherché est donc $\bigset{ (2k, k) \text{ où } k\in\Z}$.
}

\exe{}{
	On considère la fonction affine $f(x) = \frac13 -\frac23 x$.
	Montrer que $(x,y) \in \C_f$ si et seulement si $2x + 3y = 1$.
	Montrer ensuite que tous les points de la forme $(x, y) = (-3k -1 ; 2k+1)$ où $k\in\Z$ appartiennent à $\C_f$.
}{exe:2x+3y=1-integer}{
	$(x,y) \in \C_f \iff y = f(x) = \dfrac13 -\dfrac23 x \iff 3y = 1 - 2x \iff 2x + 3y = 1$, comme requis.
	
	Pour $x=-3k-1$ et $y=2k+1$, on a bien
		\[ 2x+3y = 2(-3k-1) + 3(2k+1) = -2 + 3 = 1. \]
}

\nt{
	L'équation $2x  - 3y= 3$ étant équivalente à $y = \frac23 x - 1$, c'est aussi une équation de droite.
	La première forme est appelée \emphindex{équation cartésienne}, et la deuxième \emphindex{équation réduite}.
}

\exe{}{
	On considère l'ensemble des points $(x,y)$ vérifiants $5x - 9y = 1$.
	Exprimer cet ensemble comme la courbe représentative d'une fonction affine dont on explicitera l'expression algébrique.
	Donner un point de la droite à coordonnées entières.
}{exe:isoler-y}{
	$5x - 9y = 1 \iff 5x = 1 + 9y \iff 9y = 5x -1 \iff y = \dfrac59x - \dfrac19$.
	L'ensemble des points vérifiants cette équation est donc la courbe de $f(x) = \dfrac59 x - \dfrac19$.
	
	Soit $(x, y)$ à coordonnées entières. Alors $9y = 5x -1$.
	Ainsi $9y$ doit se terminer en $9$ ou en $4$. C'est le cas de $9$ assez immédiatement ($y=1$ et donc $x=2$), mais aussi de $54$ ($y = 6$ et donc $x=11$), par exemple.
	On a trouvé deux points à coordonnées entières : $(2 ; 1)$ et $(11 ; 6)$.
}

\exe{, difficulty=2}{
	Montrer que si $(x_0,y_0)$ est solution de $5x-9y=1$, alors $(x_0+9, y_0+5)$ aussi.
	En déduire qu'il existe une infinité de points à coordonnée entière $(x,y)$ vérifiant $5x-9y=1$, et en donner 3 distincts.
}{exe:dioph59}{
	$5(x_0+9) - 9(y_0 + 5) = 5x_0 - 9y_0 + 5\times9 - 9\times5 = 5x_0 - 9y_0 = 1$.
	La solution entière $(x_0, y_0) = (2 ; 1)$ trouvée à l'exercice \ref{exe:isoler-y} permet de conclure, car ajouter 9 et ajouter 5 ne change pas le caractère entier d'un nombre.
	
	Du couple $(2 ; 1)$ on déduit le couple $(2+9 ; 1+5) = (11 ; 6)$, puis $(11+9 ; 6 + 5 ) = (20 ; 11)$.
	On vérifiera que $20 \times 5 - 11 \times 9 = 100 - 99 = 1$ pour se rassurer.
}

\section{Identifications des paramètres}

\subsection{Ordonnée à l'origine $b$}

\lem{ordonnée à l'origine}{
	Soit $f(x) = a x + b$ affine.
	Alors
		\[ f(0) = b, \]
	et donc
		\[ (0 ; b) \in \C_f. \]
}{lem:affine-b}

\newpage
\begin{multicols}{2}
\ex{}{
	Considérons la droite $(d)$ ci-contre, d'équation $y=ax+b$.
	
	Le point de $(d)$ d'abscisse nulle admet pour coordonnées $(0 ; 2)$.
	Ainsi
		\begin{align*}
			(0 ; 2) \in (d) && \iff && 2 = a\cdot0 + b,
		\end{align*}
	Et on en déduit que le paramètre $b$ vaut $2$.
	\vfill
}{}
	\begin{center}
	\includegraphics[page=8, scale=.9]{figures/fig-affines.pdf}
	\end{center}
	
\end{multicols}

\ex{}{
	Considérons une droite $(d)$ qui contient les points
		\begin{align*}
			(0;3) \in (d) && \text{ et } && (1;8) \in (d).
		\end{align*}
	On souhaite trouver la fonction affine $f$ telle que $(d)$ soit la courbe représentative de $f$.
	La propriété fondamentale \ref{cor:prop-fond} donne donc les deux équations suivantes.
		\begin{align*}
			f(0) = 3, && \text{ et } && f(1) = 8.
		\end{align*}
	Comme $f$ est affine, on sait qu'elle s'écrit $f(x) = ax+b$ pour tout $x\in\R$ et pour certains paramètres $a,b\in\R$.
	On réécrit les équations 
		\begin{align*}
			a \times 0 + b = 3 && \text{ et } && a\times1 + b = 8, \\
			b = 3 && \text{ et } && a + b = 8.
		\end{align*}
	Par suite, $b=3$, et $a= 8-b = 8-3 = 5$. Par conséquent,
		\[ f(x) = 5x + 3 \qquad \text{ pour tout } x\in\R. \]
}{}

\subsection{Coefficient directeur $a$}

\lem{coefficient directeur}{
	Soit $f(x) = a x + b$ affine.
	Alors, pour tout $x\in\R$ réel,
		\[ f(x+1) - f(x) = a. \]
	Lorsqu'on augmente l'abscisse $x$ de 1, l'ordonnée $y$ augmente de $a$.
}{lem:coeff-dir}

%\pf{Preuve du lemme \ref{lem:coeff-dir}}{
\pf{}{
	Pour calculer $f(x+1)$, posons la variable intermédiaire $t = x+1$.
	On a alors
		\[ f(x+1) = f(t) = a\cdot t + b = a\cdot (x+1) + b = a \cdot x + a + b. \]
	En soustrayant $f(x) = ax+b$, on trouve bien le résultat recherché.
		\begin{align*}
			f(x+1) - f(x) &= ax + a + b - (ax + b) \\ &= ax + a + b -ax -b \\ &= a
		\end{align*}
}

\cor{}{
	On a, en particulier,
		$ a = f(1) - f(0).$
}{cor:affine-a}

\exe{}{
	Montrer le corollaire \ref{cor:affine-a}.
}{exe:affine-a}{
	TODO
}

\exe{}{
	Pour chaque fonction affine sur $\R$ suivante, déterminer son coefficient directeur $a$ et son ordonnée à l'origine $b$ à l'aide du lemme \ref{lem:affine-b} et du corollaire \ref{cor:affine-a}.
	\begin{multicols}{2}
	\begin{enumerate}
		\item $f(x) = 2x + 1$
		\item $f(x) = 1 + 2x$
		\item $f(x) = - x$
		\item $f(x) = -42$
		\item $f(x) = 10x + 2$
		\item $f(x) = 2 + 10x$
		\item $f(x) = 1 - x$
		\item $f(x) = 0$
	\end{enumerate}
	\end{multicols}
}{exe:paramètres-affines}{
	TODO
}

\begin{multicols}{2}
		
\ex{}{
	On considère la droite ci-contre, courbe représentative $\C_f$ d'une certaine fonction affine $f(x)=ax+b$.
	
	On regarde deux points de $\C_f$ dont les abscisses sont espacées de $1$ et on regarde comme leur ordonnée évolue.
	Par exemple, $(1;0)$ et $(2,-2)$. 
	Le lemme \ref{lem:coeff-dir} implique que, en prenant $x=1$,
		\begin{align*}
			f(2) - f(1) = a.
		\end{align*}
	On en déduit que 
		\[ a = (-2) - (0) = -2. \]
	En effet, on a augmenté $x$ de $1$, et l'image a augmenté de $-2$ (ou diminué de $2$).
	
	Remarquons qu'on aurait pû prendre d'autres points, par exemple $(0;2)$ et $(1;0)$, ou $(2;-2)$ et $(3;-4)$.
	La différence des ordonnées vaut bien $-2$, le coefficient directeur.	
}{}
	\vfill\null
	\begin{center}
	\includegraphics[page=9]{figures/fig-affines.pdf}
	\end{center}
	\vfill\null
\end{multicols}

\exe{}{
	Vérifier que les points $(1;2)$ et $(4;-4)$ appartiennent bien à la droite d'équation $y = -2x+4$.
}{exe:point-sur-droite}{
	TODO
}

\exe{}{
	Décrire entièrement la droite $(d)$ contenant les points $(3;-4)$ et $(-2;14)$.
}{exe:interpolation-affine}{
	TODO
}


\nt{
	Du lemme \ref{lem:coeff-dir}, on déduit que lorsqu'on augmente l'abscisse $x$ de 1, l'ordonnée $y$ augmente de $a$.
	
	Il suit naturellement que, lorsqu'on augmente $x$ de 2, alors $y$ augmente de $a+a = 2a$.
	De plus, lorsqu'on augmente $x$ de $-1$, alors $y$ augmente de $-a$.
	
	Plus généralement, on montre facilement que $f(x+c) - f(x) = ca$ pour n'importe que $c\in\R$.
	Si on note $x' = x+c$ on a donc, en supposant $c \neq 0$ non nul,
		\[ a = \dfrac{f(x+c)-f(x)}{c} = \dfrac{f(x') - f(x)}{x' - x}. \]
	Ceci démontre la première partie du théorème \ref{thm:param-affine}.
}

\ex{}{
	Soient $(1;2)$ et $(4;-4)$ deux points du plan qui appartiennent à une droite d'équation $y=ax+b$.
	Le but est de connaître $a$ et $b$ afin de connaître entièrement la droite passant par ces deux points.
	
	En augmentant l'abscisse de 3 unités, l'ordonnée augmente de $-6$.
	D'après la remarque ci-dessus, $a=\frac{-6}3 = -2$.
	
	L'équation de la droite est donc $y=-2x + b$.
	Comme $(1 ; 2)$ lui appartient, la propriété fondamentale donne $2 = -2\cdot1 + b = -2 + b$.
	On conclut que $b=4$.
}{ex:systeme-affine}

\exe{}{
	Montrer qu'une droite ne peut pas contenir les trois points $(0;1), (1;0),$ et $(2;1)$ simultanément.
}{exe:alignés}{
	TODO
}

\exe{,difficulty=2}{
	Décrire l'ensemble $\C_f$ contenant les points $(0;1), (1;0),$ et $(2;1)$ où $f(x) = ax^2 + bx + c$.
	Grapher la parabole, par exemple en écrivant \texttt{y=x² -2x + 1} sur GeoGebra.\footnote{\href{https://www.geogebra.org/calculator}{https://www.geogebra.org/calculator}}
	
}{exe:interpolation-quadratique}{
	TODO
}

\thm{interpolation linéaire}{
	Si deux points $(x_A, y_A)$ et $(x_B, y_B)$ vérifient $x_A\neq x_B$ et appartiennent à la droite d'équation $y=ax+b$, alors
		\begin{align*}
			a = \dfrac{y_A - y_B}{x_A - x_B} && \text{et} && b = \dfrac{x_A y_B - x_B y_A}{x_A-x_B}. 
		\end{align*}
}{thm:param-affine}

\nt{
	Un moyen mnémotechnique pour retenir la formule de $a$ est
		\[ a = \dfrac{\text{différence des $y$}}{\text{différence des $x$}} = \dfrac{y-y'}{x-x'} = \dfrac{dy}{dx}. \]
	L'ordre des différences ($A$ moins $B$ ou $B$ moins $A$) doit être le même au numérateur et au dénominateur mais, même s'il ne l'est pas, on aura seulement commis une erreur de signe.
	Il est donc possible d'ignorer l'ordre des points et de décider du signe après grâce au caractère croissant ou décroissant de la fonction.
	
	La formule pour $b$ n'a pas a être retenue car, une fois $a$ connu, une appartenance d'un point suffit à donner une équation pour trouver $b$.
}

\begin{multicols}{2}
\ex{}{
	On considère la droite \mbox{$y=ax+b$} ci-contre.
	Le théorème \ref{thm:param-affine} affirme que
		\begin{align*}
			a = \dfrac{1,5 - (-3,5)}{0,5 - (-0,5)}  = \dfrac{5}{1} = 5. 
		\end{align*}
	On aurait aussi pû appliquer le lemme \ref{lem:coeff-dir} : lorsque $x$ augmente de 1 en passant de $-0,5$ à $+0,5$, l'ordonnée augmente de $5$ en passant de $-1,5$ à $3,5$.
}{}

	\begin{center}
	\includegraphics[page=10]{figures/fig-affines.pdf}
	\end{center}
\end{multicols}

\exe{}{
	Calculer $a$ et $b$ de l'exemple \ref{ex:systeme-affine} à l'aide du théorème \ref{thm:param-affine} et comparer avec les valeurs obtenues.
}{exe:a-b-affine}{
	TODO
}

\exe{}{
	Donner l'équation de la droite contenant les points $(-1;-10)$ et $(1;30)$.
	Le point $(3 ;  60)$ appartient-il à la droite ?
}{exe:a-b-affine2}{
	TODO
}

\exe{, difficulty=1}{
	Les points $(-3 ; -6)$, $(1 ;1)$, et $(3 ; 4)$ sont-ils alignés ?
}{exe:affine-alignement}{
	TODO
}

\exe{, difficulty=2}{
	Considérons une fonction quadratique 
		\[ f(x) = ax^2 + bx + c, \]
	où $a, b, c\in\R$ sont trois paramètres réels.
	Supposons de surcroît qu'on connaisse deux zéros distincts de $f$, c'est-à-dire qu'on connaisse $\alpha, \beta\in\R$ tels que $\alpha\neq\beta$ et
		\[ f(\alpha) = f(\beta) = 0. \]
	\begin{enumerate}
		\item Montrer que la fonction $g$ donnée par
			\[ g(x) = f(x) - a (x-\alpha)(x-\beta) \qquad \text{ pour tout } x\in\R \]
		est affine.
		\item Montrer que $g$ admet deux zéros distincts.
		\item En déduire, par interpolation linéaire, que $g$ est constamment nulle et donc que
			\[ f(x) = a (x-\alpha)(x-\beta)  \qquad \text{ pour tout } x\in\R.  \]
	\end{enumerate}
}{exe:thm-fond-alg-2}{
	TODO
}


\section{Parallélisme et intersection}

Considérons deux fonctions $f$ et $g$, affines sur $\R$.
Les courbes représentatives $\C_f$ et $\C_g$ sont deux droites du plan.
En géométrie, on distingue alors trois cas.
	\begin{enumerate}
		\item Les droites sont sécantes en un unique point $P$.
		\item Les droites sont confondues (ou égales).
		\item Les droites sont parallèles non confondues.
	\end{enumerate}
Le deuxième cas correspond à la situation où les deux fonction sont égales : c'est le cas $f=g$.
Deux fonctions affines sont égales si et seulement si leurs paramètres (coefficient directeur et ordonnée à l'origine) sont égaux.

Nous étudions d'abord algébriquement le cas $1$ de l'intesection.
Les techniques de résolution d'équations linéaires nous mèneront naturellement vers des situations pour lesquelles deux droites n'ont pas de point d'intersection.

\notations{
	Lorsque plusieurs équations sont valables simultanément, leur ensemble forme un \emphindex{système d'équations}, et on les regroupe à l'aide d'une accolade.
	Par exemple, le système suivant contient deux inconnues ($a$ et $b$) et deux équations.
		\[ \begin{cases*} a = 2b + 3, \\ 2a + b = -1. \end{cases*} \]
}

\ex{}{
	Considérons deux droites
		\begin{align*}
			(d) : y = 2x+3, && \et &&
			(e) : y = -4x+8.
		\end{align*}
	On souhaite trouver les points d'intersection des deux droites, c'est-à-dire décrire l'ensemble
		\[ (d) \cap (e). \]
	Soit $P(x_P;y_P)$ appartenant à l'intersection $(d) \cap (e)$.
	Comme $P \in (d),$ et $P \in (e)$, la propriété fondamentale \ref{cor:prop-fond} nous donne donc deux équations qu'on met dans un système car elles sont valables simultanément.
		\[ \begin{cases*} y_P = 2x_P + 3, \\ y_P = -4x_P + 8. \end{cases*} \]
	Comme on a bien sûr $y_P = y_P$, on peut écrire
		\begin{align*}
			y_P &= y_P \\
			2x_P + 3 &= -4x_P + 8 \\
			6x_P &= 5 \\
			x_P &=  \dfrac56
		\end{align*}
	Et par suite $y_P = 2\cdot\frac56 + 3 = \frac{14}3$.
	
	Nous avons donc obtenu que $P\left(\frac56 ; \frac{14}3\right)$ est l'unique point qui appartient à la fois à $(d)$ et $(e)$, et donc l'unique point d'intersection des droites.
	Autrement dit, 
		\[  (d) \cap (e) = \left\{ \left(\dfrac56 ; \dfrac{14}3\right) \right\}. \]
}{}

\nt{
	On se rassurera sur le résultat trouvé en vérifiant que le point d'intersection appartient bien aux deux droites ci-dessus.
		\begin{align*}
			2x_P + 3 = 2 \cdot \dfrac56 + 3 = \dfrac{14}3 = y_P, && \et &&
			-4x_P + 8 = -4\cdot\dfrac56 + 8 = \dfrac{14}3 = y_P.
		\end{align*}
}

\exe{}{
	Donner le point d'intersection des droites $(d) : y = 3-x ,$ et $(d') : y = -2x +12.$
}{exe:intersection-droites}{
	TODO
}

\notations{
	On note $\emptyset$ l'\emphindex{ensemble vide}, l'ensemble ne contenant aucun élément.
}

\exe{}{
	Montrer que $(d) \cap (d') = \emptyset$, où $(d) : y = 2x+1$, et $(d') : y = 2x-2$.
}{exe:intersection-droites2}{
	TODO
}

\exe{, difficulty=1}{
	Montrer que le point d'intersection $P(x_P ; y_P)$ de
		\begin{align*}
			(d) : y = ax+b, && \et &&
			(e) : y = a'x+b',
		\end{align*}
	vérifie $(a-a')x_P = b'-b$.
}{exe:intersection-gen}{
	TODO
}

\qs{}{
	Pour déduire $x_P$ de l'exercice \ref{exe:intersection-gen}, nous somme tentés de diviser par $a-a'$.
	
	Est-ce toujours possible ?
	Quelles conditions sur $a$ et $a'$ faut-il poser pour s'assurer qu'il existe un unique point d'intersection ?
}

\exe{, difficulty=1}{
	Considérons deux droites partageant le même coefficient directeur $a$.
		\begin{align*}
			(d) : y = ax+b, && \et &&
			(d') : y = ax+b'.
		\end{align*}
	Montrer que 
		\begin{multicols}{2}
		\begin{enumerate}[label=\roman*)]
			\item
			si $b\neq b'$, alors $(d)\cap(d') = \emptyset$ ; et
			\item
			si $b = b'$,  alors $(d)\cap(d') = (d) = (d')$.
		\end{enumerate}
		\end{multicols}
}{exe:parallélisme}{
	TODO
}

\thm{}{
	Considérons deux droites 
		\begin{align*}
			(d) : y = ax+b, && \et &&
			(d') : y = a'x+b',
		\end{align*}
	où $a, a', b, b' \in \R$ sont les paramètres des droites.
	
	On distingue alors les cas suivants sur l'intersection $(d) \cap (d')$.
		\begin{enumerate}
			\item Si $a \neq a'$, les droites $(d)$ et $(d')$ sont sécantes en un unique point d'intersection.
			\item Sinon $a = a'$, les droites $(d)$ et $(d')$ sont parallèles et on distingue deux sous-cas.
				\begin{enumerate}[label=\roman*)]
					\item Si $b\neq b'$, il n'existe aucun point d'intersection, $(d) \cap (d') = \emptyset$.
					\item Si $b=b'$, les droites sont confondues, $(d) \cap (d') = (d) = (d')$.
				\end{enumerate}
		\end{enumerate}
}{thm:parallélisme-intersection}


\exe{, difficulty=1}{
    Soit $f(x) = ax+b$ et $g(x) = a'x + b'$ pour tout $x \in \R$ deux fonctions affines de paramètres $a, a', b, b' \in\R$.

    Montrer que si $a\neq a'$, alors
        \[ \C_f \cap \C_g = \left\{ \left(\ \dfrac{b-b'}{a-a'} ; \dfrac{ab' - ba'}{a-a'} \right) \right\}. \]
}{exe:intersection-affine}{
	TODO
}

\exe{, difficulty=1}{
    Soit $f(x) = ax + b$ une fonction affine et $P(x_P;y_P)$ un point du plan.

    \begin{enumerate}
        \item 
        Montrer que la fonction affine $g$ donnée algébriquement par
            \[ g(x) = a(x-x_P) + y_P \]
        vérifie que $\C_f$ est parallèle à $\C_g$ et que $P \in \C_g$.
        \item
        Montrer que si $P\in\C_f$, alors $f=g$.
    \end{enumerate}
}{exe:parallélisme-affine}{
	TODO
}



\begin{multicols}{2}
\ex{}{
	Soit $(d) : y = 2x+3$ une droite.
	On souhaite déterminer une droite parallèle à $(d)$ et passant par le point $(2;1)$.
	Posons $(d') : y=ax+b$ cette droite et trouvons $a$ et $b$.
	
	Premièrement, le parallélisme à $(d)$ se traduit par l'égalité des coefficients directeurs :
		$ a = 2. $
	On a donc $(d') = y = 2x + b$.
	
	Deuxièmement, l'appartenance du point $(2;1)$ donne
		$ 1 = 4 + b,$
	d'où $b= 1-4 = -3$.
	
	En conclusion, $(d') : y = 2x - 3$ est la droite recherchée (en violet ci-contre).		
}{}

	\begin{center}
	\includegraphics[page=11]{figures/fig-affines.pdf}
	\end{center}
\end{multicols}

\exe{}{
	Pour chacun des couples de point $P$ et fonction affine $f$ sur $\R$, trouver la fonction affine $g$ telle que $\C_f$ et $\C_g$ soient parallèles et que $P \in \C_g$.	
	\begin{multicols}{2}
	\begin{enumerate}
		\item $P=(1;2)$ et $f(x) = 2x+1$.
		\item $P= (-2; 4)$ et $f(x) = 7-x$.
		\item $P=(-4^{12};-1,7)$ et $f(x) = 1600$.
		\item $P=(21;-50,5)$ et $f(x) = -3x + 5$.
	\end{enumerate}
	\end{multicols}
}{exe:parallélisme}{
	On pose $a, b\in\R$ les paramètres de la fonction affine $g(x)=ax+b$, $x\in\R$.
	\begin{enumerate}
		\item 
			L'information de parallélisme donne immédiatement $a=2$.
			D'où $g(x) = 2x+b$ pour tout $x\in\R$.
			
			Reste à trouver $b$ avec l'information d'appartenance.
			Celle-ci est équivalente à la contrainte
				\begin{align*}
					2 &= g(1) \\
					2 &= 2\cdot1 + b \\
					b &= 0
				\end{align*}
			D'où $g(x) = 2x$ ($x\in\R$) est la fonction affine recherchée.
		
		
		\item 
			L'information de parallélisme donne immédiatement $a=-1$.
			D'où $g(x) = -x+b$ pour tout $x\in\R$.
			
			Reste à trouver $b$ avec l'information d'appartenance.
			Celle-ci est équivalente à la contrainte
				\begin{align*}
					2 &= g(1) \\
					2 &= -1 + b \\
					b &= 3
				\end{align*}
			D'où $g(x) = -x+3$ ($x\in\R$) est la fonction affine recherchée.
			
		\item 
			L'information de parallélisme donne immédiatement $a=0$.
			D'où $g(x) =b$ pour tout $x\in\R$.
			
			Reste à trouver $b$ avec l'information d'appartenance.
			Celle-ci est équivalente à la contrainte
				\begin{align*}
					2 &= g(1) \\
					2 &=  b
				\end{align*}
			D'où $g(x) = 2$ ($x\in\R$) est la fonction affine recherchée.
			
		\item 
			L'information de parallélisme donne immédiatement $a=-3$.
			D'où $g(x) = -3x+b$ pour tout $x\in\R$.
			
			Reste à trouver $b$ avec l'information d'appartenance.
			Celle-ci est équivalente à la contrainte
				\begin{align*}
					2 &= g(1) \\
					2 &= -3\cdot1 + b \\
					b &= 5
				\end{align*}
			D'où $g(x) = -3x+5$ ($x\in\R$) est la fonction affine recherchée.
	\end{enumerate}
}


\exe{, difficulty=1}{
    Soient $f(x) = 3x^2 + 17x - 11$ et $g(x) = 2x^2 + 17x - 10$ deux fonctions quadratiques.

    Déterminer entièrement $\C_f \cap \C_g$.
}{exe:f-cap-g-deg-2}{
	TODO
}

\exe{, difficulty=2}{
    Soient $f(x) = x$ et $g(x) = x^3 - 3x^2 + 4x - 1$ deux fonctions polynomiales.

    \begin{enumerate}
        \item 
        Montrer que $(x-1)^3 = x^3 - 3x^2 + 3x - 1$ pour tout $x\in\R$.
        \item
        Déterminer entièrement $\C_f \cap \C_g$.
        \item
        Créer une fonction polynomiale $h$ de degré $4$ telle que $\C_f \cap \C_h = \bigset{ (1;1) }.$
    \end{enumerate}
}{exe:f-cap-g-deg-3-5}{
	TODO
}

\exe{, difficulty=2}{
    Soient $f(x) = -2x^2 + 7x + 2$ et $g(x) = -3x^2 + 2x +16$ deux fonctions quadratiques.
    
    \begin{enumerate}
        \item 
        Déterminer l'autre solution de $f(x)-g(x)=0$, sachant que $x=2$ en est une et donc que $f(x)-g(x) = (x-2)h(x)$, où $h$ est affine.
        \item
        Déterminer entièrement $\C_f \cap \C_g$.
        \item
        Créer une fonction polynomiale $F$ de degré $2$ telle que $\C_f \cap \C_F = \bigset{ (2;8), (-1; -7) }.$
    \end{enumerate}
}{exe:f-cap-g-deg-2bis}{
	TODO
}
% un peu beaucoup
% mettre au chapitre vecteurs ? y'a beaucoup de contenu chez les vecteurs aussi...
% jspo où le mettre
\section{Systèmes linéaires $2\times2$}

\dfn{équation cartésienne de droite}{
	Une \emphindex{équation cartésienne de droite} est une équation de la forme
		\begin{align}
			(d) : ax + by = c. \label{eq:cart}
		\end{align}
	La droite $(d)$ peut être verticale.
}{dfn:equation-cartesienne}

\nomen{
	Une \emphindex{heuristique} est une règle pragmatique qui fonctionne la plupart du temps.
	Elle donne, par exemple, une idée de stratégie de recherche, ou d'aspect d'une solution, sans être toujours optimale, ou vraie.
}

\heur{
	En voyant un système d'équations comme certaines contraintes imposées aux variables libres, on obtient l'heuristique
		\begin{center}
			( degrés de liberté ) = ( nombre de variables ) -- ( nombre de contraintes )
		\end{center}
	Le degré de liberté donne la forme des solutions : aucun degré de liberté correspond à une unique solution ; un degré de liberté correspond à une droite ; deux degrés à un plan ; etc...
	
	L'équation $y = x + 1$, à deux variables et une contrainte, représente une droite.
	
	Le système à deux variables et deux contraintes
		\[ \begin{cases*} a = 2b + 3, \\ 2a + b = -1. \end{cases*} \]
	n'a aucun degré de liberté : il n'y a \emph{probablement} qu'une seule solution.
}

\exe{}{
	Tracer la droite d'équation $4x - y = 0$ dans un repère.
}{exe:cart1}{
	TODO
}

\exe{}{
	Tracer la droite d'équation $y = 7$ dans un repère.
}{exe:cart2}{
	TODO
}

\exe{}{
	Tracer la droite d'équation $x=-2$ dans un repère.
}{exe:cart3}{
	TODO
}

\exe{}{
	Tracer les droites d'équations $2x - y = 0$, $2x - y = -2$, et $2x - y = 3$ dans un même repère.
	Que remarque-t-on ?
}{exe:parallélisme}{
	TODO
}

\exe{, difficulty = 1}{
	Montrer que les droites d'équation $ax+by =0$ et $ax+by=c$ sont parallèles, quel que soit $c\in\R$.
}{exe:parallélisme2}{
	Si $b=0$, les droites sont verticales et parallèles.
	
	Sinon, leur équations réduites sont $y = \dfrac{-a}{b}x$ et $y = \dfrac{-a}{b}x + \dfrac{c}b$, de même coefficient directeur.
	Les droites sont donc parallèles dans ce cas aussi.
}

\exe{, difficulty=1}{
	Pour chaque paire d'équations, ajouter entre elles le symbole
		\begin{enumerate}[itemindent=1.5cm]
			\item[$\implies$] si la première équation implique la seconde ;
			\item[$\impliedby$] si la deuxième équation implique la première ;
			\item[$\iff$] si les deux équations sont équivalentes ;
			\item[$\centernot\iff$] si les deux équations sont indépendantes.
		\end{enumerate}
	Spécifier l'opération appliquée pour obtenir la deuxième équation à partir de la première le cas échéant.
	
	Par abus de notation et pour gagner du temps, l'absence de symbole signifie en général que les équations sont équivalentes, mais il faut toujours faire attention à ce que cela soit bien le cas !
	
	\begin{center}
	\begin{tabular}{ccc|c|c}
		Équation 1 & Symbole & Équation 2 & Opération & \thead{Les équations sont-\\ elles équivalentes ?} \\ \hline
		$3x + 2y = 1$ &  & $6x  + 4y = 2$ & &  \\ \hline
		$-x + 2y = 1$ &  & $-x + 2y - 1 = 0$ & &  \\ \hline
		$3x + 2y = 1$ & & $-3x - 5y = -3$ & &  \\ \hline
		$x^2 = -3x$ &  & $x = -3$ & &  \\ \hline
		$-10x - y = 3$ &  & $0=0$ & &  \\ \hline
		$x-y= 3$ & & $-x + y = -3$ & &  \\ \hline
		$x -y = 0$ & & $x=y$ & &  \\ \hline
		$-x + 2y = 3$ & & $0=0$ & &  \\ \hline
		$2x + 3y + 4 = 2x + 3y + 5$ & & $1=0$ & &  \\ \hline
		$-x - y = 1$ & & $-x - y = 1$ & &  \\ \hline
		$0=0$ & & $23x - 10y = 6$ & &  \\ \hline
		$x + 2y = 3$ & & $x^2 + 2yx = 3x$ & &  \\ \hline
		$x-3y  =0$ & & $-2x+6y = 0$ & &  \\ \hline
		$-2x + 3y = -3$ & & $4x - 6y = -6$ & &  \\ \hline
		$x^2 = x$ & & $x = 1$ & &  \\ \hline
		$x = 3$ & & $x^2 = 9$ & &  \\ \hline
	\end{tabular}
	\end{center}
}{exe:systemes1}{
	\begin{center}
	\begin{tabular}{ccc|c|c}
		Équation 1 & Symbole & Équation 2 & Opération & \thead{Les équations sont-\\ elles équivalentes ?} \\ \hline
		$3x + 2y = 1$ & {$\iff$} & $6x  + 4y = 2$ & {$\times2$} & {Oui} \\ \hline
		$-x + 2y = 1$ & {$\iff$} & $-x + 2y - 1 = 0$ & {$-1$} & {Oui} \\ \hline
		$3x + 2y = 1$ & {$\centernot\iff$} & $-3x - 5y = -3$ & & {Non} \\ \hline
		$x^2 = -3x$ & {$\impliedby$} & $x = -3$ & & {Non} \\ \hline
		$-10x - y = 3$ & {$\implies$} & $0=0$ & {$\times0$} & {Non} \\ \hline
		$x-y= 3$ & {$\iff$} & $-x + y = -3$ & {$\times(-1)$} & {Oui} \\ \hline
		$x -y = 0$ &{$\iff$} & $x=y$ & {$+y$} & {Oui} \\ \hline
		$-x + 2y = 3$ & {$\implies$} & $0=0$ & {$\times0$} & {Non} \\ \hline
		$2x + 3y + 4 = 2x + 3y + 5$ & {$\iff$} & $1=0$ & {$+(-2x-3y)$} & {Oui} \\ \hline
		$-x - y = 1$ & {$\centernot\iff$} & $-x - y = 1$ & & {Non} \\ \hline
		$0=0$ & {$\impliedby$} & $23x - 10y = 6$ & & {Non} \\ \hline
		$x + 2y = 3$ & {$\implies$} & $x^2 + 2yx = 3x$ & {$\times x$} & {Non} \\ \hline
		$x-3y  =0$ & {$\iff$} & $-2x+6y = 0$ & {$\times(-2)$} & {Oui} \\ \hline
		$-2x + 3y = -3$ & {$\centernot\iff$} & $4x - 6y = -6$ & & {Non} \\ \hline
		$x^2 = x$ & {$\impliedby$} & $x = 1$ & & {Non} \\ \hline
		$x = 3$ & {$\implies$} & $x^2 = 9$ & {Mise au carré} & {Non} \\ \hline
	\end{tabular}
	\end{center}
	
}

\exe{}{
	Pour chaque système d'équations d'inconnues $x, y \in \R$, donner un système équivalent tel que les coefficients multipliant $x$ soient opposés (c'est-à-dire leur somme soit nulle).
	\setlength{\columnsep}{1cm}
	\begin{multicols}{3}
	\begin{enumerate}[label=\roman*), itemsep=20pt]
		\item $\systeme{x + 3y = 3{,}, -2x + 2y = 2.}$
		\item $\systeme{x - 7y = 1{,}, -3x + 5y = 13.}$
		\item $\systeme{2x - y = 3{,}, -x + 2y = -5.}$
		\item $\systeme{4x  - y = -4{,},  x + 5y = 2.}$
		\item $\systeme{-2x - 2y = -5{,}, 3x + 7y = 10.}$
		\item $\systeme{7x - y = 10{,}, 5x + 2y = 1.}$
	\end{enumerate}
	\end{multicols}
}{exe:systemes2}{
	\setlength{\columnsep}{1cm}
	\begin{multicols}{2}
	\begin{enumerate}[label=\roman*), itemsep=20pt]
		\item $\systeme{x + 3y = 3{,}, -2x + 2y = 2.} \iff \systeme{2x+6y=6{,},-2x+2y=2.}$
		\item $\systeme{x - 7y = 1{,}, -3x + 5y = 13.} \iff \systeme{3x - 21y = 3{,}, -3x + 5y = 13.}$
		\item $\systeme{2x - y = 3{,}, -x + 2y = -5.} \iff \systeme{2x - y = 3{,}, -2x + 4y = -10.}$
		\item $\systeme{4x  - y = -4{,},  x + 5y = 2.} \iff  \systeme{4x  - y = -4{,},  -4x - 20y = -8.}$
		\item $\systeme{-2x - 2y = -5{,}, 3x + 7y = 10.} \iff\systeme{-6x - 6y = -15{,}, 6x + 14y = 20.}$
		\item $\systeme{7x - y = 10{,}, 5x + 2y = 1.} \iff \systeme{35x - 5y = 50{,}, -35x - 14y = -7.}$
	\end{enumerate}
	\end{multicols}
}

\exe{}{
	Pour chaque système de l'exercice \ref{exe:systemes2}, 
		\begin{enumerate}
			\item combiner les équations pour trouver $y$ ; 
			\item trouver $x$ ; 
			\item vérifier que le couple $(x ;y)$ obtenu soit bien solution du système initial.
		\end{enumerate}
}{exe:systemes3}{ \, \\
	\begin{enumerate}[label=Système \roman*) :, itemsep=20pt, leftmargin=80pt]
		\item La somme des deux équations donne
			\[ 8y = 8 \iff y = 1. \]
		En remplaçant $y$ par $1$ dans la première équation, on trouve $x + 3 = 3$, et donc $x =0$.
			\[ (x ; y ) = (0; 1). \]
		
		\item La somme des deux équations donne
			\[ -16y = 16 \iff y = -1. \]
		En remplaçant $y$ par $-1$ dans la première équation, on trouve $x + 7 = 1$, et donc $x =-6$.
			\[ (x ; y ) = (-6; -1). \]
		
		\item La somme des deux équations donne
			\[ 3y = -7 \iff y = -\dfrac73. \]
		En remplaçant $y$ par $-\dfrac73$ dans la première équation, on trouve $2x +\dfrac73 = 3$, et donc $x =\dfrac13$.
			\[ (x ; y ) = \left(\dfrac13; -\dfrac73 \right). \]
		
		\item La somme des deux équations donne
			\[ -21y = -12 \iff y = \dfrac{12}{21} = \dfrac47. \]
		En remplaçant $y$ par $ \dfrac47$ dans la deuxième équation, on trouve $x +\dfrac{20}7 = 2$, et donc $x =-\dfrac67$.
			\[ (x ; y ) = \left(-\dfrac67; \dfrac47 \right). \]
		
		\item La somme des deux équations donne
			\[ 8y = 5 \iff y = \dfrac58. \]
		En remplaçant $y$ par $\dfrac58$ dans la deuxième équation, on trouve $3x +\dfrac{35}8 = 10 \iff x =\dfrac{45}{24} = \dfrac{15}8$.
			\[ (x ; y ) = \left(\dfrac{15}8; \dfrac58 \right). \]
		
		\item La somme des deux équations donne
			\[ -19y = 43 \iff y = -\dfrac{43}{19}. \]
		En remplaçant $y$ par $-\dfrac{43}{19}$ dans la première équation, on trouve $7x +\dfrac{43}{19}= 10 \iff x =\dfrac{147}{133} = \dfrac{21}{19}$.
			\[ (x ; y ) = \left( \dfrac{21}{19};  -\dfrac{43}{19} \right). \]
	\end{enumerate}
}

\exe{}{
	Donner les solutions réelles $(x;y)$ des systèmes suivantes.
	
	\begin{multicols}{3}
	\begin{enumerate}[label=\roman*), itemsep=20pt]
		\item $\sys{2x+4y=0}{-2x -2y = 2}$
		\item $\sys{-6x+3y=-3}{x -3y = -2}$
		\item $\sys{x+y = 2}{x+y=2}$
		\item $\sys{8x-4y = 6+2x-y}{7x+12y=-5+y}$
		\item $\sys{x-y = 1}{-x+y=10}$
		\item $\sys{\frac12 x - \frac23 y= -1}{\frac15 x + \frac72 y= 5}$
		\item $\sys{2x - 8y = 2}{-4x+16y=-1}$
		\item $\sys{5x + y = -2x-1-y}{8x = -1+2y}$
		\item $\systeme[yx]{2y + 12x = -3{,}, {-6}x=y + \frac32.}$
		%\item $\systeme[tz]{3t = -3-z{,}, t + 2z = 8.}$
	\end{enumerate}
	\end{multicols}
}{exe:systemes4}{\, \\
	
	\begin{enumerate}[label=\roman*), itemsep=20pt]
		\item $(x ; y) = (-2 ; 1)$
		\item $(x ; y) = (1;1)$
		\item Les deux équations sont redondantes : on a en fait qu'une seule contrainte pour deux variables et donc un degré liberté. Pour chaque choix de $x$, on peut trouver un $y$ tel que $(x;y)$ soit solution.
		L'ensemble des solutions est
			\[ \{ (x ; y) \text{ tq. } x+y=2, x, y \in \R \} = \{ (x ; y) \text{ tq. } y = -x + 2, \text{ où $x$ parcourt $\R$} \} = \C_f, \]
		où $f(x) = -x+2$.
		\item $(x ; y) = \left(\frac{17}{29} ; -\frac{24}{29}\right)$
		\item Les deux équations sont contradictoires et aucune solution n'existe.
		\item $(x ; y) = \left(-\frac{10}{113} ; \frac{162}{113}\right)$
		\item Les deux équations sont contradictoires et aucune solution n'existe.
		\item $(x ; y) = \left(-\frac{2}{15} ; -\frac{1}{30}\right)$
		\item Les deux équations sont redondantes : on a en fait qu'une seule contrainte pour deux variables et donc un degré liberté. Pour chaque choix de $x$, on peut trouver un $y$ tel que $(x;y)$ soit solution.
		L'ensemble des solutions est
			\[ \left\{ (x ; y) \text{ tq. } -6x -y = \dfrac32, x, y \in \R \right\} = \left\{ (x ; y) \tq y= -6x - \dfrac32 \text{ où $x$ parcourt $\R$} \right\} = \C_f, \]
		où $f(x) =-6x - \frac32$.
	\end{enumerate}
}


\exe{, difficulty=1}{
	On considère le système de trois équations à trois inconnues $x, y, z \in \R$ suivant.
		\[ \begin{cases} -x-2y+2z &= -2, \\ 3x - 5y + 2z &= 5, \\ 2x - 2y + 2z &= 10. \end{cases} \]
	\begin{enumerate}
		\item Ajouter aux équations 2 et 3 un multiple de la première équation pour annuler leur coefficient en $x$.
		\item Écrire le système équivalent obtenu et vérifier que les deux dernières équations forment le sous-système de deux équations à deux inconnues suivant.
		\[ \begin{cases} - 11y + 8z &= -1, \\ - 6y + 6z &= 6. \end{cases} \]
		\item Résoudre ce sous-système pour trouver $y=3$ et $z=4$.
		\item En déduire que $x=4$.
		\item Vérifier que $(x ; y ; z) = (4 ; 3 ;4)$ est bien solution du système initial.
	\end{enumerate}
}{exe:systemes5}{
	TODO
}

% TODO
%\exe{, difficulty=2}{
%	On considère le système de quatre équations à quatre inconnues réelles, $x, y, z,$ et $t$.
%		\[ \begin{cases} x+2y-2z &= 2, \\ 3x - 5y + 2z &= 5, \\ 2x - 2y + 2z &= 10. \end{cases} \]
%	\begin{enumerate}
%		\item Comme à l'exercice \ref{exe:systemes5}, annuler tous les coefficients en $x$ des trois dernières équations.
%		\item Écrire le sous-système de trois équations à trois inconnues $y, z,$ et $t$.
%		\item Résoudre le système comme à l'exercice précédent pour trouver $y, z,$ et $t$.
%		\item Trouver $x$.
%		\item Vérifier que les valeurs obtenues
%	\end{enumerate}
%}{exe:systemes5.5}{}

\exe{, difficulty=1}{
	Donner un système, le plus simple possible, de deux équations linéaires à inconnues $x, y \in \R$ qui admet uniquement $(x ; y) = (2 ; 5)$ comme solution.
	Modifier le système pour qu'aucun coefficient multipliant $x$ ou $y$ ne soit nul.
}{exe:systemes6}{
	TODO
}

\notations{
	On note $\pm$ pour signifier « plus ou moins ». Par exemple, $\pm3 = 3$ ou $-3$.
}

\exe{, difficulty=1}{
	En sommant les équations, montrer que le système suivant d'inconnues $x, y \in \R$ admet exactement quatre solutions : $(x ;y) = (\pm 2 ; \pm 3)$.
	\[ \begin{cases} x^2 - y^2 &= -5, \\ x^2 + y^2 &= 13. \end{cases} \]

	Donner un système de deux équations non linéaires à inconnues $x, y \in \R$ admettant exactement quatre solutions : $(\pm 3; \pm4)$.
}{exe:systemes7}{
	TODO
}

\exe{, difficulty=1}{
	Montrer que, pour tout $y\in\R$,
		\[ (y+1)^2 + y^2 - 13 = 2(y-2)(y+3). \]

	En combinant les équations, montrer que le système suivant d'inconnues $x, y \in \R$ admet exactement trois solutions : $(x ;y) = (0; 2),$ et $(x ;y) = \left( \pm \frac12 \sqrt{10} ; -3\right)$.
	\[ \begin{cases} 2x^2 + (y+1)^2 &= 9, \\ 2x^2 - y^2 &=-4. \end{cases} \]
}{exe:systemes8}{
	TODO
}

\exe{, difficulty=2}{
	On considère le système suivant qu'on ne cherche pas à résoudre.
		\[ \begin{cases} 3x - 2y &= 7, \\ -x - 3y &= -10. \end{cases} \]
	Supposons qu'on ait deux paires de solutions : $(x ; y)$ et $(x' ; y')$.
	
	Montrer que la différence $(a; b) = (x - x' ; y - y')$ est solution du système 
		\[ \begin{cases} 3a - 2b &= 0, \\ -a - 3b &= 0. \end{cases} \]
	On appelle ce système \emphindex{système homogène} car les coefficients constants à droite sont tous nuls.
	
	Montrer que $(a ; b) = (0 ; 0)$, et en déduire qu'on a nécessairement $(x ; y) = (x' ; y')$ : le système initial admet au plus une solution.
	
	Montrer plus généralement que n'importe quel système linéaire admet au plus une solution dès que l'unique solution du système homogène est la solution nulle.
}{exe:systemes9}{
	TODO
}
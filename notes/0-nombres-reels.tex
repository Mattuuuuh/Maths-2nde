%!TEX encoding = UTF8
%!TEX root =0-notes.tex

\chapter{Nombres rationnels, nombres réels}

\section{Ensembles de nombres}

\notations{
	Un collection \emph{non ordonnée} d'éléments est un \emph{ensemble}, noté avec des accolades et des points-virgules entre les éléments.
	
		\[ E = \bigset{ 2 ; 3 ; 5 ; 7 ; 11 }. \]
	
	Un ensemble peut contenir un nombre infini d'éléments.
	
		\[ F = \bigset{ 0 ; 2 ; 4 ; 6 ; 8 ; 10 ; \dots }. \]
}

\exe{}{
	Donner l'ensemble $A$ des entiers supérieurs ou égaux à 0 et inférieurs ou égaux à 7.
	
	Donner l'ensemble $B$ des éléments de $A$ qui sont pairs.
}{exe:set1}{
	\begin{multicols}{2}
		$A = \bigset{ 0 ; 1 ; 2 ; 3 ; 4 ; 5 ; 6 ; 7}$
		
		$B = \bigset{ 0 ; 2 ; 4 ; 6 }$
	\end{multicols}
}

\dfn{Nombres entiers}{
	On pose les nombres entiers \emph{naturels}
		\[ \N =  \bigset{ 0 ; 1 ; 2 ; \dots }, \]
	et les nombres entiers \emph{relatifs}
		\[ \Z =  \bigset{ \dots ; -2 ; -1 ; 0 ; 1 ; 2 ; \dots }. \]
}{dfn:nombres-entiers}
	
	
\notations{
	Il est aussi possible de décrire l'ensemble $F$
		\[ F =  \bigset{ 0 ; 2 ; 4 ; 6 ; 8 ; 10 ; \dots } \]
	sous la forme
		\[ F =  \bigset{ n \in \N \tqs \text{$n$ est pair} }. \]
	La deuxième forme est plus précise et évite au lecteur d'avoir à deviner la structure de l'ensemble.
}

\exe{}{
	Décrire l'ensemble $A$ de l'exercice \ref{exe:set1} sous les formes $\bigset{ n \in \N \tq \dots }$ et $\bigset{ n \in \Z \tq \dots }$.
	
	Décrire l'ensemble $B$ de l'exercice \ref{exe:set1} sous la forme $\bigset{ n \in A \tq \dots }$.
}{exe:set2}{
	\begin{multicols}{2}
		$A = \bigset{ n \in \N \tq n \leq 7}$
		
		$A = \bigset{ n \in \Z \tq 0 \leq n \leq 7}$
		
		$B = \bigset{ n \in A \tq \text{$n$ est pair}}$
	\end{multicols}
}

\dfn{Nombres rationnels}{
	Soit $a, b \in \Z$ deux entiers relatifs, avec $b$ non nul.
	
	Alors le \emph{rationnel} $\dfrac{a}{b}$ est l'unique nombre tel que
		\[ \dfrac{a}b \times b = a. \]
	
	On note l'ensemble de tous les rationnels
		\[ \Q = \Bigset{ \dfrac{a}{b} \tq a \in \Z, b \in \Z, b \neq 0}. \]
}{dfn:nombres-rationnels}

\ex{}{
	Le nombre rationnel $\dfrac12$ est l'unique nombre vérifiant
		\[ \dfrac12 \times 2 = 1. \]
	Comme $0,5 \times 2 = 1$, on a $\dfrac12  = 0,5$. C'est le \emph{développement décimal} de $\dfrac12$.
}{ex:nombres-rationnels1}

\ex{}{
	Le nombre rationnel $\dfrac13$ est l'unique nombre vérifiant
		\[ \dfrac13 \times 3 = 1. \]
}{ex:nombres-rationnels2}

\exe{}{
	Montrer que $\dfrac13 = \dfrac26$.
}{exe:reduction-fraction}{
	Par définition, $\dfrac13$ est l'unique $x\in\Q$ vérifiant $3x = 1$.
	Cependant, on a l'équivalence suivante :
		\[ 3x = 1 \iff 6x = 2. \]
	$x$ est donc égal à $\dfrac26$, par définition.
}

\exe{}{
	Montrer que $\dfrac{a}{a} = 1$ pour n'importe que $a\in\Z$ non nul.
}{exe:a-div-a}{
	Par définition, $\dfrac{a}{a}$ est l'unique $x\in\Q$ vérifiant $ax = a$.
	Comme $x=1$ vérifie cette équation, l'unicité conclut.
}

\exe{, difficulty=1}{
	Montrer que $\dfrac12 + \dfrac13 = \dfrac56$.
}{exe:somme-rationnels}{
	Posons $x = \dfrac12 + \dfrac13$.
	On a alors
		\begin{align*}
			6x &= 6 \times \dfrac12 + 6 \times \dfrac13, \\
				&= 3(2 \times \dfrac12) + 2(3 \times \dfrac13), \\
				&= 3 + 2 = 5.
		\end{align*}
	On conclut que $x = \dfrac56$.
}

\exe{, difficulty=1}{
	Montrer que $\dfrac12 \times \dfrac23 = \dfrac26$.
}{exe:produit-rationnels}{
	Posons $x = \dfrac12 \times \dfrac23$.
	On a alors, par associativité,
		\begin{align*}
			6x &= 6 \times \left(\dfrac12 \times \dfrac23\right), \\
				&= (2\times3) \times \left(\dfrac12 \times \dfrac23\right), \\
				&= \left(2\times\dfrac12\right)\times \left(3\times\dfrac23\right), \\
				&= 1 \times 2 = 2.
		\end{align*}
	On conclut que $x = \dfrac26$.
}

\thm{}{
	On a les identités suivantes, pour $a, b, c, d \in \Z$, non nuls lorsques dénominateurs.
	
	\begin{align*}
		\dfrac{a}{b} \cdot \dfrac{c}{d} = \dfrac{ac}{bd}
		&& \text{ et } &&
		\dfrac{a}{b} + \dfrac{c}{d} = \dfrac{ad + bc}{bd}.
	\end{align*}
}{thm:stabilité-Q}

\nomen{
	On appelle « corollaire » une proposition qui découle immédiatement d'un théorème mais qui mérite d'être énoncée seule.
	Le corollaire couronne ainsi le théorème.
}

\cor{}{
	Pour $a, b, c \in \Z$, non nuls lorsques dénominateurs, on a 
		\[ \dfrac{a}{b} = \dfrac{ac}{bc}. \]
}{cor:simplification-Q}

\exe{, difficulty=2}{
	Démontrer le théorème \ref{thm:stabilité-Q}.
}{exe:stabilité-Q}{
	Soit $r = \dfrac{a}{b}$ et $s = \dfrac{c}{d}$.
	Alors
		\begin{align*}
			br = a, && \text{ et } && ds = c.
		\end{align*}
	En multipliant la première équation par $d$ et la deuxième par $b$, on pourra factoriser et obtenir une équation pour $r+s$ :
		\begin{align*}
			(db)r = da, && \text{ et } && (bd)s = bc,
		\end{align*}
	implique $db(r + s) = da + bc$.
	Il suit que $r+s = \dfrac{ad + bc}{bd} \in \Q$.
	
	Similairement, en multipliant les équations, on obtient $(br)(ds)=ac \iff (bd)(rs) = ac$, et donc $r \cdot s = \dfrac{ac}{bd} \in \Q$.
}

\qs{}{
	Est-il possible d'écrire le développement décimal de $\dfrac13$ sur un feuille de papier A4 ?
	
	Le but de la prochaine section est de répondre à cette question par la négative.
}

\section{Développements décimaux}

\dfn{Nombres décimaux}{
	On dit d'un nombre $x$ qu'il est \emph{décimal} si on peut écrire son développement décimal sur un feuille de papier A4 (possiblement en écrivant très petit).
	Autrement dit, son développement décimal est fini.
	
	On note 
		\[ \D = \bigset{ x \in \Q \tq \text{ le développement décimal de $x$ est fini} }. \]
}{dfn:nombres-décimaux}

\exe{}{
	Sans calculatrice, donner la valeur numérique des fractions suivantes.
	\begin{multicols}{3}
	\begin{enumerate}[label=\roman*), leftmargin=60pt]
		\item $\dfrac1{10}$
		\item $\dfrac1{5}$
		\item $\dfrac3{10^{5}}$
		\item $\dfrac7{20}$
		\item $\dfrac{395}{50}$
		\item $\dfrac{11}{200}$
	\end{enumerate}
	\end{multicols}
}{exe:dev-decimaux}{
	\begin{multicols}{3}
	\begin{enumerate}[label=\roman*)]
		\item $\dfrac1{10} = 0,1.$
		\item $\dfrac1{5} = \dfrac2{10} = 0,2.$
		\item $\dfrac3{10^{5}} = 0,00003.$
		\item $\dfrac7{20} = \dfrac{3,5}{10} = 0,35.$
		\item $\dfrac{395}{50} = \dfrac{790}{100} = 7,9.$
		\item $\dfrac{11}{200} = \dfrac{5,5}{100} = 0,055.$
	\end{enumerate}
	\end{multicols}
}

\thm{Caractérisation de $\D$}{
	On a 
		\[ \D = \Bigset{ \dfrac{a}{10^n} \tq a \in \Z, n \in \N }. \]
}{thm:caractérisation-D}

\pf{}{
	Posons $E = \Bigset{ \dfrac{a}{10^n} \tq a \in \Z, n \in \N }$.
	Pour montrer l'égalité d'ensembles $\D = E$, il faut toujours montrer la double inclusion $\D \subseteq E$ et $E \subseteq \D$.
	Pour montrer une inclusion $\subseteq$ d'ensembles, il faut montrer que n'importe quel élément de l'ensemble de gauche (le plus petit) appartient à l'ensemble de droite (le plus grand).
	
	\begin{enumerate}[leftmargin=110pt]
		\item[\underline{Inclusion $\D \subseteq E$ :}]
		Un élément $x$ de $\D$ admet un développement décimal fini, posons $n\in\N$ sa longueur.
		Alors $10^n x = a$ est un entier relatif. Il suit que $x = \dfrac{a}{10^n} \in E$ comme requis.
		
		\item[\underline{Inclusion $E \subseteq \D$ :}]
		Un élément $x$ de $E$ est de la forme $x = \dfrac{a}{10^n}$. Comme $a$ est entier, le développement décimal de $x$ est de longueur au plus $n$, et est donc fini.
	\end{enumerate} 
}{}

\cor{}{
	On a l'inclusion d'ensembles
		\[ \D \subseteq \Q. \]
}{cor:D-in-Q}

\exe{}{
	Montrer le corollaire \ref{cor:D-in-Q}.
}{exe:caractérisation-D}{
	Tout élément de $\D$ a la forme $\dfrac{a}{10^n}$ où $a \in \Z$ et $n \in \N$.
	Comme $10^n \in \N$, l'élément est bien un ratio de deux entiers et il appartient aux rationnels.
}

\thm{}{
	Le nombre rationnel $\dfrac13$ n'est pas décimal : $\dfrac13 \not\in \D$.
}{}

\pf{}{
	Par l'absurde, supposons que $\dfrac13 = \dfrac{a}{10^n}$.
	Alors on a nécessairement 
		\[ 10^n = 3a. \]
	Comme le membre de droite est un multiple de $3$, le membre de gauche aussi.
	
	On arrive ici à une contradiction car une puissance de $10$ ne peut pas être un multiple de $3$.
	En effet, les multiples de $3$ adviennent tous les $3$ entiers ($\{0 ; 3 ; 6 ; 9 ; 12 ; \dots\}$).
	Or l'entier juste avant $10^n$ est un multiple de $3$, car c'est 
		\[ 10^n - 1 = \underbrace{99{\dots}99}_{\text{n fois}} = 3 \times \underbrace{33{\dots}33}_{\text{n fois}} . \]
	$10^n$ ne peut donc pas être multiple de $3$, une contradiction ! \Large\Lightning
}

\exe{}{
	Montrer que $\dfrac19$ n'est pas décimal.
}{exe:nondecimal}{
	Si $\dfrac19 = \dfrac{a}{10^n}$, alors $10^n = 9a$ et est multiple de $9$.
	Or l'entier d'avant est multiple de $9$ car
		\[ 10^n - 1 = \underbrace{99{\dots}99}_{\text{n fois}} = 9 \times \underbrace{11{\dots}11}_{\text{n fois}} . \]
	$10^n$ ne peut donc pas être multiple de $9$, une contradiction ! \Large\Lightning
}

\exe{}{
	Montrer que $\dfrac1{k}$ n'est pas décimal dès que $k$ ne divise aucune puissance de 10.
}{exe:nondecimal}{
	Si $\dfrac1k = \dfrac{a}{10^n}$, alors $10^n = ka$ et est multiple de $k$, une contradiction ! \Large\Lightning
}

\exe{}{
	Montrer que si $k$ divise une certaine puissance de 10, alors $\dfrac1k$ est décimal.
}{exe:nondecimal}{
	Si $k$ divise $10^n$, alors il existe un $a \in \Z$ tel que
		\[ ak = 10^n. \]
	Ainsi $\dfrac1k = \dfrac{a}{10^n} \in \D$.
}

\mprop{}{
	Le développement décimal de $\dfrac13$ est donné par
		\[ \dfrac13 = 0,333{\dots}.\]
	Il est \emph{infini} et \emph{périodique}.
}{}

\pf{}{
	Posons $x = 0,333\dots$ et montrons que $x=\dfrac13$.
	On a 
		\[ 10x = 3,333{\dots} = 3 + x. \]
	Il suit donc que $9x = 3$, et donc que $x=\dfrac13$.
}{}

\exe{}{
	Montrer que les nombres suivants sont rationnels en les exprimant sous forme de fraction d'entiers.
	\[
	\begin{aligned}
		A &= 0,666{\dots} \\
		B &= 9,999{\dots} \\
		C &= 0,121212{\dots} \\
	\end{aligned}
	\hspace{5cm}
	\begin{aligned}
		D &= 1,666{\dots} \\
		E &= 0,34777{\dots} \\
		F &= 0,123123123{\dots}
	\end{aligned}
	\]
	\[
	G = 0,123456789123456789123{\dots} \text{ (nombre d'Amandine) }
	\]
}{exe:dev-to-fraction}{
	\begin{enumerate}[label=\Alph*.]
		\item 
		La relation $10A = 6 + A$ donne $A = \dfrac23$.
		\item
		La relation $10B = 90 + B$ donne $A = 10$. Il s'avère donc que $10$ admet deux écritures décimales différentes.
		Pour se convaincre encore que $B=10$, on peut remarquer la chose suivante : il n'existe pas de nombre entre $B$ et $10$ strictement différent des deux. Or il existe toujours un nombre situé entre deux nombres distincts (leur moyenne, par exemple).
		\item
		La relation $100C = 12 + C$ donne $C = \dfrac{12}{99} = \dfrac{4}{33}$.
		\item
		La relation $10D = 15 + D$ donne $D = \dfrac{15}{9} = \dfrac53$.
		\item
		En posant $E' = 100E = 34,777{\dots}$, on obtient $10E' = 313 + E'$, d'où $E' = \dfrac{313}{9}$, et donc $E = \dfrac{1}{100}E' = \dfrac{313}{900}$.
		\item
		La relation $1000F = 123 + F$ donne $F = \dfrac{123}{999} = \dfrac{41}{333}$.
		\item 
		La relation $10^9 G = 123456789 + G$ donne $G = \dfrac{123456789}{999999999}$.
	\end{enumerate}
}

\dfn{Développement périodique}{
	On dit que le développement décimal d'un nombre est périodique dès qu'il se répète à partir d'un certain point.
}{}

\notations{
	L'expression « si et seulement si » est notée $\iff$ et traduit l'équivalence de deux propositions.
	Deux propositions $(p)$ et $(q)$ sont \emph{équivalentes} si
		\begin{enumerate}
			\item dès que $(p)$ est vraie, alors $(q)$ est vraie. On note $(p) \implies (q)$ ; et
			\item dès que $(q)$ est vraie, alors $(p)$ est vraie. On note $(q) \implies (p)$.
		\end{enumerate}
}

\thm{}{
	Soit $x$ un nombre. $x$ est rationnel si et seulement si $x$ admet un développement décimal périodique.
	Autrement dit :
	\[
		\text{Pour tout nombre $x$}, \bigl( \text{$x$ est rationnel} \bigr) \iff \big( \text{$x$ admet un développement décimal périodique}· \bigr)
	\]
}{}


\nt{
	On a donc l'égalité d'ensembles
		\[ \bigset{ \text{nombres $x$ tels que le développement décimal de $x$ est périodique} } = \Q. \]
}

\qs{}{
	Existe-t-il un nombre irrationnel ? C'est-à-dire, existe-t-il un nombre tel que son développement décimal ne soit pas périodique ?
}

\exe{, difficulty=2}{
	Donner un nombre irrationnel.
}{exe:nombre-R}{
	Voir la démonstration du théorème \ref{thm:nombre-R}.
}

\section{Nombres réels}

\thm{}{
	Il existe un nombre non rationnel.
}{thm:nombre-R}

\pf{}{
	On considère le nombre $x$ dont on construit le développement décimal de la façon suivante.
		\[ x = 0,10100100010000100000100{\dots} . \]
	Entre chaque paire de $1$, on place un nombre croissant de zéros (un, puis deux, puis trois etc…).
	
	Supposons, par l'absurde, que $x$ devienne éventuellement périodique de période $p$.
	Il existe nécessairement une série de $2p$ zéros dans sa partie périodique.
	Cette série contient une période entière, qui doit donc être entièrement nulle. 
	Le développement décimal de $x$ est ainsi éventuellement nul ($x$ est décimal), ce qui n'est pas vrai.	
	
	% un dessin serait bien ici mais je sais pas encore comment en faire un qui éclaire.
}

\dfn{Nombres réels}{
	L'ensemble de tous les nombres est noté $\R$.
	On les appelle \emph{les nombres réels}.
	
	Ils sont continus dans le sens où il n'y pas de trous comme dans les rationnels
}{dfn:nombres-réels}

\nt{
	On a ainsi le drapeau d'ensembles suivant.
		\[ \N \subseteq \Z \subseteq \D \subseteq \Q \subseteq \R. \]
}

\thm{}{
	Quelque soit $x\in\R$ réel et quelque soit $n\in\N$, il existe $r, s \in\Q$ rationnels tels que
		\begin{align*}
			r \leq x \leq s, && \text { et } && s - r \leq \dfrac{1}{10^{n}}.
		\end{align*}
	Ainsi les rationnels sont \emph{denses} dans les réels car il peuvent s'approcher d'aussi près que souhaité (ordre $10^{-n}$).
}{thm:densité-Q}


\exe{, difficulty=1}{
	Démontrer le théorème \ref{thm:densité-Q}.
}{exe:densité-Q}{
	Considérons le nombre $10^n x$, qu'on encadre par deux entiers consécutifs $a$ et $a+1$ en tronquant les chiffres après la virgule :
		\[ a \leq 10^n x \leq (a+1). \]
	En divisant par $10^n$, on obtient l'encadrement
		\[ \dfrac{a}{10^n} \leq x \leq \dfrac{a+1}{10^n}, \]
	qui vérifie bien les propriétés souhaitées car $\dfrac{a+1}{10^n} - \dfrac{a}{10^n} = \dfrac{a+1-a}{10^n} = \dfrac{1}{10^n}$.
}

\thm{de Cantor}{
	Les nombres rationnels sont énumérables, alors que les réels ne le sont pas.
}{thm:Q-dénombrable}

\nt{
	Cantor\footnote{Georg Cantor (1845-1918), mathématicien allemand.} démontre ce théorème grâce à son désormais célèbre \emph{argument de la diagonale}.
	Celui-ci en déduit que, en un certain sens (dont la définition précise est en dehors du champ d'application du cours), 
	\begin{enumerate}
		\item il y a autant de rationnels que de nombres entiers ; et
		\item il y a davantage de réels entre 0 et 1 que de rationnels dans leur intégralité.
	\end{enumerate}
}
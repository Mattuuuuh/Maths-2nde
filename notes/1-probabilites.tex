%!TEX encoding = UTF8
%!TEX root = 0-notes.tex
\chapter{Probabilités}


Le but de ce chapitre est d'étudier la notion de loi de probabilité sur un ensemble d'issues.
Une loi de probabilité est une hypothèse de départ choisie pour modéliser la réalité.
Celle-ci ne se démontre pas et peut résulter soit d'hypothèses théoriques, soit à partir de fréquences observées après un grand nombre d'expériences.
Cette deuxième méthode s'appuie sur le fait que la fréquence d'un événement s'approche de sa probabilité lorsque le nombre d'expériences augmente.
On appelle ce résultat la \og loi des grands nombres \fg.

\section{Introduction}

\subsection{Expérience, univers, événement}

\dfn{Expérience aléatoire}{
	Une expérience aléatoire est une expérience renouvelable dont on connait les résulats possibles sans qu'on puisse savoir avec certitude lequel sera réalisé.
}{dfn:exp-alea}

On introduit ici le concept d'univers et d'événement à l'aide d'ensembles finis.

\dfn{Univers $\Omega$}{
	L'universe $\Omega$ d'une expérience aléatoire est l'\textbf{ensemble} des issues possibles.
}{dfn:univers}

\ex{}{
	On tire deux cartes d'un jeu complet et on considère les couleurs des cartes tirées, sans les ordonner.
	Les $10$ issues possibles constituent l'univers :
		\[
		\Omega = \left\{
		\begin{aligned}
			&( \clubsuit, \clubsuit ), ( \clubsuit, \heartsuit ), ( \clubsuit, \spadesuit ), \\ 
			&( \clubsuit, \diamondsuit ), ( \heartsuit, \heartsuit ), ( \heartsuit, \spadesuit ), \\
			&( \heartsuit, \diamondsuit ), ( \spadesuit, \spadesuit ), ( \spadesuit, \diamondsuit ), ( \diamondsuit, \diamondsuit )
		\end{aligned}
		\right\}
		\]
	Remarquons que si on avait choisi d'ordonner les cartes (en choisir une première puis une deuxième), on aurait plutôt $16$ issues possibles, car dans ce cas $(\clubsuit, \heartsuit) \neq (\heartsuit, \clubsuit)$
}{ex:2-cartes}

\exe{1}{
	Donner les univers $\Omega$ des expériences aléatoires suivantes.
	\begin{multicols}{2}
	\begin{enumerate}[label=---]
		\item Un lancer de dé équilibré à six faces.
		\item Un lancer de pièce de monnaie.
		\item Chiffres du loto.
		\item Un lancer de dé pipé (truqué) à six faces.
	\end{enumerate}
	\end{multicols}
}{exe:univers}{
	TODO
}

\dfn{Événement}{
	Un \emph{événement} est un sous-ensemble de l'univers.
	On peut le décrire avec un ensemble ou des mots, par abus de notation.
}{dfn:énévement}

\ex{}{
	Dans l'expérience aléatoire de l'exemple \ref{ex:2-cartes}, on considère l'événement $E$ suivant.
		\begin{center}
			E : \og au moins une carte de carreau est tirée \fg.
		\end{center}
	Cet événement est associé à l'ensemble $S$ des issues de l'univers $\Omega$ vérifiant $E$.
	C'est-à-dire l'ensemble des paires de couleurs dont au moins une est de carreau.
		\begin{align*}
			S &= \{ c \in \Omega \text{ tels qu'au moins une des cartes du couple $c$ est de carreau} \} \\
			&= \left\{
			\begin{aligned}
			&( \clubsuit, \diamondsuit ), ( \heartsuit, \diamondsuit ), \\
			&( \spadesuit, \diamondsuit ), ( \diamondsuit, \diamondsuit )
			\end{aligned}
			\right\}
		\end{align*}
	Il y a $4$ issues possibles correspondant à cet événement.
}{ex:2-cartes2}

\subsection{Événement complémentaire}

\dfn{Événement complémentaire}{
	Soit $E \subseteq \Omega$ un événement.
	On pose $\overline{E}$ l'événement complémentaire à $E$ dans $\Omega$ : c'est l'ensemble des issues de $\Omega$ qui n'appartiennent pas à $E$.
	Autrement dit,
		\begin{align*}
			E \cap \overline{E} = \emptyset && \text{et} && E \cup \overline{E} = \Omega.
		\end{align*}
}{dfn:ev-compl}

\notations{
	On note aussi le complémentaire d'un événement $E$ des façons suivantes.
		\[ \overline{E} = E^c = \Omega \setminus E = \Omega - E. \]
}

\ex{}{
	On lance un dé à 6 faces, non nécessairement équilibré, et on lit le chiffre du dessus.
	Son univers d'issues est $\Omega = \bigset{ 1 ; 2 ; 3 ; 4 ; 5 ; 6}$.
	L'événement $E$ : « le chiffre obtenu est impair » correspond à l'ensemble $E = \bigset{ 1 ; 3 ; 5}$.
	Son événement complémentaire est $\overline{E}$ : « le chiffre obtenu est pair », et correspond à l'ensemble $\overline{E} = \bigset{2 ; 4 ; 6}$.
	
	Pour synthétiser tout ça, on peut dessiner un diagramme de Venn : 
	\begin{center}
	\includegraphics[page=6, scale=1.5]{figures/fig-proba.pdf}
	\end{center}
	On se convaincra sans trop de difficulté qu'un événement peut être supposé rond, comme suit.
	En outre, rien ne sert d'indiquer où le complémentaire se trouve : il est partout où l'événement n'est pas !
	\begin{center}
	\includegraphics[page=7, scale=1.5]{figures/fig-proba.pdf}
	\end{center}
}{ex:D6-impair}

\subsection{Union, intersection : diagrammes de Venn}

\dfn{Intersection, union}{	
	Pour deux ensembles $A, B$ on définit les ensembles suivants.
		\begin{enumerate}
			\item $A \cap B$ : l'intersection des deux ensembles.
			
			Un élément appartient à $A· \cap B$ dès qu'il appartient à $A$ \textbf{et} à $B$.
			\item $A \cup B$ : l'union des deux ensembles.
			
			Un élément appartient à $A \cup B$ dès qu'il appartient à $A$ \textbf{ou} à $B$.
		\end{enumerate}
}{}

\begin{center}
\includegraphics[page=1, scale=1.3]{figures/fig-proba.pdf}
\hfill
\includegraphics[page=2, scale=1.3]{figures/fig-proba.pdf}
\end{center}


\exe{}{
	Exprimer les intersections et unions suivantes sous forme d'intervalle.
	\begin{multicols}{2}
	\begin{enumerate}
		\item $\bigset{ 1 ; 2 ; -2 ; -4 } \cup \bigset{  -1; 2 ; -4 ; 1 }$
		\item $\bigset{ 1 ; 2 ; -2 ; -4 } \cap \bigset{  -1; 2 ; -4 ; 1 }$
	\end{enumerate}
	\end{multicols}
}{exe:inter-union}{
	\begin{multicols}{2}
	\begin{enumerate}
		\item $\bigset{1;2;-2;-4;-1;1}$
		\item $\bigset{ 1 ; 2 ; -4 }$
	\end{enumerate}
	\end{multicols}
}


\section{Lois de probabilité}

Lorsqu'on considère une expérience aléatoire, on associe une probabilité à chaque issue possible (c'est-à-dire chaque élément de l'univers).
C'est association est une loi de probabilité qu'on ne démontre jamais : c'est la base de la modélisation du hasard.

\dfn{Loi de probabilité}{
	Soit $\Omega = \{ \omega_1, \omega_2, \omega_3, \dots \}$ un univers fini.
	Une loi de probabilité $P$ est une fonction associant à chaque issue $\omega$ une valeur de $[0;1]$ (sa probabilité d'être réalisée).
	\begin{center}
	\begin{tabular}{|c|c|c|c|} \hline
		Issue		& $\omega_1$ & $\omega_2$ & \hspace{15pt} \dots \hspace{15pt} \\ \hline
		Probabilité	& $P(\omega_1)$ & $P(\omega_2)$ &\hspace{15pt} \dots \hspace{15pt} \\ \hline
	\end{tabular}
	\end{center}
	La loi $P$ s'étend à tout sous-ensemble $E = \{e_1, e_2, \dots \}$ de l'univers $\Omega$ par additivité : la probabilité de l'événement $E$ est la somme de la probabilité de chacun de ses éléments.
			\[ P(E) = P(e_1) + P(e_2) + \dots \]
	En outre, la probabilité de l'univers tout entier est $1$ :
		\[ P(\Omega) = P(\omega_1) + P(\omega_2) + \dots  = 1, \]
	et la probabilité de l'ensemble vide $\emptyset$ est nulle :
		\[ P(\emptyset) = 0. \]
}{dfn:loi-proba}

\nt{
	En écrivant $P(\omega)$, on abuse légèrement d'une notation. Il faudrait plutôt écrire
		\[ P\bigl(\{\omega\}\bigr), \]
	car la fonction $P$ prend uniquement des sous-ensembles de $\Omega$.
}

\thm{Inclusion-exclusion}{
	Soient $A, B \subseteq \Omega$ deux événements.
	Alors
		\[ P(A \cup B) = P(A) + P(B) - P(A\cap B). \]
}{thm:incl-excl}

%\pf{Preuve du théorème \ref{thm:incl-excl}}{
\pf{}{
	Considérons le diagramme de Venn suivant où on colorie $A$ et $B$ l'un après l'autre de la même couleur.
	En passant deux fois sur l'intersection, sa couleur devient plus sombre.
	\begin{center}
	\includegraphics[page=3, scale=1.5]{figures/fig-proba.pdf}
	\end{center}
	Lorsqu'on calcule $P(A) + P(B)$, remarquons donc qu'on compte deux fois tous les éléments appartenant à la fois à $A$ et à $B$, c'est-à-dire les éléments de $A \cap B$.
	En les retirant une fois, chaque élément de l'union $A\cup B$ est compté une seule fois et on obtient bien
		\[ P(A \cup B) = P(A) + P(B) - P(A\cap B). \]
}

\dfn{Événements disjoints}{
	Deux événements $A, B \subseteq \Omega$ sont dits disjoints dès que
		\[ A \cap B = \emptyset, \]
	l'ensemble vide.
	On a dans ce cas et d'après le théorème \ref{thm:incl-excl},
		\[ P(A \cup B) = P(A) + P(B). \]
}{}

\dfn{Équiprobabilité}{
	On parle de situation \emph{équiprobable} si chaque issue $\omega$ de l'univers $\Omega$ admet la même probabilité.
		\[ P(\omega_1) = P(\omega_2) =  P(\omega_3) = \dots, \]
	La loi qui en découle s'appelle la \emph{loi uniforme}.
}{}

\cor{}{
	En situation d'équiprobabilité, chaque issue $\omega$ a pour probabilité
		\[ P(\omega) = \dfrac{1}{\text{Nombre d'issues possibles}} = \dfrac{1}{|\Omega|}. \]
	De plus, chaque événement $E \subseteq \Omega$ a pour probabilité
		\[ P(E) = \dfrac{|E|}{|\Omega|}. \]
}{cor:equiprob}
%\pf{Démonstration du corollaire \ref{cor:equiprob}}{
\pf{}{
	Comme la probabilité de l'univers $P(\Omega)$ vaut $1$, on en déduit que
		\[ P(\Omega) = P(\omega_1) + P(\omega_2) + \dots  = 1. \]
	Chaque probabilité est la même ; notons la $p$. La somme a exactement $|\Omega|$ termes, donc
		\begin{gather*}
			|\Omega| \cdot p = 1, \\
			p = \dfrac{1}{|\Omega|}. 
		\end{gather*}
	Pour un événement $E = \{e_1, e_2, \dots \} \subseteq \Omega$, on a par définition que la probabilité de l'événement $E$ est la somme de la probabilité de chacun de ses éléments.
		\begin{align*}
			P(E) &= P(e_1) + P(e_2) + \dots \\
				&= \dfrac{1}{|\Omega|} + \dfrac{1}{|\Omega|} + \dots + \dfrac{1}{|\Omega|} \\
				&= \dfrac{|E|}{|\Omega|}.
		\end{align*}
}{}

\cor{}{
	Soient $E \subseteq \Omega$ un événement et $\overline{E}$ son complémentaire.
	Alors
		\[ P\left( \overline{E} \right) = 1 - P(E). \]
}{cor:compl}

%\pf{Démonstration du corollaire \ref{cor:compl}}{
\pf{}{
	En utilisant les propriétés de $P$, du complémentaire, et le principe d'inclusion-exclusion, on trouve
		\begin{align*}
		 1 = P(\Omega) &= P\left( E \cup \overline{E} \right), \\
		 				&= P(E) + P\left(\overline{E}\right) - P\left( E \cap \overline{E}\right), \\
		 				&= P(E) + P\left(\overline{E}\right) - P\left( \emptyset \right), \\
		 				&= P(E) + P\left(\overline{E}\right),
		\end{align*}
	ce qui conclut.
}{}

\section{Arbres de probabilité}

Les arbres de probabilité sont des schémas permettant de décrire les chemins possibles d'une expérience aléatoire à plusieurs épreuves (ou étapes).
Le point de départ est la \emph{racine} de l'arbre, et les issues possibles sont ses \emph{feuilles}.

On choisit ici de représenter un arbre de haut en bas, plutôt que de gauche à droite comme il est commun de le faire.
Il va de soi que ces deux représentations sont équivalentes.

La \emph{profondeur} de l'arbre désigne le nombre total d'épreuves de l'expérience.
On lit l'arbre en le parcourant de haut en bas, chaque embranchement correspondant à un résultat possible de l'épreuve d'après.
La somme des probabilités de chaque branche est donc nécessairement $1$.
Pour connaître la probabilité d'une issue, il faut multiplier les probabilités correspondant à chaque choix du parcours vers celle-ci.

L'ensemble des feuilles $\omega$ de l'arbre correspond à l'univers $\Omega$ de l'expérience.
On pourra donc connaître $|\Omega|$ en comptant les feuilles, et $P(\omega)$ en multipliant les probabilités qui mènent de la racine à la feuille.

\ex{}{
	On lance un D$6$ équilibré, dé à $6$ faces.
	\begin{enumerate}[label=$\bullet$]
		\item Si le numéro obtenu est un multiple de 3, on extrait au hasard une boule dans l'urne 1 qui contient 3 boules noires, 4 boules blanches et 3 boules rouges
		\item Si le numéro obtenu n'est pas un multiple de 3, on extrait une boule dans l'urne 2 qui contient 3 boules noires et 2 boules blanches.
	\end{enumerate}
	On distingue deux étapes à l'expérience : on crée donc l'arbre de profondeur $2$ suivant, où on dénote par B, N, et R les évéments \og tirer une boule blanche, noire, rouge \fg respectivement.
	
	\begin{center}
	\includegraphics[page=4, scale=1.25]{figures/fig-proba.pdf}
	\end{center}
	On ajoute les probabilités de chacun des événements sur les branches associées.
	Par exemple, de la racine, pour atteindre l'embranchement \og multiple de $3$ \fg, il y a $1$ chance sur $3$ (car seuls $3$ et $6$ sont les issues favorables).
	On ajoute donc un $\frac13$ sur la branche qui mène à cet événement, est un $\frac23$ sur l'autre.
	
	En sachant qu'on ait obtenu un nombre qui n'est pas un multiple de $3$, on tire une boule dans la deuxième urne  qui contient 3 boules noires et 2 boules blanches.
	La probabilité d'obtenir une boule blanche est donc $\frac25$, et celle d'obtenir une boule noire est $\frac35$.
	En notant toutes les probabilités ainsi calculées, on obtient l'arbre de probabilité suivant.
	
	\begin{center}
	\includegraphics[page=5, scale=1.25]{figures/fig-proba.pdf}
	\end{center}
	La probabilité d'un chemin racine-feuille est le produit des probabilités rencontrées en le parcourant.
	En outre, deux tels chemins correspondent à des événements disjoints.
	On calcule par exemple
		\[ P(N) = \dfrac23 \cdot \dfrac35 + \dfrac13 \cdot \dfrac3{10} = \dfrac12. \]
}{}

% à reporter au chapitre d'échantillonnage peut-être ?

%\section{Construction d'un modèle empirique}
%done



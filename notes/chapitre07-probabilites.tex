%!TEX encoding = UTF8
%!TEX root =notes.tex
\chapter{Probabilités}

Le but de ce chapitre est de formaliser la notion de loi de probabilité en s'appuyant sur les ensembles.
Une loi de probabilité est une hypothèse de départ choisie pour modéliser la réalité.
Celle-ci ne se démontre pas et peut résulter soit d'hypothèses théoriques, soit à partir de fréquences observées après un grand nombre d'expériences.
Cette dernière méthode s'appuie sur le fait la fréquence d'un certain résultat d'une expérience s'approche de sa probabilité.
On appelle ce résultat la \og loi des grands nombres \fg, dont il existe deux variantes : la faible et la forte.

Les objectifs de ce chapitre sont les suivants.
	\begin{enumerate}
		\item Ensemble (univers) des issues. Événements. Réunion, intersectin, complémentaire.
		\item Loi (distribution) de probabilité. Probabilité d'un événement : somme des probabilités des issues.
		\item Relation $P(A \cup B) + P(A \cap B) = P(A) + P(B)$.
		\item Dénombrement à l'aide de tableaux et d'arbres.
	\end{enumerate}
Les capacités visées sont les suivantes.
	\begin{enumerate}
		\item Utiliser des modèles théoriques de référence en comprenant que les probabilités sont définies a priori.
		\item Construire un modèle à partir de fréquences observées, en distinguant nettement modèle et réalité.
		\item Calculer des probabilités dans des cas simples : expérience aléatoire à deux ou trois épreuves.
	\end{enumerate}

\section{Introduction}

On introduit ici le concept d'univers et d'événement à l'aide d'ensembles finis.

\dfn{Expérience aléatoire}{
	Une expérience aléatoire est une expérience renouvelable dont on connait les résulats possibles sans qu'on puisse savoir avec certitude lequel sera réalisé.
}{}

\dfn{Univers $\Omega$}{
	L'universe $\Omega$ d'une expérience aléatoire est l'\textbf{ensemble} des issues possibles.
}{}

\ex{}{
	On tire deux cartes d'un jeu complet et on considère les couleurs des cartes tirées, sans les ordonner.
	Les $10$ issues possibles constituent l'univers :
		\[
		\Omega = \left\{
		\begin{aligned}
			&( \clubsuit, \clubsuit ), ( \clubsuit, \heartsuit ), ( \clubsuit, \spadesuit ), \\ 
			&( \clubsuit, \diamondsuit ), ( \heartsuit, \heartsuit ), ( \heartsuit, \spadesuit ), \\
			&( \heartsuit, \diamondsuit ), ( \spadesuit, \spadesuit ), ( \spadesuit, \diamondsuit ), ( \diamondsuit, \diamondsuit )
		\end{aligned}
		\right\}
		\]
	Remarquons que si on avait choisi d'ordonner les cartes (en choisir une première puis une deuxième), on aurait plutôt $16$ issues possibles, car dans ce cas $(\clubsuit, \heartsuit) \neq (\heartsuit, \clubsuit)$
}{ex:2-cartes}

\exe{}{
	Donner les univers $\Omega$ des expériences aléatoires suivantes.
	\begin{multicols}{2}
	\begin{enumerate}[label=---]
		\item Un lancer de dé équilibré à six faces.
		\item Un lancer de pièce de monnaie.
		\item Chiffres du loto.
		\item Un lancer de dé pipé (truqué) à six faces.
	\end{enumerate}
	\end{multicols}
}{}

\dfn{Événement}{
	Un \emph{événement} est un sous-ensemble de l'univers.
	On peut le décrire avec un ensemble ou des mots, par abus de notation.
}{}

\ex{}{
	Dans l'expérience aléatoire de l'exemple \ref{ex:2-cartes}, on considère l'événement $E$ suivant.
		\begin{center}
			E : \og au moins une carte de carreau est tirée \fg.
		\end{center}
	Cet événement est associé à l'ensemble $S$ des issues de l'univers $\Omega$ vérifiant $E$.
	C'est-à-dire l'ensemble des paires couleurs dont au moins une est de carreau.
		\begin{align*}
			S &= \{ c \in \Omega \text{ tels qu'au moins une des cartes du couple $c$ est de carreau} \} \\
			&= \left\{
			\begin{aligned}
			&( \clubsuit, \diamondsuit ), ( \heartsuit, \diamondsuit ), \\
			&( \spadesuit, \diamondsuit ), ( \diamondsuit, \diamondsuit )
			\end{aligned}
			\right\}
		\end{align*}
	Il y a $4$ issues possibles correspondant à cet événement.
}{}

\section{Lois de probabilité}

Lorsqu'on considère une expérience aléatoire, on associe une probabilité à chaque issue possible (c'est-à-dire chaque élément de l'univers).
C'est association est une loi de probabilité qu'on ne démontre jamais : c'est la base de la modélisation du hasard.

\dfn{Loi de probabilité}{
	Soit $\Omega = \{ \omega_1, \omega_2, \omega_3, \dots \}$ un univers fini.
	Une loi de probabilité $P$ est une fonction associant à chaque issue $\omega$ une valeur de $[0;1]$ (sa probabilité d'être réalisée).
	\begin{center}
	\begin{tabular}{|c|c|c|c|} \hline
		Issue		& $\omega_1$ & $\omega_2$ & \hspace{15pt} \dots \hspace{15pt} \\ \hline
		Probabilité	& $P(\omega_1)$ & $P(\omega_2)$ &\hspace{15pt} \dots \hspace{15pt} \\ \hline
	\end{tabular}
	\end{center}
	La loi $P$ s'étend à tout sous-ensemble $E = \{e_1, e_2, \dots \}$ de l'univers $\Omega$ par additivité : la probabilité de l'événement $E$ est la somme de la probabilité de chacun de ses éléments.
			\[ P(E) = P(e_1) + P(e_2) + \dots \]
	En outre, la probabilité de l'univers tout entier est toujours $1$ :
		\[ P(\Omega) = P(\omega_1) + P(\omega_2) + \dots  = 1. \]
}{}

\nt{
	En écrivant $P(\omega)$, on abuse légèrement d'une notation. Il faudrait plutôt écrire
		\[ P(\{\omega\}), \]
	car la fonction $P$ prend uniquement des sous-ensembles de $\Omega$.
}{}

\thm{Inclusion-exclusion}{
	Soient $A, B \subseteq \Omega$ deux événements.
	Alors
		\[ P(A \cup B) = P(A) + P(B) - P(A\cap B). \]
}{thm:incl-excl}

\pf{Preuve du théorème \ref{thm:incl-excl}}{
	On considère le diagramme de Venn suivant.
	
	TODO
	
	Lorsqu'on calcule $P(A) + P(B)$, remarquons qu'on compte deux fois tous les éléments appartenant à la fois à $A$ et à $B$, c'est-à-dire les éléments de $A \cap B$.
	En les retirant une fois, chaque élément de l'union $A\cup B$ est compté une seule fois et on obtient bien
		\[ P(A \cup B) = P(A) + P(B) - P(A\cap B). \]
}{}

\dfn{Équiprobabilité}{
	On parle de situation \emph{équiprobable} si chaque issue $\omega$ de l'univers $\Omega$ admet la même probabilité.
		\[ P(\omega_1) = P(\omega_2) =  P(\omega_3) = \dots, \]
	La loi qui en découle s'appelle la \emph{loi uniforme}.
}{}

\cor{}{
	En situation d'équiprobabilité, chaque issue $\omega$ a pour probabilité
		\[ P(\omega) = \dfrac{1}{\text{Nombre d'issues possibles}} = \dfrac{1}{|\Omega|}. \]
	De plus, chaque événement $E \subseteq \Omega$ a pour probabilité
		\[ P(E) = \dfrac{|E|}{|\Omega|}. \]
}{cor:equiprob}
\pf{Démonstration du corollaire \ref{cor:equiprob}}{
	Comme la probabilité de l'univers $P(\Omega)$ vaut $1$, on en déduit que
		\[ P(\Omega) = P(\omega_1) + P(\omega_2) + \dots  = 1. \]
	Chaque probabilité est la même ; notons la $p$. La somme a exactement $|\Omega|$ termes, donc
		\[ |\Omega| \cdot p = 1, \]
	et donc
		\[ p = \dfrac{1}{|\Omega|}. \]
	Pour un événement $E = \{e_1, e_2, \dots \} \subseteq \Omega$, on a par définition que la probabilité de l'événement $E$ est la somme de la probabilité de chacun de ses éléments.
		\begin{align*}
			P(E) &= P(e_1) + P(e_2) + \dots \\
				&= \dfrac{1}{|\Omega|} + \dfrac{1}{|\Omega|} + \dots + \dfrac{1}{|\Omega|} \\
				&= \dfrac{|E|}{|\Omega|}.
		\end{align*}
}{}

\dfn{Événement complémentaire}{
	Soit $E \subseteq \Omega$ un événement.
	On pose $\overline{E}$ l'événement complémentaire à $E$ dans $\Omega$ : c'est l'ensemble des issues de $\Omega$ qui n'appartiennent pas à $E$.
	Autrement dit,
		\begin{align*}
			E \cap \overline{E} = \emptyset && \text{et} && E \cup \overline{E} = \Omega.
		\end{align*}
}{}

\nt{
	On note aussi le complémentaire d'un événement $E$ des façons suivantes.
		\[ \overline{E} = E^c = \Omega \setminus E = \Omega - E. \]
}{}

\cor{}{
	Soient $E \subseteq \Omega$ un événement et $\overline{E}$ son complémentaire.
	Alors
		\[ P\left( \overline{E} \right) = 1 - P(E). \]
}{}

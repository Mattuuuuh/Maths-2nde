%!TEX encoding = UTF8
%!TEX root =notes.tex

\chapter{Vecteurs}

Le but de ce chapitre est de traiter la section \og Manipuler les vecteurs du plan \fg du bulletin officiel.
Un partie du contenu a déjà été étudiée au chapitre \ref{chap:3}, section \ref{sec:geom-plane} : géométrie plane.
Il s'agira d'une part de revoir des concepts : représentation de points dans un repère orthonormé, opérations sur les points, longueur de segment.
D'autre part, on ajoutera les notions de vecteur comme translation, de norme, de colinéarité, et de déterminant.

Le contenu du chapitre est le suivant.
	\begin{itemize}
		\item
			Vecteur $\vec{MM'}$ associé à la translation qui transforme $M$ en $M'$. Direction, sens, et norme.
		\item
			Égalité de deux vecteurs. Notation $\vec{u}$. Vecteur nul.
		\item
			Somme de deux vecteurs en lien avec l'enchaînement des translations. Relation de Chasles.
		\item
			Base orthonormée. Coordonnées d'un vecteur. Expression de la norme d'un vecteur.
		\item
			Expression des coordonnées de $\vec{AB}$ en fonction de celles de $A$ et de $B$.
		\item
			Produit d'un vecteur par un nombre réel. Colinéarité de deux vecteurs.
		\item
			Déterminant de deux vecteurs dans une base orthonormée, critère de colinéarité.
			Application à l'alignement et au parallélisme.
	\end{itemize}

Les capacités attendues sont les suivantes.
	\begin{itemize}
		\item 
			Représenter géométriquement des vecteurs.
		\item
			Construire géométriquement la somme de deux vecteurs.
		\item
			Représenter un vecteurs dont on connaît les coordonnées.
			Lire les coordonnées d'un vecteur.
		\item
			Calculer les coordonnées d'une somme de vecteurs, d'un produit d'un vecteur par un nombre réel.
		\item
			Calculer la distance entre deux points. Calculer les coordonnées du milieu d'un segment.
		\item
			Résoudre des problèmes en utilisation la représentation la plus adaptée des vecteurs.
	\end{itemize}
	
\section{Introduction}

\dfn{Translation du plan}{
	Une \emph{translation} $T$ du plan est le déplacement de tous les points du plan selon la même direction, la même distance, et dans le même sens.
	
	Pour un point $A$ quelconque, on définit l'addition
		\[ A + T \]
	comme le \emph{translaté} de $A$ par $T$.
}{}

\nt{
	Une translation $T$ déplace tous les points du plan selon la même direction, la même distance, et dans le même sens.
	On peut donc décomposer $T$ en une translation \og gauche/droite \fg~ selon l'axe des abscisses, et une translation \og haut/bas \fg selon les ordonnées.
	
	Chaque point $A(x_A ; y_A)$ voit ses coordonées modifiées de la même manière :
	celui-ci se déplace systématiquement d'une certaine quantité $x_T$ en abscisse, et $y_T$ en ordonnée.
	Ainsi, une fois déplacé, le point $A$ admet pour nouvelles coordonnées
		\[ A + T = (x_A + x_T ; y_A + y_T). \]
	Remarquons que c'est exactement comme cela qu'on a défini l'addition de deux points au chapitre \ref{chap:3}, section \ref{sec:geom-plane}, définition \ref{def:manip-points}.
	On pose donc $T = (x_T ; y_T)$ en réutilisant les notations $(x ; y)$ des points du plan.
	
	Il n'y a en fait pas de différence mathématique entre une translation et un point : à chaque translation on peut associer un point, et vice versa.
}{}


\dfn{Vecteur $\vec{AB}$}{
	Soient $A, B$ deux points du plan.
	On nomme 
		\[ T = \vec{AB} \]
	la translation $T$ du plan qui envoie le point $A$ sur le point $B$.
	On a alors la relation
		\[ A + \vec{AB} = B. \]
	
	En décomposant $T$ selon les abscisses et les ordonnées, on écrit ses coordonnées sous la forme $T = \begin{pmatrix} x_T \\ y_T \end{pmatrix}$.
}{}

\ex{}{
	Soient $A(2,5; 3), B(1,5; -4)$ deux points.
	On trace les points dans un repère, ainsi qu'une flèche envoyant $A$ sur $B$, qui correspond au vecteur $\vec{AB}$.
	
	On décompose la translation selon les deux axes : translater de $-1$ en abscisse, puis translater de $-7$ en ordonnée.
	On en déduit que $\vec{AB} = \pvec{-1}{-7}$.
	
	\begin{center}
	\begin{tikzpicture}[>=stealth, scale=1]
	\begin{axis}[xmin = -0.9, xmax=2.9, xtick={ -3, ..., 5}, ymin=-4.9, ymax=3.9, axis x line=middle, axis y line=middle, axis line style=<->, xlabel={}, ylabel={}, ytick = {-4, -3, ..., 2, 3}, grid=both]
		
		\addplot[black, mark=*, mark size = 1] (2.5,3) node[above right] {$A$};
		\addplot[black, mark=*, mark size = 1] (1.5,-4) node[right] {$B$};
	
		\draw[very thick, black, ->] (axis cs:2.5,3) -- (axis cs:1.5,-4) node[below right, pos=.5] {$\vec{AB}$};
		\draw[very thick, myr, ->] (axis cs:2.5,3) -- (axis cs:1.5,3) node[above, pos=.5] {$-1$};
		\draw[very thick, myb, ->] (axis cs:1.5,3) -- (axis cs:1.5,-4) node[left, pos=.5] {$-7$};
	\end{axis}
	\end{tikzpicture}
	\end{center}
}{ex:vec-1}

\mprop{Calcul de $\vec{AB}$}{
	Par définition, on a 
		\[ A + \vec{AB} = B, \]
	qui permet de calculer la translation $\vec{AB}$ à l'aide de la relation
		\[ \vec{AB} = B - A = \pvec{x_B - x_A}{y_B-y_A}. \]
}{prop:calcul-AB}

\cor{}{
	Pour connaître les coordonnées d'une translation, il suffit de l'appliquer à l'origine $O$ et de lire les coordonnées du translaté.
	
	En d'autres termes,
		\[ \vec{OA} = A. \]
}{cor:OAisA}

\dfn{Opérations sur les vecteurs}{
	Les vecteurs héritent des opérations légales sur les points (chapitre \ref{chap:3}, définition \ref{def:manip-points}).
	Soient $u = \pvec{x}{y}, v = \pvec{x'}{y'}$ deux vecteurs et $\kappa\in\R$ un réel. 
	Alors
	\begin{enumerate}
		\item $u+v = \pvec{x+x'}{y+y'}$ ; et
		\item $\kappa \cdot u = \pvec{\kappa \cdot x}{\kappa\cdot y}$.
	\end{enumerate}
}{}

\section{Composition de translations : somme de vecteurs}

\thm{Relation de Chasles}{
	Soient $A, B, C$ trois points du plans. 
	Alors
		\[ \vec{AB} + \vec{BC} = \vec{AC}. \]
		\begin{center}
		\og L'addition de la translation qui envoie $A$ sur $B$ à celle qui envoie $B$ sur $C$ est égale à la translation qui envoie $A$ sur $C$ \fg.
		\end{center}
}{thm:chasles}

\pf{Démonstration du théorème \ref{thm:chasles}}{
	Par calcul direct à l'aide de la proposition \ref{prop:calcul-AB},
	\begin{align*}
		\vec{AB} + \vec{BC} &= \left(B-A\right) + \left(C - B\right), \\
								&= B - A + C - B, \\
								&= C - A, \\
								&= \vec{AC}.
	\end{align*}
}{}

\ex{Somme de vecteurs géométriquement}{
	Soient ${\color{myr} u} = \pvec{2}{-1}$ et ${\color{myb} v}= \pvec{-3}{-2}$ deux vecteurs.
	
	Pour construire géométrique la somme ${\color{myr} u}+{\color{myb} v} =\pvec{2-3}{-1-2} = \pvec{-1}{-3}$, on colle bout à bout les vecteurs ${\color{myr} u}$ et ${\color{myb} v}$ comme ci-dessous.
	
	On démarre à l'origine $O$ pour pouvoir lire les coordonnées du vecteur, d'après le corollaire \ref{cor:OAisA}.
	
	\begin{center}
	\begin{tikzpicture}[>=stealth, scale=1]
	\begin{axis}[xmin = -1.9, xmax=2.9, xtick={ -3, ..., 5}, ymin=-3.9, ymax=0.9, ytick={-2}, axis x line=middle, axis y line=middle, axis line style=<->, xlabel={}, ylabel={}, grid=both]
		
		\addplot[black, mark=*, mark size = 1] (0,0) node[above right] {$O$};
	
		\draw[very thick, myr, ->] (axis cs:0, 0) -- (axis cs:2, -1) node[above right, pos=.5] {$u$};
		
		\draw[very thick, myb, ->] (axis cs:2, -1) -- (axis cs:-1, -3) node[below right, pos=.5] {$v$};
		
		\draw[very thick, black, ->] (axis cs:0, 0) -- (axis cs:-1, -3) node[left, pos=.5] {$u+v$};
	\end{axis}
	\end{tikzpicture}
	\end{center}
}{}

\section{Norme vectorielle}

Considérons $\vec{AB}$ le vecteur translatant le point $A$ sur le point $B$.
On considère la longueur $AB$ du segment $[AB]$, ce qui servira à définir la norme d'un vecteur, l'équivalent de la longueur pour les segments, et de la distance pour les nombres réels.

La formule de la longueur d'un segment (voir théorème \ref{thm:long-segment}, chapitre \ref{chap:3}) permet calculer\footnote{Remarquons que l'ordre de $A$ et de $B$ n'importe pas dans la formule, et que $|x|^2 = (x)^2$, car le carré ignore le signe.}
	\[ AB = \sqrt{ |x_B - x_A|^2 + |y_B - y_A|^2}. \]
Remarquons qu'on y voit apparaître les différences $x_B - x_A$ et $y_B - y_A$, qui sont exactement les coordonnées du vecteur
	\[ \vec{AB} = B-A. \]

\dfn{Norme d'un vecteur}{
	Soit $u = \pvec{x}{y}$ un vecteur quelconque.
	On définit sa \emph{norme} $\norm{u}$ par
		\[ \norm{u} = \sqrt{x^2 + y^2}. \]
		
	Il suit, par construction, que $\norm{\vec{AB}} = AB$.
}{}

\nt{
	Remarquons qu'on a la relation, pour $\ell = AB$, la longueur du segment $[AB]$,
		\[ \ell = \norm{\vec{AB}} = \norm{B-A} = \norm{A-B}. \]
	Ceci devrait rappeler la distance entre les nombres réels $x, y\in\R$, notée $|x-y|$ à l'aide de la valeur absolue.
}{}

\ex{}{
	Soient $A(2,5; 3), B(1,5; -4)$ les deux points de l'exemple \ref{ex:vec-1}.
	On a $\vec{AB} = B - A = \pvec{-1}{-7}$, et donc
		\[ \norm{\vec{AB}} = \norm{\pvec{-1}{-7}} = \sqrt{\left(-1\right)^2 + \left(-7\right)^2} = \sqrt{50} = \sqrt{25 \times 2} = 5 \sqrt{2}. \]
		
	En comparant avec la longueur $AB = BA = \sqrt{(2,5 - 1,5)^2 + (3 - (-4))^2} = 5 \sqrt{2}$, on obtient bien le même résultat.
	Notons qu'il y a moins de chance de faire une erreur avec la norme vectorielle car on calcule la longueur en deux étapes, avec d'abord les différences de coordonnées.
	
	\begin{center}
	\begin{tikzpicture}[>=stealth, scale=1]
	\begin{axis}[xmin = -0.9, xmax=2.9, xtick={ -3, ..., 5}, ymin=-4.9, ymax=3.9, axis x line=middle, axis y line=middle, axis line style=<->, xlabel={}, ylabel={}, ytick = {-4, -3, ..., 2, 3}, grid=both]
		
		\addplot[black, mark=*, mark size = 1] (2.5,3) node[above right] {$A$};
		\addplot[black, mark=*, mark size = 1] (1.5,-4) node[right] {$B$};
	
		\draw[very thick, black, ->] (axis cs:2.5,3) -- (axis cs:1.5,-4) node[below right, pos=.5] {$\vec{AB}$};
		\draw[very thick, myr, ->] (axis cs:2.5,3) -- (axis cs:1.5,3) node[above, pos=.5] {$-1$};
		\draw[very thick, myb, ->] (axis cs:1.5,3) -- (axis cs:1.5,-4) node[left, pos=.5] {$-7$};
	\end{axis}
	\end{tikzpicture}
	\end{center}
}{}

\thm{Homogénéité de la norme}{
	Soit $u$ un vecteur et $\kappa\in\R$ un nombre réel.
	Alors
		\[ \norm{\kappa\cdot u} = |\kappa| \cdot \norm{u}. \]
}{}

\section{Colinéarité et déterminant}

\dfn{Vecteurs colinéaires}{
	Soient $u$ et $v$ deux vecteurs.
	On dit de $u$ et $v$ qu'ils sont \emph{colinéaires} dès qu'il existe un réel $\kappa\in\R$ tel que
		\[ u = \kappa \cdot v. \]
	Deux vecteurs colinéaires admettent la même direction, mais pas forcément le même sens ni la même norme.
}{}

\nt{
	Tous les vecteurs sont colinéaires au vecteur nul (en prenant $\kappa = 0$ ci-dessus).
}{}


\mprop{}{
	Soit $A, B, C, D$ quatre points quelconques.
	
	Les vecteurs $\vec{AB}$ et $\vec{CD}$ sont colinéaires si et seulement si les droites $(AB)$ et $(CD)$ sont parallèles.
}{prop:alignement-parallélisme}

\cor{}{
	Trois points $A, B, C$ quelconques sont alignés si et seulement si les vecteurs $\vec{AB}$ et $\vec{AC}$ sont colinéaires.
}{prop:alignement-colinéarité}

\lem{}{
	Soit $(d)$ une droite non verticale et $A, B \in (d)$ deux points distincts de celle-ci.
	
	Alors $\vec{AB}$ est colinéaire à un vecteur de la forme $\pvec{1}{a}$ où $a$ est le coefficient directeur de la fonction affine $f$ telle que $\C_f = (d)$.
}{lem:vecteur-dir-coeff-dir}

\pf{Preuve du lemme \ref{lem:vecteur-dir-coeff-dir}}{
	En écrivant $A(x_A;y_B)$ et $B(x_B;y_B)$ les coordonnées des points $A$ et $B$, on a $x_A \neq x_B$ car les points sont distincts et la droite est non verticale.
	
	Ainsi, $\vec{AB} = B-A$ est égal à
		\[\vec{AB}= \pvec{x_B - x_A}{y_B-y_A} = \left(x_B - x_A\right) \pvec{1}{\frac{y_B-y_A}{x_B-x_A}}. \]
	On reconnaît bien ici le coefficient directeur de la fonction affine associée à $(d)$.
}{}

\dfn{Vecteur directeur}{
	Soit $(d)$ une droite quelconque.
	On appelle $v$ \emph{vecteur directeur} de $(d)$ si $v$ et $(d)$ admettent la même direction.
	
	Si $(d)$ est non verticale, $v$ est colinéaire à $\pvec{1}{a}$, ce qui donne immediatement le coefficient directeur de la fonction affine associée.
}{}

\nt{
	Soit $A$ un point, $v$ un vecteur, et $(d)$ la droite passant par $A$ et dirigée par $v$.
	
	Alors pour $B$ un point quelconque de $(d)$, le vecteur $\vec{AB}$ est nécessairement colinéaire à $v$.
	Par conséquent, il existe un nombre réel $k\in\R$ tel que $\vec{AB} = k \cdot v$.
	
	Or comme $\vec{AB} = B-A$, on a nécessairement
		\[ B = A + k \cdot v. \]
	Un choix arbitraire de $k$ donnera donc naissance à un nouveau point de la droite.
	Par exemple, les points $A+v, A-v, A+2v, A-\frac5{13}v, A - \sqrt{5} \cdot v$, etc... sont tous des points de la droite $(d)$.
	
	On a en fait
		\[ (d) = \left\{ A + k \cdot v \text{ où $k$ parcourt $\R$} \right\}. \]
}{}

%\mprop{}{
%	Soit $A, B, C$ trois points tels que $x_A \neq x_B \neq x_C \neq x_A$.
%	
%	Les vecteurs $\vec{AB}$ et $\vec{AC}$ sont colinéaires si et seulement si les points $A, B$, et $C$ sont alignés (c'est-à-dire qu'ils appartiennent à une même droite).
%}{prop:alignement-colinéarité}
%
% not useful
%\pf{Démonstration de la proposition \ref{prop:alignement-colinéarité}}{
%	\underline{Direction gauche-droite :}
%	
%	Si les vecteurs $\vec{AB}$ et $\vec{AC}$ sont colinéaires, il existe une constante $\kappa\in\R$ telle que
%		\[ \vec{AB} = \kappa \vec{AC}. \]
%	Par hypothèse sur les points, aucun des vecteurs n'est nul, et donc $\kappa\neq0$.
%	
%	Le coefficient directeur $a$ de la fonction affine dont la courbe représentative est $(AB)$ vérifie 
%		\[ a = \dfrac{y_B - y_A}{x_B - x_A} = \dfrac{\kappa \cdot (y_C - y_A)}{\kappa \cdot (x_C - x_A)} = \dfrac{y_C - y_A}{x_C - x_A}. \]
%	C'est donc le même coefficient directeur que celui de la fonction affine dont la courbe représentative est $(AC)$.
%	Par suite, les droites $(AB)$ et $(AC)$ sont parallèles et partagent le point $A$ : leur intersection n'est donc pas vide et elles sont nécessairement confondues.
%	
%	\underline{Direction droite-gauche :}
%	
%	Si les coefficients directeurs des fonctions affines dont les courbes représentatives sont $(AB)$ et $(AC)$ sont les mêmes, le déterminant des vecteurs $\vec{AB}$ et $\vec{AC}$ est nul, ce qui conclut d'après la suite.
%}{}

\dfn{Déterminant de deux vecteurs}{
	Soient $u = \pvec{a}{b}$ et $v=\pvec{c}{d}$ deux vecteurs.
	On définit le \emph{déterminant} $\det(u, v)$ par le nombre réel
		\begin{align*}
			\det(u,v) = \begin{vmatrix} a & c \\ b & d \end{vmatrix} = ad - bc.
		\end{align*}
}{}

\thm{}{
	Soient $u, v$ deux vecteurs.
	Alors $\det(u, v) = 0$ si et seulement si les vecteur $u$ et $v$ sont colinéaires.
	
	Le déterminant détermine donc si deux vecteurs sont colinéaires ou non.
}{}

%\ex{}{
%	Considérons les droites $(AB)$ et $(A'B')$ où
%		\begin{align*}
%			A(2; -4), && B(-3,4 ; 3,1), && \text{ et } && A'(2; -4), && B'(-3,4 ; 3,1).
%		\end{align*}
%	D'après la proposition \ref{prop:alignement-colinéarité}, les vecteurs $\vec{AB}$ et $\vec{A'B'}$ sont appelé \emph{vecteurs directeurs} des droites $(AB)$ et $(A'B')$ car ils donnent la direction (à multiple près) de celles-ci.
%
%	Si les vecteurs $\vec{AB}$ et $\vec{A'B'}$ sont colinéaires, les deux droites admettent la même direction et sont donc parallèles.
%}{}

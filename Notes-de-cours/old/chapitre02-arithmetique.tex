%!TEX encoding = UTF8
%!TEX root =notes.tex

\chapter{Arithmétique}

	\section{Diviseurs et multiples}

	\dfn{Diviseur, multiple}{
		Pour $d , n\in \N$ deux entiers naturels, on dit que 
			\[ d \textbf{ divise } n \in \N \]
		dès que $n$ s'écrit de la forme
			\[ n = d \cdot k \]
		pour un entier naturel $k \in \N$.
		On écrit alors 
			\[ d \ | \ n, \]
		et on dit également que $n$ est un \textbf{multiple} de $d$.
	}{}
	
	\ex{Nombre pair, impair}{
		 Les nombres $n$ \textbf{pairs} sont les multiples de $2$. Ils s'écrivent donc 
		 	\[ n = 2 k \]
		 pour un entier naturel $k \in \N$.
		 
		 Les nombres $n$ \textbf{impairs} se situent juste après un nombre pair et s'écrivent alors
		 	\[ n = 2 k + 1 \]
		pour un entier naturel $k \in \N$.
	}{}
	
	\qs{}{
		Soit $a \in \N$ un entier naturel.
		L'entier $ (2a + 1)^2 - 1$ est-il toujours pair ?
		
		\emph{Rappel : on a l'identité remarquable $(a+b)(a-b) = a^2 - b^2$, pour $a, b$ des nombres quelconques.}
	}
	
	\ex{Ensembles de diviseurs}{
		L'ensemble des diviseurs de $24$ est donné par
			\[ \mathcal{D}_{24} = \left\{ 1 ; 2 ; 3 ; 4 ; 6 ; 8 ; 12 ; 24 \right\}. \]
		L'ensemble des multiples de $17$ inférieurs ou égaux à $100$ est donné par
			\[ \mathcal{M} = \left\{ 0 ; 17 ; 34 ; 51 ; 68 ; 85 \right\}. \]
	}{}
	
	\nt{
		Les diviseurs se regoupent par paires : en effet, si $d$ divise $n$ et que $n = d \cdot k$, alors $k$ divise aussi $n$.
	}
	
	\qs{}{
		Étant donné que $37$ divise $111$. Montrer que $37$ divise alors aussi $555$.
		
		Plus généralement, montrer que si $a | b$, alors $a$ divise aussi tous les multiples de $b$.
	}
	
\section{Nombres premiers}

	\dfn{Nombre premier}{
		Pour $p \in \N$, $p \geq 2$, un entier naturel. On dit que $p$ est \textbf{premier} si ses seuls diviseurs sont $1$ et lui-même.
	}{}
	
	\ex{}{
		Les premiers nombres premiers sont
			\[ \{ 2 ; 3 ; 5 ; 7 ; 11 ; 13 ; \dots \}. \]
		Il y en a une infinité.
	}{}
	
	
	\thm{Théorème fondamental de l'arithmétique}{
		Tout entier $n \in \N$, $n \geq 2$ peut s'écrire de façon unique comme produit de nombres premiers.
	}{}
	
	
	\ex{}{
		\begin{multicols}{2}
		\begin{itemize}[label=$\bullet$]
			\item $32 = 2^5$
			\item $9 = 3^2$
			\item $24 = 2^3 \cdot 3$
			\item $110 = 2 \cdot 5 \cdot 11$
			\item $10^n = 2^n \cdot 5^n$
			\item $10^n \cdot 30^m = 3^m \cdot 2^{m+n} \cdot 5^{m+n}$
		\end{itemize}
		\end{multicols}	
	}{}
	
	\exe{}{
		Écrire la décomposition en produit de facteurs premiers des entiers suivants.
		
		\begin{multicols}{4}
		\begin{itemize}[label=$\bullet$]
			\item $33$
			\item $48$
			\item $110 \times 55$
			\item $35 \times 90$
		\end{itemize}
		\end{multicols}
	}{}
	
	\mprop{ }{ 
		Le rationnel $\dfrac17 \in \Q$ n'est pas un nombre décimal : $\dfrac17 \notin \D$.
	}{}
	
	\pf{Démonstration par l'absurde }{
		La preuve se décline comme suit.
		
		\begin{enumerate}
			\item Supposons, par l'absurde, que $\dfrac17 \in \D$.
			\item Par définition, $\D  = \left\{ \dfrac{a}{10^n} \text{ tel que : } a \in \Z, n \in \N \right\}$.
			\item Donc $\dfrac17$ s'écrit $\dfrac{a}{10^n}$ pour certains $a \in Z$ et $n \in \N$.
			\item D'où $\dfrac17 =  \dfrac{a}{10^n}$, et par suite $10^n = 7 \cdot a$. C'est une égalité de deux entiers naturels.
			\item À droite : $a$ étant un entier et $7$ étant premier, la décomposition en produit de premiers de $7 \cdot a$ contient $7$.
			\item Cependant, à gauche, $7$ n'apparaît pas dans la décomposition en produit de premiers de $10^n = 2^n \cdot 5^n$.
			\item L'égalité obtenue ne peut donc pas être vraie : ceci est une contradition, et $\dfrac17 \notin \D$.
		\end{enumerate}
	}

	\qs{}{
		Refaire la démonstration pour $\dfrac5{12} \notin \D$.
	}
	
	\nt{
		Si un entier naturel $d \in \N$ divise un autre entier naturel $n \in \N$, décomposition en facteurs premiers de la relation
			\[ n = d \cdot k \]
		où $k\in\N$ permet de dire la chose suivante.
		
		La puissance d'un premier $p$ dans la décompoistion de $d$ est inférieure ou égale à sa puissance dans la décomposition de $n$.
	}
	
	\mprop{}{
		Considérons un $n \in \N$.
		
		Si $n$ est pair, alors tous les multiples de $n$ sont pairs.
	}{prop:1}
	
	\exe{}{
		Cette proposition est le cas $a=2$ de la proposition suivante à démontrer.
		
		Si $a|b$, alors $a$ divise tous les multiples de $b$.
	}{}
	
\section{Coprimalité}
	
	\nt{
		Si le numérateur et le dénominateur partagent un diviseur commun, il peut s'annuler
			$\dfrac{a \cdot d}{b \cdot d} = \dfrac{a}b$.
		On dit alors qu'on réduit la fraction.
	}
	
	\dfn{Coprimalité et fractions irréductibles}{
		Deux entiers naturels $a$ et $b$ de $\N$ sont \textbf{premiers entre eux} si leur seul diviseur commun est $1$.
		
		La fraction $\dfrac{a}b$ est alors \textbf{irréductible}.
	}{}
	
	\exe{}{
		Donner l'ensemble des diviseurs communs à $100 \times 121$ et $44 \times 55$.
		Réduire la fraction $\dfrac{100 \times 121}{44 \times 55}$.
	}{}
	
	\thm{}{
		Soit $\sqrt{2}$ le nombre positif qui vérifie $\left(\sqrt{2}\right)^2 = 2$.
		Alors $\sqrt{2}$ est irrationnel : $\sqrt{2} \notin \Q$.
	}{thm:1}
	
	
	\lem{}{
		Considérons un entier naturel $n\in\N$.
		
		Si $n$ est impair, alors $n^2$ est impair.
	}{lem:1}
	
	\pf{Démonstration du lemme \ref{lem:1}}{
		Si $n$ est impair, et pour montrer que $n^2$ est impair, il suffit de montrer que $n^2 -1$ est pair.
		
		Or 
			\[ n^2 - 1 = (n+1)(n-1), \]
		Comme $n$ est impair, $n-1$ est pair, et $n^2 - 1$ est un multiple d'un nombre pair.
		
		D'après la proposition \ref{prop:1}, $n^2 -1 $ est pair, et donc $n^2$ est impair.
	}
	
	\lem{Constraposition du lemme \ref{lem:1}}{
		Considérons un entier naturel $n\in\N$.
		
		Si $n^2$ est pair, alors $n$ est pair.
	}{lem:1bis}
	
%	\pf{Preuve du lemme \ref{lem:1bis}}{
%		C'est la \emph{contraposition} du lemme \ref{lem:1}.
%	}
	
	\pf{Preuve du théorème \ref{thm:1}  }{
		La démonstration est à nouveau par l'absurde.
		
		\begin{enumerate}
			\item Supposons, par l'absurde, que $\sqrt{2} \in \Q$ soit rationnel.
			
			\item Par définition $\Q = \left\{ \dfrac{a}b \text{ tq. } a \in \Z, b\in\Z, b \neq 0 \right\}.$
			
			\item Donc $\sqrt{2}$ s'écrit $\dfrac{a}{b}$ pour certains $a,b \in \Z$, $b \neq 0$.

			\item Comme $\sqrt{2}$ est positif, et en simplifiant la fraction, on peut écrire $\sqrt{2} = \dfrac{a}{b} = \dfrac{p}{q}$ fraction irréductible avec $p, q \in \N$, $q \neq 0$.
			
			\item Par définition de $\sqrt{2}$, 
				\[ \sqrt{2}^2 = 2 = \left( \dfrac{p}q \right)^2 = \dfrac{p^2}{q^2}. \]
				
			\item D'où l'égalité d'entiers naturels $p^2 = 2 q^2$. 
				L'entier $p^2$ est donc pair, et $p$ doit l'être aussi d'après le lemme \ref{lem:1bis}.
			
			\item $p$ est multiple de $2$ et s'écrit alors comme 
				\[ p = 2 k, \]
				pour un entier naturel $k \in \N$.
				
				En substituant dans l'équation $p^2 = 2 q^2$, on trouve
					\[ (2k)^2 = 2 q^2 \qquad \iff \qquad q^2 = 2 k^2.\]
			\item D'après le lemme \ref{lem:1bis}, $q$ est pair.
			\item Ceci est une contradiction, car la fraction $\dfrac{p}q$ a été choisie irréductible, alors que $p$ et $q$ sont tous les deux pairs. 
				Finalement, $\sqrt{2} \notin \Q$.
		\end{enumerate}
	}
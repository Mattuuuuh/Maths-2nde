%!TEX encoding = UTF8
%!TEX root =notes.tex

\chapter{Fonctions de référence}

Le but de ce chapitre est de traiter la partie \og Se constituer un répertoire de fonctoins de référence \fg~ du bulletin officiel.

Le contenu du chapitre est le suivant.
	\begin{itemize}
		\item Fonctions carré, inverse, racine carrée, cube.
		\item Définitions, domaines, courbes représentatives, variations, extrema.
	\end{itemize}

Les capacités attendues sont les suivantes.
	\begin{itemize}
		\item Pour deux nombres $a$ et $b$ donnés et une fonction de référence $f$, comparer $f(a)$ et $f(b)$ numériquement ou graphiquement.
		\item Pour les fonctions affines, carré, inverse, racine carrée et cube, résoudre graphiquement ou algébriquement une équation ou une inéquation du type $f(x) = k$, $f(x) < k$.
	\end{itemize}

\section{Fonctions de référence}

\subsection{Nouvelles fonctions parentes : transformations des abscisses}

\ex{translation des abscisses}{
	Considérons $f$ une fonction quelconque sur $\R$ et $g$ définie par
		\[ g(x) = f(x+2). \]
	Pour calculer $g(0)$, on utilise donc la définition $g(0) = f(0+2) = f(2)$.
	De façon identique, $g(-3) = f(-1)$, $g(12) = f(14)$, etc…
	
	En fait, la fonction $g$ regarde dans l'avenir de la fonction $f$ : le point $(x ; f(x))$ de $\C_f$ correspond au point $(x-2 ; f(x)) = (x ; f(x)) + \pvec{-2}{0}$ sur $\C_g$.
	Graphiquement, la courbe $\C_f$ est translatée de $2$ unité vers la gauche pour obtenir $\C_g$.
	
	Pour $h(x) = f(x-1)$, on obtient les graphes suivants.
	\begin{center}
	\begin{tikzpicture}[scale=1]
		\begin{axis}[xmin = -10, xmax=10, ymin=-4.25, ymax=4.25, axis x line=middle, axis y line=middle, axis line style=->, grid=both,
		%ytick={-4,-3,...,2,3}, xtick={-11, -10,...,-4,-3},
	    	%every y tick label/.style={
	        %anchor=near yticklabel opposite,
	        %xshift=0.2em,
	    	%}
	    	]
		% g cos
		\addplot[no marks, myb, -, very thick] expression[domain=-10:10, samples=50]{.01*(x+6)*(x+3)*(x-5)}
		node[pos=.5, below]{$\mathcal{C}_f$};
		\addplot[no marks, myr, -, very thick] expression[domain=-10:10, samples=50]{.01*(x+8)*(x+5)*(x-3)}
		node[pos=.1, above]{$\mathcal{C}_g$};
		\addplot[no marks, myg, -, very thick] expression[domain=-10:10, samples=50]{.01*(x+1)*(x-2)*(x-10)}
		node[pos=.95, below]{$\mathcal{C}_h$};
		\end{axis}
	\end{tikzpicture}
	\end{center}
}{}

\ex{multiplication des abscisses}{
	Considérons $f$ une fonction quelconque sur $\R$ et $g$ définie par
		\[ g(x) = f(-x). \]
	Pour calculer $g(0)$, on utilise donc la définition $g(0) = f(-0) = f(0)$.
	De façon identique, $g(-3) = f(3)$, $g(12) = f(-12)$, etc…
	
	Tout point $(x;f(x))$ est associé au point $(-x;g(-x)) = (-x ; f(x)) \in \C_g$.
	Graphiquement, les courbes $\C_f$ et $\C_g$ sont symétriques par rapport à l'axe des ordonnées.
	
	Pour $h(x) = f(2x)$, on obtient les graphes suivants.
	\begin{center}
	\begin{tikzpicture}[scale=1]
		\begin{axis}[xmin = -10, xmax=10, ymin=-4.25, ymax=4.25, axis x line=middle, axis y line=middle, axis line style=->, grid=both,
		%ytick={-4,-3,...,2,3}, xtick={-11, -10,...,-4,-3},
	    	%every y tick label/.style={
	        %anchor=near yticklabel opposite,
	        %xshift=0.2em,
	    	%}
	    	]
		% g cos
		\addplot[no marks, myb, -, very thick] expression[domain=-10:10, samples=50]{.01*(x+6)*(x+3)*(x-5)}
		node[pos=.5, below]{$\mathcal{C}_f$};
		\addplot[no marks, myr, -, very thick] expression[domain=-10:10, samples=50]{.01*(-x+6)*(-x+3)*(-x-5)}
		node[pos=.1, above]{$\mathcal{C}_g$};
		\addplot[no marks, myg, -, very thick] expression[domain=-10:10, samples=50]{.01*(2*x+6)*(2*x+3)*(2*x-5)}
		node[pos=.95, below]{$\mathcal{C}_h$};
		\end{axis}
	\end{tikzpicture}
	\end{center}
}{}


\thm{Fonctions parentes : transformations des abscisses}{
	Soit une fonction $f$ continue sur un domaine $\D$, et $c\in\R$ un nombre réel.
		\begin{enumerate}
			\item La courbe représentative $\C_g$ de la fonction
				\[ g(x) = f(x+c) \]
			est obtenue en translatant la courbe $\C_f$ de $c$ unités vers la gauche.
			\item La courbe représentative $\C_h$ de la fonction
				\[ h(x) = f(c \cdot x) \]
				est obtenue à partir de $\C_f$ en allongeant ou contractant l'axe des abscisses d'un facteur $c$ et, si $c<0$, en appliquant une symétrie selon l'axe des ordonnées.
		\end{enumerate}
}{}

\subsection{Fonction inverse}

\subsection{Fonction racine carrée}

\subsection{Fonction cube}

\section{Position relative des courbes, parité}

\dfn{Fonction paire, fonction impaire}{
	On dit qu'une fonction réelle $f$ est 
		\begin{enumerate}
			\item \emph{paire} si $\C_f$ est symétrique par rapport à l'axe des ordonnées.
			Autrement dit, si
				\[ f(x) = f(-x) \]
			pour tout $x\in\R$ réel.
			\item \emph{impaire} si $\C_f$ est symétrique par rapport à l'origine $O$ du repère.
			Autrement dit, si
				\[ f(x) = -f(x) \]
			pour tout $x\in\R$ réel.
		\end{enumerate}
}{}





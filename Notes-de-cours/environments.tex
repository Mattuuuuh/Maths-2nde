%!TEX encoding = UTF8
%!TEX root = 0-notes.tex

%%%%%%%%%%%%%%%%%%%%%%%%%%%%
% CLEAN SETUPS
%%%%%%%%%%%%%%%%%%%%%%%%%%%%

\theoremstyle{definition}

\newtheorem{theorem}{Théorème}[chapter]
\newtheorem{corollaire}[theorem]{Corollaire}
\newtheorem{lemme}[theorem]{Lemme}
\newtheorem{proposition}[theorem]{Proposition}
\newtheorem{exercice}[theorem]{Exercice}
\newtheorem{exemple}[theorem]{Exemple}
\newtheorem{definition}[theorem]{Définition}
\newtheorem*{question}{Question}
\newtheorem*{preuve}{Preuve}
\newtheorem*{remarque}{Remarque}
\newtheorem*{strategie}{Stratégie}
\newtheorem*{methode}{Méthode}
\newtheorem*{notation}{Notation}
\newtheorem*{nomenclature}{Nomenclature}
\newtheorem{axiome}[theorem]{Axiome}
\newtheorem*{heuristique}{Heuristique}

\newtheorem*{definition*}{Définition}
\newtheorem*{lemme*}{Lemme}
\newtheorem*{proposition*}{Proposition}
\newtheorem*{theorem*}{Théorème}
\newtheorem*{corollaire*}{Corollaire}

%%%%%%%%%%%%%%%%%%%%%%%%%%%%
% CLEAN SETUPS : MDFRAMED SURROUND
%%%%%%%%%%%%%%%%%%%%%%%%%%%%

\ifclean

\usepackage[framemethod=pgf]{mdframed}
% def
\mdfdefinestyle{definition}{
	hidealllines=true,
	leftline=true,
	innerrightmargin=0,
	innertopmargin=-4pt,
}
\surroundwithmdframed[
	style=definition,
]{definition}
\surroundwithmdframed[
	style=definition,
]{definition*}

% thm
\mdfdefinestyle{theorem}{
	leftline=true,
	innertopmargin=-4pt,
	nobreak=true,
}
\surroundwithmdframed[
	style=theorem,
]{theorem}
\surroundwithmdframed[
	style=theorem,
]{theorem*}

% prop
\mdfdefinestyle{proposition}{
	leftline=true,
	innertopmargin=-4pt,
	nobreak=true,
}
\surroundwithmdframed[
	style=proposition,
]{proposition}
\surroundwithmdframed[
	style=proposition,
]{proposition*}

%%%%%%%%%%%%%%%%%%%%%%%%%%%%
% COLOURFUL SETUPS : MDFRAMED SURROUND
%%%%%%%%%%%%%%%%%%%%%%%%%%%%

\else

\usepackage[framemethod=pgf]{mdframed}
% def
\mdfdefinestyle{definition}{
	hidealllines=true,
	leftline=true,
	linecolor=BLUE_E,
	linewidth=2pt,
	innertopmargin=-4pt,
	innerrightmargin=0,
	nobreak=true,
}
\surroundwithmdframed[
	style=definition,
]{definition}
\surroundwithmdframed[
	style=definition,
]{definition*}

% thm
\mdfdefinestyle{theorem}{
	linecolor=MAROON_C,
	linewidth=2pt,
	roundcorner=4pt,
	innertopmargin=-4pt,
	nobreak=true,
}
\surroundwithmdframed[
	style=theorem,
]{theorem}
\surroundwithmdframed[
	style=theorem,
]{theorem*}

% prop
\mdfdefinestyle{proposition}{
	linecolor=GREEN_E,
	linewidth=2pt,
	innertopmargin=-4pt,
	nobreak=true,
}
\surroundwithmdframed[
	style=proposition,
]{proposition}
\surroundwithmdframed[
	style=proposition,
]{proposition*}

% lemme
\mdfdefinestyle{lemme}{
	linecolor=TEAL_E,
	linewidth=1pt,
	innertopmargin=-4pt,
	nobreak=true,
}
\surroundwithmdframed[
	style=lemme,
]{lemme}
\surroundwithmdframed[
	style=lemme,
]{lemme*}

% corollaire
\mdfdefinestyle{corollaire}{
	linecolor=YELLOW_E,
	linewidth=2pt,
	roundcorner=4pt,
	innertopmargin=-4pt,
	nobreak=true,
}
\surroundwithmdframed[
	style=corollaire,
]{corollaire}
\surroundwithmdframed[
	style=corollaire,
]{corollaire*}

\fi

%%%%%%%%%%%%%%%%%%%%%%%%%%%%
% EXERCISES 
%%%%%%%%%%%%%%%%%%%%%%%%%%%%

\usepackage[answerdelayed, lastexercise]{exercise}
\renewcommand{\ExerciseHeader}{
	\textbf{
	\theExercise.
	}
	\ifnum\ExerciseDifficulty=0
	\else
		(\theExerciseDifficulty)
	\fi
}
\renewcommand{\DifficultyMarker}{$\star$}
\renewcommand{\AnswerHeader}{
	% if exercise title is "1" then announce new chapter
	\if\ExerciseTitle1
		\hrule\vspace{1cm}
		{\LARGE
		\textbf{Exercices du chapitre \thechapter}\newline\newline
		}
	\fi
	
	\centerline{\textbf{
	Exercice \ExerciseHeaderNB
	}}
}

%%%%%%%%%%%%%%%%%%%%%%%%%%%%%%%%%%%%%%%%%%%
% MINTED FOR PYTHON ALGORITHMS
%%%%%%%%%%%%%%%%%%%%%%%%%%%%%%%%%%%%%%%%%%%

\usepackage{tcolorbox}
\tcbuselibrary{minted,breakable,xparse,skins}
\definecolor{bg}{gray}{0.95}
\DeclareTCBListing{mintedbox}{O{}m!O{}}{%
  breakable=true,
  listing engine=minted,
  listing only,
  minted language=#2,
  minted style=default,
  minted options={%
    linenos,
    gobble=0,
    breaklines=false, % otherwise it breaks for no apparent reason?
    breakafter=,,
    fontsize=\small,
    numbersep=8pt,
    tabsize=4, % tab ident = 4 spaces
    fontfamily=courier, %important pour les signes <, >
    #1},
  boxsep=0pt,
  left skip=0pt,
  right skip=0pt,
  left=25pt,
  right=0pt,
  top=3pt,
  bottom=3pt,
  arc=5pt,
  leftrule=0pt,
  rightrule=0pt,
  bottomrule=2pt,
  toprule=2pt,
  colback=bg,
  colframe=orange!70,
  enhanced,
  overlay={%
    \begin{tcbclipinterior}
    \fill[orange!20!white] (frame.south west) rectangle ([xshift=20pt]frame.north west);
    \end{tcbclipinterior}},
  #3}


%%%%%%%%%%%%%%%%%%%%%%%%%%%%
% TCOLORBOX SETUPS
%%%%%%%%%%%%%%%%%%%%%%%%%%%%

%!TEX encoding = UTF8
%!TEX root = 0-notes.tex

\chapter{Fonctions}
\label{chap:fonctions}

%\section{Fonction comme boite noire}
%
%\section{Définition algébrique d'une fonction}
%
%\section{Courbe représentative d'une fonction}
%
%\section{Implémentation}


\section{Introduction}

\dfn{fonction, variable, constante}{
	On dit qu'une quantité $y\in\R$ s'exprime \emphindex{en fonction d}'une quantité $x\in\R$ lorsque, à chaque nombre $x$ possible, on peut associer \underline{une unique} valeur $y$.
	
	On note alors $y = f(x)$, lu  « $y$ égal $f$ de $x$ », où $x$ est la \emphindex{variable}, et $f$ la \emphindex{fonction}.
	
	Si une quantité ne dépend pas de $x$, on l'appelle \emphindex{constante}, ou \emphindex{indépendante} de $x$.
	Cependant, une constante peut toujours être vue comme une fonction de $x$ !
	
	\begin{center}
	\includegraphics[page=1]{figures/fig-fonctions.pdf}
	\end{center}
}{}

\nomen{
	Lorsque $x$ vaut $2$, la valeur $y$ associée sera $f(2)$, lu « $f$ de $2$ ».
	%Ceci permet de différencier les valeurs qui émanent de différents $x\in\R$ : $f(1), f(-3), f(\frac34), \dots$.
}

\ex{}{
	Le périmètre $P$ d'un cercle de rayon $R$ est donné par $2\pi \cdot R$.
	Un rayon détermine donc un unique périmètre, et la fonction Périmètre, dépendant de la variable $R$, est donnée par
		\[ \text{Périmètre}(R) = 2\pi R. \]
}{ex:perimètre}

\exe{1}{
	Un lynx prend la fuite : il court à 15 mètres par seconde.
	\begin{enumerate}
		\item La distance parcourue par le lynx est-elle une fonction du temps écoulé depuis son départ ?
		\item Le temps écoulé depuis le départ du lynx est-il une fonction de la distance parcourue ?
	\end{enumerate}
}{exe:lynx}{
	\begin{enumerate}
		\item 
		Oui car à chaque temps est associé \textbf{une unique} distance.
		\item 
		Oui également, car à chaque distance est associé \textbf{un unique} temps.
	\end{enumerate}
}

\exe{}{
	Un étudiant jette une balle dans les airs et mesure la hauteur de la balle tous les quarts de seconde.
	Il note ses résultats dans le tableau ci-dessous.
	\begin{center}
	\def\arraystretch{1.2}
	\begin{tabular}{|c|c|c|c|c|c|c|c|}\hline
		Hauteur (cm) & 85 & 145 & 190 & 145 & 85 & 40 & 0 \\ \hline
		Temps (s) & 0 & 0,25 & 0,5 & 0,75 & 1 & 1,25 & 1,5 \\\hline
	\end{tabular}
	\end{center}
	
	\begin{enumerate}
		\item La hauteur est-elle une fonction du temps ? Justifier.
		\item Le temps est-il une fonction de la hauteur ? Justifier.
	\end{enumerate}
}{exe:table}{
	\begin{enumerate}
		\item On peut associer \textbf{une seule} hauteur à chaque temps. La hauteur peut donc être vue comme fonction du temps.
		\item On ne peut pas associer un unique temps à chaque hauteur. Par exemple, il y a deux temps distincts pour lesquels la hauteur est de $85$cm (0s et 1s).
		Le temps ne peut donc pas être vu comme fonction de la hauteur.
	\end{enumerate}
}

	
\exe{, difficulty=2}{
	\begin{enumerate}
		\item
		Montrer que le rayon $R$ d'un cercle est fonction de son périmètre $P$ et écrire la fonction $\text{Rayon}(P)$ associée.
		\item
		Montrer que l'aire $A$ d'un cercle est fonction de son rayon $R$ et écrire la fonction $\text{Aire}(R)$ associée.
		\item
		En déduire que l'aire $A$ d'un cercle est fonction de son périmètre $P$ et écrire la fonction $\text{Aire}(P)$ associée.
	\end{enumerate}
}{exe:f2}{
	\begin{enumerate}
		\item
		De la relation 
			\[ P = 2\pi \cdot R, \]
		on déduit que 
			\[ R = \dfrac{1}{2\pi} \cdot P. \]
		Remarquons qu'on a inversé le membre de gauche et le membre de droite pour extraire la forme d'une fonction : une valeur de $P$ donne une unique valeur de $R$.
		
		Ainsi on peut voir $R$ comme une fonction de $P$. On note alors $R=R(P)$, qui vérifie
		\[ R(P) = \dfrac{1}{2\pi} \cdot P. \]
		
		\item 
		La formule de l'aire
			\[ A = \pi \cdot R^2 \]
		exprime directement l'aire comme fonction du rayon : à chaque rayon possible, on peut calculer une unique aire.
		On note alors
			\[ A(R) =  \pi \cdot R^2. \]
		\item
		On souhaite exprimer l'aire $A$ comme fonction du périmètre $P$.
		Or d'après les questions précédentes, l'aire $A$ est fonction du rayon $R$, et le rayon $R$ est lui-même fonction du périmètre $P$.
		
		Ainsi, une valeur de $P$ donne une unique valeur de $R$ qui donne une unique valeur de $A$.
		En suivant le raisonnement, on peut composer les fonctions trouvées ci-dessus en considérant $A(R(P))$.
		Pour ne pas surcharger les notations, appelons plutôt $\text{Aire}$ la fonction prenant un périmètre.
			\begin{align*}
			 \text{Aire}(P) &= A\bigl(R(P)\bigr) \\ &=A\left(\dfrac{1}{2\pi} \cdot P\right) \\ &= \pi \left( \dfrac{1}{2\pi} \cdot P \right)^2 \\ &= \dfrac{\pi}{4\pi^2} \cdot P^2 = \dfrac{1}{4\pi} \cdot P^2.
			 \end{align*}
	\end{enumerate}
}

\notations{
	Pour une fonction $f$ quelconque, 
	on note $\D\subseteq\R$ un \emphindex{domaine} sur laquelle elle est bien définie et étudiée :
	$f$ admet une image $f(x)$ pour chaque $x\in\D$ du domaine d'étude.
}

%\dfn{fonction réelle, domaine}{
%	Soit $\D \subset \R$ un ensemble de nombres réels qu'on appelle \emphindex{domaine}.
%	
%	Une fonction $f$ de $\D$ dans $\R$ associe à chaque élément $x\in\D$ du domaine un nombre $f(x) \in \R$.
%	
%	On note alors
%		\begin{align*}
%			f: \D & \longrightarrow \R \\
%			x& \longmapsto f(x).
%		\end{align*}
%	
%	\begin{center}
%	\includegraphics[page=1]{figures/fig-fonctions.pdf}
%	\end{center}
%}{}

\ex{}{
	Une constante est en particulier une fonction d'une variable $x\in\R$ réelle.
	Ainsi la fonction $g$ donnée par $g(x) = 42$ pour tout $x\in\R$ 
	vaut constamment $42$, peu importe la valeur de $x$. 
	$g$ est une \emphindex{fonction constante} sur le domaine $\D = \R$.
}{ex:fonction-constante}

\exe{}{
	On considère la fonction $f$ qui à chaque triangle du plan associe la somme de ses angles.
	Quel est le domaine de $f$ ? Que vaut $f(T)$ pour tout triangle $T$ ? Que dire de $f$ ?
}{exe:somme-angles}{
	Le domaine de $f$ est l'ensemble des triangles du plan.
	Or, dans le plan, la somme des angles d'un triangle vaut toujours 180°.
	Donc $f(T) = 180^\circ$ pour tout triangle $T$ : c'est une fonction constante.
}

\exe{, difficulty=2}{
	On considère la fonction $f$ qui à chaque triangle du plan associe l'angle le plus grand de $T$.
	Montrer que $f(T) \leq 180^\circ$ pour n'importe quel triangle $T$.
	Quel est le minimum de $f$ ? Donner un triangle $T$ tel que $f(T)$ est minimal.
}{exe:max-angles}{
	L'angle d'un triangle $T$ est au plus l'angle plat, de valeur 180°. Son angle le plus grand, $f(T)$, est donc aussi borné supérieurement par 180°.
	
	Remarquons que $f(T) \geq 60^\circ$ pour tout triangle $T$, car si tous les angles d'un triangle sont inférieurs à 60°, alors leur somme ne peut pas être 180°.
	Le triangle équilatéral vérifie $f(T) = 60^\circ$, le minimum de $f$.
}
	

\dfn{antécédent, image}{
	Considérons une fonction $f$ et deux nombres réels $x, y\in\R$ vérifiants
		\[ y = f(x). \]
	On lit \og $y$ égal $f$ de $x$ \fg, et on dit alors que
		\begin{enumerate}
			\item
			$y$ est l'\emphindex{image} de $x$ par $f$ ; et
			\item
			$x$ est \underline{un} \emphindex{antécédent} de $y$ par $f$.
		\end{enumerate}
}{}

\nt{
	\begin{enumerate}
		\item 
		Chaque $x\in\D$, élément du domaine, a \underline{exactement une} image par $f$ qui est $f(x)$.
		\item
		Un nombre $y\in\R$ peut avoir \underline{un}, \underline{aucun}, ou même \underline{plusieurs} antécédents. C'est le cas par exemple de $y=0, -1, 1$ pour la fonction carré $f(x) = x^2$.
	\end{enumerate}
}

\nt{
	Le fait que $f(x) = x+1$ soit bien une fonction nous permet de dire que $a = b \implies a+1 = b+1$.
	Similairement, l'implication $a = b \implies 2a = 2b$ utilise que $f(x) = 2x$ est une fonction.
	
	En fait, les manipulations d'équations usuelles (addition, multiplication, mise au carré, etc...) prennent toutes la forme $a = b \implies f(a) = f(b)$ pour une fonction $f$ bien choisie.
}

\dfn{forme algébrique}{
	On appelle 
		\[ f(x) = 3 x+1 \]
	la forme \emphindex{algébrique} de $f$.
	On lit dans ce cas \og $f$ de $x$ égal trois $x$ plus 1 \fg.
	
	C'est une forme entièrement générale qui permet de déduire l'image $f(x)$ à partir de n'importe quel réel $x\in\D$ du domaine de $f$.
}{}

\exe{}{
	On reprend l'exercice \ref{exe:lynx}, dans lequel un lynx court à 15 mètres par seconde.
	Donner la forme algébrique de la fonction $f$ donnant la distance parcourue (mètres) au temps $t$ (secondes).
	Quel est le domaine de $f$ ?
}{exe:lynx2}{
	La distance est la vitesse multipliée par le temps : $f(t) = 15\cdot t = 15t$.
	Comme $f$ prend un temps, son domaine est tous les nombres réels positifs ou nuls.
}

\nt{
	Suite à l'exercice \ref{exe:lynx2} ci-dessus, on dira alors qu'on a écrit la distance \emphindex{en fonction} du temps $t$.
	L'expression algébrique permet de calculer facilement plusieurs images sans refaire le raisonnement « distance = vitesse fois temps » à chaque fois.
}

\notations{
	Pour n'importe quel nombre $x$ on dénote $x^2 = x \cdot x$ le produit de $x$ par lui-même.
	On lit « $x$ au carré » ou « $x$ carré ».
}

\ex{}{
	Considérons la \emphindex{fonction carré} $f(x) = x^2$ pour tout $x\in\R$.
	Calculons les images de $1; 4; -1; -0,31; \frac23 ; -\frac23;$ et $0$ par $f$.
		\begin{multicols}{3}
		\begin{enumerate}%[leftmargin=50pt]
			\item $f(1) = 1$
			\item $f(-1) = 1$
			\item $f( -0,31) = 0,0961$
			\item $f\left(\dfrac23\right) = \dfrac49$
			\item $f\left( -\dfrac23\right) = \dfrac49$
			\item $f(0) = 0$
		\end{enumerate}
		\end{multicols}
	\noindent
	On en déduit les propositions suivantes.
		\begin{multicols}{2}
		\begin{enumerate}[label=--]%, leftmargin=50pt]
			\item
			L'image de $1$ par $f$ est $1$.
			\item
			L'image de $-1$ par $f$ est $1$.
			\item
			$1$ et $-1$ sont deux antécédents de $1$ par $f$.
			\item
			Un antécédent de $0,0961$ par $f$ est $-0,31$.
			\item
			Un antécédent de $\frac49$ par $f$ est $\frac23$.
			\item
			Un antécédent de $\frac49$ par $f$ est $-\frac23$.
			\item
			$0$ est l'image et un antécédent de $0$ par $f$. %C'est d'ailleurs le seul antécédent.
		\end{enumerate}
		\end{multicols}
}{ex:carré0}

	
\exe{}{
	Considérons la fonction $f$ donnée {algébriquement} par
	\begin{align*}
		f(x) = 3x + 1
	\end{align*}
	pour tout $x\in\R$.
	
	\begin{enumerate}
		\item
		Calculer l'image par $f$ de $0$ ; de $3,1$ ; de $\frac13$ ; de $-1$ ; de $-\frac23$.
		\item
		Donner un antécédent de $1$ par $f$.
		\item
		Déterminer tous les antécédents de $6$ par $f$.	
	\end{enumerate}
}{exe:images-antécédents}{
	\begin{enumerate}
	\item
	L'image de $x$ est donnée par $f(x)$. On calcule donc
		\begin{align*}
			f(0) &= 3\cdot0 + 1 = 1 \\
			f(3,1) &= 3\cdot3,1 + 1 = 10,3 \\
			f\left(\frac13\right) &= 3\cdot\frac13 + 1 = 1 + 1 = 2 \\
			f(-1) &= 3\cdot(-1) + 1 = -2 \\
			f\left(-\frac23\right) &= 3 \cdot \left(- \frac23\right) + 1 = -1.
		\end{align*}
	\item
	La relation
		\[ f(0) = 1 \]
	implique que l'image de $0$ est $1$, et qu'un antécédent de $1$ est $0$.
	On a donc trouvé $0$, antécédent de $1$ par $f$.
	\item
	Appelons $x$ un antécédent de $6$ par $f$.
	Autrement dit, l'image de $x$ est $6$, et donc 
		\[ f(x) = 6. \]
	En utilisant l'expression algébrique de $f$, on trouve
		\begin{align*}
			f(x) &= 6, \\
			3\cdot x + 1 &= 6, \\
			3\cdot x &= 5, \\
			x &= \dfrac53.
		\end{align*}
	Ainsi $x=\frac53$ est le seul antécédent de $6$ par $f$.
	En général, certaines fonctions admettes plusieurs antécédents.
	Voir par exemple l'exercice suivant.
	\end{enumerate}


}

\exe{}{
	Un fonction $f$ admet le tableau de valeurs suivant.
		\begin{center}
		\def\arraystretch{1.2}
		\setlength\tabcolsep{20pt}
		\begin{tabular}{|c|c|c|c|c|}\hline
			$x$ & 0 & -2 & 1 & -1 \\ \hline
			$f(x)$ & 1 & 0 & 0 & 1 \\ \hline
		\end{tabular}
		\end{center}
	Parmis les expressions algébriques suivantes, lesquelles \underline{ne peuvent pas} correspondre à $f(x)$ ?
		\begin{multicols}{4}
		\begin{enumerate}[label=\roman*)]
			\item $1-x$
			\item $1+\dfrac{x}2$
			\item $\dfrac{1-x}2$
			\item $\dfrac{-x^2 - x + 2}2$
		\end{enumerate}
		\end{multicols}
}{exe:fonctions-QCM}{
	Il s'agit de discriminer les fonctions possibles en utilisant les images du tableau.
		\begin{enumerate}[label=\roman*)]
			\item $1-x$ vaut bien $1$ en $x=0$, mais l'expression vaut $3$ en $x=-2$, donc ce n'est pas l'expression de $f$.
			\item  $1+\dfrac{x}2$ vaut bien $1$ en $x=0$ et $0$ en $x=-2$, mais elle vaut $1$ en $x=1$, ce n'est donc pas l'expression de $f$.
			\item $\dfrac{1-x}2$ vaut $\frac12$ en $x=0$, ce n'est donc pas l'expression de $f$.
			\item On déduit $f(x) = \dfrac{-x^2 - x + 2}2$ est la seule expression possible, qu'on vérifiera en calculant les images de $0, -2, 1,$ et $-1$.
		\end{enumerate}

	Cette exercice deviendra vite évident après avoir terminé l'étude des fonctions affines. 
	Comme les courbes représentatives des trois premières expressions sont des droites non constantes, toutes leurs images ont un unique antécédent.
	Or ici $0$ et $1$ ont deux antécédents.
}

\exe{}{
	Remplir le tableau d'images suivant.
		\begin{center}
		\def\arraystretch{1.2}
		\setlength\tabcolsep{20pt}
		\begin{tabular}{|c|c|c|c|c|}\hline
			$x$ & 0 & -2 & 1 & -1 \\ \hline
			$f(x)$ & 1 & 3 & 0 & \\ \hline
			$g(x)$ & 1 & 0 &  & -2 \\ \hline
			$f(x)+2g(x)$ &  & & 2 & 1 \\ \hline
		\end{tabular}
		\end{center}
}{exe:fonctions-QCM-trous}{
	\begin{center}
	\def\arraystretch{1.2}
	\setlength\tabcolsep{20pt}
	\begin{tabular}{|c|c|c|c|c|}\hline
		$x$ & 0 & -2 & 1 & -1 \\ \hline
		$f(x)$ & 1 & 3 & 0 & 5 \\ \hline
		$g(x)$ & 1 & 0 & 1 & -2 \\ \hline
		$f(x)+2g(x)$ & 3 & 3 & 2 & 1 \\ \hline
	\end{tabular}
	\end{center}
}

\exe{}{
	Montrer que $g(x) = x(x+2)(x-1)(x+1)$ vérifie $g(0) = g(-2) = g(1) = g(-1) = 0$.
	Conclure que $f(x)$ et $h(x) = f(x) + g(x)$ vérifient toutes les deux le tableau de valeurs de l'exercice \ref{exe:fonctions-QCM}.
}{exe:fonctions-QCM2}{
	Lors du calcul de chacune des images, au moins un facteur est nul : le produit vaut donc toujours zéro.
}

\exe{, difficulty=1}{
	À l'aide de l'exercice \ref{exe:fonctions-QCM2}, montrer qu'il y a une infinité de fonctions $h$ différentes vérifiant le tableau de valeurs de l'exercice \ref{exe:fonctions-QCM}-
}{exe:fonctions-QCM3}{
	On prend par exemple $h(x) = f(x) + g(x), f(x) + 2g(x), f(x) + 3g(x), \dots$.
}

\nt{
	Lorsqu'on a un nombre fini d'images, on peut invalider un candidat d'expression algébrique mais jamais le valider pour tout $x\in\R$.
}

\exe{}{
	Soit $f(x) = (x-12)(x-13)$.
	Développer l'expression pour vérifier que $f(x) = x^2 - 25x + 156$.
	
	Calculer les images de 0 ; 11 ; 12 ; 13; et 14 par $f$ avec la forme adéquate et sans calculatrice.
}{exe:image-selon-forme}{
	Pour $f(0)$ et $f(1)$, on préférera la forme développée $f(x) = x^2 - 25x + 156$.
	Ainsi, $f(0) = 156$ immédiatement, et $f(1) = 1 - 25 + 156 = 132$.
	
	Pour $f(11), f(12), f(13), f(14)$, la forme factorisée est préférable, car on calcule facilement que
		\begin{align*}
			f(11) &= (11-12)(11-13) = (-1)(-2) = 2 \\
			f(12) &= (12-12)(12-13) = 0 \\
			f(13) &= (13-12)(13-13) = 0 \\
			f(14) &= (14-12)(14-13) = (2)(1) =2
		\end{align*}
}

\nomen{
	La forme $f(x) = x^2 - 25x + 156$ est la \emphindex{forme développée réduite}.
	La forme $f(x) = (x-12)(x-13)$ est la \emphindex{forme factorisée}.
}

\nt{
	Il est toujours facile de calculer $f(0)$ avec la forme développée.
	La forme factorisée, elle, est parfois plus utile pour calculer certaines images, et est toujours plus utile pour étudier le signe de $f(x)$ (voir chapitre \ref{chap:signes}).
}

\exe{}{
	Considérons la fonction $f(x) = \frac15-x$ pour tout $x\in\R$.
	
	\begin{enumerate}
		\item
		Calculer l'image par $f$ de $0$ ; de $0,2$ ; de $\frac47$ ; de $-\frac23$.
		\item
		Donner un antécédent de $0$ par $f$.
		\item
		Déterminer tous les antécédents de $5$ par $f$.	
	\end{enumerate}
}{exe:images-antécédents2}{
	\begin{enumerate}
		\item
		Calculer l'image par $f$ de $0$ ; de $0,2$ ; de $\frac47$ ; de $-\frac23$.
		\begin{align*}
			f(0) &= \frac15 \\
			f(0,2) &= \frac15 - 0,2 = 0 \\
			f\left(\frac47\right) &= \frac15 - \frac47 = \frac{7}{35} - \frac{20}{35} = \frac{-13}{35} \\
			f\left(-\frac23\right) &= \frac15 + \frac23 = \frac{3}{15} + \frac{10}{15} = \frac{13}{15}
		\end{align*}
		\item
		0,2 en est un d'après la question précédente.
		\item
		On pose $f(x) = 5$ et on résoud pour $x$.
			\begin{align*}
				f(x) &= 5 \\
				\frac15 -x &= 5 \\
				-x &= 5-\frac15 = \frac{24}{5} \\
				x &= -\frac{24}{5}
			\end{align*}
	\end{enumerate}
}

\exe{, difficulty=1}{
	Considérons les fonctions $f(x) =x^2 -6x + 9$ et $g(x)=(x-3)^2$ pour tout $x\in\R$.
	\begin{enumerate}
		\item
		Calculer les images par $f$ et $g$ de $0$ ; de $6$ ; de $3$ ; de $-\frac23$.
		\item
		Donner deux antécédents de $9$ par $f$.
		\item
		Montrer que $f(x) = g(x)$ pour tout $x\in\R$ réel.
	\end{enumerate}
}{exe:forme-canonique}{
	\begin{enumerate}
		\item
		Pour $f$, on calcule
			\begin{align*}
				f(0) &= 9, \\
				f(6) &= 6^2 - 6^2 + 9 = 9, \\
				f(3) &= 3^2 - 6\times3 + 9 = 0, \\
				f\left(-\frac23\right) &= \left(-\frac23\right)^2 - 6\left(-\frac23\right) + 9 = \frac49 + 4 + 9 = \frac49 + \frac{117}9 = \frac{121}9.
			\end{align*}
		Pour $g$, on calcule
			\begin{align*}
				g(0) &= (-3)^2 = 9, \\
				g(6) &= 3^2 = 9, \\
				g(3) &= 0^2 = 0, \\
				g\left(-\frac23\right) &= \left(-\frac23 - 3\right)^2= \left(-\dfrac{11}3\right)^2 = \frac{121}9.
			\end{align*}
		\item
		0 et 6 sont deux antécédents de 9 par $f$.
		\item
		On part toujours de la forme factorisée ($g$ ici) pour arriver vers la forme développée réduite ($f$ ici).
			\begin{align*}
				g(x) &= (x-3)^2 \\
					&= (x-3)(x-3) \\
					&= x(x-3) - 3(x-3) \\
					&= x^2 - 3x - 3x + 9 \\
					&= x^2 - 6x + 9
			\end{align*}
		Le développement de carrés parfaits deviendront immédiates après l'étude des identités remarquables.
	\end{enumerate}
}

\section{Intervalles}

\dfn{intervalle borné}{
	Un \emphindex{intervalle} borné est un segment de la droite réelle $\R$. C'est donc un ensemble de nombres.
	Il est donné par une \emphindex{borne inférieure} $a \in \R$ et une \emphindex{borne supérieure} $b\in\R$ et peut contenir ou non ses bornes.
	
	\begin{enumerate}
		\item Si $a$ et $b$ sont contenues dans l'intervalle, on le note $[a ; b]$.
		\item Si $a$  est contenue dans l'intervalle mais $b$ ne l'est pas, on le note $[a ; b [$.
		\item Si $a$ n'est pas contenue dans l'intervalle mais $b$ l'est, on le note $] a ; b]$.
		\item Si ni $a$ ni $b$ ne sont contenues dans l'intervalle, on le note $] a ; b [$.
	\end{enumerate}
}{}

\ex{}{
	Les intervalles suivants sont bornés.
	\begin{multicols}{2}
	\begin{enumerate}[label=$\bullet$]
		\item $[-1 ; 1]$
		\item $[-3 ; 1[$
		\item $\left]-10{,}341 ; \pi\right]$
		\item $\left]\sqrt{2} ; 130\right[$
	\end{enumerate}
	\end{multicols}

}{}

\dfn{intervalle non borné}{
	Un intervalle n'est pas forcément borné : une ou les deux bornes peuvent être infinies ($\pinfty$ ou $\minfty$). 
	Dans ce cas, l'intervalle n'inclut jamais l'infini car ce n'est pas un nombre.
}{}

\ex{}{
	Les intervalles suivants ne sont pas bornés.
	\begin{multicols}{2}
	\begin{enumerate}[label=$\bullet$]
		\item $]\minfty; 2]$
		\item $]\minfty ; 3[$
		\item $]0; \pinfty[$
		\item $\R = ]\minfty; \pinfty[$
	\end{enumerate}
	\end{multicols}
}{ex:3.5}

\nt{
	Un intervalle n'a de sens que si la borne inférieure est plus petite que la borne supérieure.
	De plus, un élément appartient à l'ensemble dès qu'il est plus petit que la borne supérieure, et plus grand que la borne inférieure.
	Il faut donc pouvoir noter simplement les relations \og plus petit que \fg et \og plus grand que \fg.
}

\notations{
	On définit les signes suivants correspondant à des inégalités \emphindex{strictes} et \emphindex{larges}.
	\begin{multicols}{2}
	\begin{enumerate}[leftmargin=50pt]
		\item[$<$ :] strictement inférieur à
		\item[$\leq$ :] inférieur ou égal à
		\item[$>$ :] strictement supérieur à
		\item[$\geq$ :] supérieur ou égal à
	\end{enumerate}
	\end{multicols}
	On attire l'attention sur le fait que le « ou » est inclusif : $x \leq x$ est toujours vrai. 
}

\ex{}{
	Les inégalités suivantes sont vraies. 
	\begin{multicols}{3}
	\begin{enumerate}[label=\roman*)]
		\item $1 \leq 2$
		\item $-3 \leq -2$
		\item $0 \leq 0$
		\item $1{,}02 < 1{,}1$
		\item $7{,}391 > 7{,}30001$
		\item $-4{,}001 > -4{,}0001$
	\end{enumerate}
	\end{multicols}
}{}

\nt{
	Appartenir à un intervalle est équivalent à être inférieur à la borne supérieure, et être supérieur à la borne inférieure.
	On a donc l'équivalence suivante.
		\[ x \in [a ; b] \qquad \iff \qquad x \in \R, x \geq a, \textbf{ et } x \leq b. \]
}

\notations{
	On peut noter deux inégalités sur la même lignes dès qu'elles vont dans le même sens.
	Par exemple, les deux inégalités ci-dessus peuvent être condensées en une :
		\[ x \in \R, \text{ et } a \leq x \leq b, \]
	ou encore
		\[ x \in \R, \text{ et } b \geq x \geq a. \]
}

\ex{}{
	\begin{enumerate}
		\item Prendre $x \in [-3 ; 4]$ est équivalent à prendre $x \in \R$ vérifiant $-3 \leq x \leq 4$.
		\item Prendre $x \in [-4 ; 3[$ est équivalent à prendre $x \in \R$ vérifiant $-4 \leq x < 3$.
		\item Prendre $x \in ]\minfty ; 0[$ est équivalent à prendre $x \in \R$ vérifiant $x < 0$.
		
		On dit alors que $x$ est strictement négatif.
		\item Prendre $x \in [0; \pinfty [$ est équivalent à prendre $x \in \R$ vérifiant $x \geq 0$.
		
		On dit alors que $x$ est positif ou nul.
		\item Prendre $x \in ]\minfty; \pinfty[$ est équivalent à prendre $x \in \R$.	
		
		Ceci est en fait tautologique car $]\minfty; \pinfty[ = \R$, comme vu dans l'exemple \ref{ex:3.5}.
	\end{enumerate}
}{}

\section{Représentation graphique}

\subsection{Courbe représentative}

Pour chaque $x\in\D\subset\R$, élément du domaine de $f$, il faut pouvoir facilement lire son image $f(x)$ par $f$.
À cette fin et pour chaque $x\in\D$, on crée un point d'abscisse $x$ et d'ordonnée $f(x)$.
L'ensemble de ces points est la courbe représentative de $f$.
Pour lire l'image de $x$ par $f$, il suffit alors de trouver l'ordonnée de l'unique point de la courbe d'abscisse $x$.

\dfn{courbe représentative}{
	Considérons $f : \D \rightarrow \R$ une fonction.
	
	La \emphindex{courbe représentative} de $f$, notée $\C_f$, est donnée par l'ensemble de points
		\[ \C_f = \Bigset{ \bigpar{x ; f(x)} \text{ où $x$ parcourt } \D }. \]
	On lit \og la courbe représentative de $f$ est l'ensemble des points $\bigl(x,f(x)\bigr)$ du plan où $x$ parcourt le domaine de $f$ \fg.
}{dfn:Cf}	

\nt{
	Connaître $f$ sur son domaine $\D$ c'est connaître $\C_f$, et vice versa.
	Cependant, il n'est pas toujours possible de dessiner $\C_f$ si le domaine $\D$ n'est pas borné, ou si $f$ prend des valeurs toujours plus grandes.
}

\cor{propriété fondamentale}{
	On a donc la \emphindex{propriété fondamentale} suivante valable pour tout $x, y\in\R$.
		\begin{align*}
			(x;y) \in \C_f && \iff && y = f(x).
		\end{align*}
		
	%\emph{Note : $f(x) = \dfrac1{200}(x+3,5)(x+2,5)(x-3,5)(x-4,5)(x-5,5) + 2$, polynôme de degré $5$ dans l'exemple ci-dessus.}
}{cor:prop-fond}

\begin{figure}[h!]
	\begin{center}
	 \includegraphics[page=2, scale=1.5]{figures/fig-fonctions.pdf}
	\end{center}
	\caption{Représentation de la propriété fondamentale du corollaire \ref{cor:prop-fond}.}
\end{figure}

\nomen{
	On appelle \emphindex{graphe} la représentation graphique d'une fonction.
}

\ex{}{
	Soit la fonction $f$ donnée algébriquement par
		\[ f(x) = 3x^3 -x - 3, \]
	et considérons les points $P(0;-3), Q(1; 1), R(-1; 1),$ et $S(2; 19)$.
	On se demande si les points appartiennent à la courbe représentative de $f$ ou non.
	
	D'après la propriété fondamentale \ref{cor:prop-fond}, un point $(x;y)$ appartient à la courbe de $f$ si et seulement si l'équation $y=f(x)$ est vérifiée.
	On a donc les (non) appartenances suivantes.
	\begin{multicols}{2}
		\begin{enumerate}
			\item $f(0) = -3$, et donc $P \in \C_f$.
			\item $f(1) = -1 \neq 1$, donc $Q \not\in \C_f$.
			\item $f(-1) = -5 \neq 1$, donc $R \not\in\C_f$.
			\item $f(2) = 19$, et donc $S\in\C_f$.
		\end{enumerate}
		\vfill
		
		\begin{center}
		 \includegraphics[page=3, scale=1.1]{figures/fig-fonctions.pdf}
		\end{center}
	\end{multicols}
}{}


\exe{}{
	Considérons la fonction $f$ sur $\R$ donnée algébriquement par
		\[ f(x) = \dfrac17-x. \]
	Pour chaque point suivant, déterminer s'il appartient à $\C_f$ ou non.
	
	\begin{multicols}{3}
	\begin{enumerate}[label=\roman*)]
		\item $\left(0; \frac17\right)$
		\item $\left(-\frac17 ; 0\right)$
		\item $\left(\frac27 ; -\frac17\right)$
	\end{enumerate}
	\end{multicols}

}{exe:Cf}{
	On rappelle la propriété fondamentale
		\begin{align*}
			(x;y) \in \C_f && \iff && y = f(x)
		\end{align*}
	Pour savoir si un point $(x;y)$ appartient à $\C_f$, il s'agit de vérifier si l'égalité $y=f(x)$ tient.
	
	\begin{enumerate}[label=\roman*)]
		\item On applique la propriété pour $x = 0, y= \frac17$.
		D'une part, $f(x) = f(0) = \frac17 - 0 = \frac17$, et d'autre part $y=\frac17$.
		On a donc bien $y=f(x)$ pour ce couple, et il appartient à $\C_f$.
			\[ \left(0; \dfrac17\right) \in \C_f. \]
		
		\item On choisit $(x;y) = \left(-\frac17 ; 0\right)$, et on compare $f(x) = f\left(-\frac17\right) = \frac27$ à $y=0$. 
		On a donc
			\[ \left(-\dfrac17 ; 0\right) \not\in \C_f. \]
		On a donc bien
			\[ \left(\dfrac17 ; 0\right) \in \C_f. \]
		\item On choisit $(x;y) = \left(\frac27 ; -\frac17\right)$, et on compare $f(x) = f\left(\frac27\right) = \frac{-1}7$ à $y=-\frac17$. 
		D'où
			\[ \left(\dfrac27 ; -\dfrac17\right) \not\in \C_f. \]
	\end{enumerate}
}

% for later maybe :)
%\exe{}{
%	Considérons deux fonctions $f, g$ sur $\D = ]-3 ; 3[$ données algébriquement par
%		\begin{align*}
%			f(x) = x^2 - 2x && g(x) = (x-1)^2
%		\end{align*}
%	
%	\begin{enumerate}
%		\item Esquisser les représentations graphiques de $f$ et de $g$ dans un même repère.
%		\item Démontrer que $g(x) = f(x) + 1$ pour tout $x$ du domaine.
%	\end{enumerate}
%}{exe:id-rem-graph}{
%	
%	\begin{enumerate}
%		\item On choisit plusieurs valeurs de $x\in]{-}3 ; 3[$ et on représente les points
%			\[ \bigl(x ; f(x) \bigr), \]
%		qu'on relie pour esquisser $\C_f$.
%		On fait idem pour $g$, ce qui donne le graphique ci-dessous.
%		
%		\begin{center}
%		\begin{tikzpicture}[>=stealth]
%			\begin{axis}[xmin = -3.1, xmax=3.1, ymin=-1.1, ymax=15.1, axis x line=middle, axis y line=middle, axis line style=->, grid=both]
%				\addplot[no marks, PURPLE_E, very thick, -] expression[domain=-3:3, samples=100]{x^2 -2*x}
%				node[pos=.45, below=10pt]{$\mathcal{C}_f$};
%				\addplot[no marks, BLUE_E, very thick, -] expression[domain=-3:3, samples=100]{(x-1)^2}
%				node[pos=.4, right]{$\mathcal{C}_g$};
%			\end{axis}
%		
%		\end{tikzpicture}
%		\end{center}
%		
%		
%		\item 
%		L'identité remarquable $a^2 - b^2 = (a+b)(a-b)$ donne en l'occurrence
%			\begin{align*}
%				(x-1)^2 - 1^2 &= (x-1+1) \cdot (x-1-1) \\
%								&= x \cdot (x-2) \\
%								&= x^2 - 2\cdot x = f(x).
%			\end{align*}
%			
%		On déduit alors que 
%			\[ (x-1)^2 = g(x) = f(x) + 1 = x^2 - 2\cdot x + 1 \]
%		pour tout $x\in]{-}3;3[$.
%	\end{enumerate}
%	
%}

\exe{}{
	Esquisser la courbe de la fonction $f$ sur $\D=[-2; 4]$ donnée algébriquement par
		\[ f(x) = 3. \]
	Que dire de $f$ et de $\C_f$ ?
}{exe:graph-const}{
	Pour tracer $\C_f$, on choisit des valeurs de $x$ du domaine $[{-2};4]$, on calcule $f(x)$, et on place les points $(x ; f(x))$ obtenus.
	
		\begin{center}
		\includegraphics[page=15]{figures/fig-fonctions.pdf}
		\end{center}
	
	On dit ici que $f$ est \emph{constante} car elle ne dépend pas de $x$.
	$\C_f$ est donc une droite horizontale.
}

\exe{}{
	Esquisser la courbe de la fonction $f$ sur $\D=[-5;3]$ donnée algébriquement par
		\[ f(x) = 1-x. \]
	Que dire de $\C_f$ ?
}{exe:graph-droite}{		
	$\C_f$ est une droite.
	\begin{center}
	\includegraphics[page=16]{figures/fig-fonctions.pdf}
	\end{center}
		
}

\exe{}{
	Esquisser la courbe de la fonction $f$ sur $\D=[3;10]$ donnée algébriquement par
		\[ f(x) = \dfrac3x + 1. \]
}{exe:graph-droite2}{
	Il faut choisir suffisamment de $x \in [3;10]$ du domaine afin de pouvoir bien tracer l'allure de la courbe.
	Attention, celle-ci n'est pas une droite !
	
		\begin{center}
		\includegraphics[page=17]{figures/fig-fonctions.pdf}
		\end{center}
}

\newpage

\exe{}{
	Considérons la représentation graphique suivante d'une fonction $f$ définie sur $\D = ]{-}3,4 ; 2,3[$.
	
	\begin{center}
	\includegraphics[page=9]{figures/fig-fonctions.pdf}
	\end{center}
	\begin{enumerate}
		\item\label{q1}
		Donner approximativement les images de -1 et de 2 par $f$. 
		\item Énumérer approximativement les antécédents de 2 et de 4 par $f$.
		\item Exprimer aproximativement l'ensemble $\{ x \in \D \tq f(x) \geq 4 \}$ sous forme d'intervalle.
		\item Donner approximativement un réel qui admet exactement deux antécédents par $f$.
		\item Si $f$ était définie sur $\R$ tout entier, serait-il toujours possible de connaître l'image de -2 ? Et tous les antécédents de -2 ?
	\end{enumerate}
	Supposons désormais que $f(x) = 3-2 x +\frac13 x^3$ pour tout $x\in\D$ du domaine.
	\begin{enumerate}[resume]
		\item Répondre à nouveau à la question \ref{q1} à l'aide de l'expression algébrique de $f$.
		Des valeurs exactes sont attendues.
		\item Montrer que l'image de $-3$ par $f$ est $0$ et que l'image de $0$ par $f$ est $3$.
	\end{enumerate}
}{exe:deg3}{
	\begin{enumerate}
		\item On détermine approximativement
			\begin{align*}
			f(-1) \approx 4,5 && \et && f(2) \approx 1,5.
			 \end{align*}
		\item On cherche d'abord les antécédents de 2, c'est-à-dire les nombres $x$ du domaine vérifiants
			\[ f(x) = -2. \]
		Pour ça, on se pose \og à hauteur 2 \fg : on tracer une droite horizontale d'ordonnée 2 et on regarde les points d'intersection.
			\begin{center}
			\includegraphics[page=18]{figures/fig-fonctions.pdf}
			\end{center}
		
		
		En l'occurrence, seuls $x\approx -2,8 ; 0,5 ;  \et 2,1$ fonctionnent.
			
		Pour les antécédents de 4, on fait idem en regardant les points de la droite d'ordonnée 4.
			\begin{center}
			\includegraphics[page=19]{figures/fig-fonctions.pdf}
			\end{center}
		On trouve trois valeurs approximatives : $x \approx -2,2$ et $x \approx 0,5$.
		
		\item
		On regarde quels antécédents $x \in \D$ du domaine d'étude ont une image supérieure ou égale à 4.
		En reprenant le graphe de la résolution de $f(x) = 4$, on remarque que \textbf{tous} les $x$ entre -2,2 et 0,5 ont une image supérieure à 4.
		Ainsi, 
			\[ \bigset{ x \in \D \tq f(x) \geq 4 } \approx [-2,2 ; 0,5], \]
		les bornes étant incluses car l'inégalité est large.
		
		\item 
		On a plusieurs choix ici. 
		Il faut placer une droite horizontale telle qu'elle s'intersecte exactement deux fois avec la courbe de $f$.
		Un choix clair est $4$ (en violet ci-dessous).
		Un choix moins clair est $1,1$, en faisant en sorte que la courbe frôle la droite horizontale qu'on place en ordonnée $1,1$ (en rouge ci-dessous).
		On dit alors que la droite est \emph{tangente} à la courbe.
		
			\begin{center}
			\includegraphics[page=20]{figures/fig-fonctions.pdf}
			\end{center}
		
		\item L'image est unique et ne dépend pas du domaine (tant que celui-ci contient $-2$ !).
		On peut donc toujours connaître $f(-2)$, même sans connaître $f$ en dehors du domaine.
		
		Les antécédents, eux, dépendent du domaine choisi.
		Comme on ne sait pas du tout à quoi ressemble $\C_f$ en dehors du domaine choisi, il est impossible de déterminer tous les réels $x\in\R$ antécédents de $-2$ par $f$.
		
		\item On calcule grâce à la calculatrice (ou sans...)
			\begin{align*}
				f(-1) = \dfrac{14}3 && f(2) =  \dfrac53.
			\end{align*}
		On peut également utiliser le programme Python ci-après (la notation \texttt{x**3} signifiant $x^3$).
	
		\begin{center}
		\python{images-f}
		\end{center}
		
		\item On calcule à la main que
			\begin{align*}
				f(-3) &= 3 - 2 \cdot (-3) + \dfrac13 \cdot (-3)^3, \\
					&= 3 + 6 - 9, \\
					&= 0.
			\end{align*}
		Pour l'image de $0$, remarquons que seul le terme ne dépendant pas de $x$ subsiste. 
		On l'appelle le terme \emph{constant}, et on trouve $f(0) = 3$, comme requis.
	\end{enumerate}
}


\exemulticols{}{
	Un fonction $f$ est représentée graphiquement ci-contre.
	Parmis les expressions algébriques suivantes, lequelles ne peuvent pas correspondre à $f(x)$ ?
		\begin{multicols}{2}
		\begin{enumerate}[label=\roman*)]
			\item $1-x$
			\item $\dfrac{-1-x}3$
			\item $\left(x+\dfrac13\right)^2$
			\item $-2x - \dfrac23$
		\end{enumerate}
		\end{multicols}
		\vfill\null
}{
	\begin{center}
	\includegraphics[page=21]{figures/fig-fonctions.pdf}
	\end{center}
}{exe:expr-from-graph}{
	Clairement $f(0) < 0$ est strictement négatif.
	L'expression de $f$ ne peut donc pas être la première ni la troisième.
	
	Ensuite, si l'expression de $f$ était la deuxième, on aurait $f(-1) = 0$, ce qui n'est clairement pas le cas.
	Ainsi $f(x) = -2x - \frac23$ est le seul choix possible.
}

\newpage

\exe{, difficulty=1}{
	Utiliser le repère ci-dessous pour comparer les représentations graphiques des trois fonctions suivantes données algébriquement.
		\begin{align*}
			f(x) = x^2, && g(x) = x^2 - 3, && h(x) = (x+4)^2.
		\end{align*}
		
	\begin{center}
	\includegraphics[page=22]{figures/fig-fonctions.pdf}
	\end{center}
	
	\begin{enumerate}[label=(\alph*)]
		\item
		Estimer, grâces aux courbes, les solutions $x\in\R$ des équations $f(x) = h(x)$ et $h(x) =g(x)$.
		\item
		Estimer, grâces aux courbes, à quel intervalle appartiennent les $x\in\R$ vérifiant $h(x) \leq f(x)$.
	\end{enumerate}
}{exe:y-shift}{
	On graphe les fonctions ci-dessous. 
	On remarque que $\C_g$ est juste en dessous de $\C_f$. C'est logique car
		\[ g(x) = f(x) - 3, \]
	donc quand on place les points pour $\C_f$ et $\C_g$, ceux de $\C_g$ sont $3$ en dessous de ceux de $\C_f$.
	
	On remarque aussi que $\C_h$ est la courbe de $\C_f$ décalée vers la gauche.
	C'est également cohérent car
		\[ h(x) = f(x+4). \]
	Pour connaître l'image de $x$ par $h$, on prend l'image de $x+4$ par $f$.
	On a donc par exemple $h(-4) = f(0), h(3,9) = f(0,1),$ etc... 
	On comprend bien pourquoi $\C_h$ est la $\C_f$ décalée de $4$ vers la gauche.
	
		\begin{center}
		\includegraphics[page=23]{figures/fig-fonctions.pdf}
		\end{center}
}

\exe{, difficulty=2}{
	Donner une fonction réelle et un domaine telle qu'une de ses images admet
		\begin{multicols}{2}
		\begin{enumerate}
			\item exactement un antécédent
			\item exactement deux antécédents
			\item exactement trois antécédents
			\item une infinité d'antécédents
		\end{enumerate}	
		\end{multicols}
}{exe:nb-antécédents}{
	\begin{enumerate}
		\item La fonction affine $f(x) = x$ donne une droite dont chaque image admet un unique antécédent.
		\item La fonction affine $f(x) = x^2$ donne une parabole. En résolvant $f(x) =1$, on peut démontrer que seuls $1$ et $-1$ sont les antécédents de $1$.
		\item Considérons $f(x) = (x-1)\cdot(x-2)\cdot(x-3)$. Résoudre $f(x)=0$ pour montrer que seuls $1 ; 2 ;$ et $3$ sont antécédents de $0$.
		\item Une fonction constante fonctionne bien. Par exemple $f(x) = 0$.
	\end{enumerate}
}

\exe{, difficulty=2}{
	Donner graphiquement une fonction sur $\R$ non constante telle que toutes les images de $f$ admettent un nombre infini d'antécédents.
}{exe:infinité-antécédents}{
	On peut définir par exemple la fonction \emph{signe} en incluant 0 dans les positifs (traditionnellement, $\text{signe}(0) = 0$).
		\[ \text{signe}(x) = \begin{cases*} +1 & si $x \geq 0$, \\
								-1 & si $x<0$.
				\end{cases*}. \]
	Pour une fonction plus intéressante, on pourrait prendre une fonction en vague.
	La fonction sinus fonctionne bien : entrer par exemple \texttt{y=sin(x)} sur Geogebra.
}

\newpage

\subsection{Résolution graphique d'(in)équations}

%%%%%
% en fait f = k et f < k sont des cas particuliers de f=g et f<g.
% mais c'est peut-être pousser un peu trop loin ?
%%%%%

%\subsection*{Résoudre graphiquement $f(x) = k$}
%
%
%On cherche à résoudre graphiquement une équation du type $f(x)=k$ pour un $k\in\R$ donné, c'est-à-dire trouver l'ensemble des $x\in\R$ pour lesquels l'équation est vérifiée.
%En reformulant, il s'agit de trouver l'ensemble des antécédents de $k$ par $f$. 
%
%En reprenant la solution de l'exercice \ref{exe:deg3}, on peut généraliser la stratégie à employer.
%Supposons qu'on souhaite résoudre $f(x) = 2$ où $x\in]{-}3,4 ; 2,3[$ et $\C_f$ est donnée graphiquement.
%Comme chaque point de $\C_f$ est de la forme $\bigl(x; f(x)\bigr)$, on souhaite trouver les points de la forme $(x;2)$ pour ensuite lire leur abscisse et enfin obtenir les différents $x$ du domaine.
%Pour trouver tous les points d'ordonnée $2$, on peut s'aider en créant une droite horizontale d'ordonnée $2$, comme ci-dessous.
%
%\begin{figure}[h]
%	\begin{center}
%	\includegraphics[page=8, scale=1.25]{figures/fig-fonctions.pdf}
%	\end{center}
%	\caption{Résolution graphique de l'équation $f(x) = 2$}
%\end{figure}
%Les coordonnées des points $A,B,C$ trouvés en intersectant la droite horizontale avec $\C_f$ sont
%	\begin{align*}
%		A \approx (-2,67 ; 2) && B \approx (0,52 ; 2) && C \approx (2,14 ; 2)
%	\end{align*}
%L'ensemble de solutions de l'équation $f(x)=2$ est donc approximativement donné par 
%	\[ \bigset{ -2,67 ; 0,52 ; 2,14 }. \]
%
%\thm{}{
%	Soit $f:\D \rightarrow \R$ une fonction réelle définie sur son domaine $\D$.
%	Pour tout réel $k$, les solutions dans $\D$ de l'équation $f(x)=k$ sont les abscisses des points d'intersection de $\C_f$ et de la droite d'équation $y=k$.
%}{}
%
%
%\exemulticols{}{
%	Donner l'ensemble des $x$ vérifiant $f(x) = -2$ à l'aide du graphe de $f$ ci-contre.
%}{
%	TODO graph
%}{exe:fx=k}{
%	TODO
%}
%
%
%\newpage
%\subsection*{Résoudre graphiquement $f(x) < k$}
%
%
%On cherche à résoudre graphiquement une équation du type $f(x)\leq k$ pour un $k\in\R$ donné, c'est-à-dire trouver l'ensemble des $x\in\R$ pour lesquels l'inéquation est vérifiée.
%
%On étudie à nouveau la courbe de l'exercice \ref{exe:deg3} pour généraliser la stratégie à employer.
%Supposons qu'on souhaite résoudre $f(x) \leq 2$ où $x\in]{-}3,4 ; 2,3[$ et $\C_f$ est donnée graphiquement.
%Comme chaque point de $\C_f$ est de la forme $\bigl(x; f(x)\bigr)$, on souhaite trouver les points d'ordonnée inférieure à $2$ pour ensuite lire leur abscisse et enfin obtenir les différents $x$ du domaine.
%Pour trouver tous les points d'ordonnée inférieure à $2$, on peut s'aider en créant une droite horizontale d'ordonnée $2$, comme ci-dessous.
%
%\begin{figure}[h]
%	\begin{center}
%	 \includegraphics[page=5, scale=1.25]{figures/fig-fonctions.pdf}
%	\end{center}
%	\caption{Résolution graphique de l'équation $f(x) \leq 2$}
%\end{figure}
%
%Les points de $\C_f$ d'ordonnée inférieure à $2$ ont leur abscisse appartenant à l'union d'intervalles
%	\[ ] {-}3,4 ; -2,67 ] \cup [0,52 ; 2,14]. \]
%C'est notre ensemble de solutions de l'équation $f(x) \leq 2$.
%
%\thm{}{
%	Soit $f:\D \rightarrow \R$ une fonction réelle définie sur son domaine $\D$.
%	Pour tout réel $k$, les solutions dans $\D$ de l'équation $f(x) \leq k$ sont les abscisses des points de $\C_f$ situés en dessous de la droite d'équation $y=k$.
%}{}
%
%\exemulticols{}{
%	Donner l'ensemble des $x$ vérifiant $f(x) < -2$ à l'aide du graphe de $f$ ci-contre.
%}{
%	TODO graph
%}{exe:fx=k}{
%	TODO
%}
%
%\newpage
\subsection*{Résoudre graphiquement $f(x) = g(x)$}

On cherche à résoudre graphiquement une équation du type $f(x)=g(x)$, c'est-à-dire trouver l'ensemble des $x\in\R$ pour lesquels l'équation est vérifiée.

Soient $f, g$ sur $\D = ]{-}3,4 ; 2,3[$ deux fonctions réelles dont les courbes représentatives sont données ci-dessous.

\begin{figure}[h]
	\begin{center}
	 \includegraphics[page=6, scale=1.25]{figures/fig-fonctions.pdf}
	\end{center}
	\caption{Résolution graphique de l'équation $f(x) = g(x)$}
\end{figure}
	
Imaginons un nombre réel $x\in\R$ parcourant $]{-}3,4 ; 2,3[$ en partant de la gauche et allant vers la droite.
Pour chaque $x$, le point de $\C_f$ d'abscisse $x$ est d'ordonnée $f(x)$, par définition.
Idem pour le point de $\C_g$ d'abscisse $x$ : il est d'ordonnée $g(x)$.

Ainsi quand $x$ traverse le domaine, ces deux points sont confondus si et seulement si ces deux points sont de même ordonnée, c'est-à-dire si et seulement si $f(x) = g(x)$.		
	
Les solutions $x\in]{-}3,4 ; 2,3[$ de l'équation $f(x) = g(x)$ sont donc données approximativement par l'ensemble
	\[ \bigset{ -0,8 ; 1,2 }. \]

\thm{}{
	Soient $f,g$ deux fonctions réelles définies sur un même domaine $\D$.
	Les solutions dans $\D$ de l'équation $f(x) = g(x)$ sont les abscisses des points d'intersection de $\C_f$ et $\C_g$.
}{}

\exemulticols{}{
	Donner l'ensemble des \mbox{$x \in [-2,5 ; 2,3]$} vérifiant $f(x) = g(x)$ à l'aide des graphes de $f$ et $g$ ci-contre.
}{
	\begin{center}
	\includegraphics[page=13]{figures/fig-fonctions.pdf}
	\end{center}
}{exe:f=g}{
	On trouve approximativement $x\approx-2,2$ ; $x\approx0,5$ et $x \approx 2,1$.
}

\newpage
\subsection*{Résoudre graphiquement $f(x) \leq g(x)$}

On cherche à résoudre graphiquement une équation du type $f(x)=g(x)$, c'est-à-dire trouver l'ensemble des $x\in\R$ pour lesquels l'inéquation est vérifiée.

Soient $f, g$ sur $\D = ]{-}3,4 ; 2,3[$ deux fonctions réelles dont les courbes représentatives sont données ci-dessous.

\begin{figure}[h]
	\begin{center}
	\includegraphics[page=7, scale=1.25]{figures/fig-fonctions.pdf}
	\end{center}
	\caption{Résolution graphique de l'équation $f(x) \leq g(x)$}
\end{figure}

Pour chaque $x$ parcourant $]{-}3,4 ; 2,3[$, la hauteur (l'ordonnée) du point de $\C_f$ d'abscisse $x$ est donnée par $f(x)$, et celle du point de $\C_g$ d'abscisse $x$ est donnée par $g(x)$.
Ainsi, le premier point est en dessous du second dès que $f(x) \leq g(x)$.

L'ensemble des solutions de l'inéquation $f(x) \leq g(x)$ est donc donné approximativement par
	\[ [{-}0,8 ; 1,2 ]. \]

\thm{}{
	Soient $f, g$ deux fonctions réelles définies sur un même domaine $\D$.
	Les solutions dans $\D$ de l'équation $f(x) \leq g(x)$ sont les abscisses des points de $\C_f$ situés en dessous de $\C_g$.
}{}

\notations{
	Soient deux intervalles $I, J \subseteq \R$. On note $I \cup J$ l'\emphindex{union} des ensembles $I$ et $J$ : ce sont les nombres réels qui appartiennent à $I$ \underline{ou} à $J$.
	\begin{center}
	\includegraphics[page=12, scale=1.25]{figures/fig-fonctions.pdf}
	\end{center}
}

\exemulticols{}{
	Donner l'ensemble des \mbox{$x \in [-2,5 ; 2,3]$} vérifiant $f(x) \leq g(x)$ à l'aide des graphes de $f$ et $g$ ci-contre.
	Une réponse sous forme d'union d'intervalles est attendue.
}{
	\begin{center}
	\includegraphics[page=13]{figures/fig-fonctions.pdf}
	\end{center}
}{exe:f<g}{
	On obtient approximativement $[-2,5 ; -2,2] \cup [0,5 ; 2,1]$.
	Les bornes des intervalles correspondent aux antécédents vérifiant $f(x) = g(x)$, trouvés à l'exercice \ref{exe:f=g}.
}


\exemulticols{, difficulty=1}{
	Considérons deux fonctions, $g$ et $h$, données graphiquement sur $\D = ]{-}3,4 ; 2,3[$ ci-contre.
	
	Donner approximativement l'ensemble des nombres $x$ du domaine vérifiant les (in)équations suivantes.
	\begin{enumerate}
		\itemsep1em 
		\item $g(x) = h(x)$
		\item $g(x) \leq h(x)$
		\item $h(x) \leq g(x)$
	\end{enumerate}	
}{
	\begin{center}
	\includegraphics[page=24]{figures/fig-fonctions.pdf}
	\end{center}
}{exe:eq-ineq}{
	\begin{enumerate}
		\item On cherche ici l'ensemble
			\[ \bigset{ x \in \D \text{ tels que } g(x) = h(x) }. \]
		On lit (approximativement) les abscisses des points de $\C_g \cap \C_h$, l'intersection des deux courbes :
			\[ \bigset{ x \in \D \tq g(x) = h(x) } \approx \bigset{ {-0,8}, {1,2} } \]
		\item On cherche ici l'ensemble
			\[ \bigset{ x \in \D \tq g(x) \leq h(x) }. \]
		On lit (approximativement) les abscisses lorsque les la courbe $\C_g$ est sous $\C_h$ :
			\[ \bigset{ x \in \D \tq g(x) \leq h(x) } \approx [ {-0,8} ; {1,2} ]. \]
		\item On cherche ici l'ensemble
			\[ \bigset{ x \in \D \tq h(x) \leq g(x) }. \]
		On lit (approximativement) les abscisses lorsque les la courbe $\C_h$ est sous $\C_g$.
			\[ \bigset{ x \in \D \tq g(x) \leq h(x) } \approx [{-3,4} ; {-0,8} ] \cup [ {1,2} ; {2,3} ]. \]
	\end{enumerate}
	
	Remarquons en outre les relations suivantes :
		\[ \bigset{ x \in \D \tq h(x) \leq g(x) } \cup \bigset{ x \in \D \tq h(x) \geq g(x) } = \D, \]
	ce qui n'est pas  surprenant car pour tout $x\in\D$, on a forcément $g(x) \leq h(x)$ \textbf{ou} $g(x) \geq h(x)$. 
	Ainsi, 
		\[ \bigset{ x \in \D \tq h(x) \leq g(x) \ou  h(x) \geq g(x) }  = \bigset{ x \in \D } = \D. \]
	C'est donc pour cela que les ensembles sont complémentaires par rapport à $\D$ : leur union est $\D$ tout entier.
	
	En outre, les points appartenant aux deux ensembles à la fois sont
		\begin{align*}
		 \bigset{ x \in \D \tq h(x) \leq g(x) \et h(x) \geq g(x) } \bigset{ x \in \D \tq h(x) = g(x) },
		 \end{align*}
	car $h(x) \leq g(x)$ \textbf{et} $h(x) \geq g(x)$ est équivalent à $g(x) = h(x)$ pour tout $x\in\D$.
	
	C'est donc pour ça que les bornes des intervalles étudiés sont les abscisses des points d'intersection.
	Ces points sont exactement les moments où une courbe passe au-dessus d'une autre.
}


%\newpage
\subsection{Fonctions parentes : transformations des ordonnées}

À partir de la courbe d'une fonction qu'on connaît bien, on peut déduire les courbes de toute une famille de fonctions apparentées.

\ex{ajout d'une constante}{
	Considérons $f$ une fonction quelconque sur $\R$ et $g$ définie par
		\[ g(x) = f(x)+1. \]
	Pour calculer $g(0)$, on utilise donc la définition $g(0) = f(0) + 1$.
	De façon identique, $g(-3) = f(-3) + 1$, $g(12) = f(12) + 1$, etc…
	
	D'un point $\bigpar{x; f(x)}$ de la courbe $\C_f$, on peut donc en déduire le point $\bigpar{x ; g(x)} = \bigpar{ x ; f(x)+1 }$ de la courbe $\C_g$.
	Graphiquement, la courbe $\C_f$ est translatée de $1$ unité vers le haut pour obtenir $\C_g$
	
	Pour $h(x) = f(x) - 2$, on obtient similairement les graphes de la figure \ref{fig:parentes-plus}.
}{ex:parentes-plus}

\begin{figure}[h]
	\begin{center}
	\includegraphics[page=10, scale=1.25]{figures/fig-fonctions.pdf}
	\end{center}
	\caption{Fonctions parentes de l'exemple \ref{ex:parentes-plus}.}
	\label{fig:parentes-plus}
\end{figure}

\exemulticols{}{
	Tracer les graphes des fonctions $g(x) = f(x) + 2$ et $h(x) = f(x) - 1$ dans le repère contenant $\C_f$ ci-contre.
}{
	\begin{center}
	\includegraphics[page=14]{figures/fig-fonctions.pdf}
	\end{center}
}{exe:draw-plusc}{
	\begin{center}
	\includegraphics[page=25]{figures/fig-fonctions.pdf}
	\end{center}
}

%\newpage

\ex{multiplication par une constante}{
	Considérons $f$ une fonction quelconque sur $\R$ et $g$ définie par
		\[ g(x) = 2f(x). \]
	Pour calculer $g(0)$, on utilise donc la définition $g(0) = 2f(0)$.
	De façon identique, $g(-3) = 2f(-3)$, $g(12) = 2f(12)$, etc…
	
	D'un point $(x ; f(x))$ de $\C_f$, on double son ordonnée (sa hauteur) pour obtenir $(x ; 2f(x)) = (x ; g(x))$, le point de $\C_g$ d'abscisse $x$.
	C'est ce qu'on appelle une homothétie : on agrandit ou réduit l'ordonnée de chaque point d'un même facteur.
	Si ce facteur est négatif, on fait en plus une symétrie par rapport à l'axe des abscisses.
	
	Pour $h(x) = -f(x)$, on obtient similairement les graphes de la figure \ref{fig:parentes-fois}.
}{ex:parentes-fois}

\begin{figure}[h]
	\begin{center}
	\includegraphics[page=11, scale=1.25]{figures/fig-fonctions.pdf}
	\end{center}
	\caption{Fonctions parentes de l'exemple \ref{ex:parentes-fois}.}
	\label{fig:parentes-fois}
\end{figure}


\exemulticols{}{
	Tracer les graphes des fonctions $g(x) = 2f(x)$ et $h(x) = -2f(x)$ dans le repère contenant $\C_f$ ci-contre.
}{
	\begin{center}
	\includegraphics[page=14]{figures/fig-fonctions.pdf}
	\end{center}
}{exe:draw-timesc}{
	\begin{center}
	\includegraphics[page=26]{figures/fig-fonctions.pdf}
	\end{center}
}


%\section{Implémentation}
%
%
%
%
%\exe{}{
%	Lire les deux programmes Python de la figure \ref{python:1-2}.
%	Quelles valeurs impriment-t-ils ?
%	
%	Écrire la fonction $f(x)$ de chaque programme sous forme algébrique.
%}{exe:fonction-python}{
%	TODO
%}
%
%\begin{figure}[!htb]
%	\begin{subfigure}[b]{.45\textwidth}
%\begin{mintedbox}{python}
%def f(x):
%y = 2*x-3
%return y
%
%f1 = f(1)
%f2 = f(-1/2)
%
%
%print(f1, f2)
%\end{mintedbox}
%	\caption{Programme 1.}
%	\label{python:1}
%	\end{subfigure}
%	\begin{subfigure}[b]{.45\textwidth}
%\begin{mintedbox}{python}
%def f(x):
%y = x*x + 1
%z = -2*x
%return y+z
%
%f1 = f(4)
%f2 = f(-2)
%
%print(f1, f2)
%\end{mintedbox}
%	\caption{Programme 2.}
%	\label{python:2}
%	\end{subfigure}
%	\caption{Deux fonctions implémentées en Python.}
%	\label{python:1-2}
%\end{figure}
%!TEX encoding = UTF8
%!TEX root = 0-notes.tex

\chapter{Plan cartésien}
\label{sec:geom-plane}\label{chap:plan-cartésien}

%Sur la droite réelle, on associe chaque point avec un nombre, appelé nombre réel : chaque point correspond à un nombre, et chaque nombre correspond à un point.

Dans cette section, on augmente la notion de droite réelle en ajoutant une dimension ; la droite devient le plan.
On associera alors à chaque point du plan \emph{deux}\footnote{Il s'avère qu'il est possible en théorie d'associer chaque point du plan à un unique nombre, mais la construction dépasse le champ d'application de ce cours.} nombres réel : le premier représentant sa position \og gauche/droite \fg et le deuxième sa position \og haut/bas \fg.

Un point sera alors un couple $(x, y)$ où $x, y \in \R$ sont deux nombres réels appelés respectivement l'abscisse et l'ordonnée du point.

\section{Repère orthonormé}

\dfn{Repère}{
	
	\begin{multicols}{2}

	Un repère est determiné par trois points $(O; I, J)$ comme ci-contre.
	\begin{itemize}
		\item Le point $O$ est appelé l'\emphindex{origine du repère}.
		\item La droite $(OI)$ est appelée l'\emphindex{axe des abscisses}.
		\item La droite $(OJ)$ est appelée l'\emphindex{axe des ordonées}.
		\item La graduation est \underline{toujours} régulière.
		\item[]
	\end{itemize}
	
	
	\includegraphics[page=1, scale=1.3]{figures/fig-plan.pdf}
	\end{multicols}

}{}

\dfn{Repère orthonormé}{
	Un repère $(O; I, J)$ est
	\begin{enumerate}
		\item \emphindex{orthogonal} si ses axes $(OI)$ et $(OJ)$ sont perpendiculaires ;
		\item \emphindex{normé} si les segments $[OI]$ et $[OJ]$ sont de même longueur (fixée à $1$) ;
		\item \emphindex{orthonormé} s'il est orthogonal et normé.
	\end{enumerate}
}{def:repere-orthonorme}

\nt{
	L'ordre des points $I$ et $J$ est important car il donne l'ordre de lecture des coordonnées.
	Traditionnellement, l'axe des abscisse est horizontale, et celle des ordonnée verticale.
	
	Ainsi, le point $I$ admet pour coordonnée $(1;0)$, et le point $J$ $(0;1)$.
	On notera alors :
		\begin{align*}
			I(1;0), && \et && J(0;1).
		\end{align*}
}

\ex{}{
\, \\
	\begin{center}
	\includegraphics[page=2,scale=1.5]{figures/fig-plan.pdf}
	\end{center}
}{}

\dfn{plan cartésien}{
	Le plan cartésien est l'ensemble des couples $(x, y)$ de réels :
		\[ \bigset{ (x, y) \text{ tq. } x, y\in\R }. \]
}{dfn:plan-cartésien}

\nomen{
	On appelle un élément $(x;y)$ du plan un \emphindex{point} et $x, y$ ses \emph{coordonnées}.
}

\nt{
	On appelle la plus petite graduation le \emphindex{pas de graduation}.
	Lorsqu'on trace un repère, le pas en abscisse n'est pas forcément le même que le pas en ordonnée : pour représenter les points $(-1 ; 35), (2 ; 70),$ et $(-2 ; -40)$, mieux vaut choisir un pas de $20$ en ordonnées.
	
	En outre, l'intersection des axes n'est pas forcément l'origine du repère : un repère peut commencer en $x=4$ et finir en $y=-10$ pour faciliter la représentation des points.
	
	\textbf{\warning Il est cependant important que le pas soit régulier.}
}

\exe{1}{
	Donner des points du plan (par leur coordonnées) tels que, lorsque reliés adéquatement, on puisse lire la première lettre de votre prénom.
}{exe:prénom}{
	Le prénom de l'auteur commençant par $M$, celui-ci propose les points $A(0;0), B(0;3)$, $C(1;2), D(2;3), E(2;0)$.
	\begin{center}
	\includegraphics[page=6]{figures/fig-plan.pdf}
	\end{center}
}

\exemulticols{}{
	Donner approximativement les coordonnées de chaque point du repère ci-contre.
	\begin{align*}
		&A(\phantom{2} ; \phantom{3}) \\
		&B \\
		&C \\
		&D \\
		&E
	\end{align*}
}{
	\begin{center}
	\includegraphics[page=7]{figures/fig-plan.pdf}
	\end{center}
}{exe:lecture-coord}{
	\begin{align*}
		A(2 ; 3) && B(-1 ; 2) && C(2,5 ; -1) &&
		D(0;-2) && E(-2,5 ; 0)
	\end{align*}
}

\section{Opérations dans le plan}\label{sec:longueur-milieu}

\subsection{Distance et norme}

\notations{
	Pour deux points $A$ et $B$ du plan, on note $AB$ la \emphindex{distance} entre $A$ et $B$.
}

\nt{
	La \emphindex{distance mesurée} n'est pas la même si le repère choisi n'est pas orthonormé.
	Pour se convaincre, placer le point $(10 ; 10)$ dans un repère orthonormé et un repère de pas 10, et comparer sa distance à l'origine.
	En pratique, on choisit rarement des pas identiques valant 1 unité, et on ne peut alors rien mesurer à l'aide d'une règle !
}

\notations{
	Pour n'importe quel nombre $x$ on dénote $x^2 = x \cdot x$ le produit de $x$ par lui-même.
	On lit « $x$ au carré » ou « $x$ carré ».
}

\thm{distance}{
	Soient $A(x_A, y_A)$, $B(x_B, y_B)$ deux points du plan dans un repère orthonormé.
	La distance $AB$ entre $A$ et $B$ vérifie
		\[ AB^2 = (x_A - x_B)^2 + (y_A - y_B)^2. \]
}{thm:long-segment}

\pf%{Démonstration du théorème \ref{thm:long-segment}}{
{}{
	Comme le repère est normé, on peut lire les distances sur les coordonnées.
	En particulier, la distance en la première coordonnée entre $A$ et $B$ est soit $(x_A - x_B)$, soit son opposé $(x_B - x_A)$..
	De la même façon, la distance en la deuxième coordonnée est $(y_A - y_B)$ ou $(y_B - y_A)$.
	On a donc le dessin suivant.
	
	\begin{center}
	\includegraphics[page=4, scale=1.1]{figures/fig-plan.pdf}
	\end{center}
	
	Comme le repère est orthogonal, les axes sont perpendiculaires et le triangle est donc rectangle.
	Le théorème de Pythagore s'applique donc. 
	Remarquons que le carré ignore le signe $\bigl((-E)^2 = E^2 \bigr)$, et donc que $(x_A-x_B)^2 = (x_B - x_A)^2$.
	Il suit que
		\[ AB^2 = (x_A - x_B)^2 + (y_A - y_B)^2, \]
	ce qui conclut.
}

\dfn{norme}{
	Soit $u = (x ; y)$ un point quelconque.
	On définit le carré de sa \emph{norme}, $\norm{u}^2$, par
		\[ \norm{u}^2 = x^2 + y^2. \]
}{dfn:norme-carré}

\exe{}{
	Donner le carré de la norme des points $(0 ; 0)$, $(3 ; 4)$, $(-3 ; 4)$, $(-1; -1)$.
}{exe:norme}{
	\begin{align*}
		&\norm{(0 ; 0)}^2 = 0^2 + 0^2 = 0, && \norm{(3 ; 4)}^2 = 3^2 + 4^2 = 9 + 16 = 25, \\
		&\norm{(-3 ; -4)}^2 = (-3)^2 + (-4)^2 = 9 + 16 = 25, && \norm{(1 ; -1)}^2 = 1^2 + (-1)^2 = 2.
	\end{align*}
}

\dfn{manipulations des points}{
	Soient $A(x_A, y_A)$, $B(x_B, y_B)$ deux points du plan et $\kappa \in \R$ un nombre réel quelconque.
	Les opérations suivantes sont légales.
		\begin{enumerate}
			\item L'addition de deux points : $A+B$, de coordonées $(x_A+x_B, y_A+y_B)$.
			\item La multiplication d'un point par un réel : $\kappa A$, de coordonnées $(\kappa x_A; \kappa y_A)$.
		\end{enumerate}
	\textbf{\warning On ne multiplie jamais les points ensemble ! Le produit $A$ par $B$ n'a pas de sens.}
}{def:manip-points}

\nt{
	La manipulation des points est considérée illégale par plusieurs mathématiciens du monde.
	Il s'agira alors de leur demander : quelle est la différence entre un point et un vecteur ?
	
	En cas de doute, remplacez tous les points $A(x ;y)$ par un vecteur $a = \pvec{x}{y}$ dans la suite.
	Les vecteurs seront l'objet du chapitre \ref{chap:vecteurs}.
}

\ex{}{
	Considérons $A(1;3)$ et $B(-3;2)$ deux points du plan.
	Alors 
		\begin{multicols}{3}
		\begin{enumerate}
			\item $A+B = (-2 ; 5)$
			\item $·2B = (-6 ; 4)$
			\item $-A = (-1 ; -3)$
			\item $B - A = (-4 ; -1)$
			\item $-2A - B = (1 ; -8)$
			\item $\dfrac{A-2B}{2} = \left(\dfrac72 ; \dfrac{-1}2\right)$
		\end{enumerate}
		\end{multicols}
}{}

\exe{}{
	Tracer un repère et y placer les points suivants. Il est recommandé de calculer les coordonnées des points à placer avant de tracer le repère afin de décider d'un pas adéquat.
	\begin{multicols}{3}
	\begin{enumerate}[leftmargin=50pt]
		\item $\point{A}{2}{3}$
		\item $\point{B}{-1}{3}$
		\item $C = A+B$
		\item $D=A-B$
		\item $E=\frac12A$
		\item $F=-3B$
	\end{enumerate}
	\end{multicols}
}{exe:points-à-placer}{
	\begin{center}
	\includegraphics[page=5]{figures/fig-plan.pdf}
	\end{center}
}

\mprop{reformulation du théorème \ref{thm:long-segment}}{
	La distance $AB$ entre $A$ et $B$ vérifie
		\[ AB^2 = \norm{A - B}^2 = \norm{B - A}^2. \]
}{prop:long-segment}

\exe{}{
	On considère le triangle de sommets $A(1;2), B(-3 ; 5), C(-5 ; -6)$.
	\begin{enumerate}
		\item Tracer le triangle $ABC$ dans un repère.
		\item Calculer le carré de chacun des côtés. 
		\item Que dire du triangle ?
	\end{enumerate}
}{exe:Trex}{
	\begin{multicols}{2}
	\begin{center}
	\includegraphics[page=8]{figures/fig-plan.pdf}
	\end{center}
	
	\begin{enumerate}
		\item[2.] 
			\begin{align*}
				AB^2 &= \norm{A-B}^2 = \norm{(4 ; -3)}^2 = 16 + 9 = 25, \\
				AC^2 &= \norm{A-C}^2 = \norm{(6 ; 8)}^2 = 36 + 64 = 100, \\
				BC^2 &= \norm{B-C}^2 = \norm{(2 ; 11)}^2 = 4 + 121 = 125.
			\end{align*}
		\item[3.] D'après la réciproque du théorème de Pythagore, le triangle est rectangle en $A$ car $BC^2 = AB^2 + AC^2$.
	\end{enumerate}
	\end{multicols}
}

\exe{}{
	Montrer que la distance d'un point à l'origine est sa norme.
}{exe:distO-norm}{
	L'identité $\norm{P - O} = \norm{P}$, où $O(0; 0)$ est l'origine, suffit à conclure.
}

\exe{, difficulty=1}{
	Montrer que $AB = 0$ si et seulement si $A=B$.
}{exe:norm0}{
	Comme, $AB^2 = 0$, on a $dx^2 + dy^2 = 0$, où $dx$ et $dy$ désignent les différences en $x$ et en $y$ des points $A$ et $B$.
	Comme le carré d'un nombre réel est toujours positif, on déduit que $dx = dy =0$.
}

\exe{}{
	Tracer dans un repère l'ensemble des points à distance 3 de l'origine.
}{exe:C03}{
	Cet ensemble forme un cercle de rayon 3 centré en l'origine.
	\begin{center}
	\includegraphics[page=9]{figures/fig-plan.pdf}
	\end{center}
}

\nomen{
	On dit d'un point $(x, y)$ du plan qu'il est à \emphindex{coordonnées entières} dès que $x, y \in \Z$ sont deux entiers relatifs.
}

\exe{, difficulty=2}{
	Combien de points à coordonnées entières du plan sont à distance $5$ de l'origine du repère ?
}{exe:norm5}{
	L'exercice \ref{exe:norme} nous en donne déjà 2 : $(3 ; 4)$ et $(-3 ; -4)$, dont la norme vaut $25 = 5^2$.
	On en déduit facilement que $(-3 ; 4)$ et $(3 ; -4)$ fonctionnent aussi, car le carré ignore le signe.
	
	En échangeant $x$ et $y$, on trouve aussi les points $(\pm4 ; \pm3)$, où $\pm$ signifie « plus ou moins ».
	
	Cherchons-en d'autres : $(x ; y)$ doit vérifier $x^2 + y^2 = 25$ avec $x, y$ entiers.
	En testant des valeurs de $x=0 ; 1 ; 2 ; 5$, on trouve les points $(0 ; 5), (0 ; -5)$, et $(5 ; 0), (-5 ; 0)$.
	
	Il existe donc finalement 12 tels points.
}


\exe{, difficulty=1}{
	Soient $\point{G}{-4}{-1}$ et $\point{D}{-1}{3}$ et $x\in\R$ un paramètre réel.
	Pour quel(s) $x\in\R$ est-ce que la longueur du segment entre les points $xG$ et $xD$ est-elle égale à $5$ ?
}{exe:milieu-x}{
	La longueur du segment est donnée par
		\[ 5 = \sqrt{(-4x + x)^2 + (-x - 3x)^2} = \sqrt{9x^2 + 16x^2} = \sqrt{25x^2}. \]
	En mettant l'équation au carré, on trouve
		\begin{align*}
			25 = 25x^2 && \iff &&  x^2 = 1.
		\end{align*}
	D'où on trouve les solutions $x=1$ ou $x=-1$.
}

\qs{}{
	Peut-on connaître exactement une longueur $\ell$ en connaissant son carré $\ell^2$ ?
	Plus généralement, comment connaître exactement un nombre en connaissant son carré ?
	L'étude de la racine carrée fait partie du chapitre \ref{chap:fonction-carré}.
	En attendant, on supposera l'existence d'un nombre positif noté $\sqrt3$ dont le carré vaut 3 pour résoudre les exercices suivants.
}

\exe{}{
	Considérons les points $\point{A}{1}{1}, \point{B}{3}{1}, \point{C}{2}{\sqrt{3}+1}$.
	Démontrer que le triangle $ABC$ est équilatéral en calculant le carré de la longueur de chaque côté.
	
	\emph{Un triangle équilatéral est un triangle dont les trois côtés ont la même longueur}.
}{exe:équilatéral}{
	On calcule 
		\begin{align*}
			AB^2 = {2^2 + 0^2} = 4, && AC^2 = {1^2 + \sqrt{3}^2} = 4, && BC^2 = 4.
		\end{align*}
}

\exe{}{
	Montrer que $(-2)^2 = 4$ et en déduire qu'il n'existe pas qu'une seule solution réelle à l'équation $x^2 = 4$.
}{exe:carre4}{
	Par définition, $(-2)^2 = (-2) \cdot (-2) = 4$.
	Comme $2^2 = 4$, il existe au moins deux solutions réelles à l'équation $x^2 = 4$ : 2 et $-2$.
}

\exe{, difficulty=2}{
	Montrer que $x^2 - 4 = (x-2)(x+2)$ pour tout $x\in\R$ et en déduire les deux seules solutions réelles de l'équation $x^2 = 4$.
}{exe:carre4-all}{
	Par distributivité, $(x-2)(x+2) = x^2 - 4$.
	Il suit que $x^2 = 4 \iff x^2 - 4 = 0 \iff (x-2)(x+2) = 0$.
	Or le produit de deux nombre n'est nul que si l'un des deux est nul.
	On a donc soit $x-2=0 \iff x=2$, soit $x+2=0 \iff x=-2$.
}

\exe{, difficulty=2}{
	Soient $A(1;1), B(3;1)$ deux points du plan, et $C(2;x)$ un point dépendant d'un paramètre réel $x\in\R$.
	Pour quel(s) $x\in\R$ le triangle $ABC$ est-il équilatéral ? 
}{exe:équilatéral2}{
	On souhaite que $AB = AC = BC$ soit vérifié, c'est-à-dire que
		\[ 2 = \sqrt{1 + (1-x)^2} = \sqrt{1 + (1-x)^2}. \]
	La deuxième égalité est redondante et, en mettant au carré, on trouve
		\begin{align*}
		4 = 1 + (1-x)^2 && \iff && (1-x)^2 = 3.
		\end{align*}
	En reprenant l'exercice \ref{exe:carre4-all}, on trouve deux solutions :
	\vspace{-20pt}
		\begin{multicols}{2}
		\begin{align*}
			1-x &= \sqrt{3}, \\
			x &= 1-\sqrt{3}.
		\end{align*}
			
		\begin{align*}
			1-x &= -\sqrt{3}, \\
			x &= 1 +\sqrt{3}.
		\end{align*}
		\end{multicols}
}{}

\subsection{Segment et milieu}

\exe{}{
	Représenter les points $A(1;1)$ et $B(3;-1)$ dans un repère orthonormé.
	Représenter le point
		\[ \lambda A + (1-\lambda)B, \]
	pour certaines valeurs de $\lambda$ (lu « lambda ») entre 0 et 1.
	
	Quel $\lambda$ choisir pour obtenir 
		\begin{multicols}{2}
		\begin{itemize}
			\item le point $A$ ?
			\item le point $B$ ?
			%\item le milieu du segment $[AB]$ ?
			\item le point $C\left(\dfrac32; \dfrac12\right)$ ?
		\end{itemize}
		\end{multicols}
}{exe:milieu-segment}{
	On part de $\lambda=0$ et on augment petit à petit vers 1.
	\begin{itemize}
		\item
		En $\lambda = 0$, on trouve $B$.
		\item
		En $\lambda = 0,25$, on trouve $0,25A + 0,75B = (0,25 ; 0,25) + (2,25 ; -0,75) = (2,5 ; -0,5)$
		\item
		En $\lambda = 0,5$, on trouve $0,5 A + 0,5 B = (0,5 + 1,5 ; 0,5 - 0,5) = (2 ;0)$
		\item
		En $\lambda = 0,75$, on trouve $0,75A + 0,25B = (0,75 ; 0,75) + (0,75 ; -0,25) = (1,5 ; 0,5) = C$
		\item
		En $\lambda=1$, on trouve $A$.
	\end{itemize}
	On trace en fait le segment $[AB]$ en choisissant $\lambda$ entre 0 et 1.
	
	\centering
	\includegraphics[page=10]{figures/fig-plan.pdf}
}

\dfn{segment}{
	Soient $E, F$ deux points du plan. Le segment $[EF]$ est l'ensemble des points suivant.
		\[ [EF] = \Bigset{ \lambda E + (1-\lambda)F \tq 0 \leq \lambda \leq 1 } \]
}{deg:segment}

\nt{
	La définition prend son sens en voyant l'expression $\lambda E + (1-\lambda)F$ comme une moyenne pondérée de $E$ et de $F$.
	La somme des poids vaut toujours 1 dans une moyenne et, lorsque les poids sont égaux, on obtient la moyenne classique du théorème suivant.
}

%\nt{
%	Un segment du plan est une généralisation d'un intervalle sur la droite.
%	Tous les segments du plans qui seront considérés contiennent leur bornes (contrairement aux intervalles !).
%}

\thm{}{
	Soient $A$ et $B$ deux points du plan.
	Alors le milieu $M$ du segment $[AB]$ est 
		\[ M = \dfrac{A+B}{2}. \]
		
	\begin{center}
	\includegraphics[page=3, scale=1.1]{figures/fig-plan.pdf}
	\end{center}
}{thm:milieu-segment}

\pf{}{
	Le point $P = \frac12 A + \frac12 B$ appartient au segment $[AB]$ (en prenant $\lambda=\frac12$), et vérifie
		\begin{align*}
			\norm{P-A}^2 &= \norm{ \dfrac12 A + \dfrac12 B - A}^2, \\
						&=  \norm{\dfrac12 B  -\dfrac12 A}^2, \\
						&=  \norm{\dfrac12 A  -\dfrac12 B}^2
						= \norm{P - B}^2,
		\end{align*}
	où on a utilisé que $\norm{E-F} = \norm{F-E}$ pour la troisième égalité.
	Le point $P$ est donc à équidistance de $A$ et $B$ et c'est le milieu du segment : $M = P = \frac{A+B}2$.	
}{}

\exe{}{
	Considérons les points
		\begin{align*}
			\point{A}{0}{1}, && \point{B}{-3}{0}, && \point{C}{1}{-2}, && \point{D}{-2}{-3}.
		\end{align*}
	Démontrer que le quadrilatère $BACD$ est un parallélogramme en comparant le milieu de ses deux diagonales.
	
	\emph{Un parallélogramme est un quadrilatère dont les diagonales se coupent en leur milieu}.
}{exe:parallélogramme}{
	Le milieu du segment $[BC]$ est donné par
		\[ M = \dfrac12 (B+C) = (-1;-1),\]
	et le milieu du segment $[AD]$ est donné par
		\[ M' = \dfrac12 (A+D) = (-1;-1) = M.\]
	Le quadrilatère est donc bien un parallélogramme.

	\centering
	\includegraphics[page=6]{figures/fig-exe.pdf}
}

\exe{, difficulty=1}{
	Soient $\point{A}12$ et $\point{M}3{-1}$ deux points du plan.
	Quel point $C$ faut-il choisir pour que $M$ soit le milieu du segment $[AC]$ ? Donner ses coordonnées.
}{exe:milieu2}{
	La contrainte que $M$ soit le milieu de $[AC]$ s'écrit
		\begin{align*}
			M = \dfrac12(A+C) && \iff && 2M = A+C && \iff && C = 2M-A
		\end{align*}
	Par suite,
		\[ C= 2M-A = 2\cdot\left(-\dfrac53;-1\right) - \left(-\dfrac32;\dfrac53\right) = \left(-\dfrac{11}6; -\dfrac{11}3\right). \]
	
}

\exe{, difficulty=1}{
	Soient $\point{A}{-\frac32}{\frac53}$ et $\point{M}{-\frac53}{-1}$ deux points du plan.
	Quel point $C$ faut-il choisir pour que $M$ soit le milieu du segment $[AC]$ ? Donner ses coordonnées.
}{exe:milieu2}{
	La contrainte que $M$ soit le milieu de $[AC]$ s'écrit
		\begin{align*}
			M = \dfrac12(A+C) && \iff && 2M = A+C && \iff && C = 2M-A
		\end{align*}
	Par suite,
		\[ C= 2M-A = 2\cdot\left(-\dfrac53;-1\right) - \left(-\dfrac32;\dfrac53\right) = \left(-\dfrac{11}6; -\dfrac{11}3\right). \]
	
}

\exe{, difficulty=2}{
	L'origine $O(0 ; 0)$ appartient-elle au segment $[AB]$ où $A(3;1)$ et $B(-4;-1)$ ?
}{exe:C-on-AB}{
	On souhaite savoir si $O = \lambda A + (1-\lambda)B$ pour un certain $\lambda$ entre 0 et 1.
	C'est-à-dire, si 
		\[ (0 ; 0) = \lambda(3 ;1) + (1-\lambda)(-4;-1) = \bigl(3\lambda - 4(1-\lambda) ; \lambda - (1-\lambda) \bigr) = \bigl(7\lambda - 4 ; 2 \lambda - 1 \bigr). \]
	
	D'une part, en $x$, il faut que $7\lambda- 4 = 0 \iff \lambda = \frac47$.
	D'autre part, en $y$, il faut que $2\lambda - 1 = 0 \iff \lambda = \frac12$.
	
	Comme ces deux valeurs ne sont pas les mêmes, un seul $\lambda$ ne peut pas fonctionner, et $O$ n'appartient pas au segment $[AB]$.
}

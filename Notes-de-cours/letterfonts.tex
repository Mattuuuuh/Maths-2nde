%!TEX encoding = UTF8
%!TEX root = 0-notes.tex

%fonts
\usepackage{libertinus,libertinust1math}
\usepackage[T1]{fontenc}

% for calligraphic C, D, P (important to import this after the font)
\usepackage{calrsfs}
\newcommand{\D}{\mathcal{D}}
\newcommand{\C}{\mathcal{C}}
\renewcommand{\P}{\mathcal{P}}

% Schwartz
\renewcommand{\S}{\mathcal{S}} % \S est le signe paragraphe normalement

% corps
\newcommand{\R}{\mathbb{R}}
\newcommand{\Rnn}{\mathbb{R}^{2n}}
\newcommand{\Z}{\mathbb{Z}}
\newcommand{\N}{\mathbb{N}}
\newcommand{\Q}{\mathbb{Q}}
\newcommand{\E}{\mathbb{E}}
\newcommand{\DD}{\mathbb{D}}

% order notations
\DeclareRobustCommand{\O}{%
  \text{\usefont{OMS}{cmsy}{m}{n}O}%
}

% japanese bracket
\newcommand{\japb}[1]{\langle #1 \rangle}

% arrows over partial derivatives
\newcommand{\lp}{\overleftarrow{\partial}}
\newcommand{\rp}{\overrightarrow{\partial}}

% quantization
\newcommand{\h}{\hbar}
\newcommand{\Opht}{\textrm{Op}_{\h}^{t}}
\newcommand{\Op}[2][\hbar]{\textrm{Op}_{#1}^{#2}}

% omega functions
\newcommand{\omegap}[2][\rho_0]{\omega(\partial_{#1},\partial_{#2})}
\newcommand{\omegar}[2][\rho_0]{\omega(#1,#2)}

% space before semicolon
\mathcode`\;="303B

% for \Lightning
\usepackage{marvosym}

% for \warning
% 66 or 49 idk ; depends on the computer for some reason
\newcommand{\warning}{{\fontencoding{U}\fontfamily{futs}\selectfont\char 66\relax}}

% Q(\sqrt(d)) field
\newcommand{\Qsqrt}[1]{\Q\bigl(\mspace{-3mu}\sqrt{#1}\bigr)}

% closed off square root from https://en.wikibooks.org/wiki/LaTeX/Mathematics#Roots
% simple version: does not allow cbrt{x} = sqrt[3]{x}
% New definition of square root:
% it renames \sqrt as \oldsqrt
\let\oldsqrt\sqrt
% it defines the new \sqrt in terms of the old one
\def\sqrt{\mathpalette\DHLhksqrt}
\def\DHLhksqrt#1#2{%
\setbox0=\hbox{$#1\oldsqrt{#2\,}$}\dimen0=\ht0
\advance\dimen0-0.2\ht0
\setbox2=\hbox{\vrule height\ht0 depth -\dimen0}%
{\box0\lower0.4pt\box2}}


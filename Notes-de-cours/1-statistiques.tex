%!TEX encoding = UTF8
%!TEX root = 0-notes.tex

\chapter{Statistiques descriptives}
\label{chap:statistiques}


On considère dans ce chapitre une \emphindex{série statistique}, c'est-à-dire une liste de valeurs d'une même nature qu'on souhaite étudier.
Les exemples de séries statistiques sont nombreux : une liste de notes à une évaluation, une liste de notes d'un élève unique à plusieurs évaluations, une liste de salaires bruts, une liste de températures dans un même lieu, etc…

On parle ici de \emphindex{liste} et non d'\emph{ensemble} car une valeur peut se répéter plusieurs fois, ce qui change alors la nature de la série statistique.
La moyenne de la liste $(8; 10; 12)$ est différente de la moyenne de $(8; 10; 12; 12)$.

Dans toute la section suivante, $N, p \in \N$ sont deux entiers naturels non nuls qui désignent un cardinal (de liste ou d'ensemble). 
Les valeurs $x_1, x_2, \dots$ sont toutes des nombres réels.

\dfn{étendue}{
	Considérons $X$ une série statistique.
	L'\emphindex{étendue} de la série $X$ est égale à la distance entre la plus petite et la plus grande valeur.
		\[ \bigl|\min(X) - \max(X) \bigr|. \]
}{}

\section{Moyennes pondérées}

\dfn{moyenne}{
	Considérons $X = (x_1, x_2, \dots, x_N)$ une série statistique.
	On note $\overline{X}$ sa moyenne, donnée par
		\[ \overline{X} = \dfrac{x_1 + \dots + x_N}N. \]
}{}


\exe{1}{
	Calculer la moyenne de chaque série statistique suivante.
		\begin{multicols}{2}
		\begin{enumerate}
			\item $(0; 20)$
			\item $(0; 0; 20)$
			\item $(0; 20; 20)$
			\item $(-10; 23 ; -10 ; -23 ; 10 ; 10)$
			\item $(-50 ; -100 ; -23 ; -31; -50 ; -50)$
			\item $(0; 1;2;3; \dots; 18; 19; 20)$
		\end{enumerate}
		\end{multicols}
}{exe:stats1}{
	\begin{enumerate}
		\item $\overline{X} = \frac{0+20}{2} = 10$
		\item $\overline{X} = \frac{0+0+20}{3} = \frac{20}3 \approx 6,67$
		\item $\overline{X} = \frac{0+20+20}{3} = \frac{40}3 \approx 13,33$
		\item $\overline{X} = \frac{-10+23-10-23+10+10}{6} = 0$. La série est symétrique par rapport à $0$ (chaque valeur a son opposé du même effectif).
		\item $\overline{X} = \frac{-50-100-23-31-50-50}{6} = -\frac{304}6 = -\frac{152}3 \approx -50,67$
		\item $\overline{X} = \frac{0+ 1+2+3+ \dots+ 18+ 19+ 20}{21} = 10 $. La série est symétrique par rapport à $10$.
	\end{enumerate}
	Le calcul de la dernière moyenne est généralisé à l'exercice \ref{exe:sum-all}.
}


\dfn{moyenne pondérée}{
	Soit $X = (x_1, x_2, \dots, x_N)$ une série statistique à coefficients $(c_1, \dots, c_N)$.
	
	On note $\overline{X}$ sa moyenne, donnée par
		\[ \overline{X} = \dfrac{c_1 \cdot x_1 + \dots + c_N \cdot x_N}{c_1 + \cdots + c_N}. \]
}{def:moyenne-pond}


\exe{}{
	Un élève reçoit $5$ notes listées ainsi : $(4,25 ; 10; 18,5 ; 15,5 ; 11,25)$.
	À ces notes sont associés $5$ coefficients, donnés dans le même ordre par $(0,5 ; 1,4 ; 0,1 ; 0,5; 1,1)$.
	
	Calculer la moyenne pondérée des notes.
}{exe:stats2}{
	Il y a deux manières de faire.
	D'une part, on peux calculer la somme des produits Note $\times$ Coefficient puis diviser par la somme des coefficients.
		\[ \dfrac{4,25\cdot0,5 + 10\cdot1,4 + 18,5\cdot0,1 + 15,5\cdot 0,5 + 11,25\cdot1,1}{0,5+1,4+0,1+0,5+1,1} = \dfrac{38,1}{3,6}  \approx 10,58. \]
	D'autre part, on peut calculer le poids de chaque note en normalisant les coefficients, c'est-à-dire en divisant chaque coefficient par leur somme.
	
	$p_1 = \dfrac{0,5}{3,6}$, $p_2 = \dfrac{1,4}{3,6}$, $p_3 = \dfrac{0,1}{3,6}$, $p_4 = \dfrac{0,5}{3,6}$, et $p_5 = \dfrac{1,1}{3,6}$, puis
		\[ 4,25 \cdot p_1 + 10\cdot p_2 + 18,5 \cdot p_3 + 15,5 \cdot p_4 + 11,25\cdot p_5 = 10,58. \]

	La première méthode est plus facile à employer mais elle découle en fait de la seconde : une moyenne est toujours une combinaison convexe de valeurs car les poids somment toujours à $1$.
}


\nt{
	Remarquons que la moyenne de la définition \ref{def:moyenne-pond} s'écrit aussi 
		\begin{align}
			\overline{X} &=  \dfrac{c_1 \cdot x_1 + \dots + c_N \cdot x_N}{c_1 + \cdots c_N} \nonumber\\
						&= \dfrac{c_1}{c_1 + \cdots c_N} \cdot x_1 + \dfrac{c_2}{c_1 + \cdots c_N} \cdot x_2 + \dots + \dfrac{c_N}{c_1 + \cdots c_N} \cdot x_n \label{expr:poids}
		\end{align}
	En divisant par la somme des coefficients, on a \emphindex{normalisé} les coefficients en poids, qu'on multiplie à chaque valeur.
}

\lem{}{
	Les poids $\dfrac{c_1}{c_1 + \cdots c_N}, \dfrac{c_2}{c_1 + \cdots c_N}, \dots, \dfrac{c_N}{c_1 + \cdots c_N}$ de l'expression \eqref{expr:poids} ci-dessus vérifient
		\begin{enumerate}
			\item Chaque poids est un nombre appartenant à $[0;1]$ ; et
			\item La somme des poids vaut $1$.
		\end{enumerate}
}{lem:poids-normalisés}

\nt{
	Certaines séries statistiques sont données sous la forme d'un ensemble de couples $(x, n)$ où $n$ est le nombre d'appartitions de $x$ dans la liste.
	Chaque valeur $x$ est alors distincte et on peut bien parler d'ensemble.
}{}

\dfn{moyenne avec effectifs}{
	Considérons $X = \left\{ (x_1, n_1), (x_2, n_2), \dots, (x_p, n_p) \right\}$ une série statistique avec effectifs.
	
	On note $\overline{X}$ sa moyenne, donnée par
		\begin{align}\label{eq:moyenne-effectifs}
			\overline{X} = \dfrac{n_1 \cdot x_1 + \dots + n_p \cdot x_p}{n_1 + \dots + n_p}.
		\end{align}
}{}

\exe{}{
	Pour chaque série statistique suivante, calculer la moyenne.
	\begin{multicols}{2}
	\begin{enumerate}
		\item 
			\begin{tabular}{|c|c|c|c|c|}\hline
			Valeur   & 9 & 13 & 11 & 7 \\ \hline
			Effectif & 5 & 12 & 5 & 12 \\ \hline
			\end{tabular}
			
		\item 
			\begin{tabular}{|c|c|c|c|c|}\hline
			Valeur   & 2 & 13 & 18 & 7 \\ \hline
			Effectif & 5 & 12 & 5 & 12 \\ \hline
			\end{tabular}
			
		\item 
			\begin{tabular}{|c|c|c|c|c|}\hline
			Valeur   & 0 & 20 & 2 & 5 \\ \hline
			Effectif & 5 & 14 & 10 & 2  \\ \hline
			\end{tabular}

		\item 
			\begin{tabular}{|c|c|c|c|}\hline
			Valeur   & 0 & 20 & 10 \\ \hline
			Effectif & 17 & 17 & 1 \\ \hline
			\end{tabular}


	\end{enumerate}
	\end{multicols}
}{exe:moyenne-effectif}{
	\begin{enumerate}
		\item 
			$\overline{X} = \frac{9\cdot5 + 13\cdot12 + 11\cdot5 + 7\cdot12}{5+12+5+12} = 10$
			
		\item 
			$\overline{X} = \frac{2\cdot5 + 13\cdot12 + 18\cdot5 + 7\cdot12}{5+12+5+12} = 10$
			
		\item 
			$\overline{X} = \frac{0\cdot5 + 20\cdot14 + 2\cdot10 + 5\cdot2}{5+14+10+2} = 10$

		\item 
			$\overline{X} = \frac{0\cdot17 + 20\cdot17 + 10\cdot1}{17+17+1} = 10$

	\end{enumerate}
}

\nt{
	Notons $n = n_1 + \dots + n_p$. On peut réécrire la moyenne $\overline{X}$ donnée en \eqref{eq:moyenne-effectifs} comme
		\begin{align*}
			\overline{X} &= \dfrac{n_1 \cdot x_1 + \dots + n_p \cdot x_p}{n} \\
						&= \dfrac{n_1}n \cdot x_1 + \dfrac{n_2}n x_2 + \dots + \dfrac{n_p}n x_p \\
						&= f_1 \cdot x_1 + f_2 \cdot x_2 + \dots f_p \cdot x_p,
		\end{align*}
	où $f_1 = \dfrac{n_1}n$, $f_ 2 = \dfrac{n_2}n$, etc... sont les fréquences d'apparition des valeurs dans la série statistique.
}

\dfn{fréquence d'apparition}{
	Soit $X = \left\{ (x_1, n_1), (x_2, n_2), \dots, (x_p, n_p) \right\}$ une série statistique avec effectifs.

	On pose $n = n_1 + \dots + n_p$. 
	Alors
		\begin{align*}
			f_1 = \dfrac{n_1}n, && f_ 2 = \dfrac{n_2}n, && \dots && f_ p = \dfrac{n_p}n,
		\end{align*}
	désignent les \emphindex{fréquences d'apparition} de chaque valeur $x_1, x_2, \dots, x_p$.
}
{def:freq-app}

\exe{}{
	Pour chaque série statistique suivante, remplir la ligne \og Fréquence \fg et calculer la moyenne.
		\begin{multicols}{2}
		\begin{enumerate}
			\item 
				\begin{tabular}{|c|c|c|}\hline
				Valeur   & 0 & 20 \\ \hline
				Effectif & 1 & 2  \\ \hline
				Fréquence & &  \\ \hline
				\end{tabular}
				
			\item 
				\begin{tabular}{|c|c|c|}\hline
				Valeur   & 0 & 20 \\ \hline
				Effectif & 2 & 1  \\ \hline
				Fréquence & &  \\ \hline
				\end{tabular}
				
			\item 
				\begin{tabular}{|c|c|c|c|c|}\hline
				Valeur   & 11 & 13 & 9 & 7 \\ \hline
				Effectif & 2 & 1 & 5 & 10 \\ \hline
				Fréquence & &&&  \\ \hline
				\end{tabular}
				
			\item 
				\begin{tabular}{|c|c|c|c|c|}\hline
				Valeur   & -23 & -1 & 1 & 23 \\ \hline
				Effectif & 176 & 304 &304 & 176 \\ \hline
				Fréquence & &&&  \\ \hline
				\end{tabular}

		\end{enumerate}
		\end{multicols}
}{exe:stats3}{
	\begin{enumerate}
		\item 
			La somme des effectifs $N$ est donnée par $N = 1+2 = 3$.
			Pour trouver la fréquence d'une valeur, on divise son effectif par $N$.
			
			\begin{tabular}{|c|c|c|}\hline
			Valeur   & 0 & 20 \\ \hline
			Effectif & 1 & 2  \\ \hline
			Fréquence & $\frac13$ & $\frac23$  \\ \hline
			\end{tabular}
			
			La moyenne est la somme des produits Valeur $\times$ Fréquence.
			D'où $\overline{X} = 0 \cdot \frac13 + 20 \cdot \frac23 =\dfrac{40}3 \approx 13,33$, qu'on comparera à l'exercice $1$.
			
		\item 
			La somme des effectifs $N$ est donnée par $N = 1+2 = 3$.
			Pour trouver la fréquence d'une valeur, on divise son effectif par $N$.
			
			\begin{tabular}{|c|c|c|}\hline
			Valeur   & 0 & 20 \\ \hline
			Effectif & 2 & 1  \\ \hline
			Fréquence & $\frac23$ & $\frac13$ \\ \hline
			\end{tabular}
			
			D'où $\overline{X} = 0 \cdot \frac23 + 20 \cdot \frac13 =\dfrac{20}3 \approx 6,67$, qu'on comparera à l'exercice $1$.
			
		\item 
			La somme des effectifs $N$ est donnée par $N = 2+1+5+10 = 18$.
			Pour trouver la fréquence d'une valeur, on divise son effectif par $N$.
			
			\begin{tabular}{|c|c|c|c|c|}\hline
			Valeur   & 11 & 13 & 9 & 7 \\ \hline
			Effectif & 2 & 1 & 5 & 10 \\ \hline
			Fréquence & $\frac2{18}$ & $\frac1{18}$ & $\frac5{18}$ & $\frac{10}{18}$  \\ \hline
			\end{tabular}
			
			D'où $\overline{X} = 11\cdot\frac2{18} + 13\cdot \frac1{18} + 9\cdot \frac5{18} + 7\cdot\frac{10}{18} = \dfrac{150}{18} \approx 8,33 $.
			
		\item 
			La somme des effectifs $N$ est donnée par $N = 67+12+4+17 = 100$.
			Pour trouver la fréquence d'une valeur, on divise son effectif par $N$.
			
			\begin{tabular}{|c|c|c|c|c|}\hline
			Valeur   & -23 & -1 & 1 & 23 \\ \hline
			Effectif & 67 & 12 & 4 & 17 \\ \hline
			Fréquence & $\frac{67}{100}$ & $\frac{12}{100}$ & $\frac{4}{100}$ & $\frac{17}{100}$ \\ \hline
			\end{tabular}
			
			D'où $\overline{X} = (-23) \cdot \frac{67}{100} + (-1)\cdot \frac{12}{100} + 1\cdot \frac{4}{100} + 23\cdot \frac{17}{100} = -11,58$.

	\end{enumerate}
}

\lem{}{
	Les fréquences $f_1, \dots, f_p$ de la définition \ref{def:freq-app} vérifient
		\begin{align*}
			f_1 \in [0;1], && f_2 \in [0;1], && \dots && f_p \in [0;1],
		\end{align*}
	ainsi que
		\[ f_1 + \dots + f_p = 1. \]
}{lem:fréquences-normalisées}

\nt{
	La remarque suivante est en dehors du champ d'application du cours.
	
	Du lemme \ref{lem:fréquences-normalisées}, on déduit qu'une moyenne est en fait une \emphindex{combinaison convexe} des valeurs d'une série statistique. On parle aussi de \emphindex{barycentre}. 
	Un intervalle est donné par l'ensemble de telles combinaisons des bornes inférieure et supérieure : c'est l'enveloppe convexe de ses bornes.
}

\mprop{linéarité de la moyenne}{
	Soit $X=(x_1 ; \dots ; x_N)$ une série statistique, et $\overline{X}$ sa moyenne.
	On étudie deux opérations faites sur la série statistique entière : multiplier chaque valeur par un nombre $a\in\R$, et additionner un nombre $b\in\R$ à chaque valeur.	
	
	Pour n'importe quel $a \in \R$, la série statistique
		\[ (a\cdot x_1; a\cdot x_2; \dots ; a\cdot x_N) \]
	a pour moyenne $a\cdot \overline{X}$.
	Autrement dit, $\overline{aX} = a \overline{X}$.
	
	Pour n'importe quel $b\in\R$, la série statistique
		\[ (x_1 + b; x_2 + b; \dots ; x_N + b) \]
	a pour moyenne $\overline{X} + b$.
	Autrement dit, $\overline{X+b} = \overline{X} + b$.
	
	La combinaison des deux opérations dites \emphindex{opérations linéaires} donne
		\[ \overline{aX + b} = a\overline{X} + b \]
	pour n'importe quel $a, b \in \R$.
}{}


\exe{inflation et euro fixe}{
	La moyenne des salaires mensuels d'une entreprise est de $1860$€.
	L'inflation désigne une augmentation générale des prix. Ainsi, $6\%$ d'inflation signifie que, en moyenne, les prix augmentent de $6\%$.
	Réciproquement et afin de pouvoir comparer les pouvoirs d'achats, on peut fixer les prix et voir l'inflation comme une diminution du salaire (en l'occurrence de $1-\frac{1}{1,06} \approx 5,6 \%$, d'après le théorème \ref{thm:ev-rec}).
	
	L'entreprise, après une année où l'inflation se mesurait à $6\%$ décide d'ajouter des primes aux salaires : tous les employés recevront $50$€ mensuellement en plus. Ces $50$€ sont, eux aussi, soumis à l'inflation.
	
	Quelle est la nouvelle moyenne des salaires mensuels de l'entreprise ?

}{exe:stats4}{
	Notons $X$ la série statistique des salaires avant inflation.
	On utilise ici la linéarité de la moyenne.
	
	La prime s'ajoute après l'inflation, donc la moyenne après inflation et prime s'élève à $\overline{X+50} = \overline{X} + 50$.
	
	Enfin, afin de corriger l'inflation et d'après le texte, on multiplie tous les nouveaux salaires par $(1,06)^{-1}$.
	La nouvelle moyenne est donc donnée par 
		\[ \overline{\frac{1}{1,06} \cdot \left( X + 50 \right) } = \frac{1}{1,06} \cdot \left( \overline{X} + 50 \right) = \frac{1}{1,06} \cdot (1860 + 50) \approx 1801.89. \]
	À \og euro fixe \fg \, et en moyenne, la prime de permet pas de compenser l'inflation.
}


\section{Écart type}

La moyenne ne suffit pas à elle seule à caractériser une série statistique.
L'exercice suivant considère plusieurs séries différentes ayant une même moyenne.
Le but de la section est d'introduire l'\emphindex{écart type}, noté $\sigma$ et lu \og sigma \fg, qui permet d'indiquer la \emphindex{dispersion} d'une série : son caractère plus ou moins concentré autour de sa moyenne.

\nt{
	Soit $X=(x_1 ; \dots ;x_N)$ une série statistique.
	
	En plus de la moyenne $\overline{X}$, il est aussi intéressant d'étudier la variance $\Var(X)$.
	Elle mesure la distance moyenne des $x_1, \dots, x_N$ à $\overline{X}$.
	Plus cette distance est petite, plus la valeur est proche de la moyenne.
	Globalement, si toutes les distances sont petites, la série est concentrée autour de sa moyenne, au lieu d'être dispersée.
	
	Il est tentant de calculer les distances
		\[ | x - \overline{X}| \]
	pour chaque valeur $x$, mais il est préférable de considérer le carré de la distance, c'est-à-dire
		\[ | x - \overline{X}|^2. \]
	La raison profonde sera étudiée plus tard, dans le chapitre des fonctions.
	Simplement, la valeur absolue n'est \emph{pas} lisse, alors que la fonction carré, elle, l'est.
	Comme $|x|^2 = x^2$, la valeur absolue disparaît lorsque mise au carré, et la fonction devient lisse.
}

\dfn{variance}{
	Soit $X=(x_1 ; \dots ;x_N)$ une série statistique, $\overline{X}$ sa moyenne.
	
	On dénote par $\Var(X)$ la \emphindex{variance} de $X$, donnée par
		\[ \Var(X) = \dfrac{\left(x_1 - \overline{X}\right)^2 + \dots + \left(x_N - \overline{X}\right)^2}N .\]
	La variance est toujours positive.
	
	Si $X = \left\{ (x_1, n_1), (x_2, n_2), \dots, (x_p, n_p) \right\}$ est une série statistique donnée avec effectifs, on a alors 
		\begin{align}
			\Var(X) = \dfrac{n_1 \cdot \left(x_1 - \overline{X}\right)^2 + \dots + n_p \cdot \left(x_p - \overline{X}\right)^2}{n_1 + \dots + n_p}. \label{eq:var}
		\end{align}
	
}{}

\dfn{écart type}{
	Soit $X$ une série statistique.
	L'\emphindex{écart type} $\sigma = \sigma(X)$  (lu \og sigma \fg)  est donné par
		\[ \sigma(X) = \sqrt{\Var(X) }. \]
}{}

\nt{
	Soit $X=(x_1 ; \dots ;x_N)$ une série statistique, $\overline{X}$ sa moyenne.
	
	Si on pose $Y = \left( \left(x_1 - \overline{X}\right)^2  ; \dots ; \left(x_N - \overline{X}\right)^2 \right)$, nouvelle série statistique, on a
		\[ \overline{Y} = \Var(X). \]
		
	La variance est donc bien une moyenne des distances au carré à $\overline{X}$.
}



\exe{}{
	Calculer et comparer les écart types des séries de l'exercice \ref{exe:moyenne-effectif}.
}{exe:stats5}{
	On calcule d'abord la variance puis l'écart type.
	D'après le cours, $\Var{X} = \overline{(X - \overline{X})^2}$ et $\sigma(X) = \sqrt{\Var{X}}$.
	
	\begin{enumerate}
		\item 
			\[ \Var{X} = \dfrac{5\cdot(9-10)^2 + 12 \cdot(13-10)^2 + 5\cdot(11-10)^2 + 12 \cdot(7-10)^2}{5+12+5+12} = \dfrac{226}{34}. \]
		
		D'où $\sigma(X) = \sqrt{\dfrac{226}{34}} \approx 2,578$.
		
		\item  
			\[\Var{X} = \dfrac{5\cdot(2-10)^2 + 12 \cdot(13-10)^2 + 5\cdot(18-10)^2 + 12 \cdot(7-10)^2}{5+12+5+12} = \dfrac{856}{34}. \]
		
		D'où $\sigma(X) \approx 5,02$.
		
		\item  
			\[\Var{X} = \dfrac{5\cdot(0-10)^2 + 14 \cdot(20-10)^2 + 10\cdot(2-10)^2 + 2 \cdot(5-10)^2}{5+14+10+2} = \dfrac{2590}{31} . \]
		
		D'où $\sigma(X) \approx 9,14.$
		
		\item  
			\[\Var{X} = \dfrac{17\cdot(0-10)^2 + 17 \cdot(20-10)^2 + 1\cdot(10-10)^2 }{17+17+1} = \dfrac{3400}{35}. \]
		
		D'où $\sigma(X) \approx 9,86$.
	\end{enumerate}

	Les écarts types augmentent, ce qui indique que les séries statistiques sont de moins en moins concentrées autour de leur moyenne $10$.
}

\exemulticols{, difficulty=1}{
	Calculer la moyenne et l'écart type de la série ci-contre en fonction de $k$.
	Comment évolue $\sigma$ lorsque $k=10, 20, 50, 100$ ? La série devient-elle plus ou moins concentrée autour de sa moyenne ?
}{
	\,\vfill
	\begin{center}
	\begin{tabular}{|c|c|c|c|}\hline
	Valeur   & 0 & 20 & 10 \\ \hline
	Effectif & 17 & 17 & $k$ \\ \hline
	\end{tabular}
	\end{center}
	\vfill \,
	
}{exe:stats7}{
	D'abord, la moyenne est constamment 10 :
		\[ \overline{X} = \dfrac{0\cdot17 + 20\cdot17 + 10\cdot k}{17+17+k} = \dfrac{10\cdot(34 + k)}{34+k} = 10, \]
	ce qui n'est pas surprenant car ajouter une valeur moyenne ne change pas la moyenne générale.

	On a ainsi 
		\[ \Var{X} = \dfrac{17\cdot(0-10)^2+17\cdot(20-10)^2 + k\cdot(10-10)^2}{17+17+k} = \dfrac{3400}{34+k}. \]
		
	Pour $k=1$, on retrouve l'écart type $\sigma(X) = \sqrt{\dfrac{3400}{35}} \approx 9,86$ de l'exercice \ref{exe:stats5}.
	Les autres valeurs sont listées dans le tableau ci-dessous.
	
	\begin{center}
	\begin{tabular}{|c|c|c|c|c|c|}\hline
		$k$ & 1 & 10 & 20 & 50 & 100 \\ \hline
		$\sigma(X) \approx$ & 9,86 & 8,79 & 7,93 & 6,36 & 5,04 \\ \hline
	\end{tabular}
	\end{center}
	
	On remarque que $\sigma$ diminue quand $k$ augmente.
	En effet, plus le dénominateur de la fraction $\Var{X} = \frac{3400}{34+k}$ augmente, plus la variance diminue, et donc plus l'écart type diminue aussi.
	
	La série statistique devient de plus en plus concentrée autour de sa moyenne car il y a de plus en plus de valeurs égales à celle-ci.
}

\section{Quartiles}

Dans ce qui suit, on considère $X = (x_1 ; \dots ; x_N)$ une série statistique qu'on \textbf{ordonne} : c'est-à-dire dont les valeurs sont rangées par ordre croissant.
	\[ \underbrace{x_1 \leq x_2 \leq x_3 \leq \dots}_{\text{proportion } p} \leq \dots \leq x_{N-2} \leq x_{N-1} \leq x_N. \]
La notion de quartile (et plus généralement de décile, centile, ...) permet de répondre aux questions du type : \og Quelle proportion $p$ de valeurs de $X$ sont inférieures à un seuil donné ? \fg, et \og Pour une proportion donnée $p$, quelle est le seuil qui lui correspond ? \fg.

Dans cette section, on considère les proportions $p=0,5 ; \frac14 ; et \frac34$ qui correspondent à la médiane, au premier et au troisième quartile.

\dfn{médiane}{
	La médiane d'une série $X = (x_1 ; \dots ; x_N)$ rangée par ordre croissant est donné par 
		\begin{enumerate}
			\item sa valeur centrale si $N$ est impair ; ou
			\item la moyenne de ses valeurs centrales si $N$ est pair.
		\end{enumerate}
		
	Ainsi $50\%$ des valeurs de $X$ sont inférieure à sa médiane.
}{}

\dfn{quartiles}{
	Le premier quartile $Q_1$ d'une série $X = (x_1 ; \dots ; x_N)$ rangée par ordre croissant est donné par la plus petite valeur de la série de rang supérieur ou égal à $\frac{N}4$.
	
	Ainsi $25\%$ des valeurs de $X$ sont inférieure à son premier quartile.
	
	Le troisième quartile $Q_3$ d'une série $X = (x_1 ; \dots ; x_N)$ rangée par ordre croissant est donné par la plus petite valeur de la série de rang supérieur ou égal à $\frac{3N}4$.
	
	Ainsi $75\%$ des valeurs de $X$ sont inférieure à son troisième quartile.
}{}

\dfn{écart interquartile}{
	L'écart interquartile d'une série statistique $X$ est donné par la distance entre ses deux quartiles $Q_1$ et $Q_3$, c'est-à-dire
		\[ |Q_1 - Q_3|. \]
	En supposant que $4|N$ pour faciliter les notations, on a la situation suivante.
	\[ x_1 \leq \dots \leq \underbrace{ \overbrace{x_{N/4}}^{Q_1} \leq \dots \leq \overbrace{x_{3N/4}}^{Q_3}}_{\frac{N}2 \text{ valeurs centrales}} \leq \dots \leq x_N. \]
	Les $\frac{N}2$ valeurs centrales sont à distance $|Q_1 - Q_3|$ les unes des autres.
}{}

\begin{figure}[t!]
  \centering
  \begin{subfigure}[b]{.45\textwidth}
    \centering
    \includegraphics[page=1]{figures/fig-statistiques.pdf}
  \caption{Évaluation ``Droite réelle''.}
  \label{fig:a}
  \end{subfigure}
  \hfill
  % NOT LINE BREAK!!
  \begin{subfigure}[b]{.45\textwidth}
    \centering
    \includegraphics[page=2]{figures/fig-statistiques.pdf}
  \caption{Évaluation ``Droite et plan''.}
  \end{subfigure}
    \begin{subfigure}[b]{.45\textwidth}
    \centering
    \includegraphics[page=3]{figures/fig-statistiques.pdf}
  \caption{Automatismes 1 à 5.}
  \end{subfigure}
  
  
  \caption{Histogrammes de notes (min $0$, max $20$). Les classes sont de la forme $[k;k+1[, k\in\N$. La colonne entre $12$ et $13$ compte donc toutes les notes appartenant à $[12;13[$.}
  \label{fig:hist}
\end{figure}

\exe{}{
	Écrire un tableau Valeur/Effectif pour chaque histogramme de la figure \ref{fig:hist}.
	On assignera la valeur moyenne $k+\frac12$ à chaque élément d'une classe $[k; k+1[, k\in\N$.
	Par exemple, de l'histogramme \ref{fig:a} on lit $4$ notes de valeur $16,5$.
	
	Pour chaque série obtenue, calculer
		\begin{multicols}{2}
		\begin{enumerate}
			\item La moyenne ;
			\item L'écart type ;
			\item La médiane ;
			\item Le premier quartile ;
			\item Le troisième quartile ; et
			\item L'écart interquartile.
		\end{enumerate}
		\end{multicols}
}{exe:stats8}{
	On traite la série statistique $X$ décrite par le premier histogramme \ref{fig:a}.
	On vérifie les résultats à l'aide d'un tableur : 
	voir le fichier joint \texttt{hist(a).csv}.
	
		\begin{center}
		\begin{tabular}{|c|c|c|c|c|c|c|c|c|c|c|c|c|c|c|}\hline
		Note & 2,5 & 5,5 & 6,5 &7,5 & 8,5 & 9,5 & 10,5 & 12,5 & 13,5 & 14,5 & 15,5 & 16,5 & 17,5 & 18,5 \\ \hline
		Effectif & 1 & 1 & 1 & 3 & 1 & 5 & 1 & 3 & 4 & 3 & 3 & 5 & 1 & 2 \\ \hline
		\end{tabular}
		\end{center}
		
	L'effectif total est donné par $N= 1+1+1+3+1+5+1+3+4+3+3+5+1+2 = 34$
	
		\begin{enumerate}[label=\roman*)]
			\item
				\[ \overline{X} = \dfrac{2,5+5,5+6,5+7,5\cdot3 + \dots  + 16,5\cdot5 + 17,5+18,5\cdot2}{34} = \dfrac{422}{34} \approx 12,411 \]
			\item 
				\[ \sigma(X) \approx 4,06. \]
			\item 
				$N=34$ est pair donc prend la moyenne des valeur de rang $17$ et $18$ en vérifiant que la série soit bien rangée par ordre croissant.
				Comme $1+1+1+3+1+5+1+3 = 16$, ces deux valeurs sont $13,5$ et la médiane est donc $13,5$.
			\item 
				On prend le rang immédiatement supérieur ou égal à $N/4 = 8,5$, c'est-à-dire $9$.
				Le premier quartile est donc donné par $Q_1 = 9,5$.
			\item
				On prend le rang immédiatement supérieur ou égal à $3N/4 = 25,5$, c'est-à-dire $26$.
				Le troisième quartile est donc donné par $Q_3 = 15,5$. 
			\item
				L'écart interquartile est calculé, par définition, en posant
					\[ |Q_1 - Q_3| = |9,5 - 15,5| = 6. \]
				
		\end{enumerate}
}



\exe{, difficulty=1}{
	Pour quel(s) $N\in\N$ la moyenne de la série suivante est-elle $10$ ? Comparer avec la série de l'exercice \ref{exe:moyenne-effectif}.
	
		\begin{center}
		\begin{tabular}{|c|c|c|c|c|}\hline
		Valeur   & 0 & 20 & 2 & 5 \\ \hline
		Effectif & 5 & 14 & N & 2  \\ \hline
		\end{tabular}
		\end{center}
}{exe:mean-is-10}{
	La contrainte $\overline{X} = 10$ se traduit par l'égalité suivante à résoudre pour $N\in\N$.
	
		\begin{align*}
			\overline{X} &= 10 \\
			\dfrac{5\cdot0 + 14\cdot20 + N\cdot2 + 2\cdot5}{5+14+N+2} &= 10 \\
			0  + 280 + 2\cdot N + 10 &= 10 \cdot(21 + N) \\
			290 + 2 \cdot N &= 210 + 10 \cdot N \\
			80 &= 8 \cdot N \\
			N &= 10		
		\end{align*}

	$10$ est bien un entier naturel, donc $N=10$ est l'unique solution.
}

\exe{, difficulty=1}{
	Est-il possible que la série suivante ait une moyenne de $10$ pour un certain $N\in\N$ ? Si oui, lequel ?
	
		\begin{center}
		\begin{tabular}{|c|c|c|c|c|}\hline
		Valeur   & 23 & 12 & 2 & 7 \\ \hline
		Effectif & 5 & 14 & N & 2  \\ \hline
		\end{tabular}
		\end{center}
}{exe:mean-is-10bis}{
	La contrainte $\overline{X} = 10$ se traduit par l'égalité suivante à résoudre pour $N\in\N$.
	
		\begin{align*}
			\overline{X} &= 10 \\
			\dfrac{5\cdot23 + 14\cdot12 + N\cdot2 + 2\cdot7}{5+14+N+2} &= 10 \\
			115  + 168 + 2\cdot N + 14 &= 10 \cdot(21 + N) \\
			297 + 2 \cdot N &= 210 + 10 \cdot N \\
			87 &= 8 \cdot N \\
			N &= \dfrac{87}8		
		\end{align*}

	L'unique solution candidate est donc $\dfrac{87}{8}$ qui n'est pas un entier naturel.
	Donc il n'est pas possible que cette série statistique ait une moyenne de $10$, peu importe l'effectif $N\in\N$.
	
	En prenant une valeur approchée, par exemple $N=11$, la moyenne est de $9,96$, par exactement $10$ !
}


\exe{, difficulty=1}{
	Soient $a, b \in \N$ deux entiers naturels non tous deux nuls.
	On considère la série statistique suivante, dépendant des entiers $a$ et $b$.
	
		\begin{center}
		\begin{tabular}{|c|c|c|}\hline
			Valeur   & 9 & 12 \\ \hline
			Effectif & $a$ & $b$ \\ \hline
		\end{tabular}
		\end{center}
		
	\begin{enumerate}
		\item
		Montrer que, pour les couples $(a;b) = (2;1)$ et $(a;b) = (4;2)$, la moyenne de la série est de $10$.
		\item
		Montrer que, pour tous les couples de la forme $(a;b) = \kappa\cdot(2;1)$ où $\kappa\in\N$ est un entier naturel non nul, la moyenne de la série est de $10$.
		\item
		Représenter graphiquement ces couples dans un repère en prenant $\kappa=1; 2; 3; \dots$.
		Que dire des points ?
	\end{enumerate}
}{exe:fixed-mean}{
	TODO
}


\exe{, difficulty=1}{
	Soient $a, b \in \N$ deux entiers naturels tels que $(a;b) \neq (0;0)$.
	On considère la série statistique suivante, dépendant des entiers $a$ et $b$.
	
		\begin{center}
		\begin{tabular}{|c|c|c|}\hline
			Valeur   & 7 & 12 \\ \hline
			Effectif & $a$ & $b$ \\ \hline
		\end{tabular}
		\end{center}

	\begin{enumerate}
		\item
		Montrer que, pour les couples $(a;b) = (2;3)$ et $(a;b) = (4;6)$, la moyenne de la série est de $10$.
		\item
		Montrer que, si la moyenne de la série associée à un couple $(a;b)$ est de $10$, alors le couple vérifie
			\[ 2b = 3a. \]
		\item
		En déduire que l'entier naturel $a$ est nécessairement pair.
		(on pourra raisonner par contraposition : si $a$ est impair, montrer que $3a$ est aussi impair).
		\item
		En écrivant $a=2 k$ où $k\in\N$ est un entier naturel non nul, démontrer que $b=3 k$. 
		\item
		Déduire que tous les couples $(a;b)$ dont la série associée est de moyenne $10$ sont de la forme
			\[ (a;b) = (2;3)\cdot k, \]
		où $k\in\N$ est un entier naturel non nul.
	\end{enumerate}
}{exe:mean-dioph}{
	TODO
}


\exe{, difficulty=2}{
	Soient $a, b \in \N$ deux entiers naturels tels que $(a;b) \neq (0;0)$.
	On considère la série statistique suivante, dépendant des entiers $a$ et $b$.
		\begin{center}
		\begin{tabular}{|c|c|c|c|}\hline
			Valeur   & 11 & 8 & 13 \\ \hline
			Effectif & 1 & $a$ & $b$ \\ \hline
		\end{tabular}
		\end{center}

	\begin{enumerate}
		\item
		Montrer que, pour les couples $(a;b) = (2;1)$ et $(a;b) = (5;3)$, la moyenne de la série est de $10$.
		\item
		Montrer que, si la moyenne de la série associée à un couple $(a;b)$ est de $10$, alors le couple vérifie
			\[ 2a - 3b = 1. \]
		\item
		En déduire que l'entier naturel $b$ est nécessairement impair.
		(On peut raisonner par contradiction : si $b$ est pair, montrer que l'égalité est impossible).
		\item
		En écrivant $b=2k + 1$ où $k\in\N$ est un entier naturel, démontrer que $a = 3k + 2$. 
		\item
		Déduire que tous les couples $(a;b)$ dont la série associée est de moyenne $10$ sont de la forme
			\[ (a;b) = (3;2)\cdot k + (2;1), \]
		où $k\in\N$ est un entier naturel.
		\item
		Représenter graphiquement ces couples dans un repère en prenant $k=1; 2; 3; \dots$.
		Que dire des points ?
	\end{enumerate}	
}{exe:mean-dioph2}{
	TODO
}







% j'aime pas les pie chart c'est inutile
%\def\angle{0}
%\def\radius{3}
%\def\cyclelist{{"myg","gray","myr","myb"}}
%\newcount\cyclecount \cyclecount=-1
%\newcount\ind \ind=-1
%\begin{figure}
%  \begin{tikzpicture}
%      \foreach \percent/\name in {
%        75/{préoccupation mineure (LC)},
%        6/{données insufficantes (DD)},
%        14/{éteintes ou menacées (EX à VU)},
%        5/{quasi menacées (NT)}
%    } {
%      \ifx\percent\empty\else               % If \percent is empty, do nothing
%        \global\advance\cyclecount by 1     % Advance cyclecount
%        \global\advance\ind by 1            % Advance list index
%        \ifnum3<\cyclecount                 % If cyclecount is larger than list
%          \global\cyclecount=0              %   reset cyclecount and
%          \global\ind=0                     %   reset list index
%        \fi
%        \pgfmathparse{\cyclelist[\the\ind]} % Get color from cycle list
%        \edef\color{\pgfmathresult}         %   and store as \color
%        % Draw angle and set labels
%        \draw[fill={\color!50},draw={\color}] (0,0) -- (\angle:\radius)
%          arc (\angle:\angle+\percent*3.6:\radius) -- cycle;
%        \node at (\angle+0.5*\percent*3.6:0.7*\radius) {\percent\,\%};
%        \node[pin=\angle+0.5*\percent*3.6:\name]
%          at (\angle+0.5*\percent*3.6:\radius) {};
%        \pgfmathparse{\angle+\percent*3.6}  % Advance angle
%        \xdef\angle{\pgfmathresult}         %   and store in \angle
%      \fi
%    };
%  \end{tikzpicture}
%  \label{fig:pie-chart}
%  \caption{Répartition par niveau de menace des espèces présentes en France et évaluées dans la liste rouge mondiale de l'UICN ($15579$ espèces). Source : \href{naturefrance.fr}{naturefrance.fr}, $2023$.}
%\end{figure}


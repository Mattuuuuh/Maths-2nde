%!TEX encoding = UTF8
%!TEX root = 0-notes.tex

\chapter{Vecteurs}
\label{chap:vecteurs}


\section{Introduction}

\subsection{Translation du plan}

\dfn{translation du plan}{
	Une \emphindex{translation} $T$ du plan est le déplacement de tous les points du plan selon la même direction, la même distance, et dans le même sens.
	
	Pour un point $A$ quelconque, on définit l'addition
		\[ A + T \]
	comme le \emphindex{translaté} de $A$ par $T$.
}{}

\nt{
	Une translation $T$ déplace tous les points du plan selon la même direction, la même distance, et dans le même sens.
	On peut donc décomposer $T$ en une translation \og gauche/droite \fg~ selon l'axe des abscisses, et une translation \og haut/bas \fg selon les ordonnées.
	
	Chaque point $A(x_A ; y_A)$ voit ses coordonées modifiées de la même manière :
	celui-ci se déplace systématiquement d'une certaine quantité $x_T$ en abscisse, et $y_T$ en ordonnée.
	Ainsi, une fois déplacé, le point $A$ admet pour nouvelles coordonnées
		\[ A + T = (x_A + x_T ; y_A + y_T). \]
	Remarquons que c'est exactement comme cela qu'on a défini l'addition de deux points, définition \ref{def:manip-points}.
	On pose donc $T = (x_T ; y_T)$ en réutilisant la notation $(x ; y)$ des points du plan.
	
	Il n'y a en fait pas de différence mathématique entre une translation et un point : à chaque translation on peut associer un point, et vice versa (voir corollaire \ref{cor:OAisA}).
}{}

\begin{figure}
	\centering
	\includegraphics[page=4]{figures/fig-vecteurs.pdf}
	\caption[argument for footnotemark]{Le canard Hilbert\footnotemark translaté de $x_T$ unités horizontalement puis $y_T$ unités verticalement.}
\end{figure}

\dfn{vecteur $\vec{AB}$}{
	Soient $A, B$ deux points du plan.
	On nomme 
		\[ T = \vec{AB} \]
	la translation $T$ du plan qui envoie le point $A$ sur le point $B$.
	On a alors la relation
		\[ A + \vec{AB} = B. \]
	
	En décomposant $T$ selon les abscisses et les ordonnées, on écrit ses coordonnées sous la forme $T = \begin{pmatrix} x_T \\ y_T \end{pmatrix}$.
}{}

\ex{}{
	Soient $A(2,5; 3), B(1,5; -4)$ deux points.
	On trace les points dans un repère, ainsi qu'une flèche envoyant $A$ sur $B$, qui correspond au vecteur $\vec{AB}$.
	
	On décompose la translation selon les deux axes : translater de $-1$ en abscisse, puis translater de $-7$ en ordonnée.
	On en déduit que $\vec{AB} = \pvec{-1}{-7}$.
	
	\begin{center}
	\includegraphics[page=1]{figures/fig-vecteurs.pdf}
	\end{center}
}{ex:vec-1}

\footnotetext{David Hilbert (1862-1943), mathématicien allemand. \emph{ « Wir müssen wissen, wir werden wissen. »}}

\mprop{calcul de $\vec{AB}$}{
	Par définition, on a 
		\[ A + \vec{AB} = B, \]
	qui permet de calculer la translation $\vec{AB}$ à l'aide de la relation
		\[ \vec{AB} = B - A = \pvec{x_B - x_A}{y_B-y_A}. \]
}{prop:calcul-AB}

\cor{}{
	Pour connaître les coordonnées d'une translation, il suffit de l'appliquer à l'origine $O$ et de lire les coordonnées du translaté.
	
	En d'autres termes,
		\[ \vec{OA} = A. \]
}{cor:OAisA}

\dfn{opérations sur les vecteurs}{
	Les vecteurs héritent des opérations légales sur les points (définition \ref{def:manip-points}).
	Soient $u = \pvec{x}{y}, v = \pvec{x'}{y'}$ deux vecteurs et $\kappa\in\R$ un réel. 
	Alors
	\begin{multicols}{2}
	\begin{enumerate}
		\item $u+v = \pvec{x+x'}{y+y'}$ ; et
		\item $\kappa \cdot u = \pvec{\kappa \cdot x}{\kappa\cdot y}$.
	\end{enumerate}
	\end{multicols}
}{}

\subsection{Composition de translations : somme de vecteurs}

\thm{relation de Chasles}{
	Soient $A, B, C$ trois points du plans. 
	Alors
		\[ \vec{AB} + \vec{BC} = \vec{AC}. \]
		\begin{center}
		\og L'addition de la translation qui envoie $A$ sur $B$ à celle qui envoie $B$ sur $C$ est égale à la translation qui envoie $A$ sur $C$ \fg.
		\end{center}
}{thm:chasles}

%\pf{Démonstration du théorème \ref{thm:chasles}}{
\pf{}{
	Par calcul direct à l'aide de la proposition \ref{prop:calcul-AB},
	\begin{align*}
		\vec{AB} + \vec{BC} &= \left(B-A\right) + \left(C - B\right), \\
								&= B - A + C - B, \\
								&= C - A, \\
								&= \vec{AC}.
	\end{align*}
}{}

\ex{Somme de vecteurs géométriquement}{
	Soient ${\color{RED_E} u} = \pvec{2}{-1}$ et ${\color{RED_E} v}= \pvec{-3}{-2}$ deux vecteurs.
	
	Pour construire géométrique la somme ${\color{RED_E} u}+{\color{BLUE_E} v} =\pvec{2-3}{-1-2} = \pvec{-1}{-3}$, on colle bout à bout les vecteurs ${\color{RED_E} u}$ et ${\color{BLUE_E} v}$ comme ci-dessous.
	
	On démarre à l'origine $O$ pour pouvoir lire les coordonnées du vecteur, d'après le corollaire \ref{cor:OAisA}.
	
	\begin{center}
	\includegraphics[page=2]{figures/fig-vecteurs.pdf}
	\end{center}
}{}

\exemulticols{}{
	À l'aide des vecteurs $u, v, w$ donnés dans le plan ci-contre, construire les sommes
		\begin{align*}
			a &= u+v+w, \\
			b &= -2u, \\
			c &= 3u - \dfrac12v - \dfrac13w.
		\end{align*}
	\vfill\null
}{
	\begin{center}
	\includegraphics[page=5]{figures/fig-vecteurs.pdf}
	\end{center}
}{exe:somme-geom}{
	TODO
}

\exe{}{
	À l'aide de la figure suivante et de la relation de Chasles, déterminer les sommes vectorielles suivantes en complétant les pointillés.

	\begin{center}
	\includegraphics[page=6]{figures/fig-vecteurs.pdf}
	\end{center}
	
	\begin{multicols}{2}
	\begin{enumerate}
		\item $\vec{BC} = \vec{E \dots}$
		\item $\vec{BE} = \vec{\dots F}$
		\item $\vec{AB} + \vec{FA} = \vec{F \dots} + \vec{\dots B} = \vec{\vphantom{A} \dots\dots}$
		\item $\vec{FD} + \vec{EB} = \vec{FD} + \vec{D\dots} = \vec{\vphantom{A}\dots\dots}$
		\item $\vec{AE} + \vec{GB} + \vec{DH} = \vec{AE} + \vec{E\dots} + \vec{\vphantom{A}\dots\dots} = \vec{\vphantom{A}\dots\dots}$
	\end{enumerate}
	\end{multicols}

}{exe:somme-sesamaths}{
	TODO
}

\subsection{Norme vectorielle}

On souhaite désormais parler de « longueur » de translation.
Pour cela, considérons $\vec{AB}$ le vecteur translatant le point $A$ sur le point $B$.
Un définition naturelle de la « longueur » de $\vec{AB}$ est la distance entre les points $A$ et $B$, soit $\norm{A - B}$, comme décrit dans la définition \ref{dfn:norme-carré}.
En mathématiques, on parlera alors de \emphindex{norme vectorielle}.

\dfn{norme d'un vecteur}{
	Soit $u = \pvec{x}{y}$ un vecteur quelconque.
	On définit sa \emphindex{norme} $\norm{u}$ par
		\[ \norm{u} = \sqrt{x^2 + y^2}. \]
		
	Il suit, par construction, que $\norm{\vec{AB}} = AB$.
}{dfn:norme}

\ex{}{
	Soient $A(2,5; 3), B(1,5; -4)$ les deux points de l'exemple \ref{ex:vec-1}.
	On a $\vec{AB} = B - A = \pvec{-1}{-7}$, et donc
		\[ \norm{\vec{AB}} = \norm{\pvec{-1}{-7}} = \sqrt{\left(-1\right)^2 + \left(-7\right)^2} = \sqrt{50} = \sqrt{25 \times 2} = 5 \sqrt{2}. \]
		
	En comparant avec la longueur $AB = BA = \sqrt{(2,5 - 1,5)^2 + (3 - (-4))^2} = 5 \sqrt{2}$, on obtient bien le même résultat.
	Notons qu'il y a moins de chance de faire une erreur avec la norme vectorielle car on calcule la longueur en deux étapes, avec d'abord les différences de coordonnées.
	
%	\begin{center}
%	\includegraphics[page=3]{figures/fig-vecteurs.pdf}
%	\end{center}
}{}

\exe{1}{
	Calculer la norme des vecteurs suivants. L'exprimer sous la forme $a \sqrt{b}$, où $a, b \in\N$ sont des entiers naturels et $b$ est le plus petit possible.
	\begin{multicols}{3}
	\begin{enumerate}	
		\item $u = \pvec34$
		\item $v = -\pvec34$
		\item $w = \pvec{-1}2$
		\item $u' = \pvec5{-7}$
		\item $v' = \pvec20$
		\item $w' = \pvec0{-5}$
	\end{enumerate}
	\end{multicols}
}{exe:norme-vectorielle}{
	TODO
}

\thm{homogénéité\index{homogénéité} de la norme}{
	Soit $u$ un vecteur et $\kappa\in\R$ un nombre réel.
	Alors
		\[ \norm{\kappa\cdot u} = |\kappa| \cdot \norm{u}. \]
}{}

\pf{}{
	Soit $u = \pvec{x}y$ et $\kappa\in\R$.
	Alors
	\begin{align*}
		\norm{\kappa\cdot u}^2 &= \norm{\pvec{\kappa x}{\kappa y}}^2, \\
								&= (\kappa x)^2 + (\kappa y)^2, \\
								&= \kappa^2 (x^2 + y^2) = \kappa^2 \norm{\pvec{x}y}.
	\end{align*}
	La racine carré conclut en se remémorant que $\sqrt{\kappa^2} = |\kappa|$.
}


\exe{}{
	Calculer la norme des vecteurs suivants. L'exprimer sous la forme $a \sqrt{b}$, où $a, b \in\N$ sont des entiers naturels et $b$ est le plus petit possible.
	\begin{align*}
		u = \pvec{-2}{4}, && v = 2u, && w = -\dfrac32u, && z = -\dfrac12u.
	\end{align*}
}{exe:norme-homogènre}{
	TODO
}

\exe{}{
	Pour quels réels $x\in\R$ a-t-on $\norm{x \cdot \pvec12} = 5$ ?
}{exe:homogénéité}
{
	TODO
}


\section{Colinéarité}

\dfn{vecteurs colinéaires}{
	Un vecteur $u$ est  \emphindex{vecteur colinéaire} à un vecteur $v$ dès qu'il existe un réel $\kappa\in\R$ tel que
		\begin{align*}
			u = \kappa \cdot v, && \ou && v = 0.
			%v = \kappa\cdot u.
		\end{align*}
	Deux vecteurs colinéaires admettent la même direction, mais pas forcément le même sens ni la même norme.
}{}

\nt{
	Tous les vecteurs sont colinéaires au vecteur nul (en prenant $\kappa = 0$ ci-dessus).
}{}

\exe{}{
	Montrer que la relation de colinéarité est réflexive : $u$ est toujours colinéaire à lui-même.
}{exe:col-reflex}{
	TODO
}

\exe{}{
	Montrer que la relation de colinéarité est symétrique : si $u$ est colinéaire à $v$, alors $v$ est colinéaire à $u$.
}{exe:col-sym}{
	TODO
}

\nomen{
	Par symétrie, on parle alors de \emphindex{vecteurs colinéaires} sans spécifier quel vecteur est colinéaire à l'autre.
}

\exe{}{
	Montrer que la relation de colinéarité est transitive : si $u$ est colinéaire à $v$, et $v$ est colinéaire à $w$, alors $u$ est colinéaire à $w$.
}{exe:col-trans}{
	TODO
}

\lem{}{
	Soit $(d)$ une droite non verticale et $A, B \in (d)$ deux points distincts de celle-ci.
	
	Alors $\vec{AB}$ est colinéaire à un vecteur de la forme $\pvec{1}{a}$ où $a$ est le coefficient directeur de la fonction affine $f$ telle que $\C_f = (d)$.
}{lem:vecteur-dir-coeff-dir}

%\pf{Preuve du lemme \ref{lem:vecteur-dir-coeff-dir}}{
\pf{}{
	En écrivant $A(x_A;y_B)$ et $B(x_B;y_B)$ les coordonnées des points $A$ et $B$, on a $x_A \neq x_B$ car les points sont distincts et la droite est non verticale.
	
	Ainsi, $\vec{AB} = B-A$ est égal à
		\[\vec{AB}= \pvec{x_B - x_A}{y_B-y_A} = \left(x_B - x_A\right) \pvec{1}{\frac{y_B-y_A}{x_B-x_A}}. \]
	On reconnaît bien ici le coefficient directeur de la fonction affine associée à $(d)$.
}{}

\section{Alignement et vecteur directeur}

\thm{}{
	Soit $A, B, C, D$ quatre points quelconques.
	
	Les vecteurs $\vec{AB}$ et $\vec{CD}$ sont colinéaires si et seulement si les droites $(AB)$ et $(CD)$ sont parallèles.
}{thm:alignement-parallélisme}

\exe{,difficulty=2}{
	Démontrer le théorème \ref{thm:alignement-parallélisme}.
}{exe:alignement-parallélisme}{
	TODO
}

\cor{}{
	Trois points $A, B, C$ quelconques sont alignés si et seulement si les vecteurs $\vec{AB}$ et $\vec{AC}$ sont colinéaires.
}{prop:alignement-colinéarité}

\exe{}{
	Considérons les points
		\begin{align*}
			A(-1; -6), && B(-3; -12), && C(5; 12), && D(7 ; 17).
		\end{align*}
	
	\begin{enumerate}
		\item Le point $C$ appartient-il à la droite $(AB)$ ?
		\item Le point $D$ appartient-il à la droite $(CB)$ ?
		\item Trouver un point $E$ distinct de $A, B, C, D$ et appartenant à $(AB)$.
	\end{enumerate}
}{exe:alignement}{
	TODO
}

\dfn{vecteur directeur}{
	Soit $(d)$ une droite quelconque.
	On appelle $v$ \emphindex{vecteur directeur} de $(d)$ si $v$ et $(d)$ admettent la même direction.
	
	Pour $A\in(d)$, on a alors
		\[ (d) = \bigset{ A + k \cdot v, \text{ où $k$ parcourt $\R$} }. \]
	La droite $(d)$ est égale à tous les translatés d'un de ses points $A$ par les multiples de $v$.
}{}

\nomen{
	On dit que le vecteur $v$ \emph{dirige} la droite $(d)$.
}

\nt{
	Pour deux points $A \neq B$ quelconques d'une droite $(d)$, le vecteur $\vec{AB}$ dirige $(d)$.
}

\exe{}{
	On considère un point $A(-4;5)$ et un vecteur $v = \pvec{3}{-2}$.
	Soit $(d)$ la droite passant par $A$ et dirigée par $v$.
	
	\begin{enumerate}
		\item
		Donner $4$ points distincts appartenant à $(d)$.
		\item
		Trouver la fonction affine $f$ telle que $(d) = \C_f$.
	\end{enumerate}
}{exe:vec-dir-to-affine}{
	TODO
}

\section{Déterminant}

\dfn{déterminant de deux vecteurs}{
	Soient $u = \pvec{a}{b}$ et $v=\pvec{c}{d}$ deux vecteurs.
	On définit le \emphindex{déterminant} $\det(u, v)$ par le nombre réel
		\begin{align*}
			\det(u, v) = \begin{vmatrix} a & c \\ b & d \end{vmatrix} = ad - bc.
		\end{align*}
}{}

\thm{}{
	Soient $u, v$ deux vecteurs.
	Alors $\det(u, v) = 0$ si et seulement si les vecteur $u$ et $v$ sont colinéaires.
}{}

\nt{	
	Le déterminant \emph{détermine} donc si deux vecteurs sont colinéaires ou non.
}

% la démonstration est chiante en fait
\pf{}{
	Notons $u = \pvec{a}{b}$, $v=\pvec{c}{d}$. 
	Si $u$ est nul, et donc colinéaire à tout vecteur $v$, il est facile de voir que $\det(u, v) = 0$.
	Dans ce cas, les deux propositions sont vraies et donc équivalentes.
	
	Sinon, supposons que $u\neq0$. Sans perte de généralité (quitte à échanger les coordonées), on a donc $a\neq0$.
	\begin{enumerate}[leftmargin=120pt]
		\item[\underline{Direction $\implies$ :}] 
		Si $\det(u, v) = ad - bc = 0$, alors $ad = bc$.
		Posons $\kappa = \dfrac{c}a$.
		Alors $d = \kappa b$ et $c = \kappa a$, donc $u = \kappa v$.
		
		\item[\underline{Direction $\impliedby$ :}] 
		Notons $\kappa\in\R^*$ tel que $u = \kappa v$.
		Alors $a = \kappa c$ et $b = \kappa d$.
		Il suit que $ad -bc = (\kappa c)d - (\kappa d)c = 0$.
	\end{enumerate}
}

\exe{}{
	On considère le point $A(-4; -2)$ et les deux vecteurs $u = \pvec{6}{2}$ et $v = \pvec{1}{-3}$.
	On pose de plus les trois points
		\begin{align*}
			B = A + u, && C = B + v, && D = C - u.
		\end{align*}	
	\begin{enumerate}
		\item 
		Montrer que $B(2;0), C(3;-3)$, et $D(-3;-5)$ puis dessiner le quadrilatère $ABCD$ dans un repère.
		
		\item
		\begin{enumerate}[label=(\alph*)]
			\item Calculer les vecteurs $\vec{AB}$ et $\vec{DC}$, et montrer qu'ils sont colinéaires.
			\item Calculer les vecteurs $\vec{AD}$ et $\vec{CB}$, et montrer qu'ils sont colinéaires.
			\item Montrer qu'on a $\norm{\vec{AB}}= 2\sqrt{10}$, $\norm{\vec{AD}} = \sqrt{10}$, et $\norm{\vec{BD}} = 5\sqrt{2}$.
			\item Vérifier qu'on ait bien $\det\Bigl(\vec{AB}, \vec{AD}\Bigr) = -20$.
		\end{enumerate}
		
		\item
		\begin{enumerate}[label=(\alph*)]
			\item Montrer que le quadrilatère $ABCD$ est un parallélogramme à l'aide des questions 2(a-b).
			\item Montrer que le triangle $ABD$ est rectangle en $A$ à l'aide de la réciproque du théorème de Pythagore et de la question 2(c).
			\item En déduire que $ABCD$ est un rectangle et calculer son aire à l'aide de la question 2(c).
			\item Comparer l'aire obtenue à $\left| \det\Bigl(\vec{AB}, \vec{AD}\Bigr) \right|$, la valeur absolue du déterminant calculé à la question 2(d).
		\end{enumerate}
		
	\end{enumerate}
}{exe:aire-det-particulier}{
	TODO
}

\nomen{
	On dit de deux vecteurs non nuls qu'ils sont \emphindex{vecteurs perpendiculaires} dès que les droites qu'ils dirigent sont perpendiculaires.
}

\nt{
	L'exercice \ref{exe:det-aire} vise à montrer que, lorsque les vecteurs $u$ et $v$ sont perpendiculaires, la valeur absolue du déterminant $|\det(u, v)|$ est égal à l'aire du parallélogramme de sommets $O, O+u, O+u+v, O+v$.
	
	Il s'avère que ce résultat est vrai sans l'hypothèse de perpendicularité entre les vecteurs.
}

\exe{}{
	Tracer la droite d'équation $4x - y = 0$ et le vecteur $\pvec4{-1}$ dans un même repère.
	Que remarque-t-on ?
}{exe:axbyz2}{
	TODO
}

\exe{, difficulty=1}{
	Soient $a, b \in\R$ deux nombres.
	Montrer, à l'aide du théorème de Pythagore, qu'un couple (x ; y) vérifie $ax+by=0$ si et seulement si les vecteurs $\pvec{x}{y}$ et $\pvec{a}{b}$ sont perpendiculaires.
		\[ ax+by=0 \iff \pvec{x}{y} \perp \pvec{a}{b}. \]
}{exe:produit-scalaire}{
	On considère le triangle $OAX$ où $O(0;0)$, $A(a;b)$ et $X(x ; y)$.
	Les vecteurs $\pvec{x}{y}$ et $\pvec{a}{b}$ sont perpendiculaires si et seulement si les droites $(OA)$ et $(OX)$ le sont.
	
	D'après le théorème de Pythagore (et sa réciproque), on a
		\[ (OA) \perp (OX) \iff OA^2 + OX^2 = AB^2. \]
	Or $OA^2 = a^2 + b^2$, $OB^2 = x^2 + y^2$, et $AB^2 = (a-x)^2 + (b-y)^2$.
	L'égalité de Pythagore devient donc
		\begin{align*}
			(OA) \perp (OX) &\iff a^2 + b^2 + x^2 + y^2 = (a-x)^2 + (b-y)^2, \\
				&\iff a^2 + b^2 + x^2 + y^2 = a^2 + x^2 - 2ax + b^2 + y^2 - 2by, \\
				&\iff 0 = -2(ax + by), \\
				&\iff ax + by = 0.
		\end{align*}
}

\exe{, difficulty=2}{
	Soient $u = \pvec{a}{b}, v = \pvec{c}{d}$ deux vecteurs tels que le triangle de sommets $O(0;0), O+u, O+v$ est rectangle en $O$. 
	\begin{enumerate}
		\item Faire un dessin du triangle rectangle et donner les coordonnées de chacun de ses sommets.
		\item Utiliser l'exercice \ref{exe:produit-scalaire} pour montrer que
			\[ ac + bd = 0. \]
		\item
		Montrer à l'aide de la relation ci-dessus que
			\[ \norm{u}^2 \cdot \norm{v}^2 = \bigl( \det(u, v) \bigr)^2. \]
	\end{enumerate}
}{exe:det-aire}{
	TODO
}



\exe{, difficulty=1}{
\def\arraystretch{1} % for pmatrix
	On étudie le \emphindex{système homogène} suivant qu'on ne cherche pas à résoudre.
		\[ \systeme{12x - 5y = 0{,}, -7x + 2y = 0.} \]
	\begin{enumerate}
		\item 
		Montrer que le système est équivalent à l'égalité de vecteurs
			\[ x \cdot u + y \cdot v = \pvec{0}{0}, \]
		pour les deux vecteurs $u = \pvec{12}{-7}$ et $v = \pvec{-5}{2}$.
	
		\item
		Montrer qu'une solution $(x ;y)$ non nulle existe si et seulement si $u$ et $v$ sont colinéaires, et donc si et seulement si $\det(u, v) = 0$.
		
		\item	
		Déduire dans ce cas que l'unique solution du système est la solution nulle $(x ;y) = (0;0)$.
	\end{enumerate}
}{exe:systemes10}{
	TODO
}

\exe{, difficulty=1}{
	\begin{enumerate}
		\item
		Si $u$ et $v$ sont colinéaires, et $w$ un troisième vecteur quelconque, montrer que le système $x \cdot u + y \cdot v = w$ admet une solution si et seulement si $w$ est colinéaire à $u$ et $v$.

		\item
		À partir des vecteurs colinéaires $u = \pvec23$ et $v = \pvec{-6}{-9}$ et d'un vecteur $w$ à poser, construire un système linéaire $x \cdot u + y \cdot v = w$ n'ayant aucune solution.
		\item
		À partir des vecteurs colinéaires $u = \pvec1{-4}$ et $v = \pvec{-2}{8}$ et d'un vecteur $w$ à poser, construire un système linéaire $x \cdot u + y \cdot v = w$ ayant un nombre infini de solutions.
	\end{enumerate}
}{exe:systemes11}{
	TODO
}

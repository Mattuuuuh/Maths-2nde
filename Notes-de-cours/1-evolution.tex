%!TEX encoding = UTF8
%!TEX root = 0-notes.tex

\chapter{Évolution chiffrée}
\label{chap:évolution}


\section{Sous-populations}

On considère dans cette section des populations qu'on représente par un ensemble fini non vide car l'ordre des individus n'importe pas.

\ex{ensemble d'élèves}{
  On représente l'ensemble des $34$ élèves d'une classe par l'ensemble
  \[ E = \bigset{ 1 ; 2 ; 3 ; \dots ; 32 ; 33 ; 34 }. \]
  Chaque élève est associé à un nombre (par exemple sa place dans l'ordre alphabétique de la liste d'appel).
}{ex:pop-eleves}

Une sous-population de l'ensemble $E$ est une partie de l'ensemble : c'est un ensemble $F$ inclus dans $E$.
\[ F \subseteq E. \]

\notations{
	On note $|E| \in \N$ le cardinal d'un ensemble\footnote{Il existe d'autres notations : $\text{Card}(E)$ ou $\#E$ par exemple.}, c'est-à-dire le nombre d'éléments distincts qu'il contient.
	C'est un entier naturel.
}

\ex{ensembles d'élèves pairs}{
  Soit $E$ donné à l'exemple \ref{ex:pop-eleves}.
  L'ensemble
  \[ F = \{ 2 ; 4; 6 ; \dots ; 32 ; 34 \} \]
  vérifie $F\subseteq E$ et peut donc être compris comme une sous-population des élèves de l'exemple \ref{ex:pop-eleves}.
  Ce sont en fait tous les élèves ayant une position paire dans le classement utilisé.

  Les cardinaux des ensembles sont donnés pas
  \begin{align*}
    |E| = 34 && \text{et} && |F| = 17.
  \end{align*}
}{ex:pop-eleves2}

\dfn{proportion, pourcentage de sous-population}{
  Considérons deux ensembles finis $F \subseteq E$ avec $E$ non vide.
  La proportion d'éléments de $E$ appartenant à $F$ est donnée par
  \[ p = \dfrac{|F|}{|E|}. \]
  Lorsqu'elle est exprimée sous la forme d'une fraction de $100$, on parle alors de \emphindex{pourcentage} d'éléments de $E$ appartenant à $F$.
}{def:proportion}

\ex{proportion d'élèves pairs}{
  En reprenant les notations de l'exemple \ref{ex:pop-eleves2}, on trouve
  \[ p = \dfrac{17}{34} = \dfrac12 = \dfrac{50}{100} = 50\%. \]
  Ainsi la proportion d'élèves de $E$ appartenant à $F$ est $\frac12 = 50\%$.
}{}

\nt{
  Soient $F \subseteq E$ deux ensembles finis avec $E$ non vide.
  On a alors forcément
  \[ 0 \leq |F| \leq |E|, \]
  et donc
  \begin{align*}
    0 \leq p \leq 1 && \iff && p \in [0 ; 1] && \iff 100\cdot p \in [0 ; 100].
  \end{align*}
  La proportion d'une sous-population est forcément inférieure à $1$ (et le pourcentage inférieur à $100$).
}

\nt{
  On convertit une proportion $p \in [0;1]$ en un pourcentage en la multipliant par $100$.
  Ainsi, un pourcentage peut s'approximer simplement à l'aide des deux premières décimales de $p$.
  \begin{multicols}{2}
  \begin{enumerate}
  \item $1 = 100\%$
  \item $\dfrac12 =0,5 = 50\%$
  \item $\dfrac14 = 0,25 = 25\%$
  \item $\dfrac15 = 0,2 = 20\%$
  \item $\dfrac1{10}=0,1 = 10\%$
  \item $\dfrac1{100} = 0,01 = 1\%$
  \item $\dfrac1{50} = 0,02 = 2\%$
  \item $0 = 0\%$
  \end{enumerate}
  \end{multicols}

  Les valeurs suivantes sont approximatives mais très utiles lorsqu'on souhaite estimer des pourcentages.

  \begin{multicols}{2}
  \begin{enumerate}
  \item $\dfrac13 \approx 33\%$
  \item $\dfrac23 \approx 66\%$
  \item $\dfrac17 \approx 14\%$
  \item $\dfrac19 \approx 11\%$
  \end{enumerate}
  \end{multicols}
  \vspace{3pt} % fractions goes out of bounds
}

\ex{proportion de fourmis}{
  Au total sur Terre on dénombre actuellement 16 600 espèces de fourmis parmis les 1,3 millions d'espèces d'insectes déjà décrites.

  Ainsi, la proportion de fourmis parmis les insectes est la suivante.
  \[ \dfrac{16 600}{1,3\times10^{6}} \approx 0,013 = \dfrac{1,3}{100} = 1,3 \%. \]
}{ex:fourmis}


\nt{
	Considérons $F \subseteq E$ deux ensembles finis avec $E$ non vide.
	Supposons qu'on connaisse la proportion $p = \dfrac{|F|}{|E|}$.
	Alors
	\begin{align*}
		|F| = p \cdot |E| && \text{et} && |E| = p^{-1} |F| = \dfrac1p |F|
	\end{align*}  
	En connaissant la proportion de $F$ dans $E$, on peut donc déduire la taille de $F$ à partir de celle de $E$ ou vice-versa.

	\textbf{\warning On ne peut pas déduire l'ensemble en lui-même, uniquement son cardinal !}
}

\ex{}{
  En reprenant les notations de l'exemple \ref{ex:pop-eleves2}, extraire $50\%$ de l'ensemble $E$ donne un ensemble de taille $\dfrac12 \cdot 34 = 17$.
  Cependant, il n'est pas possible de connaître ce sous-ensemble, car l'ensemble $F$ des élèves pairs et l'ensemble
  \[ G = \{ 1 ; 3 ; 5 ; \dots ; 29 ; 31 ; 33 \} \]
  des élèves impairs sont disjoints et tous les deux de cardinal $17$.
}{}

\ex{}{
  On continue l'exemple \ref{ex:fourmis}.
  En extrayant $1,3\%$ d'une population de $1,3$ millions d'individus, on obtient $16900$ individus.
  \[ \dfrac{1,3}{100} \cdot 1,3\times10^{6} = 0,013 \cdot 1,3\times10^{6} = 16~900. \]
  L'approximation ($\approx$) de l'exemple \ref{ex:fourmis} explique la différence entre les valeurs trouvées.

  Réciproquement, en sachant que $16600$ espèces constituent $1,3\%$ du total sur Terre, on calcule
  \[ 16 600 \cdot \left(\dfrac{1,3}{100}\right)^{-1} = \dfrac{16 600 \cdot 100}{1,3} \approx 1~276~923. \]
}{ex:fourmis-2}

\ex{proportions de populations imbriquées}{
  En 2023 en France, $13\%$ des espèces (faune et flore) sont considérées comme menacées à l'échelle mondiale (catégories ``danger critique'' à ``vulnérable'' de l'UICN).
  Parmis celles-ci, $23\%$ sont en danger critique. \footnote{\href{https://naturefrance.fr/indicateurs/proportion-en-france-despeces-menacees-lechelle-mondiale}{https://naturefrance.fr/indicateurs/proportion-en-france-despeces-menacees-lechelle-mondiale}}

  On cherche à calculer le pourcentage d'espèces en danger critique par rapport au nombre total d'espèces.
  Notons $N$ le nombre total d'espèces. Le nombre d'espèces menacées et donc donné par
  \[ 0,13 \cdot N. \]
  Parmis ces espèces, le nombre en danger critique est donné par
  \[ 0,23 \cdot \left(0,13 \cdot N \right) = ( 0,23 \cdot 0,13 ) \cdot N \approx 0,03 \cdot N. \]
  Ainsi, $3\%$ des espèces sont en danger critique.
}{}

\nt{
  Les proportions sont multiplicatives.
  Au même titre que prendre la moitié de la moitié revient à prendre un quart,
  prendre $50\%$ de $50\%$ revient à multiplier par $0,5 \cdot 0,5 = \left(\dfrac12\right)^2 = \dfrac14$.

  Lorsqu'on prend la moitié du tiers d'une quantité, on en prend en fait $\dfrac12 \cdot \dfrac13 = \dfrac16$.
  On peut estimer la fraction $\dfrac16$ en prenant la moitié de celle de $\dfrac13 \approx 0,33$.
  D'où $\dfrac16 \approx 16,6\%$.
}

\exe{1}{
        Une classe de Seconde comprend $25$ filles pour $9$ garçons.
        Calculer le pourcentage de filles et de garçons dans la classe.
}{exe:pop1}{
	Le nombre total d'élèves est de $25+9 = 34$.
	
	On calcule donc, pour les filles, la proportion $\dfrac{25}{34} \approx 0,73 = 73\%$.
	
	Idem pour les garçons, $\dfrac{9}{34} \approx 0,27 = 27\%$. 
	Remarquons que la somme des pourcentages est de $100\%$ car $\dfrac{25}{34} + \dfrac{9}{34} = \dfrac{34}{34} = 1 = 100\%$.
	On aurait donc pu déduire le pourcentage de garçons en calculant $100 - 73 = 27$.
}

\exe{}{
   En sachant que les 16 600 espèces de fourmis constituent environ 1,3\% du total des espèces d'insectes répertoriées sur Terre, estimer le nombre total d'espèces d'insectes. 
}{exe:pop2}{
	On a $\dfrac{16 600}{\text{nombre d'espèces d'insectes}} = 1,3\% = 0,013$.
	
	Par conséquent, 
		\[ \text{nombre d'espèces d'insectes} = \dfrac{16600}{0,013} \approx 1,3 \times 10^{6}, \]
	soit environ $1,3$ millions.
	
	Remarquons que la fraction $\dfrac{16600}{0,013}$ ne donne pas un nombre entier, car le pourcentage a été approximé (\og \emph{environ} $1,3$\% \fg).
}


\exe{}{
  En 2023 en France, $13\%$ des espèces (faune et flore) sont considérées comme menacées à l'échelle mondiale (catégories ``danger critique'' à ``vulnérable'' de l'UICN).
  Parmis celles-ci, $23\%$ sont en danger critique.
  
  Calculer le pourcentage d'espèces en danger critique par rapport au nombre total d'espèces.
}{exe:pop3}{
	Notons $E$ l'ensemble des espèces indigènes à la France, $M$ la sous-population des espèces menacées, et $D$ la sous-population des espèces en danger critique.
	On a donc la suite d'inclusions
		\[ D \subset M \subset E. \]
	
	Le texte donne les informations
		\begin{align*}
			|M| = 0,13 \cdot |E| && \text{et} && |D| = 0,23 \cdot |M|
		\end{align*}
	Par conséquent, $|D| = 0,23 \times 0,13 \cdot |E| \approx 0,03 \cdot |E|$.
	Donc les espèces en danger critique constituent $3\%$ des espèces.

	Les proportions sont ainsi multiplicatives. Attention à ne pas naïvement multiplier les pourcentages, car $13\times 23 \approx 300$.
}


\section{Évolution relative}

On généralise d'abord la notion de pourcentage de valeur à tous les nombres réels et pourcentages positifs ou nuls (en permettant les pourcentages supérieurs à $100\%$).

\dfn{}{
  Soit $x, p\in\R$ une valeur et une proportion positives ou nulles.
  On rappelle que $100p \% = p$.

  On dira \og $\bigl(100p\bigr) \%$ de $x$ \fg pour parler de la valeur $p \cdot x$.
}
{}



\ex{}{
  Un manteau est mis en vente à un prix initial de $150$€ auquel une remise de $45\%$ est appliquée.
  Celui coûte donc $55\%$ de $150$ euros, ce qui vaut
  \[ 0,55 \cdot 150 = 82,5.\]
}{}


\exe{, difficulty=1}{
  On estime la biomasse totale des fourmis sur Terre à $12$ millions de tonnes.
  Ceci serait égal à $20\%$ de la biomasse humaine.

  Estimer la biomasse totale des humains sur Terre en tonnes.
}{exe:evol3}{
	On a la relation
		\[ \dfrac{\text{biomasse des fourmis}}{\text{biomasse humaine}} = 0,2. \]
	D'où
		\[ \text{biomasse humaine} = \dfrac{12 \times 10^6}{0,2} \text{T} = 60 \times 10^6 \text{T}.\] 

}

\mprop{}{
  Soient $A$ et $B$ deux nombres réels positifs ou nuls.
  Les quantités \og $A\%$ de $B$ \fg et \og $B\%$ de $A$ \fg sont égales.
}{}


\exe{}{
  Calculer sans calculatrice les valeurs suivantes.
  \begin{multicols}{3}
    \begin{enumerate}
    \item $75\%$ de $60$
    \item $60\%$ de $75$
    \item $72\%$ de $25$
    \item $68\%$ de $20$
    \item $125\%$ de $40$
    \item $40\%$ de $125$
    \end{enumerate}
  \end{multicols}
}{exe:evol1}{
  \begin{multicols}{2}
    \begin{enumerate}
    \item $\dfrac34 \cdot 60 = 3 \cdot \dfrac{60}4 = 3 \cdot 15 = 45$
    \item $45$
    \item $\dfrac14 \cdot 72 = 18$
    \item $\dfrac15 \cdot 68 = \dfrac{136}{10} = 13,6$
    \item $40 + \dfrac14 \cdot 40 = 50$
    \item $50$
    \end{enumerate}
  \end{multicols}
}

\exe{}{
  Approximer sans calculatrice les valeurs suivantes.
  \begin{multicols}{3}
    \begin{enumerate}
    \item $33\%$ de $150$
    \item $166\%$ de $180$
    \item $11\%$ de $90$
    \item $89\%$ de $81$
    \item $16,6\%$ de $18$
    \item $83,4\%$ de $36$
    \end{enumerate}
  \end{multicols}
}{exe:evol2}{
  \begin{multicols}{2}
    \begin{enumerate}
    \item $\approx \dfrac13 \cdot 150 = 50$
    \item $\approx 180 + \dfrac23 \cdot 180 = 180 + 120 = 300$
    \item $\approx \dfrac19 \cdot 90 = 10$
    \item $\approx 81 - \dfrac19 \cdot 81 = 81 - 9 = 72$
    \item $\approx \dfrac16 \cdot 18 = 3$
    \item $\approx 36 - \dfrac16 \cdot 36 = 36 - 6 = 30$
    \end{enumerate}
  \end{multicols}
}


\exe{}{
  Calculer sans calculatrice les pourcentages suivants.
  \begin{multicols}{2}
    \begin{enumerate}
    \item $50\%$ de $60$
    \item $60\%$ de $50$
    \item $68\%$ de $25$
    \item $77\%$ de $20$
    \end{enumerate}
  \end{multicols}
}{exe:pourcentages}{
	TODO
}

\nt{
	Le pourcentage peut être également marqueur d'une évolution \emphindex{relative}
	\footnote{L'évolution \emphindex{absolue} correspond à la simple différence ``valeur finale -- valeur initiale''. Par exemple et selon le GIEC, la température moyenne mondiale a augmenté d'environ 1,1°C depuis le début de l'industrialisation ($19^{\text{è}}$ siècle).}
 d'une même quantité au fil du temps.
	On calculera alors généralement une proportion $p$ donnant l'évolution entre deux valeurs.

	\begin{align}\label{eq:evolution}
		p = \dfrac{\text{valeur initiale}}{\text{valeur finale}}.
	\end{align}
}

\mprop{}{
  Soient $a, b$ deux réels strictement positifs et $p = \dfrac{a}b$ leur rapport.
  On distingue trois cas.
  \begin{enumerate}
    \item si $a=b$, alors $p=1$ ;
    \item si $a >b$, alors $p>1$ ; et
    \item si $a < b$, alors $p<1$.
  \end{enumerate}
}{}


\nt{
  La proportion $p$ calculée en \eqref{eq:evolution} peut donc être supérieure à $1$ si la valeur initiale est plus petite que la valeur finale.
  Ceci correspond à un pourcentage supérieur à $100\%$.
}

\dfn{augmentation, diminution}{
  Une \emphindex{augmentation} d'une valeur $N$ d'une proportion $p\geq0$ correspond à la somme
  \[ N + p\cdot N = (1+p)\cdot N.\]
  Une \emphindex{diminution} d'une valeur $N$ d'une proportion $0\leq p \leq 1$ correspond à la différence
  \[ N - p\cdot N = (1-p)\cdot N.\]
}{}

\ex{}{
  En $2023$, le prix moyen du gaz naturel facturé aux ménages français s'élève à $115$€ par MWh, toutes taxes comprises (TTC).\footnote{\href{https://www.statistiques.developpement-durable.gouv.fr/media/7469}{https://www.statistiques.developpement-durable.gouv.fr/media/7469}}
  En $2022$, le prix était de $96$€.

  On calcule $\dfrac{115}{96} \approx 1,2$.
  Le prix de $2023$ est donc $120\%$ celui de $2022$, ce qui correspond à une augmentation de $20\%$.
}{}


\exe{}{
  En $2023$, le prix moyen du gaz naturel facturé aux ménages français s'élève à $115$€ par MWh, toutes taxes comprises (TTC).
  En $2022$, le prix était de $96$€.

  Calculer le pourcentage d'augmentation du prix entre l'année $2022$ et l'année $2023$.
}{exe:evol9}{
	On calcule la proportion $\dfrac{115}{96} \approx 1,20 = 120\%$.
	Celle-ci correspond à une augmentation de $20\%$.
}

\exe{}{
  En $2022$ en France, la consommation de gaz naturel s'établit à $463$ TWh.
  En $2021$, celle-ci s'élevait plutôt à $475,85$ TWh.

  Calculer le pourcentage de diminution de la consommation entre l'année $2021$ et l'année $2022$.
}{exe:evol10}{
	On calcule la proportion $\dfrac{463}{475,85} \approx 0,973 = 97,3\%$.
	Celle-ci correspond à une diminution de $2,7\%$.
}

\exe{}{
	Un marchand décide de changer le prix de sa marchandise de 1 000€ à 999€, prix psychologique.
	Il compare le nombre de ventes avant et après le changement de prix.
	\begin{enumerate}
		\item De quel pourcentage les ventes doivent-elles augmenter pour que le chiffre d'affaire reste inchangé ?
		\item De quel pourcentage les ventes doivent-elles augmenter pour que le chiffre d'affaire augmente de 10\% ?
	\end{enumerate}
}{exe:prix-psychologique}{
	TODO
}



\thm{évolutions successives}{
	L'évolution d'une valeur correspond à sa multiplication par un coefficient multiplicateur $m$.
	
	\begin{enumerate}
		\item Si $m>1$, $m=1+p$, et l'évolution est une augmentation de $100p \%$.
		\item Si $m<1$, $m=1-p$, et l'évolution est une diminution de $100p \%$.
		\item Si $m=1$, la valeur ne change pas.
	\end{enumerate}
	
	Lorsque deux évolutions successives ont lieu, les coefficients sont multipliés entre eux pour obtenir un coefficient multiplicateur global.

	\begin{center}
	\includegraphics[page=1]{figures/fig-evolution.pdf}
	\end{center}

}{thm:ev-succ}

\ex{}{
	Augmenter une quantité $N$ de $20\%$ correspond à la multiplier par $1,2$.
	Une diminution de $20\%$, elle, multiplie par $0,8$.
	
	En appliquant ces deux évolutions de façon successives, le coefficient final est donc donné par 
		\[0,8 \cdot 1,2 = 0,96, \]
	ce qui correspond à une diminution de $4\%$.
}{}

\exe{}{
        À quelle évolution correspond une augmentation de $20\%$ suivie d'une diminution de $20\%$ ?
}{exe:evol5}{
	Augmenter une quantité $N$ de $20\%$ correspond à la multiplier par $1,2$.
	Une diminution, elle, multiplie par $0,8$.
	
	La quantité finale est donné par 
		\[ 0,8 \cdot (1,2 \cdot N) = (0,8 \cdot 1,2) \cdot N = 0,96 \cdot N, \]
	qui correspond à une diminution de $4\%$.
}

\exe{, difficulty=1}{
  Une jeune femme dépose $10$€ à la banque. Celle-ci lui promet un taux d'intérêt à l'année de $3\%$.
  Ainsi, après la première année, il y aura $1,03 \cdot 10 = 10,3$€ sur son compte.
  La deuxième année, il y aura $1,03 \cdot 10,35 = 10,609$€, etc...

  À l'aide de la calculatrice, répondre aux questions suivantes.
  \begin{enumerate}
  \item Combien d'argent aura-t-elle après $5$ ans ?
  \item Combien d'argent aura-t-elle après $50$ ans ?
  \item Combien d'argent y aura-t-il sur son compte après $1000$ ans ?
  \end{enumerate}
}{exe:evol4}{
  \begin{enumerate}
  \item On multiplie $5$ fois par $1,03$, ce qui donne $1,03^5 \times 10  \approx 11,59$€.
  \item On multiplie $50$ fois par $1,03$, ce qui donne $1,03^{50} \times 10  \approx 43,84$€.
  \item On multiplie $1000$ fois par $1,03$, ce qui donne $1,03^{1000} \times 10  \approx 6,87 \times 10^{13}$€, c'est-à-dire environ $68$ billions d'euros ($1$ billion = $1000$ milliards).
  \end{enumerate}
}

\exe{, difficulty=2}{
    Considérons $p\geq 0$ une proportion et $100p$ le pourcentage associé.
    \begin{enumerate}
    \item À quelle évolution, en fonction de $p$, correspond une augmentation de $100p\%$ suivie d'une diminution de $100p\%$ ?
    \item Quel $p$ choisir pour trouver une diminution finale de $16\%$ ?
    \end{enumerate}
}{exe:evol12}{

    \begin{enumerate}
    \item Soit $N\geq0$ une quantité. Après une augmentation de $100p\%$ puis une diminution de $100p\%$,
    		la quantité est donnée par $(1-p) \cdot (1+p) \cdot N = (1-p^2) \cdot N$.
    		Ceci correspond à une diminution de $100\left(p^2\right) \%$.
    		
    		On comparera avec l'exercice 9, où $p=0,2$ et $p^2 = 0,04 = 4\%$.
    \item On pose l'égalité suivante
    		\[ 100p^2 = 16. \]
    	Remarquons que $16$ et $100$ sont tous les deux des carrés parfaits :
    		\[ p^2 = \dfrac{16}{100} = \left( \dfrac{4}{10} \right)^2, \]
	et donc $p = \dfrac{4}{10} = 40\%$, car $p\geq 0$.
	
	Vérification : au vu de la question 1, on calcule $0,4^2 = 0,16 = 16\%$.
    \end{enumerate}

}


\thm{évolution réciproque}{
	L'évolution réciproque est l'évolution qui permet de revenir à une valeur initale.
	D'après le théorème \ref{thm:ev-succ}, le coefficient multiplicateur de l'évolution réciproque est donc l'inverse de celui de l'évolution considérée.
	
	\begin{center}
	\includegraphics[page=2]{figures/fig-evolution.pdf}
	\end{center}
}{thm:ev-rec}



\ex{}{
  Une augmentation de 50\% revient à multiplier une valeur initiale $a$ par 1,5 pour obtenir une valeur finale $b$.
  \[b = 1,5\cdot a.\]
  En changeant de référentiel, on peut se demander quel pourcentage de diminution doit-on appliquer à la valeur finale $b$ pour obtenir la valeur initiale $a$ ?
  \[ a = \dfrac{1}{1,5} \cdot b = \dfrac23 b \approx 0,66 \cdot b. \]
  Pour obtenir $a$, il faut donc réduire $b$ d'environ 34\%.
}{}

\exe{}{
  Si on augmente le prix d'un objet de $150\%$, quel rabais faut-il appliquer pour retrouver le prix initial de l'objet ?
}{exe:evol6}{
	Notons $P$ le prix initial de l'objet.
	Le prix augmenté vaut donc $1,5 \cdot P$.
	Pour retrouver $P$, il faut multiplier le prix augmenté par l'inverse de $1,5$, soit $1,5^{-1} = \frac23 \approx 0,666 = 66,6\%$.
	Ceci correspond à une diminution de 33,4\%.
}


\exe{}{
  Considérons deux tailleurs, l'un à 250€ et l'autre à 360€.
  \begin{enumerate}
  \item Quelle augmentation de prix faut-il appliquer au premier tailleur pour qu'il ait le prix du second ?
  \item Quel rabais faut-il appliquer au second tailleur pour qu'il ait le prix du premier ?
  \end{enumerate}
}{exe:evol7}{
  \begin{enumerate}
  \item On calcule l'évolution $\frac{360}{250}  = 1,44 = 144\%$. Ainsi, le deuxième tailleur vaut 144\% du prix du premier : une augmentation de 44\% est nécessaire.
  \item On calcule l'évolution $\frac{250}{360}  \approx 0,7 = 70\%$. Le premier tailleur vaut environ 70\% du prix du second : une diminution de 30\% est nécessaire.
  \end{enumerate}
}

\exe{}{
	Lors d'un payement par carte bancaire, une commission valant 1\% du montant de la transaction est versée comme frais bancaire.
	Ainsi, lorsqu'un débiteur paye 100€, le créditeur reçoit 99€, et 1€ est versé à la banque.
	\begin{enumerate}
		\item Pour un payement de 200€, quelle quantité est versée en frais bancaire ? quelle quantité le créditeur reçoit-il ?
		\item Répondre à la même question pour un payement de 202€.
		\item Obtient-on la quantité initiale après une augmentation de 1\% suivie d'une diminution de 1\% ?
		\item Quelle quantité le débiteur doit-il verser pour que le créditeur reçoive exactement 200€ ? Est-ce un nombre réel ? rationnel ? décimal ? entier ?
	\end{enumerate}
}{exe:caution-matthieu-payup}{
	TODO
}

\exe{, difficulty=1}{
	Un magasin propose une remise de 15\% sur tous les articles à partir de deux articles achetés.
	Un client ne souhaite acheter qu'un seul article à 80€.
	
	\begin{enumerate}
		\item
		Le client peut-il, en ajoutant un deuxième article, obtenir un panier à 75€ ?
		Si oui, donner le prix du deuxième article au centime près.
		
		\item
		Quel est le prix maximal du deuxième article qu'il peut se permettre d'ajouter à son panier pour que le prix final reste sous les 80€ ?
		Arrondir au centime près.
	\end{enumerate}
}{exe:linvosges-matthieu}{
	
	\begin{enumerate}
		\item
		Soit $P$ le prix du deuxième article.
		On souhaite que $P$ vérifie
			\[ 0,85 (80+P) = 75. \]
		En résolvant, on obtient $P \approx 8,23$€.
		
		\item
		L'équation pour $P$ est, cette fois-ci, $0,85 (80+P) = 80$, qui donne $P \approx 14,12$€.
	\end{enumerate}
}

\exe{, difficulty=2}{
        On considère l'ensemble $E= \bigset{1; 2; \dots ; n-1 ; n}$ dépendant d'un entier naturel $n\geq1$.
        On pose $F = \bigset{k \in E \tq 2 | k}$, l'ensemble des éléments pairs de $E$.

        A-t-on toujours $\dfrac{|F|}{|E|} = \dfrac12$ ? Donner l'ensemble des entiers $n\geq1$ pour lesquels l'égalité est vraie.
}{exe:evol8}{
	En prenant $n=1; 2; 3; 4$, on conjecture que l'égalité est vraie si et seulement si $n$ est lui-même pair.
	
	D'une part, pour avoir $\dfrac{|F|}{|E|} = \dfrac12$, on doit nécessairement avoir $n = |E| = 2 \cdot |F|$, et donc $n$ pair.
	
	Réciproquement, si $n$ est pair, alors $n=2k$ et $F = \bigset{ 2\times1; 2\times2; 2\times3; \cdot ; 2\times(k-1) ; 2 \times k}$.
	Ainsi $|F| = k = \dfrac{n}2$ et l'égalité est vérifiée. 
}


\exe{, difficulty=1}{
  Si on augmente le prix d'un objet de $100p\%$ ($p\geq0$ réel), quel rabais faut-il appliquer (en fonction de $p$) pour retrouver le prix initial de l'objet ?
}{exe:evol11}{
	Soit $N\geq0$ un prix quelconque, et $(1+p)\cdot N$ le prix augmenté de $100p\%$.
	
	Pour retrouver le prix original, il faut multiplier par $(1+p)^{-1} = \dfrac{1}{1+p}$.
	La diminution correspondante est donnée par
		\[ 1 - \dfrac{1}{1+p} = \dfrac{p}{1+p}. \]
	On comparera avec l'exercice 10, où $p=0,5$, et $\dfrac{p}{1+p} = \dfrac{0,5}{1,5} = \dfrac13 \approx 33,3 \%$.
}

\exe{, difficulty=2}{
    Soit l'ensemble d'entiers $E=\bigset{m; m+1; \dots; n-1 ; n}$ dépendant de deux entiers relatifs $m<n$.
    Calculer $|E|$ en fonction de $m$ et $n$.
    Vérifier la formule avec $m=-1$ et $n=1$ en sachant que $\bigl|\bigset{-1 ; 0 ; 1 }\bigr| = 3$.
}{exe:evol13}{
	On soustrait $m$ à chaque élément de $E$, ce qui ne change pas le cardinal mais a l'avantage de simplifier l'ensemble.
		\[ |E| = \bigl| \bigset{ 0; 1 ; \dots ; n-m }\bigr|. \]
	Il y a $n-m$ entiers de $1$ à $n-m$, et donc $|E| = n-m+1$ éléments au total en n'oubliant pas $0$.

	Pour $m=-1, n=1$, on a bien $n-m+1 = (1) - (-1) + 1  = 3$.
}

\section{Ordres de grandeur}

\notations{
	On note $\R_+^*$ l'ensemble des réels strictement positifs.
	On note $\N^*$ les entiers naturels hormis 0.
}

Dans la suite, $q \in\R_+^*$ est un nombre réel strictement positif qu'on appelle \emphindex{base}.

\dfn{puissance sur $\N^*$}{
	Soit $n \in \N^*$ un entier naturel non nul.
	Alors \og $q$ puissance $n$ \fg~est égal à
		\[ q^{n} = \underbrace{q \times q \times \cdots \times q}_{\text{$n$ fois}}. \]
	En particulier,
		\begin{align*}
			q^1 = q && q^2 = q\times q && q^3 = q \times q \times q
		\end{align*}
}{dfn:puissance-N}

\mprop{propriétés des puissances}{
	Soient $a, b \in \Z$. Alors
		\[ q^{a} \times q^{b} = q^{a+b}, \]
	et 
		\[ \bigl( q^{a}\bigr)^b = q^{a\times b}. \]
	En particulier, on a 
		\begin{align*}
			q^0 = 1 && q^{-1} = \dfrac1q &&  q^{-a} = \dfrac1{q^a}
		\end{align*}
}{prop:propriétés-puissances}

\exe{}{
	Sans calculatrice, exprimer les nombres suivants sous la forme $q^n$, où $q \in \N$ et $n\in\Z$ sont des entiers.
	\begin{multicols}{3}
	\begin{enumerate}[label=\roman*)]
		\item $10^3 \times 10^5$
		\item $\left(4^5\right)^2$
		\item $\dfrac{5^3}{5^3}$
		\item $1$
		\item $\dfrac{2^4}{2^7}$
		\item $\left(2^{-1}\right)^3$
		\item $\left(2^{3}\right)^{-1}$
		\item $\left(\dfrac{1}{7^2}\right)^6$
		\item $\dfrac{10^{12}}{10^{-12}}$
		\item $\dfrac{10^{-5}}{10^{6}}$
	\end{enumerate}
	\end{multicols}
}{exe:puissances}{
	TODO
}

\exe{}{
	On estime que, dans l'univers, il y a au moins
		\begin{itemize}
			\item $10^{11}$ galaxies ; que chacune contient
			\item $10^{11}$ étoiles ; dont la masse moyenne est de
			\item $10^{32}$ kilogrammes ; et que chaque gramme de matière contient
			\item $10^{24}$ atomes.
		\end{itemize}
	Estimer le nombre d'atomes dans l'univers observable à partir de ces données.
}{exe:atomes-univers}{
	TODO
}

\dfn{ordre de grandeur}{
	Un nombre a pour  \emphindex{ordre de grandeur} $10^n$, avec $n\in\Z$ entier relatif, sa puissance de $10$ la plus proche.
}{dfn:ordre-grandeur}

\ex{}{
	On a $2^{40} \approx 1,09 \times 10^{12}$, et donc son ordre de grandeur est $10^{12}$.
	Il faut $13$ chiffres pour écrire $2^{40}$.
}{ex:ordre-grandeur}

\exe{}{
	Montrer qu'on a environ $2^{10} \approx 10^3$. 	
	\begin{enumerate}
		\item En déduire approximativement l'ordre de grandeur de $2^{20}$ et le nombre de chiffres nécessaires pour l'écrire.
		\item En déduire approximativement l'ordre de grandeur de $2^{35}$ et le nombre de chiffres nécessaires pour l'écrire.
	\end{enumerate}
}{exe:grandeur-binaire}{
	TODO
}

% redondant un peu
\exe{}{
	Montrer qu'on a environ $6^{8} \approx 10^6$ et $7^6 \approx 10^5$ : ce sont leur \emph{ordre de grandeur}.
	
	\begin{enumerate}
		\item En déduire l'ordre de grandeur de $6^{16}$ et le nombre de chiffres nécessaires pour l'écrire.
		\item En déduire l'ordre de grandeur de $7^{18}$ et le nombre de chiffres nécessaires pour l'écrire.
		\item En déduire l'ordre de grandeur de $6^{24} \times 7^{12}$ et le nombre de chiffres nécessaires pour l'écrire.
	\end{enumerate}
}{exe:grandeur-6-7}{
	TODO
}

\exe{}{
	Montrer que $10^{50} - 10^{30} = 10^{50} \bigl( 1 - 10^{-20} \bigr)$.
	Décrire le développement décimal de $1-10^{-20}$. De quel entier est-il très proche ?
	
	En déduire l'ordre de grandeur de $10^{50} - 10^{30}$.
}{exe:grandeur-soustraction}{
	TODO
}

\exe{}{
	Exprimer en écriture scientifique la valeur obtenue après une diminution de 60\% de $10^{60}$.
	Quel est son ordre de grandeur ?
}{exe:grandeur-évolution}{
	TODO
}

\exe{, difficulty=2}{
	Le but de l'exercice est de calculer la valeur de $q^N$ le plus vite possible, où $q\in\R$ est un nombre réel et $N\in\N$ un entier naturel.
	On se restreint aux opérations $+$ et $\times$ afin de pouvoir compter les opérations arithmétique élémentaires (la puissance n'en est pas une !).
	
	\begin{enumerate}
		\item Implémenter la fonction \texttt{puissance(q,N)} figure \ref{fig:fast-exp} qui renvoie $q^N$.
		\item Combien de multiplications sont nécessaires pour que \texttt{puissance(q,4)} termine ? et  \texttt{puissance(q,10)} ? et \texttt{puissance(q,N)} ?
		\item Considérons l'algorithme \texttt{fastexp} de la figure \ref{fig:fast-exp}. Que retourne l'appel \texttt{fastexp(q, 1)} ? et \texttt{fastexp(q, 2)},  \texttt{fastexp(q, 3)} ?
		\item Combien de multiplications sont nécessaires pour que \texttt{fastexp(q,4)} termine ?
		\item Comparer le nombre de multiplications nécessaires au calcul de $q^{1024}$ en utilisant \texttt{puissance(q,1024)} versus \texttt{fastexp(q, 10)}.
		\item Montrer que si $N=2^n$, alors  \texttt{fastexp(q, n)} retourne $q^N$ en $n$ opérations arithmétiques.
		\item Réécrire \texttt{fastexp} sous forme récursive en quatre lignes.
	\end{enumerate}
}{exe:fast-exp}{
	TODO
}

\begin{figure}
	\begin{multicols}{2}
	\python{fastexp}
	\end{multicols}
	\caption{Deux fonctions d'exponentiation.}
	\label{fig:fast-exp}
\end{figure}



